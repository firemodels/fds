% !TEX root = FDTs_Validation_Guide.tex

\chapter{HGL Temperature}
\label{HGL_Chapter}

\section{Natural Ventilation}

For a compartment with natural ventilation, the method of McCaffrey, Quintiere, and Harkleroad (MQH) can be used. In this method, the hot gas layer temperature increase, $\Delta T_g$, above ambient ($T_g$ - $T_\infty$) is given as follows

\be
\Delta T_g = 6.85 \left( \frac{\dot Q^2}{A_v \sqrt{H_v} h_k A_T} \right)^{1/3}
\label{eq:MQH}
\ee

\noindent where

\be
t_p = \left( \frac{\rho c}{k} \right) \left( \frac{\delta}{2} \right)^2
\label{eq:MQH_tp}
\ee

\be
h_k = \frac{k}{\delta} \mathrm{\ for\ } t > t_p
\label{eq:MQH_hk_gt}
\ee

\be
h_k = \left( \frac{k \rho c}{t} \right)^{1/2} \mathrm{\ for\ } t \le t_p
\label{eq:MQH_hk_lt}
\ee

In Eq.~\ref{eq:MQH}, $\Delta T_g$ is the upper gas temperature rise above ambient~($^\circ$C), $\dot Q$ is the heat release rate (HRR) of the fire~(kW), $A_v$ is the total area of the ventilation openings~(m$^2$), $H_v$ is the height of the ventilation opening~(m$^2$), $h_k$ is the heat transfer coefficient~(W/m$^2$/K), and $A_T$ is the total area of the compartment enclosing surfaces, excluding areas of vent openings~(m$^2$). 

The parameter $t_p$ is the thermal penetration time of the wall lining material, which is given by Eq.~\ref{eq:MQH_tp}. If the burning time $t$ is greater than $t_p$, then the heat transfer coefficient $h_k$ is calculated using Eq.~\ref{eq:MQH_hk_gt}. In Eq.~\ref{eq:MQH_hk_gt}, $k$ is the thermal conductivity of the interior lining~(kW/m/K), and $\delta$ is the thickness of the interior lining. If the burning time $t$ is less than $t_p$, then the heat transfer coefficient $h_k$ is calculated using Eq.~\ref{eq:MQH_hk_lt}. In Eq.~\ref{eq:MQH_hk_lt}, $k$ is the thermal conductivity of the interior lining~(kW/m/K), $\rho$ is the density of the interior lining~(kg/m$^3$), and $c$ is the specific heat of the interior lining~(kJ/kg/K).

%The smoke layer height can be determined by
%
%\be
%Z = H (1 + 2 \left( \frac{0.05}{\rho_G} \right) Q^{0.333} T H^{0.667}/(3 L W) )^{-1.5}
%\ee
%
%where
%
%\be
%\rho_g = \frac{353}{T_g}
%\ee
%
%If the smoke layer height is below the door height, then the door height is used as the layer height.


\clearpage


\section{Forced Ventilation}

\subsection{FPA Method}

For a compartment with forced ventilation, the method of Foote, Pagni, and Alvares (FPA) can be used. In this method, the upper-layer gas temperature increase above ambient is given as a function of the fire HRR, the compartment ventilation flow rate, the gas-specific heat capacity, the compartment surface area, and an effective heat transfer coefficient. The approximate compartment hot gas layer temperature increase, $\Delta T_g$, above ambient ($T_g$ - $T_\infty$) is given by the following equation

\be
\Delta T_g = \left[ 1 + 0.63 \left( \frac{\dot Q}{\dot m c_p T_\infty} \right)^{0.72} \left( \frac{h_k A_T}{\dot m c_p} \right)^{-0.36} \right] T_\infty
\label{eq:FPA}
\ee

\noindent where

\be
t_p = \left( \frac{\rho c}{k} \right) \left( \frac{\delta}{2} \right)^2
\label{eq:FPA_tp}
\ee

\be
h_k = \frac{k}{\delta} \mathrm{\ for\ } t > t_p
\label{eq:FPA_hk_gt}
\ee

\be
h_k = \left( \frac{k \rho c}{t} \right)^{1/2} \mathrm{\ for\ } t \le t_p
\label{eq:FPA_hk_lt}
\ee

In Eq.~\ref{eq:FPA}, $\Delta T_g$ is the hot gas layer temperature rise above ambient~($^\circ$C), $\dot Q$ is the HRR of the fire~(kW), $\dot m$ is the compartment ventilation mass flow rate~(kg/s), $c_p$ is the specific heat of air (kJ/kg/K), $T_\infty$ is the ambient air temperature~($^\circ$C), $h_k$ is the heat transfer coefficient~(kW/m$^2$/K), and $A_T$ is the total area of compartment enclosing surfaces~(m$^2$).

The parameter $t_p$ is the thermal penetration time of the wall lining material, which is given by Eq.~\ref{eq:FPA_tp}. If the burning time $t$ is greater than $t_p$, then the heat transfer coefficient $h_k$ is calculated using Eq.~\ref{eq:FPA_hk_gt}. In Eq.~\ref{eq:FPA_hk_gt}, $k$ is the thermal conductivity of the interior lining~(kW/m/K), and $\delta$ is the thickness of the interior lining. If the burning time $t$ is less than $t_p$, then the heat transfer coefficient $h_k$ is calculated using Eq.~\ref{eq:FPA_hk_lt}. In Eq.~\ref{eq:FPA_hk_lt}, $k$ is the thermal conductivity of the interior lining~(kW/m/K), $\rho$ is the density of the interior lining~(kg/m$^3$), and $c$ is the specific heat of the interior lining~(kJ/kg/K).


\clearpage


\subsection{Deal and Beyler Method}

For a compartment with forced ventilation, the method of Deal and Beyler (DB) can be used. In this method, the approximate compartment hot gas layer temperature increase, $\Delta T_g$, above ambient ($T_g$ - $T_\infty$) is given by the following equation

\be
\Delta T_g = \left( \frac{\dot Q}{\dot m c_p + h_k A_T} \right)
\label{eq:DB}
\ee

\noindent where

\be
h_k = 0.4\ \mathrm{max} \left( \sqrt{\frac{k \rho c}{t}} , \frac{k}{\delta} \right)
\label{eq:DB_hk}
\ee

In Eq.~\ref{eq:DB}, $\Delta T_g$ is the hot gas layer temperature rise above ambient~($^\circ$C), $\dot Q$ is the HRR of the fire~(kW), $\dot m$ is the compartment ventilation mass flow rate~(kg/s), $c_p$ is the specific heat of air (kJ/kg/K), $T_\infty$ is the ambient air temperature~($^\circ$C), $h_k$ is the heat transfer coefficient~(kW/m$^2$/K), and $A_T$ is the total area of compartment enclosing surfaces~(m$^2$).

In Eq.~\ref{eq:DB_hk}, $k$ is the thermal conductivity of the interior lining~(kW/m/K), $\rho$ is the density of the interior lining~(kg/m$^3$), $c$ is the specific heat of the interior lining~(kJ/kg/K), $t$ is the exposure time~(s), and $\delta$ is the thickness of the interior lining~(m). This model is only valid for times up to 2000 seconds.



\clearpage


\section{No Ventilation}

For a compartment with no ventilation (closed doors) and constant HRR, the method of Belyer can be used. In this method, the compartment hot gas layer temperature increase, $\Delta T_g$, above ambient ($T_g$ - $T_\infty$) is given by the following equation

\be
\Delta T_g = \frac{2 K_2}{K_1^2} (K_1 \sqrt{t} - 1 + e^{-K_1 \sqrt{t}})
\label{eq:Beyler}
\ee

\noindent where

\be
K_1 = \frac{2(0.4\sqrt{k \rho c}) A_T}{m c_p}
\label{eq:Beyler_K1}
\ee

\be
K_2 = \frac{\dot Q}{m c_p}
\label{eq:Beyler_K2}
\ee

In Eq.~\ref{eq:Beyler}, $\Delta T_g$ is the hot gas layer temperature rise above ambient~($^\circ$C), and $t$ is the exposure time~(s). In Eq.~\ref{eq:Beyler_K1}, $k$ is the thermal conductivity of the interior lining~(kW/m/K), $\rho$ is the density of the interior lining~(kg/m$^3$), $c$ is the specific heat of the interior lining~(kJ/kg/K), $A_T$ the total area of compartment enclosing surfaces~(m$^2$), $m$ is the mass of gas in the compartment~(kg), and $c_p$ is the specific heat of air~(kJ/kg/K). In Eq.~\ref{eq:Beyler_K2}, $\dot Q$ is the HRR of the fire~(kW).


\clearpage


\section{Summary of Hot Gas Layer Temperature Predictions}

Summary scatter plots of the HGL temperature predictions are shown on the following pages.

\begin{figure}[ht]
\begin{center}
\begin{tabular}{l}
\includegraphics[width=4.0in]{SCRIPT_FIGURES/Scatterplots/HGL_Temperature_MQH}
\end{tabular}
\end{center}
\caption[Summary of HGL temperature predictions for natural ventilation tests]
{Summary of HGL temperature predictions for natural ventilation tests using the MQH method.}
\label{HGL_Summary_Natural_Ventilation}
\end{figure}

\begin{figure}[p]
\begin{center}
\begin{tabular}{l}
\includegraphics[width=4.0in]{SCRIPT_FIGURES/Scatterplots/HGL_Temperature_FPA} \\
\includegraphics[width=4.0in]{SCRIPT_FIGURES/Scatterplots/HGL_Temperature_DB}
\end{tabular}
\end{center}
\caption[Summary of HGL temperature predictions for forced ventilation tests]
{Summary of HGL temperature predictions for forced ventilation tests using the FPA method (top) and DB method (bottom).}
\label{HGL_Summary_Forced_Ventilation}
\end{figure}

\begin{figure}[p]
\begin{center}
\begin{tabular}{l}
\includegraphics[width=4.0in]{SCRIPT_FIGURES/Scatterplots/HGL_Temperature_Beyler}
\end{tabular}
\end{center}
\caption[Summary of HGL temperature predictions for no ventilation tests]
{Summary of HGL temperature predictions for no ventilation tests using the Beyler method.}
\label{HGL_Summary_No_Ventilation}
\end{figure}

