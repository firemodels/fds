%
% -------------------  smoke notes ------------------------
%

\section{Overview}

Visualization techniques such as those used to view color shaded
contours are useful for analyzing data quantitatively but are not
as suitable for applications such as using a virtual fire fighter
trainer or examining the obscuration effects of smoke in an
outdoor environment.  These applications require visualization
methods that display simulation data in a realistic manner. These
methods need to determine opacities and colors from simulation
data.  This paper describes two methods, slice rendered and volume
rendered, for visualizing smoke realistically.  Each is based on
an approximation of the radiation transport equation\cite{} and
differ mainly in path length over which integrals are performed.

A complete method for visualizing smoke and fire taking into
account absorption, emission, and scattering (both in and out)
requires the solution of the the volume rendering equation\cite{}
also called the radiation transport equation (RTE)\cite{}. The
solution of this integro-differential equation requires large
amounts of memory and computational resources.  Approximations and
simplifications are required in order to display smoke at
interactive frame rates.  The main approximation is to take
advantage of the low albedo character of smoke which results in
less scattering events than would occur in a high albedo media
such as a cloud.

Two approaches are described for visualizing smoke. The first
approach, a slice rendering method, integrates a simplified form
of the radiation transport for each vertex in each simulation grid
or slice plane. The integral's path length is one slice plane to
the next. The resulting partially transparent slice planes are
combined by the video hardware to form one image. The slice planes
are oriented to be perpendicular to the viewer's line of sight.
The second approach, a volume rendering method, again integrates a
simplified form of the radiation transport equation but from the
front of the solution domain to the back rather than across just
one slice plane. Video hardware is again exploited, this time to
compute a line integral for each screen pixel.  Opacities and
color for both methods are computed using soot density and
temperature data derived from a fire simulation. Colors are
determined from a prescribed colormap which is a function of
temperature.


%\section{Background}

%\cite{stam:99}\cite{stam:95}\cite{fedkiw:01}\cite{stam:00}
%\cite{PRESS88a} \cite{marchingcubes}

%\cite{stam:99a}\cite{sayood:96a}\cite{Bajaj:01a}\cite{Levoy:90a}
%\cite{Sabella:88a}\cite{Gardner:85a}

%
% -------------------  Slice Rendering ------------------------
%

\section{Slice Rendered Smoke}


%
% -------------------  Computing Opacity ------------------------
%

\subsection{Computing Opacity}
The process of computing opacity at nodes in a slice plane is
illustrated schematically in Figure \ref{figsmokesetup}. A ray
travels from the background to the observer through intervening
smoke. At each step towards the observer, light is absorbed,
scattered or emitted by the smoke.  Emission effects are ignored.
Scattering effects presently are only accounted for in the value
of the total mass extinction coefficient used.  Contributions are
assumed to be from both absorption and scattering. The obscuration
is computed along each ray one grid plane at a time, using Beer's
law as follows.  The $\alpha$ values are pre-computed by FDS using
Beer's law~\cite{Siegel:2001}
\begin{equation}
\alpha=1-\exp(-ks\Delta x) \label{eq:alpha}
\end{equation}

\noindent for a particular view direction (down the x axis) where
$\Delta x$ is this distance between two nodes, $k$ is the total
soot mass extinction coefficient and $s$ be the soot density.
Beer's law is an empirical relationship relating light absorption
to the material properties of the media the light is travelling

\begin{figure}[\figoptions]
\begin{center}
\includegraphics[width=4.0in]{figures/smoke_setup}
\end{center}
\caption {Smoke is drawn by blending the smoke color with the
background color.  The amount of blending depends on the amount of
obscuration as determined from soot density and path length.}
\label{figsmokesetup}
\end{figure}

through, in this case soot or smoke.

The $\alpha$ parameter in equation (\ref{eq:alpha}) is used by
OpenGL to blend smoke planes with the current background, the same
blending process discussed in Section \ref{blending}.  The
$\alpha$ parameter used here also represents an opacity, 0.0 for
completely transparent, 1.0 for completely opaque.

%
% -------------------  Adjusting Opacity ------------------------
%

\subsection{Adjusting Opacity}

The absorption parameter, $\alpha$, needs to be adjusted when the
view direction is not aligned along the axis orthogonal to the
viewing planes (as in Figure \ref{figray}), the distance between
adjacent smoke planes changes, or viewing planes are skipped.
\begin{figure}[\figoptions]
\centerline{\includegraphics[width=5.0in]{figures/forney_figure4}}
\caption [Diagram illustrating the adjustment needed to opaqueness
parameter, $\alpha$, for non axis aligned views.] { Diagram
illustrating adjustment needed to opaqueness parameter, $\alpha$,
for non axis aligned views. The $\alpha$ value along the ray
containing the $\hat{x}$ segment needs to be larger to account for
the longer path length. } \label{figray}
\end{figure}

Ten million exponential operations per second are required to
display smoke with corrected $\alpha$'s at 10 frames per second if
the simulation has grid dimensions of $100\times 100\times 100$.
Recent advances in CPU and video hardware makes these types of
visualizations possible. These corrections may also be performed
in the video card (GPU), resulting in increased display rates
because the GPU performs the corrections simultaneously at all or
many of the grid nodes rather than one at a time as the CPU would.

The $\alpha$ obscurations are pre-computed using the distance
$\Delta x$ between adjacent planes along the x-axis. The adjusted
$\hat{\alpha}$ expressed in terms of $\Delta\hat{x}$ is given by
\begin{equation}
\label{eq:adjusted}
\hat{\alpha}=1-\exp(-ks\Delta \hat{x})\\
\end{equation}
where $\Delta\hat{x}$ is the distance between planes along the line of site.
Equations (\ref{eq:alpha}) and (\ref{eq:adjusted}) may be used to
solve for $\hat{\alpha}$ in terms of $\alpha$ to obtain
\begin{equation}
\label{eq:alphahat}
\hat{\alpha}=1-(1-\alpha)^{\Delta\hat{x}/\Delta x}
\end{equation}
after noting that
\begin{eqnarray*}
1-\hat{\alpha}=\exp(-ks\Delta\hat{x})=\exp(-ks\Delta
x)^{\Delta\hat{x}/\Delta x}=(1-\alpha)^{\Delta\hat{x}/\Delta x}
\end{eqnarray*}

The computation of equation (\ref{eq:alphahat}) is expensive
because the exponential is computed at each grid node for every
time step.  In addition, numerical cancellation may occur for
small $\alpha$ leading to loss of significant digits. Both
problems may be solved by expanding equation (\ref{eq:alphahat})
in a Taylor series and keeping only the first few terms:
\begin{eqnarray*}
\hat{\alpha}\approx \alpha r -
\frac{\alpha^2}{2}r(r-1)+\frac{\alpha^3}{6}r(r-1)(r-2)
\end{eqnarray*}
where $r=\sec(\theta)=\Delta \hat{x}/\Delta
x=||x_p-x_e||/n\cdot(x_p-x_e)$, $n$ is the unit vector normal to
the current plane being drawn, $\theta$ is the angle between the
view direction and $n$, $x_e$ is the observers position and $x_p$
is the vertex being drawn (along the view direction).  These terms
are illustrated in Figure \ref{figray}.

When planes are skipped, equation (\ref{eq:alphahat}) may be
simplified.  In particular, when every 2nd plane is skipped,
$\Delta\hat{x}/\Delta x=2$, so that equation (\ref{eq:alphahat})
simplifies to
\begin{eqnarray*}
\hat{\alpha}=1-(1-\alpha)^2=2\alpha-\alpha^2
\end{eqnarray*}

The video hardware uses $\alpha$ values contained in the smoke
planes to obscure the background much like a camera uses a neutral
density filter to darken a scene.  Extending the analogy,
Smokeview uses one spatial/time varying {\em numerical}\ neutral
density filter for each plane of smoke data.  On a node by node
basis then, each smoke plane obscures the current image stored in
the OpenGL back buffer by the amount $(1-\alpha)$ to form a new
back buffer image.  Figure \ref{figplume} illustrates this process
showing several snapshots of a fire plume. The final image in the
lower right is the most realistic. A simplistic description of one
step of this process is given by

\begin{eqnarray*}
\mbox{new buffer image} = (1-\alpha)\times \mbox{old buffer image}
\end{eqnarray*}


\begin{figure}[\figoptions]
\begin{center}
\begin{tabular}{cc}
\includegraphics[height=4.0in]{figures/splume_20_27}&
\includegraphics[height=4.0in]{figures/splume_17_27}\\
slices 20 to 27&slices 17 to 27\\
\includegraphics[height=4.0in]{figures/splume_14_27}&
\includegraphics[height=4.0in]{figures/splume_11_27}\\
slices 14 to 27&slices 11 to 27
\end{tabular}
\end{center}
\caption [Smoke plume visualized using several vertical parallel
partially transparent planes.] {Smoke plume visualized using
several vertical parallel partially transparent planes. The
transparency is computing using soot density which is in turn
computed using conservation equations for mass, momentum and
energy. } \label{figplume}
\end{figure}

\noindent This process is repeated for each smoke plane. Figure
\ref{figsmoke3d} illustrates this process showing smoke and fire
in a townhouse kitchen fire.

The visualization is performed by displaying a series of partially
transparent planes. For illustration, these planes are made more
conspicuous (in Figure \ref{figsmoke3d}a) by skipping smoke planes
(displaying every third plane) and orienting them along the `x'
axis. Figure \ref{figsmoke3d}b shows the visualization as it
normally appears with all slice planes shown and oriented along a
plane most perpendicular to the view direction.
\begin{figure}[\figoptions]
\begin{center}
\begin{tabular}{l}
\includegraphics[height=3.75in]{figures/thouse5c_skip}\\
a) slices skipped and oriented along `x' directions\\
\includegraphics[height=3.75in]{figures/thouse5c_full}\\
b) all slices shown and oriented towards viewer \\
\end{tabular}
\end{center}
\caption{Realistic visualization of a townhouse kitchen fire simulated
using FDS5. Planes in the top image are drawn to be conspicuous by
skipping 2 out of every 3 planes and by aligning planes along the
x axis. All planes in the bottom image are displayed (none are
skipped) and they are aligned to be closest to perpendicular of
all possible plane orientations.
  }
\label{figsmoke3d}%
\end{figure}

%
% -------------------  Lighting Smoke ------------------------
%

%\chapter{Lighting Smoke}


%
% -------------------  Orienting smoke planes ------------------------
%

\subsection{Orienting smoke planes}

Smoke opacity data computed as described in previous sections is
stored in a 3D array. This array corresponds to the solution
domain as set up in an FDS input file (or some other model). Smoke
planes are drawn in Smokeview through this data.  The orientation
is chosen to be most perpendicular to the viewers line of sight. A
plane orientation exactly perpendicular to the view direction
could be drawn if one is willing to pay the added CPU cost of
interpolating opacity values between grid nodes.

Figure \ref{figDIRA} illustrates this process showing three view
directions and the corresponding smoke plane orientations that
would be used. Off-axis viewing is minimized by selecting the view
planes orientation that minimizes the angle between the planes
normal direction and the view direction. This angle, $\theta$, is
illustrated in Figure \ref{figDIRB}, and is given by
\begin{eqnarray*}
\cos(\theta)=\frac{n\cdot v_e}{||n||~||v_e||}
\end{eqnarray*}

\noindent where $n$ is normal vector for the candidate smoke
plane, and $v_e$ is the view direction vector.  In OpenGL, the
view direction vector, $v_e$, is computed by simply obtaining the
modelview matrix, $M$ and multiplying it by the vector,
$(0,0,1)^T$ or equivalently the third row of $M$.

\begin{figure}
\begin{tabular}{ccc}
\includegraphics[width=2.0in]{figures/figDIR1a}&
\includegraphics[width=2.0in]{figures/figDIR1b}&
\includegraphics[width=2.0in]{figures/figDIR1c}\\
a) smoke planes parallel to $y$ axis& b) smoke planes parallel to
$y=x$ axis)&
c) smoke planes parallel to $x$ axis\\
\end{tabular}
\caption{View of smoke planes from above.  Smoke Planes are
oriented so that they are {\em most perpendicular}\ to the line of sight }
\label{figDIRA}
\end{figure}

\begin{figure}
\centerline{\includegraphics[width=6.0in]{figures/figDIR2}}
\caption{Diagram illustrating the angle between the line of sight
and smoke plane normal vector.  View planes are chosen to minimize
this angle.} \label{figDIRB}
\end{figure}

%
% -------------------  Compression ------------------------
%

\subsection{Compressing Smoke Data}

The opacity parameters are computed at each node for all time
steps. As noted earlier, the space required to store these values
can easily exceed the 2GB file size limit found on 32 bit
operating systems. Techniques are required to reduce storage
requirements.

Compression for this application occurs in two steps.  First, a
four to one compression level is achieved by using Beer's law to
convert soot density, a four byte floating point quantity, to
opacity, a one byte quantity.  Video cards presently use only one
byte to represent opacity. Next, the sequence of opacity values
are compressed using run-length encoding. Run length encoding
works as follows.

\begin{enumerate}
\item Represent four or more consecutive identical characters as
$\# n c$ where $\#$ is a special character denoting the beginning
of a repeated sequence, $n$ is the number of repeats and $c$ is
the character repeated.  $n$ can be up to 254 (255 is used to
represent the {\em special}\ character). \item Represent
characters not repeated four or more times as is.
\end{enumerate}

So, the character string {\tt aaaaaabbbbcc}\ would be encoded {\tt \#6b\#4bcc}.

Run length encoding provides a reasonably good compression ratio,
is simple to implement and more importantly can be decompressed
quickly. This last property is important for any compression
scheme chosen because it is a rate limiting step in the process
that Smokeview uses to display smoke data. The CPU time required
to compute the smoke flow can easily exceed one minute of CPU time
per outputted time step, so extra time used to produce a more
compact file is affordable. However, each data frame is
decompressed {\em on the fly}\ so a compression format that can be
rapidly decompressed is critical.

Smokezip is a software program developed as a companion to
Smokeview and FDS to compress FDS data files. A second compression
scheme is used by Smokezip to compress FDS files even more
compactly.  Smokezip uses the ZLIB compression library available
at (\hhref{http://zlib.net/}).

\section{Volume Rendered Smoke}
\subsection{Terminology}
\subsection{Radiation Transport Equation}
\newcommand{\siga}{ \sigma_a(x) }
\newcommand{\sigt}{ \sigma_t(x) }
\newcommand{\sigs}{ \sigma_s(x) }
\newcommand{\sigts}{ \sigma_t(s) }
\newcommand{\Le}{ L_e(x) }
\newcommand{\Lexo}{ L_e(x,\omega) }
\newcommand{\Lxo}{ L(x,\omega) }
\newcommand{\dLdx}{ \frac{dL}{dx}(x)}
\newcommand{\intf}[2]{ \exp\left({\int_#1^#2 \sigts ds}\right) }
\newcommand{\intff}[2]{ {\int_#1^#2 \sigts ds} }
\newcommand{\intmf}[2]{ \exp\left({-\int_#1^#2 \sigts ds}\right) }
\newcommand{\intmff}[2]{ {-\int_#1^#2 \sigts ds} }

The radiation transport equation is used to calculate radiance
within a region with participating media due to one or more light
sources. The change in radiance along a ray with direction
$\omega$ may be expressed by the radiation transport equation
given by

\begin{eqnarray}
\label{eq:fullrte}
 \left(\omega\cdot\nabla\right)\Lxo =
-\underbrace{\siga\Lxo}_\mathrm{absorption}-\underbrace{\sigs\Lxo}_\mathrm{out-scattering}
+ \underbrace{\siga\Lexo}_\mathrm{emission} +
\underbrace{\sigs\int_{4\pi}p(x,\omega,\omega')L_i(x,\omega')d\omega'}_\mathrm{in-scattering}
\end{eqnarray}

\noindent where, as illustrated in Figure \ref{figRadiance},
$\Lxo$ is the radiance at $x$ along a direction $\omega$, $\siga$
is the absorption coefficient, $\sigs$ is the scattering
coefficient, $\Lexo$ is the radiance emitted at $x$ in direction
$\omega$ and $p(x,\omega,\omega')$ is the fraction of light moving
along direction $\omega'$ scattered along direction $\omega$.
Absorption, and out-scattering causes radiance to decrease while
emission and in-scattering causes radiance to increase.
\begin{figure}[\figoptions]
\begin{center}
\includegraphics[width=6.0in]{figures/rte_setup}
\end{center}
\caption{xxx.} \label{figRadiance}
\end{figure}

\subsection{Approximating the Radiation Transport Equation}
\subsubsection{Absorption and Out-Scattering - Beer-Lambert Law}
Equation (\ref{eq:fullrte}) may be simplified by neglecting the
in-scattering and emission terms and combining the absorption and
out-scattering terms using $\sigt=\siga+\sigs$ to obtain

\begin{eqnarray*}
\dLdx&=&-\sigt L(x)\\
L(0)&=&L_0
\end{eqnarray*}
which has solution
\begin{equation}
\label{eq:beer1} \frac{L(D)}{L_0}=\intmf{0}{D}
\end{equation}

If $\sigt$ is constant then equation (\ref{eq:beer1}) simplifies
to
\begin{equation}
\label{eq:beer2} \frac{L(D)}{L_0}=\exp(-\sigma_t D)
\end{equation}
which is known as the Beer-Lambert law\cite{}.


\subsubsection{Absorption, Out-Scattering and Emission}
Alternatively, if the emission term is also included when
simplifying equation (\ref{eq:fullrte})  the following
differential equation results.

\begin{eqnarray}
\label{eq:simple_rte}
\dLdx&=&-\sigt L(x) + \siga L_e(x)\\
 L(0)&=&L_0\nonumber
\end{eqnarray}

This equation may be solved by rearranging terms and applying the
integrating factor $\exp(\int_0^x \sigts ds)$ to both sides to
obtain
\begin{eqnarray*}
\intf{0}{x}\left(\dLdx+\sigt L(x)\right)&=&  \intf{0}{x}\siga \Le\\
\frac{d}{dx}\left(\intf{0}{x} L(x)\right)&=& \intf{0}{x}\siga \Le
\end{eqnarray*}
Integrating both sides and substituting the integration limits
results in
\begin{eqnarray*}
\left.\intf{0}{x} L(x)\right|_0^D&=& \int_0^D\intf{0}{x}\siga \Le dx \\
\intf{0}{D} L(D)-L_0&=& \int_0^D\intf{0}{x}\siga \Le dx
\end{eqnarray*}
Solving for $L(D)$ after noting that
$\intff{0}{x}-\intff{0}{D}=-\intff{x}{D}$ results in
\begin{eqnarray*}
L(D)&=&\intmf{0}{D} L_0+ \int_0^D\intmf{x}{D}\siga \Le dx\\
\end{eqnarray*}
The radiance at $D$ is then given by
\begin{equation}
 L(D)=\tau(0,D)L_0 + \int_0^D\tau(x,D)\siga\Le dx
\end{equation}
where $\tau(a,b)$ is the optical depth between $a$ and $b$ given
by $\tau(a,b)=\intmf{a}{b}$ .

\begin{figure}[\figoptions]
\begin{center}
\includegraphics[width=4.0in]{figures/smoke_setup2}
\end{center}
\caption {Smoke is drawn by blending the smoke color with the
background color.  The amount of blending depends on the amount of
obscuration as determined from soot density and path length.}
\label{figsmokesetup2}
\end{figure}

\subsection{Combining Multiple Solutions}
Equation (\ref{eq:rtesoln}), is the solution to an approximate form of the radiation transport equation over a path
from $x=0$ to $x=D$.  As illustrated in Figure \ref{figsmokesetup3}, 
suppose that this path is split at $D_1$ into two parts, the first part from $0$ to $D_1$ and the second part from $D_1$ to $D$.  
Further, suppose that the RTE solution is known on each part.
This section describes how these two solutions may be combined to form a solution
over the entire interval $0$ to $D$.

Given a radiance of $L_0$ at $x=0$, the radiance at $D$, denoted $L(D)$ is given by
\begin{eqnarray}
\label{eq:rtesoln}
L(D)&=&\tau(0,D)L_0 + \int_0^D\tau(x,D)\siga\Le dx
\end{eqnarray}
This is the RTE solution over the entire interval $0$ to $D$.
Similarly, $D_1$ and $D$ is given by
\begin{eqnarray*}
L_1(D_1)&=&\tau(0,D_1)L_0 + \int_0^{D_1}\tau(x,D_1)\siga\Le dx\\
L_2(D)&=&\tau(D_1,D)L_0 + \int_{D_1}^{D}\tau(x,D)\siga\Le dx
\end{eqnarray*}
where the initial conditions are $L_1(0)=L_0$ and $L_2(D_1)=L_0$.

The optical depth, $\tau(a,b)$, may be expressed as $\tau(a,b)=\tau(a,x)\tau(x,b)$ since
\begin{eqnarray*}
\tau(a,b)&=&\exp\left(-\int_a^b \sigt ds\right)\\
&=&\exp\left(-\int_a^x \sigt ds -\int_x^b \sigt ds\right)\\
&=&\exp\left(-\int_a^x \sigt ds\right)\exp\left(-\int_x^b \sigt ds\right)\\
&=&\tau(a,x)\tau(x,b)
\end{eqnarray*}
Using this property, the RTE solution, $L(x)$, over the full interval may be re-written in terms of solutions $L_1(x)$ and $L_2(x)$ using
\begin{eqnarray*}
L(D)&=&\tau(0,D)L_0 + \int_0^D\tau(x,D)\siga\Le dx\\
&=&\tau(0,D_1)\tau(D_1,D)L_0 + \int_0^{D_1}\tau(x,D_1)\tau(D_1,D)\Le dx + \int_{D_1}^{D}\tau(x,D)\Le dx\\
&=&\tau(D_1,D)\left(\tau(0,D_1)L_0 + \int_0^{D_1}\tau(x,D_1)\Le dx\right) + \int_{D_1}^{D}\tau(x,D)\Le dx\\
&=&\tau(D_1,D)L_1(D_1)+L_2(D)-\tau(D_1,D)L_0\\
&=&\tau(D_1,D)(L_1(D_1)-L_0) + L_2(D)
\end{eqnarray*}
This relationship may be exploited to draw smoke one mesh at a time using local properties of each mesh without having to solve the RTE equation over the entire interval at once.

\begin{figure}[\figoptions]
\begin{center}
\includegraphics[width=4.0in]{figures/smoke_setup3}
\end{center}
\caption {Smoke is drawn by blending the smoke color with the
background color.  The amount of blending depends on the amount of
obscuration as determined from soot density and path length.}
\label{figsmokesetup3}
\end{figure}
