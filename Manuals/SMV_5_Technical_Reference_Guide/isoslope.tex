
\begin{figure}[\figoptions]
\begin{center}
\includegraphics[width=5.0in]{figures/3point_line_smooth}
\end{center}
\caption{Setup for determining the slope of a smooth curve passing through three points.}
\label{figlinesmooth}%
\end{figure}

The use of harmonic averages to construct average normal vectors at isosurface vertices is justified by the following two
dimensional example.  As illustrated in Figure \ref{figlinesmooth},
consider the quadratic curve $y(x)=A+Bx+Cx^2$ constrained by $y(-\Delta x_2)=0$, $y(0)=1$ and $y(\Delta x_1)=0$ .  The slope of this
curve at $x=0$ is given by $y'(0)=B$.  Likewise, the slope of the vector perpendicular to this curve at $x=0$ is
$-1/B$.  The coefficient $B$ may be determined from the two simultaneous equations $y(\Delta
x_1)=0$ and  $(y(-\Delta x_2)=0$ (note that $A=1$ since $y(0)=1$) or
\begin{eqnarray*}
1+B\Delta x_1 + C \Delta x_1^2 &= &0\\
1-B\Delta x_2 + C \Delta x_2^2 &= &0
\end{eqnarray*}
which has solution
\begin{eqnarray*}
B&=&\frac{\Delta x_1^2-\Delta x_2^2}{\Delta x_1\Delta x_2(\Delta x_1+\Delta x_2)}=
\frac{\Delta x_1-\Delta x_2}{\Delta x_1\Delta x_2}\\
\end{eqnarray*}
The slope $N$ of the normal at $x=0$ is then given by
\begin{eqnarray*}
N=-\frac{1}{B}&=&\frac{1}{1/\Delta x_1-1/\Delta x_2}
\end{eqnarray*}
The slope $N_1$ of the normal to the line segment between $(0,1)$ and $(\Delta x_1,0)$ is $N_1=\Delta x_1$.
The slope $N_2$ of the normal to the line segment between $(-\Delta x_2,0)$ and (0,1) is $N_2=-\Delta x_2$.
Therefore the average slope $N$ may be expressed in terms of $N_1$ and $N_2$ as
\begin{eqnarray*}
N=\frac{1}{1/N_1+1/N_2}
\end{eqnarray*}
