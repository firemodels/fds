% !TEX root = Correlation_Guide.tex

\chapter{Smoke Detector Activation Time}
\label{Smoke_Detector_Activation_Time_Chapter}

\section{Temperature Rise Method}

\subsection*{Description}

In this method, the prediction of smoke detector activation time is identical to that for a sprinkler (as described in Chapter~\ref{Sprinkler_Activation_Time_Chapter}). Bukowski and Averill~\cite{Bukowski:2} suggest an activation temperature of 5~$^\circ$C to be typical of many residential smoke alarms. It is assumed that the smoke detectors are low-RTI devices ($\textrm{RTI}=5~\si{(m.s)^{1/2}}$).

Note that some of these cases assume a quasi-steady approach for a fire source $\dot Q$ that follows a specified t-squared growth, which was specified as $\dot Q = \alpha t^2$ up to a cutoff time of $t\sb{fire}$. After the time $t\sb{fire}$, the fire HRR was steady.

\subsection*{Verification}

Test: NIST Smoke Alarms, Test SDC02

\begin{table}[!ht]
\caption[Verification input parameters, smoke detector activation time]
{Verification input parameters, smoke detector activation time.}
\begin{center}
\begin{tabular}{|l|c|}
\hline
                              &              \\
\rb{Input Parameter}          &  \rb{Value}  \\ \hline \hline
$\alpha$ (kW/s$^2$)           &  0.00463     \\ \hline
Location Factor (-)           &  1           \\ \hline
$t\sb{fire}$ (s)              &  300         \\ \hline
$r$ (m)                       &  1.3         \\ \hline
$H$ (m)                       &  2.1         \\ \hline
$\Delta T\sb{c}$ ($^\circ$C)  &  5           \\ \hline
RTI (\si{(m.s)^{1/2}})        &  5           \\ \hline
$T_\infty$ ($^\circ$C)        &  21          \\ \hline
\end{tabular}
\end{center}
\end{table}

\noindent Expected result: At 28~s, the HRR $\dot Q$ is 3.63~kW, and the activation time $t\sb{act}$ is 35.0~s.

% Note that the FDTs spreadsheet uses incorrect coefficients in Alpert's ceiling jet correlation, which gives an activation time of 35.12 s.


\clearpage


\subsection*{Validation}

\begin{figure}[!ht]
\begin{center}
\begin{tabular}{l}
\includegraphics[width=4.0in]{SCRIPT_FIGURES/Scatterplots/Smoke_Detector_Activation_Time_Temperature_Rise}
\end{tabular}
\end{center}
\caption[Summary of smoke detector activation time predictions (Temperature Rise)]
{Summary of smoke detector activation time predictions using the Temperature Rise method.}
\label{Smoke_Detector_Activation_Summary_Temperature_Rise}
\end{figure}


\clearpage


\section{Milke Method}

\subsection*{Description}

The correlation of Milke~\cite{Milke:1} predicts that the time of smoke detector activation, $t\sb{act}$, is given by
\be
t\sb{act} = \frac{X H^{4/3}}{\dot Q^{1/3}}
\label{eq:Milke}
\ee
where $H$ is the height of the ceiling above the top of the fuel~(\si{ft}), and $\dot Q$ is the HRR~(\si{kW}). The constant $X$ is given by
\be
X = 4.6 \times 10^{-4} (Y^2) + 2.7 \times 10^{-15} (Y^6)
\label{eq:Milke_X}
\ee
and the constant $Y$ is given by
\be
Y = \frac{\Delta T\sb{c} H^{5/3}}{\dot Q^{2/3}}
\label{eq:Milke_Y}
\ee
where $\Delta T\sb{c}$ is the temperature rise of gases under the ceiling required for the smoke detector to activate~($^\circ$F).

Note that some of these cases assume a quasi-steady approach for a fire source $\dot Q$ that follows a specified t-squared growth, which was specified as $\dot Q = \alpha t^2$ up to a cutoff time of $t\sb{fire}$. After the time $t\sb{fire}$, the fire HRR was steady.

\subsection*{Verification}

Test: NIST Smoke Alarms, Test SDC02

\begin{table}[!ht]
\caption[Verification input parameters, smoke detector activation time]
{Verification input parameters, smoke detector activation time.}
\begin{center}
\begin{tabular}{|l|c|}
\hline
                              &              \\
\rb{Input Parameter}          &  \rb{Value}  \\ \hline \hline
$\alpha$ (kW/s$^2$)           &  0.00463     \\ \hline
$t\sb{fire}$ (s)              &  300         \\ \hline
$H$ (m)                       &  2.1         \\ \hline
$\Delta T\sb{c}$ ($^\circ$C)  &  5           \\ \hline
\end{tabular}
\end{center}
\end{table}

\noindent Expected result: At 30~s, the HRR $\dot Q$ is 4.2~kW, and the activation time $t\sb{act}$ is 30.8~s.

% Note: The FDTs spreadsheets use an incorrect unit conversion factor from kW to Btu/s, which gives an activation time of 25.4~s.


\clearpage


\subsection*{Validation}

\begin{figure}[!ht]
\begin{center}
\begin{tabular}{l}
\includegraphics[width=4.0in]{SCRIPT_FIGURES/Scatterplots/Smoke_Detector_Activation_Time_Milke}
\end{tabular}
\end{center}
\caption[Summary of smoke detector activation time predictions (Milke)]
{Summary of smoke detector activation time predictions using the method of Milke.}
\label{Smoke_Detector_Activation_Summary_Milke}
\end{figure}


\clearpage


\section{Mowrer Method}

\subsection*{Description}

The correlation of Mowrer~\cite{Mowrer:1} predicts that the time of smoke detector activation, $t\sb{act}$, is given by
\be
t\sb{act} = t\sb{pl} + t\sb{cj}
\label{eq:Mowrer}
\ee
where the transport lag time of the plume, $t\sb{pl}$, is given by
\be
t\sb{pl} = C\sb{pl} \frac{H^{4/3}}{\dot Q^{1/3}}
\label{eq:Mowrer_tpl}
\ee
where $C\sb{pl}$ is the plume lag time constant~(0.67), $H$ is the height of the ceiling above the fuel~(\si{m}), and $\dot Q$ is the HRR~(\si{kW}).
The transport lag time of the ceiling jet, $t\sb{cj}$, is given by
\be
t\sb{cj} = \frac{1}{C\sb{cj}} \frac{r^{11/6}}{\dot Q^{1/3} H^{1/2}}
\label{eq:Mowrer_tcj}
\ee
where $C\sb{cj}$ is the ceiling jet time lag time constant~(1.2), and $r$ is the radial distance to the detector~(\si{m}).

Note that some of these cases assume a quasi-steady approach for a fire source $\dot Q$ that follows a specified t-squared growth, which was specified as $\dot Q = \alpha t^2$ up to a cutoff time of $t\sb{fire}$. After the time $t\sb{fire}$, the fire HRR was steady.

\subsection*{Verification}

Test: NIST Smoke Alarms, Test SDC02

\begin{table}[!ht]
\caption[Verification input parameters, smoke detector activation time]
{Verification input parameters, smoke detector activation time.}
\begin{center}
\begin{tabular}{|l|c|}
\hline
                      &              \\
\rb{Input Parameter}  &  \rb{Value}  \\ \hline \hline
$\alpha$ (kW/s$^2$)   &  0.00463     \\ \hline
$t\sb{fire}$ (s)      &  300         \\ \hline
$C\sb{pl}$ (-)        &  0.67        \\ \hline
$C\sb{cj}$ (-)        &  1.2         \\ \hline
$r$ (m)               &  1.3         \\ \hline
$H$ (m)               &  2.1         \\ \hline
\end{tabular}
\end{center}
\end{table}

\noindent Expected result: At 5~s, the HRR $\dot Q$ is 0.116~kW, and the activation time $t\sb{act}$ is 5.6~s.


\clearpage


\subsection*{Validation}

\begin{figure}[!ht]
\begin{center}
\begin{tabular}{l}
\includegraphics[width=4.0in]{SCRIPT_FIGURES/Scatterplots/Smoke_Detector_Activation_Time_Mowrer}
\end{tabular}
\end{center}
\caption[Summary of smoke detector activation time predictions (Mowrer)]
{Summary of smoke detector activation time predictions using the method of Mowrer (bottom).}
\label{Smoke_Detector_Activation_Summary_Mowrer}
\end{figure}