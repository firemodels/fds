\documentclass[11pt]{book}

%%%%%%%%%%%%%%%%%%%%%%%%%%%%%%%%%%%%%%%%%%%%%%%%%%%%%%%%%%%%%%%%%%%%%%%%%%%%%%%%%%%%%%%%%%%%%%%%%%%
%                                                                                                 %
% The mathematical style of these documents follows                                               %
%                                                                                                 %
% A. Thompson and B.N. Taylor. The NIST Guide for the Use of the International System of Units.   %
%    NIST Special Publication 881, 2008.                                                          %
%                                                                                                 %
% http://www.nist.gov/pml/pubs/sp811/index.cfm                                                    %
%                                                                                                 %
%%%%%%%%%%%%%%%%%%%%%%%%%%%%%%%%%%%%%%%%%%%%%%%%%%%%%%%%%%%%%%%%%%%%%%%%%%%%%%%%%%%%%%%%%%%%%%%%%%%

% $Date$
% $Revision$
% $Author$

%%%%%%%%%%%%%%%%%%%%%%%%%%%%%%%%%%%%%%%%%%%%%%%%%%%%%%%%%%%%%%%%%%%%%%%%%%%%%%%%%%%%%%%%%%%%%%%%%%%
%                                                                                                 %
% The mathematical style of these documents follows                                               %
%                                                                                                 %
% A. Thompson and B.N. Taylor. The NIST Guide for the Use of the International System of Units.   %
%    NIST Special Publication 881, 2008.                                                          %
%                                                                                                 %
% http://www.nist.gov/pml/pubs/sp811/index.cfm                                                    %
%                                                                                                 %
%%%%%%%%%%%%%%%%%%%%%%%%%%%%%%%%%%%%%%%%%%%%%%%%%%%%%%%%%%%%%%%%%%%%%%%%%%%%%%%%%%%%%%%%%%%%%%%%%%%

% Packages which force the use of better TeX coding
% Mostly from http://tex.stackexchange.com/q/19264
%%\RequirePackage[l2tabu, orthodox]{nag}
%%\usepackage{fixltx2e}
%\usepackage{isomath} % Disabled for the moment because it changes the syntax for bold and roman Greek math symbols
%%\usepackage[all,warning]{onlyamsmath}
%\usepackage{strict} % Commented out for now because it is uncommon. A copy of style.sty is in Manuals/LaTeX_Style_Files/.

\usepackage{times,mathptmx}
\usepackage[pdftex]{graphicx} % use \usepackage[pdftex,demo]{graphicx} to suppress images
\usepackage{tabularx}
\usepackage{multirow}
%\usepackage{pdfsync}
\usepackage{tikz}
\usepackage{bm}
\usepackage{pgfplots}
%\pgfplotsset{compat=1.7}
\usepackage{tocloft}
\usepackage{color}
\definecolor{linknavy}{rgb}{0,0,0.50196}
\definecolor{linkred}{rgb}{1,0,0}
\definecolor{linkblue}{rgb}{0,0,1}
\usepackage{amsmath}
\usepackage{cancel}
\usepackage{float}
\usepackage{caption}
\usepackage{pict2e}
\usepackage{graphpap}
\usepackage{rotating}
\usepackage{geometry}
\usepackage{relsize}
\usepackage{longtable}
\usepackage{xltabular}
\usepackage{lscape}
\usepackage{booktabs}
\usepackage{colortbl}
\definecolor{lavender}{rgb}{0.9, 0.9, 0.98}
\usepackage{amssymb}
\usepackage{threeparttable}
\usepackage{makeidx} % Create index at end of document
\usepackage[nottoc,notlof,notlot]{tocbibind} % Put the bibliography and index in the ToC
\usepackage{lastpage} % Automatic last page number reference.
\usepackage[T1]{fontenc}
\usepackage{enumerate}
\usepackage{upquote}
\usepackage{moreverb}
\usepackage{morefloats}
\usepackage[section]{placeins}
\usepackage{scrextend}
\usepackage{needspace}
\usepackage[backend=biber, style=numeric, sorting=none, backref=true]{biblatex}

\newcommand{\nopart}{\expandafter\def\csname Parent-1\endcsname{}} % To fix table of contents in pdf.
\newcommand{\ct}[1]{\lstinline{#1}}
\newcommand{\tct}[1]{\lstinline[basicstyle=\scriptsize\ttfamily]!#1!}

\usepackage{siunitx}

\usepackage{listings}
\usepackage{textcomp}
\lstset{
    tabsize=4,
    rulecolor=,
    language=Fortran,
        basicstyle=\small\ttfamily,
        upquote=true,
        aboveskip={\baselineskip},
        belowskip={\baselineskip},
        columns=fixed,
        extendedchars=true,
        breaklines=true,
        breakatwhitespace=true,
        frame=none,
        showtabs=false,
        showspaces=false,
        showstringspaces=false,
        identifierstyle=\ttfamily,
        keywordstyle=\color[rgb]{0,0,0},
        commentstyle=\color[rgb]{0,0,0},
        stringstyle=\color[rgb]{0,0,0},
        literate={\_}{}{0\discretionary{\_}{}{\_}}
                 {/}{}{0\discretionary{/}{}{/}}%
}

\usepackage{xr-hyper}
\usepackage[pdftex,
        colorlinks=true,
        urlcolor=linkblue,     % \href{...}{...} external (URL)
        citecolor=linkred,     % citation number colors
        linkcolor=linknavy,    % \ref{...} and \pageref{...}
        pdfproducer={pdflatex},
        pdfpagemode=UseNone,
        bookmarksopen=true,
        plainpages=false,
        verbose]{hyperref}

% The Following commented code makes the ``Draft'' watermark on each page.
%\usepackage{eso-pic}
%\usepackage{type1cm}
%\makeatletter
%   \AddToShipoutPicture{
%     \setlength{\@tempdimb}{.5\paperwidth}
%     \setlength{\@tempdimc}{.5\paperheight}
%     \setlength{\unitlength}{1pt}
%     \put(\strip@pt\@tempdimb,\strip@pt\@tempdimc){
%     \makebox(0,0){\rotatebox{45}{\textcolor[gray]{0.75}{\fontsize{8cm}\selectfont{RC6}}}}}
% }
%\makeatother

\captionsetup[figure]{font=small}

\setlength{\textwidth}{6.5in}
\setlength{\textheight}{9.0in}
\setlength{\topmargin}{0.in}
\setlength{\headheight}{0.in}
\setlength{\headsep}{0.in}
\setlength{\parindent}{0.25in}
\setlength{\oddsidemargin}{0.0in}
\setlength{\evensidemargin}{0.0in}
\setlength{\leftmargini}{\parindent}        % Controls the indenting of the "bullets" in a list
\cftsetindents{section}{.25in}{0.40in}      % Distance from left margin to section number; Width of section number and space before section title
\cftsetindents{subsection}{0.65in}{0.60in}  % Distance from left margin to subsection number; Width of subsection number and space before subsection title
\setlength{\cftfignumwidth}{0.45in}         % Width of figure number and space before figure caption in the list of figures
\setlength{\cfttabnumwidth}{0.45in}         % Width of table number and space before table caption in the list of tables

\makeatletter
\setlength{\@fptop}{0pt}                    % Figures on separate pages pushed to the top
\setlength{\@fpbot}{0pt plus 1fil}
\makeatother

\newcommand{\authortitlesigs}
{
\begin{flushright}
Kevin McGrattan \\
Simo Hostikka \\
Jason Floyd \\
Randall McDermott \\
Marcos Vanella \\
Eric Mueller \\
Chandan Paul
\end{flushright}
}

\newcommand{\logosigs}{
\begin{minipage}[b]{6.25in}
\parbox[b]{.5\textwidth}{\flushleft{\includegraphics[height=1.5in]{../Bibliography/FDS_Logo_lock}}}
\hfill
\parbox[b]{.5\textwidth}{\flushright{\includegraphics[height=1in]{../Bibliography/nistident_flright_vec}}}
\end{minipage}
}

\newcommand{\authorsigs}
{
\begin{flushright}
Kevin McGrattan \\
Randall McDermott \\
Marcos Vanella \\
Eric Mueller \\
{\em Fire Research Division, Engineering Laboratory, Gaithersburg, Maryland} \\[.1in]
Simo Hostikka \\
{\em Aalto University, Espoo, Finland} \\[.1in]
Jason Floyd \\
{\em Fire Safety Research Institute, UL Research Institutes, Columbia, Maryland} \\[.1in]
Chandan Paul \\
{\em The George Washington University, Washington, D.C.}
\end{flushright}
}

\newcommand{\titlesigs}
{
\small
\begin{flushright}
U.S. Department of Commerce \\
{\em Howard Lutnick, Secretary} \\
\hspace{1in} \\
National Institute of Standards and Technology \\
{\em Craig Burkhardt, Acting NIST Director and Acting Under Secretary of Commerce for Standards and Technology}
\end{flushright}
}


\newcommand{\disclaimer}[1]
{
\begin{minipage}[t]{6.25in}
\fontsize{10}{12}\selectfont
\begin{flushright}
Certain commercial entities, equipment, or materials may be identified in this \\
document in order to describe an experimental procedure or concept adequately. \\
Such identification is not intended to imply recommendation or endorsement by the \\
National Institute of Standards and Technology, nor is it intended to imply that the \\
entities, materials, or equipment are necessarily the best available for the purpose.
\end{flushright}
\vspace{3in}
\large
\flushright{\bf National Institute of Standards and Technology Special Publication #1 \\
Natl.~Inst.~Stand.~Technol.~Spec.~Publ.~#1, \pageref{LastPage} pages (October 2013) \\
CODEN: NSPUE2}
\vfill
\hspace{1in}
\end{minipage}
}



\newcommand{\gforneybio}
{
\item[Glenn Forney] is a computer scientist at the Engineering Laboratory of NIST.  He received a
bachelor of science degree in mathematics from Salisbury State College and a master of
science and a doctorate in mathematics from Clemson University.  He joined NIST
in 1986 (then the National Bureau of Standards) and has since worked on developing tools that
provide a better understanding of fire phenomena, most notably Smokeview, a software tool for visualizing
Fire Dynamics Simulator data.
}

\newcommand{\smvoverview}
{
This guide is part of a three volume set of companion documents describing how to use Smokeview
in Volume I, the Smokeview User's Guide~\cite{Smokeview_Users_Guide}, describing technical details of how the visualizations are performed in Volume II, the Smokeview Technical Reference Guide~\cite{Smokeview_Tech_Guide}, and presents example cases
verifying the various visualization capabilities of Smokeview in Volume III, the Smokeview Verification Guide~\cite{Smokeview_Verification_Guide}.  Details on the use and technical background of the Fire Dynamics Simulator is contained in the FDS User's~\cite{FDS_Users_Guide} and Technical reference guide~\cite{FDS_Math_Guide}
respectively.
}

% commands to use for "official" cover and title pages
% see smokeview verification guide to see how they are used

\newcommand{\headerA}[1]{
\begin{flushright}
\fontsize{20}{24}\selectfont
\bf{NIST Special Publication #1}
\end{flushright}
}


\newcommand{\headerB}[1]{
\begin{flushright}
\fontsize{28}{33.6}\selectfont
\bf{#1}
\end{flushright}
}

\newcommand{\headerC}[1]{
\vspace{.15in}
\begin{flushright}
\fontsize{12}{14}\selectfont
#1
\end{flushright}
}

\newcommand{\headerD}[1]{
\begin{flushright}
\fontsize{12}{14}\selectfont
http://dx.doi.org/10.6028/NIST.SP.#1
\end{flushright}
}



\newcommand{\dod}[2]{\frac{\partial #1}{\partial #2}}
\newcommand{\DoD}[2]{\frac{\mathrm{D} #1}{\mathrm{D} #2}}
\newcommand{\dsods}[2]{\frac{\partial^2 #1}{\partial #2^2}}
\renewcommand{\d}{\,\mathrm{d}}
\newcommand{\dx}{\delta x}
\newcommand{\dy}{\delta y}
\newcommand{\dz}{\delta z}
\newcommand{\degF}{$^\circ$F}
\newcommand{\degC}{$^\circ$C}
\newcommand{\x}{x}
\newcommand{\y}{y}
\newcommand{\z}{z}
\newcommand{\dt}{\delta t}
\newcommand{\dn}{\delta n}
\newcommand{\cH}{H}
\newcommand{\hu}{u}
\newcommand{\hv}{v}
\newcommand{\hw}{w}
\newcommand{\la}{\lambda}
\newcommand{\bO}{{\Omega}}
\newcommand{\bo}{{\mathbf{\omega}}}
\newcommand{\btau}{\mathbf{\tau}}
\newcommand{\bdelta}{{\mathbf{\delta}}}
\newcommand{\sumyw}{\sum (Y_\alpha/W_\alpha)}
\newcommand{\oW}{\overline{W}}
\newcommand{\om}{\ensuremath{\omega}}
\newcommand{\omx}{\omega_x}
\newcommand{\omy}{\omega_y}
\newcommand{\omz}{\omega_z}
\newcommand{\erf}{\hbox{erf}}
\newcommand{\erfc}{\hbox{erfc}}
\newcommand{\bF}{{\mathbf{F}}}
\newcommand{\bG}{{\mathbf{G}}}
\newcommand{\bof}{{\mathbf{f}}}
\newcommand{\bq}{{\mathbf{q}}}
\newcommand{\br}{{\mathbf{r}}}
\newcommand{\bu}{{\mathbf{u}}}
\newcommand{\bx}{{\mathbf{x}}}
\newcommand{\bk}{{\mathbf{k}}}
\newcommand{\bv}{{\mathbf{v}}}
\newcommand{\bg}{{\mathbf{g}}}
\newcommand{\bn}{{\mathbf{n}}}
\newcommand{\bS}{{\mathbf{S}}}
\newcommand{\bW}{\overline{W}}
\newcommand{\dS}{d{\mathbf{S}}}
\newcommand{\bs}{{\mathbf{s}}}
\newcommand{\bI}{{\mathbf{I}}}
\newcommand{\hp}{H}
\newcommand{\trho}{\tilde{\rho}}
\newcommand{\dph}{{\delta\phi}}
\newcommand{\dth}{{\delta\theta}}
\newcommand{\tp}{\tilde{p}}
\newcommand{\bp}{\overline{p}}
\newcommand{\dQ}{\dot{Q}}
\newcommand{\dq}{\dot{q}}
\newcommand{\dbq}{\dot{\mathbf{q}}}
\newcommand{\dm}{\dot{m}}
\newcommand{\ha}{\frac{1}{2}}
\newcommand{\ft}{\frac{4}{3}}
\newcommand{\ot}{\frac{1}{3}}
\newcommand{\fofi}{\frac{4}{5}}
\newcommand{\of}{\frac{1}{4}}
\newcommand{\twth}{\frac{2}{3}}
\newcommand{\R}{R}
\newcommand{\be}{\begin{equation}}
\newcommand{\ee}{\end{equation}}
\newcommand{\RE}{\hbox{Re}}
\newcommand{\LE}{\hbox{Le}}
\newcommand{\PR}{\hbox{Pr}}
\newcommand{\PE}{\hbox{Pe}}
\newcommand{\NU}{\hbox{Nu}}
\newcommand{\SC}{\hbox{Sc}}
\newcommand{\SH}{\hbox{Sh}}
\newcommand{\WE}{\hbox{We}}
\newcommand{\OI}{\text{\tiny \hbox{OI}}}
\newcommand{\COTWO}{\text{\tiny \hbox{CO}$_2$}}
\newcommand{\HTWOO}{\text{\tiny \hbox{H}$_2$\hbox{O}}}
\newcommand{\OTWO}{\text{\tiny \hbox{O}$_2$}}
\newcommand{\NTWO}{\text{\tiny \hbox{N}$_2$}}
\newcommand{\CO}{\text{\tiny \hbox{CO}}}
\newcommand{\HCN}{\text{\tiny \hbox{HCN}}}
\newcommand{\F}{\text{\tiny \hbox{F}}}
\newcommand{\C}{\text{\tiny \hbox{C}}}
\newcommand{\Hy}{\text{\tiny \hbox{H}}}
\newcommand{\So}{\text{\tiny \hbox{S}}}
\newcommand{\M}{\text{\tiny \hbox{M}}}
\newcommand{\xx}{\text{\tiny \hbox{x}}}
\newcommand{\yy}{\text{\tiny \hbox{y}}}
\newcommand{\zz}{\text{\tiny \hbox{z}}}
\newcommand{\smvlines}{120~000}

\newcommand{\calH}{\mathcal{H}}
\newcommand{\calR}{\mathcal{R}}

\newcommand{\dif}{\mathrm{d}}
\newcommand{\Div}{\nabla\cdot}
\newcommand{\D}{\mbox{D}}
\newcommand{\mhalf}{\mbox{$\frac{1}{2}$}}
\newcommand{\thalf}{\mbox{\tiny $\frac{1}{2}$}}
\newcommand{\tripleprime}{{\prime\prime\prime}}
\newcommand{\ppp}{{\prime\prime\prime}}
\newcommand{\pp}{{\prime\prime}}

\newcommand{\superscript}[1]{\ensuremath{^{\textrm{\tiny #1}}}}
\newcommand{\subscript}[1]{\ensuremath{_{\textrm{\tiny #1}}}}

\newcommand{\rb}[1]{\raisebox{1.5ex}[0pt]{#1}}

\newcommand{\Ra}{$\Rightarrow$}
\newcommand{\hhref}[1]{\href{#1}{{\tt #1}}}
\newcommand{\fdsinput}[1]{{\scriptsize\verbatiminput{../../Verification/Visualization/#1}}}

\definecolor{AQUAMARINE}{rgb}{0.49804,1.00000,0.83137}
\definecolor{ANTIQUE WHITE}{rgb}{0.98039,0.92157,0.84314}
\definecolor{BEIGE}{rgb}{0.96078,0.96078,0.86275}
\definecolor{BLACK}{rgb}{0.00000,0.00000,0.00000}
\definecolor{BLUE}{rgb}{0.00000,0.00000,1.00000}
\definecolor{BLUE VIOLET}{rgb}{0.54118,0.16863,0.88627}
\definecolor{BRICK}{rgb}{0.61176,0.40000,0.12157}
\definecolor{BROWN}{rgb}{0.64706,0.16471,0.16471}
\definecolor{BURNT SIENNA}{rgb}{0.54118,0.21176,0.05882}
\definecolor{BURNT UMBER}{rgb}{0.54118,0.20000,0.14118}
\definecolor{CADET BLUE}{rgb}{0.37255,0.61961,0.62745}
\definecolor{CHOCOLATE}{rgb}{0.82353,0.41176,0.11765}
\definecolor{COBALT}{rgb}{0.23922,0.34902,0.67059}
\definecolor{CORAL}{rgb}{1.00000,0.49804,0.31373}
\definecolor{CYAN}{rgb}{0.00000,1.00000,1.00000}
\definecolor{DIM GRAY }{rgb}{0.41176,0.41176,0.41176}
\definecolor{EMERALD GREEN}{rgb}{0.00000,0.78824,0.34118}
\definecolor{FIREBRICK}{rgb}{0.69804,0.13333,0.13333}
\definecolor{FLESH}{rgb}{1.00000,0.49020,0.25098}
\definecolor{FOREST GREEN}{rgb}{0.13333,0.54510,0.13333}
\definecolor{GOLD }{rgb}{1.00000,0.84314,0.00000}
\definecolor{GOLDENROD}{rgb}{0.85490,0.64706,0.12549}
\definecolor{GRAY}{rgb}{0.50196,0.50196,0.50196}
\definecolor{GREEN}{rgb}{0.00000,1.00000,0.00000}
\definecolor{GREEN YELLOW}{rgb}{0.67843,1.00000,0.18431}
\definecolor{HONEYDEW}{rgb}{0.94118,1.00000,0.94118}
\definecolor{HOT PINK}{rgb}{1.00000,0.41176,0.70588}
\definecolor{INDIAN RED}{rgb}{0.80392,0.36078,0.36078}
\definecolor{INDIGO}{rgb}{0.29412,0.00000,0.50980}
\definecolor{IVORY}{rgb}{1.00000,1.00000,0.94118}
\definecolor{IVORY BLACK}{rgb}{0.16078,0.14118,0.12941}
\definecolor{KELLY GREEN}{rgb}{0.00000,0.50196,0.00000}
\definecolor{KHAKI}{rgb}{0.94118,0.90196,0.54902}
\definecolor{LAVENDER}{rgb}{0.90196,0.90196,0.98039}
\definecolor{LIME GREEN}{rgb}{0.19608,0.80392,0.19608}
\definecolor{MAGENTA}{rgb}{1.00000,0.00000,1.00000}
\definecolor{MAROON}{rgb}{0.50196,0.00000,0.00000}
\definecolor{MELON}{rgb}{0.89020,0.65882,0.41176}
\definecolor{MIDNIGHT BLUE}{rgb}{0.09804,0.09804,0.43922}
\definecolor{MINT}{rgb}{0.74118,0.98824,0.78824}
\definecolor{NAVY}{rgb}{0.00000,0.00000,0.50196}
\definecolor{OLIVE}{rgb}{0.50196,0.50196,0.00000}
\definecolor{OLIVE DRAB}{rgb}{0.41961,0.55686,0.13725}
\definecolor{ORANGE}{rgb}{1.00000,0.50196,0.00000}
\definecolor{ORANGE RED}{rgb}{1.00000,0.27059,0.00000}
\definecolor{ORCHID}{rgb}{0.85490,0.43922,0.83922}
\definecolor{PINK}{rgb}{1.00000,0.75294,0.79608}
\definecolor{POWDER BLUE}{rgb}{0.69020,0.87843,0.90196}
\definecolor{PURPLE}{rgb}{0.50196,0.00000,0.50196}
\definecolor{RASPBERRY}{rgb}{0.52941,0.14902,0.34118}
\definecolor{RED}{rgb}{1.00000,0.00000,0.00000}
\definecolor{ROYAL BLUE}{rgb}{0.25490,0.41176,0.88235}
\definecolor{SALMON}{rgb}{0.98039,0.50196,0.44706}
\definecolor{SANDY BROWN}{rgb}{0.95686,0.64314,0.37647}
\definecolor{SEA GREEN}{rgb}{0.32941,1.00000,0.62353}
\definecolor{SEPIA}{rgb}{0.36863,0.14902,0.07059}
\definecolor{SIENNA}{rgb}{0.62745,0.32157,0.17647}
\definecolor{SILVER}{rgb}{0.75294,0.75294,0.75294}
\definecolor{SKY BLUE}{rgb}{0.52941,0.80784,0.92157}
\definecolor{SLATEBLUE}{rgb}{0.41569,0.35294,0.80392}
\definecolor{SLATE GRAY}{rgb}{0.43922,0.50196,0.56471}
\definecolor{SPRING GREEN}{rgb}{0.00000,1.00000,0.49804}
\definecolor{STEEL BLUE}{rgb}{0.27451,0.50980,0.70588}
\definecolor{TAN}{rgb}{0.82353,0.70588,0.54902}
\definecolor{TEAL}{rgb}{0.00000,0.50196,0.50196}
\definecolor{THISTLE}{rgb}{0.84706,0.74902,0.84706}
\definecolor{TOMATO }{rgb}{1.00000,0.38824,0.27843}
\definecolor{TURQUOISE}{rgb}{0.25098,0.87843,0.81569}
\definecolor{VIOLET}{rgb}{0.93333,0.50980,0.93333}
\definecolor{VIOLET RED}{rgb}{0.81569,0.12549,0.56471}
\definecolor{WHITE}{rgb}{1.00000,1.00000,1.00000}
\definecolor{YELLOW}{rgb}{1.00000,1.00000,0.00000}

\floatstyle{boxed}
\newfloat{notebox}{H}{lon}
\newfloat{warning}{H}{low}

% Set default longtable alignment
\setlength\LTleft{0pt}
\setlength\LTright{0pt}

% Prevent large paragraph separations
\raggedbottom

% Allow multi-line equations to span page breaks
\allowdisplaybreaks

% Conditional to activate Unstructured Geometry text:
\newif\ifcompgeom
\compgeomtrue

\IfFileExists{../Bibliography/gitrevision.tex}
{\newcommand{\gitrevision}{FDS6.5.3-739-g9e39475}
}
{\newcommand{\gitrevision}{unknown} }

\externaldocument[UG-]{FDS_User_Guide}

\includeonly{Overview_Chapter,Survey_Chapter,Experiment_Chapter, Error_Chapter, HGL_Chapter, Plume_Chapter, Ceiling_Jet_Chapter,Velocity_Chapter, Species_Chapter,Pressure_Chapter, Surface_Temperature_Chapter, Heat_Flux_Chapter, Suppression_Chapter, Burning_Rate_Chapter, Wind_Chapter}

%\includeonly{Experiment_Chapter,HGL_Chapter,Velocity_Chapter}

\addbibresource{../Bibliography/FDS_refs.bib}
\addbibresource{../Bibliography/FDS_general.bib}
\addbibresource{../Bibliography/FDS_mathcomp.bib}


\begin{document}

\pagestyle{empty}

\begin{minipage}[t][9in][s]{6.5in}

\headerA{
1018-3\\
Sixth Edition\\
}

\headerB{
Fire Dynamics Simulator\\
Technical Reference Guide\\
Volume 3: Validation\\
}

\headerC{
\authortitlesigs
}

\vfill

\headerD{1018}

\vfill

\logosigs

\end{minipage}

\newpage

\hspace{5in}

\newpage

\begin{minipage}[t][9in][s]{6.5in}

\headerA{
1018-3\\
Sixth Edition\\
}

\headerB{
Fire Dynamics Simulator\\
Technical Reference Guide\\
Volume 3: Validation\\
}

\headerC{
\authorsigs
}

\headerD{1018}

\headerC{
\flushright{\today \\
Revision:~\gitrevision}}


\vfill

\flushright{\includegraphics[width=1in]{../Bibliography/doc} }

\titlesigs

\end{minipage}

\newpage

\disclaimer{1018-3}




\newpage

\frontmatter

\pagestyle{plain}

\chapter{Authors}

The Fire Dynamics Simulator and Smokeview are the products of an international collaborative effort led by
the National Institute of Standards and Technology (NIST) and VTT Technical Research Centre of Finland. Its developers and
contributors are listed below.

\vspace{0.5in}

\begin{flushleft}

Principal Developers (in alphabetical order) \\ [0.2in]

Jason Floyd, Hughes Associates, Inc., Baltimore, Maryland, USA \\
Glenn Forney, NIST \\
Simo Hostikka, VTT \\
Timo Korhonen, VTT  \\
Randall McDermott, NIST \\
Kevin McGrattan, NIST \\ [0.5in]

Contributers \\ [0.2in]

Elizabeth Blanchard, Centre Scientifique et Technique du B\^{a}timent (CSTB), Paris, France \\
Susan Killian, hhpberlin, Germany \\
Charles Luo, Global Engineering and Materials, Inc., Princeton, New Jersey, USA \\
Anna Matala, VTT \\
William Mell, U.S. Forest Service, Seattle, Washington, USA \\
Christian Rogsch, Neustadt/Wstr., Germany \\
Topi Sikanen, VTT \\
Ben Trettel, University of Maryland, USA \\
Craig Weinschenk, NIST

\end{flushleft}


\chapter{About the Authors}

\begin{description}

\item[Elizabeth Blanchard] is a fire protection engineer at the French building agency CSTB. She holds a master of science degree in mathematical modeling and a doctorate in mechanics and thermal engineering. She is mainly involved at CSTB in the research program concerning water spray.

\item[Jason Floyd] is a Senior Engineer at Hughes Associates, Inc., in Baltimore, Maryland. He received a bachelors of science and Ph.D. in the Nuclear Engineering Program of the University of Maryland. After graduating, he won a National Research Council Post-Doctoral Fellowship at the Building and Fire Research Laboratory of NIST, where he developed the combustion algorithm within FDS. He is a principal developer of the combustion model and control logic within FDS.

\item[Glenn Forney] is a computer scientist in the Engineering Laboratory of NIST. He received a bachelors of science degree in mathematics from Salisbury State College in 1978 and a master of science and a doctorate in mathematics at Clemson University in 1980 and 1984.  He joined the NIST staff in 1986 (then the National Bureau of Standards) and has since worked on developing tools that provide a better understanding of fire phenomena, most notably Smokeview, a software tool for visualizing Fire Dynamics Simulation data.

\item[Simo Hostikka] is a Senior Research Scientist at VTT Technical Research Centre of Finland. He received a master of science (technology) degree in 1997 and a doctorate in 2008 from the Department of Engineering Physics and Mathematics of the Helsinki University of Technology.  He is the principal developer of the radiation and solid phase sub-models within FDS.

\item[Susan Kilian] is a mathematician with numerics and scientific computing expertise. She received her diploma from the University of Heidelberg and received her doctorate from the Technical University of Dortmund in 2002. Since 2007 she has been a research scientist for hhpberlin, a fire safety engineering firm located in Berlin, Germany. Her research interests include high performance computing and the development of efficient parallel solvers for the pressure Poisson equation. 

\item[Charles Luo] is a Senior Research Scientist at Global Engineering and Materials, Inc., in Princeton, New Jersey. He received a B.S.~in theoretical and applied mechanics from the University of Science and Technology of China in 2002, and a doctorate in mechanical engineering from the State University of New York at Buffalo in 2010. His research interests include fire-structure interaction, immersed boundary methods, and fire response of composite and aluminum structures.

\item[Anna Matala] is a Research Scientist at VTT Technical Research Centre of Finland and a PhD candidate at Aalto University School of Science. She received her M.Sc.~degree in Systems and Operations Research from Helsinki University of Technology in 2008. Her research concentrates on pyrolysis modelling and parameter estimation in fire simulations.

\item[Randall McDermott] joined the research staff of the Building and Fire Research Lab in 2008. He received a B.S.~from the University of Tulsa in Chemical Engineering in 1994 and a doctorate at the University of Utah in 2005. His research interests include subgrid-scale models and numerical methods for large-eddy simulation, adaptive mesh refinement, immersed boundary methods, and Lagrangian particle methods.

\item[Kevin McGrattan] is a mathematician in the Engineering Laboratory of NIST. He received a bachelors of science degree from the School of Engineering and Applied Science of Columbia University in 1987 and a doctorate at the Courant Institute of New York University in 1991. He joined the NIST staff in 1992 and has since worked on the development of fire models, most notably the Fire Dynamics Simulator.

\item[William (Ruddy) Mell] is an applied mathematician currently at the U.S. Forest Service in Seattle, Washington. He holds a B.S. degree from the University of Minnesota (1981) and doctorate from the University of Washington (1994). His research interests include the development of large eddy simulation methods and sub-models applicable to the physics of large fires in buildings, vegetation, and the wildland-urban interface.

\item[Christian Rogsch] received a Diploma degree (like M.Sc.) in Safety Engineering from the University of Wuppertal, Germany. He works on shared-memory parallelization (OpenMP) of the Fire Dynamics Simulator.

\item[Topi Sikanen] is a Research Scientist at VTT Technical Research Centre of Finland and a graduate student at Aalto University School of Science. He received his M.Sc.~degree in Systems and Operations Research from Helsinki University of Technology in 2008. He works on the Lagrangian particle and liquid evaporation models. 

\item[Ben Trettel] is a graduate student at the University of Maryland. He received a B.S.~degree from the University of Maryland in Mechanical Engineering in 2011. He develops models for the transport of Lagrangian particles for the Fire Dynamics Simulator.

\item[Craig Weinschenk] joined the Fire Research Division as a National Research Council Postdoctoral Research Associate in 2011. He received a B.S.~from Rowan University in Mechanical Engineering in 2006, an M.S.~from the University of Texas-Austin in Mechanical Engineering in 2007, and a doctorate from the University of Texas-Austin in Mechanical Engineering in 2011. His research interests include numerical combustion, quadrature method of moments, and human factors research of fire-fighting tactics.

\end{description}





\chapter{Preface}

This is Volume 3 of the FDS Technical Reference Guide. Volume 1 describes the mathematical model and numerical method. Volume 2 documents past and present model verification work. Instructions for using FDS are contained in a separate User's Guide~\cite{FDS_Users_Guide}.

The FDS Technical Reference Guide is based in part on the ``Standard Guide for Evaluating the Predictive Capability of Deterministic Fire Models,'' ASTM~E~1355~\cite{ASTM:E1355}. ASTM~E~1355 defines {\em model evaluation} as ``the process of quantifying the accuracy of chosen results from a model when applied for a specific use.'' The model evaluation process consists of two main components: verification and validation. {\em Verification} is a process to check the correctness of the solution of the governing equations. Verification does not imply that the governing equations are appropriate; only that the equations are being solved correctly. {\em Validation} is a process to determine the appropriateness of the governing equations as a mathematical model of the physical phenomena of interest. Typically, validation involves comparing model results with experimental measurement. Differences that cannot be explained in terms of numerical errors in the model or uncertainty in the measurements are attributed to the assumptions and simplifications of the physical model.

Evaluation is critical to establishing both the acceptable uses and limitations of a model. Throughout its development, FDS has undergone various forms of evaluation, both at NIST and beyond. This volume provides a survey of validation work conducted to date to evaluate FDS.

\chapter{Disclaimer}

The US Department of Commerce makes no warranty, expressed or implied,
to users of the Fire Dynamics Simulator (FDS), and accepts no responsibility for its use.
Users of FDS assume sole responsibility under Federal law for determining
the appropriateness of its use in any particular application;
for any conclusions drawn from the results of its use; and for any
actions taken or not taken as a result of analysis performed using these tools.

Users are warned that FDS is intended for use only by those competent
in the fields of fluid dynamics, thermodynamics, heat transfer, combustion, and fire science,
and is intended only to supplement the informed judgment of the qualified user.
The software package is a computer model that may or may not have predictive
capability when applied to a specific set of factual circumstances.
Lack of accurate predictions by the model could lead to erroneous
conclusions with regard to fire safety. All results should be evaluated by an informed user.

Throughout this document, the mention of computer hardware or commercial
software does not constitute endorsement by NIST, nor does it indicate that
the products are necessarily those best suited for the intended purpose.


\chapter{Acknowledgments}
\label{acksection}

The following individuals and organizations played a role in the validation process of FDS.
\begin{itemize}
\item The US Nuclear Regulatory Commission Office of Research has funded key validation experiments, the preparation of the FDS manuals, and the development of various sub-models that are of importance in the area of nuclear power plant safety. Special thanks to Mark Salley, David Stroup, and Jason Dreisbach for their efforts and support.
\item Anthony Hamins of NIST directed the NIST/NRC and WTC experiments, conducted smaller methane burner measurements, and quantified the experimental uncertainty of these and other experiments used in this study. Alex Maranghides was the Director of the Large Fire Laboratory at NIST at the time these tests were conducted, and he helped to design the experiments. Therese McAllister oversaw the instrumentation of the structural steel during the WTC experiments.
\item Anthony Hamins of NIST developed the technique of evaluating experimental uncertainty that is used throughout this Guide. Blaza Toman of the Statistical Engineering Division of NIST developed the method of quantifying the model uncertainty.
\item Rick Peacock of NIST assisted in the interpretation of results from the ``NBS Multi-Room Test Series,'' a set of three room fire experiments conducted at the National Bureau of Standards (now NIST) in the mid-1980's.
\item Bryan Klein, currently employed at Thunderhead Engineering, Inc., assisted in the development of techniques to automatically generate the plots that are found throughout this Guide.
\item Bill Pitts, Nelson Bryner, and Erik Johnsson of NIST contributed and interpreted test data for the ``NIST Reduced Scale Enclosure Experiments.'' Matthew Bundy, Erik Johnsson, Paul Fuss, David Lenhart, Sung Chan Kim, and Andrew Lock of NIST contributed similar data collected within a full-scale standard compartment in 2010.
\item Rodney Bryant of NIST contributed velocity profile data for the ``Bryant Doorway'' series.
\item Anthony Putorti and Scott Bareham of NIST contributed temperature measurements from plate thermometer experiments in a cone calorimeter.
\item David Sheppard, currently of the Bureau of Alcohol, Tobacco and Firearms (ATF), conducted the experiments referred to as the ``UL/NFPRF Test Series'' on behalf of the Fire Protection Research Foundation (then known as the National Fire Protection Research Foundation) while working at Underwriters Labs in Northbrook, Illinois. Sheppard, along with Bryan Klein, currently employed at Thunderhead Engineering, Inc., conducted the experiments referred to as the ``ATF Corridors'' series in 2008.
\item Jerry Back, Craig Beyler and Phil DiNenno of Hughes Associates and Pat Tatem of the Naval Research Laboratory contributed experimental data for the ``HAI/NRL Wall Fire'' series. Thanks also to Craig Beyler for assistance with the data for the ``Beyler Hood Experiments.''
\item Ken Steckler provided details about the ``Steckler Compartment Experiments'' of 1979.
\item Jianping Zhang at the University of Ulster contributed heat flux measurements from the SBI apparatus.
\item At the University of Maryland, Professor Fred~Mowrer and Phil Friday were the first to apply FDS to the NRC-sponsored experiments referred to in this document as the ``FM/SNL Test Series'' (Factory Mutual and Sandia National Laboratories conducted these experiments).
\item Jukka Vaari of VTT, Finland, contributed the Cup Burner test cases.
\item Steve Nowlen of Sandia National Laboratory provided valuable information about the FM/SNL series, and he also conducted the CAROLFIRE experiments.
\item Ulf Wickstr\"{o}m of SP, Sweden, contributed experimental data from a series of experiments (SP AST) that were designed to evaluate the feasibility of using plate thermometer measurements as boundary conditions for a heat conduction calculation within several types of steel beams. The adiabatic surface temperature concept was tested in both the experiments and model.
\item Jeremy Thornock at the University of Utah provided data on the Sandia helium plume.
\item Sheldon Teiszen at Sandia National Laboratories, Albuquerque, provided detailed statistics for the helium plume and pool fire experiments conducted in the Sandia FLAME facility.
\item Taylor Myers, a student at the University of Maryland and a Summer Undergraduate Research Fellow (SURF) at NIST, analyzed the Vettori Flat and Sloped Ceiling sprinkler experiments. Thanks also to Bob Vettori of the U.S. Nuclear Regulatory Commission and formerly of NIST for his help in locating the original test data and laboratory notebooks.
\item Hans la Cour-Harbo, a masters degree student at the Technical University of Denmark, provided guidance and insight on the NRCC Facade experiments. Scott Bareham of NIST provided the technical drawing of the test enclosure.
\item Prof.~Stanislav Stoliarov and graduate students Mark McKinnon and Jing Li of the University of Maryland provided the properties of several polymers for the FAA Polymers example.
\item Michael Spearpoint and masters degree students Roger Harrison and Rob Fleury of the University of Canterbury, New Zealand, supplied measurements of mass entrainment rates into spill plumes (Harrison Spill Plumes) and heat flux measurements from propane burner fires (Fleury Heat Flux).
\item Ezti Oztekin of the Federal Aviation Administration (FAA) developed the FAA Cargo Compartments cases based on experiments sponsored by the FAA.
\item Topi Sikanen of VTT, Finland, and Jonathan Wahlqvist of Lund University, Sweden, contributed FDS input files for the PRISME DOOR series.
\item Paul Tyson, a student at the University of Ulster, Northern Ireland, contributed the input files and supporting documents for the NRCC Smoke Tower experiments.
%\item The Wind Engineering chapter was developed with help from Scott Hemley (2012 NIST SURF student), Dilip Banerjee, Donghun Yeo, Marc Levitan, and Emil Simiu, all from the NIST Engineering Laboratory, Materials and Structural Systems Division.
\item James White, a student at the University of Maryland, provided documentation and input files for the UMD Line Burner cases.
\item Charlie Hopkin and Michael Spearpoint of Olsson Fire \& Risk provided the data and FDS input files for experiments conducted by Adam Bittern at the University of Christchurch, New Zealand.
\item The simulations of liquefied natural gas (LNG) dispersion experiments that are described in this report were originally designed by Jeffrey Engerer and Anay Luketa of Sandia National Laboratories on behalf of the Pipeline and Hazardous Materials Safety Administration of the U.S. Department of Transportation.
\item Rahul Kallada Janardhan, a doctoral student at Aalto University, Finland, prepared the BST/FRS wood crib fire spread simulations.
\item Deepak Paudel, a doctoral student at Aalto University, Finland, prepared the models and datasets for the Insulation Material Fire Resistance tests.
\item Aleksi Rinta-Paavola, a doctoral student at Aalto University, Finland, carried out tests and prepared the simulations of Aalto Wood Experiments.
\end{itemize}


\cleardoublepage
\phantomsection
\addcontentsline{toc}{chapter}{Contents}
\tableofcontents

\cleardoublepage
\phantomsection
\addcontentsline{toc}{chapter}{List of Figures}
\listoffigures

\cleardoublepage
\phantomsection
\addcontentsline{toc}{chapter}{List of Tables}
\listoftables

\chapter{List of Acronyms}

\begin{tabbing}
\hspace{1.5in} \= \\
ALOFT \> A Large Outdoor Fire plume Trajectory model \\
AST \> Adiabatic Surface Temperature \\
ASTM \> American Society for Testing and Materials \\
ATF \> Bureau of Alcohol, Tobacco, Firearms, and Explosives \\
BRE \> British Research Establishment \\
CAROLFIRE \> Cable Response to Live Fire Test Program \\
CFAST \> Consolidated Model of Fire Growth and Smoke Transport \\
CFT \> Critical Flame Temperature \\
DNS \> Direct Numerical Simulation \\
FAA \> Federal Aviation Administration \\
FDS \> Fire Dynamics Simulator \\
FLAME \> Fire Laboratory for Accreditation of Models by Experimentation \\
FM \> Factory Mutual Global \\
FSE \>  Full-Scale Enclosure \\
HAI \> Hughes Associates, Inc. \\
HDPE \> high density polyethylene \\
HGL \> Hot Gas Layer \\
HIPS \> high-impact polystyrene \\
HRR \> Heat Release Rate \\
ISO \> International Standards Organization \\
LEMTA \> Laboratoire d'Energ\a'{e}tique et de M\a'{e}chanique Th\a'{e}orique et Appliqu\a'{e}e \\
LES \> Large Eddy Simulation \\
LLNL \> Lawrence Livermore National Laboratory \\
LNG \> Liquified Natural Gas \\
MEC \> Minimum Extinguishing Concentration \\
NBS \> National Bureau of Standards (former name of NIST) \\
NFPRF \> National Fire Protection Research Foundation \\
NIST \> National Institute of Standards and Technology \\
NRC \> Nuclear Regulatory Commission \\
NRCC \> National Research Council of Canada \\
NRL \> Naval Research Laboratory \\
PDPA \> Phase Doppler Particle Analyzer \\
PIV \> Particle Image Velocimetry \\
PMMA \> poly(methyl methacrylate) \\
PRISME \> Propagation d'un incendie pour des sc\a'{e}narios multi-locaux \a'{e}l\a'{e}mentaires \\
PVC \> Polyvinyl chloride \\
RANS \> Reynolds Averaged Navier-Stokes \\
RSE \> Reduced-Scale Enclosure \\
SBI \>  Single Burning Item \\
SNL \> Sandia National Laboratory \\
SP \>  Statens Provningsanstalt (Technical Research Institute of Sweden) \\
TGA \> Thermal Gravimetric Analysis \\
THIEF \> Thermally-Induced Electrical Failure \\
UL  \> Underwriters Laboratories \\
USN \> United States Navy \\
VTFRL \> Virginia Tech Fire Research Laboratory \\
VTT \> Valtion Teknillinen Tutkimuskeskus (Technical Research Centre of Finland) \\
WTC \> World Trade Center \\
\end{tabbing}



\mainmatter

\chapter{Model Overview}

This chapter presents general information about the Fire Dynamics Simulator, following the basic framework set forth in
ASTM E 1355~\cite{ASTM:E1355}.

\section{Basic Description of FDS}


\subsection{Type of Model}

FDS is a Computational Fluid Dynamics (CFD) model of fire-driven fluid flow. The model numerically solves a form of the Navier-Stokes equations appropriate for low-speed, thermally-driven flow with an emphasis on smoke and heat transport from fires. The partial derivatives in the conservation equations for mass, momentum and energy are approximated by finite differences, and the solution is updated in time on a three-dimensional, rectilinear grid. Thermal radiation is computed using a finite volume technique on the same grid as the flow solver. Lagrangian particles are used to simulate smoke movement, sprinkler discharge, and fuel sprays.

Smokeview is a companion program to FDS that produces images and animations of the results. In recent years, its developer, Glenn Forney, has
added to Smokeview the ability to visualize fire and smoke in a fairly realistic way. In a sense, Smokeview now is, via its three-dimensional
renderings, an integral part of the physical model, as it allows one to assess the visibility within a fire compartment in ways that ordinary
scientific visualization software cannot.

Although not part of the FDS/Smokeview suite maintained at NIST, there are several third-party and proprietary ``add-ons'' to FDS either available
commercially or privately maintained by individual users. Most notably, there are several Graphical User Interfaces (GUIs) that can be used
to create the input file containing all the necessary information needed to perform a simulation.



\subsection{Version History}

Version 1 of FDS was publicly released in February 2000, version 2 in December 2001, version 3 in November 2002, and version 4 in July 2004. The
present version of FDS is 5, first released in October, 2007.

Starting with FDS 5, a formal revision management system has been implemented to track changes to the FDS source code. The open-source program
development tools are provided by an Internet-based organization known as Google Code (code.google.com).

The version number for FDS has three parts.  For example, FDS 5.2.12 indicates that this is FDS 5, the fifth major release. The 2 indicates a
significant upgrade, but still within the framework of FDS 5.  The 12 indicates the twelveth minor upgrade of 5.2, mostly bug fixes and minor user
requests.


\subsection{Model Developers}


Currently, FDS is maintained by the Building and Fire Research Laboratory (BFRL) of National Institute of Standards and Technology. The developers at
NIST have formed a loose collaboration of interested stakeholders, including:
\begin{itemize}
\item VTT Technical Research Centre of Finland, a research and testing
laboratory similar to NIST
\item The Society of Fire Protection Engineers (SFPE) who conduct training classes on the use of FDS
\item Fire protection engineering firms that use the software
\item Engineering departments at various universities with a particular emphasis on fire
\end{itemize}
BFRL awards grants on a competitive basis to external organizations who conduct research in fire science and engineering. Some of these grants have
been used to assist the development of FDS. The role of the grantee in supporting day to day development varies. Not all of the developers outside of
NIST are grantees.

Starting with Version 5, the FDS development team uses an Internet-based development environment called GoogleCode, a free service of the search
engine company, Google. GoogleCode is a widely used service designed to assist open source software development by providing a repository for source
code, revision control, program distribution, bug tracking, and various other very useful services.

Each member of the FDS development team has an account and password access to the FDS repository. In addition, anonymous access is available to all
interested users, who can receive the latest versions of the source code, manuals, and other items. Anonymous users simply do not have the power to
commit changes to any of these items. The power to commit changes to FDS or its manuals can be granted to anyone on a case by case basis.

The FDS manuals are typeset using \LaTeX, specifically, PDF \LaTeX. The \LaTeX files are essentially text files that are under SVN (Subversion)
control. The figures are either in the form of PDF or jpeg files, depending on whether they are vector or raster format. There are a variety of
\LaTeX packages available, including MiKTeX. The FDS developers edit the manuals as part of the day to day upkeep of the model. Different editions of
the manuals are distinguished by date.


\subsection{Intended Uses}

Throughout its development, FDS has been aimed at solving practical
fire problems in fire protection engineering, while at the same time
providing a tool to study fundamental fire dynamics and combustion.
FDS can be used to model the following phenomena:
\begin{itemize}
\setlength{\itemsep}{0.0in}
\item Low speed transport of heat and combustion products from fire
\item Radiative and convective heat transfer between the gas and solid surfaces
\item Pyrolysis
\item Flame spread and fire growth
\item Sprinkler, heat detector, and smoke detector activation
\item Sprinkler sprays and suppression by water
\end{itemize}
Although FDS was designed specifically for fire simulations,
it can be used for other low-speed fluid flow simulations that do not necessarily
include fire or thermal effects. To date, about half of the
applications of the model have been for design of smoke control
systems and sprinkler/detector activation studies.
The other half consist of residential and industrial fire reconstructions.


\subsection{Input Parameters}

All of the input parameters required by FDS to describe a particular
scenario are conveyed via a single text file created by the user.
The file contains information about the numerical grid, ambient environment, building geometry, material
properties, combustion kinetics, and desired output quantities.
The numerical grid consists of one or more rectilinear meshes with (usually) uniform cells. All geometric features of the
scenario must conform to this numerical grid. Objects smaller than a single grid cell are either approximated
as a single cell, or rejected. The building geometry is input as a series of rectangular blocks. Boundary conditions are
applied to solid surfaces as rectangular patches. Materials are defined by their thermal conductivity, specific heat,
density, thickness, and burning behavior. There are various ways that this information is conveyed, depending on the
desired level of detail.

Any simulation of a real fire scenario involves specifying material properties for the walls, floor, ceiling,
and furnishings. FDS treats all of these objects as multi-layered solids, thus the physical parameters for many real
objects can only be viewed as approximations to the actual properties. Describing these materials in the input file is
the single most challenging task for the user. Thermal properties such as conductivity, specific heat,
density, and thickness can be found in various handbooks, or in manufacturers literature, or from bench-scale measurements.
The burning behavior of materials at different heat fluxes is more difficult to describe, and the properties more difficult
to obtain. Even though entire books are devoted to the
subject~\cite{Babrauskas:2}, it is still difficult to find information on a particular item.

A significant part of the FDS input file directs the code to output various quantities in various ways. Much like in an
actual experiment, the user must decide before the calculation begins what information to save. There is no way to
recover information after the calculation is over if it was not requested at the start.

A complete description of the input parameters required by FDS can be found in the FDS User's Guide~\cite{FDS_Users_Guide}.


\subsection{Output Quantities}

FDS computes the temperature, density, pressure, velocity and chemical composition within each numerical
grid cell at each discrete time step. There are typically hundreds of thousands to millions of grid
cells and thousands to hundreds of thousands of time steps. In addition, FDS computes at solid surfaces
the temperature, heat flux, mass loss rate, and various other quantities. The user must carefully select what
data to save, much like one would do in designing an actual experiment. Even though only a small fraction of
the computed information can be saved, the output typically consists of fairly large data files. Typical
output quantities for the gas phase include:
\begin{itemize}
\setlength{\itemsep}{0.0in}
\item Gas temperature
\item Gas velocity
\item Gas species concentration (water vapor, CO$_2$, CO, N$_2$)
\item Smoke concentration and visibility estimates
\item Pressure
\item Heat release rate per unit volume
\item Mixture fraction (or air/fuel ratio)
\item Gas density
\item Water droplet mass per unit volume
\end{itemize}
On solid surfaces, FDS predicts additional quantities associated with the energy balance between
gas and solid phase, including
\begin{itemize}
\setlength{\itemsep}{0.0in}
\item Surface and interior temperature
\item Heat flux, both radiative and convective
\item Burning rate
\item Water droplet mass per unit area
\end{itemize}
Global quantities recorded by the program include:
\begin{itemize}
\setlength{\itemsep}{0.0in}
\item Total Heat Release Rate (HRR)
\item Sprinkler and detector activation times
\item Mass and energy fluxes through openings or solids
\end{itemize}
Time histories of various quantities at a single point in space or global
quantities like the fire's heat release rate (HRR) are saved in simple, comma-delimited text files that
can be plotted using a spreadsheet program.
However, most field or surface data are visualized with a program called Smokeview, a tool specifically
designed to analyze data generated by FDS. FDS and Smokeview are used in concert to model and visualize fire phenomena.
Smokeview performs this visualization by presenting animated tracer particle flow,
animated contour slices of computed gas variables and animated surface data.
Smokeview also presents contours and vector plots of static data anywhere
within a scene at a fixed time.

A complete list of FDS output quantities and formats is given in Ref.~\cite{FDS_Users_Guide}.
Details on the use of Smokeview are found in Ref.~\cite{Smokeview_Users_Guide}.




\subsection{Governing Equations, Assumptions and Numerics}

Following is a brief description of
the major components of FDS. Detailed information regarding the assumptions and governing equations associated
with the model is provided in Section~\ref{govequations}.
\begin{description}
\item[Hydrodynamic Model] FDS
solves numerically a form of the Navier-Stokes equations appropriate
for low-speed, thermally-driven flow with an emphasis on
smoke and heat transport from fires. The core algorithm is an
explicit predictor-corrector scheme that is second order accurate in space
and time. Turbulence is treated by means of the Smagorinsky form of
Large Eddy Simulation (LES). It is possible to perform a Direct
Numerical Simulation (DNS) if the underlying numerical grid is fine
enough. LES is the default mode of operation.
\item[Combustion Model]
For most applications, FDS uses a combustion model based on the mixing limited, infinitely fast reaction of lumped species.
Lumped species are conserved scalar quantities that represent a mixture of species such as air which is a mixture of nitrogren, oxygen, water vapor, and carbon dioxide.  In FDS versions prior to 6, these lumped species were referred to as mixture fraction parameters.  As with FDS 5, the reaction of fuel and oxygen is not necessarily instantaneous and complete, and there are
several optional schemes that are designed to predict the extent of combustion in under-ventilated spaces.
The mass fractions of all of the major reactants and products can
be derived from the lumped species by means of ``state relations,''
expressions arrived at by a
combination of simplified analysis and measurement.
\item[Radiation Transport] Radiative heat transfer is included in the
model via the solution of the radiation transport equation for a gray
gas. In a limited number of cases, a wide band model can be used in
place of the gray gas model to provide a better spectral accuracy. The
radiation equation is solved using a technique similar to a finite
volume method for convective transport, thus the name given to it is
the Finite Volume Method (FVM). Using approximately 100 discrete
angles, the finite volume solver requires about 20~\% of the total CPU
time of a calculation, a modest cost given the complexity of radiation
heat transfer.  Water droplets can absorb and scatter thermal
radiation. This is important in cases involving mist sprinklers, but
also plays a role in all sprinkler cases. The absorption and
scattering coefficients are based on Mie theory. The scattering from
the gaseous species and soot is not included in the model.
\item[Geometry]
FDS approximates the governing equations on one or more rectilinear grids. The
user prescribes rectangular obstructions that are forced to conform
with the underlying grid.
\item[Boundary Conditions]
All solid surfaces are assigned thermal boundary conditions, plus
information about the burning behavior of the material.
Heat and mass transfer to and from solid surfaces is
usually handled with empirical correlations, although it is possible
to compute directly the heat and mass transfer when performing a
Direct Numerical Simulation (DNS).
\item[Sprinklers and Detectors] The activation of sprinklers and heat and smoke detectors
is modeled using fairly simple correlations of thermal inertia for
sprinklers and heat detectors, and transport lag for smoke detectors.
Sprinkler sprays are modeled by Lagrangian particles that represent a sampling of the
water droplets ejected from the sprinkler.
\end{description}


\subsection{Limitations}

Although FDS can address most fire scenarios, there are limitations in all of its various
algorithms. Some of the more prominent limitations of the model are listed here. More
specific limitations are discussed as part of the description of the governing equations
in Section~\ref{govequations}.
\begin{description}
\item[Low Speed Flow Assumption] The use of FDS is limited to low-speed\footnote{Mach numbers less than about 0.3} flow
with an emphasis on smoke and heat transport from fires. This assumption rules out using the model for any scenario
involving flow speeds approaching the speed of sound, such as explosions, choke flow at nozzles, and detonations.
\item[Rectilinear Geometry] The efficiency of FDS is due to the simplicity of its rectilinear numerical grid and the
use of a fast, direct solver for the pressure field.
This can be a limitation in some situations where certain geometric features
do not conform to the rectangular grid, although most building components do. There are techniques in FDS to
lessen the effect of ``sawtooth'' obstructions used to represent non-rectangular objects, but these cannot be expected
to produce good results if, for example, the intent of the calculation is to study boundary layer effects. For most
practical large-scale simulations, the increased grid resolution afforded by the fast pressure solver offsets the
approximation of a curved boundary by small rectangular grid cells.
\item[Fire Growth and Spread]
Because the model was originally designed to analyze industrial-scale fires,
it can be used reliably when the heat release rate (HRR) of the fire is specified and the
transport of heat and exhaust products is the principal aim of the simulation.
In these cases, the model predicts flow velocities and temperatures to an accuracy within
10~\% to 20~\% of experimental measurements, depending on the resolution of the numerical grid
\footnote{It is extremely rare to
find measurements of local velocities and/or temperatures from fire experiments that
have reported error estimates that are less than 10~\%. Thus, the most accurate
calculations using FDS do not introduce significantly greater errors in these quantities
than the vast majority of fire experiments.}.
However, for fire scenarios where the heat release rate is {\em predicted} rather than {\em specified},
the uncertainty of the model is higher.
There are several reasons for this: (1) properties of real materials
and real fuels are often unknown or difficult to obtain, (2) the physical processes of combustion,
radiation and solid phase heat transfer are more complicated than their mathematical representations
in FDS, (3) the results of calculations are sensitive to both the numerical and physical parameters.
Current research is aimed at improving this situation, but it is safe to say that
modeling fire growth and spread will always require a higher level of
user skill and judgment than that required for modeling the transport of smoke and heat from specified fires.
\item[Combustion]
For most applications, FDS uses a mixing-controlled, lumped species based combustion model.
Lumped species are conserved scalar quantities that represent mixtures of gas species.  For most applications, these mixtures are air, fuel plus an optional diluent, and combustion products.  In its simplest form, the model assumes that combustion is mixing-controlled, and that the
reaction of fuel and oxygen is infinitely fast, regardless of the temperature.
For large-scale, well-ventilated
fires, this is a good assumption. However, if a fire is in an
under-ventilated compartment, or if a suppression agent like water
mist or CO$_2$ is introduced, fuel and oxygen are allowed to mix and not burn, according to a few empirically-based criteria.
The physical mechanisms underlying these phenomena are complex, and are tied closely to the flame temperature and local strain rate, neither of
which are readily-available in a large scale fire simulation.
Subgrid-scale modeling of gas phase suppression and
extinction is still an area of active research in the combustion
community. Until reliable models can be developed for building-scale
fire simulations, simple empirical rules can be used that
prevent burning from taking place when the atmosphere immediately
surrounding the fire cannot sustain the combustion. Details are found in
Section~\ref{combustionsection}.
\item[Radiation] Radiative heat transfer is included in the model via
the solution of the radiation transport equation (RTE) for a gray gas, and
in some limited cases using a wide band model.  The RTE is solved
using a technique similar to finite volume methods for convective
transport, thus the name given to it is the Finite Volume Method
(FVM). There are several limitations of the model. First, the
absorption coefficient for the smoke-laden gas is a complex function
of its composition and temperature. Because of the simplified
combustion model, the chemical composition of the smokey gases,
especially the soot content, can effect both the absorption and
emission of thermal radiation.  Second, the radiation transport is
discretized via approximately 100 solid angles, although the user may
choose to use more angles. For targets far away from a localized
source of radiation, like a growing fire, the discretization can lead
to a non-uniform distribution of the radiant energy. This error is
called ``ray effect'' and can be seen in the visualization of surface
temperatures, where ``hot spots'' show the effect of the finite number
of solid angles. The problem can be lessened by the inclusion of more
solid angles, but at a price of longer computing times. In most cases,
the radiative flux to far-field targets is not as important as those
in the near-field, where coverage by the default number of angles is
much better.
\end{description}



\clearpage
\section{Peer Review Process}

FDS is reviewed both internally and externally. All documents issued by the
National Institute of Standards and Technology are formally reviewed internally by members of
the staff. The theoretical basis of FDS is laid out in the present document, and is
subject to internal review by staff members who are not active participants in the development
of the model, but who are members of the Fire Research Division and are considered experts in
the fields of fire and combustion. Externally, papers detailing various parts of FDS are
regularly published in peer-reviewed journals and conference proceedings. In addition, FDS
is used world-wide by fire protection engineering firms who review the technical details of
the model related to their particular application. Some of these firms also publish in the
open literature reports documenting internal efforts to validate the model for a particular
use. Many of these studies are referenced in Volume 3 of the FDS Technical Reference Guide~\cite{FDS_Tech_Guide}.


\subsection{Survey of the Relevant Fire and Combustion Literature}

\label{Relevantdocs}

FDS has two separate manuals --
the FDS Technical Reference Guide~\cite{FDS_Tech_Guide}
and the FDS User's Guide~\cite{FDS_Users_Guide}. The Technical Reference Guide is broken into three volumes: (1)~Mathematical Model, (2)~Verification, and (3)~Experimental Validation.
Smokeview has its own User's Guide~\cite{Smokeview_Users_Guide}. The FDS and Smokeview User Guides only describe the mechanics of using the
computer programs. The
Technical Reference Guides provides the theory, algorithm details, and verification and validation work.

There are numerous sources that describe various parts of the
model. The basic set of equations solved in FDS was formulated by Rehm
and Baum in the {\em Journal of Research of the National Bureau of
Standards}~\cite{Rehm:1}.  The basic hydrodynamic algorithm evolved at
NIST through the 1980s and 1990s, incorporating fairly well-known
numerical schemes that are documented in books by Anderson, Tannehill
and Pletcher~\cite{Anderson:1}, Peyret and Taylor~\cite{Peyret:1}, and
Ferziger and Peri\'{c}~\cite{Ferziger:1}. This last book provides a
good description of the large eddy simulation technique and provides
references to many current publications on the subject.  Numerical
techniques appropriate for combustion systems are described by Oran
and Boris~\cite{Oran:1}.  The mixture fraction combustion model is
described in a review article by Bilger~\cite{Bilger:AnnRev}. Basic
heat transfer theory is provided by Holman~\cite{Holman:1} and
Incropera~\cite{Incropera:1}. Thermal radiation is described in Siegel
and Howell~\cite{Siegel:1}.

Much of the current knowledge of fire science and engineering
is found in the {\em SFPE Handbook of Fire Protection Engineering}~\cite{SFPE}. Popular textbooks in fire protection
engineering include those by Drysdale~\cite{Drysdale:1} and Quintiere~\cite{Quintiere:2}. On-going research in
fire and combustion is documented in several periodicals and conference proceedings.
The International Association of Fire Safety Science (IAFSS)
organizes a conference every two years, the proceedings of which are frequently referenced by fire researchers.
Interscience Communications, a London-based publisher of several fire-related journals, hosts a conference known as Interflam roughly
every three years in the United Kingdom.
The Combustion Institute hosts an international symposium on combustion every two years, and in addition to the
proceedings of this symposium, the organization publishes its own journal, {\em Combustion and Flame}.
The papers appearing in the IAFSS conference proceedings,
the Combustion Symposium proceedings, and {\em Combustion and Flame} are all peer-reviewed, while those appearing in the
Interflam proceedings are selected based on the submission of a short abstract.
Both the Society for Fire Protection Engineers (SFPE) and the National Fire Protection Association (NFPA) publish
peer-reviewed technical journals entitled the {\em Journal of Fire Protection Engineering} and {\em Fire Technology}.
Other often-cited, peer-reviewed technical journals include the {\em Fire Safety Journal}, {\em Fire and Materials}, {\em Combustion
Science and Technology}, {\em Combustion Theory and Modeling} and the {\em Journal of Heat Transfer}.

Research at NIST is documented in various ways beyond contributions made by staff to external journals and conferences.
NIST publishes several forms of internal reports, special publications, and its own journal called the {\em Journal
of Research of NIST}. An internal report, referred to as a NISTIR (NIST Inter-agency Report), is a convenient means to disseminate information,
especially when the quantity of data exceeds what could normally be accepted by a journal. Often parts of a NISTIR are
published externally, with the NISTIR itself serving as the complete record of the work performed. Previous versions of the
FDS Technical Reference Guide and User's Guide were published as NISTIRs. The current FDS and Smokeview manuals are
being published as NIST Special Publications, distinguished from NISTIRs by the fact that they are permanently archived.
Work performed by an outside person or organization working
under a NIST grant or contract is published in the form of a NIST Grant/Contract Report (GCR).
All work performed by the staff of the Building and Fire Research Laboratory at NIST beyond 1993 is permanently stored in
electronic form and made freely available via the Internet and yearly-released compact disks (CDs) or other electronic media.




\subsection{Review of the Theoretical Basis of the Model}

\label{JustAA}
The technical approach and assumptions of the model have been presented in
the peer-reviewed scientific literature and at technical conferences cited in the previous section.
The major assumptions of the model, for example the large eddy simulation technique and the combustion
model, have undergone a roughly 40 year development and are now documented in popular introductory text books.
More specific sub-models, like the sprinkler spray routine or the various pyrolysis models, have yet to be developed to
this extent. As a consequence, all documents produced by NIST staff are required to go
through an internal editorial review and approval process.
This process is designed to ensure compliance with the technical requirements,
policy, and editorial quality required by NIST.
The technical review includes a critical evaluation of the technical content and
methodology, statistical treatment of data, uncertainty analysis, use of appropriate
reference data and units, and bibliographic references.
The FDS and Smokeview manuals are first reviewed by a member of the Fire Research Division,
then by the immediate supervisor of the author of the document,
then by the chief of the Fire Research Division, and finally by a reader from
outside the division. Both the immediate supervisor and the division chief are
technical experts in the field. Once the document has been reviewed, it is
then brought before the Editorial Review Board (ERB),
a body of representatives from all the NIST laboratories.
At least one reader is designated by the Board for each document that it accepts for
review. This last reader is selected based on technical competence and impartiality.
The reader is usually from outside the division producing the document and is
responsible for checking that the document conforms with NIST policy on units, uncertainty
and scope. He/she does not need to be a technical expert in fire or combustion.

Recently, the US Nuclear Regulatory Commission (US NRC) published a seven-volume report on its own verification and validation
study of five different fire models used for nuclear power plant applications~\cite{NUREG_1824}. Two of the models are essentially a set
of empirically-based correlations in the form of engineering ``spread sheets.'' Two of the models are classic two-zone fire models, one of which
is the NIST developed CFAST. FDS is the sole CFD model in the study. More on the study and its results can be found in Volume~3 of the
FDS Technical Reference Guide~\cite{FDS_Tech_Guide}.

Besides formal internal and peer review, FDS is subjected to continuous scrutiny because
it is available free of charge to the general public and is used
internationally by those involved in fire safety design and post-fire reconstruction.
The quality of the FDS and Smokeview User Guides is checked implicitly by the fact that the
majority of model users have not taken a formal training course in the actual use of the model, but
are able to read the supporting documents, perform a few sample simulations, and then systematically build
up a level of expertise appropriate for their applications. The developers receive daily feedback from
users on the clarity of the documentation and add clarifications
when needed. Before new versions of the model are released, there is a several month ``beta test'' period
in which users test the new version using the updated documentation. This process is similar,
although less formal, to that which most computer software programs undergo.
Also, the source code for FDS is released publicly, and has been used at
various universities world-wide, both in the classroom as a teaching tool as well as for research.
As a result, flaws in the theoretical development and the computer program itself
have been identified and corrected. As FDS continues to evolve, the user base will continue to
serve as a means to evaluate the model. We consider this process as important to the development of FDS as the formal
internal and external peer-review processes.


\clearpage
\section{Development Process}

Changes are made to the FDS source code daily, and tracked via revision control software. However, these daily changes do not constitute a change to
the version number. After the developers determine that enough changes have been made to the source, they release a new minor upgrade, 5.2.12 to
5.2.13, for example. This happens every few weeks. A change from 5.2 to 5.3 might happen only a few times a year, when significant improvements have
been made to the model physics.

There is no formal process by which FDS is updated. Each developer works on various routines, and makes changes as warranted. Minor bugs are fixed
without any communication (the developers are in different locations), but more significant changes are discussed via email or telephone calls. A
suite of simple verification calculations (included in this document) are routinely run to ensure that the daily bug fixes have not altered any of
the important algorithms. A suite of validation calculations (also included here) are run with each significant upgrade. Significant changes to FDS
are made based on the following criteria, in no particular order:
\begin{description}
\item[Better Physics:] The goal of any model is to be {\em predictive}, but it also must be reliable. FDS is a blend of empirical and
deterministic sub-models, chosen based on their robustness, consistency, and reliability. Any new sub-model must demonstrate that it is of comparable
or superior accuracy to its empirical counterpart.
\item[Modest CPU Increase:] If a proposed algorithm doubles the calculation time but yields only a marginal improvement in accuracy, it is
likely to be rejected. Also, the various routines in FDS are expected to consume CPU time in proportion to their overall importance. For example, the
radiation transport algorithm consumes about 25~\% of the CPU time, consistent with the fact that about one-fourth to one-third of the fire's energy
is emitted as thermal radiation.
\item[Simpler Algorithm:] If a new algorithm does what the old one did using fewer lines of code, it is almost always accepted, so long as
it does not decrease functionality.
\item[Increased or Comparable Accuracy:] The validation experiments that are part of this guide serve as the metric for new routines. It is
not enough for a new algorithm to perform well in a few cases. It must show clear improvement across the suite of experiments. If the accuracy is
only comparable to the previous version, then some other criteria must be satisfied.
\item[Acceptance by the Fire Protection Community:] Especially in regard to fire-specific devices, like sprinklers and smoke detectors, the
algorithms in FDS often are based on their acceptance among the practicing engineers.
\end{description}




\chapter{Survey of Past Validation Work}


In this  chapter, a survey of  FDS validation work  will be presented. Some of the work has been  performed at NIST, some by its grantees and some by
engineering firms using the model.  Because each organization has its  own reasons for  validating the model, the  referenced papers and reports do
not follow any particular guidelines. Some of the works only provide  a qualitative assessment  of the model,  concluding that the  model  agreement
with  a  particular  experiment  is ``good''  or ``reasonable.'' Sometimes, the conclusion is that the model works well in certain cases, not as well
in others. These studies are included in the survey because the references  are useful to other model users who may have a similar application  and
are interested in even qualitative assessment. It is important to note  that some of the papers point out flaws in early releases of FDS that have
been corrected or improved in more recent  releases. Some of  the issues raised, however,  are still subjects of  active research. The  research
agenda for FDS  is greatly influenced  by   the  feedback   provided  by  users,   often  through publication of validation efforts.


\section{Validation Work with Pre-Release Versions of FDS}

FDS was officially released in  2000. However, for two decades various CFD codes using the basic FDS hydrodynamic framework were developed at NIST
for  different applications and  for research. In the  mid 1990s, many of  these different codes were consolidated  into what eventually became FDS.
Before FDS, the various  models were referred  to as LES, NIST-LES, LES3D,  IFS (Industrial Fire Simulator), and  ALOFT (A Large Outdoor Fire Plume
Trajectory).

The  NIST LES model  describes the  transport of  smoke and  hot gases during  a fire  in an  enclosure using  the  Boussinesq approximation, where
it is assumed that the density and temperature variations in the flow                           are                          relatively
small~\cite{Rehm:1,Rehm:SIAM83,Rehm:ANM85,Rehm:IAFSS3}.     Such    an approximation  can be  applied  to a  fire  plume away  from the  fire itself.
Much of  the early  work  with this  form of  the model  was devoted  to  the  formulation of  the  low  Mach  number form  of  the Navier-Stokes
equations and  the development  of the  basic numerical algorithm.   Early validation  efforts  compared the  model with  salt water
experiments~\cite{Baum:1,McGrattan:1,Rehm:IAFSS5},   and  fire plumes~\cite{Baum:IAFSS5,Baum:2,Baum:3,Baum:4}.  Clement validated the hydrodynamic
model  in FDS by  measuring salt water flows  using Laser Induced   dye   Fluorescence~(LIF)~\cite{Clement:1}.  An   interesting finding  of this
work was  that the  transition from  a laminar  to a turbulent plume is very difficult  to predict with any technique other than DNS.

Eventually, the  Boussinesq approximation was  dropped and simulations began  to   include  more  fire-specific   phenomena.  Simulations  of
enclosure   fires   were   compared   to  experiments   performed   by Steckler~\cite{McGrattan:4}.  Mell~{\em  et al.}~\cite{Mell:1} studied small
helium  plumes, with particular attention to  the relative roles of  baroclinic torque and  buoyancy as  sources of  vorticity.  Cleary {\em  et
al.}~\cite{LES:6}  used   the  LES  model  to  simulate  the environment  seen by  multi-sensor fire  detectors and  performed some simple validation
work to check the model before using it.  Large fire experiments were performed by NIST  at the FRI test facility in Japan, and at US Naval aircraft
hangars in Hawaii and Iceland~\cite{Davis:1}. Room   airflow   applications   were   considered  by   Emmerich   and
McGrattan~\cite{Emmerich:1,Emmerich:2}.

These early validation efforts were encouraging, but still pointed out the  need  to  improve  the  hydrodynamic  model  by  introducing  the
Smagorinsky form of large eddy simulation.  This addition improved the stability  of the  model  because of  the  relatively simple  relation between
the  local strain rate  and the turbulent viscosity.  There is both   a   physical  and   numerical   benefit   to  the   Smagorinsky model.
Physically,  the viscous term used  in the model  has the right functional  form to describe  sub-grid mixing  processes. Numerically, local
oscillations in the computed  flow quantities are damped if they become  large   enough  to  threaten  the  stability   of  the  entire calculation.

During the 1980s and 1990s,  the Building and Fire Research Laboratory at NIST studied the burning of  crude oil under the sponsorship of the US
Minerals Management Service.  The aim of the work was to assess the feasibility of using burning as a means to remove spilled oil from the sea
surface. As part of the  effort, Rehm and Baum developed a special application of the LES model called ALOFT. The model was a spin-off of the
two-dimensional LES enclosure  model, in which a three-dimensional steady-state plume was computed  as a two-dimensional evolution of the lateral
wind field  generated by a large fire blown  in a steady wind. The ALOFT model is based on  large eddy simulation in that it attempts to resolve the
relevant scales of a large, bent-over plume. Validation work  was  performed  by  simulating  the plumes  from  several  large experimental burns of
crude oil in which aerial and ground sampling of smoke       particulate       was       performed~\cite{McGrattan:4a}. Yamada~\cite{ALOFT:2}
performed  a validation  of the ALOFT  model for 10~m oil  tank fire. The results  indicate that the  prediction of the plume  cross  section  500~m
from   the  fire  agree  well  with  the experimental observations.




\section{Validation of FDS since 2000}

There is an  on-going effort at NIST and elsewhere  to validate FDS as new capabilities are  added. To date, most of  the validation work has
evaluated the  model's ability  to predict the  transport of  heat and exhaust products from  a fire through an enclosure.  In these studies, the
heat release rate is usually prescribed, along with the production rates of  various products  of combustion.  More  recently, validation efforts
have moved  beyond  just transport  issues  to consider  fire growth, flame spread,  suppression, sprinkler/detector activation, and other
fire-specific phenomena.

The  validation work  discussed below  can be  organized  into several categories: Comparisons with full-scale tests conducted especially for the
chosen  evaluation,   comparisons   with  previously   published full-scale  test data,  comparisons with  standard  tests, comparisons with
documented  fire experience,  and  comparisons with  engineering correlations.  There is no single  method by which the predictions and measurements
are compared.   Formal, rigorous validation exercises are time-consuming  and  expensive.  Most  validation exercises  are  done simply to assess if
the model can be used for a very specific purpose. While  not  comprehensive on  their  own,  these studies  collectively constitute a valuable
assessment of the model.


\subsection{Comparison with Full-Scale Tests Conducted Specifically for the Chosen Evaluation}

As part of the NIST investigation  of the World Trade Center fires and collapse,  a series  of large  scale fire  experiments  were performed
specifically  to  validate  FDS~\cite{Hamins:WTC1}.   The  tests  were performed in  a rectangular  compartment 7.2~m long  by 3.6~m  wide by 3.8~m
tall.   The fires were  fueled by heptane  for some tests  and a heptane/toluene mixture  for the others.  The results of the experiments and simulations
are included in detail in this Guide.

A second set of experiments to validate FDS for use in the World Trade Center  investigation is  documented  in Ref.~\cite{Hamins:WTC2}.  The experiments
are not described as part of this Guide. The intent
of  these tests  was to  evaluate the ability  of the  model to simulate the growth  of a fire burning 3  office workstations within a compartment of
dimensions 11~m by 7~m by 4~m, open at one end to mimic the ventilation  of 5  windows similar to  those in  WTC 1 and  2. Six tests  were performed
with various  initial conditions  exploring the effect of jet fuel spray and ceiling tiles covering the surface of the desks and carpet. Measurements
were  made of the heat release rate and compartment  gas   temperatures  at  four   locations  using  vertical thermocouple arrays. Six different
material samples were tested in the NIST  cone calorimeter:  desk,  chair, paper,  computer case,  privacy panel, and  carpet. Data for the  carpet,
desk and  privacy panel were input directly into FDS, with the other 3 materials lumped together to form an  idealized fuel type.  Open burns of
single  workstations were used to  calibrate the simplified fuel  package. Then FDS  was used to make  blind  predictions  of   the  3  workstation
fires  within  the compartment. Peak  heat release rates and  temperatures were predicted to within 20~\% for all tests.


\subsection{Comparison with Engineering Correlations}

There are  several examples of  fire flows that have  been extensively studied, so much  so that a set of  engineering correlations combining the
results  of   many  experiments   have  been   developed.  These correlations are  useful to modelers because of  their simplicity. The most studied
phenomena include fire  plumes, ceiling jets,  and flame heights.

Although much  of the  early validation work  before FDS  was released involved fire plumes, it remains an active area of interest. One study by
Chow  and Yin~\cite{Chow:1}  surveys  the  performance of  various models in predicting plume temperatures and entrainment.  They compare various
correlations, a  RANS (Reynolds-Averaged Navier-Stokes) model, and FDS.  Simulations  were carried out that replicated  a 470~kW fire with a
diameter of  1~m and an  unbounded ceiling.  A  numerical grid size of  96 by  96 by  96 cells was  used in  the FDS  calculation and provided
results  that agreed  well with those  predicted by  the RANS model and the various correlations.

Battaglia~{\em  et al.}~\cite{Battaglia:1} used  FDS to  simulate fire whirls.   First,  the  model  was  shown to  reproduce  the  McCaffrey
correlation  of  a  fire  plume,   then  it  was  shown  to  reproduce qualitatively certain features  of fire whirls. At the  time, FDS used
Lagrangian elements to introduce heat  from the fire (no longer used), and this  combustion model could not replicate  the extreme stretching of the
core of the flame zone.

Quintiere and Ma~\cite{Ma:2,Ma:3} compared predicted flame heights and plume  centerline temperatures to  empirical correlations.   For plume
temperature,   the  Heskestad   correlation~\cite{SFPE:Heskestad}  was chosen.  Favorable  agreement was found  in the plume region,  but the results
near  the  flame  region  were found  to  be  grid-dependent, especially for  low $Q^*$  fires.  At this  same time,  researchers at NIST were
reaching similar  conclusions, and it  was noticed  by both teams  that a  critical parameter  for the  model is  $D^*/\dx$, where $D^*$ is the
characteristic fire diameter and $\dx$  is the grid cell size.  If  this parameter  is  sufficiently  large,  the fire  can  be considered
well-resolved  and  agreement  with various  flame  height correlations was found. If the parameter is not large enough, the fire is not
well-resolved and adjustments  must be made to  the combustion routine to account for it.



\subsection{Comparisons with Previously Published Full-Scale Test Data}
\label{prevpub}

Experiments  conducted  solely   for  model  validation  are  somewhat rare.  More common  are validation  studies  that use  data from  past
experiments.  This   section  contains  brief   descriptions  of  work published  comparing  FDS with  past  experiments  or correlations  of
experimental data.

\subsubsection{Pool Fires}


Xin~{\em et  al.}~\cite{Xin:JSS2005} used FDS to model  a 1~m diameter methane pool  fire.  The  computational domain was  2~m by 2~m  by 4~m with  a
uniform  grid  size of  2.5  cm.  The  predicted results  were compared  to   experimental  data  and  found   to  qualitatively  and quantitatively
reproduce   the  velocity  field.   The  same  authors performed a similar study of a 7.1~cm methane burner~\cite{Xin:CF2005} and a helium
plume~\cite{Xin:CS2002}.

Hostikka~{\em  et al.}~\cite{Hostikka:3} modeled  small pool  fires of methane and methanol to test  the FDS radiation solver for low-sooting fires.
They conclude that  the predicted  radiative fluxes  for both fuels  are  higher than  measured  values,  especially  at small  heat release rates,
due to an over-prediction of the gas temperature.

Hietaniemi,  Hostikka and  Vaari~\cite{Hietaniemi:1}  consider heptane pool fires of various diameters.  Predictions of the burning rate as a
function  of  diameter  follow  the  trend observed  in  a  number  of experimental studies.  Their results show an improvement  in the model over
the earlier work with  methanol fires, due to improvements in the radiation  routine  and the  fact  that  heptane  is more  sooty  than methanol,
simplifying  the treatment of radiation.   The authors point out  that reliable  predictions of  the burning  rate of  liquid fuels require roughly
twice as fine a grid spanning the burner than would be necessary to predict plume velocities and temperatures. The reason for this is  the prediction
of the heat  feedback to the  burning surface necessary to {\em predict} rather  than to {\em prescribe} the burning rate.


\subsubsection{Airflows in Non-Fire Compartments}

The low Mach number assumption in FDS is appropriate not only to fire, but to  most building  ventilation scenarios.  An  example of  how the model
can  be used  to  assess indoor  air  quality  is presented  by Musser~{\em  et  al.}~\cite{Musser:1}.   The  test compartment  was  a displacement
ventilation  test   room   that  contained   computers, furniture, and  lighting fixtures as well as  heated rectangular boxes intended to  represent
occupants.  A detailed description  of the test configuration is  given by Yuan~{\em et  al.}~\cite{Yuan:1}.  The room is ventilated with  cool
supply air introduced via  a diffuser that is mounted on a side  wall near the floor. The air rises  as it is warmed by heat sources  and exits
through a return duct  located in the upper portion  of  the  room.  The   flow  pattern  is  intended  to  remove contaminants by sweeping  them
upward at the source  and removing them from the room.  Sulphur  hexafluoride, SF$_6$, was introduced into the compartment during the  experiment as
a tracer gas  near the breathing zone  of  the   occupants.   Temperature,  tracer  concentration,  and velocity were  measured during the
experiments.   For temperature, the two finest grids (50 by 36 by  24 and 64 by 45 by 30) produced results in  which the  agreement between  the
measurement  and  prediction was considered  acceptable.  The agreement  for the  tracer concentrations were  not as  good.  It  was suggested  that
the  difference  could be related to  the way  the source  of the tracer  gas was  modeled.  The comparison  of   velocity  data  was  deemed
reasonable,  given  the limitations of the velocity probes at low velocities.

In another  study, Musser and  Tan~\cite{Musser:2} used FDS  to assess the  design  of ventilation  systems  for  facilities  in which  train
locomotives  operate.  Although there  is  only  a  limited amount  of validation, the  study is useful  in demonstrating a practical  use of FDS for
a non-fire scenario.

Mniszewski~\cite{Mniszewski:1}  used  FDS  to  model  the  release  of flammable gases in simple enclosures and open areas. In this work, the gases
were not ignited.

Kerber  and Walton  provided a  comparison between  FDS version  1 and experiments on positive pressure ventilation in a full-scale enclosure without
a fire.   The model predictions of velocity  were within 10~\% to 20~\% of the experimental values~\cite{Kerber:1}.


\subsubsection{Wind Engineering}

Most applications  of FDS involve fires within  buildings. However, it can be used to model thermal  plumes in the open and wind impinging on the
exterior   of   a   building.    Rehm,   McGrattan,   Baum   and Simiu~\cite{LES:4} use the LES solver to estimate surface pressures on simple
rectangular blocks in  a crosswind, and compare these estimates to experimental measurements.  In a subsequent paper~\cite{Rehm:WS02}, they consider
the qualitative  effects of multiple buildings and trees on a wind field.

A   different    approach   to   wind    is   taken   by    Wang   and Joulain~\cite{Wang:IAFSS2002}. They  consider a  small fire in  a wind tunnel
0.4~m wide  and  0.7~m tall  with  flow speeds  of 0.5~m/s  to 2.5~m/s.  Much  of  the  comparison with  experiment  is  qualitative, including
flame shape,  lean,  length.  They also  use  the model  to determine  the  predominant  modes  of  heat  transfer  for  different operating
conditions. To  assess  the combustion,  they implement  an ``Eddy Break-up''  combustion model~\cite{Magnussen:1} and  compare it to the mixture
fraction approach used by FDS.  The two models perform better or  worse, depending on  the operating conditions. Some  of the weaknesses of the
mixture fraction model as implemented in FDS version 2 are addressed in subsequent versions. The Eddy Break-up approach has not been implemented in
the official version of FDS.

Chang and Meroney~\cite{ChangJWE2003} compared the results of FDS with the  commercial CFD  package  FLUENT in  simulating  the transport  of
pollutants   from  steady   point  sources   in  an   idealized  urban environment.  FLUENT  employs a  variety  of  RANS (Reynolds  Averaged
Navier-Stokes)  closure  methods,   whereas  FDS  employs  large  eddy simulation (LES).   The results of the numerical  models were compared with
wind tunnel measurements within a 1:50 scale physical model of an urban street "canyon".


\subsubsection{Growing Fires}
\label{growingfires}

Vettori~\cite{Vettori:1} modeled two different fire growth rates in an obstructed ceiling geometry.  The rectangular compartment was 9.2~m by 5.6~m
by 2.4~m  with a hollow steel door to  the outside that remained closed during the tests. An open wooden stairway led to an upper floor with the same
dimensions as the fire compartment below.  Wooden joists measuring 0.038~m by 0.24~m were spaced at 0.41~m intervals across the ceiling and  were
supported  by a single  steel beam that  spanned the width of the  room.  A rectangular methane gas  burner measuring 0.7~m by 1.0~m by 0.31~m was
placed  in the corner of the chamber.  Slow and fast  burning  fires  that  reached  1055~kW  in  600  s  and  150  s, respectively,  were
monitored.   Four   vertical  arrays  of  Type  K thermocouples were used to measure temperatures during the tests.  The FDS model used four grid
refinements and piecewise linear grid spacing for each fire growth rate (slow  and fast). For the fast growing fire, the predicted  temperatures were
within  20~\% of the  measured values and within  10~\% for the slow  growing fire. In  general, finer grids produced better agreement.

In a follow-up report,  Vettori~\cite{Vettori:2} extended his study to include sloped ceilings, with  and without obstructions. He found that the
difference between  predicted and  measured  sprinkler activation times varied  between 4~\%  and 26~\% for  all cases studied.  He also noted that
FDS was able to predict the first activation of a sprinkler twice  as far  from  the fire  as  another; caused  presumably by  the re-direction of
smoke by the beams on the ceiling.

Floyd~\cite{Floyd:5,Floyd:6} validated  FDS by comparing  the modeling results with  measurements from fire tests at  the Heiss-Dampf Reaktor (HDR)
facility.  The structure was originally the containment building for a nuclear power reactor  in Germany. The cylindrical structure was 20~m in
diameter and  50~m in height  topped by a  hemispherical dome 10~m  in radius.   The building  was divided  into eight  levels.  The total  volume of
the building  was approximately  11,000~m$^3$.  From 1984  to 1991, four  fire test  series were  performed within  the HDR facility.  The T51  test
series consisted of eleven  propane gas tests and three  wood crib  tests.  To avoid  permanently damaging  the test facility, a special set of  test
rooms were constructed, consisting of a fire  room with a narrow  door, a long corridor  wrapping around the reactor vessel  shield wall, and  a
curtained area centered  beneath a maintenance  hatch.   The  fire   room  walls  were  lined  with  fire brick. The doorway and corridor walls had
the same construction as the test chamber. Six gas burners were mounted in the fire room.  The fuel source was propane gas mixed with  10~\% air fed
at a constant rate to one of the  six burners.  For comparison with the  FDS model, only the fire room, hallway and curtained  region was input into
the model, for a total of 450,000 grid cells.  The burners were defined within FDS as separate vents  with a constant  inlet velocity.  Two sets  of
burners were created, the first set at the physical location of the burners as the source  of fuel and second set  directly above the first  set as a
source for  ambient air.   The data was  presented at  fifteen minutes into  the fire.
The FDS model predicted the  layer height and temperature of the space to within 10~\% of the experimental values~\cite{Floyd:5}.

FDS predictions of fire growth and smoke movement in large spaces were presented by  Kashef~\cite{Kashef:1}.  The experiments  were conducted at the
National Research Council  Canada.  The tests were performed in a compartment with  dimensions of 9~m by 6~m by  5.5~m with 32 exhaust inlets and a
single supply fan.  A burner  generated fires ranging in size  from 15~kW  to 1000~kW.   FDS produced  good predictions  of the experimental layer
temperatures and interface heights,  but there was some disagreement in the shape of the temperature profiles.


\subsubsection{Flame Spread}
\label{flame spread}

FDS was  evaluated to predict  the heat transfer  to the wall  from an adjacent pool fire.   The experimental results were based  on the work by Beck
{\em et al.}  The  predicted heat flux was  in agreement with the experimental  results.  The temperatures  are within 30~\%  of the measured values
near the base of  the wall but  decrease more rapidly than  the  experimental  measurements.   The  difference  between  the experimental and
predicted values  can be attributed to the combustion model within FDS.

The flame spread  calculations from FDS were compared  to the vertical flame  spread over  a 5~m  slab of  PMMA performed  by  Factory Mutual
Research  Corporation (FMRC).   The  predicted flame  spread rate  was within  0.3~m/s  for any  point  in  time  during the  analysis.   The
comparison at  the quasi-steady  burning rate once  the full  slab was burning     shows    that     FDS    over-estimated     the    burning
rate~\cite{Ma:2,Ma:3}.

A   charring  model   was   implemented  in   FDS   by  Hostikka   and McGrattan~\cite{Hostikka:2}.  The model  is a  simplification  of work done at
NIST by Ritchie  {\em et al.}~\cite{Ritchie:1}.  The charring model was  first used to  predict the burning  rate of a  small wooden sample in the
cone calorimeter.  The results were  more favorable for higher imposed heat fluxes. For  low imposed fluxes, the heat transfer at the edge  of the
sample was more pronounced,  and more difficult to model  accurately.   Full-scale room  tests  with  wood paneling  were modeled, but  the results
were  judged to be grid-dependent.  This was likely a consequence of the  gas phase spatial resolution, rather than the solid phase. The authors
concluded that it is difficult to predict the growth rate of a fire  in a wood-lined room without ``tuning'' the pyrolysis rate  coefficients. For
real  wood products, it  is unlikely that all  of the  necessary properties can  be obtained  easily. Thus, grid sensitivity  and uncertain  material
properties make  {\em blind} predictions of fire  growth on real materials beyond  the reach of the current version of the model. However, the model
can still be used for a qualitative assessment  of fire behavior as long  as the uncertainty in the flame spread rate is recognized.


\subsubsection{Response of Active and Passive Fire Protection}

A   significant  validation  effort   for  sprinkler   activation  and suppression was  a project entitled the  International Fire Sprinkler, Smoke
and Heat Vent, Draft  Curtain Fire Test Project organized by the National   Fire  Protection   Research  Foundation~\cite{McGrattan:5}. Thirty-nine
large scale  fire  tests were  conducted at  Underwriters Laboratories in  Northbrook, IL.  The  tests were aimed  at evaluating the performance of
various  fire protection systems in large buildings with  flat ceilings, like  warehouses and  ``big box''  retail stores. All the  tests were
conducted  under a 30~m by  30~m adjustable-height platform in a 37~m by 37~m by 15~m high test bay. At the time, FDS had not been publicly released
and  was referred to as the Industrial Fire Simulator (IFS), but it was essentially the same as FDS version 1.

For  model  validation  of  sprinkler activation,  the  most  valuable experiments performed were a series  of heptane spray fires.  With the spray
burner in different  locations, with and without draft curtains, with  and  without  vertical  vents,  the model  made  predictions  of sprinkler
activation and upper layer temperatures.  For all tests, the first ring  of sprinklers surrounding the fire  activated within 15~\% of the
experimental  times; within 25~\% for the  second ring. The gas temperatures near the ceiling were  predicted to within about 15~\% of the measured
values.

Most of the full-scale experiments performed during the project used a heptane  spray  burner  to   generate  controlled  fires  of  1~MW  to 10~MW.
However, 5  experiments  were performed  with  6~m high  racks containing the  Factory Mutual Standard Plastic Commodity,  or Group A Plastic. To
model these  fires, bench scale experiments were performed to characterize the burning behavior of the commodity, and larger test fires  provided
validation  data  with  which  to   test  the  model predictions    of    the     burning    rate    and    flame    spread
behavior~\cite{Hamins:1,Hamins:IAFSS2002}.     Two   to    four   tier configurations  were  evaluated.  For  the  period  of  time prior  to
application of water, the simulated heat release rate was within 20~\% of the experimental  heat release rates.  It should  be noted that the model
was very  sensitive to the thermal parameters  and the numerical grid when used to model the fire growth in the piled commodity tests.


High rack storage fires of pool chemicals were modeled by Olenick~{\em et  al.}~\cite{Olenick:1}  to  determine  the  validity  of  sprinkler
activation predictions  of FDS.  The model was  compared to full-scale fires conducted  in January, 2000  at Southwest Research  Institute in San
Antonio,  Texas.  The results indicated that  the model accurately predicted sprinkler activation and the over-pressurization of the test
compartment.

FDS  has  been  used  to  study  the behavior  of  a  fire  undergoing suppression     by     a    water     mist     system.     Kim     and
Ryou~\cite{Kim:BE2003,Kim:IJACR2004}   compared  FDS   predictions  to results of compartment fire tests  with and without the application of a water
mist. The cooling and oxygen dilution were predicted to within about 10~\% of the measurements, but the simulations failed to predict the complete
extinguishment of a hexane pool fire. The authors suggest that this is a result of the combustion model rather than the spray or droplet model.

Another study  of water  mist suppression using  FDS was  conducted by Hume   at   the    University   of   Canterbury,   Christchurch,   New
Zealand~\cite{Hume:Masters}. Full-scale  experiments were performed in which a fine  water mist was combined with  a displacement ventilation system
to protect occupants and electrical equipment in the event of a fire.  Simulations of  these experiments  with FDS  showed qualitative agreement, but
the version of the  model used in the study (version 3) was not able  to predict accurately the decrease  in heat release rate of the fire.

Hostikka    and   McGrattan~\cite{Hostikka:FSJ2006}    evaluated   the absorption of  thermal radiation by water sprays.  They considered two sets of
experimental data and concluded  that FDS has  the ability to predict the  attenuation of thermal radiation  ``when the hydrodynamic interaction
between   the  droplets  is   weak.''  However,  modeling interacting sprays would require a more costly coalescence model. They also note that  the
results of the model were  sensitive to grid size, angular discretization, and droplet sampling.

\subsubsection{Airflows in Fire Compartments}

Cochard~\cite{Cochard:1} used  FDS to  study the ventilation  within a tunnel. He  compared the model  results with a full-scale  tunnel fire
experiment conducted  as part of the  Massachusetts Highway Department Memorial Tunnel Fire Ventilation  Test Program.  The test consisted of a
single  point supply of  fresh air through  a 28~m$^2$ opening  in a 135~m tunnel.  The ventilation was started 2 min after the ignition of a  40~MW
fire.  Fifteen  temperature  measurement  trees were  placed within  the  tunnel and  replicated  within  the  model. Depending  on location,  the
difference between  predicted and  measured temperature rise ranged from 10~\% to 20~\%.

McGrattan and Hamins~\cite{McGrattan:HST} also applied FDS to simulate two of the Memorial Tunnel Fire Tests as validation for the use of the model
in  studying  an  actual  fire in  the  Howard  Street  Tunnel, Baltimore,  Maryland,  July  2001.  The  experiments  chosen  for  the comparison
were unventilated. One  experiment was  a 20~MW  fire; the other a 50~MW fire.  FDS predictions of peak near-ceiling temperatures were within
50~$^\circ$C of the measured peak temperatures, which were 600~$^\circ$C and 800~$^\circ$C, respectively.

Friday studied the  use of FDS in large  scale mechanically ventilated spaces.   The ventilated  enclosure  was provided  with air  injection rates
of  1 to 12 air  changes per hour  and a fire with  heat release rates ranging  from 0.5~MW to  2~MW.  The test measurements  and model output were
compared to assess the accuracy of FDS~\cite{Friday:1}.

Zhang {\em  et al.}~\cite{Zhang:2} utilized  the FDS model  to predict turbulence characteristics  of the flow and temperature  fields due to fire
in a compartment.   The experimental  data was  acquired through tests that replicated a half-scale ISO Room Fire Test.  Two cases were explored:
the heat  source in  the center  of the  room and  the heat source adjacent to a wall.  The heat source was a heating element with an output of
12~kW/m$^2$ and was  assumed stable after 300 s.  For the first case,  the predicted  average velocity and  temperature profiles were found to
``agree reasonably  well.''  Near the ceiling, the model under-predicted   temperature   and   over-predicted  velocity.    The predicted  intensity
of  the temperature  fluctuation  ``agree[d] very well'' at  all points  except those directly  adjacent to  the burner. The turbulent heat flux was
found to be larger in the region above the heat source.

The  second case also  used a  burner with  a 12~kW/m$^2$  heat source located  at the  wall.  As  with the  first case,  the  predicted mean
velocities  agree  with  the  experimental  results  except  near  the ceiling.   The  temperatures  near   the  ceiling  were  found  to  be
over-predicted by FDS.  The  intensity of the velocity fluctuation was found to  ``agree well''  with the experimental  data except  near the
ceiling.   The  predicted  intensity  of the  temperature  fluctuation agrees ``very well''  with the experimental data except  in the region near
the  middle of the room.  This  might be due to  the influence of the door  sill.  Overall, in  both cases, the predicted  values agreed well  with
the  experimental values  in  all regions  except near  the ceiling.


The ability of  version 1 of FDS to  accurately predict smoke detector activation was studied by D'Souza~\cite{DSouza:1}. The smoke transport model
within FDS  was tested and compared with UL  217 test data.  The second step  in this research was  to further validate  the model with full-scale
multi-compartment fire tests.  The results  indicated that FDS is capable of predicting  smoke detector activation when used with smoke  detector
lag correlations  that  correct  for  the time  delay associated with smoke having to penetrate the detector housing.

Another study of smoke detector  activation was carried out by Brammer at  the University  of Canterbury,  New  Zealand~\cite{Brammer:1}. Two fire
tests from  a series  performed  in a  two-story residence  were simulated, and  smoke detector  activation times were  predicted using three
different methods. The methods consisted of either a temperature correlation,  a time-lagged  function  of the  optical  density, or  a thermal
device much like a heat detector.  The purpose was to identify ways to reliably predict smoke detector activation using typical model output like
temperature and  smoke concentration. It was remarked that simulating  the  early stage  of  the  fire  is critical  to  reliable prediction.

Cleary~\cite{Cleary:1} also provided a comparison between FDS computed gas  velocity,  temperature  and  concentrations at  various  detector
locations.   The research concluded  that multi-room  fire simulations with the FDS model can accurately predict the conditions that a sensor might
experience during a real fire  event.  The FDS model was able to predict the smoke and gas concentrations, heat, and flow velocities at various
detector locations to within 15~\% of measurements.

Piergoirgio~{\em  et al.}~\cite{Piergiorgio:1} provided  a qualitative analysis of FDS applied to a  truck fire within a tunnel.  The goal of their
analysis  was to describe the  spread of the  toxic gases within the tunnels, to determine the  places not involved in the spreading of combustion
products  and to quantify  the oxygen, carbon  monoxide and hydrochloric acid concentrations during the fire.

Edwards~{\em  et al.}~\cite{Edwards:SME2005,Edwards:FSJ}  used  FDS to determine the critical air velocity  for smoke reversal in a tunnel as a
function of  the fire intensity, and his  results compared favorably with   experimental  results.   In  a   further  study,   Edwards  and
Hwang~\cite{Edwards:SME2006}  applied FDS to  study fire  spread along combustibles in  a ventilated mine  entry. Analyses such as  these are
intended for planning and implementation of ventilation changes during mine fire fighting and rescue operations.


\subsubsection{Combustion Model}

Floyd~{\em et al.}~\cite{Floyd:1,Floyd:6} compared the radiation model of  FDS version 2  with full-scale  data from  the Virginia  Tech Fire
Research Laboratory (VTFRL).  The  test compartment was outfitted with equipment   capable   of  taking   temperature,   air  velocity,   gas
concentrations, unburned hydrocarbon  and heat flux measurements.  The test facility consisted of  a single compartment geometrically similar to the
ISO 9705 standard compartment with dimensions of 1.2~m by 1.8~m by  1.2~m  in height.   The  ceiling  and  walls were  constructed  of fiberboard
over  a steel  shell  with  a  floor of  concrete.   Three baseline experiments  were completed with  fires ranging in  size from 90~kW  to 440~kW.
Overall,  FDS  predicted  the  temperatures  to within  15~\%  of  the measured  temperatures.  The  FDS velocity  measurements  followed the trend
of  the test  data  but did  not  replicate  it.  The  outgoing velocities  were under-predicted by  30~\% to  40~\% and  the incoming velocities
were  over-predicted by 40~\%. FDS predicted  the heat flux gauge response to within 10~\%  of the measured values.  The radiation model  in FDS
predicted the  measured  fluxes to  within 15~\%.

Xin   and  Gore~\cite{Xin:JSS2003}   compared   FDS  predictions   and measurements  of the  spectral radiation  intensities of  small fires. The
fuel flow rates for  methane and ethylene burners were selected so that the  Froude numbers  matched that of  liquid toluene  pool fires. The heat
release rate was 4.2~kW  for the methane flame and 3.4~kW for the ethylene flame.  Line of sight spectral radiation intensities were measured  at
six  downstream  locations.    The  spectral  radiation intensity calculations were performed by post-processing the transient scalar  distributions
provided  by FDS.   The calculated  and measured spectral  radiation  intensities were  found  to  be in  ``excellent'' agreement for the gas
radiation bands.

Zhang~{\em et al.}~\cite{Zhang:1} compared the experimental results of a  circular methane  gas burner  to predictions  computed by  FDS. The
compartment was 2.8~m by 2.8~m  by 2.2~m high with natural ventilation from a  standard door.  Good agreement was  found for  the temperature
prediction at the doorway where  the radiation model was used. The FDS model predicted the temperatures at the corner of the room better than other
models  compared by the group.  It was found  that, overall, FDS predicted temperatures  well but the prescribed  turbulent Prandtl and Schmidt
numbers play an important  role in determining the accuracy of the model.


Bundy,  Dillon and  Hamins~\cite{Dillon:1,Hamins:FPE2005}  studied the use of FDS  in providing data and correlations  for fire investigators to
support their investigations.   A paraffin  wax candle  was placed within  a  0.61~m by  0.61~m  by  0.76~m  plexi-glass enclosure.   The chamber was
raised 20~mm off the surface to reveal 44 uniformly spaced 6~mm diameter holes.   The holes provided oxygen to  the flame without subjecting the
flame to a draft.   A 150~mm hole was  provided at the top of the enclosure to allow  for the heat and combustion products to exit the space.  The
heat flux  from the candle flame was modeled with FDS.  The model  provides a prediction of the heat  flux of the candle at a height of  56~mm above
the base of the flame  with an accuracy of 5~\%. The flux is under predicted  by 16~\% at 76~mm above the base of the  flame. The remainder  of the
predictions show  flux measurements were under-predicted by 15~\% to 40~\% of the measured values.



\subsection{Comparison with Standard Tests}

Standard fire tests are  performed at various testing laboratories and universities  around  the world.   While  most  were  not designed  as
validation tools, they nevertheless  can be used as relatively simple, well characterized fire experiments.

An extensive  amount of  validation work with  FDS version 4  has been performed    by   Hietaniemi,    Hostikka,   and    Vaari    at   VTT,
Finland~\cite{Hietaniemi:1}.  The  case studies are  comprised of fire experiments ranging  in scale  from the cone  calorimeter (ISO~5660-1, 2002)
to full-scale fire tests such as the room corner test (ISO~9705, 1993).  Comparisons are  also  made  between FDS  4  results and  data obtained  in
the  SBI (Single  Burning Item)  Euro-classification test apparatus (EN  13823, 2002) as  well as data  obtained in two  {\em ad hoc} experimental
configurations: one is  similar to the  room corner test but  has only partial linings and  the other is a  space to study fires in building
cavities. In the study of upholstered furniture, the experimental configurations  are the cone  and furniture calorimeters, and the  ISO room. For
liquid  pool fires, comparison is  made to data obtained  by  numerous  researchers.   The burning  materials  include spruce  timber, MDF  (Medium
Density  Fiber) board,  PVC  wall carpet, upholstered furniture, cables with plastic sheathing, and heptane.

The scope of the VTT work is considerable. Assessing the accuracy of the model must be done on a case by case basis. In some cases, predictions of
the burning rate of the material were based solely on its fundamental properties, as in the heptane pool fire simulations. In other cases, some
properties of the material are unknown, as in the spruce timber simulations. Thus, some of the simulations are true predictions, some are
calibrations. The intent of the authors was to provide guidance to engineers using the model as to appropriate grid sizes and material properties. In
many cases, the numerical grid was made fairly coarse to account for the fact that in practice, FDS is used to model large spaces of which the fuel
may only comprise a small fraction.


\subsection{Comparison with Documented Fire Experience}

Documented fire experience includes known behavior of materials in fires, eyewitness accounts of real fires, observed post fire conditions, and other
means. To date, several actual fires have been reconstructed using FDS. One case study performed by NIST is documented in Ref.~\cite{Madrzykowski:1}.
Two fire fighters were killed and one severely injured in a townhouse fire in Washington, D.C. during the evening of May 30, 1999. Questions arose
about the injuries the fire fighters had sustained, the lack of thermal damage in the living room where a fallen fire fighter was found and why the
fire fighters never opened their hose lines to protect themselves or to extinguish the fire.

To answer some of the questions, a rectangular volume of 10~m by 6~m by 5.1~m was divided into 76,500 cells in the FDS model. The FDS results that
best replicated the observed fire behavior indicated that the opening of the basement sliding glass door provided oxygen to a pre-heated,
under-ventilated fire. Flashover was estimated to occur in less than 60 s following the entry of fire fighters into the basement. The resulting fire
gases flowed up the basement stairs and moved across the living room ceiling towards the back wall of the townhouse. These hot gases came in direct
contact with the fire fighters who were killed. The hot gases traversed the townhouse in less than 2 s, giving the fire fighters little time to
respond. The model showed that the oxygen level was too low to support flaming and, therefore, the fire fighters did not have a visual cue of the
thermal conditions until it was too late. Results of the FDS study were shared with the D.C. fire department and have been made available via a
multi-media CD-ROM to other fire departments across the country.

Another case study performed at NIST involved a fire in a Houston restaurant~\cite{Texas}. On the morning of February 14, 2000, a fire started in the
office area of a fast food restaurant. Two fire fighters died when the roof collapsed. The FDS model was used to simulate the fire. The fuel was
assumed to be the contents of a typical office, and the fire was assumed to have a slowly growing heat release rate peaking at 6~MW. Multiple vents
were modeled and the time at which they opened replicated the fire fighters' actions after arrival.  The model provided a visual representation of
the fire during the initial phases until the collapse of the roof.

NIST also performed a case study on a fire that killed three children and three fire fighters on the morning of December 22, 1999~\cite{Iowa}. The
fire started on top of a stove in a two-story residence. FDS was used to simulate the fire.  The fuel packages consisted of several furniture items
in the kitchen and living room with heat release rates reaching 5.2~MW. The model results indicated the critical event in the fire was flashover of
the kitchen. The fire became a multi-room event after flashover with temperatures increasing to over 600~$^\circ$C. The hot gases spread quickly from
the living room to the stairway on the second floor trapping the fire fighters.

Outside of NIST, FDS has been used to investigate many actual fires, but very few of these studies are documented in the literature. Exceptions
include a study by Rein~{\em et al.}~\cite{Rein:Interflam2004} looking at several fire events using an analytical fire growth model, the NIST zone
model CFAST, and FDS. A similar study was performed several years earlier by Spearpoint {\em et al.}~\cite{Spearpoint:ICFRE3} as a class exercise at
the University of Maryland. During the SFPE Professional Development Week in the fall of 2001, a workshop was held in which several engineers related
their experiences using FDS as a forensic tool~\cite{Carpenter:SFPE2001}. The role of carbon monoxide in the deaths of three fire fighters was
studied by Christensen and Icove~\cite{Christensen:JFS}. There is little quantitative validation of the model afforded by these studies. However, the
degree to which the model is able to reproduce observed behavior can be used as an indicator of the model's strengths and


\chapter{Description of Experiments}

This chapter contains a brief description of the experiments that were used for model validation. Only enough detail is included here to provide a
general understanding of the model simulations. Anyone wishing to use the experimental measurements for validation ought to consult the cited test reports for a 
comprehensive description.



\section{VTT Large Hall Tests}

The experiments are described in Ref.~\cite{Hostikka:Hall}. The series consisted of 8 experiments, but because of replicates only three unique fire
scenarios. The experiments were undertaken to study the movement of smoke in a large hall with a sloped ceiling. The tests were conducted inside the
VTT Fire Test Hall, with dimensions of 19 m (62 ft) high by 27 m (89 ft) long by 14 m (46 ft) wide. Figure shows detailed plan, side and perspective
schematic diagrams of the experimental arrangement. Each test involved a single heptane pool fire, ranging from 2~MW to 4~MW. Figure is a photo of a
2~MW fire. Four types of measurements were used in the present evaluation -- the HGL temperature and depth, average flame height and the plume
temperature. Three vertical arrays of thermocouples, plus two thermocouples in the plume, were compared to model simulation results. The HGL
temperature and height were reduced from an average of the three TC trees using the standard algorithm. The ceiling jet temperature was not
considered, because the ceiling in the test hall is not flat, and the standard model algorithm is not appropriate for these conditions.

The VTT test report lacks some information needed to model the experiments, so some information was based on private communications with the
principal investigator, Simo Hostikka. The information used to conduct the model simulations is presented in Table 2-3, including information on the
fire, the compartment, and the ventilation.

Surface Materials: The walls and ceiling of the test hall consist of a 1 mm (0.039 in) thick layer of sheet metal on top of a 5 cm (2 in) layer of
mineral wool. The floor was constructed of concrete. The report does not provide thermal properties of these materials. Thermophysical properties of
the materials that were used in the simulations are given in Chapter 3.

Natural Ventilation: In Cases 1 and 2, all doors were closed, and ventilation was restricted to infiltration through the building envelope. Precise
information on air infiltration during these tests is not available. The scientists who conducted the experiments recommend a leakage area of about 2
m2 (20 ft2), distributed uniformly throughout the enclosure. By contrast, in Case 3, the doors located in each end wall (Doors 1 and 2, respectively)
were open to the external ambient environment. These doors are each 0.8 m (2.6 ft) wide by 4 m (5 ft) high, and are located such that their centers
are 9.3 m (30.5 ft) from the south wall.

Mechanical Ventilation: The test hall had a single mechanical exhaust duct, located in the roof space, running along the center of the building. This
duct had a circular section with a diameter of 1 m (40 in), and opened horizontally to the hall at a distance of 12 m (39 ft) from the floor and 10.5
m (34.4 ft) from the west wall. Mechanical exhaust ventilation was operational for Case 3, with a constant volume flow rate of 11 m3/s drawn through
the 1 m (40 in) diameter exhaust duct. 2-14 Heat Release Rate: Each test used a single fire source with its center located 16 m (52 ft) from the west
wall and 7.4 m (24.3 ft) from the south wall. For all tests, the fuel was heptane in a circular steel pan that was partially filled with water. The
pan had a diameter of 1.17 m (46.0 in) for Case 1 and 1.6 m (63 in) for Cases 2 and 3. In each case, the fuel surface was 1 m (40 in) above the
floor. The trays were placed on load cells, and the HRR was calculated from the mass loss rate (see definition in Chapter 3). For the three cases,
the fuel mass loss rate was averaged from individual replicate tests. In the HRR estimation, the heat of combustion (taken as 44.6 kJ/g) and the
combustion efficiency for n-heptane was used. Hostikka suggests a value of 0.8 for the combustion efficiency. Bundy [Ref. 23] estimates the
efficiency of a 500 kW heptane pool fire to be equal to 0.97. Tewarson reports a value of 0.93 for a 10 cm pool [Ref. 17]. The magnitude of the
combustion efficiency is a complicated function of fire size, ventilation, and other effects. Consideration of the chemical structure of a fire
suggests that the combustion efficiency should decrease as the fire size grows. Available data confirms this [Ref. 24]. The size of a compartment may
also impact this parameter, but there is little information in the fire literature that addresses this point. In summary, there is little certainty
in the actual value of the combustion efficiency in this experiment. In this report, a combustion efficiency of 0.85 � 0.12
(or � 14 %) is recommended for the BE #2 pool fire tests, based on engineering judgment. Due
to the relatively large value of the uncertainty associated with .a, the uncertainty in HRR is dominated by the uncertainty in the combustion
efficiency. Uncertainty in the mass loss rate measurement also contributed to the overall uncertainty, and the uncertainty in HRR was estimated
as 15 %. Figures 2-9 to 2-11 show the prescribed HRR as a function of time during Cases 1 to 3,
respectively. Tables 2-4 to 2-6 represent the mass loss and estimated HRR associated with Figures 2-9 to 2-11, respectively.

Radiative Fraction: The radiative fraction was assigned a value of 0.35, similar to many smoky hydrocarbons [Ref. 19]. The relative combined expanded
(2s) uncertainty in this parameter was assigned a
value of �20 %, which is typical of uncertainty values reported in the literature for this
parameter.



\clearpage

\section{UL/NFPRF Sprinkler, Vent, and Draft Curtain Study}
\label{UL_NFPRF_Description}

In January, 1997, a series of 22 heptane spray burner experiments was conducted at the Large Scale Fire Test Facility at Underwriters Laboratories
(UL) in Northbrook, Illinois~\cite{Sheppard:1}. The objective of the experiments was to characterize the temperature and flow field for fire
scenarios with a controlled heat release rate in the presence of sprinklers, draft curtains and a single smoke \& heat vent.

\subsubsection{Test Facility}

The Large Scale Fire Test Facility at UL contains a 37~m by 37~m (120~ft by 120~ft) main fire test cell, equipped with a 30.5~m by 30.5~m (100~ft by
100~ft) adjustable height ceiling. The layout of the experiments is shown in Fig.~\ref{layout}. One 1.2~m by 2.4~m (4~ft by 8~ft) vent was installed
among 49 upright sprinklers on a 3~m by 3~m (10~ft by 10~ft) spacing.

\begin{figure}[p]
\begin{center}
\setlength{\unitlength}{.05416667in}
\begin{picture}(120,120)

\linethickness{1.mm} \put(0,0){\framebox(120,120)[tc]{North Wall}} \linethickness{.5mm} \put(10,10){\framebox(100,100)[tc]{Adjustable Height
Ceiling}}

\thinlines \put(117,67){\vector(0,-1){67}} \put(117,73){\vector(0, 1){47}} \put(117,70){\makebox(0,0){$120'$}} \put(113,57){\vector(0,-1){47}}
\put(111,110){\line(1,0){4.}} \put(111, 10){\line(1,0){4.}} \put(113,63){\vector(0, 1){47}} \put(113,60){\makebox(0,0){$100'$}}
\put(30.9,12.83){\dashbox{1}(67.1,71.17)[tc]{Draft Curtains}} \put(27.9,40){\vector(0,-1){27.17}} \put(27.9,46){\vector(0, 1){38.0}}
\put(25.9,84.){\line(1,0){4.}} \put(25.9,12.83){\line(1,0){4.}} \put(27.9,43){\makebox(0,0){$71'2''$}} \put(64.0,87.){\vector(-1,0){33.1}}
\put(72.0,87.){\vector( 1,0){26.0}} \put(30.9,85.){\line(0,1){4.}} \put(98.0,85.){\line(0,1){4.}} \put(68.0,87.){\makebox(0,0){$67'1''$}}

\put(16.0,87.){\vector(-1,0){6.}} \put(24.0,87.){\vector( 1,0){6.92}} \put(20.0,87.){\makebox(0,0){$20'11''$}} \put(101.,87.){\vector(-1,0){3.}}
\put(107.,87.){\vector( 1,0){3.}} \put(104.,87.){\makebox(0,0){$12'$}}

\put(27.9,100){\vector(0,1){10.}} \put(27.9,94){\vector(0,-1){10.}} \put(27.9,97){\makebox(0,0){$26'$}}

\put(27.9,8){\vector(0,1){2.}} \put(27.9,8){\line(1,0){3.}} \put(30.9,8){\makebox(0,0)[l]{$2'10''$}}

\put(55.08,14.83){\line(-1,0){2.}} \put(54.08,16.83){\vector(0,-1){2.}} \put(54.08,7.83){\vector(0,1){5.}} \put(54.08,7.83){\line(1,0){3.}}
\put(57.08,7.83){\makebox(0,0)[l]{$2'$}}

\put(85.08,24.83){\line(0,-1){2.}} \put(95.08,24.83){\line(0,-1){2.}} \put(93.08,23.83){\vector(1,0){2.}} \put(87.08,23.83){\vector(-1,0){2.}}
\put(103.00,23.83){\vector(-1,0){5.}} \put(103.00,23.83){\line(0,-1){3.}} \put(103.00,20.83){\makebox(0,0)[ct]{$2'11''$}}
\put(90.08,23.83){\makebox(0,0)[c]{$10'$}}

\thicklines \put(78.08,55.83){\framebox(4,8){ }}

\put(78.58,58.33){\framebox(3,3)[c]{A}} \put(78.58,68.33){\framebox(3,3)[c]{B}} \put(88.58,58.33){\framebox(3,3)[c]{C}}
\put(58.58,38.33){\framebox(3,3)[c]{D}}

\thinlines

\multiput(35.08,14.83)(0,10){7}{\circle*{.8}} \multiput(45.08,14.83)(0,10){7}{\circle*{.8}} \multiput(55.08,14.83)(0,10){7}{\circle*{.8}}
\multiput(65.08,14.83)(0,10){7}{\circle*{.8}} \multiput(75.08,14.83)(0,10){7}{\circle*{.8}} \multiput(85.08,14.83)(0,10){7}{\circle*{.8}}
\multiput(95.08,14.83)(0,10){7}{\circle*{.8}} \tiny \put(35.48,15.23){98} \put(45.48,15.23){91} \put(55.48,15.23){84} \put(65.48,15.23){81}
\put(75.48,15.23){78} \put(85.48,15.23){75} \put(95.48,15.23){72} \put(35.48,25.23){99} \put(45.48,25.23){92} \put(55.48,25.23){85}
\put(65.48,25.23){82} \put(75.48,25.23){79} \put(85.48,25.23){76} \put(95.48,25.23){73} \put(35.48,35.23){100} \put(45.48,35.23){93}
\put(55.48,35.23){86} \put(65.48,35.23){83} \put(75.48,35.23){80} \put(85.48,35.23){77} \put(95.48,35.23){74} \put(35.48,45.23){101}
\put(45.48,45.23){94} \put(55.48,45.23){87} \put(65.48,45.23){62} \put(75.48,45.23){58} \put(85.48,45.23){54} \put(95.48,45.23){50}
\put(35.48,55.23){102} \put(45.48,55.23){95} \put(55.48,55.23){88} \put(65.48,55.23){63} \put(75.48,55.23){59} \put(85.48,55.23){55}
\put(95.48,55.23){51} \put(35.48,65.23){103} \put(45.48,65.23){96} \put(55.48,65.23){89} \put(65.48,65.23){64} \put(75.48,65.23){60}
\put(85.48,65.23){56} \put(95.48,65.23){52} \put(35.48,75.23){104} \put(45.48,75.23){97} \put(55.48,75.23){90} \put(65.48,75.23){65}
\put(75.48,75.23){61} \put(85.48,75.23){57} \put(95.48,75.23){53} \put(70.08,49.83){\makebox(0,0)[c]{68}} \put(70.08,59.83){\makebox(0,0)[c]{69}}
\put(70.08,69.83){\makebox(0,0)[c]{70}} \put(80.08,49.83){\makebox(0,0)[c]{67}} \put(90.08,49.83){\makebox(0,0)[c]{66}}
\put(90.08,69.83){\makebox(0,0)[c]{71}}

\multiput(80.08,56.83)(0,1){7}{\circle*{.2}} \put(80.58,62.83){\line(1,0){22.5}} \put(104.,62.83){\makebox(0,0)[l]{43}}
\put(104.,61.33){\makebox(0,0)[l]{44}} \put(104.,59.83){\makebox(0,0)[l]{45}} \put(104.,58.33){\makebox(0,0)[l]{46}}
\put(104.,56.83){\makebox(0,0)[l]{47}} \put(104.,55.33){\makebox(0,0)[l]{48}} \put(104.,53.83){\makebox(0,0)[l]{49}}

\normalsize

\end{picture}
\end{center}
\caption[Plan view of the UL/NFPRF Experiments.] {\bf Plan view of the UL/NFPRF Experiments. The sprinklers are indicated by the solid circles and
are spaced 3~m (10~ft) apart. The number beside each sprinkler location indicates the channel number of the nearest thermocouple. The vent dimensions
are 4~ft by 8~ft. The boxed letters A, B, C and D indicate burner positions. Corresponding to each burner position is a vertical array of
thermocouples. Thermocouples 1--9 hang 7, 22, 36, 50, 64, 78, 92, 106 and 120~in from the ceiling, respectively, above Position A. Thermocouples 10
and 11 are positioned above and below the ceiling tile directly above Position B, followed by 12--20 that hang at the same levels below the ceiling
as 1--9. The same pattern is followed at Positions C and D, with thermocouples 21--31 at C and 32--42 at D.} \label{layout}
\end{figure}

The ceiling was raised to a height of 7.6~m (25~ft) and instrumented with thermocouples and other measurement devices. The ceiling was constructed of
0.6~m by 1.2~m by 1.6~cm (2~ft by 4~ft by 5/8~in) UL fire rated Armstrong Ceramaguard (Item 602B) ceiling tiles. The manufacturer reported the
thermal properties of the material to be: specific heat 753 J/kg$\cdot$K, thermal diffusivity $2.6 \times 10^{-7}$~m$^2$/s, conductivity
0.0611~W/m$\cdot$K, and density 313~kg/m$^3$.

Draft curtains 1.8~m (6~ft) deep were installed for 16 of the 22 tests, enclosing an area of about 450~m$^2$ (4,800~ft$^2$). The curtains were
constructed of 1.4~m (54~in) wide sheets of 18 gauge sheet metal.

The sprinklers used were Central ELO-231 (Extra Large Orifice) uprights. The orifice diameter of this sprinkler is reported by the manufacturer to be
nominally 0.64~in, the reference actuation temperature is reported by the manufacturer to be 74$^\circ$C (165$^\circ$F). The RTI (Response Time
Index) and C-factor (Conductivity factor) were reported by UL to be 148~(m$\cdot$s)$^\ha$ (268~(ft$\cdot$s)$^\ha$) and 0.7~(m/s)$^\ha$
(1.3~(ft/s)$^\ha$), respectively~\cite{Sheppard:1}. When installed, the sprinkler deflector was located 8~cm (3~in) below the ceiling. The thermal
element of the sprinkler was located 11~cm (4.25~in) below the ceiling. The sprinklers were installed with 3~m by 3~m (10~ft by 10~ft) spacing in a
system designed to deliver a constant 0.34~L/(s$\cdot$m$^2$) (0.50 gpm/ft$^2$) discharge density when supplied by a 131~kPa (19~psi) discharge
pressure

A single UL listed double leaf fire vent with steel covers and steel curb was installed in the adjustable height ceiling in the position shown in
Fig.~\ref{layout}. The vent is designed to open manually or automatically. The vent doors were recessed into the ceiling about 0.3~m (1~ft).

\subsubsection{Fire and Heat Release Rate}

The heptane spray burner consisted of a 1~m by 1~m (40~in by 40~in) square of 1/2~in pipe supported by four cement blocks 0.6~m (2~ft) off the floor.
Four atomizing spray nozzles were used to provide a free spray of heptane that was then ignited. For all but one of the tests, the total heat release
rate from the fire was manually ramped up following the curve
$$ \dot{Q} = \dot{Q}_0 \; \left( \frac{t}{\tau} \right)^2 $$
where $\dot{Q}_0=10$~MW and $\tau=75$~s ($\tau=150$~s was used in Test I-16). The fire growth curve was followed until a specified fire size was
reached or the first sprinkler activated. After either of these events, the fire size was maintained at that level until conditions reached roughly a
steady state, {\em i.e.} the temperatures recorded near the ceilings remained steady and no more sprinkler activations occurred.

The heat release rate from the burner was confirmed by placing it under the large product calorimeter at UL, ramping up the flow of heptane in the
same manner as in the tests, and measuring the total and convective heat release rates. It was found that the convective heat release rate was
0.65$\pm$0.02 of the total.

\subsubsection{Instrumentation}

The instrumentation for the tests consisted of thermocouples, gas analysis equipment, and pressure transducers. The locations of the instrumentation
are referenced in the plan view of the facility (Fig.~\ref{layout}).

Temperature measurements were recorded at 104 locations. Type K 0.0625~in diameter Inconel sheathed thermocouples were positioned to measure (i)
temperatures near the ceiling, (ii) temperatures of the ceiling jet, and (iii) temperatures near the vent. The thermocouples numbered 50--65 were
positioned near the sprinklers, 10~cm (4~in) below the ceiling. These were intended to measure near-sprinkler gas temperatures as well as to detect
sprinkler activation when wetted. Thermocouples 66--104 were placed 5~cm (2~in) below the ceiling. Thermocouples 43--49 ran down the centerline of
the vent at the level of the ceiling, and were spaced 0.3~m (1~ft) apart. Thermocouples 1--42 were mounted on arrays hanging above each fire
location. The positions are noted in the caption to Fig.~\ref{layout}.

Oxygen, carbon dioxide and carbon monoxide sampling probes were placed at the ground (5~cm (3~in) from the floor, 2~m (6~ft) from the burner), and at
the vent (15~cm (6~in) below the ceiling, vent center).


\clearpage

\section{NIST/NRC Test Series}

These experiments, sponsored by the US NRC and conducted at NIST, consisted of 15 large-scale experiments performed in June 2003. All 15 tests were
included in the validation study. The experiments are documented in Ref.~\cite{Hamins:SP1013-1}. The fire sizes ranged from 350 kW to 2.2 MW in a compartment with dimensions
21.7~m by 7.1~m by 3.8~m high, designed to represent a compartment in a nuclear power plant containing power and control cables.
The walls and ceiling were covered with two layers of marinate boards, each layer 0.0125~m thick. The floor
was covered with one layer of gypsum board on top of a layer of plywood. Thermo-physical and optical properties of the marinate
and other materials used in the compartment are given in Ref.~\cite{Hamins:SP1013-1}. The room had one door and a mechanical air injection and extraction
system. Ventilation conditions, the fire size, and fire location were varied. Numerous measurements (approximately 350 per test) were made including
gas and surface temperatures, heat fluxes and gas velocities.

Natural Ventilation: The compartment had a 2 m by 2 m door in the middle of the west wall. Some of the tests had a closed door and no mechanical
ventilation (Tests 2, 7, 8, 13, and 17), and in those tests the measured compartment leakage was an important consideration. Ref. [6] reports leakage
area based on measurements performed prior to Tests 1, 2, 7, 8, and 13. For the closed door tests, the leakage area used in the simulations ought to
be based on the last available measurement. It should be noted that the chronological order of the tests differed from the numerical order [Ref. 6].
For Test 4, it is recommended that the leakage area measured before Test 2 be used. For Tests 10 and 16, it is recommended that the leakage area
measured before Test 7 be used.

Mechanical Ventilation: The mechanical ventilation and exhaust was used during Tests 4, 5, 10, and 16, providing about 5 air changes per hour. The
door was closed during Test 4 and open during Tests 5, 10, and 16. The supply duct was positioned on the south wall, about 2 m off the floor. An
exhaust duct of equal area to the supply duct was positioned on the opposite wall at a comparable location. The flow rates through the supply and
exhaust ducts were measured in detail during breaks in the testing, in the absence of a fire. During the tests, the flows were monitored with single
bi-directional probes during the tests themselves.

Heat Release Rate: A single nozzle was used to spray liquid hydrocarbon fuels onto a 1 m by 2 m fire pan that was about 0.02 m deep. The test plan
originally called for the use of two nozzles to provide the fuel spray. Experimental observation suggested that the fire was less unsteady with the
use of a single nozzle. In addition, it was observed that the actual extent of the liquid pool was well-approximated by a 1 m circle in the
center of the pan. For safety reasons, the fuel flow was terminated when the lower-layer oxygen concentration
dropped to approximately 15~\% by volume.
The fuel used in 14 of the tests was heptane, while toluene was used for one test. The HRR was
determined using oxygen consumption calorimetry. The recommended uncertainty values
were 17~\% for all of the tests.

Radiative Fraction: The radiative fraction was measured in an independent study for the same fuels using the same spray burner as used in the test
series [Ref. 18]. The value of the radiative fraction and its
uncertainty were reported as 0.44 � 16 \% and 0.40 � 23 \% for heptane and toluene, respectively.

\clearpage

\section{McCaffrey's Plume Experiments}

In 1979, at the National Bureau of Standards (now NIST), Bernard McCaffrey measured centerline temperature and velocity profiles above a porous,
refractory burner. There were five distinct heat release rates, ranging from 14~kW to 57~kW. The fuel was natural gas. The burner was square, 0.3~m on each side.
The results of the experiments are reported in Reference~\cite{McCaffrey:NBSIR_79-1910}.



\section{NIST Diffusion Flame Test Series}

Smyth {\em et al.} conducted diffusion flame experiments at NIST used a methane/air Wolfhard-Parker slot burner.  The experiments are described in
detail in Refs.~\cite{Norton:1,Smyth:1}.  The Wolfhard-Parker slot burner consists of an 8~mm wide
central slot flowing fuel surrounded by two 16~mm wide slots flowing dry air with 1~mm separations between the slots.
The slots are 41~mm in length.  Measurements were made of all major species and a number of minor species along with temperature
and velocity.  Experimental uncertainties have been reported as 5~\% for temperature  and 10~\% to 20~\%
for the major species.

\section{Beyler Hood Experiments}

Beyler performed a large number of experiments involving a variety of fuels, fire sizes, burner diameters, and
burner distances beneath a hood~\cite{Beyler:Hood}.  The hood consisted of concentric cylinders separated
by an air gap.  The inner cylinder was shorter than the outer and this allowed combustion products to be removed
uniformly from the hood perimeter.  The exhaust gases were then analyzed to determine species concentrations.
The burner could be raised and lowered with respect to the bottom edge of the hood.  Based on the published
measurement uncertainties, species errors are estimated at 6~\%.

\section{NIST Reduced Scale Enclosure CO Production Test Series}

The CO production test series used the NIST Reduced Scale Enclosure (RSE)~\cite{Bryner:1}.  The RSE is a
40~\% scaled version of the ISO 9705 compartment.
It measures 0.98 m wide by 1.46 m deep by 0.98 m tall.  The compartment contains a door centered on the small
face that measures 0.48 m wide by 0.81 m tall.  A 15 cm diameter natural gas burner was positioned in the
center of the compartment.  The burner was on a stand so that its top was 15~cm above the floor.
Species measurements were made inside the upper layer of the compartment at the front near the door
and near the rear of the compartment.

\section{NIST Flame Radiation Test Series}

Hamins {\em et al.} performed a series of tests on circular gas
burners measuring the radial and vertical radiative heat flux profiles
outside the flame region. The tests are described
in~\cite{Hostikka:3}. Tests at three burner diameters, 0.10 m, 0.38 m
and 1.0 m are used for validation. The parameters of the tests are
listed in the table below.

\begin{table}[h!]
\begin{center}
%\caption{Details of the Hamins CH4 tests}
\label{tab:Hamins_CH4}
\vspace{0.1in}
\begin{tabular}{|l|c|c|c|}
\hline
Test no. & Diameter  & Fuel        & HRRPUA      \\
         &   (m)     &             & (kW/m$^2$)  \\ \hline
\hline
1        & 0.1       & CH$_4$      & 53.8        \\ \hline
5        & 0.1       & CH$_4$      & 240         \\ \hline
23       & 0.38      & CH$_4$      & 295         \\ \hline
21       & 0.38      & CH$_4$      & 1550        \\ \hline
7        & 1.0       & Natural gas & 62.4        \\ \hline
19       & 1.0       & Natural gas & 206         \\ \hline
\end{tabular}
\end{center}
\end{table}



\newcommand{\paper}{chapter }
%\newcommand{\paper}{paper }

\chapter{Quantifying Model Uncertainty}

\section{Introduction}

Although there are numerous definitions of uncertainty in regard to fire modeling, the most clearly understood is
probably by way of a simple example.
Suppose a fire model is used to estimate the likelihood that electrical control cables would be damaged due to
a fire.  It is assumed that the fire is ignited within an electrical cabinet, and that
damage to exterior cables is assumed to occur when the surface temperature of any cable reaches 400~$^\circ$C.
The model predicts that the maximum surface temperature of the cables would be 350~$^\circ$C.
Does this mean that there is no chance of damage? The answer is no, because the input parameters, like the heat release rate of
the cabinet fire, and the model assumptions, like the way the fire and the cables are modeled, are uncertain. The effect of the
two combined -- the parameter uncertainty and the model uncertainty -- represent the total uncertainty in the
predicted temperature.

Uncertainty analysis has long been a part of fire science. Indeed, fire modeling has its origins in risk assessment and uncertainty analysis. The scenario above,
for example, was considered by Sui and Apostolakis in the early 1980s~\cite{Sui:RE1982} as part of their development of risk models for
nuclear power plants. The fire models at the
time were relatively simple, which made the methods for quantifying their uncertainty reasonably tractable for practicing engineers. Over the past
thirty years, however, both fire modeling and the corresponding methods of uncertainty analysis have become far more complex. While both are still
areas of active research, it can be argued that while fire modeling has become commonplace among practicing fire protection engineers, uncertainty
analysis has not. There are a number of reasons for this, including:
\begin{enumerate}
\item It is time-consuming,
\item It is difficult to understand, and
\item It is typically not demanded by authorities having jurisdiction (AHJ).
\end{enumerate}
Whether or not these are valid does not change the fact that detailed uncertainty analyses are rarely performed in practice. At best, model developers
and users publish validation studies whose conclusions are expressed in ways like this: ``The model generally over-predicts the measured temperatures
by about 10~\%,'' or ``The model predictions are within about 20~\% of the measured heat fluxes.'' This information helps to inform the decision of
the Authority Having Jurisdiction (AHJ), but only in the most qualitative sense. In particular, if the model has been shown to over-predict the quantity of interest, it is considered
``conservative,'' and its subsequent predictions are assessed in that light.

As performance based design increases the use of fire modeling, and the results of the model are incorporated within a statistical framework,
like a probabilistic risk assessment (PRA), there is a need for a more formal approach to uncertainty that is still tractable for the practicing
engineer. Using the example above, the answer to the question as to whether or not cables would be damaged in a fire could be expressed in terms of
a statistical distribution like the graph shown in Fig.~\ref{bell_curve}. The area underneath the curve is 1, and
the area indicated by the shaded region is the probability that even though the model has predicted 350~$^\circ$C, there is still a chance that
the cables could be damaged because of the uncertainty in the model and its input parameters.
\begin{figure}[t]
\begin{center}
\includegraphics[width=5.in]{FIGURES/bell_curve}
\end{center}
\caption[Demonstration of model uncertainty.]{Plot showing a possible way of expressing the uncertainty of the model prediction.}
\label{bell_curve}
\end{figure}

The uncertainty in the prediction of a fire model results from the model's input parameters and
the model's physical and mathematical assumptions. The two are commonly referred to as
the {\em parameter uncertainty} and the {\em model uncertainty}, respectively. In general, there are numerous input
parameters and numerous subroutines that are part of any fire model calculation. Uncertainty analysis seeks to
assess the impact of each on the final prediction, a considerable problem given all the variables involved.
Fire protection engineering has two advantages that make this problem more tractable. First,
a number of useful empirical correlations have
been developed over the past few decades that provide simple mathematical relationships between the most important input parameters
and output quantities that are of interest in a fire analysis. Second, there is a considerable amount of
experimental measurements against which to compare the results of numerical simulations.

This \paper suggests a methodology for quantifying the {\em model uncertainty}; that is, the accuracy of the model, not its
input parameters. The procedure is not a dramatic departure from the current
practice of fire model validation in that it relies mainly on comparisons of model predictions and experimental measurements. In
assessing the accuracy of the measurements, however, there is a need to quantify how the measured parameters in the experiments
affect the reported results. This suggests a method of handling {\em parameter uncertainty} for the model as well.



\subsection{Sources of Model Uncertainty}

A deterministic fire model is based on fundamental conservation laws of mass, momentum and energy,
applied either to entire compartments or smaller control
volumes that make up the compartments. A CFD model may use millions of control volumes to compute the
solution of the Navier-Stokes equations.
However, it does not actually solve the Navier-Stokes equations, but rather an approximate form of these equations.
The approximation involves simplifying
physical assumptions, like the various techniques for treating subgrid-scale turbulence.
One critical approximation is the discretization of the governing equations. For example,
the partial derivative of the density, $\rho$,
with respect to the spatial coordinate, $x$, can be written in approximate form as:
\be \frac{\partial \rho}{\partial x} = \frac{\rho_{i+1} - \rho_{i-1}}{2 \, \dx} + \mathcal{O}(\dx^2) \ee
where $\dx$ is the grid spacing chosen by the model user.
The second term on the right represents all of the terms of order $\dx^2$ and higher in the Taylor
series expansion and are known collectively as the
{\em discretization error}. These extra terms are simply dropped from
the equation set, the argument being that they become smaller and smaller with decreasing grid cell size, $\dx$.
The effect of these neglected terms is captured, to
some extent, by the subgrid-scale turbulence model, but that is yet another approximation of the true physics.
What effect do these approximations have on
the predicted results? It is very difficult to determine based on an analysis of the discretized equations.
One possibility for estimating
the magnitude of the discretization error is to perform a detailed
convergence analysis, but this still does not answer a
question like, ``What is the uncertainty of the model prediction of the gas
temperature at a particular location in the room at a particular point in time?''

To make matters worse, there are literally dozens of subroutines that make up a CFD fire model,
from its transport equations, radiation solver, solid phase heat transfer routines, pyrolysis model,
empirical mass, momentum and energy transfer routines at the wall, and so on.
It has been suggested by some that
a means to quantify the model uncertainty is to combine the uncertainties of all the model
components in a way similar to that of the input parameters.
However, such an exercise is very difficult, especially for a computational fluid dynamics (CFD) model,
for a number of reasons. First, fire involves
a complicated interaction of gas and solid phase phenomena that are closely coupled.
Second, grid sensitivity in a CFD model or the error associated with
a two-layer assumption in a zone model are dependent on the particular fire scenario.
Third, fire is an inherently transient phenomenon in which relatively small
changes in events, like a door opening or sprinkler actuation, can lead to significant changes in outcome.

Rather than attempt to decompose the model into its constituent parts and assess the uncertainty of
each, the strategy adopted here is to compare model predictions to as many
experiments as possible. This has been the traditional approach for quantifying model uncertainty in fire
protection engineering because of the relative abundance of test data.
However, before the model uncertainty can be quantified, the uncertainty of the
measurements against which the model predictions are compared must be quantified. This is discussed in the next section.


\subsection{Experimental Uncertainty}

In a recent fire model validation study conducted by the U.S.~Nuclear Regulatory Commission~\cite{NUREG_1824},
Hamins estimated the experimental uncertainty for several full-scale fire experiments. There were
two uncertainty estimates needed for each quantity of interest. The first was an estimate, expressed in the
form of a 95~\% confidence interval, of the
uncertainty in the measurement of the quantity itself. For example, reported gas and surface temperatures
were made with thermocouples of various designs (bare-bead,
shielded, aspirated) with different size beads, metals, and so on. For each, one can estimate the
uncertainty in the reported measurement.

Next, the
uncertainty of the measurements of the reported test parameters was estimated, including the heat release rate,
leakage area, ventilation rate, material
properties, and so on. It was then necessary to calculate how the uncertainty in these parameters contributed to
the uncertainty of the reported measurement. To do this, Hamins
examined a number of empirical formulae that are widely used in fire protection engineering to determine the
most important test parameters and their effect on the measured results. These formulae provided the means of propagating
the parameter uncertainties through the experiment.
For example, it has been shown~\cite{SFPE:Walton} that the hot gas layer temperature rise, $T-T_0$, due to
a compartment fire is proportional to the heat release rate, $\dQ$, raised to the two-thirds power:
\be T-T_0 = C \, \dQ^{\frac{2}{3}} \ee
The constant, $C$, involves a number of geometric and thermo-physical parameters that are unique to the given
fire scenario. By way of differentials, this empirical relationship can be expressed in the form:
\be \frac{\Delta T}{T-T_0} \approx \frac{2}{3} \, \frac{\Delta \dQ}{\dQ}  \ee
In words, the relative change in the temperature rise is approximately two-thirds the relative change
in the heat release rate. Assuming that the numerical model exhibits the same functional relationship between the compartment
temperature and the heat release rate, there is now a way to express the uncertainty of the model prediction as a function
of the uncertainty of this most
important input parameter. Often, the uncertainty of a measurement is expressed in the form of a 95~\% confidence interval,
(two standard deviations or 2$\sigma$).
Thus, if the heat release rate of a fire is assumed with 95~\% confidence to be
within 15~\% of the reported measurement, then the temperature
predicted by the model has an uncertainty\footnote{An uncertainty that is expressed in the form of a
percentage typically means one or two {\em relative} standard deviations, often denoted with a tilde,
$\widetilde{\sigma}=\sigma/\mu$.} of at least 10~\%.

Table~\ref{Parameter_Uncertainty} lists the most important physical parameters associated with various
measured quantities in the experiments and their power dependence.


\begin{table}[t]
\caption{Sensitivity of model outputs from Volume 2 of NUREG-1824~\cite{NUREG_1824}. }
\begin{center}
\begin{tabular}{|l|c|c|}
\hline
Output Quantity                                 & Input Parameter(s)    & Power Dependence \\ \hline \hline
HGL Temperature                                 & HRR                   & 2/3    \\ \hline
HGL Depth                                       & Door Height           & 1      \\ \hline
Ceiling Jet Temperature                         & HRR                   &        \\ \hline
Plume Temperature                               & HRR                   &        \\ \hline
Gas Concentration                               & HRR                   & 1/2    \\ \hline
                                                & HRR                   & 1      \\ \cline{2-3}
\raisebox{1.5ex}[0pt]{Smoke Concentration}      & Soot Yield            & 1      \\ \hline
                                                & HRR                   & 2      \\ \cline{2-3}
Compartment Pressure                            & Leakage Rate          & 2      \\ \cline{2-3}
                                                & Ventilation Rate      & 2      \\ \hline
Heat Flux                                       & Heat Flux             & 4/3    \\ \hline
Surface Temperature                             & HRR                   & 2/3    \\ \hline
\end{tabular}
\end{center}
\label{Parameter_Uncertainty}
\end{table}


It was assumed that the measurement and parameter uncertainty are uncorrelated, and they were combined by
quadrature (summing of squares) to yield a combined
experimental uncertainty. Another way to look at this is to recognize that the combined uncertainty is
represented as the diagonal of the rectangle formed
from the horizontal and vertical uncertainty/error bars in Fig.~\ref{scatterplot}.

Hamins performed this exercise for ten quantities of interest in the U.S. NRC validation study.
The results are summarized in Table~\ref{Uncertainty}, with
each combined uncertainty reported in the form of a 95~\% confidence interval ({\em i.e.} $2 \, \widetilde{\sigma}_E$).
The tilde above the $\sigma$ denotes a
{\em relative uncertainty}, which is a convenient way to report it because it is assumed that the uncertainty in the
reported value is proportional to
its magnitude. This assumption is made throughout the analysis for both measurement uncertainty and model error.
The assumption is based on a
qualitative assessment of dozens of scatter plots similar to that shown in Fig.~\ref{scatterplot} that show
the scattered points to form an expanding ``wedge''
about the diagonal line, or some other off-diagonal line due to an assumed bias in the model predictions.
This assessment is a critical component of the
analysis described in the next section.

\begin{table}[t]
\caption{Summary of Hamins' uncertainty estimates~\cite{NUREG_1824}. }
\begin{center}
\begin{tabular}{|l|c|}
\hline
Measured Quantity               & Combined Relative       \\
                                & Uncertainty, $2 \, \widetilde{\sigma}_E$       \\ \hline \hline
HGL Temperature                 & 0.14    \\ \hline
HGL Depth                       & 0.13    \\ \hline
Ceiling Jet Temperature         & 0.16    \\ \hline
Plume Temperature               & 0.14    \\ \hline
Gas Concentrations              & 0.09    \\ \hline
Smoke Concentration             & 0.33    \\ \hline
Pressure with Ventilation       & 0.80    \\ \hline
Pressure without Ventilation    & 0.40    \\ \hline
Heat Flux                       & 0.20    \\ \hline
Surface Temperature             & 0.14    \\ \hline
\end{tabular}
\end{center}
\label{Uncertainty}
\end{table}






\section{Calculating Model Uncertainty}

This section describes a method for calculating the {\em model uncertainty}. In terms of the example given above, this
is the answer to the question, ``350~$^\circ$C plus or minus what?'' The focus is on the model, not the parameter, uncertainty;
thus, it is assumed, for the moment, that the input parameters are not subject to any uncertainty.
In that case, how good is the prediction, assuming the fire scenario is exactly defined?
The answer to this question is based solely on
the model's track record for predicting cable temperatures in controlled validation experiments.

A fire model validation study typically consists of comparing point measurements from a wide variety of fire experiments
with corresponding model predictions.
Figure~\ref{temp_history} is a typical result for a single point measurement, and given that usually
dozens of such measurements are made during each experiment,
and potentially dozens of experiments are conducted as part of a test series, hundreds of such plots can be
produced for any given quantity of interest.
\begin{figure}[t]
\begin{center}
\includegraphics[height=2.5in]{FIGURES/sample_time_history}
\end{center}
\caption[Sample time history plots.]{Example of a typical time history comparison of model prediction and experimental measurement.}
\label{temp_history}
\end{figure}
Usually, the data is condensed into a more tractable form by way of a single metric with which to
compare the two curves like the ones shown in Fig.~\ref{temp_history}. Peacock {\em et al.}~\cite{Peacock:FSJ1999}
discuss various possible metrics. A commonly used metric is simply to compare the measured and predicted peak values.
If the data is spiky, some form of time-averaging can be used. Regardless of the exact form of the metric, what results from
this exercise is a pair of numbers for each plot, $(E_i,M_i)$, that can be depicted graphically as shown in Fig.~\ref{scatterplot}.
The diagonal line in the plot indicates where a prediction and measurement agree.
But because there is uncertainty associated with each, it cannot be said that the model is perfect if its predictions
agree exactly with measurements.
There needs to be a way of quantifying the uncertainties of each before any conclusions can be drawn.
Such an exercise would result in the uncertainty
bars\footnote{The data in Fig.~\ref{scatterplot} was extracted from Ref.~\cite{NUREG_1824}.
The uncertainty bars are for demonstration only.}
shown in the figure. The
horizontal bar associated with each point represents the uncertainty in the measurement itself.
For example, a thermocouple used to measure a gas
temperature has uncertainty. The vertical bar, however, results from the combination of the model
and parameter uncertainty, like the heat release rate and material properties that are reported by the experimentalist.
The {\em experimental uncertainty} is the combination of the measurement and input parameter uncertainty.
Decoupling the model from the experimental uncertainty is the subject of the analysis below.

\begin{figure}[t]
\begin{center}
\includegraphics[height=3.in]{FIGURES/scatterplot}
\end{center}
\caption[Sample scatter plot.]{Example of a typical scatter plot of model predictions and experimental measurements.}
\label{scatterplot}
\end{figure}


Actually, the ``plus or minus'' uncertainty bounds are expressed as a statistical distribution. In other words, the
prediction is not expressed merely as 350~$^\circ$C, but rather as a random variable with a given mean, $\mu$, and standard
deviation, $\sigma$. There are a few very important assumptions to make at the start:
\begin{enumerate}
\item The distribution is assumed to be normal.
\item The degree to which the model under or over-predicts the ``truth'' is expressed as a multiplicative bias factor, $\delta$; thus,
the mean of the distribution is $M/\delta$, where $M$ is the model prediction.
\item The standard deviation of the distribution is expressed as a fixed percentage of the mean,
$\widetilde{\sigma}=\sigma/\mu$.
\end{enumerate}
Under these assumptions, the ``true'' temperature of the cable, $\theta$, is expressed in the
following way\footnote{$N(\mu,\sigma^2)$ denotes a normal (Gaussian) distribution
with mean, $\mu$, and standard deviation, $\sigma$.}:
\be
   \theta \; | \; 350 \sim N \left( \frac{350}{\delta} \; , \; \widetilde{\sigma}_M^2
   \left( \frac{350}{\delta} \right)^2 \right) \label{truthexample}
\ee
In words, given a model prediction of 350~$^\circ$C, the expected value of the true temperature is assumed to be normally distributed
with the given mean and standard deviation. The values of $\delta$ and $\widetilde{\sigma}_M$ are calculated based on the
comparison of model predictions with experimental measurements. This calculation is discussed below. First, however, the
assumptions above require justification.

The first assumption is based on the observation that the results of past validation exercises,
when plotted as shown in Fig.~\ref{scatterplot}, suggest
that the difference between predicted and measured values is roughly proportional to the magnitude of the measured value.
Furthermore, given the
complexity of the models, it would be difficult to postulate a more precise functional relationship.
The same is true of the form of the distribution. Given the complexity of the models and the experiments,
it would be difficult to justify
any particular distribution. In fact, the very reason for developing this method of quantifying model
error based on validation
experiments is because the model algorithm itself is too complicated to work with directly
as a means of estimating the error. However, if
the distributions were better characterized, the methodology presented here could be generalized appropriately.

With these assumptions in mind, a relatively simple method of analyzing the data can be developed.
A general discussion of Bayesian data analysis can be found in
Ref.~\cite{Gelman:Stats}.
Assume that the set of model predictions and the corresponding set of experimental measurements are denoted
$M_i$ and $E_i$, respectively, where $i$ ranges from 1 to $n$ and both $M_i$ and $E_i$ are positive numbers
expressing the increase in the value of a quantity above its ambient.
As mentioned above, measurements from full-scale fire experiments often lack uncertainty estimates. In cases where the uncertainty is
reported, it is usually expressed as either a standard deviation or confidence interval about the measured value. In other words, there is rarely
a reported systematic bias in the measurement because if a bias can be quantified, the reported values are adjusted accordingly.
For this reason, assume that a given experimental measurement, $E_i$, is normally
distributed about the ``true'' value, $\theta_i$, and there is no systematic bias:
\be
   E \; | \; \theta \sim N(\theta \; , \; \sigma_E^2) \label{expunc}
\ee
The notation\footnote{Note that the subscript, $i$, has been dropped merely to reduce the notational clutter.}
$E \; | \; \theta$ means that $E$ is conditional on a particular value of $\theta$.
This is the usual way of defining a likelihood function.
It is convenient to use the so-called delta method\footnote{Given the random variable $X \sim N(\mu,\sigma^2)$, the
delta method~\cite{Oehlert:1992} provides a way to estimate the distribution of a function of $X$:
$$g(X) \sim N \left( g(\mu) + g''(\mu) \, \sigma^2/2 \, , \, (g'(\mu) \, \sigma)^2\right)$$} to obtain the approximate distribution
\be
   \ln E \; | \; \theta \sim N \left( \ln \theta - \frac{\widetilde{\sigma}_E^2}{2} \, , \,\widetilde{\sigma}_E^2 \right) \label{eeq}
\ee
The purpose of applying the natural log to the random variable is so that its variance can be expressed in terms of the
relative uncertainty, $\widetilde{\sigma}_E=\sigma_E/\theta$. This is convenient because it is assumed that the relative
uncertainty is constant for each quantity of interest. The quantities and uncertainty values are listed in Table~\ref{Uncertainty}.

It cannot be assumed, as in the case of the experimental measurements, that the model predictions have no systematic bias. Instead,
it is assumed that the model predictions are normally distributed about the true values
multiplied by a bias factor, $\delta$:
\be
   M \; | \; \theta \sim N \left(\delta \, \theta \, , \, \sigma_M^2 \right) \label{mdist}
\ee
The standard deviation, $\sigma_M$, is the model-intrinsic uncertainty, {\em i.e.} model error.
This and the bias factor, $\delta$, are the parameters that are sought.
Again, the delta method renders a distribution for $\ln M$ whose parameters can be expressed in terms of a
relative standard deviation:
\be
   \ln M \; | \; \theta \sim N \left(\ln \delta +\ln \theta - \frac{\widetilde{\sigma}_M^2}{2} \; , \;
   \widetilde{\sigma}_M^2 \right) \quad ; \quad \widetilde{\sigma}_M=\frac{\sigma_M}{\delta \, \theta} \label{meq}
\ee
Combining Eq.~(\ref{eeq}) with Eq.~(\ref{meq}) yields:
\be
   \ln M  - \ln E \sim N \left( \ln \delta - \frac{\widetilde{\sigma}_M^2}{2}+\frac{\widetilde{\sigma}_E^2}{2} \; ,
   \; \widetilde{\sigma}_M^2+\widetilde{\sigma}_E^2 \right) \label{lnMeq}
\ee
What is now needed is a way to estimate the mean and standard deviation of this combined distribution. First, define:
\be
   \overline{\ln M} = \frac{1}{n} \, \sum_{i=1}^n \, \ln M_i  \quad ; \quad
   \overline{\ln E} = \frac{1}{n} \, \sum_{i=1}^n \, \ln E_i
\ee
The least squares estimate\footnote{Note that $\hat{\sigma}$ denotes an estimate of $\sigma$.} of the standard
deviation of the combined distribution is defined as:
\be
   \widehat{ \widetilde{\sigma}_M^2 } + \widetilde{\sigma}_E^2  = \frac{1}{n-1} \sum_{i=1}^n \,
   \left[ \left(\ln M_i - \ln E_i \right) - \left( \overline{\ln M} - \overline{\ln E} \right)  \right]^2 \label{stdev}
\ee
Recall that $\widetilde{\sigma}_E$ is known and the expression on the right can be evaluated using the pairs of measured and
predicted values. An estimate of $\delta$ can be found using the mean of the distribution:
\be
   \hat{\delta} = \exp \left( \overline{\ln M}-\overline{\ln E}+\frac{ \widehat{\widetilde{\sigma}_M^2}}{2}-\frac{\widetilde{\sigma}_E^2}{2} \right)
\ee
Taking the assumed normal distribution of the model prediction, $M$, in Eq.~(\ref{mdist}) and using
a Bayesian argument\footnote{The form of Bayes theorem used here states that the posterior distribution is the product of
the prior distribution and the likelihood function, normalized by their integral:
$f(\theta|M)= p(\theta) \, f(M|\theta)/\int p(\theta) \, f(M|\theta) \, d\theta$.
A constant prior is also known as a Jeffreys prior~\cite{Gelman:Stats}.}
with a non-informative prior for $\theta$, the posterior distribution can be expressed:
\be
   \delta \, \theta \; | \; M \sim N \left( M \; , \; \sigma_M^2 \right) \label{thetaeq}
\ee
The assumption of a non-informative prior implies that there is not sufficient information about the
prior distribution ({\em i.e.} the true value) of
$\theta$ to assume anything other than a uniform\footnote{A uniform distribution means that for any two equally sized intervals of the real line,
there is an equal likelihood that the random variable takes a value in one of them.} distribution.
This is equivalent to saying that the modeler has not biased the model input parameters to compensate for a known
bias in the model output. For example, if a particular model has been shown to over-predict compartment temperature, and the modeler has reduced the specified heat release
rate to better estimate the true temperature, then it can no longer be assumed that the prior distribution of the true temperature is uniform.
Still another way to look at this is by analogy to target shooting. Suppose a particular rifle
has a manufacturers defect such that, on average, it shoots 10~cm to the left of the target. It must be assumed that any given shot by a marksman without this knowledge is
going to strike 10~cm to the left of the intended target. However, if the marksman knows of the defect, he or she will probably aim 10~cm to the right of the
intended target to compensate for the defect. If that is the case, it can no longer be assumed that the intended target was 10~cm to the right of the bullet hole.

In summary, assuming that the modeler has not modified the input parameters to compensate for a known bias in the model,
the true value of the model output quantity, $\theta$, given the model prediction, $M$, is assumed to be normally distributed with the following mean and standard deviation:
\be
   \theta \; | \; M \sim N \left( \frac{M}{\hat{\delta}} \; , \; \widehat{\widetilde{\sigma}_M^2} \left( \frac{M}{\hat{\delta}} \right)^2 \right) \label{truth}
\ee
This formula has been obtained by dividing by the bias factor, $\delta$, in Eq.~(\ref{thetaeq}).
Below, there is an example of how one might make use of this formula in practice.
First, however, the accuracy of the procedure just described needs to be verified.

\section{Verifying the Procedure}

The statistical analysis described in the previous section is difficult to understand without a fairly good background in Bayesian analysis. However,
the calculation itself is no more difficult than determining means and standard deviations of a few columns of numbers and it can be easily done with
a simple spreadsheet program.

To better illustrate the process, and also to verify this procedure, start with 1000 uniformly distributed
random numbers, $\theta_i$, between 0 and 1000. These numbers represent a particular quantity of interest, like a gas temperature at a particular
point and at a particular time, and it is assumed that these values
have no uncertainty -- they represent the ``truth.'' Of course, the true values can never be known, but for this hypothetical exercise they are assumed.
Next, with $\theta_i$ as the mean, a normally-distributed random variable is chosen that represents a hypothetical measurement, $E_i$.
The relative standard deviation of the distribution, $\widetilde{\sigma}_E$, is assumed known. In the same way, a normally-distributed
random variable is chosen that represents a hypothetical
model prediction, $M_i$. The mean of distribution $\delta \, \theta_i$ and the relative standard deviation, $\widetilde{\sigma}_M$,
is specified.
This procedure creates 1000 pairs of $(E_i,M_i)$ with which the procedure outlined in the previous section can be tested.
Using the 1000 pairs of values and the experimental uncertainty, $\widetilde{\sigma}_E$, the specified model bias, $\hat{\delta}$,
and relative error, $\widetilde{\sigma}_M$, of the hypothetical model predictions should be accurately estimated.

Two examples are considered. For the first example, assume that the model has no bias ($\delta=1$) and that the relative uncertainty of the measurements and the
relative error of the model are both 0.1, or 10~\%. The scatter plot on the left side of Fig.~\ref{Case_1_Scatter} displays the model predicted values compared to the
true values. The dashed lines indicate the 95~\% confidence interval; that is, it is expected that 95~\% of the points should fall between these
lines, whose slopes are plus and minus 20~\% ($2\, \widetilde{\sigma}_M$) of the diagonal line. Of course, this plot cannot exist in a real situation, because the true
values are never known. Instead, the only way to present the data is via the scatter plot on the right side of Fig.~\ref{Case_1_Scatter}. Here, the measured values, $E_i$, are
compared with the predicted values, $M_i$. The same dashed lines are carried over from the plot on the left. Because the predicted values are being compared
with measurements that have uncertainty, it appears that the model error is greater than it actually is because the data points are now scattered noticeably beyond the
original uncertainty bounds. In this hypothetical example, the model is
assumed to be as accurate as the measurements, yet the comparison makes it seem as if the model is less accurate than the experiments. The procedure outlined
above, which makes use of only the measured and predicted values, is able to extract from the data the fact that the hypothetical model has a
relative error of 10~\%, not the roughly 15~\% that one would infer from the plot if the experimental uncertainty were not taken into account.

\begin{figure}[t]
\begin{center}
\includegraphics[height=3.2in]{FIGURES/Case_1_Scatter_T_vs_M}
\includegraphics[height=3.2in]{FIGURES/Case_1_Scatter_E_vs_M}
\end{center}
\caption[Verification of the model error calculation, Case 1.]{(Left) A comparison of predicted and ``true'' values for 1000 hypothetical
experiments in which the model predictions and experimental measurements have the same accuracy.
(Right) The same data, except now the predicted values are compared to measured values. On both plots, the uncertainty bounds apply to both
the predicted and measured values.}
\label{Case_1_Scatter}
\end{figure}


For the second example, the hypothetical model predictions and the experimental measurements are more realistic. Assume now that the
measured quantity is surface temperature, and that the relative uncertainty of the measured values, $\widetilde{\sigma}_E=0.07$, is obtained from Table~\ref{Uncertainty}.
We assume the model has a bias factor of 1.03 and its relative error is 0.10 in order to generate a set of hypothetical model predictions.
A graphical representation of the data is shown in Fig.~\ref{Case_2_Scatter}. Note that the
experimental uncertainty bounds are represented by solid lines, and the model bias and relative error are represented by the dashed lines.
Each are expressed as 95~\% confidence intervals ($2 \, \widetilde{\sigma}$).

\begin{figure}[t]
\begin{center}
\includegraphics[height=3.2in]{FIGURES/Case_2_Scatter_T_vs_M}
\includegraphics[height=3.2in]{FIGURES/Case_2_Scatter_E_vs_M}
\end{center}
\caption[Verification of the model error calculation, Case 2.]{(Left) A comparison of predicted and ``true'' values for 1000 hypothetical
experiments. The short dashed lines indicate the bias, $\delta=1.03$, and the relative error, $2\widetilde{\sigma}_M=0.20$, of the model predictions.
The solid, off-diagonal lines indicate the relative uncertainty, $2\widetilde{\sigma}_E=0.14$ of the experimental measurements.
(Right) The same data, except now the predicted values are compared to measured values. On both plots, the uncertainty bounds are the same.}
\label{Case_2_Scatter}
\end{figure}

The statistical procedure outlined above ought to provide estimates of the values of the bias factor, $\delta$, and the model error, $\widetilde{\sigma}_M$, for the
hypothetical model predictions. The results of seven random trials are shown in Table~\ref{trials}. For each trial, values representing the model
predictions and experimental measurements were randomly selected based on the assumed distributions. The calculation procedure discussed above was applied to these
hypothetical values. The resulting estimates of the model bias and relative error were not exactly the same as those used to generate the data
because the calculation procedure relies on truncated Taylor series approximations.
However, given the fact that the experimental uncertainty estimate,
$\widetilde{\sigma}_E$, is often only a gross approximation in its own right, the accuracy of the procedure is more than adequate, as indicated by the average values
of the seven trials.

\begin{table}[t]
\caption{Estimated bias and relative error from random trials used to verify the analysis. }
\begin{center}
\begin{tabular}{|c|c|c|}
\hline
Trial   & Bias Factor      & Relative Error \\ \hline \hline
Exact   & 1.030            &    0.100            \\ \hline \hline
1       & 1.031            &    0.097            \\ \hline
2       & 1.028            &    0.101            \\ \hline
3       & 1.026            &    0.108            \\ \hline
4       & 1.034            &    0.100            \\ \hline
5       & 1.027            &    0.105            \\ \hline
6       & 1.035            &    0.100            \\ \hline
7       & 1.029            &    0.097            \\ \hline \hline
Average & 1.030            &    0.101            \\ \hline
\end{tabular}
\end{center}
\label{trials}
\end{table}





\section{Making Use of the Model Error}

The previous sections describes a method of quantifying the model error by comparing its predictions with experimental measurements. But how does one make use of the computed
model bias and relative error? This is best answered with an example. Suppose the model is being used to estimate the likelihood that
electrical control cables could be damaged due to
a fire in a compartment. Damage is assumed to occur when the surface temperature of any cable reaches 400~$^\circ$C. It is also assumed that the fire is
ignited within an electrical cabinet and the heat release rate of the fire is a specified function of time, and that all other input
parameters for the model are known and provided. Finally, it is assumed, for the time being, that there is no uncertainty
associated with any of these assumptions. What is the likelihood that the cables would be damaged if that fire were to occur? The calculation is performed, and the
model predicts that the maximum surface temperature of the cables is 350~$^\circ$C. Does this mean that there is no chance of damage, assuming that the input parameters
and assumptions are not in question at the moment? The answer is no, because the model itself is subject to error. So what is the chance that the
cables could actually reach temperatures greater than 400~$^\circ$C?

Before answering this question, first consider past experiments for which model predictions have been compared to measured surface temperatures of objects
with similar thermal characteristics as the cables in question. How ``similar'' the experiment is to the hypothetical scenario under study can be quantified by way of
various parameters, like the thermal inertia of the object, the size of the fire, the size of the compartment, and so on. Next, the results of the validation study can be
analyzed following the procedure spelled out above, which provides an estimate of the bias factor, $\delta$, and relative error, $\tilde{\sigma}_M$, for the model
predictions of this particular quantity. For the sake of argument, assume a bias factor is 1.03; that is, on average, the model has been shown to over-predict
surface temperatures by 3~\%. Also assume that the relative error has been calculated and it is 0.10~. These are the same values that were assumed in the second example
of the previous section.
Now, consider the graph shown in Fig.~\ref{bell_curve}.
The vertical lines indicate the ``threshold'' temperature at which damage is assumed to occur (400~$^\circ$C), and the temperature predicted by the
model (350~$^\circ$C). Given an ambient temperature of 20~$^\circ$C, the predicted temperature rise, $M$, is 330~$^\circ$C.
The bell curve is taken as a normal distribution whose mean and standard deviation are obtained from Eq.~(\ref{truth}) and calculated here:
\be \mu = 20 + \frac{M}{\delta} = 20 + \frac{330}{1.03} = 340.4 \; ^\circ \hbox{C}  \quad ; \quad
   \sigma = \widetilde{\sigma}_M \, \frac{M}{\hat{\delta}} = 0.10 \times \frac{330}{1.03} = 32 \; ^\circ \hbox{C}  \ee
respectively. The shaded area beneath the bell curve is the probability (0.03 in this case) that the ``true'' temperature can exceed 400~$^\circ$C.
This means that there is a 3~\% chance that the cables could
become damaged {\em based solely on the fact that the model is not a perfect representation of reality.}

The obvious question to ask at this point is what if it cannot be assumed that the specified fire, material properties, and other input parameters are
known exactly? How does that affect our estimate of the likelihood of cable damage? The procedure above has only provided a way of expressing the model
error. What if the specified fire in the electrical cabinet is actually chosen from a distribution of heat release rates?
What if the material properties of the cables are not known exactly? Assuming one could quantify the uncertainty of all of the input parameters, and assuming that the model error
and the input uncertainty are uncorrelated, it is possible to combine the two following the procedure that was described above that is used to determine the combined
experimental uncertainty. Of
course, the uncertainty associated with the input parameters would need to be quantified and propagated through the model. The end result would be a widening of
the distribution shown in Fig.~\ref{bell_curve} and an increase in the likelihood of cable damage, assuming that the parameters used in the ``base case'' were
all taken as the mean values of their respective distributions. In fact, depending on the scenario, the uncertainty associated with the input parameters can far outweigh
the model error. For example, it was discussed above that the upper layer temperature in a compartment is proportional to the heat release rate to the two-thirds power.
Hamins~\cite{NUREG_1824} demonstrates that the surface temperature of an object also exhibits the same sensitivity to the HRR. Suppose that in a compartment fire
analysis, the HRR is chosen from a distribution derived from experimental measurements of the burning rate of the potential fuel sources in the room. Suppose a similar
analysis is done for the other input parameters that have a significant impact on the results. Assuming that the
input parameters are normally-distributed, and that the
simulation makes use of the mean values, the combined uncertainty in the result could
be expressed in terms of the model error, $\widetilde{\sigma}_M$, plus contributions from the uncertainty in the input parameters:
\be
   \widetilde{\sigma}^2 = \widetilde{\sigma}_M^2 + \sum_i p_i^2 \widetilde{\sigma}_i^2  \label{comb_unc}
\ee
The factors, $p_i$, represent the power dependencies of the individual input parameters. For example, a prediction of the surface temperature of an object has a power dependence of
$p=0.67$ on the HRR. In essence, Eq.~(\ref{comb_unc}) combines the sensitivity and error analyses to produce a single estimate of the total uncertainty of the
model result. To the person evaluating the analysis, like the AHJ, this is really all that matters. However, for the model developers and model users, it is useful to
decompose the total uncertainty into its constituent parts. That encourages the developers to reduce as much as possible the value of $\widetilde{\sigma}_M$ and the users to reduce
the values of $\widetilde{\sigma}_i$. It also indicates the major sources of uncertainty so that resources can be spent wisely addressing the most important ones.



\section{Limitations}

The above verification exercises are valuable in assuring that the procedure works as designed, but it also points out a few issues that need to be addressed. First, any
statistical procedure is based on the law of averages, or, in other words, more data is better than less. It is usually not possible to conduct a large number of
fire experiments
to assess the accuracy of the model in predicting each quantity of interest. Sometimes there are only a few data points with which to estimate the model error. Worse yet,
there may not be enough information about the experimental procedure to estimate the uncertainty of the reported measurements. In such cases, it may be better
to simply present the comparison of model prediction and experimental measurement as a series of plots like Fig.~\ref{temp_history}, or in the form of a scatter plot
(Fig.~\ref{scatterplot}) without any uncertainty or error bars. The value of the validation process outlined above is that it is
possible at any step to stop and accept the raw output of the study as the basis for making an assessment of the model.
The uncertainty analysis and error quantification is valid only to
the extent that sufficient experimental data with quantified uncertainty estimates is available.
It does more harm than good to attempt to quantify the model error
with insufficient means to do so.

Another concern with the above procedure is that an over-estimate of the experimental uncertainty will result in an under-estimate of the model error. Keep
in mind that the model can never be declared more accurate than the experimental measurements against which it is compared. This rule is demonstrated
mathematically by Eq.~(\ref{stdev}) where it is observed that an over-estimate of the experimental uncertainty, $\widetilde{\sigma}_E$, can result in
the square root of a negative number, an imaginary number. In the case that the computed model error is less than the estimated experimental uncertainty, the
latter must be re-evaluated, or the number of data points needs to be seriously questioned.



\section{Conclusion}

A procedure has been proposed to estimate model error by way of comparisons of model predictions with experimental measurements whose uncertainty has been
quantified. For clarity of presentation, issues associated with the selection of experiments, metrics of comparison, and presentation of results, have not been
discussed in detail because these decisions are application-specific and best left to the end user or AHJ.


\chapter{Hot Gas Layer Temperature and Depth}

FDS, like any CFD model, does not perform a direct calculation of the HGL temperature or height. These are constructs unique to two-zone models, like
CFAST and MAGIC. Nevertheless, FDS does make predictions of gas temperature at the same locations as the thermocouples in the experiments, and these
values can be reduced in the same manner as the experimental measurements to produce an ``average'' HGL temperature and height.  Regardless of the
validity of the reduction method, the FDS predictions of the HGL temperature and height ought to be representative of the accuracy of its predictions
of the individual thermocouple measurements that are used in the HGL reduction. The temperature measurements from all six test series are used to
compute an HGL temperature and height with which to compare to FDS.  The same layer reduction method is used for five of the six test series. Only
the NBS Multi-Room series uses another method.

A brief description of each test series is included below, followed by graphs comparing the predicted and measured HGL temperature and layer height.
A summary table is provided at the end of the section that displays the relative differences between predictions and measurements for all six test
series.  Note that the calculation of relative difference is based on the temperature rise above ambient, and the layer depth, that is, the distance
from the ceiling to where the hot gas layer descends.  Where the model over-predicts the HGL temperature or the depth of the HGL, the relative
difference is a positive number. This convention is used throughout this report where the model over-predicts the severity of the fire, the relative
difference is positive; where it under-predicts, the difference is negative.


\section{NIST/WTC Test Series}




\section{VTT Large Hall Test Series}

The HGL temperature and depth are calculated from the averaged gas temperatures from three vertical thermocouple arrays using the standard reduction
method. There are 10 thermocouples in each vertical array, spaced 2 m apart in the lower two-thirds of the hall, and 1 m apart near the ceiling.
Figure~\ref{VTT_Overview} presents a snapshot from one of the simulations. Note in the figure that all of the obstructions, including the slanted
roof and exhaust duct, are approximated in the model as rectangular to conform with the rectilinear grid.





\section{NIST/NRC Test Series}

The NIST/NRC series consisted of 15 liquid spray fire tests with different heat release rates, pan locations, and ventilation conditions. The basic
geometry, including the numerical grid, is shown in Figure~\ref{NIST_NRC_Overview}. Gas temperatures were measured using seven floor-to-ceiling
thermocouple arrays (or ``trees'') distributed throughout the compartment.  The average hot gas layer temperature and height are calculated using
thermocouple Trees 1, 2, 3, 5, 6 and 7. Tree 4 was not used because one of its thermocouples (4-9) malfunctioned during most of the experiments.

A few observations about the simulations:
\begin{itemize}
\item During Tests 4, 5, 10 and 16 a fan blew air into the compartment through a vent in the south wall.
The measured velocity profile of the fan is not uniform, with the bulk of the air blowing from the lower third of the duct towards the ceiling at a
roughly 45? angle.  The exact flow pattern is difficult to replicate in the model, thus, the results for Tests 4, 5, 10 and 16 should be evaluated
with this in mind. The effect of the fan on the hot gas layer is small, but it does have a some effect on target temperatures near the vent.
\item For all of the tests involving a fan, the predicted HGL height rises after the fire is extinguished,
while the measured HGL drops.  This appears to be a curious artifact of the layer reduction algorithm. It is not included in the calculation of the
relative difference.
\item In the closed door tests, the hot gas layer descends all the way to the floor.
However, the reduction method, used on both the measured and predicted temperatures, does not account for the formation of a single layer, and
therefore does not indicate that the layer drops all the way to the floor. This is neither a flaw in the measurements nor in FDS, but rather in the
layer reduction method.
\item The HGL reduction method produces spurious results in the first few minutes of each test because no clear layer has yet formed.
These early times are not included in the relative difference calculation.
\end{itemize}



\section{FM/SNL Test Series}

Tests 4, 5, and 21 from the FM/SNL test series are selected for comparison. The hot gas layer temperature and height are calculated using the
standard method. The thermocouple arrays that are referred to as Sectors 1, 2 and 3 are averaged (with an equal weighting for each) for Tests 4 and
5. For Test 21, only Sectors 1 and 3 are used, as Sector 2 falls within the smoke plume.

Note the following:
\begin{itemize}
\item The HGL heights, both the measured and predicted, are somewhat noisy due to the effect of ventilation ducts in the upper layer.
\item The ventilation was turned off after 9 min in Test 5,
the effect of which was a slight increase in both the measured and predicted HGL temperature.
\item The measured HGL temperature is noticeably greater than the prediction in Test 21.
This is possibly due to an increase in the HRR towards the end of the test.  The simulations all used fixed HRRs after the 4 min ramp up.
\end{itemize}



\section{NBS Multi-Room Test Series}

This series of experiments consists of two relatively small rooms connected by a long corridor. The fire is located in one of the rooms.  Eight
vertical arrays of thermocouples are positioned throughout the test space: one in the burn room, one near the door of the burn room, three in the
corridor, one in the exit to the outside at the far end of the corridor, one near the door of the other or ``target'' room, and one inside the target
room.  Four of the eight arrays have been selected for comparison with model prediction: the array in the burn room (BR), the array in the middle of
the corridor (18 ft from the BR), the array at the far end of the corridor (38 ft from the BR), and the array in the target room (TR).  In Tests 100A
and 100O, the target room is closed, in which case the array in the exit (EXI) doorway is used. The test director reduced the layer information
individually for the eight thermocouple arrays using an alternative method. These results are included in the original data sets. However, for the
current validation study, the selected TC trees were reduced using the conventional method common to all the experiments considered.  The results are
presented below.


\section{Summary of Hot Gas Layer Temperature and Height}

\begin{figure}[p]
\begin{center}
\begin{tabular}{l}
\includegraphics[width=4.0in]{FIGURES/ScatterPlots/HGL_Temperature} \\
\vspace{0.25in} \\
\includegraphics[width=4.0in]{FIGURES/ScatterPlots/HGL_Depth} \\
\vspace{0.25in}
\end{tabular}
\caption{Summary of HGL Results.}
\end{center}
\end{figure}


\chapter{Fire Plumes}

\section{Plume Temperature}

Heskestad (1995) provided a simple correlation for estimating the increase in centerline temperature above ambient ($T_p$ - $T_\infty$) of a fire plume as a function of ceiling height and HRR

\be
\Delta T_p = \frac{9.1 \left( \frac{T_\infty}{g c_p^2 \rho_{a}^2} \right)^{1/3} \dot Q_c^{2/3}}{(z-z_0)^{5/3}}
\label{eq:Heskestad}
\ee

\noindent where

\be
\dot Q_c = \dot Q (1 - \chi_r)
\label{eq:Heskestad_Qc}
\ee

\be
z_0 = -1.02 D + 0.083 \dot Q^{2/5}
\label{eq:Heskestad_z0}
\ee

\be
D = \sqrt{\frac{4 A}{\pi}}
\label{eq:Heskestad_D}
\ee

In Eq.~\ref{eq:Heskestad}, $\Delta T_p$ is the plume centerline temperature rise above ambient~($^\circ$C), $T_\infty$ is the ambient air temperature~($^\circ$C), $g$ is the acceleration of gravity~(m/s$^2$), $c_p$ is the specific heat of air~(kJ/kg-K), $\rho_{a}$ is the ambient air density~(kg/m$^3$), $\dot Q_c$ is the convective HRR~(kW), $z$ is the elevation above the fire source~(m), and $z_0$ is the hypothetical virtual origin of the fire~(m).

In Eq.~\ref{eq:Heskestad_Qc}, $\dot Q$ is the total HRR~(kW), and $\chi_r$ is the radiative fraction~(-). In Eq.~\ref{eq:Heskestad_z0}, $D$ is the diameter of the fire source~(m). In Eq.~\ref{eq:Heskestad_D}, $A$ is the area of the fire source~(m$^2$).

\clearpage

\section{Summary of Plume Temperature}

Summary scatter plots of the plume temperature predictions are given on the following pages.

\begin{figure}[ht]
\begin{center}
\begin{tabular}{l}
\includegraphics[width=4.0in]{FIGURES/Scatterplots/Plume_Temperature}
\end{tabular}
\end{center}
\caption[Summary of plume temperature predictions.]
{Summary of plume temperature predictions.}
\label{Plume_Summary}
\end{figure}



\chapter{Ceiling Jets and Device Activation}

FDS is a computational fluid dynamics (CFD) model and has no explicit ceiling jet model.
Rather, temperatures throughout the fire compartment are computed directly from the governing conservation equations.
Nevertheless, temperature measurements near the ceiling can be used to evaluate the model's ability to predict the flow of
hot gases across a relatively flat ceiling. Measurements for this category are available from the NIST/NRC and the FM/SNL series.

\section{WTC Test Series}

Aspirated thermocouples were positioned 3~m to the west (TTRW1) and 2~m to the east (TTRE1) of the fire pan, 18~cm below the ceiling.

\begin{figure}[p]
\begin{tabular*}{\textwidth}{l@{\extracolsep{\fill}}r}
\includegraphics[height=2.2in]{FIGURES/WTC/WTC_01_v5_Ceiling_Jet} &
\includegraphics[height=2.2in]{FIGURES/WTC/WTC_02_v5_Ceiling_Jet} \\
\includegraphics[height=2.2in]{FIGURES/WTC/WTC_03_v5_Ceiling_Jet} &
\includegraphics[height=2.2in]{FIGURES/WTC/WTC_04_v5_Ceiling_Jet} \\
\includegraphics[height=2.2in]{FIGURES/WTC/WTC_05_v5_Ceiling_Jet} &
\includegraphics[height=2.2in]{FIGURES/WTC/WTC_06_v5_Ceiling_Jet}
\end{tabular*}
\label{WTC_Jet}
\end{figure}

\section{NIST/NRC Test Series}

The thermocouple nearest the ceiling in Tree 7, located towards the back of the compartment,
has been chosen as a surrogate for the ceiling jet temperature.
Curiously, the difference between measured and predicted temperatures is noticeably greater for the open door tests.
Certainly, the open door changes the flow pattern of the exhaust gases.
However, the predicted HGL heights for the open door tests, shown in the previous section,
do not show a noticeable difference from their closed door counterparts.
The predicted HGL temperatures are only slightly less than those measured in the open door tests,
due in large part to the contribution of Tree 7 in the layer reduction calculation.


\begin{figure}[p]
\begin{tabular*}{\textwidth}{l@{\extracolsep{\fill}}r}
\includegraphics[height=2.2in]{FIGURES/NIST_NRC/NIST_NRC_01_v5_Ceiling_Jet} &
\includegraphics[height=2.2in]{FIGURES/NIST_NRC/NIST_NRC_07_v5_Ceiling_Jet} \\
\includegraphics[height=2.2in]{FIGURES/NIST_NRC/NIST_NRC_02_v5_Ceiling_Jet} &
\includegraphics[height=2.2in]{FIGURES/NIST_NRC/NIST_NRC_08_v5_Ceiling_Jet} \\
\includegraphics[height=2.2in]{FIGURES/NIST_NRC/NIST_NRC_04_v5_Ceiling_Jet} &
\includegraphics[height=2.2in]{FIGURES/NIST_NRC/NIST_NRC_10_v5_Ceiling_Jet} \\
\includegraphics[height=2.2in]{FIGURES/NIST_NRC/NIST_NRC_13_v5_Ceiling_Jet} &
\includegraphics[height=2.2in]{FIGURES/NIST_NRC/NIST_NRC_16_v5_Ceiling_Jet}
\end{tabular*}
\label{NIST_NRC_Jet_Closed}
\end{figure}

\begin{figure}[p]
\begin{tabular*}{\textwidth}{l@{\extracolsep{\fill}}r}
\includegraphics[height=2.2in]{FIGURES/NIST_NRC/NIST_NRC_17_v5_Ceiling_Jet} &
 \\
\includegraphics[height=2.2in]{FIGURES/NIST_NRC/NIST_NRC_03_v5_Ceiling_Jet} &
\includegraphics[height=2.2in]{FIGURES/NIST_NRC/NIST_NRC_09_v5_Ceiling_Jet} \\
\includegraphics[height=2.2in]{FIGURES/NIST_NRC/NIST_NRC_05_v5_Ceiling_Jet} &
\includegraphics[height=2.2in]{FIGURES/NIST_NRC/NIST_NRC_14_v5_Ceiling_Jet} \\
\includegraphics[height=2.2in]{FIGURES/NIST_NRC/NIST_NRC_15_v5_Ceiling_Jet} &
\includegraphics[height=2.2in]{FIGURES/NIST_NRC/NIST_NRC_18_v5_Ceiling_Jet}
\end{tabular*}
\label{NIST_NRC_Jet_Open}
\end{figure}


\section{FM/SNL Test Series}

The near-ceiling thermocouples in Sectors 1 and 3 have been chosen as surrogates for the ceiling jet temperature.
The results are shown below.  The only noticeable discrepancy is in Test 21, and it is the same pattern that
was observed in the HGL temperature comparison for this test.

\begin{figure}[p]
\begin{tabular*}{\textwidth}{l@{\extracolsep{\fill}}r}
\includegraphics[height=2.2in]{FIGURES/FM_SNL/FM_SNL_04_v5_Ceiling_Jet} &
\includegraphics[height=2.2in]{FIGURES/FM_SNL/FM_SNL_05_v5_Ceiling_Jet} \\
\includegraphics[height=2.2in]{FIGURES/FM_SNL/FM_SNL_21_v5_Ceiling_Jet} &
\end{tabular*}
\label{FM_SNL_Ceiling_Jet}
\end{figure}


\clearpage

\section{ATF Corridors Series}

This series of experiments involved two fairly long corridors connected by a staircase. The fire, a natural gas sand
burner, was located on the first level at the end of the corridor away from the stairwell, which was located at the
other end. The corridor was closed at the end where the fire was located, and open at the same position on the 
second level. Two-way flow occurred on both levels because make-up air flowed from the opening on the second level down
the stairs to the first. The only opening to the enclosure was the open end of the second-level corridor.

Temperatures were measured with 7 thermocouple trees. Tree A was located fairly close to the fire on the first level. Tree~B
was located halfway down the first-level corridor. Tree~C was close to the stairwell entrance on the first level. Tree~D was located
in the doorway of the stairwell on the first level. Tree~E was located roughly along the vertical centerline of the 
stairwell. Tree~F was located near the stairwell opening on the second level. Tree~G was located near the exit at the
other end of the second-level corridor. The graphs on the following pages show the top and bottom TC from each tree for
the given fire sizes of 50~kW, 100~kW, 250~kW, 500~kW, and a mixed HRR ``pulsed'' fire.

\begin{figure}[p]
\begin{tabular*}{\textwidth}{l@{\extracolsep{\fill}}r}
\includegraphics[height=2.2in]{FIGURES/ATF_Corridors/ATF_Corridors_Jet_Temp_A_050_kW} &
\includegraphics[height=2.2in]{FIGURES/ATF_Corridors/ATF_Corridors_Jet_Temp_B_050_kW} \\
\includegraphics[height=2.2in]{FIGURES/ATF_Corridors/ATF_Corridors_Jet_Temp_C_050_kW} &
\includegraphics[height=2.2in]{FIGURES/ATF_Corridors/ATF_Corridors_Jet_Temp_D_050_kW} \\
\includegraphics[height=2.2in]{FIGURES/ATF_Corridors/ATF_Corridors_Jet_Temp_E_050_kW} &
\includegraphics[height=2.2in]{FIGURES/ATF_Corridors/ATF_Corridors_Jet_Temp_F_050_kW} \\
\includegraphics[height=2.2in]{FIGURES/ATF_Corridors/ATF_Corridors_Jet_Temp_G_050_kW} &
\end{tabular*}
\label{ATF_Corridors_Jet_Temp_50_kW}
\end{figure}

\begin{figure}[p]
\begin{tabular*}{\textwidth}{l@{\extracolsep{\fill}}r}
\includegraphics[height=2.2in]{FIGURES/ATF_Corridors/ATF_Corridors_Jet_Temp_A_100_kW} &
\includegraphics[height=2.2in]{FIGURES/ATF_Corridors/ATF_Corridors_Jet_Temp_B_100_kW} \\
\includegraphics[height=2.2in]{FIGURES/ATF_Corridors/ATF_Corridors_Jet_Temp_C_100_kW} &
\includegraphics[height=2.2in]{FIGURES/ATF_Corridors/ATF_Corridors_Jet_Temp_D_100_kW} \\
\includegraphics[height=2.2in]{FIGURES/ATF_Corridors/ATF_Corridors_Jet_Temp_E_100_kW} &
\includegraphics[height=2.2in]{FIGURES/ATF_Corridors/ATF_Corridors_Jet_Temp_F_100_kW} \\
\includegraphics[height=2.2in]{FIGURES/ATF_Corridors/ATF_Corridors_Jet_Temp_G_100_kW} &
\end{tabular*}
\label{ATF_Corridors_Jet_Temp_100_kW}
\end{figure}

\begin{figure}[p]
\begin{tabular*}{\textwidth}{l@{\extracolsep{\fill}}r}
\includegraphics[height=2.2in]{FIGURES/ATF_Corridors/ATF_Corridors_Jet_Temp_A_250_kW} &
\includegraphics[height=2.2in]{FIGURES/ATF_Corridors/ATF_Corridors_Jet_Temp_B_250_kW} \\
\includegraphics[height=2.2in]{FIGURES/ATF_Corridors/ATF_Corridors_Jet_Temp_C_250_kW} &
\includegraphics[height=2.2in]{FIGURES/ATF_Corridors/ATF_Corridors_Jet_Temp_D_250_kW} \\
\includegraphics[height=2.2in]{FIGURES/ATF_Corridors/ATF_Corridors_Jet_Temp_E_250_kW} &
\includegraphics[height=2.2in]{FIGURES/ATF_Corridors/ATF_Corridors_Jet_Temp_F_250_kW} \\
\includegraphics[height=2.2in]{FIGURES/ATF_Corridors/ATF_Corridors_Jet_Temp_G_250_kW} &
\end{tabular*}
\label{ATF_Corridors_Jet_Temp_250_kW}
\end{figure}

\begin{figure}[p]
\begin{tabular*}{\textwidth}{l@{\extracolsep{\fill}}r}
\includegraphics[height=2.2in]{FIGURES/ATF_Corridors/ATF_Corridors_Jet_Temp_A_500_kW} &
\includegraphics[height=2.2in]{FIGURES/ATF_Corridors/ATF_Corridors_Jet_Temp_B_500_kW} \\
\includegraphics[height=2.2in]{FIGURES/ATF_Corridors/ATF_Corridors_Jet_Temp_C_500_kW} &
\includegraphics[height=2.2in]{FIGURES/ATF_Corridors/ATF_Corridors_Jet_Temp_D_500_kW} \\
\includegraphics[height=2.2in]{FIGURES/ATF_Corridors/ATF_Corridors_Jet_Temp_E_500_kW} &
\includegraphics[height=2.2in]{FIGURES/ATF_Corridors/ATF_Corridors_Jet_Temp_F_500_kW} \\
\includegraphics[height=2.2in]{FIGURES/ATF_Corridors/ATF_Corridors_Jet_Temp_G_500_kW} &
\end{tabular*}
\label{ATF_Corridors_Jet_Temp_500_kW}
\end{figure}

\begin{figure}[p]
\begin{tabular*}{\textwidth}{l@{\extracolsep{\fill}}r}
\includegraphics[height=2.2in]{FIGURES/ATF_Corridors/ATF_Corridors_Jet_Temp_A_Mix_kW} &
\includegraphics[height=2.2in]{FIGURES/ATF_Corridors/ATF_Corridors_Jet_Temp_B_Mix_kW} \\
\includegraphics[height=2.2in]{FIGURES/ATF_Corridors/ATF_Corridors_Jet_Temp_C_Mix_kW} &
\includegraphics[height=2.2in]{FIGURES/ATF_Corridors/ATF_Corridors_Jet_Temp_D_Mix_kW} \\
\includegraphics[height=2.2in]{FIGURES/ATF_Corridors/ATF_Corridors_Jet_Temp_E_Mix_kW} &
\includegraphics[height=2.2in]{FIGURES/ATF_Corridors/ATF_Corridors_Jet_Temp_F_Mix_kW} \\
\includegraphics[height=2.2in]{FIGURES/ATF_Corridors/ATF_Corridors_Jet_Temp_G_Mix_kW} &
\end{tabular*}
\label{ATF_Corridors_Jet_Temp_Mix_kW}
\end{figure}





\begin{figure}[p]
\begin{center}
\begin{tabular}{c}
\includegraphics[width=5.0in]{FIGURES/ScatterPlots/Ceiling_Jet_Temperature} \\
\vspace{0.25in}
\end{tabular}
\end{center}
\caption[Summary of ceiling jet temperature predictions, WTC, NIST/NRC and FM/SNL test series.]
{Summary of ceiling jet temperature predictions, WTC, NIST/NRC and FM/SNL test series.}
\end{figure}



\clearpage

\section{UL/NFPRF Sprinkler, Vent, and Draft Curtain Experiments}
\label{UL_NFPRF:Results}

The ceiling jet is an important fire phenomenon because of the presence of automatic fire protection devices at the ceiling, like
sprinklers and smoke/heat vents. The results of the UL/NFPRF experiments provide useful data to assess the accuracy of FDS in predicting
the velocity and temperature near the ceiling, and consequently the resulting activation of sprinklers.
The UL/NFPRF test results (Series I) are summarized in Table~\ref{ULmatrix}.

The figures on the following pages display the number of sprinklers actuated as a function of time. 
The results are then summarized in Fig.~\ref{UL_NFPRF}. Note that there are no experimental uncertainty bounds on the plot because it is difficult to estimate the
combined uncertainty related to the various parameters that are input into the model. At the bottom of Fig.~\ref{UL_NFPRF}, the results of three replicate experiments
demonstrate that the total number of actuated sprinklers in each experiment is repeatable, even though individual actuation times may vary. Based on these
three replicates, there is very little, if any, uncertainty in the total number of actuated sprinklers for each test. However, the test report~\cite{Sheppard:1} does not
include uncertainty estimates for the heat release rate, thermal properties of the ceiling, sprinkler RTI, conductivity factor, actuation temperature,
median droplet diameter, and various other parameters that have been input into the model. Consequently, it is not possible to estimate the uncertainty in the
total number of actuated sprinklers due to the uncertainty in the reported parameters. The only sensitivity analysis conducted for this set of experiments was
to change the median volumetric droplet size from 1000~$\mu$m to 750~$\mu$m, which led to a reduction of approximately 50~\% in the number of predicted sprinkler actuations.

\begin{figure}[p]
\begin{tabular*}{\textwidth}{l@{\extracolsep{\fill}}r}
\includegraphics[height=2.2in]{FIGURES/UL_NFPRF/UL_NFPRF_1_01_Actuations} &
\includegraphics[height=2.2in]{FIGURES/UL_NFPRF/UL_NFPRF_1_02_Actuations} \\
\includegraphics[height=2.2in]{FIGURES/UL_NFPRF/UL_NFPRF_1_03_Actuations} &
\includegraphics[height=2.2in]{FIGURES/UL_NFPRF/UL_NFPRF_1_04_Actuations} \\
\includegraphics[height=2.2in]{FIGURES/UL_NFPRF/UL_NFPRF_1_05_Actuations} &
\includegraphics[height=2.2in]{FIGURES/UL_NFPRF/UL_NFPRF_1_06_Actuations} \\
\includegraphics[height=2.2in]{FIGURES/UL_NFPRF/UL_NFPRF_1_07_Actuations} &
\includegraphics[height=2.2in]{FIGURES/UL_NFPRF/UL_NFPRF_1_08_Actuations} \\
\end{tabular*}
\label{UL_NFPRF_1}
\end{figure}

\begin{figure}[p]
\begin{tabular*}{\textwidth}{l@{\extracolsep{\fill}}r}
\includegraphics[height=2.2in]{FIGURES/UL_NFPRF/UL_NFPRF_1_09_Actuations} &
\includegraphics[height=2.2in]{FIGURES/UL_NFPRF/UL_NFPRF_1_10_Actuations} \\
\includegraphics[height=2.2in]{FIGURES/UL_NFPRF/UL_NFPRF_1_11_Actuations} &
\includegraphics[height=2.2in]{FIGURES/UL_NFPRF/UL_NFPRF_1_12_Actuations} \\
\includegraphics[height=2.2in]{FIGURES/UL_NFPRF/UL_NFPRF_1_13_Actuations} &
\includegraphics[height=2.2in]{FIGURES/UL_NFPRF/UL_NFPRF_1_14_Actuations} \\
\includegraphics[height=2.2in]{FIGURES/UL_NFPRF/UL_NFPRF_1_15_Actuations} &
\includegraphics[height=2.2in]{FIGURES/UL_NFPRF/UL_NFPRF_1_16_Actuations} \\
\end{tabular*}
\label{UL_NFPRF_2}
\end{figure}

\begin{figure}[p]
\begin{tabular*}{\textwidth}{l@{\extracolsep{\fill}}r}
\includegraphics[height=2.2in]{FIGURES/UL_NFPRF/UL_NFPRF_1_17_Actuations} &
\includegraphics[height=2.2in]{FIGURES/UL_NFPRF/UL_NFPRF_1_18_Actuations} \\
\includegraphics[height=2.2in]{FIGURES/UL_NFPRF/UL_NFPRF_1_19_Actuations} &
\includegraphics[height=2.2in]{FIGURES/UL_NFPRF/UL_NFPRF_1_20_Actuations} \\
\includegraphics[height=2.2in]{FIGURES/UL_NFPRF/UL_NFPRF_1_21_Actuations} &
\includegraphics[height=2.2in]{FIGURES/UL_NFPRF/UL_NFPRF_1_22_Actuations} 
\end{tabular*}
\label{UL_NFPRF_3}
\end{figure}

\begin{figure}[p]
\begin{center}
\includegraphics[width=3.5in]{FIGURES/ScatterPlots/UL_NFPRF_Actuations}
\end{center}
\begin{tabular*}{\textwidth}{l@{\extracolsep{\fill}}r}
\includegraphics[height=2.2in]{FIGURES/UL_NFPRF/UL_NFPRF_1_01_08_Actuations} &
\includegraphics[height=2.2in]{FIGURES/UL_NFPRF/UL_NFPRF_1_04_07_Actuations}
\end{tabular*}
\begin{center}
\includegraphics[height=2.2in]{FIGURES/UL_NFPRF/UL_NFPRF_1_09_10_Actuations} 
\end{center}`
\caption[Summary of sprinkler actuation predictions, UL/NFPRF test series.]
{Above: Comparison of predicted and measured sprinkler activation times for the UL/NFPRF Test Series I. Below:
The results of three replicate experiments.}
\label{UL_NFPRF}
\end{figure}


\chapter{Gas Velocity}

Gas velocity is often measured at compartment inlets and outlets as part of a global assessment of mass and
energy conservation.  This chapter contains measurements of gas velocity and related quantities.

\section{Steckler Compartment Experiments}

\begin{figure}[p]
\begin{tabular*}{\textwidth}{l@{\extracolsep{\fill}}r}
\includegraphics[height=2.2in]{FIGURES/Steckler_Compartment/Steckler_010_Temp} &
\includegraphics[height=2.2in]{FIGURES/Steckler_Compartment/Steckler_011_Temp} \\
\includegraphics[height=2.2in]{FIGURES/Steckler_Compartment/Steckler_012_Temp} &
\includegraphics[height=2.2in]{FIGURES/Steckler_Compartment/Steckler_612_Temp} \\
\includegraphics[height=2.2in]{FIGURES/Steckler_Compartment/Steckler_013_Temp} &
\includegraphics[height=2.2in]{FIGURES/Steckler_Compartment/Steckler_014_Temp} \\
\includegraphics[height=2.2in]{FIGURES/Steckler_Compartment/Steckler_018_Temp} &
\includegraphics[height=2.2in]{FIGURES/Steckler_Compartment/Steckler_710_Temp}
\end{tabular*}
\label{Steckler_Temp_1}
\end{figure}

\begin{figure}[p]
\begin{tabular*}{\textwidth}{l@{\extracolsep{\fill}}r}
\includegraphics[height=2.2in]{FIGURES/Steckler_Compartment/Steckler_810_Temp} &
\includegraphics[height=2.2in]{FIGURES/Steckler_Compartment/Steckler_016_Temp} \\
\includegraphics[height=2.2in]{FIGURES/Steckler_Compartment/Steckler_017_Temp} &
\includegraphics[height=2.2in]{FIGURES/Steckler_Compartment/Steckler_022_Temp} \\
\includegraphics[height=2.2in]{FIGURES/Steckler_Compartment/Steckler_023_Temp} &
\includegraphics[height=2.2in]{FIGURES/Steckler_Compartment/Steckler_030_Temp} \\
\includegraphics[height=2.2in]{FIGURES/Steckler_Compartment/Steckler_041_Temp} &
\includegraphics[height=2.2in]{FIGURES/Steckler_Compartment/Steckler_019_Temp}
\end{tabular*}
\label{Steckler_Temp_2}
\end{figure}

\begin{figure}[p]
\begin{tabular*}{\textwidth}{l@{\extracolsep{\fill}}r}
\includegraphics[height=2.2in]{FIGURES/Steckler_Compartment/Steckler_020_Temp} &
\includegraphics[height=2.2in]{FIGURES/Steckler_Compartment/Steckler_021_Temp} \\
\includegraphics[height=2.2in]{FIGURES/Steckler_Compartment/Steckler_114_Temp} &
\includegraphics[height=2.2in]{FIGURES/Steckler_Compartment/Steckler_144_Temp} \\
\includegraphics[height=2.2in]{FIGURES/Steckler_Compartment/Steckler_212_Temp} &
\includegraphics[height=2.2in]{FIGURES/Steckler_Compartment/Steckler_242_Temp} \\
\includegraphics[height=2.2in]{FIGURES/Steckler_Compartment/Steckler_410_Temp} &
\includegraphics[height=2.2in]{FIGURES/Steckler_Compartment/Steckler_210_Temp}
\end{tabular*}
\label{Steckler_Temp_3}
\end{figure}

\begin{figure}[p]
\begin{tabular*}{\textwidth}{l@{\extracolsep{\fill}}r}
\includegraphics[height=2.2in]{FIGURES/Steckler_Compartment/Steckler_310_Temp} &
\includegraphics[height=2.2in]{FIGURES/Steckler_Compartment/Steckler_240_Temp} \\
\includegraphics[height=2.2in]{FIGURES/Steckler_Compartment/Steckler_116_Temp} &
\includegraphics[height=2.2in]{FIGURES/Steckler_Compartment/Steckler_122_Temp} \\
\includegraphics[height=2.2in]{FIGURES/Steckler_Compartment/Steckler_224_Temp} &
\includegraphics[height=2.2in]{FIGURES/Steckler_Compartment/Steckler_324_Temp} \\
\includegraphics[height=2.2in]{FIGURES/Steckler_Compartment/Steckler_220_Temp} &
\includegraphics[height=2.2in]{FIGURES/Steckler_Compartment/Steckler_221_Temp}
\end{tabular*}
\label{Steckler_Temp_4}
\end{figure}

\begin{figure}[p]
\begin{tabular*}{\textwidth}{l@{\extracolsep{\fill}}r}
\includegraphics[height=2.2in]{FIGURES/Steckler_Compartment/Steckler_514_Temp} &
\includegraphics[height=2.2in]{FIGURES/Steckler_Compartment/Steckler_544_Temp} \\
\includegraphics[height=2.2in]{FIGURES/Steckler_Compartment/Steckler_512_Temp} &
\includegraphics[height=2.2in]{FIGURES/Steckler_Compartment/Steckler_542_Temp} \\
\includegraphics[height=2.2in]{FIGURES/Steckler_Compartment/Steckler_610_Temp} &
\includegraphics[height=2.2in]{FIGURES/Steckler_Compartment/Steckler_510_Temp} \\
\includegraphics[height=2.2in]{FIGURES/Steckler_Compartment/Steckler_540_Temp} &
\includegraphics[height=2.2in]{FIGURES/Steckler_Compartment/Steckler_517_Temp}
\end{tabular*}
\label{Steckler_Temp_5}
\end{figure}

\begin{figure}[p]
\begin{tabular*}{\textwidth}{l@{\extracolsep{\fill}}r}
\includegraphics[height=2.2in]{FIGURES/Steckler_Compartment/Steckler_622_Temp} &
\includegraphics[height=2.2in]{FIGURES/Steckler_Compartment/Steckler_522_Temp} \\
\includegraphics[height=2.2in]{FIGURES/Steckler_Compartment/Steckler_524_Temp} &
\includegraphics[height=2.2in]{FIGURES/Steckler_Compartment/Steckler_541_Temp} \\
\includegraphics[height=2.2in]{FIGURES/Steckler_Compartment/Steckler_520_Temp} &
\includegraphics[height=2.2in]{FIGURES/Steckler_Compartment/Steckler_521_Temp} \\
\includegraphics[height=2.2in]{FIGURES/Steckler_Compartment/Steckler_513_Temp} &
\includegraphics[height=2.2in]{FIGURES/Steckler_Compartment/Steckler_160_Temp}
\end{tabular*}
\label{Steckler_Temp_6}
\end{figure}

\begin{figure}[p]
\begin{tabular*}{\textwidth}{l@{\extracolsep{\fill}}r}
\includegraphics[height=2.2in]{FIGURES/Steckler_Compartment/Steckler_163_Temp} &
\includegraphics[height=2.2in]{FIGURES/Steckler_Compartment/Steckler_164_Temp} \\
\includegraphics[height=2.2in]{FIGURES/Steckler_Compartment/Steckler_165_Temp} &
\includegraphics[height=2.2in]{FIGURES/Steckler_Compartment/Steckler_162_Temp} \\
\includegraphics[height=2.2in]{FIGURES/Steckler_Compartment/Steckler_167_Temp} &
\includegraphics[height=2.2in]{FIGURES/Steckler_Compartment/Steckler_161_Temp} \\
\includegraphics[height=2.2in]{FIGURES/Steckler_Compartment/Steckler_166_Temp} &

\end{tabular*}
\label{Steckler_Temp_7}
\end{figure}

\begin{figure}[p]
\begin{center}
\begin{tabular}{l}
\includegraphics[width=4.0in]{FIGURES/ScatterPlots/HGL_Temperature_Steckler} 
\end{tabular}
\caption{Summary of Velocity Results.}
\end{center}
\end{figure}



\begin{figure}[p]
\begin{tabular*}{\textwidth}{l@{\extracolsep{\fill}}r}
\includegraphics[height=2.2in]{FIGURES/Steckler_Compartment/Steckler_010_Vel} &
\includegraphics[height=2.2in]{FIGURES/Steckler_Compartment/Steckler_011_Vel} \\
\includegraphics[height=2.2in]{FIGURES/Steckler_Compartment/Steckler_012_Vel} &
\includegraphics[height=2.2in]{FIGURES/Steckler_Compartment/Steckler_612_Vel} \\
\includegraphics[height=2.2in]{FIGURES/Steckler_Compartment/Steckler_013_Vel} &
\includegraphics[height=2.2in]{FIGURES/Steckler_Compartment/Steckler_014_Vel} \\
\includegraphics[height=2.2in]{FIGURES/Steckler_Compartment/Steckler_018_Vel} &
\includegraphics[height=2.2in]{FIGURES/Steckler_Compartment/Steckler_710_Vel}
\end{tabular*}
\label{Steckler_Vel_1}
\end{figure}

\begin{figure}[p]
\begin{tabular*}{\textwidth}{l@{\extracolsep{\fill}}r}
\includegraphics[height=2.2in]{FIGURES/Steckler_Compartment/Steckler_810_Vel} &
\includegraphics[height=2.2in]{FIGURES/Steckler_Compartment/Steckler_016_Vel} \\
\includegraphics[height=2.2in]{FIGURES/Steckler_Compartment/Steckler_017_Vel} &
\includegraphics[height=2.2in]{FIGURES/Steckler_Compartment/Steckler_022_Vel} \\
\includegraphics[height=2.2in]{FIGURES/Steckler_Compartment/Steckler_023_Vel} &
\includegraphics[height=2.2in]{FIGURES/Steckler_Compartment/Steckler_030_Vel} \\
\includegraphics[height=2.2in]{FIGURES/Steckler_Compartment/Steckler_041_Vel} &
\includegraphics[height=2.2in]{FIGURES/Steckler_Compartment/Steckler_019_Vel}
\end{tabular*}
\label{Steckler_Vel_2}
\end{figure}

\begin{figure}[p]
\begin{tabular*}{\textwidth}{l@{\extracolsep{\fill}}r}
\includegraphics[height=2.2in]{FIGURES/Steckler_Compartment/Steckler_020_Vel} &
\includegraphics[height=2.2in]{FIGURES/Steckler_Compartment/Steckler_021_Vel} \\
\includegraphics[height=2.2in]{FIGURES/Steckler_Compartment/Steckler_114_Vel} &
\includegraphics[height=2.2in]{FIGURES/Steckler_Compartment/Steckler_144_Vel} \\
\includegraphics[height=2.2in]{FIGURES/Steckler_Compartment/Steckler_212_Vel} &
\includegraphics[height=2.2in]{FIGURES/Steckler_Compartment/Steckler_242_Vel} \\
\includegraphics[height=2.2in]{FIGURES/Steckler_Compartment/Steckler_410_Vel} &
\includegraphics[height=2.2in]{FIGURES/Steckler_Compartment/Steckler_210_Vel}
\end{tabular*}
\label{Steckler_Vel_3}
\end{figure}

\begin{figure}[p]
\begin{tabular*}{\textwidth}{l@{\extracolsep{\fill}}r}
\includegraphics[height=2.2in]{FIGURES/Steckler_Compartment/Steckler_310_Vel} &
\includegraphics[height=2.2in]{FIGURES/Steckler_Compartment/Steckler_240_Vel} \\
\includegraphics[height=2.2in]{FIGURES/Steckler_Compartment/Steckler_116_Vel} &
\includegraphics[height=2.2in]{FIGURES/Steckler_Compartment/Steckler_122_Vel} \\
\includegraphics[height=2.2in]{FIGURES/Steckler_Compartment/Steckler_224_Vel} &
\includegraphics[height=2.2in]{FIGURES/Steckler_Compartment/Steckler_324_Vel} \\
\includegraphics[height=2.2in]{FIGURES/Steckler_Compartment/Steckler_220_Vel} &
\includegraphics[height=2.2in]{FIGURES/Steckler_Compartment/Steckler_221_Vel}
\end{tabular*}
\label{Steckler_Vel_4}
\end{figure}

\begin{figure}[p]
\begin{tabular*}{\textwidth}{l@{\extracolsep{\fill}}r}
\includegraphics[height=2.2in]{FIGURES/Steckler_Compartment/Steckler_514_Vel} &
\includegraphics[height=2.2in]{FIGURES/Steckler_Compartment/Steckler_544_Vel} \\
\includegraphics[height=2.2in]{FIGURES/Steckler_Compartment/Steckler_512_Vel} &
\includegraphics[height=2.2in]{FIGURES/Steckler_Compartment/Steckler_542_Vel} \\
\includegraphics[height=2.2in]{FIGURES/Steckler_Compartment/Steckler_610_Vel} &
\includegraphics[height=2.2in]{FIGURES/Steckler_Compartment/Steckler_510_Vel} \\
\includegraphics[height=2.2in]{FIGURES/Steckler_Compartment/Steckler_540_Vel} &
\includegraphics[height=2.2in]{FIGURES/Steckler_Compartment/Steckler_517_Vel}
\end{tabular*}
\label{Steckler_Vel_5}
\end{figure}

\begin{figure}[p]
\begin{tabular*}{\textwidth}{l@{\extracolsep{\fill}}r}
\includegraphics[height=2.2in]{FIGURES/Steckler_Compartment/Steckler_622_Vel} &
\includegraphics[height=2.2in]{FIGURES/Steckler_Compartment/Steckler_522_Vel} \\
\includegraphics[height=2.2in]{FIGURES/Steckler_Compartment/Steckler_524_Vel} &
\includegraphics[height=2.2in]{FIGURES/Steckler_Compartment/Steckler_541_Vel} \\
\includegraphics[height=2.2in]{FIGURES/Steckler_Compartment/Steckler_520_Vel} &
\includegraphics[height=2.2in]{FIGURES/Steckler_Compartment/Steckler_521_Vel} \\
\includegraphics[height=2.2in]{FIGURES/Steckler_Compartment/Steckler_513_Vel} &
\includegraphics[height=2.2in]{FIGURES/Steckler_Compartment/Steckler_160_Vel}
\end{tabular*}
\label{Steckler_Vel_6}
\end{figure}

\begin{figure}[p]
\begin{tabular*}{\textwidth}{l@{\extracolsep{\fill}}r}
\includegraphics[height=2.2in]{FIGURES/Steckler_Compartment/Steckler_163_Vel} &
\includegraphics[height=2.2in]{FIGURES/Steckler_Compartment/Steckler_164_Vel} \\
\includegraphics[height=2.2in]{FIGURES/Steckler_Compartment/Steckler_165_Vel} &
\includegraphics[height=2.2in]{FIGURES/Steckler_Compartment/Steckler_162_Vel} \\
\includegraphics[height=2.2in]{FIGURES/Steckler_Compartment/Steckler_167_Vel} &
\includegraphics[height=2.2in]{FIGURES/Steckler_Compartment/Steckler_161_Vel} \\
\includegraphics[height=2.2in]{FIGURES/Steckler_Compartment/Steckler_166_Vel} &

\end{tabular*}
\label{Steckler_Vel_7}
\end{figure}

\begin{figure}[p]
\begin{center}
\begin{tabular}{l}
\includegraphics[width=4.0in]{FIGURES/ScatterPlots/Velocity_Steckler}
\end{tabular}
\caption{Summary of Velocity Results.}
\end{center}
\end{figure}


\clearpage


\section{NIST/WTC Test Series}

\begin{figure}[p]
\begin{tabular*}{\textwidth}{l@{\extracolsep{\fill}}r}
\includegraphics[height=2.2in]{FIGURES/WTC/WTC_01_v5_Inlet_Velocity} &
\includegraphics[height=2.2in]{FIGURES/WTC/WTC_01_v5_Outlet_Velocity} \\
\includegraphics[height=2.2in]{FIGURES/WTC/WTC_02_v5_Inlet_Velocity} &
\includegraphics[height=2.2in]{FIGURES/WTC/WTC_02_v5_Outlet_Velocity} \\
\includegraphics[height=2.2in]{FIGURES/WTC/WTC_03_v5_Inlet_Velocity} &
\includegraphics[height=2.2in]{FIGURES/WTC/WTC_03_v5_Outlet_Velocity}
\end{tabular*}
\label{NIST_WTC_Velocity_1}
\end{figure}


\begin{figure}[p]
\begin{tabular*}{\textwidth}{l@{\extracolsep{\fill}}r}
\includegraphics[height=2.2in]{FIGURES/WTC/WTC_04_v5_Inlet_Velocity} &
\includegraphics[height=2.2in]{FIGURES/WTC/WTC_04_v5_Outlet_Velocity} \\
\includegraphics[height=2.2in]{FIGURES/WTC/WTC_05_v5_Inlet_Velocity} &
\includegraphics[height=2.2in]{FIGURES/WTC/WTC_05_v5_Outlet_Velocity} \\
\includegraphics[height=2.2in]{FIGURES/WTC/WTC_06_v5_Inlet_Velocity} &
\includegraphics[height=2.2in]{FIGURES/WTC/WTC_06_v5_Outlet_Velocity}
\end{tabular*}
\label{NIST_WTC_Velocity_2}
\end{figure}



\begin{figure}[ht]
\begin{tabular*}{\textwidth}{l@{\extracolsep{\fill}}r}
\includegraphics[width=3.0in]{FIGURES/ScatterPlots/Velocity} &

\end{tabular*}
\caption{Summary of Velocity Results.}
\end{figure}


% !TEX root = FDS_Validation_Guide.tex

\chapter{Gas Species and Smoke}

For most applications, FDS uses a single step, mixing-controlled combustion model. The products of combustion are ``lumped'' together and tracked as a single gas mixture. These products include CO$_2$, H$_2$O, CO, and soot. However, in some cases, the combustion is incomplete due to a lack of oxygen. In others, a multiple-step reaction scheme is used to predict the production of CO.

\section{Major Combustion Products, O$_2$ and CO$_2$}

For any hydrocarbon fuel, the major combustion products are oxygen and carbon dioxide. Accurate predictions of these gases requires knowledge of the chemical composition of the fuel and an accurate transport algorithm for the combustion products.

\clearpage

\subsection{DelCo Trainers}

Oxygen and carbon dioxide measurements were made at several locations in the one and two level DelCo training structures. See Sec.~\ref{DelCo_Description} for their exact locations.

\begin{figure}[!h]
\begin{tabular*}{\textwidth}{l@{\extracolsep{\fill}}r}
\includegraphics[height=2.15in]{SCRIPT_FIGURES/DelCo_Trainers/Test_02_CO2} &
\includegraphics[height=2.15in]{SCRIPT_FIGURES/DelCo_Trainers/Test_02_O2} \\
\includegraphics[height=2.15in]{SCRIPT_FIGURES/DelCo_Trainers/Test_03_CO2} &
\includegraphics[height=2.15in]{SCRIPT_FIGURES/DelCo_Trainers/Test_03_O2} \\
\includegraphics[height=2.15in]{SCRIPT_FIGURES/DelCo_Trainers/Test_04_CO2} &
\includegraphics[height=2.15in]{SCRIPT_FIGURES/DelCo_Trainers/Test_04_O2}
\end{tabular*}
\caption[DelCo Trainers, CO$_2$ and O$_2$ concentration, Tests 2-4]{DelCo Trainers, CO$_2$ and O$_2$ concentration, Tests 2-4.}
\label{DelCo_CO2_O2_1}
\end{figure}

\newpage

\begin{figure}[p]
\begin{tabular*}{\textwidth}{l@{\extracolsep{\fill}}r}
\includegraphics[height=2.15in]{SCRIPT_FIGURES/DelCo_Trainers/Test_05_CO2} &
\includegraphics[height=2.15in]{SCRIPT_FIGURES/DelCo_Trainers/Test_05_O2} \\
\includegraphics[height=2.15in]{SCRIPT_FIGURES/DelCo_Trainers/Test_06_CO2} &
\includegraphics[height=2.15in]{SCRIPT_FIGURES/DelCo_Trainers/Test_06_O2}
\end{tabular*}
\caption[DelCo Trainers, CO$_2$ and O$_2$ concentration, Tests 5-6]{DelCo Trainers, CO$_2$ and O$_2$ concentration, Tests 5-6.}
\label{DelCo_CO2_O2_2}
\end{figure}

\begin{figure}[p]
\begin{tabular*}{\textwidth}{l@{\extracolsep{\fill}}r}
\includegraphics[height=2.15in]{SCRIPT_FIGURES/DelCo_Trainers/Test_22_CO2} &
\includegraphics[height=2.15in]{SCRIPT_FIGURES/DelCo_Trainers/Test_22_O2} \\
\includegraphics[height=2.15in]{SCRIPT_FIGURES/DelCo_Trainers/Test_23_CO2} &
\includegraphics[height=2.15in]{SCRIPT_FIGURES/DelCo_Trainers/Test_23_O2} \\
\includegraphics[height=2.15in]{SCRIPT_FIGURES/DelCo_Trainers/Test_24_CO2} &
\includegraphics[height=2.15in]{SCRIPT_FIGURES/DelCo_Trainers/Test_24_O2} \\
\includegraphics[height=2.15in]{SCRIPT_FIGURES/DelCo_Trainers/Test_25_CO2} &
\includegraphics[height=2.15in]{SCRIPT_FIGURES/DelCo_Trainers/Test_25_O2}
\end{tabular*}
\caption[DelCo Trainers, CO$_2$ and O$_2$ concentration, Tests 22-25]{DelCo Trainers, CO$_2$ and O$_2$ concentration, Tests 22-25.}
\label{DelCo_CO2_O2_3}
\end{figure}


\clearpage

\subsection{FAA Cargo Compartments}

Carbon dioxide and carbon monoxide were measured near the ceiling in the forward, middle, and aft sections of the compartment. Note that all but the middle compartment concentrations were measured in Tests~2~and~3.


\begin{figure}[h]
\begin{tabular*}{\textwidth}{l@{\extracolsep{\fill}}r}
\includegraphics[height=2.15in]{SCRIPT_FIGURES/FAA_Cargo_Compartments/FAA_Cargo_Compartments_Test_1_CO2} &
\includegraphics[height=2.15in]{SCRIPT_FIGURES/FAA_Cargo_Compartments/FAA_Cargo_Compartments_Test_1_CO} \\
\includegraphics[height=2.15in]{SCRIPT_FIGURES/FAA_Cargo_Compartments/FAA_Cargo_Compartments_Test_2_CO2} &
\includegraphics[height=2.15in]{SCRIPT_FIGURES/FAA_Cargo_Compartments/FAA_Cargo_Compartments_Test_2_CO} \\
\includegraphics[height=2.15in]{SCRIPT_FIGURES/FAA_Cargo_Compartments/FAA_Cargo_Compartments_Test_3_CO2} &
\includegraphics[height=2.15in]{SCRIPT_FIGURES/FAA_Cargo_Compartments/FAA_Cargo_Compartments_Test_3_CO}
\end{tabular*}
\caption[FAA Cargo Compartment experiments, CO$_2$ and O$_2$ concentration]{FAA Cargo Compartment experiments, CO$_2$ and O$_2$ concentration.}
\label{FAA_Cargo_CO2_CO}
\end{figure}

\clearpage

\subsection{LLNL Enclosure}

Oxygen and carbon dioxide concentrations were reported for a single location within the LLNL Enclosure. The test report~\cite{Foote:LLNL1986} specifies the lateral location of the sensors (1.5~m from the wall opposite the exhaust duct, along the room centerline), but the reported height is only ``top of room.'' In the model, the sensors were located 4.3~m off the floor, 0.2~m below the ceiling of the compartment.

For some experiments, only two concentrations are reported, one being the ambient value. The results of these experiments are shown for completeness, but the values are not included in the calculation of the accuracy statistics.

Also, the test report does not indicate whether the reported volume fractions were taken to be ``wet'' or ``dry''; that is, whether the water vapor was condensed out of the sample. For the sake of comparison, the FDS results are assumed to be ``dry''.

Further details of the modeling can be found in Sec.~\ref{LLNL_Enclosure_Description}.

\newpage

\begin{figure}[p]
\begin{tabular*}{\textwidth}{l@{\extracolsep{\fill}}r}
\includegraphics[height=2.15in]{SCRIPT_FIGURES/LLNL_Enclosure/LLNL_01_O2} &
\includegraphics[height=2.15in]{SCRIPT_FIGURES/LLNL_Enclosure/LLNL_01_CO2} \\
\includegraphics[height=2.15in]{SCRIPT_FIGURES/LLNL_Enclosure/LLNL_02_O2} &
\includegraphics[height=2.15in]{SCRIPT_FIGURES/LLNL_Enclosure/LLNL_02_CO2} \\
\includegraphics[height=2.15in]{SCRIPT_FIGURES/LLNL_Enclosure/LLNL_03_O2} &
\includegraphics[height=2.15in]{SCRIPT_FIGURES/LLNL_Enclosure/LLNL_03_CO2} \\
\includegraphics[height=2.15in]{SCRIPT_FIGURES/LLNL_Enclosure/LLNL_04_O2} &
\includegraphics[height=2.15in]{SCRIPT_FIGURES/LLNL_Enclosure/LLNL_04_CO2}
\end{tabular*}
\caption[LLNL Enclosure, O$_2$ and CO$_2$ concentration, Tests 1-4]{LLNL Enclosure, O$_2$ and CO$_2$ concentration, Tests 1-4.}
\label{LLNL_Gas_1}
\end{figure}

\begin{figure}[p]
\begin{tabular*}{\textwidth}{l@{\extracolsep{\fill}}r}
\includegraphics[height=2.15in]{SCRIPT_FIGURES/LLNL_Enclosure/LLNL_05_O2} &
\includegraphics[height=2.15in]{SCRIPT_FIGURES/LLNL_Enclosure/LLNL_05_CO2} \\
\includegraphics[height=2.15in]{SCRIPT_FIGURES/LLNL_Enclosure/LLNL_06_O2} &
\includegraphics[height=2.15in]{SCRIPT_FIGURES/LLNL_Enclosure/LLNL_06_CO2} \\
\includegraphics[height=2.15in]{SCRIPT_FIGURES/LLNL_Enclosure/LLNL_07_O2} &
\includegraphics[height=2.15in]{SCRIPT_FIGURES/LLNL_Enclosure/LLNL_07_CO2} \\
\includegraphics[height=2.15in]{SCRIPT_FIGURES/LLNL_Enclosure/LLNL_08_O2} &
\includegraphics[height=2.15in]{SCRIPT_FIGURES/LLNL_Enclosure/LLNL_08_CO2}
\end{tabular*}
\caption[LLNL Enclosure, O$_2$ and CO$_2$ concentration, Tests 5-8]{LLNL Enclosure, O$_2$ and CO$_2$ concentration, Tests 5-8.}
\label{LLNL_Gas_2}
\end{figure}

\begin{figure}[p]
\begin{tabular*}{\textwidth}{l@{\extracolsep{\fill}}r}
\includegraphics[height=2.15in]{SCRIPT_FIGURES/LLNL_Enclosure/LLNL_09_O2} &
\includegraphics[height=2.15in]{SCRIPT_FIGURES/LLNL_Enclosure/LLNL_09_CO2} \\
\includegraphics[height=2.15in]{SCRIPT_FIGURES/LLNL_Enclosure/LLNL_10_O2} &
\includegraphics[height=2.15in]{SCRIPT_FIGURES/LLNL_Enclosure/LLNL_10_CO2} \\
\includegraphics[height=2.15in]{SCRIPT_FIGURES/LLNL_Enclosure/LLNL_11_O2} &
\includegraphics[height=2.15in]{SCRIPT_FIGURES/LLNL_Enclosure/LLNL_11_CO2} \\
\includegraphics[height=2.15in]{SCRIPT_FIGURES/LLNL_Enclosure/LLNL_12_O2} &
\includegraphics[height=2.15in]{SCRIPT_FIGURES/LLNL_Enclosure/LLNL_12_CO2}
\end{tabular*}
\caption[LLNL Enclosure, O$_2$ and CO$_2$ concentration, Tests 9-12]{LLNL Enclosure, O$_2$ and CO$_2$ concentration, Tests 9-12.}
\label{LLNL_Gas_3}
\end{figure}

\begin{figure}[p]
\begin{tabular*}{\textwidth}{l@{\extracolsep{\fill}}r}
\includegraphics[height=2.15in]{SCRIPT_FIGURES/LLNL_Enclosure/LLNL_13_O2} &
\includegraphics[height=2.15in]{SCRIPT_FIGURES/LLNL_Enclosure/LLNL_13_CO2} \\
\includegraphics[height=2.15in]{SCRIPT_FIGURES/LLNL_Enclosure/LLNL_14_O2} &
\includegraphics[height=2.15in]{SCRIPT_FIGURES/LLNL_Enclosure/LLNL_14_CO2} \\
\includegraphics[height=2.15in]{SCRIPT_FIGURES/LLNL_Enclosure/LLNL_15_O2} &
\includegraphics[height=2.15in]{SCRIPT_FIGURES/LLNL_Enclosure/LLNL_15_CO2} \\
\includegraphics[height=2.15in]{SCRIPT_FIGURES/LLNL_Enclosure/LLNL_16_O2} &
\includegraphics[height=2.15in]{SCRIPT_FIGURES/LLNL_Enclosure/LLNL_16_CO2}
\end{tabular*}
\caption[LLNL Enclosure, O$_2$ and CO$_2$ concentration, Tests 13-16]{LLNL Enclosure, O$_2$ and CO$_2$ concentration, Tests 13-16.}
\label{LLNL_Gas_4}
\end{figure}

\begin{figure}[p]
\begin{tabular*}{\textwidth}{l@{\extracolsep{\fill}}r}
\includegraphics[height=2.15in]{SCRIPT_FIGURES/LLNL_Enclosure/LLNL_17_O2} &
\includegraphics[height=2.15in]{SCRIPT_FIGURES/LLNL_Enclosure/LLNL_17_CO2} \\
\includegraphics[height=2.15in]{SCRIPT_FIGURES/LLNL_Enclosure/LLNL_18_O2} &
\includegraphics[height=2.15in]{SCRIPT_FIGURES/LLNL_Enclosure/LLNL_18_CO2} \\
\includegraphics[height=2.15in]{SCRIPT_FIGURES/LLNL_Enclosure/LLNL_19_O2} &
\includegraphics[height=2.15in]{SCRIPT_FIGURES/LLNL_Enclosure/LLNL_19_CO2} \\
\includegraphics[height=2.15in]{SCRIPT_FIGURES/LLNL_Enclosure/LLNL_20_O2} &
\includegraphics[height=2.15in]{SCRIPT_FIGURES/LLNL_Enclosure/LLNL_20_CO2}
\end{tabular*}
\caption[LLNL Enclosure, O$_2$ and CO$_2$ concentration, Tests 17-20]{LLNL Enclosure, O$_2$ and CO$_2$ concentration, Tests 17-20.}
\label{LLNL_Gas_5}
\end{figure}

\begin{figure}[p]
\begin{tabular*}{\textwidth}{l@{\extracolsep{\fill}}r}
\includegraphics[height=2.15in]{SCRIPT_FIGURES/LLNL_Enclosure/LLNL_21_O2} &
\includegraphics[height=2.15in]{SCRIPT_FIGURES/LLNL_Enclosure/LLNL_21_CO2} \\
\includegraphics[height=2.15in]{SCRIPT_FIGURES/LLNL_Enclosure/LLNL_22_O2} &
\includegraphics[height=2.15in]{SCRIPT_FIGURES/LLNL_Enclosure/LLNL_22_CO2} \\
\includegraphics[height=2.15in]{SCRIPT_FIGURES/LLNL_Enclosure/LLNL_23_O2} &
\includegraphics[height=2.15in]{SCRIPT_FIGURES/LLNL_Enclosure/LLNL_23_CO2} \\
\includegraphics[height=2.15in]{SCRIPT_FIGURES/LLNL_Enclosure/LLNL_24_O2} &
\includegraphics[height=2.15in]{SCRIPT_FIGURES/LLNL_Enclosure/LLNL_24_CO2}
\end{tabular*}
\caption[LLNL Enclosure, O$_2$ and CO$_2$ concentration, Tests 21-24]{LLNL Enclosure, O$_2$ and CO$_2$ concentration, Tests 21-24.}
\label{LLNL_Gas_6}
\end{figure}

\begin{figure}[p]
\begin{tabular*}{\textwidth}{l@{\extracolsep{\fill}}r}
\includegraphics[height=2.15in]{SCRIPT_FIGURES/LLNL_Enclosure/LLNL_25_O2} &
\includegraphics[height=2.15in]{SCRIPT_FIGURES/LLNL_Enclosure/LLNL_25_CO2} \\
\includegraphics[height=2.15in]{SCRIPT_FIGURES/LLNL_Enclosure/LLNL_26_O2} &
\includegraphics[height=2.15in]{SCRIPT_FIGURES/LLNL_Enclosure/LLNL_26_CO2} \\
\includegraphics[height=2.15in]{SCRIPT_FIGURES/LLNL_Enclosure/LLNL_27_O2} &
\includegraphics[height=2.15in]{SCRIPT_FIGURES/LLNL_Enclosure/LLNL_27_CO2} \\
\includegraphics[height=2.15in]{SCRIPT_FIGURES/LLNL_Enclosure/LLNL_28_O2} &
\includegraphics[height=2.15in]{SCRIPT_FIGURES/LLNL_Enclosure/LLNL_28_CO2}
\end{tabular*}
\caption[LLNL Enclosure, O$_2$ and CO$_2$ concentration, Tests 25-28]{LLNL Enclosure, O$_2$ and CO$_2$ concentration, Tests 25-28.}
\label{LLNL_Gas_7}
\end{figure}

\begin{figure}[p]
\begin{tabular*}{\textwidth}{l@{\extracolsep{\fill}}r}
\includegraphics[height=2.15in]{SCRIPT_FIGURES/LLNL_Enclosure/LLNL_29_O2} &
\includegraphics[height=2.15in]{SCRIPT_FIGURES/LLNL_Enclosure/LLNL_29_CO2} \\
\includegraphics[height=2.15in]{SCRIPT_FIGURES/LLNL_Enclosure/LLNL_30_O2} &
\includegraphics[height=2.15in]{SCRIPT_FIGURES/LLNL_Enclosure/LLNL_30_CO2} \\
\includegraphics[height=2.15in]{SCRIPT_FIGURES/LLNL_Enclosure/LLNL_31_O2} &
\includegraphics[height=2.15in]{SCRIPT_FIGURES/LLNL_Enclosure/LLNL_31_CO2} \\
\includegraphics[height=2.15in]{SCRIPT_FIGURES/LLNL_Enclosure/LLNL_32_O2} &
\includegraphics[height=2.15in]{SCRIPT_FIGURES/LLNL_Enclosure/LLNL_32_CO2}
\end{tabular*}
\caption[LLNL Enclosure, O$_2$ and CO$_2$ concentration, Tests 29-32]{LLNL Enclosure, O$_2$ and CO$_2$ concentration, Tests 29-32.}
\label{LLNL_Gas_8}
\end{figure}

\begin{figure}[p]
\begin{tabular*}{\textwidth}{l@{\extracolsep{\fill}}r}
\includegraphics[height=2.15in]{SCRIPT_FIGURES/LLNL_Enclosure/LLNL_33_O2} &
\includegraphics[height=2.15in]{SCRIPT_FIGURES/LLNL_Enclosure/LLNL_33_CO2} \\
\includegraphics[height=2.15in]{SCRIPT_FIGURES/LLNL_Enclosure/LLNL_34_O2} &
\includegraphics[height=2.15in]{SCRIPT_FIGURES/LLNL_Enclosure/LLNL_34_CO2} \\
\includegraphics[height=2.15in]{SCRIPT_FIGURES/LLNL_Enclosure/LLNL_35_O2} &
\includegraphics[height=2.15in]{SCRIPT_FIGURES/LLNL_Enclosure/LLNL_35_CO2} \\
\includegraphics[height=2.15in]{SCRIPT_FIGURES/LLNL_Enclosure/LLNL_36_O2} &
\includegraphics[height=2.15in]{SCRIPT_FIGURES/LLNL_Enclosure/LLNL_36_CO2}
\end{tabular*}
\caption[LLNL Enclosure, O$_2$ and CO$_2$ concentration, Tests 33-36]{LLNL Enclosure, O$_2$ and CO$_2$ concentration, Tests 33-36.}
\label{LLNL_Gas_9}
\end{figure}

\begin{figure}[p]
\begin{tabular*}{\textwidth}{l@{\extracolsep{\fill}}r}
\includegraphics[height=2.15in]{SCRIPT_FIGURES/LLNL_Enclosure/LLNL_37_O2} &
\includegraphics[height=2.15in]{SCRIPT_FIGURES/LLNL_Enclosure/LLNL_37_CO2} \\
\includegraphics[height=2.15in]{SCRIPT_FIGURES/LLNL_Enclosure/LLNL_38_O2} &
\includegraphics[height=2.15in]{SCRIPT_FIGURES/LLNL_Enclosure/LLNL_38_CO2} \\
\includegraphics[height=2.15in]{SCRIPT_FIGURES/LLNL_Enclosure/LLNL_39_O2} &
\includegraphics[height=2.15in]{SCRIPT_FIGURES/LLNL_Enclosure/LLNL_39_CO2} \\
\includegraphics[height=2.15in]{SCRIPT_FIGURES/LLNL_Enclosure/LLNL_40_O2} &
\includegraphics[height=2.15in]{SCRIPT_FIGURES/LLNL_Enclosure/LLNL_40_CO2}
\end{tabular*}
\caption[LLNL Enclosure, O$_2$ and CO$_2$ concentration, Tests 37-40]{LLNL Enclosure, O$_2$ and CO$_2$ concentration, Tests 37-40.}
\label{LLNL_Gas_10}
\end{figure}

\begin{figure}[p]
\begin{tabular*}{\textwidth}{l@{\extracolsep{\fill}}r}
\includegraphics[height=2.15in]{SCRIPT_FIGURES/LLNL_Enclosure/LLNL_41_O2} &
\includegraphics[height=2.15in]{SCRIPT_FIGURES/LLNL_Enclosure/LLNL_41_CO2} \\
\includegraphics[height=2.15in]{SCRIPT_FIGURES/LLNL_Enclosure/LLNL_42_O2} &
\includegraphics[height=2.15in]{SCRIPT_FIGURES/LLNL_Enclosure/LLNL_42_CO2} \\
\includegraphics[height=2.15in]{SCRIPT_FIGURES/LLNL_Enclosure/LLNL_43_O2} &
\includegraphics[height=2.15in]{SCRIPT_FIGURES/LLNL_Enclosure/LLNL_43_CO2} \\
\includegraphics[height=2.15in]{SCRIPT_FIGURES/LLNL_Enclosure/LLNL_44_O2} &
\includegraphics[height=2.15in]{SCRIPT_FIGURES/LLNL_Enclosure/LLNL_44_CO2}
\end{tabular*}
\caption[LLNL Enclosure, O$_2$ and CO$_2$ concentration, Tests 41-44]{LLNL Enclosure, O$_2$ and CO$_2$ concentration, Tests 41-44.}
\label{LLNL_Gas_11}
\end{figure}

\begin{figure}[p]
\begin{tabular*}{\textwidth}{l@{\extracolsep{\fill}}r}
\includegraphics[height=2.15in]{SCRIPT_FIGURES/LLNL_Enclosure/LLNL_45_O2} &
\includegraphics[height=2.15in]{SCRIPT_FIGURES/LLNL_Enclosure/LLNL_45_CO2} \\
\includegraphics[height=2.15in]{SCRIPT_FIGURES/LLNL_Enclosure/LLNL_46_O2} &
\includegraphics[height=2.15in]{SCRIPT_FIGURES/LLNL_Enclosure/LLNL_46_CO2} \\
\includegraphics[height=2.15in]{SCRIPT_FIGURES/LLNL_Enclosure/LLNL_47_O2} &
\includegraphics[height=2.15in]{SCRIPT_FIGURES/LLNL_Enclosure/LLNL_47_CO2} \\
\includegraphics[height=2.15in]{SCRIPT_FIGURES/LLNL_Enclosure/LLNL_48_O2} &
\includegraphics[height=2.15in]{SCRIPT_FIGURES/LLNL_Enclosure/LLNL_48_CO2}
\end{tabular*}
\caption[LLNL Enclosure, O$_2$ and CO$_2$ concentration, Tests 45-48]{LLNL Enclosure, O$_2$ and CO$_2$ concentration, Tests 45-48.}
\label{LLNL_Gas_12}
\end{figure}

\begin{figure}[p]
\begin{tabular*}{\textwidth}{l@{\extracolsep{\fill}}r}
\includegraphics[height=2.15in]{SCRIPT_FIGURES/LLNL_Enclosure/LLNL_49_O2} &
\includegraphics[height=2.15in]{SCRIPT_FIGURES/LLNL_Enclosure/LLNL_49_CO2} \\
\includegraphics[height=2.15in]{SCRIPT_FIGURES/LLNL_Enclosure/LLNL_50_O2} &
\includegraphics[height=2.15in]{SCRIPT_FIGURES/LLNL_Enclosure/LLNL_50_CO2} \\
\includegraphics[height=2.15in]{SCRIPT_FIGURES/LLNL_Enclosure/LLNL_51_O2} &
\includegraphics[height=2.15in]{SCRIPT_FIGURES/LLNL_Enclosure/LLNL_51_CO2} \\
\includegraphics[height=2.15in]{SCRIPT_FIGURES/LLNL_Enclosure/LLNL_52_O2} &
\includegraphics[height=2.15in]{SCRIPT_FIGURES/LLNL_Enclosure/LLNL_52_CO2}
\end{tabular*}
\caption[LLNL Enclosure, O$_2$ and CO$_2$ concentration, Tests 49-52]{LLNL Enclosure, O$_2$ and CO$_2$ concentration, Tests 49-52.}
\label{LLNL_Gas_13}
\end{figure}

\begin{figure}[p]
\begin{tabular*}{\textwidth}{l@{\extracolsep{\fill}}r}
\includegraphics[height=2.15in]{SCRIPT_FIGURES/LLNL_Enclosure/LLNL_53_O2} &
\includegraphics[height=2.15in]{SCRIPT_FIGURES/LLNL_Enclosure/LLNL_53_CO2} \\
\includegraphics[height=2.15in]{SCRIPT_FIGURES/LLNL_Enclosure/LLNL_54_O2} &
\includegraphics[height=2.15in]{SCRIPT_FIGURES/LLNL_Enclosure/LLNL_54_CO2} \\
\includegraphics[height=2.15in]{SCRIPT_FIGURES/LLNL_Enclosure/LLNL_55_O2} &
\includegraphics[height=2.15in]{SCRIPT_FIGURES/LLNL_Enclosure/LLNL_55_CO2} \\
\includegraphics[height=2.15in]{SCRIPT_FIGURES/LLNL_Enclosure/LLNL_56_O2} &
\includegraphics[height=2.15in]{SCRIPT_FIGURES/LLNL_Enclosure/LLNL_56_CO2}
\end{tabular*}
\caption[LLNL Enclosure, O$_2$ and CO$_2$ concentration, Tests 53-56]{LLNL Enclosure, O$_2$ and CO$_2$ concentration, Tests 53-56.}
\label{LLNL_Gas_14}
\end{figure}

\begin{figure}[p]
\begin{tabular*}{\textwidth}{l@{\extracolsep{\fill}}r}
\includegraphics[height=2.15in]{SCRIPT_FIGURES/LLNL_Enclosure/LLNL_57_O2} &
\includegraphics[height=2.15in]{SCRIPT_FIGURES/LLNL_Enclosure/LLNL_57_CO2} \\
\includegraphics[height=2.15in]{SCRIPT_FIGURES/LLNL_Enclosure/LLNL_58_O2} &
\includegraphics[height=2.15in]{SCRIPT_FIGURES/LLNL_Enclosure/LLNL_58_CO2} \\
\includegraphics[height=2.15in]{SCRIPT_FIGURES/LLNL_Enclosure/LLNL_59_O2} &
\includegraphics[height=2.15in]{SCRIPT_FIGURES/LLNL_Enclosure/LLNL_59_CO2} \\
\includegraphics[height=2.15in]{SCRIPT_FIGURES/LLNL_Enclosure/LLNL_60_O2} &
\includegraphics[height=2.15in]{SCRIPT_FIGURES/LLNL_Enclosure/LLNL_60_CO2}
\end{tabular*}
\caption[LLNL Enclosure, O$_2$ and CO$_2$ concentration, Tests 57-60]{LLNL Enclosure, O$_2$ and CO$_2$ concentration, Tests 57-60.}
\label{LLNL_Gas_15}
\end{figure}

\begin{figure}[p]
\begin{tabular*}{\textwidth}{l@{\extracolsep{\fill}}r}
\includegraphics[height=2.15in]{SCRIPT_FIGURES/LLNL_Enclosure/LLNL_61_O2} &
\includegraphics[height=2.15in]{SCRIPT_FIGURES/LLNL_Enclosure/LLNL_61_CO2} \\
\includegraphics[height=2.15in]{SCRIPT_FIGURES/LLNL_Enclosure/LLNL_62_O2} &
\includegraphics[height=2.15in]{SCRIPT_FIGURES/LLNL_Enclosure/LLNL_62_CO2} \\
\includegraphics[height=2.15in]{SCRIPT_FIGURES/LLNL_Enclosure/LLNL_63_O2} &
\includegraphics[height=2.15in]{SCRIPT_FIGURES/LLNL_Enclosure/LLNL_63_CO2} \\
\includegraphics[height=2.15in]{SCRIPT_FIGURES/LLNL_Enclosure/LLNL_64_O2} &
\includegraphics[height=2.15in]{SCRIPT_FIGURES/LLNL_Enclosure/LLNL_64_CO2}
\end{tabular*}
\caption[LLNL Enclosure, O$_2$ and CO$_2$ concentration, Tests 61-64]{LLNL Enclosure, O$_2$ and CO$_2$ concentration, Tests 61-64.}
\label{LLNL_Gas_16}
\end{figure}


\clearpage


\subsection{NIST/NRC Experiments}

The following pages present comparisons of oxygen and carbon dioxide concentration predictions and measurements for the
NIST/NRC series. There were two oxygen measurements, one in the upper layer, one in the lower.  There was only one carbon
dioxide measurement in the upper layer.

\begin{figure}[h]
\begin{tabular*}{\textwidth}{l@{\extracolsep{\fill}}r}
\includegraphics[height=2.15in]{SCRIPT_FIGURES/NIST_NRC/NIST_NRC_17_Oxygen} &
\includegraphics[height=2.15in]{SCRIPT_FIGURES/NIST_NRC/NIST_NRC_17_CO2} \\
\includegraphics[height=2.15in]{SCRIPT_FIGURES/NIST_NRC/NIST_NRC_03_Oxygen} &
\includegraphics[height=2.15in]{SCRIPT_FIGURES/NIST_NRC/NIST_NRC_03_CO2} \\
\includegraphics[height=2.15in]{SCRIPT_FIGURES/NIST_NRC/NIST_NRC_09_Oxygen} &
\includegraphics[height=2.15in]{SCRIPT_FIGURES/NIST_NRC/NIST_NRC_09_CO2}
\end{tabular*}
\caption[NIST/NRC experiments, CO$_2$ and O$_2$ concentration, Tests 3, 9, 17]{NIST/NRC experiments, CO$_2$ and O$_2$ concentration, Tests 3, 9, 17.}
\label{NIST_NRC_Gas_Open_1}
\end{figure}

\newpage

\begin{figure}[p]
\begin{tabular*}{\textwidth}{l@{\extracolsep{\fill}}r}
\includegraphics[height=2.15in]{SCRIPT_FIGURES/NIST_NRC/NIST_NRC_05_Oxygen} &
\includegraphics[height=2.15in]{SCRIPT_FIGURES/NIST_NRC/NIST_NRC_05_CO2} \\
\includegraphics[height=2.15in]{SCRIPT_FIGURES/NIST_NRC/NIST_NRC_14_Oxygen} &
\includegraphics[height=2.15in]{SCRIPT_FIGURES/NIST_NRC/NIST_NRC_14_CO2} \\
\includegraphics[height=2.15in]{SCRIPT_FIGURES/NIST_NRC/NIST_NRC_15_Oxygen} &
\includegraphics[height=2.15in]{SCRIPT_FIGURES/NIST_NRC/NIST_NRC_15_CO2} \\
\includegraphics[height=2.15in]{SCRIPT_FIGURES/NIST_NRC/NIST_NRC_18_Oxygen} &
\includegraphics[height=2.15in]{SCRIPT_FIGURES/NIST_NRC/NIST_NRC_18_CO2}
\end{tabular*}
\caption[NIST/NRC experiments, CO$_2$ and O$_2$ concentration, Tests 5, 14, 15, 18]{NIST/NRC experiments, CO$_2$ and O$_2$ concentration, Tests 5, 14, 15, 18.}
\label{NIST_NRC_Gas_Open_2}
\end{figure}

\begin{figure}[p]
\begin{tabular*}{\textwidth}{l@{\extracolsep{\fill}}r}
\includegraphics[height=2.15in]{SCRIPT_FIGURES/NIST_NRC/NIST_NRC_01_Oxygen} &
\includegraphics[height=2.15in]{SCRIPT_FIGURES/NIST_NRC/NIST_NRC_01_CO2} \\
\includegraphics[height=2.15in]{SCRIPT_FIGURES/NIST_NRC/NIST_NRC_07_Oxygen} &
\includegraphics[height=2.15in]{SCRIPT_FIGURES/NIST_NRC/NIST_NRC_07_CO2} \\
\includegraphics[height=2.15in]{SCRIPT_FIGURES/NIST_NRC/NIST_NRC_02_Oxygen} &
\includegraphics[height=2.15in]{SCRIPT_FIGURES/NIST_NRC/NIST_NRC_02_CO2} \\
\includegraphics[height=2.15in]{SCRIPT_FIGURES/NIST_NRC/NIST_NRC_08_Oxygen} &
\includegraphics[height=2.15in]{SCRIPT_FIGURES/NIST_NRC/NIST_NRC_08_CO2}
\end{tabular*}
\caption[NIST/NRC experiments, CO$_2$ and O$_2$ concentration, Tests 1, 2, 7, 8]{NIST/NRC experiments, CO$_2$ and O$_2$ concentration, Tests 1, 2, 7, 8.}
\label{NIST_NRC_Gas_Closed_1}
\end{figure}

\begin{figure}[p]
\begin{tabular*}{\textwidth}{l@{\extracolsep{\fill}}r}
\includegraphics[height=2.15in]{SCRIPT_FIGURES/NIST_NRC/NIST_NRC_04_Oxygen} &
\includegraphics[height=2.15in]{SCRIPT_FIGURES/NIST_NRC/NIST_NRC_04_CO2} \\
\includegraphics[height=2.15in]{SCRIPT_FIGURES/NIST_NRC/NIST_NRC_10_Oxygen} &
\includegraphics[height=2.15in]{SCRIPT_FIGURES/NIST_NRC/NIST_NRC_10_CO2} \\
\includegraphics[height=2.15in]{SCRIPT_FIGURES/NIST_NRC/NIST_NRC_13_Oxygen} &
\includegraphics[height=2.15in]{SCRIPT_FIGURES/NIST_NRC/NIST_NRC_13_CO2} \\
\includegraphics[height=2.15in]{SCRIPT_FIGURES/NIST_NRC/NIST_NRC_16_Oxygen} &
\includegraphics[height=2.15in]{SCRIPT_FIGURES/NIST_NRC/NIST_NRC_16_CO2}
\end{tabular*}
\caption[NIST/NRC experiments, CO$_2$ and O$_2$ concentration, Tests 4, 10, 13, 16]{NIST/NRC experiments, CO$_2$ and O$_2$ concentration, Tests 4, 10, 13, 16.}
\label{NIST_NRC_Gas_Closed_2}
\end{figure}


\clearpage

\subsection{NRCC Smoke Tower}

In the NRCC Smoke Tower experiments, there were oxygen and carbon dioxide analyzers in the stair shaft on the second floor just outside the door of the fire compartment.

\begin{figure}[!ht]
\begin{tabular*}{\textwidth}{l@{\extracolsep{\fill}}r}
\includegraphics[height=2.15in]{SCRIPT_FIGURES/NRCC_Smoke_Tower/BK-R_Fire_Lobby_O2} &
\includegraphics[height=2.15in]{SCRIPT_FIGURES/NRCC_Smoke_Tower/BK-R_Fire_Lobby_CO2} \\
\includegraphics[height=2.15in]{SCRIPT_FIGURES/NRCC_Smoke_Tower/CMP-R_Fire_Lobby_O2} &
\includegraphics[height=2.15in]{SCRIPT_FIGURES/NRCC_Smoke_Tower/CMP-R_Fire_Lobby_CO2}
\end{tabular*}
\caption[NRCC Smoke Tower, CO$_2$ and O$_2$ concentration, Tests BK-R and COMP-R]{NRCC Smoke Tower, CO$_2$ and O$_2$ concentration, Tests BK-R and COMP-R.}
\label{NRCC_Smoke_Tower_O2_CO2_1}
\end{figure}

\begin{figure}[!ht]
\begin{tabular*}{\textwidth}{l@{\extracolsep{\fill}}r}
\includegraphics[height=2.15in]{SCRIPT_FIGURES/NRCC_Smoke_Tower/CLC-I-R_Fire_Lobby_O2} &
\includegraphics[height=2.15in]{SCRIPT_FIGURES/NRCC_Smoke_Tower/CLC-I-R_Fire_Lobby_CO2} \\
\includegraphics[height=2.15in]{SCRIPT_FIGURES/NRCC_Smoke_Tower/CLC-II-R_Fire_Lobby_O2} &
\includegraphics[height=2.15in]{SCRIPT_FIGURES/NRCC_Smoke_Tower/CLC-II-R_Fire_Lobby_CO2}
\end{tabular*}
\caption[NRCC Smoke Tower, CO$_2$ and O$_2$ concentration, Tests CLC-I-R and CLC-II-R]{NRCC Smoke Tower, CO$_2$ and O$_2$ concentration, Tests CLC-I-R and CLC-II-R.}
\label{NRCC_Smoke_Tower_O2_CO2_2}
\end{figure}


\clearpage

\subsection{PRISME DOOR Experiments}

Each compartment in the PRISME DOOR experiments contained an oxygen and carbon dioxide measurement in the upper (haut) and lower (bas) layers.

\begin{figure}[!ht]
\begin{tabular*}{\textwidth}{l@{\extracolsep{\fill}}r}
\includegraphics[height=2.15in]{SCRIPT_FIGURES/PRISME/PRS_D1_Room_1_O2} &
\includegraphics[height=2.15in]{SCRIPT_FIGURES/PRISME/PRS_D1_Room_1_CO2} \\
\includegraphics[height=2.15in]{SCRIPT_FIGURES/PRISME/PRS_D2_Room_1_O2} &
\includegraphics[height=2.15in]{SCRIPT_FIGURES/PRISME/PRS_D2_Room_1_CO2} \\
\includegraphics[height=2.15in]{SCRIPT_FIGURES/PRISME/PRS_D3_Room_1_O2} &
\includegraphics[height=2.15in]{SCRIPT_FIGURES/PRISME/PRS_D3_Room_1_CO2}
\end{tabular*}
\caption[PRISME DOOR experiments, CO$_2$ and O$_2$ concentration, Room 1, Tests 1-3]{PRISME DOOR experiments, CO$_2$ and O$_2$ concentration, Room 1, Tests 1-3.}
\label{PRISME_Gas_1}
\end{figure}

\newpage

\begin{figure}[p]
\begin{tabular*}{\textwidth}{l@{\extracolsep{\fill}}r}
\includegraphics[height=2.15in]{SCRIPT_FIGURES/PRISME/PRS_D4_Room_1_O2} &
\includegraphics[height=2.15in]{SCRIPT_FIGURES/PRISME/PRS_D4_Room_1_CO2} \\
\includegraphics[height=2.15in]{SCRIPT_FIGURES/PRISME/PRS_D5_Room_1_O2} &
\includegraphics[height=2.15in]{SCRIPT_FIGURES/PRISME/PRS_D5_Room_1_CO2} \\
\includegraphics[height=2.15in]{SCRIPT_FIGURES/PRISME/PRS_D6_Room_1_O2} &
\includegraphics[height=2.15in]{SCRIPT_FIGURES/PRISME/PRS_D6_Room_1_CO2}
\end{tabular*}
\caption[PRISME DOOR experiments, CO$_2$ and O$_2$ concentration, Room 1, Tests 4-6]{PRISME DOOR experiments, CO$_2$ and O$_2$ concentration, Room 1, Tests 4-6.}
\label{PRISME_Gas_2}
\end{figure}

\begin{figure}[p]
\begin{tabular*}{\textwidth}{l@{\extracolsep{\fill}}r}
\includegraphics[height=2.15in]{SCRIPT_FIGURES/PRISME/PRS_D1_Room_2_O2} &
\includegraphics[height=2.15in]{SCRIPT_FIGURES/PRISME/PRS_D1_Room_2_CO2} \\
\includegraphics[height=2.15in]{SCRIPT_FIGURES/PRISME/PRS_D2_Room_2_O2} &
\includegraphics[height=2.15in]{SCRIPT_FIGURES/PRISME/PRS_D2_Room_2_CO2} \\
\includegraphics[height=2.15in]{SCRIPT_FIGURES/PRISME/PRS_D3_Room_2_O2} &
\includegraphics[height=2.15in]{SCRIPT_FIGURES/PRISME/PRS_D3_Room_2_CO2}
\end{tabular*}
\caption[PRISME DOOR experiments, CO$_2$ and O$_2$ concentration, Room 2, Tests 1-3]{PRISME DOOR experiments, CO$_2$ and O$_2$ concentration, Room 2, Tests 1-3.}
\label{PRISME_Gas_3}
\end{figure}

\begin{figure}[p]
\begin{tabular*}{\textwidth}{l@{\extracolsep{\fill}}r}
\includegraphics[height=2.15in]{SCRIPT_FIGURES/PRISME/PRS_D4_Room_2_O2} &
\includegraphics[height=2.15in]{SCRIPT_FIGURES/PRISME/PRS_D4_Room_2_CO2} \\
\includegraphics[height=2.15in]{SCRIPT_FIGURES/PRISME/PRS_D5_Room_2_O2} &
\includegraphics[height=2.15in]{SCRIPT_FIGURES/PRISME/PRS_D5_Room_2_CO2} \\
\includegraphics[height=2.15in]{SCRIPT_FIGURES/PRISME/PRS_D6_Room_2_O2} &
\includegraphics[height=2.15in]{SCRIPT_FIGURES/PRISME/PRS_D6_Room_2_CO2}
\end{tabular*}
\caption[PRISME DOOR experiments, CO$_2$ and O$_2$ concentration, Room 2, Tests 4-6]{PRISME DOOR experiments, CO$_2$ and O$_2$ concentration, Room 2, Tests 4-6.}
\label{PRISME_Gas_4}
\end{figure}

\clearpage

\subsection{PRISME SOURCE Experiments}

The compartment in the PRISME SOURCE experiments contained an oxygen and carbon dioxide measurement in the upper (haut) and lower (bas) layers.

\begin{figure}[!ht]
\begin{tabular*}{\textwidth}{l@{\extracolsep{\fill}}r}
\includegraphics[height=2.15in]{SCRIPT_FIGURES/PRISME/PRS_SI_D1_Room_2_O2} &
\includegraphics[height=2.15in]{SCRIPT_FIGURES/PRISME/PRS_SI_D1_Room_2_CO2} \\
\includegraphics[height=2.15in]{SCRIPT_FIGURES/PRISME/PRS_SI_D2_Room_2_O2} &
\includegraphics[height=2.15in]{SCRIPT_FIGURES/PRISME/PRS_SI_D2_Room_2_CO2} \\
\includegraphics[height=2.15in]{SCRIPT_FIGURES/PRISME/PRS_SI_D3_Room_2_O2} &
\includegraphics[height=2.15in]{SCRIPT_FIGURES/PRISME/PRS_SI_D3_Room_2_CO2} \\
\includegraphics[height=2.15in]{SCRIPT_FIGURES/PRISME/PRS_SI_D4_Room_2_O2} &
\includegraphics[height=2.15in]{SCRIPT_FIGURES/PRISME/PRS_SI_D4_Room_2_CO2}
\end{tabular*}
\caption[PRISME SOURCE experiments, CO$_2$ and O$_2$ concentration, Room 2, Tests 1-4]{PRISME SOURCE experiments, CO$_2$ and O$_2$ concentration, Room 2, Tests 1-4.}
\label{PRISME_SOURCE_Gas_1}
\end{figure}

\newpage

\begin{figure}[p]
\begin{tabular*}{\textwidth}{l@{\extracolsep{\fill}}r}
\includegraphics[height=2.15in]{SCRIPT_FIGURES/PRISME/PRS_SI_D5_Room_2_O2} &
\includegraphics[height=2.15in]{SCRIPT_FIGURES/PRISME/PRS_SI_D5_Room_2_CO2} \\
\includegraphics[height=2.15in]{SCRIPT_FIGURES/PRISME/PRS_SI_D5a_Room_2_O2} &
\includegraphics[height=2.15in]{SCRIPT_FIGURES/PRISME/PRS_SI_D5a_Room_2_CO2} \\
\includegraphics[height=2.15in]{SCRIPT_FIGURES/PRISME/PRS_SI_D6_Room_2_O2} &
\includegraphics[height=2.15in]{SCRIPT_FIGURES/PRISME/PRS_SI_D6_Room_2_CO2} \\
\includegraphics[height=2.15in]{SCRIPT_FIGURES/PRISME/PRS_SI_D6a_Room_2_O2} &
\includegraphics[height=2.15in]{SCRIPT_FIGURES/PRISME/PRS_SI_D6a_Room_2_CO2}
\end{tabular*}
\caption[PRISME SOURCE experiments, CO$_2$ and O$_2$ concentration, Room 2, Tests 5-6]{PRISME SOURCE experiments, CO$_2$ and O$_2$ concentration, Room 2, Tests 5-6.}
\label{PRISME_SOURCE_Gas_2}
\end{figure}

\clearpage


\subsection{WTC Experiments}

The following pages present comparisons of oxygen and carbon dioxide concentration predictions and measurements for the
WTC experiments. There was only one measurement of each made near the ceiling of the compartment roughly 2~m from the fire.


\begin{figure}[h]
\begin{tabular*}{\textwidth}{l@{\extracolsep{\fill}}r}
\includegraphics[height=2.15in]{SCRIPT_FIGURES/WTC/WTC_01_Oxygen} &
\includegraphics[height=2.15in]{SCRIPT_FIGURES/WTC/WTC_01_CO2} \\
\includegraphics[height=2.15in]{SCRIPT_FIGURES/WTC/WTC_02_Oxygen} &
\includegraphics[height=2.15in]{SCRIPT_FIGURES/WTC/WTC_02_CO2} \\
\includegraphics[height=2.15in]{SCRIPT_FIGURES/WTC/WTC_03_Oxygen} &
\includegraphics[height=2.15in]{SCRIPT_FIGURES/WTC/WTC_03_CO2}
\end{tabular*}
\caption[WTC experiments, CO$_2$ and O$_2$ concentration, Tests 1-3]{WTC experiments, CO$_2$ and O$_2$ concentration, Tests 1-3.}
\label{NIST_WTC_Oxygen_CO2_1}
\end{figure}

\newpage

\begin{figure}[p]
\begin{tabular*}{\textwidth}{l@{\extracolsep{\fill}}r}
\includegraphics[height=2.15in]{SCRIPT_FIGURES/WTC/WTC_04_Oxygen} &
\includegraphics[height=2.15in]{SCRIPT_FIGURES/WTC/WTC_04_CO2} \\
\includegraphics[height=2.15in]{SCRIPT_FIGURES/WTC/WTC_05_Oxygen} &
\includegraphics[height=2.15in]{SCRIPT_FIGURES/WTC/WTC_05_CO2} \\
\includegraphics[height=2.15in]{SCRIPT_FIGURES/WTC/WTC_06_Oxygen} &
\includegraphics[height=2.15in]{SCRIPT_FIGURES/WTC/WTC_06_CO2}
\end{tabular*}
\caption[WTC experiments, CO$_2$ and O$_2$ concentration, Tests 4-6]{WTC experiments, CO$_2$ and O$_2$ concentration, Tests 4-6.}
\label{NIST_WTC_Oxygen_CO2_2}
\end{figure}

\clearpage


\subsection{UMD Line Burner}
\label{UMD_Line_Burner_species}

Oxygen concentration measurements were made across the coflow section of the burner.  Fig.~\ref{UMD_Line_Burner_methane_O2_p18_O2} shows mean volume fraction O$_2$ profiles for two heights, $z$, above the burner surface for the experiment with nitrogen dilution of the coflowing air to 18 vol.~\% O$_2$ with methane as fuel.  Notice that the O$_2$ level at the outer edge of the burner is the ambient value of 21 vol.~\%.  Further experimental details may be found in White et al.~\cite{White:2015}. FDS simulations are performed at three grid resolutions corresponding to $W/\delta x = 4, 8, 16$, where $W = 5$ cm is the width of the line burner (see Fig.~\ref{fig:umd_line_burner_plan_view}).

\begin{figure}[h]
\begin{tabular*}{\textwidth}{l@{\extracolsep{\fill}}r}
\includegraphics[height=2.15in]{SCRIPT_FIGURES/UMD_Line_Burner/methane_O2_p18_O2_z_p125} &
\includegraphics[height=2.15in]{SCRIPT_FIGURES/UMD_Line_Burner/methane_O2_p18_O2_z_p250}
\end{tabular*}
\caption[UMD\_Line\_Burner oxygen concentration profiles]
{Measured and predicted mean oxygen concentration profiles at 18 vol \% O$_2$.}
\label{UMD_Line_Burner_methane_O2_p18_O2}
\end{figure}

\clearpage


\subsection{Summary of Major Combustion Products Predictions}
\label{Carbon Dioxide Concentration}
\label{Oxygen Concentration}


\begin{figure}[!h]
\centering
\begin{tabular}{c}
\includegraphics[height=4in]{SCRIPT_FIGURES/ScatterPlots/FDS_Carbon_Dioxide_Concentration} \\
\includegraphics[height=4in]{SCRIPT_FIGURES/ScatterPlots/FDS_Oxygen_Concentration}
\end{tabular}
\caption[Summary of major gas species predictions]
{Summary of major gas species predictions.}
\end{figure}

\clearpage


\section{Smoke}

\subsection{FM Burner Experiments}
\label{FM_Burner_Soot}

Mean and rms soot volume fraction measurements were made above a 13.7~cm (inner) diameter, 15 kW ethylene burner. Figures~\ref{FM_Burner_Soot_1} through \ref{FM_Burner_Soot_6} display the radial profiles located at heights of 0.5, 1.0, 1.5, 2.5, and 3.5 burner diameters, $D$. Figures~\ref{FM_Burner_Soot_PDF_20p9} through \ref{FM_Burner_Soot_PDF_15p2} display the probability distributions (PDFs) at the five heights and radii of 0~cm, 2~cm, 4~cm, 6~cm, and 8~cm.

\begin{figure}[!h]
\begin{tabular*}{\textwidth}{l@{\extracolsep{\fill}}r}
\includegraphics[height=2.15in]{SCRIPT_FIGURES/FM_Burner/LII_Radial_Mean_Soot_0p5D_20p9} &
\includegraphics[height=2.15in]{SCRIPT_FIGURES/FM_Burner/LII_Radial_RMS_Soot_0p5D_20p9} \\
\includegraphics[height=2.15in]{SCRIPT_FIGURES/FM_Burner/LII_Radial_Mean_Soot_1p0D_20p9} &
\includegraphics[height=2.15in]{SCRIPT_FIGURES/FM_Burner/LII_Radial_RMS_Soot_1p0D_20p9} \\
\includegraphics[height=2.15in]{SCRIPT_FIGURES/FM_Burner/LII_Radial_Mean_Soot_1p5D_20p9} &
\includegraphics[height=2.15in]{SCRIPT_FIGURES/FM_Burner/LII_Radial_RMS_Soot_1p5D_20p9}
\end{tabular*}
\caption[FM Burner experiments, plume mean and rms soot volume fraction, 20.9~\% O$_2$]
{FM Burner experiments, plume mean and rms soot volume fraction at heights of 0.5, 1.0, and 1.5 burner diameters, D, 20.9~\% O$_2$.}
\label{FM_Burner_Soot_1}
\end{figure}

\newpage

\begin{figure}[p]
\begin{tabular*}{\textwidth}{l@{\extracolsep{\fill}}r}
\includegraphics[height=2.15in]{SCRIPT_FIGURES/FM_Burner/LII_Radial_Mean_Soot_2p5D_20p9} &
\includegraphics[height=2.15in]{SCRIPT_FIGURES/FM_Burner/LII_Radial_RMS_Soot_2p5D_20p9} \\
\includegraphics[height=2.15in]{SCRIPT_FIGURES/FM_Burner/LII_Radial_Mean_Soot_3p5D_20p9} &
\includegraphics[height=2.15in]{SCRIPT_FIGURES/FM_Burner/LII_Radial_RMS_Soot_3p5D_20p9}
\end{tabular*}
\caption[FM Burner experiments, plume mean and rms soot volume fraction, 20.9~\% O$_2$]
{FM Burner experiments, plume mean and rms soot volume fraction at heights of 2.5 and 3.5 burner diameters, D, 20.9~\% O$_2$.}
\label{FM_Burner_Soot_2}
\end{figure}

\begin{figure}[p]
\begin{tabular*}{\textwidth}{l@{\extracolsep{\fill}}r}
\includegraphics[height=2.15in]{SCRIPT_FIGURES/FM_Burner/LII_Radial_Mean_Soot_0p5D_16p8} &
\includegraphics[height=2.15in]{SCRIPT_FIGURES/FM_Burner/LII_Radial_RMS_Soot_0p5D_16p8} \\
\includegraphics[height=2.15in]{SCRIPT_FIGURES/FM_Burner/LII_Radial_Mean_Soot_1p0D_16p8} &
\includegraphics[height=2.15in]{SCRIPT_FIGURES/FM_Burner/LII_Radial_RMS_Soot_1p0D_16p8} \\
\includegraphics[height=2.15in]{SCRIPT_FIGURES/FM_Burner/LII_Radial_Mean_Soot_1p5D_16p8} &
\includegraphics[height=2.15in]{SCRIPT_FIGURES/FM_Burner/LII_Radial_RMS_Soot_1p5D_16p8}
\end{tabular*}
\caption[FM Burner experiments, plume mean and rms soot volume fraction, 16.8~\% O$_2$]
{FM Burner experiments, plume mean and rms soot volume fraction at heights of 0.5, 1.0, and 1.5 burner diameters, D, 16.8~\% O$_2$.}
\label{FM_Burner_Soot_3}
\end{figure}

\begin{figure}[p]
\begin{tabular*}{\textwidth}{l@{\extracolsep{\fill}}r}
\includegraphics[height=2.15in]{SCRIPT_FIGURES/FM_Burner/LII_Radial_Mean_Soot_2p5D_16p8} &
\includegraphics[height=2.15in]{SCRIPT_FIGURES/FM_Burner/LII_Radial_RMS_Soot_2p5D_16p8} \\
\includegraphics[height=2.15in]{SCRIPT_FIGURES/FM_Burner/LII_Radial_Mean_Soot_3p5D_16p8} &
\includegraphics[height=2.15in]{SCRIPT_FIGURES/FM_Burner/LII_Radial_RMS_Soot_3p5D_16p8}
\end{tabular*}
\caption[FM Burner experiments, plume mean and rms soot volume fraction, 16.8~\% O$_2$]
{FM Burner experiments, plume mean and rms soot volume fraction at heights of 2.5 and 3.5 burner diameters, D, 16.8~\% O$_2$.}
\label{FM_Burner_Soot_4}
\end{figure}

\begin{figure}[p]
\begin{tabular*}{\textwidth}{l@{\extracolsep{\fill}}r}
\includegraphics[height=2.15in]{SCRIPT_FIGURES/FM_Burner/LII_Radial_Mean_Soot_0p5D_15p2} &
\includegraphics[height=2.15in]{SCRIPT_FIGURES/FM_Burner/LII_Radial_RMS_Soot_0p5D_15p2} \\
\includegraphics[height=2.15in]{SCRIPT_FIGURES/FM_Burner/LII_Radial_Mean_Soot_1p0D_15p2} &
\includegraphics[height=2.15in]{SCRIPT_FIGURES/FM_Burner/LII_Radial_RMS_Soot_1p0D_15p2} \\
\includegraphics[height=2.15in]{SCRIPT_FIGURES/FM_Burner/LII_Radial_Mean_Soot_1p5D_15p2} &
\includegraphics[height=2.15in]{SCRIPT_FIGURES/FM_Burner/LII_Radial_RMS_Soot_1p5D_15p2}
\end{tabular*}
\caption[FM Burner experiments, plume mean and rms soot volume fraction, 15.2~\% O$_2$]
{FM Burner experiments, plume mean and rms soot volume fraction at heights of 0.5, 1.0, and 1.5 burner diameters, D, 15.2~\% O$_2$.}
\label{FM_Burner_Soot_5}
\end{figure}

\begin{figure}[p]
\begin{tabular*}{\textwidth}{l@{\extracolsep{\fill}}r}
\includegraphics[height=2.15in]{SCRIPT_FIGURES/FM_Burner/LII_Radial_Mean_Soot_2p5D_15p2} &
\includegraphics[height=2.15in]{SCRIPT_FIGURES/FM_Burner/LII_Radial_RMS_Soot_2p5D_15p2} \\
\includegraphics[height=2.15in]{SCRIPT_FIGURES/FM_Burner/LII_Radial_Mean_Soot_3p5D_15p2} &
\includegraphics[height=2.15in]{SCRIPT_FIGURES/FM_Burner/LII_Radial_RMS_Soot_3p5D_15p2}
\end{tabular*}
\caption[FM Burner experiments, plume mean and rms soot volume fraction, 15.2~\% O$_2$]
{FM Burner experiments, plume mean and rms soot volume fraction at heights of 2.5 and 3.5 burner diameters, D, 15.2~\% O$_2$.}
\label{FM_Burner_Soot_6}
\end{figure}


\begin{sidewaysfigure}[p]
\begin{tabular*}{\textwidth}{ccccc}
\includegraphics[width=1.5in]{SCRIPT_FIGURES/FM_Burner/LII_Soot_PDF_0p5D_r=0cm_20p9} &
\includegraphics[width=1.5in]{SCRIPT_FIGURES/FM_Burner/LII_Soot_PDF_0p5D_r=2cm_20p9} &
\includegraphics[width=1.5in]{SCRIPT_FIGURES/FM_Burner/LII_Soot_PDF_0p5D_r=4cm_20p9} &
\includegraphics[width=1.5in]{SCRIPT_FIGURES/FM_Burner/LII_Soot_PDF_0p5D_r=6cm_20p9} &
\includegraphics[width=1.5in]{SCRIPT_FIGURES/FM_Burner/LII_Soot_PDF_0p5D_r=8cm_20p9} \\
\includegraphics[width=1.5in]{SCRIPT_FIGURES/FM_Burner/LII_Soot_PDF_1p0D_r=0cm_20p9} &
\includegraphics[width=1.5in]{SCRIPT_FIGURES/FM_Burner/LII_Soot_PDF_1p0D_r=2cm_20p9} &
\includegraphics[width=1.5in]{SCRIPT_FIGURES/FM_Burner/LII_Soot_PDF_1p0D_r=4cm_20p9} &
\includegraphics[width=1.5in]{SCRIPT_FIGURES/FM_Burner/LII_Soot_PDF_1p0D_r=6cm_20p9} &
\includegraphics[width=1.5in]{SCRIPT_FIGURES/FM_Burner/LII_Soot_PDF_1p0D_r=8cm_20p9} \\
\includegraphics[width=1.5in]{SCRIPT_FIGURES/FM_Burner/LII_Soot_PDF_1p5D_r=0cm_20p9} &
\includegraphics[width=1.5in]{SCRIPT_FIGURES/FM_Burner/LII_Soot_PDF_1p5D_r=2cm_20p9} &
\includegraphics[width=1.5in]{SCRIPT_FIGURES/FM_Burner/LII_Soot_PDF_1p5D_r=4cm_20p9} &
\includegraphics[width=1.5in]{SCRIPT_FIGURES/FM_Burner/LII_Soot_PDF_1p5D_r=6cm_20p9} &
\includegraphics[width=1.5in]{SCRIPT_FIGURES/FM_Burner/LII_Soot_PDF_1p5D_r=8cm_20p9} \\
\includegraphics[width=1.5in]{SCRIPT_FIGURES/FM_Burner/LII_Soot_PDF_2p5D_r=0cm_20p9} &
\includegraphics[width=1.5in]{SCRIPT_FIGURES/FM_Burner/LII_Soot_PDF_2p5D_r=2cm_20p9} &
\includegraphics[width=1.5in]{SCRIPT_FIGURES/FM_Burner/LII_Soot_PDF_2p5D_r=4cm_20p9} &
\includegraphics[width=1.5in]{SCRIPT_FIGURES/FM_Burner/LII_Soot_PDF_2p5D_r=6cm_20p9} &
\includegraphics[width=1.5in]{SCRIPT_FIGURES/FM_Burner/LII_Soot_PDF_1p5D_r=8cm_20p9} \\
\includegraphics[width=1.5in]{SCRIPT_FIGURES/FM_Burner/LII_Soot_PDF_3p5D_r=0cm_20p9} &
\includegraphics[width=1.5in]{SCRIPT_FIGURES/FM_Burner/LII_Soot_PDF_3p5D_r=2cm_20p9} &
\includegraphics[width=1.5in]{SCRIPT_FIGURES/FM_Burner/LII_Soot_PDF_3p5D_r=4cm_20p9} &
\includegraphics[width=1.5in]{SCRIPT_FIGURES/FM_Burner/LII_Soot_PDF_3p5D_r=6cm_20p9} &
\includegraphics[width=1.5in]{SCRIPT_FIGURES/FM_Burner/LII_Soot_PDF_3p5D_r=8cm_20p9} \\
\end{tabular*}
\caption[FM Burner experiments, soot volume fraction PDFs, 20.9~\% O$_2$]
{FM Burner experiments, soot volume fraction PDFs, 20.9~\% O$_2$.}
\label{FM_Burner_Soot_PDF_20p9}
\end{sidewaysfigure}

\begin{sidewaysfigure}[p]
\begin{tabular*}{\textwidth}{ccccc}
\includegraphics[width=1.5in]{SCRIPT_FIGURES/FM_Burner/LII_Soot_PDF_0p5D_r=0cm_16p8} &
\includegraphics[width=1.5in]{SCRIPT_FIGURES/FM_Burner/LII_Soot_PDF_0p5D_r=2cm_16p8} &
\includegraphics[width=1.5in]{SCRIPT_FIGURES/FM_Burner/LII_Soot_PDF_0p5D_r=4cm_16p8} &
\includegraphics[width=1.5in]{SCRIPT_FIGURES/FM_Burner/LII_Soot_PDF_0p5D_r=6cm_16p8} &
\includegraphics[width=1.5in]{SCRIPT_FIGURES/FM_Burner/LII_Soot_PDF_0p5D_r=8cm_16p8} \\
\includegraphics[width=1.5in]{SCRIPT_FIGURES/FM_Burner/LII_Soot_PDF_1p0D_r=0cm_16p8} &
\includegraphics[width=1.5in]{SCRIPT_FIGURES/FM_Burner/LII_Soot_PDF_1p0D_r=2cm_16p8} &
\includegraphics[width=1.5in]{SCRIPT_FIGURES/FM_Burner/LII_Soot_PDF_1p0D_r=4cm_16p8} &
\includegraphics[width=1.5in]{SCRIPT_FIGURES/FM_Burner/LII_Soot_PDF_1p0D_r=6cm_16p8} &
\includegraphics[width=1.5in]{SCRIPT_FIGURES/FM_Burner/LII_Soot_PDF_1p0D_r=8cm_16p8} \\
\includegraphics[width=1.5in]{SCRIPT_FIGURES/FM_Burner/LII_Soot_PDF_1p5D_r=0cm_16p8} &
\includegraphics[width=1.5in]{SCRIPT_FIGURES/FM_Burner/LII_Soot_PDF_1p5D_r=2cm_16p8} &
\includegraphics[width=1.5in]{SCRIPT_FIGURES/FM_Burner/LII_Soot_PDF_1p5D_r=4cm_16p8} &
\includegraphics[width=1.5in]{SCRIPT_FIGURES/FM_Burner/LII_Soot_PDF_1p5D_r=6cm_16p8} &
\includegraphics[width=1.5in]{SCRIPT_FIGURES/FM_Burner/LII_Soot_PDF_1p5D_r=8cm_16p8} \\
\includegraphics[width=1.5in]{SCRIPT_FIGURES/FM_Burner/LII_Soot_PDF_2p5D_r=0cm_16p8} &
\includegraphics[width=1.5in]{SCRIPT_FIGURES/FM_Burner/LII_Soot_PDF_2p5D_r=2cm_16p8} &
\includegraphics[width=1.5in]{SCRIPT_FIGURES/FM_Burner/LII_Soot_PDF_2p5D_r=4cm_16p8} &
\includegraphics[width=1.5in]{SCRIPT_FIGURES/FM_Burner/LII_Soot_PDF_2p5D_r=6cm_16p8} &
\includegraphics[width=1.5in]{SCRIPT_FIGURES/FM_Burner/LII_Soot_PDF_1p5D_r=8cm_16p8} \\
\includegraphics[width=1.5in]{SCRIPT_FIGURES/FM_Burner/LII_Soot_PDF_3p5D_r=0cm_16p8} &
\includegraphics[width=1.5in]{SCRIPT_FIGURES/FM_Burner/LII_Soot_PDF_3p5D_r=2cm_16p8} &
\includegraphics[width=1.5in]{SCRIPT_FIGURES/FM_Burner/LII_Soot_PDF_3p5D_r=4cm_16p8} &
\includegraphics[width=1.5in]{SCRIPT_FIGURES/FM_Burner/LII_Soot_PDF_3p5D_r=6cm_16p8} &
\includegraphics[width=1.5in]{SCRIPT_FIGURES/FM_Burner/LII_Soot_PDF_3p5D_r=8cm_16p8} \\
\end{tabular*}
\caption[FM Burner experiments, soot volume fraction PDFs, 16.8~\% O$_2$]
{FM Burner experiments, soot volume fraction PDFs, 16.8~\% O$_2$.}
\label{FM_Burner_Soot_PDF_16p8}
\end{sidewaysfigure}

\begin{sidewaysfigure}[p]
\begin{tabular*}{\textwidth}{ccccc}
\includegraphics[width=1.5in]{SCRIPT_FIGURES/FM_Burner/LII_Soot_PDF_0p5D_r=0cm_15p2} &
\includegraphics[width=1.5in]{SCRIPT_FIGURES/FM_Burner/LII_Soot_PDF_0p5D_r=2cm_15p2} &
\includegraphics[width=1.5in]{SCRIPT_FIGURES/FM_Burner/LII_Soot_PDF_0p5D_r=4cm_15p2} &
\includegraphics[width=1.5in]{SCRIPT_FIGURES/FM_Burner/LII_Soot_PDF_0p5D_r=6cm_15p2} &
\includegraphics[width=1.5in]{SCRIPT_FIGURES/FM_Burner/LII_Soot_PDF_0p5D_r=8cm_15p2} \\
\includegraphics[width=1.5in]{SCRIPT_FIGURES/FM_Burner/LII_Soot_PDF_1p0D_r=0cm_15p2} &
\includegraphics[width=1.5in]{SCRIPT_FIGURES/FM_Burner/LII_Soot_PDF_1p0D_r=2cm_15p2} &
\includegraphics[width=1.5in]{SCRIPT_FIGURES/FM_Burner/LII_Soot_PDF_1p0D_r=4cm_15p2} &
\includegraphics[width=1.5in]{SCRIPT_FIGURES/FM_Burner/LII_Soot_PDF_1p0D_r=6cm_15p2} &
\includegraphics[width=1.5in]{SCRIPT_FIGURES/FM_Burner/LII_Soot_PDF_1p0D_r=8cm_15p2} \\
\includegraphics[width=1.5in]{SCRIPT_FIGURES/FM_Burner/LII_Soot_PDF_1p5D_r=0cm_15p2} &
\includegraphics[width=1.5in]{SCRIPT_FIGURES/FM_Burner/LII_Soot_PDF_1p5D_r=2cm_15p2} &
\includegraphics[width=1.5in]{SCRIPT_FIGURES/FM_Burner/LII_Soot_PDF_1p5D_r=4cm_15p2} &
\includegraphics[width=1.5in]{SCRIPT_FIGURES/FM_Burner/LII_Soot_PDF_1p5D_r=6cm_15p2} &
\includegraphics[width=1.5in]{SCRIPT_FIGURES/FM_Burner/LII_Soot_PDF_1p5D_r=8cm_15p2} \\
\includegraphics[width=1.5in]{SCRIPT_FIGURES/FM_Burner/LII_Soot_PDF_2p5D_r=0cm_15p2} &
\includegraphics[width=1.5in]{SCRIPT_FIGURES/FM_Burner/LII_Soot_PDF_2p5D_r=2cm_15p2} &
\includegraphics[width=1.5in]{SCRIPT_FIGURES/FM_Burner/LII_Soot_PDF_2p5D_r=4cm_15p2} &
\includegraphics[width=1.5in]{SCRIPT_FIGURES/FM_Burner/LII_Soot_PDF_2p5D_r=6cm_15p2} &
\includegraphics[width=1.5in]{SCRIPT_FIGURES/FM_Burner/LII_Soot_PDF_1p5D_r=8cm_15p2} \\
\includegraphics[width=1.5in]{SCRIPT_FIGURES/FM_Burner/LII_Soot_PDF_3p5D_r=0cm_15p2} &
\includegraphics[width=1.5in]{SCRIPT_FIGURES/FM_Burner/LII_Soot_PDF_3p5D_r=2cm_15p2} &
\includegraphics[width=1.5in]{SCRIPT_FIGURES/FM_Burner/LII_Soot_PDF_3p5D_r=4cm_15p2} &
\includegraphics[width=1.5in]{SCRIPT_FIGURES/FM_Burner/LII_Soot_PDF_3p5D_r=6cm_15p2} &
\includegraphics[width=1.5in]{SCRIPT_FIGURES/FM_Burner/LII_Soot_PDF_3p5D_r=8cm_15p2} \\
\end{tabular*}
\caption[FM Burner experiments, soot volume fraction PDFs, 15.2~\% O$_2$]
{FM Burner experiments, soot volume fraction PDFs, 15.2~\% O$_2$.}
\label{FM_Burner_Soot_PDF_15p2}
\end{sidewaysfigure}



\clearpage


\subsection{FM/FPRF Data Center Experiments}
\label{FM Smoke Concentration}

Results of the low exhaust rate (78 ACH) and high exhaust rate (265 ACH) tests for propylene and cables is shown in the figure below. Each test had three measurement locations (subfloor, ceiling, and ceiling plenum); however, not all locations for all tests had a measurement above background noise in the laser signal.

\begin{figure}[!h]
\begin{tabular*}{\textwidth}{l@{\extracolsep{\fill}}r}
\includegraphics[height=2.15in]{SCRIPT_FIGURES/FM_FPRF_Datacenter/FM_Datacenter_Low_C3H6_SF} &
\includegraphics[height=2.15in]{SCRIPT_FIGURES/FM_FPRF_Datacenter/FM_Datacenter_High_C3H6_SF} \\
\includegraphics[height=2.15in]{SCRIPT_FIGURES/FM_FPRF_Datacenter/FM_Datacenter_Low_C3H6_HA} &
\includegraphics[height=2.15in]{SCRIPT_FIGURES/FM_FPRF_Datacenter/FM_Datacenter_High_C3H6_HA} \\
\includegraphics[height=2.15in]{SCRIPT_FIGURES/FM_FPRF_Datacenter/FM_Datacenter_Low_Cable_SF} &
\includegraphics[height=2.15in]{SCRIPT_FIGURES/FM_FPRF_Datacenter/FM_Datacenter_High_Cable_SF}
\end{tabular*}
\caption[FM/FPRF Data Center, smoke concentration, propylene and cable sources]{FM/FPRF Data Center, smoke concentration, low and high exhaust rate, propylene and cable sources.}
\label{FM_FPRF_Datacenter_Smoke}
\end{figure}

\clearpage

\subsection{NIST/NRC Experiments}
\label{Smoke Concentration}

The figures on the following pages contain comparisons of measured and predicted smoke concentration at one measuring station in the upper layer. The experiments are divided into two. Figure~\ref{smoke_closed} displays the results of the experiments in which the compartment door was closed. Figure~\ref{smoke_open} displays the results of the experiments in which the compartment door was open. The results are significantly different---the simulations of the closed door experiments predict smoke concentrations that are more than twice that of the experiments. The reason for this is that FDS has not accounted for the agglomeration and deposition of the smoke on the walls, something that is far more important when the compartment is not well-ventilated.

\newpage

\begin{figure}[p]
\begin{tabular*}{\textwidth}{l@{\extracolsep{\fill}}r}
\includegraphics[height=2.15in]{SCRIPT_FIGURES/NIST_NRC/NIST_NRC_01_Smoke} &
\includegraphics[height=2.15in]{SCRIPT_FIGURES/NIST_NRC/NIST_NRC_07_Smoke} \\
\includegraphics[height=2.15in]{SCRIPT_FIGURES/NIST_NRC/NIST_NRC_02_Smoke} &
\includegraphics[height=2.15in]{SCRIPT_FIGURES/NIST_NRC/NIST_NRC_08_Smoke} \\
\includegraphics[height=2.15in]{SCRIPT_FIGURES/NIST_NRC/NIST_NRC_04_Smoke} &
\includegraphics[height=2.15in]{SCRIPT_FIGURES/NIST_NRC/NIST_NRC_10_Smoke} \\
\includegraphics[height=2.15in]{SCRIPT_FIGURES/NIST_NRC/NIST_NRC_13_Smoke} &
\includegraphics[height=2.15in]{SCRIPT_FIGURES/NIST_NRC/NIST_NRC_16_Smoke}
\end{tabular*}
\caption[NIST/NRC experiments, smoke concentration, closed door experiments]{NIST/NRC experiments, smoke concentration, closed door experiments.}
\label{smoke_closed}
\end{figure}

\begin{figure}[p]
\begin{tabular*}{\textwidth}{l@{\extracolsep{\fill}}r}
\includegraphics[height=2.15in]{SCRIPT_FIGURES/NIST_NRC/NIST_NRC_03_Smoke} &
\includegraphics[height=2.15in]{SCRIPT_FIGURES/NIST_NRC/NIST_NRC_09_Smoke} \\
\includegraphics[height=2.15in]{SCRIPT_FIGURES/NIST_NRC/NIST_NRC_05_Smoke} &
\includegraphics[height=2.15in]{SCRIPT_FIGURES/NIST_NRC/NIST_NRC_14_Smoke} \\
\includegraphics[height=2.15in]{SCRIPT_FIGURES/NIST_NRC/NIST_NRC_15_Smoke} &
\includegraphics[height=2.15in]{SCRIPT_FIGURES/NIST_NRC/NIST_NRC_18_Smoke}
\end{tabular*}
\caption[NIST/NRC experiments, smoke concentration, open door experiments]{NIST/NRC experiments, smoke concentration, open door experiments.}
\label{smoke_open}
\end{figure}


\begin{figure}[p]
\begin{center}
\begin{tabular}{c}
\includegraphics[height=4.0in]{SCRIPT_FIGURES/ScatterPlots/FDS_Smoke_Concentration}
\end{tabular}
\end{center}
\caption[Summary of smoke concentration predictions]{Summary of smoke concentration predictions.}
\end{figure}

\clearpage

\subsection{FAA Cargo Compartments}
\label{Smoke Obscuration}

Beam obscuration measurements were made at different locations within the compartment (see Fig.~\ref{FAA_Cargo_probe_locations}). The data is presented below in terms of percent transmission per meter, $100(I/I_0)^{1/L}$, where $I$ is the light intensity and $L$ is the beam pathlength in units of meters.

\begin{figure}[h]
\begin{tabular*}{\textwidth}{l@{\extracolsep{\fill}}r}
\includegraphics[height=2.15in]{SCRIPT_FIGURES/FAA_Cargo_Compartments/FAA_Cargo_Compartments_Test_1_Ceiling_Transmission} &
\includegraphics[height=2.15in]{SCRIPT_FIGURES/FAA_Cargo_Compartments/FAA_Cargo_Compartments_Test_1_Cargo_Transmission} \\
\includegraphics[height=2.15in]{SCRIPT_FIGURES/FAA_Cargo_Compartments/FAA_Cargo_Compartments_Test_2_Ceiling_Transmission} &
\includegraphics[height=2.15in]{SCRIPT_FIGURES/FAA_Cargo_Compartments/FAA_Cargo_Compartments_Test_2_Cargo_Transmission} \\
\includegraphics[height=2.15in]{SCRIPT_FIGURES/FAA_Cargo_Compartments/FAA_Cargo_Compartments_Test_3_Ceiling_Transmission} &
\includegraphics[height=2.15in]{SCRIPT_FIGURES/FAA_Cargo_Compartments/FAA_Cargo_Compartments_Test_3_Cargo_Transmission}
\end{tabular*}
\caption[FAA Cargo Compartments experiments, smoke obscuration]{FAA Cargo Compartments experiments, smoke obscuration.}
\end{figure}

\newpage

\begin{figure}[p]
\begin{center}
\begin{tabular}{c}
\includegraphics[height=4.0in]{SCRIPT_FIGURES/ScatterPlots/FDS_Smoke_Obscuration}
\end{tabular}
\end{center}
\caption[Summary of smoke obscuration predictions]{Summary of smoke obscuration predictions.}
\end{figure}


\clearpage

\section{Aerosols}

\subsection{Sippola Aerosol Deposition Experiments}
\label{Aerosol Deposition Velocity}

Figure~\ref{Sippola_Aerosol_Deposition_Velocity} compares the measured and predicted aerosol deposition velocities in the Sippola experiments, and Figure~\ref{Summary_Aerosol_Deposition_Velocity} shows a summary of the results. Details of the experiment and simulation are found in Sec.~\ref{Sippola_Aerosol_Deposition_Description}

\begin{figure}[!ht]
\begin{center}
\begin{tabular}{c}
\includegraphics[height=2.15in]{SCRIPT_FIGURES/Sippola_Aerosol_Deposition/Sippola_Aerosol_Ceiling_Deposition} \\
\includegraphics[height=2.15in]{SCRIPT_FIGURES/Sippola_Aerosol_Deposition/Sippola_Aerosol_Wall_Deposition} \\
\includegraphics[height=2.15in]{SCRIPT_FIGURES/Sippola_Aerosol_Deposition/Sippola_Aerosol_Floor_Deposition}
\end{tabular}
\end{center}
\caption[Predicted and measured aerosol deposition velocities, Sippola experiments]
{Predicted and measured aerosol deposition velocities, Sippola experiments.}
\label{Sippola_Aerosol_Deposition_Velocity}
\end{figure}

\begin{figure}[!ht]
\begin{center}
\begin{tabular}{c}
\includegraphics[height=4.0in]{SCRIPT_FIGURES/ScatterPlots/FDS_Aerosol_Deposition_Velocity} \\
\vspace{0.25in} \\
\end{tabular}
\end{center}
\caption[Summary of aerosol deposition velocity predictions]
{Summary of aerosol deposition velocity predictions.}
\label{Summary_Aerosol_Deposition_Velocity}
\end{figure}

\subsection{NIST Soot Deposition Gauge Experiments}
\label{Aerosol Deposition}

Figure~\ref{NIST Soot Deposition Gauge deposited mass} compares the measured and predicted aerosol mass deposition velocities in the NIST Soot Deposition Gauge experiments, and Figure~\ref{Summary_NIST_SDG} shows a summary of the results. Details of the experiment and simulation are found in Sec.~\ref{NIST_SDG_Description}

\begin{figure}[!ht]
   \begin{center}
      \begin{tabular}{c}
         \includegraphics[height=2.15in]{SCRIPT_FIGURES/NIST_Deposition_Gauge/NIST_SDG_100} \\
         \includegraphics[height=2.15in]{SCRIPT_FIGURES/NIST_Deposition_Gauge/NIST_SDG_200}
      \end{tabular}
   \end{center}
   \caption[Predicted and measured aerosol deposited mass, NIST Soot Deposition Gauge exp]
   {Predicted and measured aerosol deposited mass, NIST Soot Deposition Gauge experiments.}
   \label{NIST Soot Deposition Gauge deposited mass}
\end{figure}

\begin{figure}[!ht]
   \begin{center}
      \begin{tabular}{c}
         \includegraphics[height=4.0in]{SCRIPT_FIGURES/ScatterPlots/FDS_Aerosol_Deposition} \\
         \vspace{0.25in} \\
      \end{tabular}
   \end{center}
   \caption[Summary of aerosol deposited mass predictions]
   {Summary of aerosol deposited mass predictions.}
   \label{Summary_NIST_SDG}
\end{figure}

\clearpage

\section{Droplet Evaporation}

This section presents the results of simulations of liquid droplet evaporation experiments. The titles of the sections below are named for the experimentalists.

\subsection{Ranz and Marshall}

A description of the experiments is included in Sec.~\ref{Ranz_Marshall_Description}. Figure~\ref{RM_Fig_8} shows the results of predicting the drop diameter of a single droplet evaporating in dry air.

\begin{figure}[!h]
    \centering
    \includegraphics[height=2.15in]{SCRIPT_FIGURES/Ranz_Marshall/Ranz_Marshall_Time_Dep}
    \caption[Droplet diameter for the Ranz and Marshall experiment]
    {Measured and predicted droplet diameter for the Ranz and Marshall experiment shown in Fig.~8 of \cite{Ranz}.}
    \label{RM_Fig_8}
\end{figure}

Figure~\ref{RM_Tables} compares the measured and predicted evaporation rates for Table 1 through Table 4 in \cite{Ranz}. The Table 1 experiments were water droplets in ambient air, the Table 2 experiments were water droplets at in warm air (less than boiling), the Table 3 experiments were water droplets in hot air (greater than boiling), and the Table 4 experiments were benzene droplets in ambient air.  A summary of the results is shown in Fig.~\ref{RM_Summary}.

\begin{figure}[!h]
   \begin{tabular*}{\textwidth}{l@{\extracolsep{\fill}}r}
      \includegraphics[height=2.15in]{SCRIPT_FIGURES/Ranz_Marshall/Ranz_Marshall_Table_1} &
      \includegraphics[height=2.15in]{SCRIPT_FIGURES/Ranz_Marshall/Ranz_Marshall_Table_2} \\
      \includegraphics[height=2.15in]{SCRIPT_FIGURES/Ranz_Marshall/Ranz_Marshall_Table_3} &
      \includegraphics[height=2.15in]{SCRIPT_FIGURES/Ranz_Marshall/Ranz_Marshall_Table_4}
   \end{tabular*}
   \caption[Evaporation rates for the Ranz and Marshall experiments]{Evaporation rates for the Ranz and Marshall experiments~\cite{Ranz}.}
   \label{RM_Tables}
\end{figure}

\begin{figure}[!ht]
   \begin{center}
      \begin{tabular}{c}
         \includegraphics[height=3.0in]{SCRIPT_FIGURES/ScatterPlots/FDS_Evaporation_Rate} \\
         \vspace{0.25in} \\
      \end{tabular}
   \end{center}
   \caption[Summary of evaporation rates for the Ranz and Marshall experiments]{Summary of evaporation rates for the Ranz and Marshall experiments~\cite{Ranz}.}
   \label{RM_Summary}
\end{figure}

\FloatBarrier

\subsection{Fujita et al.}

A description of the experiments is included in Sec.~\ref{Fujita_exp}. Figure~\ref{Fujita_drop_area} shows the droplet area normalized by its initial area, and Fig.~\ref{Fujita_drop_T} shows the droplet temperature change.

\begin{figure}[!h]
\begin{tabular*}{\textwidth}{l@{\extracolsep{\fill}}r}
\includegraphics[height=2.15in]{SCRIPT_FIGURES/Droplet_Evaporation/fujita_up8_hp0_d} &
\includegraphics[height=2.15in]{SCRIPT_FIGURES/Droplet_Evaporation/fujita_up8_hp3_d} \\
\includegraphics[height=2.15in]{SCRIPT_FIGURES/Droplet_Evaporation/fujita_u2p0_hp0_d} &
\includegraphics[height=2.15in]{SCRIPT_FIGURES/Droplet_Evaporation/fujita_u2p0_hp3_d}
\end{tabular*}
\caption[Normalized droplet area for the Fujita experiments]{Normalized droplet area for the Fujita experiments for varying Reynolds number (Re) and Humidity (H)}
\label{Fujita_drop_area}
\end{figure}

\begin{figure}[!ht]
\begin{tabular*}{\textwidth}{l@{\extracolsep{\fill}}r}
\includegraphics[height=2.15in]{SCRIPT_FIGURES/Droplet_Evaporation/fujita_up8_hp0_T} &
\includegraphics[height=2.15in]{SCRIPT_FIGURES/Droplet_Evaporation/fujita_up8_hp3_T} \\
\includegraphics[height=2.15in]{SCRIPT_FIGURES/Droplet_Evaporation/fujita_u2p0_hp0_T} &
\includegraphics[height=2.15in]{SCRIPT_FIGURES/Droplet_Evaporation/fujita_u2p0_hp3_T}
\end{tabular*}
\caption[Droplet temperature change, Fujita et al. experiments]{Droplet temperature change for the Fujita experiments for varying Reynolds number (Re) and Humidity (H).}
\label{Fujita_drop_T}
\end{figure}

\FloatBarrier

\subsection{Gavin}

A description of the experiments is included in Sec.~\ref{Gavin_exp}. Figure~\ref{Gavin_plots} shows the simulation results for the droplet terminal velocity.

\begin{figure}[!h]
\begin{tabular*}{\textwidth}{l@{\extracolsep{\fill}}r}
\includegraphics[height=2.15in]{SCRIPT_FIGURES/Droplet_Evaporation/gavin_557} &
\includegraphics[height=2.15in]{SCRIPT_FIGURES/Droplet_Evaporation/gavin_769}
\end{tabular*}
\caption[Droplet terminal velocity for the Gavin experiments]{Droplet terminal velocity for the Gavin experiments.}
\label{Gavin_plots}
\end{figure}

\FloatBarrier

\subsection{Kolaitis and Founti}

A description of the experiments is included in Sec.~\ref{Kolaitis_exp}. Figure~\ref{Kolaitis_d2} shows the simulation results for the squared droplet diameter. Figure~\ref{Kolaitis_T} shows the simulation results for the droplet surface temperature.

\begin{figure}[!h]
   \begin{tabular*}{\textwidth}{l@{\extracolsep{\fill}}r}
      \includegraphics[height=2.15in]{SCRIPT_FIGURES/Droplet_Evaporation/kolaitis_decane_d2} &
      \includegraphics[height=2.15in]{SCRIPT_FIGURES/Droplet_Evaporation/kolaitis_ethanol_d2} \\
      \includegraphics[height=2.15in]{SCRIPT_FIGURES/Droplet_Evaporation/kolaitis_heptane_1_d2} &
      \includegraphics[height=2.15in]{SCRIPT_FIGURES/Droplet_Evaporation/kolaitis_heptane_2_d2}
   \end{tabular*}
   \caption[Square of the droplet diameter, Kolaitis and Founti]{Square of the droplet diameter, Kolaitis and Founti.}
   \label{Kolaitis_d2}
\end{figure}

\begin{figure}[!ht]
   \begin{tabular*}{\textwidth}{l@{\extracolsep{\fill}}r}
      \includegraphics[height=2.15in]{SCRIPT_FIGURES/Droplet_Evaporation/kolaitis_decane_T} & \\
      \includegraphics[height=2.15in]{SCRIPT_FIGURES/Droplet_Evaporation/kolaitis_heptane_1_T} &
      \includegraphics[height=2.15in]{SCRIPT_FIGURES/Droplet_Evaporation/kolaitis_heptane_2_T}
   \end{tabular*}
   \caption[Droplet surface temperature, Kolaitis and Founti]{Droplet surface temperature, Kolaitis and Founti.}
   \label{Kolaitis_T}
\end{figure}

\FloatBarrier

\subsection{Maqua et al.}

A description of the experiments is included in Sec.~\ref{Maqua_exp}. Figure~\ref{Maqua_plots} shows the droplet surface temperature for the two experiments.

\begin{figure}[!ht]
\begin{tabular*}{\textwidth}{l@{\extracolsep{\fill}}r}
\includegraphics[height=2.15in]{SCRIPT_FIGURES/Droplet_Evaporation/maqua_acetone} &
\includegraphics[height=2.15in]{SCRIPT_FIGURES/Droplet_Evaporation/maqua_ethanol}
\end{tabular*}
\caption[Droplet surface temperature for the Maqua et al. experiments]{Droplet surface temperature for the Maqua et al. experiments.}
\label{Maqua_plots}
\end{figure}

\FloatBarrier

\subsection{Taflin Experiments}

A description of the experiments is included in Sec.~\ref{Taflin_exp}. Figure~\ref{Taflin_plots} shows the droplet diameter for the two Taflin experiments.

\begin{figure}[!h]
\begin{tabular*}{\textwidth}{l@{\extracolsep{\fill}}r}
\includegraphics[height=2.15in]{SCRIPT_FIGURES/Droplet_Evaporation/taflin_43p9} &
\includegraphics[height=2.15in]{SCRIPT_FIGURES/Droplet_Evaporation/taflin_56p6}
\end{tabular*}
\caption[Droplet diameter for the Taflin experiments]{Droplet diameter for the Taflin experiments.}
\label{Taflin_plots}
\end{figure}



\clearpage

\section{Products of Incomplete Combustion}

Predicting the concentration of products of incomplete combustion is challenging because it requires information about the chemical composition of the fuel and the multiple reactions that convert fuel to products. FDS contains a fairly general framework by which users can specify the reaction mechanism, and the examples in the following subsections highlight some of the more commonly used schemes.


\subsection{Smyth Slot Burner Experiment}
\label{Smyth_Reactions}

Figure~\ref{Smyth_Slot_Burner_temp} shows predicted and measured temperatures at three elevations above the Smyth slot burner (see Sec.~\ref{Smyth_Slot_Burner_Description} for details). Figures~\ref{Smyth_Slot_Burner_fuel_ox} through \ref{Smyth_Slot_Burner_h2o_h2} show predicted and measured concentrations of CH$_4$, O$_2$, CO, CO$_2$, H$_2$O and H$_2$ at the same three elevations. The reported uncertainty in the species concentration measurements ranges from 10~\% to 20~\%. The abbreviation ``FR'' in the labels means that the conversion of CO to CO$_2$ is modeled using a finite-rate, reversible reaction. The word ``Fast'' implies that this reaction is infinitely fast, but occurs following the first reaction, also infinitely fast.

\begin{figure}[!h]
\begin{tabular*}{\textwidth}{l@{\extracolsep{\fill}}r}
\includegraphics[height=2.15in]{SCRIPT_FIGURES/Smyth_Slot_Burner/Smyth_Slot_Burner_11mm_Temperature} &
\includegraphics[height=2.15in]{SCRIPT_FIGURES/Smyth_Slot_Burner/Smyth_Slot_Burner_9mm_Temperature} \\
\multicolumn{2}{c}{\includegraphics[height=2.15in]{SCRIPT_FIGURES/Smyth_Slot_Burner/Smyth_Slot_Burner_7mm_Temperature}}
\end{tabular*}
\caption[Temperature predictions at 7~mm, 9~mm, and 11~mm above burner, Smyth experiment]
{Predicted and measured temperature at 7~mm, 9~mm, and 11~mm above a methane-air slot burner.}
\label{Smyth_Slot_Burner_temp}
\end{figure}

\newpage

\begin{figure}[p]
\begin{tabular*}{\textwidth}{l@{\extracolsep{\fill}}r}
\includegraphics[height=2.15in]{SCRIPT_FIGURES/Smyth_Slot_Burner/Smyth_Slot_Burner_11mm_Fuel} &
\includegraphics[height=2.15in]{SCRIPT_FIGURES/Smyth_Slot_Burner/Smyth_Slot_Burner_11mm_Oxygen} \\
\includegraphics[height=2.15in]{SCRIPT_FIGURES/Smyth_Slot_Burner/Smyth_Slot_Burner_9mm_Fuel} &
\includegraphics[height=2.15in]{SCRIPT_FIGURES/Smyth_Slot_Burner/Smyth_Slot_Burner_9mm_Oxygen} \\
\includegraphics[height=2.15in]{SCRIPT_FIGURES/Smyth_Slot_Burner/Smyth_Slot_Burner_7mm_Fuel} &
\includegraphics[height=2.15in]{SCRIPT_FIGURES/Smyth_Slot_Burner/Smyth_Slot_Burner_7mm_Oxygen}
\end{tabular*}
\caption[CH$_4$ and O$_2$ volume fractions at 11~mm, 9~mm, and 7~mm above burner, Smyth burner]
{Predicted and measured CH$_4$ and O$_2$ volume fractions at 11~mm, 9~mm, and 7~mm above a methane-air slot burner.}
\label{Smyth_Slot_Burner_fuel_ox}
\end{figure}

\begin{figure}[p]
\begin{tabular*}{\textwidth}{l@{\extracolsep{\fill}}r}
\includegraphics[height=2.15in]{SCRIPT_FIGURES/Smyth_Slot_Burner/Smyth_Slot_Burner_11mm_Carbon_Dioxide} &
\includegraphics[height=2.15in]{SCRIPT_FIGURES/Smyth_Slot_Burner/Smyth_Slot_Burner_11mm_Carbon_Monoxide} \\
\includegraphics[height=2.15in]{SCRIPT_FIGURES/Smyth_Slot_Burner/Smyth_Slot_Burner_9mm_Carbon_Dioxide} &
\includegraphics[height=2.15in]{SCRIPT_FIGURES/Smyth_Slot_Burner/Smyth_Slot_Burner_9mm_Carbon_Monoxide} \\
\includegraphics[height=2.15in]{SCRIPT_FIGURES/Smyth_Slot_Burner/Smyth_Slot_Burner_7mm_Carbon_Dioxide} &
\includegraphics[height=2.15in]{SCRIPT_FIGURES/Smyth_Slot_Burner/Smyth_Slot_Burner_7mm_Carbon_Monoxide}
\end{tabular*}
\caption[CO$_2$ and CO volume fractions at 11~mm, 9~mm, and 7~mm above burner, Smyth burner]
{Predicted and measured CO$_2$ and CO volume fractions at 11~mm, 9~mm, and 7~mm above a methane-air slot burner.}
\label{Smyth_Slot_Burner_co_co2}
\end{figure}

\begin{figure}[p]
\begin{tabular*}{\textwidth}{l@{\extracolsep{\fill}}r}
\includegraphics[height=2.15in]{SCRIPT_FIGURES/Smyth_Slot_Burner/Smyth_Slot_Burner_11mm_Water} &
\includegraphics[height=2.15in]{SCRIPT_FIGURES/Smyth_Slot_Burner/Smyth_Slot_Burner_11mm_Hydrogen} \\
\includegraphics[height=2.15in]{SCRIPT_FIGURES/Smyth_Slot_Burner/Smyth_Slot_Burner_9mm_Water} &
\includegraphics[height=2.15in]{SCRIPT_FIGURES/Smyth_Slot_Burner/Smyth_Slot_Burner_9mm_Hydrogen} \\
\includegraphics[height=2.15in]{SCRIPT_FIGURES/Smyth_Slot_Burner/Smyth_Slot_Burner_7mm_Water} &
\includegraphics[height=2.15in]{SCRIPT_FIGURES/Smyth_Slot_Burner/Smyth_Slot_Burner_7mm_Hydrogen}
\end{tabular*}
\caption[H$_2$O and H$_2$ volume fractions at 11~mm, 9~mm, and 7~mm above burner, Smyth burner]
{Predicted and measured H$_2$O and H$_2$ volume fractions at 11~mm, 9~mm, and 7~mm above a methane-air slot burner.}
\label{Smyth_Slot_Burner_h2o_h2}
\end{figure}



\clearpage

\subsection{Beyler Hood Experiments}

Fig.~\ref{Beyler_Species} compares measured and predicted species mass fractions in the Beyler Hood experiments. Both measured and predicted values are time-averaged. The FDS results are taken at the extraction vent, whereas the measurements were made downstream of the vent. Details of the experiments can be found in Sec.~\ref{Beyler_Hood_Description}.



\begin{figure}[h]
\begin{tabular*}{\textwidth}{c@{\extracolsep{\fill}}c}
\includegraphics[height=2.8in]{SCRIPT_FIGURES/Beyler_Hood/Beyler_Hood_O2} &
\includegraphics[height=2.8in]{SCRIPT_FIGURES/Beyler_Hood/Beyler_Hood_CO2} \\
\includegraphics[height=2.8in]{SCRIPT_FIGURES/Beyler_Hood/Beyler_Hood_H2O} &
\includegraphics[height=2.8in]{SCRIPT_FIGURES/Beyler_Hood/Beyler_Hood_CO} \\
\includegraphics[height=2.8in]{SCRIPT_FIGURES/Beyler_Hood/Beyler_Hood_Soot} &
\includegraphics[height=2.8in]{SCRIPT_FIGURES/Beyler_Hood/Beyler_Hood_UHC}
\end{tabular*}
\caption[Summary of gas species predictions, Beyler hood experiments]
{Comparison of measured and predicted species concentrations in the Beyler hood experiments}
\label{Beyler_Species}
\end{figure}

\clearpage

\subsection{NIST Reduced Scale Enclosure (RSE) Experiments, 1994}
\label{sec:NIST_RSE_1994}

Figures~\ref{NIST_RSE_1994_CO} through \ref{NIST_RSE_1994_H2O} show the measured and predicted CO, CO$_2$ and O$_2$, and H$_2$O concentrations. Figure~\ref{NIST_RSE_1994_temp} shows the measured and predicted thermocouple temperatures in the front and rear of the compartment. The measurements were made 10~cm below the ceiling and 30~cm from the left side wall. The front position was 10~cm from the wall with the door; the back position was 30~cm from the rear wall. Details of the experiments are found in Sec.~\ref{NIST_RSE_1994_Description}.

\begin{figure}[!h]
\begin{tabular*}{\textwidth}{l@{\extracolsep{\fill}}r}
\includegraphics[height=2.15in]{SCRIPT_FIGURES/NIST_RSE_1994/NIST_RSE_1994_CO_Front} &
\includegraphics[height=2.15in]{SCRIPT_FIGURES/NIST_RSE_1994/NIST_RSE_1994_CO_Rear}
\end{tabular*}
\caption[Comparison of measured and predicted CO concentration, NIST RSE experiments]{Comparison of measured and predicted CO concentration, NIST RSE experiments.}
\label{NIST_RSE_1994_CO}
\end{figure}

\begin{figure}[!h]
\begin{tabular*}{\textwidth}{l@{\extracolsep{\fill}}r}
\includegraphics[height=2.15in]{SCRIPT_FIGURES/NIST_RSE_1994/NIST_RSE_1994_CO2_Front} &
\includegraphics[height=2.15in]{SCRIPT_FIGURES/NIST_RSE_1994/NIST_RSE_1994_CO2_Rear}
\end{tabular*}
\caption[Comparison of measured and predicted CO$_2$ concentration, NIST RSE experiments]{Comparison of measured and predicted CO$_2$ concentration, NIST RSE experiments.}
\label{NIST_RSE_1994_CO2}
\end{figure}

\newpage

\begin{figure}[!h]
\begin{tabular*}{\textwidth}{l@{\extracolsep{\fill}}r}
\includegraphics[height=2.15in]{SCRIPT_FIGURES/NIST_RSE_1994/NIST_RSE_1994_O2_Front} &
\includegraphics[height=2.15in]{SCRIPT_FIGURES/NIST_RSE_1994/NIST_RSE_1994_O2_Rear}
\end{tabular*}
\caption[Comparison of measured and predicted O$_2$ concentration, NIST RSE experiments]{Comparison of measured and predicted O$_2$ concentration, NIST RSE experiments.}
\label{NIST_RSE_1994_O2}
\end{figure}

\begin{figure}[!h]
\begin{tabular*}{\textwidth}{l@{\extracolsep{\fill}}r}
\includegraphics[height=2.15in]{SCRIPT_FIGURES/NIST_RSE_1994/NIST_RSE_1994_H2O_Front} &
\includegraphics[height=2.15in]{SCRIPT_FIGURES/NIST_RSE_1994/NIST_RSE_1994_H2O_Rear}
\end{tabular*}
\caption[Comparison of measured and predicted H$_2$O concentration, NIST RSE experiments]{Comparison of measured and predicted H$_2$O concentration, NIST RSE experiments.}
\label{NIST_RSE_1994_H2O}
\end{figure}

\begin{figure}[!h]
\begin{tabular*}{\textwidth}{l@{\extracolsep{\fill}}r}
\includegraphics[height=2.15in]{SCRIPT_FIGURES/NIST_RSE_1994/NIST_RSE_1994_TC_Front} &
\includegraphics[height=2.15in]{SCRIPT_FIGURES/NIST_RSE_1994/NIST_RSE_1994_TC_Rear}
\end{tabular*}
\caption[Comparison of measured and predicted temperature, NIST RSE experiments]{Comparison of measured and predicted temperature, NIST RSE experiments.}
\label{NIST_RSE_1994_temp}
\end{figure}

\clearpage



\subsection{NIST Reduced-Scale Enclosure (RSE) Experiments, 2007}
\label{sec:NIST_RSE_2007}

Species and temperature measurements were made in the front and rear of the compartment, 10~cm below the ceiling, 29~cm from the right wall (looking into the compartment), and 10~cm from the front wall or 29~cm from the rear wall. The compartment was 0.95~m wide by 1.42~m deep by 0.98~m tall with a door 0.48~m wide by 0.81~m tall centered on one of the short walls.  Measurements of temperature, O$_2$, CO$_2$, CO, and unburned hydrocarbon concentration were made near the ceiling in the front and back of the compartment. Details of the experiments are found in Sec.~\ref{NIST_RSE_2007_Description}.

\begin{figure}[!ht]
\begin{tabular*}{\textwidth}{l@{\extracolsep{\fill}}r}
\includegraphics[height=2.15in]{SCRIPT_FIGURES/NIST_RSE_2007/RSE_01_CO2} &
\includegraphics[height=2.15in]{SCRIPT_FIGURES/NIST_RSE_2007/RSE_01_CO} \\
\includegraphics[height=2.15in]{SCRIPT_FIGURES/NIST_RSE_2007/RSE_01_O2} &
\includegraphics[height=2.15in]{SCRIPT_FIGURES/NIST_RSE_2007/RSE_01_THC} \\
\includegraphics[height=2.15in]{SCRIPT_FIGURES/NIST_RSE_2007/RSE_01_temp} &
\includegraphics[height=2.15in]{SCRIPT_FIGURES/NIST_RSE_2007/RSE_01_HRR}
\end{tabular*}
\caption[Summary of Test 1, NIST RSE 2007]{Summary of Test 1, NIST RSE 2007.}
\label{NIST_RSE_2007_1}
\end{figure}

\newpage

\begin{figure}[p]
\begin{tabular*}{\textwidth}{l@{\extracolsep{\fill}}r}
\includegraphics[height=2.15in]{SCRIPT_FIGURES/NIST_RSE_2007/RSE_02_CO2} &
\includegraphics[height=2.15in]{SCRIPT_FIGURES/NIST_RSE_2007/RSE_02_CO} \\
\includegraphics[height=2.15in]{SCRIPT_FIGURES/NIST_RSE_2007/RSE_02_O2} &
\includegraphics[height=2.15in]{SCRIPT_FIGURES/NIST_RSE_2007/RSE_02_THC} \\
\includegraphics[height=2.15in]{SCRIPT_FIGURES/NIST_RSE_2007/RSE_02_temp} &
\includegraphics[height=2.15in]{SCRIPT_FIGURES/NIST_RSE_2007/RSE_02_HRR}
\end{tabular*}
\caption[Summary of Test 2, NIST RSE 2007]{Summary of Test 2, NIST RSE 2007.}
\label{NIST_RSE_2007_2}
\end{figure}

\begin{figure}[p]
\begin{tabular*}{\textwidth}{l@{\extracolsep{\fill}}r}
\includegraphics[height=2.15in]{SCRIPT_FIGURES/NIST_RSE_2007/RSE_03_CO2} &
\includegraphics[height=2.15in]{SCRIPT_FIGURES/NIST_RSE_2007/RSE_03_CO} \\
\includegraphics[height=2.15in]{SCRIPT_FIGURES/NIST_RSE_2007/RSE_03_O2} &
\includegraphics[height=2.15in]{SCRIPT_FIGURES/NIST_RSE_2007/RSE_03_THC} \\
\includegraphics[height=2.15in]{SCRIPT_FIGURES/NIST_RSE_2007/RSE_03_temp} &
\includegraphics[height=2.15in]{SCRIPT_FIGURES/NIST_RSE_2007/RSE_03_HRR}
\end{tabular*}
\caption[Summary of Test 3, NIST RSE 2007]{Summary of Test 3, NIST RSE 2007.}
\label{NIST_RSE_2007_3}
\end{figure}

\begin{figure}[p]
\begin{tabular*}{\textwidth}{l@{\extracolsep{\fill}}r}
\includegraphics[height=2.15in]{SCRIPT_FIGURES/NIST_RSE_2007/RSE_04_CO2} &
\includegraphics[height=2.15in]{SCRIPT_FIGURES/NIST_RSE_2007/RSE_04_CO} \\
\includegraphics[height=2.15in]{SCRIPT_FIGURES/NIST_RSE_2007/RSE_04_O2} &
\includegraphics[height=2.15in]{SCRIPT_FIGURES/NIST_RSE_2007/RSE_04_THC} \\
\includegraphics[height=2.15in]{SCRIPT_FIGURES/NIST_RSE_2007/RSE_04_temp} &
\includegraphics[height=2.15in]{SCRIPT_FIGURES/NIST_RSE_2007/RSE_04_HRR}
\end{tabular*}
\caption[Summary of Test 4, NIST RSE 2007]{Summary of Test 4, NIST RSE 2007.}
\label{NIST_RSE_2007_4}
\end{figure}

\begin{figure}[p]
\begin{tabular*}{\textwidth}{l@{\extracolsep{\fill}}r}
\includegraphics[height=2.15in]{SCRIPT_FIGURES/NIST_RSE_2007/RSE_05_CO2} &
\includegraphics[height=2.15in]{SCRIPT_FIGURES/NIST_RSE_2007/RSE_05_CO} \\
\includegraphics[height=2.15in]{SCRIPT_FIGURES/NIST_RSE_2007/RSE_05_O2} &
\includegraphics[height=2.15in]{SCRIPT_FIGURES/NIST_RSE_2007/RSE_05_THC} \\
\includegraphics[height=2.15in]{SCRIPT_FIGURES/NIST_RSE_2007/RSE_05_temp} &
\includegraphics[height=2.15in]{SCRIPT_FIGURES/NIST_RSE_2007/RSE_05_HRR}
\end{tabular*}
\caption[Summary of Test 5, NIST RSE 2007]{Summary of Test 5, NIST RSE 2007.}
\label{NIST_RSE_2007_5}
\end{figure}

\begin{figure}[p]
\begin{tabular*}{\textwidth}{l@{\extracolsep{\fill}}r}
\includegraphics[height=2.15in]{SCRIPT_FIGURES/NIST_RSE_2007/RSE_06_CO2} &
\includegraphics[height=2.15in]{SCRIPT_FIGURES/NIST_RSE_2007/RSE_06_CO} \\
\includegraphics[height=2.15in]{SCRIPT_FIGURES/NIST_RSE_2007/RSE_06_O2} &
\includegraphics[height=2.15in]{SCRIPT_FIGURES/NIST_RSE_2007/RSE_06_THC} \\
\includegraphics[height=2.15in]{SCRIPT_FIGURES/NIST_RSE_2007/RSE_06_temp} &
\includegraphics[height=2.15in]{SCRIPT_FIGURES/NIST_RSE_2007/RSE_06_HRR}
\end{tabular*}
\caption[Summary of Test 6, NIST RSE 2007]{Summary of Test 6, NIST RSE 2007.}
\label{NIST_RSE_2007_6}
\end{figure}

\begin{figure}[p]
\begin{tabular*}{\textwidth}{l@{\extracolsep{\fill}}r}
\includegraphics[height=2.15in]{SCRIPT_FIGURES/NIST_RSE_2007/RSE_07_CO2} &
\includegraphics[height=2.15in]{SCRIPT_FIGURES/NIST_RSE_2007/RSE_07_CO} \\
\includegraphics[height=2.15in]{SCRIPT_FIGURES/NIST_RSE_2007/RSE_07_O2} &
\includegraphics[height=2.15in]{SCRIPT_FIGURES/NIST_RSE_2007/RSE_07_THC} \\
\includegraphics[height=2.15in]{SCRIPT_FIGURES/NIST_RSE_2007/RSE_07_temp} &
\includegraphics[height=2.15in]{SCRIPT_FIGURES/NIST_RSE_2007/RSE_07_HRR}
\end{tabular*}
\caption[Summary of Test 7, NIST RSE 2007]{Summary of Test 7, NIST RSE 2007.}
\label{NIST_RSE_2007_7}
\end{figure}

\begin{figure}[p]
\begin{tabular*}{\textwidth}{l@{\extracolsep{\fill}}r}
\includegraphics[height=2.15in]{SCRIPT_FIGURES/NIST_RSE_2007/RSE_10_CO2} &
\includegraphics[height=2.15in]{SCRIPT_FIGURES/NIST_RSE_2007/RSE_10_CO} \\
\includegraphics[height=2.15in]{SCRIPT_FIGURES/NIST_RSE_2007/RSE_10_O2} &
\includegraphics[height=2.15in]{SCRIPT_FIGURES/NIST_RSE_2007/RSE_10_THC} \\
\includegraphics[height=2.15in]{SCRIPT_FIGURES/NIST_RSE_2007/RSE_10_temp} &
\includegraphics[height=2.15in]{SCRIPT_FIGURES/NIST_RSE_2007/RSE_10_HRR}
\end{tabular*}
\caption[Summary of Test 10, NIST RSE 2007]{Summary of Test 10, NIST RSE 2007.}
\label{NIST_RSE_2007_10}
\end{figure}

\begin{figure}[p]
\begin{tabular*}{\textwidth}{l@{\extracolsep{\fill}}r}
\includegraphics[height=2.15in]{SCRIPT_FIGURES/NIST_RSE_2007/RSE_11_CO2} &
\includegraphics[height=2.15in]{SCRIPT_FIGURES/NIST_RSE_2007/RSE_11_CO} \\
\includegraphics[height=2.15in]{SCRIPT_FIGURES/NIST_RSE_2007/RSE_11_O2} &
\includegraphics[height=2.15in]{SCRIPT_FIGURES/NIST_RSE_2007/RSE_11_THC} \\
\includegraphics[height=2.15in]{SCRIPT_FIGURES/NIST_RSE_2007/RSE_11_temp} &
\includegraphics[height=2.15in]{SCRIPT_FIGURES/NIST_RSE_2007/RSE_11_HRR}
\end{tabular*}
\caption[Summary of Test 11, NIST RSE 2007]{Summary of Test 11, NIST RSE 2007.}
\label{NIST_RSE_2007_11}
\end{figure}

\begin{figure}[p]
\begin{tabular*}{\textwidth}{l@{\extracolsep{\fill}}r}
\includegraphics[height=2.15in]{SCRIPT_FIGURES/NIST_RSE_2007/RSE_12_CO2} &
\includegraphics[height=2.15in]{SCRIPT_FIGURES/NIST_RSE_2007/RSE_12_CO} \\
\includegraphics[height=2.15in]{SCRIPT_FIGURES/NIST_RSE_2007/RSE_12_O2} &
\includegraphics[height=2.15in]{SCRIPT_FIGURES/NIST_RSE_2007/RSE_12_THC} \\
\includegraphics[height=2.15in]{SCRIPT_FIGURES/NIST_RSE_2007/RSE_12_temp} &
\includegraphics[height=2.15in]{SCRIPT_FIGURES/NIST_RSE_2007/RSE_12_HRR}
\end{tabular*}
\caption[Summary of Test 12, NIST RSE 2007]{Summary of Test 12, NIST RSE 2007.}
\label{NIST_RSE_2007_12}
\end{figure}

\begin{figure}[p]
\begin{tabular*}{\textwidth}{l@{\extracolsep{\fill}}r}
\includegraphics[height=2.15in]{SCRIPT_FIGURES/NIST_RSE_2007/RSE_15_CO2} &
\includegraphics[height=2.15in]{SCRIPT_FIGURES/NIST_RSE_2007/RSE_15_CO} \\
\includegraphics[height=2.15in]{SCRIPT_FIGURES/NIST_RSE_2007/RSE_15_O2} &
\includegraphics[height=2.15in]{SCRIPT_FIGURES/NIST_RSE_2007/RSE_15_THC} \\
\includegraphics[height=2.15in]{SCRIPT_FIGURES/NIST_RSE_2007/RSE_15_temp} &
\includegraphics[height=2.15in]{SCRIPT_FIGURES/NIST_RSE_2007/RSE_15_HRR}
\end{tabular*}
\caption[Summary of Test 15, NIST RSE 2007]{Summary of Test 15, NIST RSE 2007.}
\label{NIST_RSE_2007_15}
\end{figure}

\begin{figure}[p]
\begin{tabular*}{\textwidth}{l@{\extracolsep{\fill}}r}
\includegraphics[height=2.15in]{SCRIPT_FIGURES/NIST_RSE_2007/RSE_16_CO2} &
\includegraphics[height=2.15in]{SCRIPT_FIGURES/NIST_RSE_2007/RSE_16_CO} \\
\includegraphics[height=2.15in]{SCRIPT_FIGURES/NIST_RSE_2007/RSE_16_O2} &
\includegraphics[height=2.15in]{SCRIPT_FIGURES/NIST_RSE_2007/RSE_16_THC} \\
\includegraphics[height=2.15in]{SCRIPT_FIGURES/NIST_RSE_2007/RSE_16_temp} &
\includegraphics[height=2.15in]{SCRIPT_FIGURES/NIST_RSE_2007/RSE_16_HRR}
\end{tabular*}
\caption[Summary of Test 16, NIST RSE 2007]{Summary of Test 16, NIST RSE 2007.}
\label{NIST_RSE_2007_16}
\end{figure}


\clearpage

\subsection{NIST Full-Scale Enclosure (FSE) Experiments, 2008}
\label{sec:NIST_FSE_2008}

Species concentrations and temperature measurements were made at the front and rear of the compartment. Details of the experiments are found in Sec.~\ref{NIST_FSE_2008_Description}.

\begin{figure}[!ht]
\begin{tabular*}{\textwidth}{l@{\extracolsep{\fill}}r}
\includegraphics[height=2.15in]{SCRIPT_FIGURES/NIST_FSE_2008/ISONG3_Carbon_Dioxide} &
\includegraphics[height=2.15in]{SCRIPT_FIGURES/NIST_FSE_2008/ISONG3_Carbon_Monoxide} \\
\includegraphics[height=2.15in]{SCRIPT_FIGURES/NIST_FSE_2008/ISONG3_Oxygen} &
\includegraphics[height=2.15in]{SCRIPT_FIGURES/NIST_FSE_2008/ISONG3_Unburned_Hydrocarbons} \\
\includegraphics[height=2.15in]{SCRIPT_FIGURES/NIST_FSE_2008/ISONG3_Temperature} &
\includegraphics[height=2.15in]{SCRIPT_FIGURES/NIST_FSE_2008/ISONG3_HRR}
\end{tabular*}
\caption[Summary of ISONG3, NIST FSE 2008]{Summary of ISONG3, NIST FSE 2008.}
\label{NIST_FSE_1994_ISONG3}
\end{figure}

\newpage

\begin{figure}[p]
\begin{tabular*}{\textwidth}{l@{\extracolsep{\fill}}r}
\includegraphics[height=2.15in]{SCRIPT_FIGURES/NIST_FSE_2008/ISOHept4_Carbon_Dioxide} &
\includegraphics[height=2.15in]{SCRIPT_FIGURES/NIST_FSE_2008/ISOHept4_Carbon_Monoxide} \\
\includegraphics[height=2.15in]{SCRIPT_FIGURES/NIST_FSE_2008/ISOHept4_Oxygen} &
\includegraphics[height=2.15in]{SCRIPT_FIGURES/NIST_FSE_2008/ISOHept4_Unburned_Hydrocarbons} \\
\includegraphics[height=2.15in]{SCRIPT_FIGURES/NIST_FSE_2008/ISOHept4_Temperature} &
\includegraphics[height=2.15in]{SCRIPT_FIGURES/NIST_FSE_2008/ISOHept4_HRR}
\end{tabular*}
\caption[Summary of ISOHept4, NIST FSE 2008]{Summary of ISOHept4, NIST FSE 2008.}
\label{NIST_FSE_1994_ISOHept4}
\end{figure}

\begin{figure}[p]
\begin{tabular*}{\textwidth}{l@{\extracolsep{\fill}}r}
\includegraphics[height=2.15in]{SCRIPT_FIGURES/NIST_FSE_2008/ISOHept5_Carbon_Dioxide} &
\includegraphics[height=2.15in]{SCRIPT_FIGURES/NIST_FSE_2008/ISOHept5_Carbon_Monoxide} \\
\includegraphics[height=2.15in]{SCRIPT_FIGURES/NIST_FSE_2008/ISOHept5_Oxygen} &
\includegraphics[height=2.15in]{SCRIPT_FIGURES/NIST_FSE_2008/ISOHept5_Unburned_Hydrocarbons} \\
\includegraphics[height=2.15in]{SCRIPT_FIGURES/NIST_FSE_2008/ISOHept5_Temperature} &
\includegraphics[height=2.15in]{SCRIPT_FIGURES/NIST_FSE_2008/ISOHept5_HRR}
\end{tabular*}
\caption[Summary of ISOHept5, NIST FSE 2008]{Summary of ISOHept5, NIST FSE 2008.}
\label{NIST_FSE_1994_ISOHept5}
\end{figure}

\begin{figure}[p]
\begin{tabular*}{\textwidth}{l@{\extracolsep{\fill}}r}
\includegraphics[height=2.15in]{SCRIPT_FIGURES/NIST_FSE_2008/ISOHept8_Carbon_Dioxide} &
\includegraphics[height=2.15in]{SCRIPT_FIGURES/NIST_FSE_2008/ISOHept8_Carbon_Monoxide} \\
\includegraphics[height=2.15in]{SCRIPT_FIGURES/NIST_FSE_2008/ISOHept8_Oxygen} &
\includegraphics[height=2.15in]{SCRIPT_FIGURES/NIST_FSE_2008/ISOHept8_Unburned_Hydrocarbons} \\
\includegraphics[height=2.15in]{SCRIPT_FIGURES/NIST_FSE_2008/ISOHept8_Temperature} &
\includegraphics[height=2.15in]{SCRIPT_FIGURES/NIST_FSE_2008/ISOHept8_HRR}
\end{tabular*}
\caption[Summary of ISOHept8, NIST FSE 2008]{Summary of ISOHept8, NIST FSE 2008.}
\label{NIST_FSE_1994_ISOHept8}
\end{figure}

\begin{figure}[p]
\begin{tabular*}{\textwidth}{l@{\extracolsep{\fill}}r}
\includegraphics[height=2.15in]{SCRIPT_FIGURES/NIST_FSE_2008/ISOHept9_Carbon_Dioxide} &
\includegraphics[height=2.15in]{SCRIPT_FIGURES/NIST_FSE_2008/ISOHept9_Carbon_Monoxide} \\
\includegraphics[height=2.15in]{SCRIPT_FIGURES/NIST_FSE_2008/ISOHept9_Oxygen} &
\includegraphics[height=2.15in]{SCRIPT_FIGURES/NIST_FSE_2008/ISOHept9_Unburned_Hydrocarbons} \\
\includegraphics[height=2.15in]{SCRIPT_FIGURES/NIST_FSE_2008/ISOHept9_Temperature} &
\includegraphics[height=2.15in]{SCRIPT_FIGURES/NIST_FSE_2008/ISOHept9_HRR}
\end{tabular*}
\caption[Summary of ISOHept9, NIST FSE 2008]{Summary of ISOHept9, NIST FSE 2008.}
\label{NIST_FSE_1994_ISOHept9}
\end{figure}

\begin{figure}[p]
\begin{tabular*}{\textwidth}{l@{\extracolsep{\fill}}r}
\includegraphics[height=2.15in]{SCRIPT_FIGURES/NIST_FSE_2008/ISONylon10_Carbon_Dioxide} &
\includegraphics[height=2.15in]{SCRIPT_FIGURES/NIST_FSE_2008/ISONylon10_Carbon_Monoxide} \\
\includegraphics[height=2.15in]{SCRIPT_FIGURES/NIST_FSE_2008/ISONylon10_Oxygen} &
\includegraphics[height=2.15in]{SCRIPT_FIGURES/NIST_FSE_2008/ISONylon10_Unburned_Hydrocarbons} \\
\includegraphics[height=2.15in]{SCRIPT_FIGURES/NIST_FSE_2008/ISONylon10_Temperature} &
\includegraphics[height=2.15in]{SCRIPT_FIGURES/NIST_FSE_2008/ISONylon10_HRR}
\end{tabular*}
\caption[Summary of ISONylon10, NIST FSE 2008]{Summary of ISONylon10, NIST FSE 2008.}
\label{NIST_FSE_1994_ISONylon10}
\end{figure}

\begin{figure}[p]
\begin{tabular*}{\textwidth}{l@{\extracolsep{\fill}}r}
\includegraphics[height=2.15in]{SCRIPT_FIGURES/NIST_FSE_2008/ISOPP11_Carbon_Dioxide} &
\includegraphics[height=2.15in]{SCRIPT_FIGURES/NIST_FSE_2008/ISOPP11_Carbon_Monoxide} \\
\includegraphics[height=2.15in]{SCRIPT_FIGURES/NIST_FSE_2008/ISOPP11_Oxygen} &
\includegraphics[height=2.15in]{SCRIPT_FIGURES/NIST_FSE_2008/ISOPP11_Unburned_Hydrocarbons} \\
\includegraphics[height=2.15in]{SCRIPT_FIGURES/NIST_FSE_2008/ISOPP11_Temperature} &
\includegraphics[height=2.15in]{SCRIPT_FIGURES/NIST_FSE_2008/ISOPP11_HRR}
\end{tabular*}
\caption[Summary of ISOPP11, NIST FSE 2008]{Summary of ISOPP11, NIST FSE 2008.}
\label{NIST_FSE_1994_ISOPropylene11}
\end{figure}

\begin{figure}[p]
\begin{tabular*}{\textwidth}{l@{\extracolsep{\fill}}r}
\includegraphics[height=2.15in]{SCRIPT_FIGURES/NIST_FSE_2008/ISOHeptD12_Carbon_Dioxide} &
\includegraphics[height=2.15in]{SCRIPT_FIGURES/NIST_FSE_2008/ISOHeptD12_Carbon_Monoxide} \\
\includegraphics[height=2.15in]{SCRIPT_FIGURES/NIST_FSE_2008/ISOHeptD12_Oxygen} &
\includegraphics[height=2.15in]{SCRIPT_FIGURES/NIST_FSE_2008/ISOHeptD12_Unburned_Hydrocarbons} \\
\includegraphics[height=2.15in]{SCRIPT_FIGURES/NIST_FSE_2008/ISOHeptD12_Temperature} &
\includegraphics[height=2.15in]{SCRIPT_FIGURES/NIST_FSE_2008/ISOHeptD12_HRR}
\end{tabular*}
\caption[Summary of ISOHeptD12, NIST FSE 2008]{Summary of ISOHeptD12, NIST FSE 2008.}
\label{NIST_FSE_1994_ISOHeptD12}
\end{figure}

\begin{figure}[p]
\begin{tabular*}{\textwidth}{l@{\extracolsep{\fill}}r}
\includegraphics[height=2.15in]{SCRIPT_FIGURES/NIST_FSE_2008/ISOHeptD13_Carbon_Dioxide} &
\includegraphics[height=2.15in]{SCRIPT_FIGURES/NIST_FSE_2008/ISOHeptD13_Carbon_Monoxide} \\
\includegraphics[height=2.15in]{SCRIPT_FIGURES/NIST_FSE_2008/ISOHeptD13_Oxygen} &
\includegraphics[height=2.15in]{SCRIPT_FIGURES/NIST_FSE_2008/ISOHeptD13_Unburned_Hydrocarbons} \\
\includegraphics[height=2.15in]{SCRIPT_FIGURES/NIST_FSE_2008/ISOHeptD13_Temperature} &
\includegraphics[height=2.15in]{SCRIPT_FIGURES/NIST_FSE_2008/ISOHeptD13_HRR}
\end{tabular*}
\caption[Summary of ISOHeptD13, NIST FSE 2008]{Summary of ISOHeptD13, NIST FSE 2008.}
\label{NIST_FSE_1994_ISOHeptD13}
\end{figure}

\begin{figure}[p]
\begin{tabular*}{\textwidth}{l@{\extracolsep{\fill}}r}
\includegraphics[height=2.15in]{SCRIPT_FIGURES/NIST_FSE_2008/ISOPropD14_Carbon_Dioxide} &
\includegraphics[height=2.15in]{SCRIPT_FIGURES/NIST_FSE_2008/ISOPropD14_Carbon_Monoxide} \\
\includegraphics[height=2.15in]{SCRIPT_FIGURES/NIST_FSE_2008/ISOPropD14_Oxygen} &
\includegraphics[height=2.15in]{SCRIPT_FIGURES/NIST_FSE_2008/ISOPropD14_Unburned_Hydrocarbons} \\
\includegraphics[height=2.15in]{SCRIPT_FIGURES/NIST_FSE_2008/ISOPropD14_Temperature} &
\includegraphics[height=2.15in]{SCRIPT_FIGURES/NIST_FSE_2008/ISOPropD14_HRR}
\end{tabular*}
\caption[Summary of ISOPropD14, NIST FSE 2008]{Summary of ISOPropD14, NIST FSE 2008.}
\label{NIST_FSE_1994_ISOPropD14}
\end{figure}

\begin{figure}[p]
\begin{tabular*}{\textwidth}{l@{\extracolsep{\fill}}r}
\includegraphics[height=2.15in]{SCRIPT_FIGURES/NIST_FSE_2008/ISOProp15_Carbon_Dioxide} &
\includegraphics[height=2.15in]{SCRIPT_FIGURES/NIST_FSE_2008/ISOProp15_Carbon_Monoxide} \\
\includegraphics[height=2.15in]{SCRIPT_FIGURES/NIST_FSE_2008/ISOProp15_Oxygen} &
\includegraphics[height=2.15in]{SCRIPT_FIGURES/NIST_FSE_2008/ISOProp15_Unburned_Hydrocarbons} \\
\includegraphics[height=2.15in]{SCRIPT_FIGURES/NIST_FSE_2008/ISOProp15_Temperature} &
\includegraphics[height=2.15in]{SCRIPT_FIGURES/NIST_FSE_2008/ISOProp15_HRR}
\end{tabular*}
\caption[Summary of ISOProp15, NIST FSE 2008]{Summary of ISOProp15, NIST FSE 2008.}
\label{NIST_FSE_1994_ISOProp15}
\end{figure}

\begin{figure}[p]
\begin{tabular*}{\textwidth}{l@{\extracolsep{\fill}}r}
\includegraphics[height=2.15in]{SCRIPT_FIGURES/NIST_FSE_2008/ISOStyrene16_Carbon_Dioxide} &
\includegraphics[height=2.15in]{SCRIPT_FIGURES/NIST_FSE_2008/ISOStyrene16_Carbon_Monoxide} \\
\includegraphics[height=2.15in]{SCRIPT_FIGURES/NIST_FSE_2008/ISOStyrene16_Oxygen} &
\includegraphics[height=2.15in]{SCRIPT_FIGURES/NIST_FSE_2008/ISOStyrene16_Unburned_Hydrocarbons} \\
\includegraphics[height=2.15in]{SCRIPT_FIGURES/NIST_FSE_2008/ISOStyrene16_Temperature} &
\includegraphics[height=2.15in]{SCRIPT_FIGURES/NIST_FSE_2008/ISOStyrene16_HRR}
\end{tabular*}
\caption[Summary of ISOStyrene16, NIST FSE 2008]{Summary of ISOStyrene16, NIST FSE 2008.}
\label{NIST_FSE_1994_ISOStyrene16}
\end{figure}

\begin{figure}[p]
\begin{tabular*}{\textwidth}{l@{\extracolsep{\fill}}r}
\includegraphics[height=2.15in]{SCRIPT_FIGURES/NIST_FSE_2008/ISOStyrene17_Carbon_Dioxide} &
\includegraphics[height=2.15in]{SCRIPT_FIGURES/NIST_FSE_2008/ISOStyrene17_Carbon_Monoxide} \\
\includegraphics[height=2.15in]{SCRIPT_FIGURES/NIST_FSE_2008/ISOStyrene17_Oxygen} &
\includegraphics[height=2.15in]{SCRIPT_FIGURES/NIST_FSE_2008/ISOStyrene17_Unburned_Hydrocarbons} \\
\includegraphics[height=2.15in]{SCRIPT_FIGURES/NIST_FSE_2008/ISOStyrene17_Temperature} &
\includegraphics[height=2.15in]{SCRIPT_FIGURES/NIST_FSE_2008/ISOStyrene17_HRR}
\end{tabular*}
\caption[Summary of ISOStyrene17, NIST FSE 2008]{Summary of ISOStyrene17, NIST FSE 2008.}
\label{NIST_FSE_1994_ISOStyrene17}
\end{figure}


\begin{figure}[p]
\begin{tabular*}{\textwidth}{l@{\extracolsep{\fill}}r}
\includegraphics[height=2.15in]{SCRIPT_FIGURES/NIST_FSE_2008/ISOPP18_Carbon_Dioxide} &
\includegraphics[height=2.15in]{SCRIPT_FIGURES/NIST_FSE_2008/ISOPP18_Carbon_Monoxide} \\
\includegraphics[height=2.15in]{SCRIPT_FIGURES/NIST_FSE_2008/ISOPP18_Oxygen} &
\includegraphics[height=2.15in]{SCRIPT_FIGURES/NIST_FSE_2008/ISOPP18_Unburned_Hydrocarbons} \\
\includegraphics[height=2.15in]{SCRIPT_FIGURES/NIST_FSE_2008/ISOPP18_Temperature} &
\includegraphics[height=2.15in]{SCRIPT_FIGURES/NIST_FSE_2008/ISOPP18_HRR}
\end{tabular*}
\caption[Summary of ISOPP18, NIST FSE 2008]{Summary of ISOPP18, NIST FSE 2008.}
\label{NIST_FSE_1994_ISOPP18}
\end{figure}

\begin{figure}[p]
\begin{tabular*}{\textwidth}{l@{\extracolsep{\fill}}r}
\includegraphics[height=2.15in]{SCRIPT_FIGURES/NIST_FSE_2008/ISOHept19_Carbon_Dioxide} &
\includegraphics[height=2.15in]{SCRIPT_FIGURES/NIST_FSE_2008/ISOHept19_Carbon_Monoxide} \\
\includegraphics[height=2.15in]{SCRIPT_FIGURES/NIST_FSE_2008/ISOHept19_Oxygen} &
\includegraphics[height=2.15in]{SCRIPT_FIGURES/NIST_FSE_2008/ISOHept19_Unburned_Hydrocarbons} \\
\includegraphics[height=2.15in]{SCRIPT_FIGURES/NIST_FSE_2008/ISOHept19_Temperature} &
\includegraphics[height=2.15in]{SCRIPT_FIGURES/NIST_FSE_2008/ISOHept19_HRR}
\end{tabular*}
\caption[Summary of ISOHept19, NIST FSE 2008]{Summary of ISOHept19, NIST FSE 2008.}
\label{NIST_FSE_1994_ISOHept19}
\end{figure}

\begin{figure}[p]
\begin{tabular*}{\textwidth}{l@{\extracolsep{\fill}}r}
\includegraphics[height=2.15in]{SCRIPT_FIGURES/NIST_FSE_2008/ISOToluene20_Carbon_Dioxide} &
\includegraphics[height=2.15in]{SCRIPT_FIGURES/NIST_FSE_2008/ISOToluene20_Carbon_Monoxide} \\
\includegraphics[height=2.15in]{SCRIPT_FIGURES/NIST_FSE_2008/ISOToluene20_Oxygen} &
\includegraphics[height=2.15in]{SCRIPT_FIGURES/NIST_FSE_2008/ISOToluene20_Unburned_Hydrocarbons} \\
\includegraphics[height=2.15in]{SCRIPT_FIGURES/NIST_FSE_2008/ISOToluene20_Temperature} &
\includegraphics[height=2.15in]{SCRIPT_FIGURES/NIST_FSE_2008/ISOToluene20_HRR}
\end{tabular*}
\caption[Summary of ISOToluene20, NIST FSE 2008]{Summary of ISOToluene20, NIST FSE 2008.}
\label{NIST_FSE_1994_ISOToluene20}
\end{figure}

\begin{figure}[p]
\begin{tabular*}{\textwidth}{l@{\extracolsep{\fill}}r}
\includegraphics[height=2.15in]{SCRIPT_FIGURES/NIST_FSE_2008/ISOStyrene21_Carbon_Dioxide} &
\includegraphics[height=2.15in]{SCRIPT_FIGURES/NIST_FSE_2008/ISOStyrene21_Carbon_Monoxide} \\
\includegraphics[height=2.15in]{SCRIPT_FIGURES/NIST_FSE_2008/ISOStyrene21_Oxygen} &
\includegraphics[height=2.15in]{SCRIPT_FIGURES/NIST_FSE_2008/ISOStyrene21_Unburned_Hydrocarbons} \\
\includegraphics[height=2.15in]{SCRIPT_FIGURES/NIST_FSE_2008/ISOStyrene21_Temperature} &
\includegraphics[height=2.15in]{SCRIPT_FIGURES/NIST_FSE_2008/ISOStyrene21_HRR}
\end{tabular*}
\caption[Summary of ISOStyrene21, NIST FSE 2008]{Summary of ISOStyrene21, NIST FSE 2008.}
\label{NIST_FSE_1994_ISOStyrene21}
\end{figure}

\begin{figure}[p]
\begin{tabular*}{\textwidth}{l@{\extracolsep{\fill}}r}
\includegraphics[height=2.15in]{SCRIPT_FIGURES/NIST_FSE_2008/ISOHept22_Carbon_Dioxide} &
\includegraphics[height=2.15in]{SCRIPT_FIGURES/NIST_FSE_2008/ISOHept22_Carbon_Monoxide} \\
\includegraphics[height=2.15in]{SCRIPT_FIGURES/NIST_FSE_2008/ISOHept22_Oxygen} &
\includegraphics[height=2.15in]{SCRIPT_FIGURES/NIST_FSE_2008/ISOHept22_Unburned_Hydrocarbons} \\
\includegraphics[height=2.15in]{SCRIPT_FIGURES/NIST_FSE_2008/ISOHept22_Temperature} &
\includegraphics[height=2.15in]{SCRIPT_FIGURES/NIST_FSE_2008/ISOHept22_HRR}
\end{tabular*}
\caption[Summary of ISOHept22, NIST FSE 2008]{Summary of ISOHept22, NIST FSE 2008.}
\label{NIST_FSE_1994_ISOHept22}
\end{figure}

\begin{figure}[p]
\begin{tabular*}{\textwidth}{l@{\extracolsep{\fill}}r}
\includegraphics[height=2.15in]{SCRIPT_FIGURES/NIST_FSE_2008/ISOHept23_Carbon_Dioxide} &
\includegraphics[height=2.15in]{SCRIPT_FIGURES/NIST_FSE_2008/ISOHept23_Carbon_Monoxide} \\
\includegraphics[height=2.15in]{SCRIPT_FIGURES/NIST_FSE_2008/ISOHept23_Oxygen} &
\includegraphics[height=2.15in]{SCRIPT_FIGURES/NIST_FSE_2008/ISOHept23_Unburned_Hydrocarbons} \\
\includegraphics[height=2.15in]{SCRIPT_FIGURES/NIST_FSE_2008/ISOHept23_Temperature} &
\includegraphics[height=2.15in]{SCRIPT_FIGURES/NIST_FSE_2008/ISOHept23_HRR}
\end{tabular*}
\caption[Summary of ISOHept23, NIST FSE 2008]{Summary of ISOHept23, NIST FSE 2008.}
\label{NIST_FSE_1994_ISOHept23}
\end{figure}

\begin{figure}[p]
\begin{tabular*}{\textwidth}{l@{\extracolsep{\fill}}r}
\includegraphics[height=2.15in]{SCRIPT_FIGURES/NIST_FSE_2008/ISOHept24_Carbon_Dioxide} &
\includegraphics[height=2.15in]{SCRIPT_FIGURES/NIST_FSE_2008/ISOHept24_Carbon_Monoxide} \\
\includegraphics[height=2.15in]{SCRIPT_FIGURES/NIST_FSE_2008/ISOHept24_Oxygen} &
\includegraphics[height=2.15in]{SCRIPT_FIGURES/NIST_FSE_2008/ISOHept24_Unburned_Hydrocarbons} \\
\includegraphics[height=2.15in]{SCRIPT_FIGURES/NIST_FSE_2008/ISOHept24_Temperature} &
\includegraphics[height=2.15in]{SCRIPT_FIGURES/NIST_FSE_2008/ISOHept24_HRR}
\end{tabular*}
\caption[Summary of ISOHept24, NIST FSE 2008]{Summary of ISOHept24, NIST FSE 2008.}
\label{NIST_FSE_1994_ISOHept24}
\end{figure}

\begin{figure}[p]
\begin{tabular*}{\textwidth}{l@{\extracolsep{\fill}}r}
\includegraphics[height=2.15in]{SCRIPT_FIGURES/NIST_FSE_2008/ISOHept25_Carbon_Dioxide} &
\includegraphics[height=2.15in]{SCRIPT_FIGURES/NIST_FSE_2008/ISOHept25_Carbon_Monoxide} \\
\includegraphics[height=2.15in]{SCRIPT_FIGURES/NIST_FSE_2008/ISOHept25_Oxygen} &
\includegraphics[height=2.15in]{SCRIPT_FIGURES/NIST_FSE_2008/ISOHept25_Unburned_Hydrocarbons} \\
\includegraphics[height=2.15in]{SCRIPT_FIGURES/NIST_FSE_2008/ISOHept25_Temperature} &
\includegraphics[height=2.15in]{SCRIPT_FIGURES/NIST_FSE_2008/ISOHept25_HRR}
\end{tabular*}
\caption[Summary of ISOHept25, NIST FSE 2008]{Summary of ISOHept25, NIST FSE 2008.}
\label{NIST_FSE_1994_ISOHept25}
\end{figure}

\begin{figure}[p]
\begin{tabular*}{\textwidth}{l@{\extracolsep{\fill}}r}
\includegraphics[height=2.15in]{SCRIPT_FIGURES/NIST_FSE_2008/ISOHept26_Carbon_Dioxide} &
\includegraphics[height=2.15in]{SCRIPT_FIGURES/NIST_FSE_2008/ISOHept26_Carbon_Monoxide} \\
\includegraphics[height=2.15in]{SCRIPT_FIGURES/NIST_FSE_2008/ISOHept26_Oxygen} &
\includegraphics[height=2.15in]{SCRIPT_FIGURES/NIST_FSE_2008/ISOHept26_Unburned_Hydrocarbons} \\
\includegraphics[height=2.15in]{SCRIPT_FIGURES/NIST_FSE_2008/ISOHept26_Temperature} &
\includegraphics[height=2.15in]{SCRIPT_FIGURES/NIST_FSE_2008/ISOHept26_HRR}
\end{tabular*}
\caption[Summary of ISOHept26, NIST FSE 2008]{Summary of ISOHept26, NIST FSE 2008.}
\label{NIST_FSE_1994_ISOHept26}
\end{figure}

\begin{figure}[p]
\begin{tabular*}{\textwidth}{l@{\extracolsep{\fill}}r}
\includegraphics[height=2.15in]{SCRIPT_FIGURES/NIST_FSE_2008/ISOHept27_Carbon_Dioxide} &
\includegraphics[height=2.15in]{SCRIPT_FIGURES/NIST_FSE_2008/ISOHept27_Carbon_Monoxide} \\
\includegraphics[height=2.15in]{SCRIPT_FIGURES/NIST_FSE_2008/ISOHept27_Oxygen} &
\includegraphics[height=2.15in]{SCRIPT_FIGURES/NIST_FSE_2008/ISOHept27_Unburned_Hydrocarbons} \\
\includegraphics[height=2.15in]{SCRIPT_FIGURES/NIST_FSE_2008/ISOHept27_Temperature} &
\includegraphics[height=2.15in]{SCRIPT_FIGURES/NIST_FSE_2008/ISOHept27_HRR}
\end{tabular*}
\caption[Summary of ISOHept27, NIST FSE 2008]{Summary of ISOHept27, NIST FSE 2008.}
\label{NIST_FSE_1994_ISOHept27}
\end{figure}

\begin{figure}[p]
\begin{tabular*}{\textwidth}{l@{\extracolsep{\fill}}r}
\includegraphics[height=2.15in]{SCRIPT_FIGURES/NIST_FSE_2008/ISOHept28_Carbon_Dioxide} &
\includegraphics[height=2.15in]{SCRIPT_FIGURES/NIST_FSE_2008/ISOHept28_Carbon_Monoxide} \\
\includegraphics[height=2.15in]{SCRIPT_FIGURES/NIST_FSE_2008/ISOHept28_Oxygen} &
\includegraphics[height=2.15in]{SCRIPT_FIGURES/NIST_FSE_2008/ISOHept28_Unburned_Hydrocarbons} \\
\includegraphics[height=2.15in]{SCRIPT_FIGURES/NIST_FSE_2008/ISOHept28_Temperature} &
\includegraphics[height=2.15in]{SCRIPT_FIGURES/NIST_FSE_2008/ISOHept28_HRR}
\end{tabular*}
\caption[Summary of ISOHept28, NIST FSE 2008]{Summary of ISOHept28, NIST FSE 2008.}
\label{NIST_FSE_1994_ISOHept28}
\end{figure}

\begin{figure}[p]
\begin{tabular*}{\textwidth}{l@{\extracolsep{\fill}}r}
\includegraphics[height=2.15in]{SCRIPT_FIGURES/NIST_FSE_2008/ISOToluene29_Carbon_Dioxide} &
\includegraphics[height=2.15in]{SCRIPT_FIGURES/NIST_FSE_2008/ISOToluene29_Carbon_Monoxide} \\
\includegraphics[height=2.15in]{SCRIPT_FIGURES/NIST_FSE_2008/ISOToluene29_Oxygen} &
\includegraphics[height=2.15in]{SCRIPT_FIGURES/NIST_FSE_2008/ISOToluene29_Unburned_Hydrocarbons} \\
\includegraphics[height=2.15in]{SCRIPT_FIGURES/NIST_FSE_2008/ISOToluene29_Temperature} &
\includegraphics[height=2.15in]{SCRIPT_FIGURES/NIST_FSE_2008/ISOToluene29_HRR}
\end{tabular*}
\caption[Summary of ISOToluene29, NIST FSE 2008]{Summary of ISOToluene29, NIST FSE 2008.}
\label{NIST_FSE_1994_ISOToluene29}
\end{figure}

\begin{figure}[p]
\begin{tabular*}{\textwidth}{l@{\extracolsep{\fill}}r}
\includegraphics[height=2.15in]{SCRIPT_FIGURES/NIST_FSE_2008/ISOPropanol30_Carbon_Dioxide} &
\includegraphics[height=2.15in]{SCRIPT_FIGURES/NIST_FSE_2008/ISOPropanol30_Carbon_Monoxide} \\
\includegraphics[height=2.15in]{SCRIPT_FIGURES/NIST_FSE_2008/ISOPropanol30_Oxygen} &
\includegraphics[height=2.15in]{SCRIPT_FIGURES/NIST_FSE_2008/ISOPropanol30_Unburned_Hydrocarbons} \\
\includegraphics[height=2.15in]{SCRIPT_FIGURES/NIST_FSE_2008/ISOPropanol30_Temperature} &
\includegraphics[height=2.15in]{SCRIPT_FIGURES/NIST_FSE_2008/ISOPropanol30_HRR}
\end{tabular*}
\caption[Summary of ISOPropanol30, NIST FSE 2008]{Summary of ISOPropanol30, NIST FSE 2008.}
\label{NIST_FSE_1994_ISOPropanol30}
\end{figure}

\begin{figure}[p]
\begin{tabular*}{\textwidth}{l@{\extracolsep{\fill}}r}
\includegraphics[height=2.15in]{SCRIPT_FIGURES/NIST_FSE_2008/ISONG32_Carbon_Dioxide} &
\includegraphics[height=2.15in]{SCRIPT_FIGURES/NIST_FSE_2008/ISONG32_Carbon_Monoxide} \\
\includegraphics[height=2.15in]{SCRIPT_FIGURES/NIST_FSE_2008/ISONG32_Oxygen} &
\includegraphics[height=2.15in]{SCRIPT_FIGURES/NIST_FSE_2008/ISONG32_Unburned_Hydrocarbons} \\
\includegraphics[height=2.15in]{SCRIPT_FIGURES/NIST_FSE_2008/ISONG32_Temperature} &
\includegraphics[height=2.15in]{SCRIPT_FIGURES/NIST_FSE_2008/ISONG32_HRR}
\end{tabular*}
\caption[Summary of ISONG32, NIST FSE 2008]{Summary of ISONG32, NIST FSE 2008.}
\label{NIST_FSE_1994_ISONG32}
\end{figure}


\clearpage

\subsection{NIST Pool Fires}
\label{sec:NIST_Pool_Fires}

Falkenstein-Smith et al.~\cite{Falkenstein-Smith:2019} made mean temperature and product species concentration measurements along the centerline above 30.5~cm diameter liquid pool fires of acetone, ethanol, and methanol; and a 37.0~cm diameter burner of methane and propane. The measurements were made using a gas chromatograph/mass spectrometer system (GC/MSD). The volume fraction of each species was calculated via the number of moles identified by the GC/MSD at each centerline point. Soot mass fractions were measured during the gas sampling process. Note that the species measurements include the vapor form of the primary fuel molecule, in addition to intermediate fuel species. FDS does not model the decomposition of the fuel molecule into intermediate hydrocarbon species except for CO and H$_2$.

The mean centerline temperature profiles are found in Sec.~\ref{NIST_Pool_Fires_Plume_Temps}. The species concentrations are found in Figs.~\ref{NIST_Pool_Fires_Acetone} through~\ref{NIST_Pool_Fires_Propane_34kW} on the following pages.

\newpage

\begin{figure}[p]
\begin{tabular*}{\textwidth}{l@{\extracolsep{\fill}}r}
\includegraphics[height=2.15in]{SCRIPT_FIGURES/NIST_Pool_Fires/NIST_Acetone_O2_CL} &
\includegraphics[height=2.15in]{SCRIPT_FIGURES/NIST_Pool_Fires/NIST_Acetone_Fuel_CL} \\
\includegraphics[height=2.15in]{SCRIPT_FIGURES/NIST_Pool_Fires/NIST_Acetone_CO2_CL} &
\includegraphics[height=2.15in]{SCRIPT_FIGURES/NIST_Pool_Fires/NIST_Acetone_CO_CL}    \\
\includegraphics[height=2.15in]{SCRIPT_FIGURES/NIST_Pool_Fires/NIST_Acetone_H2O_CL} &
\includegraphics[height=2.15in]{SCRIPT_FIGURES/NIST_Pool_Fires/NIST_Acetone_H2_CL} \\
\includegraphics[height=2.15in]{SCRIPT_FIGURES/NIST_Pool_Fires/NIST_Acetone_N2_CL} &
\includegraphics[height=2.15in]{SCRIPT_FIGURES/NIST_Pool_Fires/NIST_Acetone_Soot_CL}
\end{tabular*}
\caption[NIST Pool Fires, centerline product species, acetone]{NIST Pool Fires, centerline product species, acetone.}
\label{NIST_Pool_Fires_Acetone}
\end{figure}

\begin{figure}[p]
\begin{tabular*}{\textwidth}{l@{\extracolsep{\fill}}r}
\includegraphics[height=2.15in]{SCRIPT_FIGURES/NIST_Pool_Fires/NIST_Ethanol_O2_CL} &
\includegraphics[height=2.15in]{SCRIPT_FIGURES/NIST_Pool_Fires/NIST_Ethanol_Fuel_CL} \\
\includegraphics[height=2.15in]{SCRIPT_FIGURES/NIST_Pool_Fires/NIST_Ethanol_CO2_CL} &
\includegraphics[height=2.15in]{SCRIPT_FIGURES/NIST_Pool_Fires/NIST_Ethanol_CO_CL}    \\
\includegraphics[height=2.15in]{SCRIPT_FIGURES/NIST_Pool_Fires/NIST_Ethanol_H2O_CL} &
\includegraphics[height=2.15in]{SCRIPT_FIGURES/NIST_Pool_Fires/NIST_Ethanol_H2_CL} \\
\includegraphics[height=2.15in]{SCRIPT_FIGURES/NIST_Pool_Fires/NIST_Ethanol_N2_CL} &
\includegraphics[height=2.15in]{SCRIPT_FIGURES/NIST_Pool_Fires/NIST_Ethanol_Soot_CL}
\end{tabular*}
\caption[NIST Pool Fires, centerline product species, ethanol]{NIST Pool Fires, centerline product species, ethanol.}
\label{NIST_Pool_Fires_Ethanol}
\end{figure}

\begin{figure}[p]
\begin{tabular*}{\textwidth}{l@{\extracolsep{\fill}}r}
\includegraphics[height=2.15in]{SCRIPT_FIGURES/NIST_Pool_Fires/NIST_Methanol_O2_CL} &
\includegraphics[height=2.15in]{SCRIPT_FIGURES/NIST_Pool_Fires/NIST_Methanol_Fuel_CL} \\
\includegraphics[height=2.15in]{SCRIPT_FIGURES/NIST_Pool_Fires/NIST_Methanol_CO2_CL} &
\includegraphics[height=2.15in]{SCRIPT_FIGURES/NIST_Pool_Fires/NIST_Methanol_CO_CL}    \\
\includegraphics[height=2.15in]{SCRIPT_FIGURES/NIST_Pool_Fires/NIST_Methanol_H2O_CL} &
\includegraphics[height=2.15in]{SCRIPT_FIGURES/NIST_Pool_Fires/NIST_Methanol_H2_CL} \\
\includegraphics[height=2.15in]{SCRIPT_FIGURES/NIST_Pool_Fires/NIST_Methanol_N2_CL} &
\includegraphics[height=2.15in]{SCRIPT_FIGURES/NIST_Pool_Fires/NIST_Methanol_Soot_CL}
\end{tabular*}
\caption[NIST Pool Fires, centerline product species, methanol]{NIST Pool Fires, centerline product species, methanol.}
\label{NIST_Pool_Fires_Methanol}
\end{figure}

\begin{figure}[p]
\begin{tabular*}{\textwidth}{l@{\extracolsep{\fill}}r}
\includegraphics[height=2.15in]{SCRIPT_FIGURES/NIST_Pool_Fires/NIST_Methane_O2_CL} &
\includegraphics[height=2.15in]{SCRIPT_FIGURES/NIST_Pool_Fires/NIST_Methane_Fuel_CL} \\
\includegraphics[height=2.15in]{SCRIPT_FIGURES/NIST_Pool_Fires/NIST_Methane_CO2_CL} &
\includegraphics[height=2.15in]{SCRIPT_FIGURES/NIST_Pool_Fires/NIST_Methane_CO_CL}    \\
\includegraphics[height=2.15in]{SCRIPT_FIGURES/NIST_Pool_Fires/NIST_Methane_H2O_CL} &
\includegraphics[height=2.15in]{SCRIPT_FIGURES/NIST_Pool_Fires/NIST_Methane_H2_CL} \\
\includegraphics[height=2.15in]{SCRIPT_FIGURES/NIST_Pool_Fires/NIST_Methane_N2_CL} &
\includegraphics[height=2.15in]{SCRIPT_FIGURES/NIST_Pool_Fires/NIST_Methane_Soot_CL}
\end{tabular*}
\caption[NIST Pool Fires, centerline product species, methane]{NIST Pool Fires, centerline product species, methane.}
\label{NIST_Pool_Fires_Methane}
\end{figure}

\begin{figure}[p]
\begin{tabular*}{\textwidth}{l@{\extracolsep{\fill}}r}
\includegraphics[height=2.15in]{SCRIPT_FIGURES/NIST_Pool_Fires/NIST_Propane_20kW_O2_CL} &
\includegraphics[height=2.15in]{SCRIPT_FIGURES/NIST_Pool_Fires/NIST_Propane_20kW_Fuel_CL} \\
\includegraphics[height=2.15in]{SCRIPT_FIGURES/NIST_Pool_Fires/NIST_Propane_20kW_CO2_CL} &
\includegraphics[height=2.15in]{SCRIPT_FIGURES/NIST_Pool_Fires/NIST_Propane_20kW_CO_CL}    \\
\includegraphics[height=2.15in]{SCRIPT_FIGURES/NIST_Pool_Fires/NIST_Propane_20kW_H2O_CL} &
\includegraphics[height=2.15in]{SCRIPT_FIGURES/NIST_Pool_Fires/NIST_Propane_20kW_H2_CL} \\
\includegraphics[height=2.15in]{SCRIPT_FIGURES/NIST_Pool_Fires/NIST_Propane_20kW_N2_CL} &
\includegraphics[height=2.15in]{SCRIPT_FIGURES/NIST_Pool_Fires/NIST_Propane_20kW_Soot_CL}
\end{tabular*}
\caption[NIST Pool Fires, centerline product species, propane, 20 kW]{NIST Pool Fires, centerline product species, propane, 20 kW.}
\label{NIST_Pool_Fires_Propane_20kW}
\end{figure}

\begin{figure}[p]
\begin{tabular*}{\textwidth}{l@{\extracolsep{\fill}}r}
\includegraphics[height=2.15in]{SCRIPT_FIGURES/NIST_Pool_Fires/NIST_Propane_34kW_O2_CL} &
\includegraphics[height=2.15in]{SCRIPT_FIGURES/NIST_Pool_Fires/NIST_Propane_34kW_Fuel_CL} \\
\includegraphics[height=2.15in]{SCRIPT_FIGURES/NIST_Pool_Fires/NIST_Propane_34kW_CO2_CL} &
\includegraphics[height=2.15in]{SCRIPT_FIGURES/NIST_Pool_Fires/NIST_Propane_34kW_CO_CL}    \\
\includegraphics[height=2.15in]{SCRIPT_FIGURES/NIST_Pool_Fires/NIST_Propane_34kW_H2O_CL} &
\includegraphics[height=2.15in]{SCRIPT_FIGURES/NIST_Pool_Fires/NIST_Propane_34kW_H2_CL} \\
\includegraphics[height=2.15in]{SCRIPT_FIGURES/NIST_Pool_Fires/NIST_Propane_34kW_N2_CL} &
\includegraphics[height=2.15in]{SCRIPT_FIGURES/NIST_Pool_Fires/NIST_Propane_34kW_Soot_CL}
\end{tabular*}
\caption[NIST Pool Fires, centerline product species, propane, 34 kW]{NIST Pool Fires, centerline product species, propane, 34 kW.}
\label{NIST_Pool_Fires_Propane_34kW}
\end{figure}
\clearpage

%\subsection{PRISME DOOR Experiments}
%
%Each compartment in the PRISME DOOR experiments contained carbon monoxide measurements in the upper (haut) and lower (bas) layers.
%
%\begin{figure}[!ht]
%\begin{tabular*}{\textwidth}{l@{\extracolsep{\fill}}r}
%\includegraphics[height=2.15in]{SCRIPT_FIGURES/PRISME/PRS_D1_Room_1_CO} &
%\includegraphics[height=2.15in]{SCRIPT_FIGURES/PRISME/PRS_D2_Room_1_CO} \\
%\includegraphics[height=2.15in]{SCRIPT_FIGURES/PRISME/PRS_D3_Room_1_CO} &
%\includegraphics[height=2.15in]{SCRIPT_FIGURES/PRISME/PRS_D4_Room_1_CO} \\
%\includegraphics[height=2.15in]{SCRIPT_FIGURES/PRISME/PRS_D5_Room_1_CO} &
%\includegraphics[height=2.15in]{SCRIPT_FIGURES/PRISME/PRS_D6_Room_1_CO}
%\end{tabular*}
%\label{PRISME_CO_1}
%\end{figure}
%
%\begin{figure}[p]
%\begin{tabular*}{\textwidth}{l@{\extracolsep{\fill}}r}
%\includegraphics[height=2.15in]{SCRIPT_FIGURES/PRISME/PRS_D1_Room_2_CO} &
%\includegraphics[height=2.15in]{SCRIPT_FIGURES/PRISME/PRS_D2_Room_2_CO} \\
%\includegraphics[height=2.15in]{SCRIPT_FIGURES/PRISME/PRS_D3_Room_2_CO} &
%\includegraphics[height=2.15in]{SCRIPT_FIGURES/PRISME/PRS_D4_Room_2_CO} \\
%\includegraphics[height=2.15in]{SCRIPT_FIGURES/PRISME/PRS_D5_Room_2_CO} &
%\includegraphics[height=2.15in]{SCRIPT_FIGURES/PRISME/PRS_D6_Room_2_CO}
%\end{tabular*}
%\label{PRISME_CO_2}
%\end{figure}
%
%\clearpage

\subsection{Summary, Products of Incomplete Combustion}
\label{Carbon Monoxide Concentration}

\begin{figure}[h]
\begin{center}
\begin{tabular}{c}
\includegraphics[height=4.0in]{SCRIPT_FIGURES/ScatterPlots/FDS_Carbon_Monoxide_Concentration}
\end{tabular}
\end{center}
\caption[Summary of carbon monoxide predictions]{Summary of carbon monoxide predictions.}
\label{Summary_CO_Concentration}
\end{figure}

\clearpage

\section{Helium Release in a Reduced Scale Garage Geometry}
\label{Species Concentration}

FDS simulations were performed to predict the helium release and dispersion in a reduced scale garage geometry. The figures on the following pages show the comparison between the FDS predictions and the measured values for the eighteen experiments. Table~\ref{NIST_He_2009_Parameters} lists the experimental parameters, including the release duration, release location (21~cm off the floor at the center of the compartment, 21~cm off the floor and 5~cm from the center of the rear wall, and 2.5~cm below the ceiling at the center of the compartment), and the leak area (single small vent, 2.4~cm by 2.4~cm, at the center of the front wall, single large vent, 3.05~cm by 3.05~cm, at the center of the front wall, and a pair of vents, 2.15~cm by 2.15~cm, centered on the front wall, 2.5~cm from the floor and ceiling, respectively). The seven sensors were located 37.5~cm from the front and side walls, at heights of 9~cm, 19~cm, 28~cm, 37~cm, 47~cm, 56~cm, and 65~cm off the floor. In the figures on the following pages, the highest concentrations correspond to the highest measurement locations.

\begin{table}[h]
\centering
\caption[Test parameters of the NIST\_He\_2009 experiments]{Test parameters of the NIST\_He\_2009 experiments.}
\begin{tabular}{|c|c|c|c|}
\hline
Test         &  Release          &  Release          &  Leak                         \\
Label        &  Duration (h)     &  Location         &  Configuration                \\ \hline \hline
3600-LC-SSV  &  1                &  Lower Center     &  Single Small Vent            \\ \hline
3600-LC-SLV  &  1                &  Lower Center     &  Single Large Vent            \\ \hline
3600-LC-ULV  &  1                &  Lower Center     &  Dual Vents                   \\ \hline
3600-LR-SSV  &  1                &  Lower Rear       &  Single Small Vent            \\ \hline
3600-LR-SLV  &  1                &  Lower Rear       &  Single Large Vent            \\ \hline
3600-LR-ULV  &  1                &  Lower Rear       &  Dual Vents                   \\ \hline
3600-UC-SSV  &  1                &  Upper Center     &  Single Small Vent            \\ \hline
3600-UC-SLV  &  1                &  Upper Center     &  Single Large Vent            \\ \hline
3600-UC-ULV  &  1                &  Upper Center     &  Dual Vents                   \\ \hline
14400-LC-SSV &  4                &  Lower Center     &  Single Small Vent            \\ \hline
14400-LC-SLV &  4                &  Lower Center     &  Single Large Vent            \\ \hline
14400-LC-ULV &  4                &  Lower Center     &  Dual Vents                   \\ \hline
14400-LR-SSV &  4                &  Lower Rear       &  Single Small Vent            \\ \hline
14400-LR-SLV &  4                &  Lower Rear       &  Single Large Vent            \\ \hline
14400-LR-ULV &  4                &  Lower Rear       &  Dual Vents                   \\ \hline
14400-UC-SSV &  4                &  Upper Center     &  Single Small Vent            \\ \hline
14400-UC-SLV &  4                &  Upper Center     &  Single Large Vent            \\ \hline
14400-UC-ULV &  4                &  Upper Center     &  Dual Vents                   \\ \hline
\end{tabular}
\label{NIST_He_2009_Parameters}
\end{table}

\newpage

\begin{figure}[p]
\begin{tabular*}{\textwidth}{l@{\extracolsep{\fill}}r}
\includegraphics[height=2.15in]{SCRIPT_FIGURES/NIST_He_2009/NIST_He_3600_LC_SSV} &
\includegraphics[height=2.15in]{SCRIPT_FIGURES/NIST_He_2009/NIST_He_3600_LC_SLV} \\
\includegraphics[height=2.15in]{SCRIPT_FIGURES/NIST_He_2009/NIST_He_3600_LC_ULV} &
\includegraphics[height=2.15in]{SCRIPT_FIGURES/NIST_He_2009/NIST_He_3600_LR_SSV} \\
\includegraphics[height=2.15in]{SCRIPT_FIGURES/NIST_He_2009/NIST_He_3600_LR_SLV} &
\includegraphics[height=2.15in]{SCRIPT_FIGURES/NIST_He_2009/NIST_He_3600_LR_ULV}
\end{tabular*}
\caption[Results of the NIST\_He\_2009 experiments]{Comparison of measured (solid lines) and predicted (dashed lines) helium concentrations in the NIST\_He\_2009 experiments.}
\label{NIST_Hydrogen_Species_1}
\end{figure}

\begin{figure}[p]
\begin{tabular*}{\textwidth}{l@{\extracolsep{\fill}}r}
\includegraphics[height=2.15in]{SCRIPT_FIGURES/NIST_He_2009/NIST_He_3600_UC_SSV} &
\includegraphics[height=2.15in]{SCRIPT_FIGURES/NIST_He_2009/NIST_He_3600_UC_SLV} \\
\includegraphics[height=2.15in]{SCRIPT_FIGURES/NIST_He_2009/NIST_He_3600_UC_ULV} &
\includegraphics[height=2.15in]{SCRIPT_FIGURES/NIST_He_2009/NIST_He_14400_LC_SSV} \\
\includegraphics[height=2.15in]{SCRIPT_FIGURES/NIST_He_2009/NIST_He_14400_LC_SLV} &
\includegraphics[height=2.15in]{SCRIPT_FIGURES/NIST_He_2009/NIST_He_14400_LC_ULV}
\end{tabular*}
\caption[Results of the NIST\_He\_2009 experiments]{Comparison of measured (solid lines) and predicted (dashed lines) helium concentrations in the NIST\_He\_2009 experiments.}
\label{NIST_Hydrogen_Species_2}
\end{figure}

\begin{figure}[p]
\begin{tabular*}{\textwidth}{l@{\extracolsep{\fill}}r}
\includegraphics[height=2.15in]{SCRIPT_FIGURES/NIST_He_2009/NIST_He_14400_LR_SSV} &
\includegraphics[height=2.15in]{SCRIPT_FIGURES/NIST_He_2009/NIST_He_14400_LR_SLV} \\
\includegraphics[height=2.15in]{SCRIPT_FIGURES/NIST_He_2009/NIST_He_14400_LR_ULV} &
\includegraphics[height=2.15in]{SCRIPT_FIGURES/NIST_He_2009/NIST_He_14400_UC_SSV} \\
\includegraphics[height=2.15in]{SCRIPT_FIGURES/NIST_He_2009/NIST_He_14400_UC_SLV} &
\includegraphics[height=2.15in]{SCRIPT_FIGURES/NIST_He_2009/NIST_He_14400_UC_ULV}
\end{tabular*}
\caption[Results of the NIST\_He\_2009 experiments]{Comparison of measured (solid lines) and predicted (dashed lines) helium concentrations in the NIST\_He\_2009 experiments.}
\label{NIST_Hydrogen_Species_3}
\end{figure}

\begin{figure}[p]
\begin{center}
\begin{tabular}{c}
\includegraphics[height=4.0in]{SCRIPT_FIGURES/ScatterPlots/FDS_Species_Concentration}
\end{tabular}
\end{center}
\caption[Summary of species concentration predictions]{Summary of species concentration predictions.}
\label{Summary_Species_Concentration}
\end{figure}




% !TEX root = FDS_Validation_Guide.tex

\chapter{Pressure}

In FDS, the pressure is decomposed into a temporally-varying background pressure plus a temporally and spatially-varying perturbation that drives the flow. The former can be thought of as the ``over-pressure'' which increases if heat is introduced into a closed compartment. In real buildings, leakage and ventilation affect the compartment ``over-pressure'' along with the fire.

\section{FM/FPRF Datacenter Experiments}

Measurements made during flow mapping in the FM datacenter mockup included two pairs of differential pressure transmitters. One pair measured the pressure difference between the subfloor (SF) and the cold aisle (CA).  The other pair measured the pressure difference between the hot aisle (HA) and the ceiling plenum (CP). A comparison of measured and predicted pressures for the exhaust rate (78 ACH) and high exhaust rate (265 ACH) tests  is shown in the figure below.

\begin{figure}[!ht]
\begin{tabular*}{\textwidth}{l@{\extracolsep{\fill}}r}
\includegraphics[height=2.15in]{SCRIPT_FIGURES/FM_FPRF_Datacenter/FM_Datacenter_Veltest_Low_Pres} &
\includegraphics[height=2.15in]{SCRIPT_FIGURES/FM_FPRF_Datacenter/FM_Datacenter_Veltest_High_Pres}
\end{tabular*}
\caption[FM/FPRF Data Center, differential pressure]{FM/FPRF Data Center, differential pressure (Left - low exhaust rate, Right - high exhaust rate)}
\label{FM_FPRF_Datacenter_Pres}
\end{figure}

\section{NIST/NRC Experiments}

Comparisons between measured and predicted pressures for the NIST/NRC series are shown on the following pages. For those tests in which the door to the compartment was open, the over-pressures were only a few Pascals, whereas when the door was closed, the over-pressures were several hundred Pascals. The pressure within the compartment was measured at a single point, near the floor. For the simulations of the closed door tests, the compartment is assumed to leak via a small uniform flow distributed over the walls and ceiling. The flow rate is calculated based on the assumption that the leakage rate is proportional to the measured leakage area times the square root of compartment over-pressure.

Note that for the closed door tests, there is often a dramatic drop in the predicted compartment pressure. This is the result of the assumption in FDS that the heat release rate is decreased to zero in one second at the time in the experiment when the fuel flow was stopped for safety reasons.  In reality, the fire did not extinguish immediately because there was an excess of fuel in the pan following the flow stoppage. For the purpose of model comparison, the peak over-pressures are compared in the closed door tests, and the peak (albeit small) under-pressures are compared in the open door tests.

\newpage

\begin{figure}[p]
\begin{tabular*}{\textwidth}{l@{\extracolsep{\fill}}r}
\includegraphics[height=2.15in]{SCRIPT_FIGURES/NIST_NRC/NIST_NRC_01_Pressure} &
\includegraphics[height=2.15in]{SCRIPT_FIGURES/NIST_NRC/NIST_NRC_07_Pressure} \\
\includegraphics[height=2.15in]{SCRIPT_FIGURES/NIST_NRC/NIST_NRC_02_Pressure} &
\includegraphics[height=2.15in]{SCRIPT_FIGURES/NIST_NRC/NIST_NRC_08_Pressure} \\
\includegraphics[height=2.15in]{SCRIPT_FIGURES/NIST_NRC/NIST_NRC_04_Pressure} &
\includegraphics[height=2.15in]{SCRIPT_FIGURES/NIST_NRC/NIST_NRC_10_Pressure} \\
\includegraphics[height=2.15in]{SCRIPT_FIGURES/NIST_NRC/NIST_NRC_13_Pressure} &
\includegraphics[height=2.15in]{SCRIPT_FIGURES/NIST_NRC/NIST_NRC_16_Pressure}
\end{tabular*}
\caption[NIST/NRC experiments, compartment pressure, Tests 1, 2, 4, 7, 8, 10, 13, 16]{NIST/NRC experiments, compartment pressure, Tests 1, 2, 4, 7, 8, 10, 13, 16.}
\label{NIST_NRC_Pressure_Closed}
\end{figure}

\begin{figure}[p]
\begin{tabular*}{\textwidth}{l@{\extracolsep{\fill}}r}
\includegraphics[height=2.15in]{SCRIPT_FIGURES/NIST_NRC/NIST_NRC_17_Pressure} &
   \\
\includegraphics[height=2.15in]{SCRIPT_FIGURES/NIST_NRC/NIST_NRC_03_Pressure} &
\includegraphics[height=2.15in]{SCRIPT_FIGURES/NIST_NRC/NIST_NRC_09_Pressure} \\
\includegraphics[height=2.15in]{SCRIPT_FIGURES/NIST_NRC/NIST_NRC_05_Pressure} &
\includegraphics[height=2.15in]{SCRIPT_FIGURES/NIST_NRC/NIST_NRC_14_Pressure} \\
\includegraphics[height=2.15in]{SCRIPT_FIGURES/NIST_NRC/NIST_NRC_15_Pressure} &
\includegraphics[height=2.15in]{SCRIPT_FIGURES/NIST_NRC/NIST_NRC_18_Pressure}
\end{tabular*}
\caption[NIST/NRC experiments, compartment pressure, Tests 3, 5, 9, 14, 15, 17, 18]{NIST/NRC experiments, compartment pressure, Tests 3, 5, 9, 14, 15, 17, 18.}
\label{NIST_NRC_Pressure_Open}
\end{figure}

\clearpage

\section{LLNL Enclosure Experiments}

The reported compartment pressure in the LLNL Enclosure experiments was taken near the ceiling of the compartment, 0.6~m from the wall including the exhaust duct, and 0.6~m from the wall opposite the wall with the door. Further details of the experiments can be found in Sec.~\ref{LLNL_Enclosure_Description}.

In the figures on the following pages, the open circles represent the measured pressure; the solid line represents the predicted pressure. The predicted pressures are time-averaged over a time interval of 30~s, whereas the measurements appear to be instantaneous values separated by hundreds or thousands of seconds. Because of this, the short-duration pressure spike that is typical of fires within relatively tight compartments is seen in the model prediction but not necessarily the measured data. The comparison of measurement and prediction is based on the final few pressure points, not the initial spike.

The results of all 64 experiments are plotted for completeness, but a few of the results were excluded from the computation of the summary statistics because the fire self-extinguished near the time of the last pressure measurement, sometimes leading to a reported final pressure being less than the initial pressure, typical when there is a sudden decrease in the heat release rate. In other cases, the difference between initial and final measured pressure was too small to make a meaningful comparison.

For cases where the door to the compartment was open, the measured gauge pressures at the start of the experiment ranged from 0~Pa to 10~Pa. There is not enough information in the test report to explain why the starting pressures were not 0~Pa; thus, the measured pressures were adjusted so that the starting pressure is 0~Pa.

\newpage

\begin{figure}[p]
\begin{tabular*}{\textwidth}{l@{\extracolsep{\fill}}r}
\includegraphics[height=2.15in]{SCRIPT_FIGURES/LLNL_Enclosure/LLNL_01_Pres} &
\includegraphics[height=2.15in]{SCRIPT_FIGURES/LLNL_Enclosure/LLNL_02_Pres} \\
\includegraphics[height=2.15in]{SCRIPT_FIGURES/LLNL_Enclosure/LLNL_03_Pres} &
\includegraphics[height=2.15in]{SCRIPT_FIGURES/LLNL_Enclosure/LLNL_04_Pres} \\
\includegraphics[height=2.15in]{SCRIPT_FIGURES/LLNL_Enclosure/LLNL_05_Pres} &
\includegraphics[height=2.15in]{SCRIPT_FIGURES/LLNL_Enclosure/LLNL_06_Pres} \\
\includegraphics[height=2.15in]{SCRIPT_FIGURES/LLNL_Enclosure/LLNL_07_Pres} &
\includegraphics[height=2.15in]{SCRIPT_FIGURES/LLNL_Enclosure/LLNL_08_Pres}
\end{tabular*}
\caption[LLNL Enclosure experiments, compartment pressure, Tests 1-8]{LLNL Enclosure experiments, compartment pressure, Tests 1-8.}
\label{LLNL_Enclosure_Pres_1}
\end{figure}

\begin{figure}[p]
\begin{tabular*}{\textwidth}{l@{\extracolsep{\fill}}r}
\includegraphics[height=2.15in]{SCRIPT_FIGURES/LLNL_Enclosure/LLNL_09_Pres} &
\includegraphics[height=2.15in]{SCRIPT_FIGURES/LLNL_Enclosure/LLNL_10_Pres} \\
\includegraphics[height=2.15in]{SCRIPT_FIGURES/LLNL_Enclosure/LLNL_11_Pres} &
\includegraphics[height=2.15in]{SCRIPT_FIGURES/LLNL_Enclosure/LLNL_12_Pres} \\
\includegraphics[height=2.15in]{SCRIPT_FIGURES/LLNL_Enclosure/LLNL_13_Pres} &
\includegraphics[height=2.15in]{SCRIPT_FIGURES/LLNL_Enclosure/LLNL_14_Pres} \\
\includegraphics[height=2.15in]{SCRIPT_FIGURES/LLNL_Enclosure/LLNL_15_Pres} &
\includegraphics[height=2.15in]{SCRIPT_FIGURES/LLNL_Enclosure/LLNL_16_Pres}
\end{tabular*}
\caption[LLNL Enclosure experiments, compartment pressure, Tests 9-16]{LLNL Enclosure experiments, compartment pressure, Tests 9-16.}
\label{LLNL_Enclosure_Pres_2}
\end{figure}

\begin{figure}[p]
\begin{tabular*}{\textwidth}{l@{\extracolsep{\fill}}r}
\includegraphics[height=2.15in]{SCRIPT_FIGURES/LLNL_Enclosure/LLNL_17_Pres} &
\includegraphics[height=2.15in]{SCRIPT_FIGURES/LLNL_Enclosure/LLNL_18_Pres} \\
\includegraphics[height=2.15in]{SCRIPT_FIGURES/LLNL_Enclosure/LLNL_19_Pres} &
\includegraphics[height=2.15in]{SCRIPT_FIGURES/LLNL_Enclosure/LLNL_20_Pres} \\
\includegraphics[height=2.15in]{SCRIPT_FIGURES/LLNL_Enclosure/LLNL_21_Pres} &
\includegraphics[height=2.15in]{SCRIPT_FIGURES/LLNL_Enclosure/LLNL_22_Pres} \\
\includegraphics[height=2.15in]{SCRIPT_FIGURES/LLNL_Enclosure/LLNL_23_Pres} &
\includegraphics[height=2.15in]{SCRIPT_FIGURES/LLNL_Enclosure/LLNL_24_Pres}
\end{tabular*}
\caption[LLNL Enclosure experiments, compartment pressure, Tests 17-24]{LLNL Enclosure experiments, compartment pressure, Tests 17-24.}
\label{LLNL_Enclosure_Pres_3}
\end{figure}

\begin{figure}[p]
\begin{tabular*}{\textwidth}{l@{\extracolsep{\fill}}r}
\includegraphics[height=2.15in]{SCRIPT_FIGURES/LLNL_Enclosure/LLNL_25_Pres} &
\includegraphics[height=2.15in]{SCRIPT_FIGURES/LLNL_Enclosure/LLNL_26_Pres} \\
\includegraphics[height=2.15in]{SCRIPT_FIGURES/LLNL_Enclosure/LLNL_27_Pres} &
\includegraphics[height=2.15in]{SCRIPT_FIGURES/LLNL_Enclosure/LLNL_28_Pres} \\
\includegraphics[height=2.15in]{SCRIPT_FIGURES/LLNL_Enclosure/LLNL_29_Pres} &
\includegraphics[height=2.15in]{SCRIPT_FIGURES/LLNL_Enclosure/LLNL_30_Pres} \\
\includegraphics[height=2.15in]{SCRIPT_FIGURES/LLNL_Enclosure/LLNL_31_Pres} &
\includegraphics[height=2.15in]{SCRIPT_FIGURES/LLNL_Enclosure/LLNL_32_Pres}
\end{tabular*}
\caption[LLNL Enclosure experiments, compartment pressure, Tests 25-32]{LLNL Enclosure experiments, compartment pressure, Tests 25-32.}
\label{LLNL_Enclosure_Pres_4}
\end{figure}

\begin{figure}[p]
\begin{tabular*}{\textwidth}{l@{\extracolsep{\fill}}r}
\includegraphics[height=2.15in]{SCRIPT_FIGURES/LLNL_Enclosure/LLNL_33_Pres} &
\includegraphics[height=2.15in]{SCRIPT_FIGURES/LLNL_Enclosure/LLNL_34_Pres} \\
\includegraphics[height=2.15in]{SCRIPT_FIGURES/LLNL_Enclosure/LLNL_35_Pres} &
\includegraphics[height=2.15in]{SCRIPT_FIGURES/LLNL_Enclosure/LLNL_36_Pres} \\
\includegraphics[height=2.15in]{SCRIPT_FIGURES/LLNL_Enclosure/LLNL_37_Pres} &
\includegraphics[height=2.15in]{SCRIPT_FIGURES/LLNL_Enclosure/LLNL_38_Pres} \\
\includegraphics[height=2.15in]{SCRIPT_FIGURES/LLNL_Enclosure/LLNL_39_Pres} &
\includegraphics[height=2.15in]{SCRIPT_FIGURES/LLNL_Enclosure/LLNL_40_Pres}
\end{tabular*}
\caption[LLNL Enclosure experiments, compartment pressure, Tests 33-40]{LLNL Enclosure experiments, compartment pressure, Tests 33-40.}
\label{LLNL_Enclosure_Pres_5}
\end{figure}

\begin{figure}[p]
\begin{tabular*}{\textwidth}{l@{\extracolsep{\fill}}r}
\includegraphics[height=2.15in]{SCRIPT_FIGURES/LLNL_Enclosure/LLNL_41_Pres} &
\includegraphics[height=2.15in]{SCRIPT_FIGURES/LLNL_Enclosure/LLNL_42_Pres} \\
\includegraphics[height=2.15in]{SCRIPT_FIGURES/LLNL_Enclosure/LLNL_43_Pres} &
\includegraphics[height=2.15in]{SCRIPT_FIGURES/LLNL_Enclosure/LLNL_44_Pres} \\
\includegraphics[height=2.15in]{SCRIPT_FIGURES/LLNL_Enclosure/LLNL_45_Pres} &
\includegraphics[height=2.15in]{SCRIPT_FIGURES/LLNL_Enclosure/LLNL_46_Pres} \\
\includegraphics[height=2.15in]{SCRIPT_FIGURES/LLNL_Enclosure/LLNL_47_Pres} &
\includegraphics[height=2.15in]{SCRIPT_FIGURES/LLNL_Enclosure/LLNL_48_Pres}
\end{tabular*}
\caption[LLNL Enclosure experiments, compartment pressure, Tests 41-48]{LLNL Enclosure experiments, compartment pressure, Tests 41-48.}
\label{LLNL_Enclosure_Pres_6}
\end{figure}

\begin{figure}[p]
\begin{tabular*}{\textwidth}{l@{\extracolsep{\fill}}r}
\includegraphics[height=2.15in]{SCRIPT_FIGURES/LLNL_Enclosure/LLNL_49_Pres} &
\includegraphics[height=2.15in]{SCRIPT_FIGURES/LLNL_Enclosure/LLNL_50_Pres} \\
\includegraphics[height=2.15in]{SCRIPT_FIGURES/LLNL_Enclosure/LLNL_51_Pres} &
\includegraphics[height=2.15in]{SCRIPT_FIGURES/LLNL_Enclosure/LLNL_52_Pres} \\
\includegraphics[height=2.15in]{SCRIPT_FIGURES/LLNL_Enclosure/LLNL_53_Pres} &
\includegraphics[height=2.15in]{SCRIPT_FIGURES/LLNL_Enclosure/LLNL_54_Pres} \\
\includegraphics[height=2.15in]{SCRIPT_FIGURES/LLNL_Enclosure/LLNL_55_Pres} &
\includegraphics[height=2.15in]{SCRIPT_FIGURES/LLNL_Enclosure/LLNL_56_Pres}
\end{tabular*}
\caption[LLNL Enclosure experiments, compartment pressure, Tests 49-56]{LLNL Enclosure experiments, compartment pressure, Tests 49-56.}
\label{LLNL_Enclosure_Pres_7}
\end{figure}

\begin{figure}[p]
\begin{tabular*}{\textwidth}{l@{\extracolsep{\fill}}r}
\includegraphics[height=2.15in]{SCRIPT_FIGURES/LLNL_Enclosure/LLNL_57_Pres} &
\includegraphics[height=2.15in]{SCRIPT_FIGURES/LLNL_Enclosure/LLNL_58_Pres} \\
\includegraphics[height=2.15in]{SCRIPT_FIGURES/LLNL_Enclosure/LLNL_59_Pres} &
\includegraphics[height=2.15in]{SCRIPT_FIGURES/LLNL_Enclosure/LLNL_60_Pres} \\
\includegraphics[height=2.15in]{SCRIPT_FIGURES/LLNL_Enclosure/LLNL_61_Pres} &
\includegraphics[height=2.15in]{SCRIPT_FIGURES/LLNL_Enclosure/LLNL_62_Pres} \\
\includegraphics[height=2.15in]{SCRIPT_FIGURES/LLNL_Enclosure/LLNL_63_Pres} &
\includegraphics[height=2.15in]{SCRIPT_FIGURES/LLNL_Enclosure/LLNL_64_Pres}
\end{tabular*}
\caption[LLNL Enclosure experiments, compartment pressure, Tests 57-64]{LLNL Enclosure experiments, compartment pressure, Tests 57-64.}
\label{LLNL_Enclosure_Pres_8}
\end{figure}



\clearpage

\section{PRISME DOOR Experiments}

The PRISME experiments were conducted in a relatively well-sealed set of compartments with a well-controlled ventilation system. Supply air was forced into and exhaust products extracted from the test compartments via two fans and a fairly extensive ventilation network. The air flow rates and nodal pressures were measured throughout the system. The FDS simulations included the ventilation system, and for each segment of the network a loss coefficient was calculated so as to match the initial conditions of the experiments. The plots to follow show the predicted and measured compartment pressures and supply and exhaust flows. These air flows were predicted by the model, based on the initial specification of the ventilation system.

\newpage

\begin{figure}[p]
\begin{tabular*}{\textwidth}{l@{\extracolsep{\fill}}r}
\includegraphics[height=2.15in]{SCRIPT_FIGURES/PRISME/PRS_D1_Room_1_Pressure} &
\includegraphics[height=2.15in]{SCRIPT_FIGURES/PRISME/PRS_D1_Room_1_Supply_Exhaust} \\
\includegraphics[height=2.15in]{SCRIPT_FIGURES/PRISME/PRS_D2_Room_1_Pressure} &
\includegraphics[height=2.15in]{SCRIPT_FIGURES/PRISME/PRS_D2_Room_1_Supply_Exhaust} \\
\includegraphics[height=2.15in]{SCRIPT_FIGURES/PRISME/PRS_D3_Room_1_Pressure} &
\includegraphics[height=2.15in]{SCRIPT_FIGURES/PRISME/PRS_D3_Room_1_Supply_Exhaust}
\end{tabular*}
\caption[PRISME DOOR, compartment pressure and supply/exhaust, Room 1, Tests 1-3]{PRISME DOOR, compartment pressure and supply/exhaust, Room 1, Tests 1-3.}
\label{PRISME_Room_1_Pressure_1-3}
\end{figure}

\begin{figure}[p]
\begin{tabular*}{\textwidth}{l@{\extracolsep{\fill}}r}
\includegraphics[height=2.15in]{SCRIPT_FIGURES/PRISME/PRS_D4_Room_1_Pressure} &
\includegraphics[height=2.15in]{SCRIPT_FIGURES/PRISME/PRS_D4_Room_1_Supply_Exhaust} \\
\includegraphics[height=2.15in]{SCRIPT_FIGURES/PRISME/PRS_D5_Room_1_Pressure} &
\includegraphics[height=2.15in]{SCRIPT_FIGURES/PRISME/PRS_D5_Room_1_Supply_Exhaust} \\
\includegraphics[height=2.15in]{SCRIPT_FIGURES/PRISME/PRS_D6_Room_1_Pressure} &
\includegraphics[height=2.15in]{SCRIPT_FIGURES/PRISME/PRS_D6_Room_1_Supply_Exhaust}
\end{tabular*}
\caption[PRISME DOOR, compartment pressure and supply/exhaust, Room 1, Tests 4-6]{PRISME DOOR, compartment pressure and supply/exhaust, Room 1, Tests 4-6.}
\label{PRISME_Room_1_Pressure_4-6}
\end{figure}

\begin{figure}[p]
\begin{tabular*}{\textwidth}{l@{\extracolsep{\fill}}r}
\includegraphics[height=2.15in]{SCRIPT_FIGURES/PRISME/PRS_D1_Room_2_Pressure} &
\includegraphics[height=2.15in]{SCRIPT_FIGURES/PRISME/PRS_D1_Room_2_Supply_Exhaust} \\
\includegraphics[height=2.15in]{SCRIPT_FIGURES/PRISME/PRS_D2_Room_2_Pressure} &
\includegraphics[height=2.15in]{SCRIPT_FIGURES/PRISME/PRS_D2_Room_2_Supply_Exhaust} \\
\includegraphics[height=2.15in]{SCRIPT_FIGURES/PRISME/PRS_D3_Room_2_Pressure} &
\includegraphics[height=2.15in]{SCRIPT_FIGURES/PRISME/PRS_D3_Room_2_Supply_Exhaust}
\end{tabular*}
\caption[PRISME DOOR, compartment pressure and supply/exhaust, Room 2, Tests 1-3]{PRISME DOOR, compartment pressure and supply/exhaust, Room 2, Tests 1-3.}
\label{PRISME_Room_2_Pressure_1-3}
\end{figure}

\begin{figure}[p]
\begin{tabular*}{\textwidth}{l@{\extracolsep{\fill}}r}
\includegraphics[height=2.15in]{SCRIPT_FIGURES/PRISME/PRS_D4_Room_2_Pressure} &
\includegraphics[height=2.15in]{SCRIPT_FIGURES/PRISME/PRS_D4_Room_2_Supply_Exhaust} \\
\includegraphics[height=2.15in]{SCRIPT_FIGURES/PRISME/PRS_D5_Room_2_Pressure} &
\includegraphics[height=2.15in]{SCRIPT_FIGURES/PRISME/PRS_D5_Room_2_Supply_Exhaust} \\
\includegraphics[height=2.15in]{SCRIPT_FIGURES/PRISME/PRS_D6_Room_2_Pressure} &
\includegraphics[height=2.15in]{SCRIPT_FIGURES/PRISME/PRS_D6_Room_2_Supply_Exhaust}
\end{tabular*}
\caption[PRISME DOOR, compartment pressure and supply/exhaust, Room 2, Tests 4-6]{PRISME DOOR, compartment pressure and supply/exhaust, Room 2, Tests 4-6.}
\label{PRISME_Room_2_Pressure_4-6}
\end{figure}


\clearpage

\section{PRISME SOURCE Experiments}

The PRISME SOURCE experiments were conducted in a single compartment with a well-controlled ventilation system. Supply air was forced into and exhaust products extracted from the test compartment via two fans and a fairly extensive ventilation network. The air flow rates and nodal pressures were measured throughout the system. The FDS simulations included the ventilation system, and for each segment of the network a loss coefficient was calculated so as to match the initial conditions of the experiments. The plots to follow show the predicted and measured compartment pressures and supply and exhaust flows. These air flows were predicted by the model, based on the initial specification of the ventilation system.

\newpage

\begin{figure}[p]
\begin{tabular*}{\textwidth}{l@{\extracolsep{\fill}}r}
\includegraphics[height=2.15in]{SCRIPT_FIGURES/PRISME/PRS_SI_D1_Room_2_Pressure} &
\includegraphics[height=2.15in]{SCRIPT_FIGURES/PRISME/PRS_SI_D1_Room_2_Supply_Exhaust} \\
\includegraphics[height=2.15in]{SCRIPT_FIGURES/PRISME/PRS_SI_D2_Room_2_Pressure} &
\includegraphics[height=2.15in]{SCRIPT_FIGURES/PRISME/PRS_SI_D2_Room_2_Supply_Exhaust} \\
\includegraphics[height=2.15in]{SCRIPT_FIGURES/PRISME/PRS_SI_D3_Room_2_Pressure} &
\includegraphics[height=2.15in]{SCRIPT_FIGURES/PRISME/PRS_SI_D3_Room_2_Supply_Exhaust} \\
\includegraphics[height=2.15in]{SCRIPT_FIGURES/PRISME/PRS_SI_D4_Room_2_Pressure} &
\includegraphics[height=2.15in]{SCRIPT_FIGURES/PRISME/PRS_SI_D4_Room_2_Supply_Exhaust}
\end{tabular*}
\caption[PRISME SOURCE, pressure and supply/exhaust flow rates, Tests 1, 2, 3 and 4]{PRISME SOURCE, pressure and supply/exhaust flow rates, Tests 1, 2, 3 and 4.}
\label{PRISME_SOURCE_Room_2_Pressure_1}
\end{figure}

\begin{figure}[p]
\begin{tabular*}{\textwidth}{l@{\extracolsep{\fill}}r}
\includegraphics[height=2.15in]{SCRIPT_FIGURES/PRISME/PRS_SI_D5_Room_2_Pressure} &
\includegraphics[height=2.15in]{SCRIPT_FIGURES/PRISME/PRS_SI_D5_Room_2_Supply_Exhaust} \\
\includegraphics[height=2.15in]{SCRIPT_FIGURES/PRISME/PRS_SI_D5a_Room_2_Pressure} &
\includegraphics[height=2.15in]{SCRIPT_FIGURES/PRISME/PRS_SI_D5a_Room_2_Supply_Exhaust} \\
\includegraphics[height=2.15in]{SCRIPT_FIGURES/PRISME/PRS_SI_D6_Room_2_Pressure} &
\includegraphics[height=2.15in]{SCRIPT_FIGURES/PRISME/PRS_SI_D6_Room_2_Supply_Exhaust} \\
\includegraphics[height=2.15in]{SCRIPT_FIGURES/PRISME/PRS_SI_D6a_Room_2_Pressure} &
\includegraphics[height=2.15in]{SCRIPT_FIGURES/PRISME/PRS_SI_D6a_Room_2_Supply_Exhaust}
\end{tabular*}
\caption[PRISME SOURCE, pressure and supply/exhaust flow rates, Tests 5, 5a, 6 and 6a]{PRISME SOURCE, pressure and supply/exhaust flow rates, Tests 5, 5a, 6 and 6a.}
\label{PRISME_SOURCE_Room_2_Pressure_2}
\end{figure}

\clearpage

\section{Pr\'{e}trel/IRSN Water Spray Experiments}

A description of the experiment is found in Sec.~\ref{PRISME_Description} and Refs.~\cite{Beji:FSJ2023,Pretrel:FSJ2017}. Briefly, a propane burner is placed in the corner of an 8.67~m long, 4.88~m wide, 3.9~m tall closed compartment equipped with two water spray nozzles that are activated after a hot gas layer has developed. Air is supplied at a rate of approximately 0.7~m$^3$/s. A sketch of the compartment is shown in Fig.~\ref{L4_sketch}. The propane burner has a steady heat release rate of approximately 292~kW. The water spray nozzles are positioned 297~cm above the floor and are activated at 6~min, 54~s and deactivated at 11~min, 30~s. The combined flow rate is 109~L/min with a spray angle of approximately 90$^\circ$ and median volumetric droplet diameter of 470~$\mu$m.

Shown in Fig.~\ref{Pretrel_water_spray_figs} are plots of the compartment pressure, exhaust and supply rates, CO$_2$ and O$_2$ volume fractions, and near ceiling temperature. The temperature is measured at a height of 390~cm at the southeast (SE) thermocouple array. The gas concentration measurements are made at this same location at heights of 89~cm (bas/low) and 314~cm (haut/high). The last plot in Fig.~\ref{Pretrel_water_spray_figs} displays the energy budget; that is, the amount of the fire's energy lost to the walls, the exhaust vent, and the water droplets. The method for measuring these quantities is given in Ref.~\cite{Pretrel:FSJ2017}.

\begin{figure}[!ht]
\setlength{\unitlength}{.0075in}
\begin{picture}(867,500)
\linethickness{1.0mm} 
\put(0,0){\framebox(867,488)[tc]{East Wall}} 
\linethickness{0.25mm} 
\put(360,244){\circle*{10}} 
\put(330,254){Nozzle} 
\put(360,244){\circle{300}} 
\put(660,244){\circle*{10}} 
\put(630,254){Nozzle} 
\put(660,244){\circle{300}} 
\put(0,0){\framebox(120,100)[c]{Burner}}
\put(20,224){\framebox(40,40)[c]{ }}
\put(15,205){Supply}
\put(40,270){\vector(0,1){30}}
\put(780,224){\framebox(40,40)[c]{ }}
\put(765,268){Exhaust}
\put(800,188){\vector(0,1){30}}
\put(248,240){\circle{5}} 
\put(223,215){TG NE} 
\put(432,238){\circle{5}} 
\put(407,213){TG CC} 
\put(648,332){\circle{5}} 
\put(600,307){CO$_2$, O$_2$, CO} 
\put(623,347){TG SE} 
\put(640,138){\circle{5}} 
\put(615,113){TG SW} 
\end{picture}
\caption[Plan view of the Pr\'{e}trel water spray experiments]{Plan view of the Pr\'{e}trel water spray experiments. The points labelled ``TG'' are the locations of thermocouple arrays.}
\label{L4_sketch}
\end{figure}

\newpage

\begin{figure}[p]
\begin{tabular*}{\textwidth}{l@{\extracolsep{\fill}}r}
\includegraphics[height=2.15in]{SCRIPT_FIGURES/PRISME/PR2_FES_PA2a_O2} &
\includegraphics[height=2.15in]{SCRIPT_FIGURES/PRISME/PR2_FES_PA2a_CO2} \\
\includegraphics[height=2.15in]{SCRIPT_FIGURES/PRISME/PR2_FES_PA2a_Pressure} &
\includegraphics[height=2.15in]{SCRIPT_FIGURES/PRISME/PR2_FES_PA2a_Supply_Exhaust} \\
\multicolumn{2}{c}{\includegraphics[height=2.15in]{SCRIPT_FIGURES/PRISME/PR2_FES_PA2a_TG_L4_SE_390}} \\
\multicolumn{2}{c}{\includegraphics[height=2.15in]{SCRIPT_FIGURES/PRISME/PR2_FES_PA2a_Energy_Balance}} 
\end{tabular*}
\caption[Various measurements from the Pr\'{e}trel water spray experiment]{Various measurements from the Pr\'{e}trel water spray experiment.}
\label{Pretrel_water_spray_figs}
\end{figure}



\clearpage

\section{UL/NIJ House Experiments}

A description and drawings from these experiments are included in Sec.~\ref{UL_NIJ_Description}

For both the one and two-story house experiments, pressures were measured at three elevations in various locations. All pressure taps were installed within 0.3~m of a wall. In the single-story house, the taps were located 0.3~m (1~ft), 1.2~m (4~ft), and 2.1~m (7~ft) below the ceiling. In the two-story house, the taps were located 0.3~m (1~ft), 2.4~m (8~ft), and 4.6~m (15~ft) below the ceiling in the family room atrium (9PT), and 0.3~m (1~ft), 1.2~m (4~ft), and 2.1~m (7~ft) below the ceiling in all other rooms. Note that for the two-story house, Test~6, the den was closed and no pressure measurements are reported for this room.

The plots on the following pages compare the measured pressures with corresponding model predictions. Note that the pressures are reported relative to the ambient pressure at the given elevation. That is, all reported pressures are zero at the time of ignition.

\newpage

\begin{figure}[p]
\begin{tabular*}{\textwidth}{l@{\extracolsep{\fill}}r}
\includegraphics[height=2.15in]{SCRIPT_FIGURES/UL_NIJ_Houses/UL_NIJ_Single_Story_Gas_1_Pressure_1} &
\includegraphics[height=2.15in]{SCRIPT_FIGURES/UL_NIJ_Houses/UL_NIJ_Single_Story_Gas_1_Pressure_2} \\
\includegraphics[height=2.15in]{SCRIPT_FIGURES/UL_NIJ_Houses/UL_NIJ_Single_Story_Gas_1_Pressure_3} &
\includegraphics[height=2.15in]{SCRIPT_FIGURES/UL_NIJ_Houses/UL_NIJ_Single_Story_Gas_1_Pressure_4} \\
\includegraphics[height=2.15in]{SCRIPT_FIGURES/UL_NIJ_Houses/UL_NIJ_Single_Story_Gas_1_Pressure_5} &
\includegraphics[height=2.15in]{SCRIPT_FIGURES/UL_NIJ_Houses/UL_NIJ_Single_Story_Gas_1_Pressure_6} \\
\end{tabular*}
\caption[UL/NIJ Experiments, Pressure, Single-Story (Ranch) House, Test 1]{UL/NIJ Experiments, Pressure, Single-Story (Ranch) House, Test 1.}
\label{UL_NIJ_Pres_Ranch_1}
\end{figure}

\begin{figure}[p]
\begin{tabular*}{\textwidth}{l@{\extracolsep{\fill}}r}
\includegraphics[height=2.15in]{SCRIPT_FIGURES/UL_NIJ_Houses/UL_NIJ_Single_Story_Gas_2_Pressure_1} &
\includegraphics[height=2.15in]{SCRIPT_FIGURES/UL_NIJ_Houses/UL_NIJ_Single_Story_Gas_2_Pressure_2} \\
\includegraphics[height=2.15in]{SCRIPT_FIGURES/UL_NIJ_Houses/UL_NIJ_Single_Story_Gas_2_Pressure_3} &
\includegraphics[height=2.15in]{SCRIPT_FIGURES/UL_NIJ_Houses/UL_NIJ_Single_Story_Gas_2_Pressure_4} \\
\includegraphics[height=2.15in]{SCRIPT_FIGURES/UL_NIJ_Houses/UL_NIJ_Single_Story_Gas_2_Pressure_5} &
\includegraphics[height=2.15in]{SCRIPT_FIGURES/UL_NIJ_Houses/UL_NIJ_Single_Story_Gas_2_Pressure_6} \\
\end{tabular*}
\caption[UL/NIJ Experiments, Pressure, Single-Story (Ranch) House, Test 2]{UL/NIJ Experiments, Pressure, Single-Story (Ranch) House, Test 2.}
\label{UL_NIJ_Pres_Ranch_2}
\end{figure}

\begin{figure}[p]
\begin{tabular*}{\textwidth}{l@{\extracolsep{\fill}}r}
\includegraphics[height=2.15in]{SCRIPT_FIGURES/UL_NIJ_Houses/UL_NIJ_Single_Story_Gas_5_Pressure_1} &
\includegraphics[height=2.15in]{SCRIPT_FIGURES/UL_NIJ_Houses/UL_NIJ_Single_Story_Gas_5_Pressure_2} \\
\includegraphics[height=2.15in]{SCRIPT_FIGURES/UL_NIJ_Houses/UL_NIJ_Single_Story_Gas_5_Pressure_3} &
\includegraphics[height=2.15in]{SCRIPT_FIGURES/UL_NIJ_Houses/UL_NIJ_Single_Story_Gas_5_Pressure_4} \\
\includegraphics[height=2.15in]{SCRIPT_FIGURES/UL_NIJ_Houses/UL_NIJ_Single_Story_Gas_5_Pressure_5} &
\includegraphics[height=2.15in]{SCRIPT_FIGURES/UL_NIJ_Houses/UL_NIJ_Single_Story_Gas_5_Pressure_6} \\
\end{tabular*}
\caption[UL/NIJ Experiments, Pressure, Single-Story (Ranch) House, Test 5]{UL/NIJ Experiments, Pressure, Single-Story (Ranch) House, Test 5.}
\label{UL_NIJ_Pres_Ranch_5}
\end{figure}

\begin{figure}[p]
\begin{tabular*}{\textwidth}{l@{\extracolsep{\fill}}r}
\includegraphics[height=2.15in]{SCRIPT_FIGURES/UL_NIJ_Houses/UL_NIJ_Two_Story_Gas_1_Pressure_1} &
\includegraphics[height=2.15in]{SCRIPT_FIGURES/UL_NIJ_Houses/UL_NIJ_Two_Story_Gas_1_Pressure_3} \\
\includegraphics[height=2.15in]{SCRIPT_FIGURES/UL_NIJ_Houses/UL_NIJ_Two_Story_Gas_1_Pressure_4} &
\includegraphics[height=2.15in]{SCRIPT_FIGURES/UL_NIJ_Houses/UL_NIJ_Two_Story_Gas_1_Pressure_5} \\
\includegraphics[height=2.15in]{SCRIPT_FIGURES/UL_NIJ_Houses/UL_NIJ_Two_Story_Gas_1_Pressure_6} &
\includegraphics[height=2.15in]{SCRIPT_FIGURES/UL_NIJ_Houses/UL_NIJ_Two_Story_Gas_1_Pressure_8} \\
\includegraphics[height=2.15in]{SCRIPT_FIGURES/UL_NIJ_Houses/UL_NIJ_Two_Story_Gas_1_Pressure_9}
\end{tabular*}
\caption[UL/NIJ Experiments, Pressure, Two-Story (Colonial) House, Test 1]{UL/NIJ Experiments, Pressure, Two-Story (Colonial) House, Test 1.}
\label{UL_NIJ_Pres_Colonial_1}
\end{figure}

\begin{figure}[p]
\begin{tabular*}{\textwidth}{l@{\extracolsep{\fill}}r}
\includegraphics[height=2.15in]{SCRIPT_FIGURES/UL_NIJ_Houses/UL_NIJ_Two_Story_Gas_4_Pressure_1} &
\includegraphics[height=2.15in]{SCRIPT_FIGURES/UL_NIJ_Houses/UL_NIJ_Two_Story_Gas_4_Pressure_3} \\
\includegraphics[height=2.15in]{SCRIPT_FIGURES/UL_NIJ_Houses/UL_NIJ_Two_Story_Gas_4_Pressure_4} &
\includegraphics[height=2.15in]{SCRIPT_FIGURES/UL_NIJ_Houses/UL_NIJ_Two_Story_Gas_4_Pressure_5} \\
\includegraphics[height=2.15in]{SCRIPT_FIGURES/UL_NIJ_Houses/UL_NIJ_Two_Story_Gas_4_Pressure_6} &
\includegraphics[height=2.15in]{SCRIPT_FIGURES/UL_NIJ_Houses/UL_NIJ_Two_Story_Gas_4_Pressure_8} \\
\includegraphics[height=2.15in]{SCRIPT_FIGURES/UL_NIJ_Houses/UL_NIJ_Two_Story_Gas_4_Pressure_9}
\end{tabular*}
\caption[UL/NIJ Experiments, Pressure, Two-Story (Colonial) House, Test 4]{UL/NIJ Experiments, Pressure, Two-Story (Colonial) House, Test 4.}
\label{UL_NIJ_Pres_Colonial_4}
\end{figure}

\begin{figure}[p]
\begin{tabular*}{\textwidth}{l@{\extracolsep{\fill}}r}
\includegraphics[height=2.15in]{SCRIPT_FIGURES/UL_NIJ_Houses/UL_NIJ_Two_Story_Gas_6_Pressure_1} &
\includegraphics[height=2.15in]{SCRIPT_FIGURES/UL_NIJ_Houses/UL_NIJ_Two_Story_Gas_6_Pressure_3} \\
\includegraphics[height=2.15in]{SCRIPT_FIGURES/UL_NIJ_Houses/UL_NIJ_Two_Story_Gas_6_Pressure_4} &
\includegraphics[height=2.15in]{SCRIPT_FIGURES/UL_NIJ_Houses/UL_NIJ_Two_Story_Gas_6_Pressure_5} \\
\includegraphics[height=2.15in]{SCRIPT_FIGURES/UL_NIJ_Houses/UL_NIJ_Two_Story_Gas_6_Pressure_6} &
\includegraphics[height=2.15in]{SCRIPT_FIGURES/UL_NIJ_Houses/UL_NIJ_Two_Story_Gas_6_Pressure_9}
\end{tabular*}
\caption[UL/NIJ Experiments, Pressure, Two-Story (Colonial) House, Test 6]{UL/NIJ Experiments, Pressure, Two-Story (Colonial) House, Test 6.}
\label{UL_NIJ_Pres_Colonial_6}
\end{figure}

\clearpage

\section{Summary of Pressure Predictions}
\label{Compartment Over-Pressure}

\begin{figure}[h!]
\begin{center}
\begin{tabular}{c}
\includegraphics[height=4in]{SCRIPT_FIGURES/Scatterplots/FDS_Compartment_Pressure}
\end{tabular}
\end{center}
\caption[Summary of pressure predictions]{Summary of pressure predictions for open and closed compartments.}
\label{Pressure_Summary}
\end{figure}



\chapter{Surface Temperature}

All solid surfaces in an FDS model are assigned thermal boundary conditions.
Heat and mass transfer to and from solid surfaces is
usually handled with empirical correlations, although it is possible
to compute directly the heat and mass transfer when performing a
Direct Numerical Simulation (DNS). Heat conduction into a solid surface is calculated via a one-dimensional solution of
the heat equation in either cartesian or cylindrical coordinates. The latter is useful for cables.



\section{WTC Test Series, Steel Structural Members and ``Slug'' Calorimeters}

The compartment for the WTC experiments contained a hollow box column roughly 0.5~m from the fire pan, two trusses over the top
of the pan, and one or two steel bars resting on the lower truss flanges. In Tests 1, 2 and 3, the steel was bare, and in Tests 4, 5 and 6, the
steel was coated with various thicknesses of sprayed fire-resistive materials.

The column was instrumented near its base (about
0.5~m from the floor, middle (1.5~m), and upper (2.5~m). Four measurements of steel (and insulation) temperatures were made at each location, for
each of its four sides.

In FDS, these elements were modeled using thin sheet obstructions with a resolution of 10~cm.

Five cylinders (``slugs'') of nickel 200 ($\ge$ 99~\% nickel), 25.4~cm long and 10.2~cm in diameter, were positioned
50~cm north of the centerline in the WTC experiments. Slugs 1 through 5 were 2.92~m, 1.82~m, 0.57~m, 0.05~m, and 1.56~m, respectively, from the
longitudinal axis of the fire pan. All the slugs were 50~cm north of the lateral axis. The fire pan measured 2~m by 1~m. Four thermocouples were
inserted into each slug at various locations. All four temperatures for each slug were virtually indistinguishable.

In FDS, rectangular obstructions were used to model the slugs, but the one-dimensional heat conduction calculation was performed using
cylindrical coordinates.



\begin{figure}[p]
\begin{tabular*}{\textwidth}{l@{\extracolsep{\fill}}r}
\includegraphics[height=2.2in]{FIGURES/WTC/WTC_01_v5_Upper_Column_Steel_Temp} &
\includegraphics[height=2.2in]{FIGURES/WTC/WTC_02_v5_Upper_Column_Steel_Temp} \\
\includegraphics[height=2.2in]{FIGURES/WTC/WTC_03_v5_Upper_Column_Steel_Temp} &
\includegraphics[height=2.2in]{FIGURES/WTC/WTC_04_v5_Upper_Column_Steel_Temp} \\
\includegraphics[height=2.2in]{FIGURES/WTC/WTC_05_v5_Upper_Column_Steel_Temp} &
\includegraphics[height=2.2in]{FIGURES/WTC/WTC_06_v5_Upper_Column_Steel_Temp}
\end{tabular*}
\label{NIST_WTC_Upper_Column_Steel}
\end{figure}

\begin{figure}[p]
\begin{tabular*}{\textwidth}{l@{\extracolsep{\fill}}r}
\includegraphics[height=2.2in]{FIGURES/WTC/WTC_01_v5_Middle_Column_Steel_Temp} &
\includegraphics[height=2.2in]{FIGURES/WTC/WTC_02_v5_Middle_Column_Steel_Temp} \\
\includegraphics[height=2.2in]{FIGURES/WTC/WTC_03_v5_Middle_Column_Steel_Temp} &
\includegraphics[height=2.2in]{FIGURES/WTC/WTC_04_v5_Middle_Column_Steel_Temp} \\
\includegraphics[height=2.2in]{FIGURES/WTC/WTC_05_v5_Middle_Column_Steel_Temp} &
\includegraphics[height=2.2in]{FIGURES/WTC/WTC_06_v5_Middle_Column_Steel_Temp}
\end{tabular*}
\label{NIST_WTC_Middle_Column_Steel}
\end{figure}

\begin{figure}[p]
\begin{tabular*}{\textwidth}{l@{\extracolsep{\fill}}r}
\includegraphics[height=2.2in]{FIGURES/WTC/WTC_01_v5_Lower_Column_Steel_Temp} &
\includegraphics[height=2.2in]{FIGURES/WTC/WTC_02_v5_Lower_Column_Steel_Temp} \\
\includegraphics[height=2.2in]{FIGURES/WTC/WTC_03_v5_Lower_Column_Steel_Temp} &
\includegraphics[height=2.2in]{FIGURES/WTC/WTC_04_v5_Lower_Column_Steel_Temp} \\
\includegraphics[height=2.2in]{FIGURES/WTC/WTC_05_v5_Lower_Column_Steel_Temp} &
\includegraphics[height=2.2in]{FIGURES/WTC/WTC_06_v5_Lower_Column_Steel_Temp}
\end{tabular*}
\label{NIST_WTC_Lower_Column_Steel}
\end{figure}

\begin{figure}[p]
\begin{tabular*}{\textwidth}{l@{\extracolsep{\fill}}r}
\includegraphics[height=2.2in]{FIGURES/WTC/WTC_01_v5_Truss_A_Upper_Steel_Temp} &
\includegraphics[height=2.2in]{FIGURES/WTC/WTC_02_v5_Truss_A_Upper_Steel_Temp} \\
\includegraphics[height=2.2in]{FIGURES/WTC/WTC_03_v5_Truss_A_Upper_Steel_Temp} &
\includegraphics[height=2.2in]{FIGURES/WTC/WTC_04_v5_Truss_A_Upper_Steel_Temp} \\
\includegraphics[height=2.2in]{FIGURES/WTC/WTC_05_v5_Truss_A_Upper_Steel_Temp} &
\includegraphics[height=2.2in]{FIGURES/WTC/WTC_06_v5_Truss_A_Upper_Steel_Temp}
\end{tabular*}
\label{NIST_WTC_Truss_A_Upper_Steel_Temp}
\end{figure}

\begin{figure}[p]
\begin{tabular*}{\textwidth}{l@{\extracolsep{\fill}}r}
\includegraphics[height=2.2in]{FIGURES/WTC/WTC_01_v5_Truss_A_Middle_Steel_Temp} &
\includegraphics[height=2.2in]{FIGURES/WTC/WTC_02_v5_Truss_A_Middle_Steel_Temp} \\
\includegraphics[height=2.2in]{FIGURES/WTC/WTC_03_v5_Truss_A_Middle_Steel_Temp} &
\includegraphics[height=2.2in]{FIGURES/WTC/WTC_04_v5_Truss_A_Middle_Steel_Temp} \\
\includegraphics[height=2.2in]{FIGURES/WTC/WTC_05_v5_Truss_A_Middle_Steel_Temp} &
\includegraphics[height=2.2in]{FIGURES/WTC/WTC_06_v5_Truss_A_Middle_Steel_Temp}
\end{tabular*}
\label{NIST_WTC_Truss_A_Middle_Steel_Temp}
\end{figure}

\begin{figure}[p]
\begin{tabular*}{\textwidth}{l@{\extracolsep{\fill}}r}
\includegraphics[height=2.2in]{FIGURES/WTC/WTC_01_v5_Truss_A_Lower_Steel_Temp} &
\includegraphics[height=2.2in]{FIGURES/WTC/WTC_02_v5_Truss_A_Lower_Steel_Temp} \\
\includegraphics[height=2.2in]{FIGURES/WTC/WTC_03_v5_Truss_A_Lower_Steel_Temp} &
\includegraphics[height=2.2in]{FIGURES/WTC/WTC_04_v5_Truss_A_Lower_Steel_Temp} \\
\includegraphics[height=2.2in]{FIGURES/WTC/WTC_05_v5_Truss_A_Lower_Steel_Temp} &
\includegraphics[height=2.2in]{FIGURES/WTC/WTC_06_v5_Truss_A_Lower_Steel_Temp}
\end{tabular*}
\label{NIST_WTC_Truss_A_Lower_Steel_Temp}
\end{figure}

\begin{figure}[p]
\begin{tabular*}{\textwidth}{l@{\extracolsep{\fill}}r}
\includegraphics[height=2.2in]{FIGURES/WTC/WTC_01_v5_Truss_B_Upper_Steel_Temp} &
\includegraphics[height=2.2in]{FIGURES/WTC/WTC_02_v5_Truss_B_Upper_Steel_Temp} \\
\includegraphics[height=2.2in]{FIGURES/WTC/WTC_03_v5_Truss_B_Upper_Steel_Temp} &
\includegraphics[height=2.2in]{FIGURES/WTC/WTC_04_v5_Truss_B_Upper_Steel_Temp} \\
\includegraphics[height=2.2in]{FIGURES/WTC/WTC_05_v5_Truss_B_Upper_Steel_Temp} &
\includegraphics[height=2.2in]{FIGURES/WTC/WTC_06_v5_Truss_B_Upper_Steel_Temp}
\end{tabular*}
\label{NIST_WTC_Truss_B_Upper_Steel_Temp}
\end{figure}

\begin{figure}[p]
\begin{tabular*}{\textwidth}{l@{\extracolsep{\fill}}r}
\includegraphics[height=2.2in]{FIGURES/WTC/WTC_01_v5_Truss_B_Middle_Steel_Temp} &
\includegraphics[height=2.2in]{FIGURES/WTC/WTC_02_v5_Truss_B_Middle_Steel_Temp} \\
\includegraphics[height=2.2in]{FIGURES/WTC/WTC_03_v5_Truss_B_Middle_Steel_Temp} &
\includegraphics[height=2.2in]{FIGURES/WTC/WTC_04_v5_Truss_B_Middle_Steel_Temp} \\
\includegraphics[height=2.2in]{FIGURES/WTC/WTC_05_v5_Truss_B_Middle_Steel_Temp} &
\includegraphics[height=2.2in]{FIGURES/WTC/WTC_06_v5_Truss_B_Middle_Steel_Temp}
\end{tabular*}
\label{NIST_WTC_Truss_B_Middle_Steel_Temp}
\end{figure}

\begin{figure}[p]
\begin{tabular*}{\textwidth}{l@{\extracolsep{\fill}}r}
\includegraphics[height=2.2in]{FIGURES/WTC/WTC_01_v5_Truss_B_Lower_Steel_Temp} &
\includegraphics[height=2.2in]{FIGURES/WTC/WTC_02_v5_Truss_B_Lower_Steel_Temp} \\
\includegraphics[height=2.2in]{FIGURES/WTC/WTC_03_v5_Truss_B_Lower_Steel_Temp} &
\includegraphics[height=2.2in]{FIGURES/WTC/WTC_04_v5_Truss_B_Lower_Steel_Temp} \\
\includegraphics[height=2.2in]{FIGURES/WTC/WTC_05_v5_Truss_B_Lower_Steel_Temp} &
\includegraphics[height=2.2in]{FIGURES/WTC/WTC_06_v5_Truss_B_Lower_Steel_Temp}
\end{tabular*}
\label{NIST_WTC_Truss_B_Lower_Steel_Temp}
\end{figure}


\begin{figure}[p]
\begin{tabular*}{\textwidth}{l@{\extracolsep{\fill}}r}
\includegraphics[height=2.2in]{FIGURES/WTC/WTC_01_v5_Bar_1_Steel_Temp} &
\includegraphics[height=2.2in]{FIGURES/WTC/WTC_02_v5_Bar_1_Steel_Temp} \\
\includegraphics[height=2.2in]{FIGURES/WTC/WTC_03_v5_Bar_1_Steel_Temp} &
\includegraphics[height=2.2in]{FIGURES/WTC/WTC_04_v5_Bar_1_Steel_Temp} \\
\includegraphics[height=2.2in]{FIGURES/WTC/WTC_05_v5_Bar_1_Steel_Temp} &
\includegraphics[height=2.2in]{FIGURES/WTC/WTC_06_v5_Bar_1_Steel_Temp}
\end{tabular*}
\label{NIST_WTC_Bar_1_Steel_Temp}
\end{figure}

\begin{figure}[p]
\begin{tabular*}{\textwidth}{l@{\extracolsep{\fill}}r}
\includegraphics[height=2.2in]{FIGURES/WTC/WTC_01_v5_Slug_1_Temp} &
\includegraphics[height=2.2in]{FIGURES/WTC/WTC_02_v5_Slug_1_Temp} \\
\includegraphics[height=2.2in]{FIGURES/WTC/WTC_03_v5_Slug_1_Temp} &
\includegraphics[height=2.2in]{FIGURES/WTC/WTC_04_v5_Slug_1_Temp} \\
\includegraphics[height=2.2in]{FIGURES/WTC/WTC_05_v5_Slug_1_Temp} &
\includegraphics[height=2.2in]{FIGURES/WTC/WTC_06_v5_Slug_1_Temp}
\end{tabular*}
\label{NIST_WTC_Slug_1_Temp}
\end{figure}

\begin{figure}[p]
\begin{tabular*}{\textwidth}{l@{\extracolsep{\fill}}r}
\includegraphics[height=2.2in]{FIGURES/WTC/WTC_01_v5_Slug_2_Temp} &
\includegraphics[height=2.2in]{FIGURES/WTC/WTC_02_v5_Slug_2_Temp} \\
\includegraphics[height=2.2in]{FIGURES/WTC/WTC_03_v5_Slug_2_Temp} &
\includegraphics[height=2.2in]{FIGURES/WTC/WTC_04_v5_Slug_2_Temp} \\
\includegraphics[height=2.2in]{FIGURES/WTC/WTC_05_v5_Slug_2_Temp} &
\includegraphics[height=2.2in]{FIGURES/WTC/WTC_06_v5_Slug_2_Temp}
\end{tabular*}
\label{NIST_WTC_Slug_2_Temp}
\end{figure}

\begin{figure}[p]
\begin{tabular*}{\textwidth}{l@{\extracolsep{\fill}}r}
\includegraphics[height=2.2in]{FIGURES/WTC/WTC_01_v5_Slug_3_Temp} &
\includegraphics[height=2.2in]{FIGURES/WTC/WTC_02_v5_Slug_3_Temp} \\
\includegraphics[height=2.2in]{FIGURES/WTC/WTC_03_v5_Slug_3_Temp} &
\includegraphics[height=2.2in]{FIGURES/WTC/WTC_04_v5_Slug_3_Temp} \\
\includegraphics[height=2.2in]{FIGURES/WTC/WTC_05_v5_Slug_3_Temp} &
\includegraphics[height=2.2in]{FIGURES/WTC/WTC_06_v5_Slug_3_Temp}
\end{tabular*}
\label{NIST_WTC_Slug_3_Temp}
\end{figure}

\begin{figure}[p]
\begin{tabular*}{\textwidth}{l@{\extracolsep{\fill}}r}
\includegraphics[height=2.2in]{FIGURES/WTC/WTC_01_v5_Slug_4_Temp} &
\includegraphics[height=2.2in]{FIGURES/WTC/WTC_02_v5_Slug_4_Temp} \\
\includegraphics[height=2.2in]{FIGURES/WTC/WTC_03_v5_Slug_4_Temp} &
\includegraphics[height=2.2in]{FIGURES/WTC/WTC_04_v5_Slug_4_Temp} \\
\includegraphics[height=2.2in]{FIGURES/WTC/WTC_05_v5_Slug_4_Temp} &
\includegraphics[height=2.2in]{FIGURES/WTC/WTC_06_v5_Slug_4_Temp}
\end{tabular*}
\label{NIST_WTC_Slug_4_Temp}
\end{figure}

\begin{figure}[p]
\begin{tabular*}{\textwidth}{l@{\extracolsep{\fill}}r}
\includegraphics[height=2.2in]{FIGURES/WTC/WTC_01_v5_Slug_5_Temp} &
\includegraphics[height=2.2in]{FIGURES/WTC/WTC_02_v5_Slug_5_Temp} \\
\includegraphics[height=2.2in]{FIGURES/WTC/WTC_03_v5_Slug_5_Temp} &
\includegraphics[height=2.2in]{FIGURES/WTC/WTC_04_v5_Slug_5_Temp} \\
\includegraphics[height=2.2in]{FIGURES/WTC/WTC_05_v5_Slug_5_Temp} &
\includegraphics[height=2.2in]{FIGURES/WTC/WTC_06_v5_Slug_5_Temp}
\end{tabular*}
\label{NIST_WTC_Slug_5_Temp}
\end{figure}


\begin{figure}[p]
\begin{center}
\begin{tabular}{c}
\includegraphics[width=5.0in]{FIGURES/ScatterPlots/Steel_Temperature} \\
\vspace{0.25in}
\end{tabular}
\end{center}
\caption[Summary of steel temperature predictions, WTC test series.]
{Summary of steel temperature predictions for the WTC test series.}
\end{figure}


\clearpage

\section{SP 2009 Adiabatic Surface Temperature Experiments}

Comparisons of FDS predictions of gas, plate thermometer, and steel temperatures for three compartment experiments conducted at SP, Sweden are
presented on the following pages. Each experiment was conducted in a standard compartment, 3.6~m long by 2.4~m wide by 2.4~m high, with a 0.8~m wide by
2.0~m high door centered on the narrow wall. Each experiment used a constant 450~kW propane burner and a single beam suspended 20~cm below the ceiling
along the centerline of the compartment. There were three measurement stations along the beam at lengths of 0.9~m (Position A), 1.8~m (Position B), and
2.7~m (Position C) from the far wall where the fire was either positioned in the corner (Tests 1 and 2), or the center (Test 3). The beam in Test 1 was
a rectangular steel tube filled with an insulation material. The beam in Tests 2 and 3 was an I-beam. Details can be found in the test report~\cite{Wickstrom_AST}.

Each page to follow contains the results for a single experiment and measuring station. There are nine in all. In addition to predictions of the gas,
plate thermometer, and steel temperatures, there are predictions of the adiabatic surface temperature at the surface of the plate thermometers. 
FDS includes a calculation of
the Adiabatic Surface Temperature (AST), a quantity that is
representative of the heat flux to a solid surface. Following the idea
proposed by Wickstr\"{o}m~\cite{Wickstrom:Interflam2007}, the
following equation can be solved via a simple iterative technique to
determine an effective gas temperature, $T_{\hbox{\tiny AST}}$:
\be \dot{q}_r'' + \dot{q}_c'' = \epsilon \sigma \, \left(
T_{\hbox{\tiny AST}}^4 - T_w^4 \right) + h (T_{\hbox{\tiny AST}} - T_w)  \ee
The sum $\dot{q}_r'' + \dot{q}_c''$ is the {\em net} heat flux onto
the solid surface, whose temperature is $T_w$. The AST is what would be measured by a perfect plate thermometer. In fact, it can be shown that
the calculated plate and adiabatic surface temperatures become equivalent after a certain period of time. The ``measured'' AST in the plots are
derived from the measured plate thermometer temperatures.


\begin{figure}[p]
\begin{tabular*}{\textwidth}{l@{\extracolsep{\fill}}r}
\includegraphics[height=2.2in]{FIGURES/SP2009_AST/SP2009_AST_Test_1_Sta_A_Pos_1_and_2_Gas} &
\includegraphics[height=2.2in]{FIGURES/SP2009_AST/SP2009_AST_Test_1_Sta_A_Pos_3_and_4_Gas} \\
\includegraphics[height=2.2in]{FIGURES/SP2009_AST/SP2009_AST_Test_1_Sta_A_Pos_1_and_2_PT} &
\includegraphics[height=2.2in]{FIGURES/SP2009_AST/SP2009_AST_Test_1_Sta_A_Pos_3_and_4_PT} \\
\includegraphics[height=2.2in]{FIGURES/SP2009_AST/SP2009_AST_Test_1_Sta_A_Pos_1_and_2_AST} &
\includegraphics[height=2.2in]{FIGURES/SP2009_AST/SP2009_AST_Test_1_Sta_A_Pos_3_and_4_AST} \\
\includegraphics[height=2.2in]{FIGURES/SP2009_AST/SP2009_AST_Test_1_Sta_A_Pos_1_and_2_Steel} &
\includegraphics[height=2.2in]{FIGURES/SP2009_AST/SP2009_AST_Test_1_Sta_A_Pos_3_and_4_Steel}
\end{tabular*}
\label{SP_Test_1_Station_A}
\end{figure}

\begin{figure}[p]
\begin{tabular*}{\textwidth}{l@{\extracolsep{\fill}}r}
\includegraphics[height=2.2in]{FIGURES/SP2009_AST/SP2009_AST_Test_1_Sta_B_Pos_2_and_4_Gas} &
  \\
\includegraphics[height=2.2in]{FIGURES/SP2009_AST/SP2009_AST_Test_1_Sta_B_Pos_1_and_2_PT} &
\includegraphics[height=2.2in]{FIGURES/SP2009_AST/SP2009_AST_Test_1_Sta_B_Pos_3_and_4_PT} \\
\includegraphics[height=2.2in]{FIGURES/SP2009_AST/SP2009_AST_Test_1_Sta_B_Pos_1_and_2_AST} &
\includegraphics[height=2.2in]{FIGURES/SP2009_AST/SP2009_AST_Test_1_Sta_B_Pos_3_and_4_AST} \\
\includegraphics[height=2.2in]{FIGURES/SP2009_AST/SP2009_AST_Test_1_Sta_B_Pos_1_and_2_Steel} &
\includegraphics[height=2.2in]{FIGURES/SP2009_AST/SP2009_AST_Test_1_Sta_B_Pos_3_and_4_Steel}
\end{tabular*}
\label{SP_Test_1_Station_B}
\end{figure}

\begin{figure}[p]
\begin{tabular*}{\textwidth}{l@{\extracolsep{\fill}}r}
\includegraphics[height=2.2in]{FIGURES/SP2009_AST/SP2009_AST_Test_1_Sta_C_Pos_2_and_4_Gas} &
  \\
\includegraphics[height=2.2in]{FIGURES/SP2009_AST/SP2009_AST_Test_1_Sta_C_Pos_1_and_2_PT} &
\includegraphics[height=2.2in]{FIGURES/SP2009_AST/SP2009_AST_Test_1_Sta_C_Pos_3_and_4_PT} \\
\includegraphics[height=2.2in]{FIGURES/SP2009_AST/SP2009_AST_Test_1_Sta_C_Pos_1_and_2_AST} &
\includegraphics[height=2.2in]{FIGURES/SP2009_AST/SP2009_AST_Test_1_Sta_C_Pos_3_and_4_AST} \\
\includegraphics[height=2.2in]{FIGURES/SP2009_AST/SP2009_AST_Test_1_Sta_C_Pos_1_and_2_Steel} &
\includegraphics[height=2.2in]{FIGURES/SP2009_AST/SP2009_AST_Test_1_Sta_C_Pos_3_and_4_Steel}
\end{tabular*}
\label{SP_Test_1_Station_C}
\end{figure}


\begin{figure}[p]
\begin{tabular*}{\textwidth}{l@{\extracolsep{\fill}}r}
\includegraphics[height=2.2in]{FIGURES/SP2009_AST/SP2009_AST_Test_2_Sta_A_Pos_1_and_2_Gas} &
\includegraphics[height=2.2in]{FIGURES/SP2009_AST/SP2009_AST_Test_2_Sta_A_Pos_3_and_4_Gas} \\
\includegraphics[height=2.2in]{FIGURES/SP2009_AST/SP2009_AST_Test_2_Sta_A_Pos_1_and_2_PT} &
\includegraphics[height=2.2in]{FIGURES/SP2009_AST/SP2009_AST_Test_2_Sta_A_Pos_3_and_4_PT} \\
\includegraphics[height=2.2in]{FIGURES/SP2009_AST/SP2009_AST_Test_2_Sta_A_Pos_1_and_2_AST} &
\includegraphics[height=2.2in]{FIGURES/SP2009_AST/SP2009_AST_Test_2_Sta_A_Pos_3_and_4_AST} \\
\includegraphics[height=2.2in]{FIGURES/SP2009_AST/SP2009_AST_Test_2_Sta_A_Pos_1_and_2_Steel} &
\includegraphics[height=2.2in]{FIGURES/SP2009_AST/SP2009_AST_Test_2_Sta_A_Pos_3_and_4_Steel}
\end{tabular*}
\label{SP_Test_2_Station_A}
\end{figure}

\begin{figure}[p]
\begin{tabular*}{\textwidth}{l@{\extracolsep{\fill}}r}
\includegraphics[height=2.2in]{FIGURES/SP2009_AST/SP2009_AST_Test_2_Sta_B_Pos_2_and_4_Gas} &
  \\
\includegraphics[height=2.2in]{FIGURES/SP2009_AST/SP2009_AST_Test_2_Sta_B_Pos_1_and_2_PT} &
\includegraphics[height=2.2in]{FIGURES/SP2009_AST/SP2009_AST_Test_2_Sta_B_Pos_3_and_4_PT} \\
\includegraphics[height=2.2in]{FIGURES/SP2009_AST/SP2009_AST_Test_2_Sta_B_Pos_1_and_2_AST} &
\includegraphics[height=2.2in]{FIGURES/SP2009_AST/SP2009_AST_Test_2_Sta_B_Pos_3_and_4_AST} \\
\includegraphics[height=2.2in]{FIGURES/SP2009_AST/SP2009_AST_Test_2_Sta_B_Pos_1_and_2_Steel} &
\includegraphics[height=2.2in]{FIGURES/SP2009_AST/SP2009_AST_Test_2_Sta_B_Pos_3_and_4_Steel}
\end{tabular*}
\label{SP_Test_2_Station_B}
\end{figure}

\begin{figure}[p]
\begin{tabular*}{\textwidth}{l@{\extracolsep{\fill}}r}
\includegraphics[height=2.2in]{FIGURES/SP2009_AST/SP2009_AST_Test_2_Sta_C_Pos_2_and_4_Gas} &
  \\
\includegraphics[height=2.2in]{FIGURES/SP2009_AST/SP2009_AST_Test_2_Sta_C_Pos_1_and_2_PT} &
\includegraphics[height=2.2in]{FIGURES/SP2009_AST/SP2009_AST_Test_2_Sta_C_Pos_3_and_4_PT} \\
\includegraphics[height=2.2in]{FIGURES/SP2009_AST/SP2009_AST_Test_2_Sta_C_Pos_1_and_2_AST} &
\includegraphics[height=2.2in]{FIGURES/SP2009_AST/SP2009_AST_Test_2_Sta_C_Pos_3_and_4_AST} \\
\includegraphics[height=2.2in]{FIGURES/SP2009_AST/SP2009_AST_Test_2_Sta_C_Pos_1_and_2_Steel} &
\includegraphics[height=2.2in]{FIGURES/SP2009_AST/SP2009_AST_Test_2_Sta_C_Pos_3_and_4_Steel}
\end{tabular*}
\label{SP_Test_2_Station_C}
\end{figure}


\begin{figure}[p]
\begin{tabular*}{\textwidth}{l@{\extracolsep{\fill}}r}
\includegraphics[height=2.2in]{FIGURES/SP2009_AST/SP2009_AST_Test_3_Sta_A_Pos_1_and_2_Gas} &
\includegraphics[height=2.2in]{FIGURES/SP2009_AST/SP2009_AST_Test_3_Sta_A_Pos_3_and_4_Gas} \\
\includegraphics[height=2.2in]{FIGURES/SP2009_AST/SP2009_AST_Test_3_Sta_A_Pos_1_and_2_PT} &
\includegraphics[height=2.2in]{FIGURES/SP2009_AST/SP2009_AST_Test_3_Sta_A_Pos_3_and_4_PT} \\
\includegraphics[height=2.2in]{FIGURES/SP2009_AST/SP2009_AST_Test_3_Sta_A_Pos_1_and_2_AST} &
\includegraphics[height=2.2in]{FIGURES/SP2009_AST/SP2009_AST_Test_3_Sta_A_Pos_3_and_4_AST} \\
\includegraphics[height=2.2in]{FIGURES/SP2009_AST/SP2009_AST_Test_3_Sta_A_Pos_1_and_2_Steel} &
\includegraphics[height=2.2in]{FIGURES/SP2009_AST/SP2009_AST_Test_3_Sta_A_Pos_3_and_4_Steel}
\end{tabular*}
\label{SP_Test_3_Station_A}
\end{figure}

\begin{figure}[p]
\begin{tabular*}{\textwidth}{l@{\extracolsep{\fill}}r}
\includegraphics[height=2.2in]{FIGURES/SP2009_AST/SP2009_AST_Test_3_Sta_B_Pos_2_and_4_Gas} &
  \\
\includegraphics[height=2.2in]{FIGURES/SP2009_AST/SP2009_AST_Test_3_Sta_B_Pos_1_and_2_PT} &
\includegraphics[height=2.2in]{FIGURES/SP2009_AST/SP2009_AST_Test_3_Sta_B_Pos_3_and_4_PT} \\
\includegraphics[height=2.2in]{FIGURES/SP2009_AST/SP2009_AST_Test_3_Sta_B_Pos_1_and_2_AST} &
\includegraphics[height=2.2in]{FIGURES/SP2009_AST/SP2009_AST_Test_3_Sta_B_Pos_3_and_4_AST} \\
\includegraphics[height=2.2in]{FIGURES/SP2009_AST/SP2009_AST_Test_3_Sta_B_Pos_1_and_2_Steel} &
\includegraphics[height=2.2in]{FIGURES/SP2009_AST/SP2009_AST_Test_3_Sta_B_Pos_3_and_4_Steel}
\end{tabular*}
\label{SP_Test_3_Station_B}
\end{figure}

\begin{figure}[p]
\begin{tabular*}{\textwidth}{l@{\extracolsep{\fill}}r}
\includegraphics[height=2.2in]{FIGURES/SP2009_AST/SP2009_AST_Test_3_Sta_C_Pos_2_and_4_Gas} &
  \\
\includegraphics[height=2.2in]{FIGURES/SP2009_AST/SP2009_AST_Test_3_Sta_C_Pos_1_and_2_PT} &
\includegraphics[height=2.2in]{FIGURES/SP2009_AST/SP2009_AST_Test_3_Sta_C_Pos_3_and_4_PT} \\
\includegraphics[height=2.2in]{FIGURES/SP2009_AST/SP2009_AST_Test_3_Sta_C_Pos_1_and_2_AST} &
\includegraphics[height=2.2in]{FIGURES/SP2009_AST/SP2009_AST_Test_3_Sta_C_Pos_3_and_4_AST} \\
\includegraphics[height=2.2in]{FIGURES/SP2009_AST/SP2009_AST_Test_3_Sta_C_Pos_1_and_2_Steel} &
\includegraphics[height=2.2in]{FIGURES/SP2009_AST/SP2009_AST_Test_3_Sta_C_Pos_3_and_4_Steel}
\end{tabular*}
\label{SP_Test_3_Station_C}
\end{figure}





\begin{figure}[p]
\begin{center}
\begin{tabular}{c}
\includegraphics[width=5.0in]{FIGURES/ScatterPlots/SP_Temperature} \\
\vspace{0.25in}
\end{tabular}
\end{center}
\caption[Summary of gas, plate thermometer, and steel temperatures, SP2009 AST Experiments.]
{Summary of gas, plate thermometer, and steel temperatures, SP2009 AST Experiments.}
\end{figure}

\clearpage

\section{NIST/NRC Test Series, Cables}

Cables in various types (power and control), and configurations (horizontal, vertical, in trays or free-hanging), were installed in
the test compartment.
For each of the four cable targets considered, measurements of the local gas temperature, surface temperature, radiative heat flux,
and total heat flux are available.  The following pages display comparisons of these quantities for
Control Cable B, Horizontal Cable Tray D, Power Cable F and Vertical Cable Tray G.
FDS does not have a detailed solid phase model that can account for the heat transfer within the bundled,
cylindrical, non-homogenous cables.  For the bundled cables within horizontal and vertical trays (Targets D and G),
FDS assumes them to be rectangular slabs of thickness comparable to the diameter of the individual cables.
For the free-hanging cables B and F, FDS assumes them to be cylinders of uniform composition into which it
computes the radial heat transfer as a function of the heat flux to a designated location.
The superposition of gas temperature, heat flux and surface temperature in the figures on the following pages
provides information about how cables heat up in fires.  Favorable or unfavorable predictions of cable surface
temperatures can often be explained in terms of comparable errors in the prediction of the thermal environment in the vicinity of the cable.

\begin{figure}[p]
\begin{tabular*}{\textwidth}{l@{\extracolsep{\fill}}r}
\includegraphics[height=2.2in]{FIGURES/NIST_NRC/NIST_NRC_01_v5_Cable_B_Temp} &
\includegraphics[height=2.2in]{FIGURES/NIST_NRC/NIST_NRC_07_v5_Cable_B_Temp} \\
\includegraphics[height=2.2in]{FIGURES/NIST_NRC/NIST_NRC_02_v5_Cable_B_Temp} &
\includegraphics[height=2.2in]{FIGURES/NIST_NRC/NIST_NRC_08_v5_Cable_B_Temp} \\
\includegraphics[height=2.2in]{FIGURES/NIST_NRC/NIST_NRC_04_v5_Cable_B_Temp} &
\includegraphics[height=2.2in]{FIGURES/NIST_NRC/NIST_NRC_10_v5_Cable_B_Temp} \\
\includegraphics[height=2.2in]{FIGURES/NIST_NRC/NIST_NRC_13_v5_Cable_B_Temp} &
\includegraphics[height=2.2in]{FIGURES/NIST_NRC/NIST_NRC_16_v5_Cable_B_Temp}
\end{tabular*}
\label{NIST_NRC_Cable_B_Closed}
\end{figure}

\begin{figure}[p]
\begin{tabular*}{\textwidth}{l@{\extracolsep{\fill}}r}
\includegraphics[height=2.2in]{FIGURES/NIST_NRC/NIST_NRC_03_v5_Cable_B_Temp} &
\includegraphics[height=2.2in]{FIGURES/NIST_NRC/NIST_NRC_09_v5_Cable_B_Temp} \\
\includegraphics[height=2.2in]{FIGURES/NIST_NRC/NIST_NRC_05_v5_Cable_B_Temp} &
\includegraphics[height=2.2in]{FIGURES/NIST_NRC/NIST_NRC_14_v5_Cable_B_Temp} \\
\includegraphics[height=2.2in]{FIGURES/NIST_NRC/NIST_NRC_15_v5_Cable_B_Temp} &
\includegraphics[height=2.2in]{FIGURES/NIST_NRC/NIST_NRC_18_v5_Cable_B_Temp}
\end{tabular*}
\label{NIST_NRC_Cable_B_Open}
\end{figure}

\begin{figure}[p]
\begin{tabular*}{\textwidth}{l@{\extracolsep{\fill}}r}
\includegraphics[height=2.2in]{FIGURES/NIST_NRC/NIST_NRC_01_v5_Cable_D_Temp} &
\includegraphics[height=2.2in]{FIGURES/NIST_NRC/NIST_NRC_07_v5_Cable_D_Temp} \\
\includegraphics[height=2.2in]{FIGURES/NIST_NRC/NIST_NRC_02_v5_Cable_D_Temp} &
\includegraphics[height=2.2in]{FIGURES/NIST_NRC/NIST_NRC_08_v5_Cable_D_Temp} \\
\includegraphics[height=2.2in]{FIGURES/NIST_NRC/NIST_NRC_04_v5_Cable_D_Temp} &
\includegraphics[height=2.2in]{FIGURES/NIST_NRC/NIST_NRC_10_v5_Cable_D_Temp} \\
\includegraphics[height=2.2in]{FIGURES/NIST_NRC/NIST_NRC_13_v5_Cable_D_Temp} &
\includegraphics[height=2.2in]{FIGURES/NIST_NRC/NIST_NRC_16_v5_Cable_D_Temp}
\end{tabular*}
\label{NIST_NRC_Cable_D_Closed}
\end{figure}

\begin{figure}[p]
\begin{tabular*}{\textwidth}{l@{\extracolsep{\fill}}r}
\includegraphics[height=2.2in]{FIGURES/NIST_NRC/NIST_NRC_03_v5_Cable_D_Temp} &
\includegraphics[height=2.2in]{FIGURES/NIST_NRC/NIST_NRC_09_v5_Cable_D_Temp} \\
\includegraphics[height=2.2in]{FIGURES/NIST_NRC/NIST_NRC_05_v5_Cable_D_Temp} &
\includegraphics[height=2.2in]{FIGURES/NIST_NRC/NIST_NRC_14_v5_Cable_D_Temp} \\
\includegraphics[height=2.2in]{FIGURES/NIST_NRC/NIST_NRC_15_v5_Cable_D_Temp} &
\includegraphics[height=2.2in]{FIGURES/NIST_NRC/NIST_NRC_18_v5_Cable_D_Temp}
\end{tabular*}
\label{NIST_NRC_Cable_D_Open}
\end{figure}

\begin{figure}[p]
\begin{tabular*}{\textwidth}{l@{\extracolsep{\fill}}r}
\includegraphics[height=2.2in]{FIGURES/NIST_NRC/NIST_NRC_01_v5_Cable_F_Temp} &
\includegraphics[height=2.2in]{FIGURES/NIST_NRC/NIST_NRC_07_v5_Cable_F_Temp} \\
\includegraphics[height=2.2in]{FIGURES/NIST_NRC/NIST_NRC_02_v5_Cable_F_Temp} &
\includegraphics[height=2.2in]{FIGURES/NIST_NRC/NIST_NRC_08_v5_Cable_F_Temp} \\
\includegraphics[height=2.2in]{FIGURES/NIST_NRC/NIST_NRC_04_v5_Cable_F_Temp} &
\includegraphics[height=2.2in]{FIGURES/NIST_NRC/NIST_NRC_10_v5_Cable_F_Temp} \\
\includegraphics[height=2.2in]{FIGURES/NIST_NRC/NIST_NRC_13_v5_Cable_F_Temp} &
\includegraphics[height=2.2in]{FIGURES/NIST_NRC/NIST_NRC_16_v5_Cable_F_Temp}
\end{tabular*}
\label{NIST_NRC_Cable_F_Closed}
\end{figure}

\begin{figure}[p]
\begin{tabular*}{\textwidth}{l@{\extracolsep{\fill}}r}
\includegraphics[height=2.2in]{FIGURES/NIST_NRC/NIST_NRC_03_v5_Cable_F_Temp} &
\includegraphics[height=2.2in]{FIGURES/NIST_NRC/NIST_NRC_09_v5_Cable_F_Temp} \\
\includegraphics[height=2.2in]{FIGURES/NIST_NRC/NIST_NRC_05_v5_Cable_F_Temp} &
\includegraphics[height=2.2in]{FIGURES/NIST_NRC/NIST_NRC_14_v5_Cable_F_Temp} \\
\includegraphics[height=2.2in]{FIGURES/NIST_NRC/NIST_NRC_15_v5_Cable_F_Temp} &
\includegraphics[height=2.2in]{FIGURES/NIST_NRC/NIST_NRC_18_v5_Cable_F_Temp}
\end{tabular*}
\label{NIST_NRC_Cable_F_Open}
\end{figure}

\begin{figure}[p]
\begin{tabular*}{\textwidth}{l@{\extracolsep{\fill}}r}
\includegraphics[height=2.2in]{FIGURES/NIST_NRC/NIST_NRC_01_v5_Cable_G_Temp} &
\includegraphics[height=2.2in]{FIGURES/NIST_NRC/NIST_NRC_07_v5_Cable_G_Temp} \\
\includegraphics[height=2.2in]{FIGURES/NIST_NRC/NIST_NRC_02_v5_Cable_G_Temp} &
\includegraphics[height=2.2in]{FIGURES/NIST_NRC/NIST_NRC_08_v5_Cable_G_Temp} \\
\includegraphics[height=2.2in]{FIGURES/NIST_NRC/NIST_NRC_04_v5_Cable_G_Temp} &
\includegraphics[height=2.2in]{FIGURES/NIST_NRC/NIST_NRC_10_v5_Cable_G_Temp} \\
\includegraphics[height=2.2in]{FIGURES/NIST_NRC/NIST_NRC_13_v5_Cable_G_Temp} &
\includegraphics[height=2.2in]{FIGURES/NIST_NRC/NIST_NRC_16_v5_Cable_G_Temp}
\end{tabular*}
\label{NIST_NRC_Cable_G_Closed}
\end{figure}

\begin{figure}[p]
\begin{tabular*}{\textwidth}{l@{\extracolsep{\fill}}r}
\includegraphics[height=2.2in]{FIGURES/NIST_NRC/NIST_NRC_03_v5_Cable_G_Temp} &
\includegraphics[height=2.2in]{FIGURES/NIST_NRC/NIST_NRC_09_v5_Cable_G_Temp} \\
\includegraphics[height=2.2in]{FIGURES/NIST_NRC/NIST_NRC_05_v5_Cable_G_Temp} &
\includegraphics[height=2.2in]{FIGURES/NIST_NRC/NIST_NRC_14_v5_Cable_G_Temp} \\
\includegraphics[height=2.2in]{FIGURES/NIST_NRC/NIST_NRC_15_v5_Cable_G_Temp} &
\includegraphics[height=2.2in]{FIGURES/NIST_NRC/NIST_NRC_18_v5_Cable_G_Temp}
\end{tabular*}
\label{NIST_NRC_Cable_G_Open}
\end{figure}

\begin{figure}[p]
\begin{center}
\begin{tabular}{c}
\includegraphics[width=5.0in]{FIGURES/ScatterPlots/Cable_Temperature} \\
\vspace{0.25in}
\end{tabular}
\end{center}
\caption[Summary of cable surface temperature predictions, NIST/NRC test series.]
{Summary of cable surface temperature predictions for the NIST/NRC test series.}
\end{figure}




\clearpage

\section{WTC Ceiling and Wall Temperatures}

The following pages contain comparisons of predicted and measured ceiling temperatures, both at the surface and beneath a layer of
marinite board. Table~\ref{WTC_Ceiling} below lists the coordinates of the measurement locations relative to the center of the fire pan.
Names with ``IN'' appended are measurements made under the marinite board.


\begin{table}[h!]
\caption{Locations of ceiling surface temperature measurements relative to the fire pan in the WTC series.}
\begin{center}
\begin{tabular}{|l|c|c|c|}
\hline
Name                & $x$ (m)   & $y$ (m)   & $z$ (m)   \\ \hline \hline
TCC                 & 0.62      & 0.07      & 3.82      \\ \hline
TCN3                & 0.62      & 0.67      & 3.82      \\ \hline
TCS3                & 0.62      & -0.53     & 3.82      \\ \hline
TCE7                & 2.18      & 0.07      & 3.82      \\ \hline
TCW7                & -1.15     & 0.07      & 3.82      \\ \hline \hline
TCCIN               & 0.62      & 0.07      & 3.83      \\ \hline
TCN3IN              & 0.62      & 0.67      & 3.83      \\ \hline
TCS3IN              & 0.62      & -0.53     & 3.83      \\ \hline
TCE4IN              & 1.28      & 0.07      & 3.83      \\ \hline
TCW4IN              & 0.05      & 0.07      & 3.83      \\ \hline
\end{tabular}
\end{center}
\label{WTC_Ceiling}
\end{table}

\begin{figure}[p]
\begin{tabular*}{\textwidth}{l@{\extracolsep{\fill}}r}
\includegraphics[height=2.2in]{FIGURES/WTC/WTC_01_v5_Ceiling_Temp_NS} &
\includegraphics[height=2.2in]{FIGURES/WTC/WTC_02_v5_Ceiling_Temp_NS} \\
\includegraphics[height=2.2in]{FIGURES/WTC/WTC_03_v5_Ceiling_Temp_NS} &
\includegraphics[height=2.2in]{FIGURES/WTC/WTC_04_v5_Ceiling_Temp_NS} \\
\includegraphics[height=2.2in]{FIGURES/WTC/WTC_05_v5_Ceiling_Temp_NS} &
\includegraphics[height=2.2in]{FIGURES/WTC/WTC_06_v5_Ceiling_Temp_NS}
\end{tabular*}
\label{NIST_WTC_Ceiling_NS}
\end{figure}

\begin{figure}[p]
\begin{tabular*}{\textwidth}{l@{\extracolsep{\fill}}r}
\includegraphics[height=2.2in]{FIGURES/WTC/WTC_01_v5_Ceiling_Temp_EW} &
\includegraphics[height=2.2in]{FIGURES/WTC/WTC_02_v5_Ceiling_Temp_EW} \\
\includegraphics[height=2.2in]{FIGURES/WTC/WTC_03_v5_Ceiling_Temp_EW} &
\includegraphics[height=2.2in]{FIGURES/WTC/WTC_04_v5_Ceiling_Temp_EW} \\
\includegraphics[height=2.2in]{FIGURES/WTC/WTC_05_v5_Ceiling_Temp_EW} &
\includegraphics[height=2.2in]{FIGURES/WTC/WTC_06_v5_Ceiling_Temp_EW}
\end{tabular*}
\label{NIST_WTC_Ceiling_EW}
\end{figure}

\begin{figure}[p]
\begin{tabular*}{\textwidth}{l@{\extracolsep{\fill}}r}
\includegraphics[height=2.2in]{FIGURES/WTC/WTC_01_v5_Wall_Temp_98_100_102} &
\includegraphics[height=2.2in]{FIGURES/WTC/WTC_02_v5_Wall_Temp_98_100_102} \\
\includegraphics[height=2.2in]{FIGURES/WTC/WTC_03_v5_Wall_Temp_98_100_102} &
\includegraphics[height=2.2in]{FIGURES/WTC/WTC_04_v5_Wall_Temp_98_100_102} \\
\includegraphics[height=2.2in]{FIGURES/WTC/WTC_05_v5_Wall_Temp_98_100_102} &
\includegraphics[height=2.2in]{FIGURES/WTC/WTC_06_v5_Wall_Temp_98_100_102}
\end{tabular*}
\label{NIST_WTC_Wall_98_100_102}
\end{figure}

\begin{figure}[p]
\begin{tabular*}{\textwidth}{l@{\extracolsep{\fill}}r}
\includegraphics[height=2.2in]{FIGURES/WTC/WTC_01_v5_Wall_Temp_103_105_106} &
\includegraphics[height=2.2in]{FIGURES/WTC/WTC_02_v5_Wall_Temp_103_105_106} \\
\includegraphics[height=2.2in]{FIGURES/WTC/WTC_03_v5_Wall_Temp_103_105_106} &
\includegraphics[height=2.2in]{FIGURES/WTC/WTC_04_v5_Wall_Temp_103_105_106} \\
\includegraphics[height=2.2in]{FIGURES/WTC/WTC_05_v5_Wall_Temp_103_105_106} &
\includegraphics[height=2.2in]{FIGURES/WTC/WTC_06_v5_Wall_Temp_103_105_106}
\end{tabular*}
\label{NIST_WTC_Wall_103_105_106}
\end{figure}

\begin{figure}[p]
\begin{tabular*}{\textwidth}{l@{\extracolsep{\fill}}r}
\includegraphics[height=2.2in]{FIGURES/WTC/WTC_01_v5_Wall_Temp_107_109_110} &
\includegraphics[height=2.2in]{FIGURES/WTC/WTC_02_v5_Wall_Temp_107_109_110} \\
\includegraphics[height=2.2in]{FIGURES/WTC/WTC_03_v5_Wall_Temp_107_109_110} &
\includegraphics[height=2.2in]{FIGURES/WTC/WTC_04_v5_Wall_Temp_107_109_110} \\
\includegraphics[height=2.2in]{FIGURES/WTC/WTC_05_v5_Wall_Temp_107_109_110} &
\includegraphics[height=2.2in]{FIGURES/WTC/WTC_06_v5_Wall_Temp_107_109_110}
\end{tabular*}
\label{NIST_WTC_Wall_107_109_110}
\end{figure}

\begin{figure}[p]
\begin{tabular*}{\textwidth}{l@{\extracolsep{\fill}}r}
\includegraphics[height=2.2in]{FIGURES/WTC/WTC_01_v5_Inner_Ceiling_Temp_NS} &
\includegraphics[height=2.2in]{FIGURES/WTC/WTC_02_v5_Inner_Ceiling_Temp_NS} \\
\includegraphics[height=2.2in]{FIGURES/WTC/WTC_03_v5_Inner_Ceiling_Temp_NS} &
\includegraphics[height=2.2in]{FIGURES/WTC/WTC_04_v5_Inner_Ceiling_Temp_NS} \\
\includegraphics[height=2.2in]{FIGURES/WTC/WTC_05_v5_Inner_Ceiling_Temp_NS} &
\includegraphics[height=2.2in]{FIGURES/WTC/WTC_06_v5_Inner_Ceiling_Temp_NS}
\end{tabular*}
\label{NIST_WTC_Inner_Ceiling_NS}
\end{figure}

\begin{figure}[p]
\begin{tabular*}{\textwidth}{l@{\extracolsep{\fill}}r}
\includegraphics[height=2.2in]{FIGURES/WTC/WTC_01_v5_Inner_Ceiling_Temp_EW} &
\includegraphics[height=2.2in]{FIGURES/WTC/WTC_02_v5_Inner_Ceiling_Temp_EW} \\
\includegraphics[height=2.2in]{FIGURES/WTC/WTC_03_v5_Inner_Ceiling_Temp_EW} &
\includegraphics[height=2.2in]{FIGURES/WTC/WTC_04_v5_Inner_Ceiling_Temp_EW} \\
\includegraphics[height=2.2in]{FIGURES/WTC/WTC_05_v5_Inner_Ceiling_Temp_EW} &
\includegraphics[height=2.2in]{FIGURES/WTC/WTC_06_v5_Inner_Ceiling_Temp_EW}
\end{tabular*}
\label{NIST_WTC_Inner_Ceiling_EW}
\end{figure}

\begin{figure}[p]
\begin{tabular*}{\textwidth}{l@{\extracolsep{\fill}}r}
\includegraphics[height=2.2in]{FIGURES/WTC/WTC_01_v5_Inner_Wall_Temp} &
\includegraphics[height=2.2in]{FIGURES/WTC/WTC_02_v5_Inner_Wall_Temp} \\
\includegraphics[height=2.2in]{FIGURES/WTC/WTC_03_v5_Inner_Wall_Temp} &
\includegraphics[height=2.2in]{FIGURES/WTC/WTC_04_v5_Inner_Wall_Temp} \\
\includegraphics[height=2.2in]{FIGURES/WTC/WTC_05_v5_Inner_Wall_Temp} &
\includegraphics[height=2.2in]{FIGURES/WTC/WTC_06_v5_Inner_Wall_Temp}
\end{tabular*}
\label{NIST_WTC_Inner_Wall}
\end{figure}

\begin{figure}[p]
\begin{center}
\begin{tabular}{c}
\includegraphics[width=5.0in]{FIGURES/ScatterPlots/WTC_Surface_Temperature}
\end{tabular}
\end{center}
\caption[Summary of wall and ceiling temperature predictions, WTC test series.]
{Summary of wall and ceiling temperature predictions for the WTC test series.}
\end{figure}


\clearpage





\section{NIST/NRC Test Series, Compartment Walls, Floor and Ceiling}

Thirty-six heat flux gauges were positioned at various locations on all four walls of the compartment,
plus the ceiling and floor.  Comparisons between measured and predicted heat fluxes and surface temperatures are shown
on the following pages for a selected number of locations.
Over half of the measurement points are in roughly the same relative location to the fire and hence
the measurements and predictions are similar.  For this reason, data for the east and north walls are shown
because the data from the south and west walls are comparable.  Data from the south wall is used in cases where
the corresponding instrument on the north wall failed, or in cases where the fire is positioned close to the south wall.
For each test, eight locations are used for comparison, two on the long (mainly north) wall,
two on the short (east) wall, two on the floor, and two on the ceiling.  Of the two locations for each panel,
one is considered in the far-field, relatively remote from the fire; one is in the near-field,
relatively close to the fire.  How close or far varies from test to test.
The two short wall locations are equally remote from the fire; thus, one location is in the lower layer, one in the upper.
Table~\ref{NIST_NRC_Wall_Coords} lists the locations for each test.


\vspace{\baselineskip}

\begin{table}[h!]
\caption[Wall thermocouple positions for the NIST/NRC series.]
{Wall thermocouple positions for the NIST/NRC series. The origin of the coordinate system at the floor in the southwest
corner of the compartment.}
\begin{center}
\begin{tabular}{|l|c|c|c||l|c|c|c|}
\hline
Name              & $x$   & $y$  & $z$      & Name              & $x$   & $y$   & $z$       \\ \hline \hline
TC North U-1-2    & 3.85  & 7.04 & 1.49     & TC Floor U-1-2    & 3.08  & 3.51  & 0         \\ \hline
TC North U-2-2    & 3.86  & 7.04 & 3.71     & TC Floor U-2-2    & 9.08  & 1.94  & 0         \\ \hline
TC North U-3-2    & 9.48  & 7.04 & 1.86     & TC Floor U-3-2    & 9.06  & 5.97  & 0         \\ \hline
TC North U-4-2    & 12.07 & 7.04 & 1.88     & TC Floor U-4-2    & 10.86 & 2.38  & 0         \\ \hline
TC North U-5-2    & 17.69 & 7.04 & 1.49     & TC Floor C-5-2    & 10.93 & 5.20  & 0.01      \\ \hline
TC North U-6-2    & 17.69 & 7.04 & 3.69     & TC Floor U-6-2    & 13.13 & 1.99  & 0         \\ \hline
TC South U-1-2    & 3.86  & 0    & 1.49     & TC Floor U-7-2    & 13.00 & 5.92  & 0         \\ \hline
TC South U-2-2    & 3.86  & 0    & 3.82     & TC Floor U-8-2    & 18.63 & 3.54  & 0         \\ \hline
TC South U-3-2    & 9.54  & 0    & 1.86     & TC Ceiling U-1-2  & 3.04  & 3.60  & 3.82      \\ \hline
TC South U-4-2    & 12.08 & 0    & 1.86     & TC Ceiling C-2-2  & 8.99  & 2.00  & 3.82      \\ \hline
TC South U-5-2    & 17.69 & 0    & 1.50     & TC Ceiling C-3-2  & 9.03  & 5.97  & 3.82      \\ \hline
TC South U-6-2    & 17.74 & 0    & 3.70     & TC Ceiling C-4-2  & 10.79 & 2.38  & 3.82      \\ \hline
TC East U-1-2     & 21.66 & 1.52 & 1.12     & TC Ceiling C-5-2  & 10.79 & 5.20  & 3.82      \\ \hline
TC East U-2-2     & 21.66 & 1.52 & 2.40     & TC Ceiling C-6-2  & 13.00 & 2.07  & 3.82      \\ \hline
TC East U-3-2     & 21.66 & 5.68 & 1.13     & TC Ceiling C-7-2  & 12.84 & 5.98  & 3.82      \\ \hline
TC East U-4-2     & 21.66 & 5.70 & 2.42     & TC Ceiling U-8-2  & 18.71 & 3.54  & 3.82      \\ \hline
\end{tabular}
\end{center}
\label{NIST_NRC_Wall_Coords}
\end{table}


\begin{figure}[p]
\begin{tabular*}{\textwidth}{l@{\extracolsep{\fill}}r}
\includegraphics[height=2.2in]{FIGURES/NIST_NRC/NIST_NRC_01_v5_Long_Wall_Temp} &
\includegraphics[height=2.2in]{FIGURES/NIST_NRC/NIST_NRC_07_v5_Long_Wall_Temp} \\
\includegraphics[height=2.2in]{FIGURES/NIST_NRC/NIST_NRC_02_v5_Long_Wall_Temp} &
\includegraphics[height=2.2in]{FIGURES/NIST_NRC/NIST_NRC_08_v5_Long_Wall_Temp} \\
\includegraphics[height=2.2in]{FIGURES/NIST_NRC/NIST_NRC_04_v5_Long_Wall_Temp} &
\includegraphics[height=2.2in]{FIGURES/NIST_NRC/NIST_NRC_10_v5_Long_Wall_Temp} \\
\includegraphics[height=2.2in]{FIGURES/NIST_NRC/NIST_NRC_13_v5_Long_Wall_Temp} &
\includegraphics[height=2.2in]{FIGURES/NIST_NRC/NIST_NRC_16_v5_Long_Wall_Temp}
\end{tabular*}
\label{NIST_NRC_Long_Wall_Temp_Closed}
\end{figure}

\begin{figure}[p]
\begin{tabular*}{\textwidth}{l@{\extracolsep{\fill}}r}
\includegraphics[height=2.2in]{FIGURES/NIST_NRC/NIST_NRC_03_v5_Long_Wall_Temp} &
\includegraphics[height=2.2in]{FIGURES/NIST_NRC/NIST_NRC_09_v5_Long_Wall_Temp} \\
\includegraphics[height=2.2in]{FIGURES/NIST_NRC/NIST_NRC_05_v5_Long_Wall_Temp} &
\includegraphics[height=2.2in]{FIGURES/NIST_NRC/NIST_NRC_14_v5_Long_Wall_Temp} \\
\includegraphics[height=2.2in]{FIGURES/NIST_NRC/NIST_NRC_15_v5_Long_Wall_Temp} &
\includegraphics[height=2.2in]{FIGURES/NIST_NRC/NIST_NRC_18_v5_Long_Wall_Temp}
\end{tabular*}
\label{NIST_NRC_Long_Wall_Temp_Open}
\end{figure}

\begin{figure}[p]
\begin{tabular*}{\textwidth}{l@{\extracolsep{\fill}}r}
\includegraphics[height=2.2in]{FIGURES/NIST_NRC/NIST_NRC_01_v5_Short_Wall_Temp} &
\includegraphics[height=2.2in]{FIGURES/NIST_NRC/NIST_NRC_07_v5_Short_Wall_Temp} \\
\includegraphics[height=2.2in]{FIGURES/NIST_NRC/NIST_NRC_02_v5_Short_Wall_Temp} &
\includegraphics[height=2.2in]{FIGURES/NIST_NRC/NIST_NRC_08_v5_Short_Wall_Temp} \\
\includegraphics[height=2.2in]{FIGURES/NIST_NRC/NIST_NRC_04_v5_Short_Wall_Temp} &
\includegraphics[height=2.2in]{FIGURES/NIST_NRC/NIST_NRC_10_v5_Short_Wall_Temp} \\
\includegraphics[height=2.2in]{FIGURES/NIST_NRC/NIST_NRC_13_v5_Short_Wall_Temp} &
\includegraphics[height=2.2in]{FIGURES/NIST_NRC/NIST_NRC_16_v5_Short_Wall_Temp}
\end{tabular*}
\label{NIST_NRC_Short_Wall_Temp_Closed}
\end{figure}

\begin{figure}[p]
\begin{tabular*}{\textwidth}{l@{\extracolsep{\fill}}r}
\includegraphics[height=2.2in]{FIGURES/NIST_NRC/NIST_NRC_03_v5_Short_Wall_Temp} &
\includegraphics[height=2.2in]{FIGURES/NIST_NRC/NIST_NRC_09_v5_Short_Wall_Temp} \\
\includegraphics[height=2.2in]{FIGURES/NIST_NRC/NIST_NRC_05_v5_Short_Wall_Temp} &
\includegraphics[height=2.2in]{FIGURES/NIST_NRC/NIST_NRC_14_v5_Short_Wall_Temp} \\
\includegraphics[height=2.2in]{FIGURES/NIST_NRC/NIST_NRC_15_v5_Short_Wall_Temp} &
\includegraphics[height=2.2in]{FIGURES/NIST_NRC/NIST_NRC_18_v5_Short_Wall_Temp}
\end{tabular*}
\label{NIST_NRC_Short_Wall_Temp_Open}
\end{figure}

\begin{figure}[p]
\begin{tabular*}{\textwidth}{l@{\extracolsep{\fill}}r}
\includegraphics[height=2.2in]{FIGURES/NIST_NRC/NIST_NRC_01_v5_Ceiling_Temp} &
\includegraphics[height=2.2in]{FIGURES/NIST_NRC/NIST_NRC_07_v5_Ceiling_Temp} \\
\includegraphics[height=2.2in]{FIGURES/NIST_NRC/NIST_NRC_02_v5_Ceiling_Temp} &
\includegraphics[height=2.2in]{FIGURES/NIST_NRC/NIST_NRC_08_v5_Ceiling_Temp} \\
\includegraphics[height=2.2in]{FIGURES/NIST_NRC/NIST_NRC_04_v5_Ceiling_Temp} &
\includegraphics[height=2.2in]{FIGURES/NIST_NRC/NIST_NRC_10_v5_Ceiling_Temp} \\
\includegraphics[height=2.2in]{FIGURES/NIST_NRC/NIST_NRC_13_v5_Ceiling_Temp} &
\includegraphics[height=2.2in]{FIGURES/NIST_NRC/NIST_NRC_16_v5_Ceiling_Temp}
\end{tabular*}
\label{NIST_NRC_Ceiling_Temp_Closed}
\end{figure}

\begin{figure}[p]
\begin{tabular*}{\textwidth}{l@{\extracolsep{\fill}}r}
\includegraphics[height=2.2in]{FIGURES/NIST_NRC/NIST_NRC_03_v5_Ceiling_Temp} &
\includegraphics[height=2.2in]{FIGURES/NIST_NRC/NIST_NRC_09_v5_Ceiling_Temp} \\
\includegraphics[height=2.2in]{FIGURES/NIST_NRC/NIST_NRC_05_v5_Ceiling_Temp} &
\includegraphics[height=2.2in]{FIGURES/NIST_NRC/NIST_NRC_14_v5_Ceiling_Temp} \\
\includegraphics[height=2.2in]{FIGURES/NIST_NRC/NIST_NRC_15_v5_Ceiling_Temp} &
\includegraphics[height=2.2in]{FIGURES/NIST_NRC/NIST_NRC_18_v5_Ceiling_Temp}
\end{tabular*}
\label{NIST_NRC_Ceiling_Temp_Open}
\end{figure}

\begin{figure}[p]
\begin{tabular*}{\textwidth}{l@{\extracolsep{\fill}}r}
\includegraphics[height=2.2in]{FIGURES/NIST_NRC/NIST_NRC_01_v5_Floor_Temp} &
\includegraphics[height=2.2in]{FIGURES/NIST_NRC/NIST_NRC_07_v5_Floor_Temp} \\
\includegraphics[height=2.2in]{FIGURES/NIST_NRC/NIST_NRC_02_v5_Floor_Temp} &
\includegraphics[height=2.2in]{FIGURES/NIST_NRC/NIST_NRC_08_v5_Floor_Temp} \\
\includegraphics[height=2.2in]{FIGURES/NIST_NRC/NIST_NRC_04_v5_Floor_Temp} &
\includegraphics[height=2.2in]{FIGURES/NIST_NRC/NIST_NRC_10_v5_Floor_Temp} \\
\includegraphics[height=2.2in]{FIGURES/NIST_NRC/NIST_NRC_13_v5_Floor_Temp} &
\includegraphics[height=2.2in]{FIGURES/NIST_NRC/NIST_NRC_16_v5_Floor_Temp}
\end{tabular*}
\label{NIST_NRC_Floor_Temp_Closed}
\end{figure}

\begin{figure}[p]
\begin{tabular*}{\textwidth}{l@{\extracolsep{\fill}}r}
\includegraphics[height=2.2in]{FIGURES/NIST_NRC/NIST_NRC_03_v5_Floor_Temp} &
\includegraphics[height=2.2in]{FIGURES/NIST_NRC/NIST_NRC_09_v5_Floor_Temp} \\
\includegraphics[height=2.2in]{FIGURES/NIST_NRC/NIST_NRC_05_v5_Floor_Temp} &
\includegraphics[height=2.2in]{FIGURES/NIST_NRC/NIST_NRC_14_v5_Floor_Temp} \\
\includegraphics[height=2.2in]{FIGURES/NIST_NRC/NIST_NRC_15_v5_Floor_Temp} &
\includegraphics[height=2.2in]{FIGURES/NIST_NRC/NIST_NRC_18_v5_Floor_Temp}
\end{tabular*}
\label{NIST_NRC_Floor_Temp_Open}
\end{figure}



\begin{figure}[p]
\begin{center}
\begin{tabular}{c}
\includegraphics[width=5.0in]{FIGURES/ScatterPlots/Wall_Temperature}
\end{tabular}
\end{center}
\caption[Summary of wall, floor and ceiling temperature predictions, NIST/NRC test series.]
{Summary of wall, floor and ceiling temperature predictions for the NIST/NRC test series.}
\end{figure}


\chapter{Heat Flux and Surface Temperature}



\clearpage

\section{NIST/WTC Test Series, Steel Structural Members}

A single box column was installed in the test compartment, about 1~m away from the fire. The column was instrumented near its base (about
0.5~m from the floor, middle (1.5~m), and upper (2.5~m). Four measurements of steel (and insulation) temperatures were made at each location, for
each of its four sides.

\clearpage

\subsection{NIST/WTC Test Series, Column Steel Temperatures}

\vspace{1in}

\begin{figure}[h!]
\begin{tabular*}{\textwidth}{l@{\extracolsep{\fill}}r}
\includegraphics[width=2.6in]{FIGURES/WTC_01_v5_Upper_Column_Steel_Temp} &
\includegraphics[width=2.6in]{FIGURES/WTC_02_v5_Upper_Column_Steel_Temp} \\
\includegraphics[width=2.6in]{FIGURES/WTC_03_v5_Upper_Column_Steel_Temp} &
\includegraphics[width=2.6in]{FIGURES/WTC_04_v5_Upper_Column_Steel_Temp} \\
\includegraphics[width=2.6in]{FIGURES/WTC_05_v5_Upper_Column_Steel_Temp} &
\includegraphics[width=2.6in]{FIGURES/WTC_06_v5_Upper_Column_Steel_Temp}
\end{tabular*}
\caption{Upper Column Steel Temperatures for the NIST/WTC Test Series.}
\label{NIST_WTC_Upper_Column_Steel}
\end{figure}

\begin{figure}[p]
\begin{tabular*}{\textwidth}{l@{\extracolsep{\fill}}r}
\includegraphics[width=2.6in]{FIGURES/WTC_01_v5_Middle_Column_Steel_Temp} &
\includegraphics[width=2.6in]{FIGURES/WTC_02_v5_Middle_Column_Steel_Temp} \\
\includegraphics[width=2.6in]{FIGURES/WTC_03_v5_Middle_Column_Steel_Temp} &
\includegraphics[width=2.6in]{FIGURES/WTC_04_v5_Middle_Column_Steel_Temp} \\
\includegraphics[width=2.6in]{FIGURES/WTC_05_v5_Middle_Column_Steel_Temp} &
\includegraphics[width=2.6in]{FIGURES/WTC_06_v5_Middle_Column_Steel_Temp}
\end{tabular*}
\caption{Middle Column Steel Temperatures for the NIST/WTC Test Series.}
\label{NIST_WTC_Middle_Column_Steel}
\end{figure}

\begin{figure}[p]
\begin{tabular*}{\textwidth}{l@{\extracolsep{\fill}}r}
\includegraphics[width=2.6in]{FIGURES/WTC_01_v5_Lower_Column_Steel_Temp} &
\includegraphics[width=2.6in]{FIGURES/WTC_02_v5_Lower_Column_Steel_Temp} \\
\includegraphics[width=2.6in]{FIGURES/WTC_03_v5_Lower_Column_Steel_Temp} &
\includegraphics[width=2.6in]{FIGURES/WTC_04_v5_Lower_Column_Steel_Temp} \\
\includegraphics[width=2.6in]{FIGURES/WTC_05_v5_Lower_Column_Steel_Temp} &
\includegraphics[width=2.6in]{FIGURES/WTC_06_v5_Lower_Column_Steel_Temp}
\end{tabular*}
\caption{Lower Column Steel Temperatures for the NIST/WTC Test Series.}
\label{NIST_WTC_Lower_Column_Steel}
\end{figure}

\clearpage



\subsection{NIST/WTC Test Series, Truss A Steel Temperatures}

\vspace{1in}

\begin{figure}[h!]
\begin{tabular*}{\textwidth}{l@{\extracolsep{\fill}}r}
\includegraphics[width=2.6in]{FIGURES/WTC_01_v5_Truss_A_Upper_Steel_Temp} &
\includegraphics[width=2.6in]{FIGURES/WTC_02_v5_Truss_A_Upper_Steel_Temp} \\
\includegraphics[width=2.6in]{FIGURES/WTC_03_v5_Truss_A_Upper_Steel_Temp} &
\includegraphics[width=2.6in]{FIGURES/WTC_04_v5_Truss_A_Upper_Steel_Temp} \\
\includegraphics[width=2.6in]{FIGURES/WTC_05_v5_Truss_A_Upper_Steel_Temp} &
\includegraphics[width=2.6in]{FIGURES/WTC_06_v5_Truss_A_Upper_Steel_Temp}
\end{tabular*}
\caption{Truss A Upper Flange Steel Temperatures for the NIST/WTC Test Series.}
\label{NIST_WTC_Truss_A_Upper_Steel}
\end{figure}

\begin{figure}[p]
\begin{tabular*}{\textwidth}{l@{\extracolsep{\fill}}r}
\includegraphics[width=2.6in]{FIGURES/WTC_01_v5_Truss_A_Middle_Steel_Temp} &
\includegraphics[width=2.6in]{FIGURES/WTC_02_v5_Truss_A_Middle_Steel_Temp} \\
\includegraphics[width=2.6in]{FIGURES/WTC_03_v5_Truss_A_Middle_Steel_Temp} &
\includegraphics[width=2.6in]{FIGURES/WTC_04_v5_Truss_A_Middle_Steel_Temp} \\
\includegraphics[width=2.6in]{FIGURES/WTC_05_v5_Truss_A_Middle_Steel_Temp} &
\includegraphics[width=2.6in]{FIGURES/WTC_06_v5_Truss_A_Middle_Steel_Temp}
\end{tabular*}
\caption{Truss A Web Steel Temperatures for the NIST/WTC Test Series.}
\label{NIST_WTC_Truss_A_Middle_Steel}
\end{figure}

\begin{figure}[p]
\begin{tabular*}{\textwidth}{l@{\extracolsep{\fill}}r}
\includegraphics[width=2.6in]{FIGURES/WTC_01_v5_Truss_A_Lower_Steel_Temp} &
\includegraphics[width=2.6in]{FIGURES/WTC_02_v5_Truss_A_Lower_Steel_Temp} \\
\includegraphics[width=2.6in]{FIGURES/WTC_03_v5_Truss_A_Lower_Steel_Temp} &
\includegraphics[width=2.6in]{FIGURES/WTC_04_v5_Truss_A_Lower_Steel_Temp} \\
\includegraphics[width=2.6in]{FIGURES/WTC_05_v5_Truss_A_Lower_Steel_Temp} &
\includegraphics[width=2.6in]{FIGURES/WTC_06_v5_Truss_A_Lower_Steel_Temp}
\end{tabular*}
\caption{Truss A Lower Flange Steel Temperatures for the NIST/WTC Test Series.}
\label{NIST_WTC_Truss_A_Lower_Steel}
\end{figure}

\clearpage


\subsection{NIST/WTC Test Series, Truss B Steel Temperatures}

\vspace{1in}

\begin{figure}[h!]
\begin{tabular*}{\textwidth}{l@{\extracolsep{\fill}}r}
\includegraphics[width=2.6in]{FIGURES/WTC_01_v5_Truss_B_Upper_Steel_Temp} &
\includegraphics[width=2.6in]{FIGURES/WTC_02_v5_Truss_B_Upper_Steel_Temp} \\
\includegraphics[width=2.6in]{FIGURES/WTC_03_v5_Truss_B_Upper_Steel_Temp} &
\includegraphics[width=2.6in]{FIGURES/WTC_04_v5_Truss_B_Upper_Steel_Temp} \\
\includegraphics[width=2.6in]{FIGURES/WTC_05_v5_Truss_B_Upper_Steel_Temp} &
\includegraphics[width=2.6in]{FIGURES/WTC_06_v5_Truss_B_Upper_Steel_Temp}
\end{tabular*}
\caption{Truss B Upper Flange Steel Temperatures for the NIST/WTC Test Series.}
\label{NIST_WTC_Truss_B_Upper_Steel}
\end{figure}

\begin{figure}[p]
\begin{tabular*}{\textwidth}{l@{\extracolsep{\fill}}r}
\includegraphics[width=2.6in]{FIGURES/WTC_01_v5_Truss_B_Middle_Steel_Temp} &
\includegraphics[width=2.6in]{FIGURES/WTC_02_v5_Truss_B_Middle_Steel_Temp} \\
\includegraphics[width=2.6in]{FIGURES/WTC_03_v5_Truss_B_Middle_Steel_Temp} &
\includegraphics[width=2.6in]{FIGURES/WTC_04_v5_Truss_B_Middle_Steel_Temp} \\
\includegraphics[width=2.6in]{FIGURES/WTC_05_v5_Truss_B_Middle_Steel_Temp} &
\includegraphics[width=2.6in]{FIGURES/WTC_06_v5_Truss_B_Middle_Steel_Temp}
\end{tabular*}
\caption{Truss B Web Steel Temperatures for the NIST/WTC Test Series.}
\label{NIST_WTC_Truss_B_Middle_Steel}
\end{figure}

\begin{figure}[p]
\begin{tabular*}{\textwidth}{l@{\extracolsep{\fill}}r}
\includegraphics[width=2.6in]{FIGURES/WTC_01_v5_Truss_B_Lower_Steel_Temp} &
\includegraphics[width=2.6in]{FIGURES/WTC_02_v5_Truss_B_Lower_Steel_Temp} \\
\includegraphics[width=2.6in]{FIGURES/WTC_03_v5_Truss_B_Lower_Steel_Temp} &
\includegraphics[width=2.6in]{FIGURES/WTC_04_v5_Truss_B_Lower_Steel_Temp} \\
\includegraphics[width=2.6in]{FIGURES/WTC_05_v5_Truss_B_Lower_Steel_Temp} &
\includegraphics[width=2.6in]{FIGURES/WTC_06_v5_Truss_B_Lower_Steel_Temp}
\end{tabular*}
\caption{Truss B Lower Flange Steel Temperatures for the NIST/WTC Test Series.}
\label{NIST_WTC_Truss_B_Lower_Steel}
\end{figure}

\clearpage


\subsection{NIST/WTC Test Series, Steel Bar Temperatures}

\vspace{1in}

\begin{figure}[h!]
\begin{tabular*}{\textwidth}{l@{\extracolsep{\fill}}r}
\includegraphics[width=2.6in]{FIGURES/WTC_01_v5_Bar_1_Steel_Temp} &
\includegraphics[width=2.6in]{FIGURES/WTC_02_v5_Bar_1_Steel_Temp} \\
\includegraphics[width=2.6in]{FIGURES/WTC_03_v5_Bar_1_Steel_Temp} &
\includegraphics[width=2.6in]{FIGURES/WTC_04_v5_Bar_1_Steel_Temp} \\
\includegraphics[width=2.6in]{FIGURES/WTC_05_v5_Bar_1_Steel_Temp} &
\includegraphics[width=2.6in]{FIGURES/WTC_06_v5_Bar_1_Steel_Temp}
\end{tabular*}
\caption{Bar 1 Steel Temperatures for the NIST/WTC Test Series.}
\label{NIST_WTC Bar_1_Steel}
\end{figure}

\begin{figure}[h!]
\begin{tabular*}{\textwidth}{l@{\extracolsep{\fill}}r}
\includegraphics[width=2.6in]{FIGURES/WTC_01_v5_Bar_2_Steel_Temp} &
\includegraphics[width=2.6in]{FIGURES/WTC_02_v5_Bar_2_Steel_Temp} \\
\includegraphics[width=2.6in]{FIGURES/WTC_03_v5_Bar_2_Steel_Temp} &
 \\
\end{tabular*}
\caption{Bar 2 Steel Temperatures for NIST/WTC Tests 1, 2 and 3.}
\label{NIST_WTC Bar_2_Steel}
\end{figure}



\clearpage


\subsection{NIST/WTC Test Series, Nickel ``Slug'' Temperatures}

\vspace{1in}

\begin{figure}[h!]
\begin{tabular*}{\textwidth}{l@{\extracolsep{\fill}}r}
\includegraphics[width=2.6in]{FIGURES/WTC_01_v5_Slug_1_Temp} &
\includegraphics[width=2.6in]{FIGURES/WTC_02_v5_Slug_1_Temp} \\
\includegraphics[width=2.6in]{FIGURES/WTC_03_v5_Slug_1_Temp} &
\includegraphics[width=2.6in]{FIGURES/WTC_04_v5_Slug_1_Temp} \\
\includegraphics[width=2.6in]{FIGURES/WTC_05_v5_Slug_1_Temp} &
\includegraphics[width=2.6in]{FIGURES/WTC_06_v5_Slug_1_Temp}
\end{tabular*}
\caption{Nickel ``Slug'' 1 Temperatures for the NIST/WTC Test Series.}
\label{NIST_WTC Slug_1}
\end{figure}

\begin{figure}[h!]
\begin{tabular*}{\textwidth}{l@{\extracolsep{\fill}}r}
\includegraphics[width=2.6in]{FIGURES/WTC_01_v5_Slug_2_Temp} &
\includegraphics[width=2.6in]{FIGURES/WTC_02_v5_Slug_2_Temp} \\
\includegraphics[width=2.6in]{FIGURES/WTC_03_v5_Slug_2_Temp} &
\includegraphics[width=2.6in]{FIGURES/WTC_04_v5_Slug_2_Temp} \\
\includegraphics[width=2.6in]{FIGURES/WTC_05_v5_Slug_2_Temp} &
\includegraphics[width=2.6in]{FIGURES/WTC_06_v5_Slug_2_Temp}
\end{tabular*}
\caption{Nickel ``Slug'' 2 Temperatures for the NIST/WTC Test Series.}
\label{NIST_WTC Slug_2}
\end{figure}

\begin{figure}[h!]
\begin{tabular*}{\textwidth}{l@{\extracolsep{\fill}}r}
\includegraphics[width=2.6in]{FIGURES/WTC_01_v5_Slug_3_Temp} &
\includegraphics[width=2.6in]{FIGURES/WTC_02_v5_Slug_3_Temp} \\
\includegraphics[width=2.6in]{FIGURES/WTC_03_v5_Slug_3_Temp} &
\includegraphics[width=2.6in]{FIGURES/WTC_04_v5_Slug_3_Temp} \\
\includegraphics[width=2.6in]{FIGURES/WTC_05_v5_Slug_3_Temp} &
\includegraphics[width=2.6in]{FIGURES/WTC_06_v5_Slug_3_Temp}
\end{tabular*}
\caption{Nickel ``Slug'' 3 Temperatures for the NIST/WTC Test Series.}
\label{NIST_WTC Slug_3}
\end{figure}

\begin{figure}[h!]
\begin{tabular*}{\textwidth}{l@{\extracolsep{\fill}}r}
\includegraphics[width=2.6in]{FIGURES/WTC_01_v5_Slug_4_Temp} &
\includegraphics[width=2.6in]{FIGURES/WTC_02_v5_Slug_4_Temp} \\
\includegraphics[width=2.6in]{FIGURES/WTC_03_v5_Slug_4_Temp} &
\includegraphics[width=2.6in]{FIGURES/WTC_04_v5_Slug_4_Temp} \\
\includegraphics[width=2.6in]{FIGURES/WTC_05_v5_Slug_4_Temp} &
\includegraphics[width=2.6in]{FIGURES/WTC_06_v5_Slug_4_Temp}
\end{tabular*}
\caption{Nickel ``Slug'' 4 Temperatures for the NIST/WTC Test Series.}
\label{NIST_WTC Slug_4}
\end{figure}

\begin{figure}[h!]
\begin{tabular*}{\textwidth}{l@{\extracolsep{\fill}}r}
\includegraphics[width=2.6in]{FIGURES/WTC_01_v5_Slug_5_Temp} &
\includegraphics[width=2.6in]{FIGURES/WTC_02_v5_Slug_5_Temp} \\
\includegraphics[width=2.6in]{FIGURES/WTC_03_v5_Slug_5_Temp} &
\includegraphics[width=2.6in]{FIGURES/WTC_04_v5_Slug_5_Temp} \\
\includegraphics[width=2.6in]{FIGURES/WTC_05_v5_Slug_5_Temp} &
\includegraphics[width=2.6in]{FIGURES/WTC_06_v5_Slug_5_Temp}
\end{tabular*}
\caption{Nickel ``Slug'' 5 Temperatures for the NIST/WTC Test Series.}
\label{NIST_WTC Slug_5}
\end{figure}




\clearpage

\section{NIST/NRC Test Series, Cables}

Cables in various types (power and control), and configurations (horizontal, vertical, in trays or free-hanging), were installed in
the test compartment.
For each of the four cable targets considered, measurements of the local gas temperature, surface temperature, radiative heat flux,
and total heat flux are available.  The following pages display comparisons of these quantities for
Control Cable B, Horizontal Cable Tray D, Power Cable F and Vertical Cable Tray G.
FDS does not have a detailed solid phase model that can account for the heat transfer within the bundled,
cylindrical, non-homogenous cables.  For the bundled cables within horizontal and vertical trays (Targets D and G),
FDS assumes them to be rectangular slabs of thickness comparable to the diameter of the individual cables.
For the free-hanging cables B and F, FDS assumes them to be cylinders of uniform composition into which it
computes the radial heat transfer as a function of the heat flux to a designated location.
The superposition of gas temperature, heat flux and surface temperature in the figures on the following pages
provides information about how cables heat up in fires.  Favorable or unfavorable predictions of cable surface
temperatures can often be explained in terms of comparable errors in the prediction of the thermal environment in the vicinity of the cable.

\clearpage

\subsection{Free-Hanging Control Cable B}

\vspace{1in}

\begin{figure}[h!]
\begin{tabular*}{\textwidth}{l@{\extracolsep{\fill}}r}
\includegraphics[width=2.6in]{FIGURES/NIST_NRC_01_v5_B_Cable_Gas_Temp_4-8} &
\includegraphics[width=2.6in]{FIGURES/NIST_NRC_07_v5_B_Cable_Gas_Temp_4-8} \\
\includegraphics[width=2.6in]{FIGURES/NIST_NRC_01_v5_B_Cable_Heat_Flux} &
\includegraphics[width=2.6in]{FIGURES/NIST_NRC_07_v5_B_Cable_Heat_Flux} \\
\includegraphics[width=2.6in]{FIGURES/NIST_NRC_01_v5_B_Cable_TC} &
\includegraphics[width=2.6in]{FIGURES/NIST_NRC_07_v5_B_Cable_TC}
\end{tabular*}
\caption{NIST/NRC Series, Cable B, Replicate Tests 1 and 7.}
\label{NIST_NRC_B_1_and_7}
\end{figure}

\begin{figure}[h]
\begin{tabular*}{\textwidth}{l@{\extracolsep{\fill}}r}
\includegraphics[width=2.6in]{FIGURES/NIST_NRC_02_v5_B_Cable_Gas_Temp_4-8} &
\includegraphics[width=2.6in]{FIGURES/NIST_NRC_08_v5_B_Cable_Gas_Temp_4-8} \\
\includegraphics[width=2.6in]{FIGURES/NIST_NRC_02_v5_B_Cable_Heat_Flux} &
\includegraphics[width=2.6in]{FIGURES/NIST_NRC_08_v5_B_Cable_Heat_Flux} \\
\includegraphics[width=2.6in]{FIGURES/NIST_NRC_02_v5_B_Cable_TC} &
\includegraphics[width=2.6in]{FIGURES/NIST_NRC_08_v5_B_Cable_TC}
\end{tabular*}
\caption{NIST/NRC Series, Cable B, Replicate Tests 2 and 8.}
\label{NIST_NRC_B_2_and_8}
\end{figure}

\begin{figure}[h]
\begin{tabular*}{\textwidth}{l@{\extracolsep{\fill}}r}
\includegraphics[width=2.6in]{FIGURES/NIST_NRC_04_v5_B_Cable_Gas_Temp_4-8} &
\includegraphics[width=2.6in]{FIGURES/NIST_NRC_10_v5_B_Cable_Gas_Temp_4-8} \\
\includegraphics[width=2.6in]{FIGURES/NIST_NRC_04_v5_B_Cable_Heat_Flux} &
\includegraphics[width=2.6in]{FIGURES/NIST_NRC_10_v5_B_Cable_Heat_Flux} \\
\includegraphics[width=2.6in]{FIGURES/NIST_NRC_04_v5_B_Cable_TC} &
\includegraphics[width=2.6in]{FIGURES/NIST_NRC_10_v5_B_Cable_TC}
\end{tabular*}
\caption{NIST/NRC Series, Cable B, Replicate Tests 4 and 10.}
\label{NIST_NRC_B_4_and_10}
\end{figure}

\begin{figure}[h]
\begin{tabular*}{\textwidth}{l@{\extracolsep{\fill}}r}
\includegraphics[width=2.6in]{FIGURES/NIST_NRC_13_v5_B_Cable_Gas_Temp_4-8} &
\includegraphics[width=2.6in]{FIGURES/NIST_NRC_16_v5_B_Cable_Gas_Temp_4-8} \\
\includegraphics[width=2.6in]{FIGURES/NIST_NRC_13_v5_B_Cable_Heat_Flux} &
\includegraphics[width=2.6in]{FIGURES/NIST_NRC_16_v5_B_Cable_Heat_Flux} \\
\includegraphics[width=2.6in]{FIGURES/NIST_NRC_13_v5_B_Cable_TC} &
\includegraphics[width=2.6in]{FIGURES/NIST_NRC_16_v5_B_Cable_TC}
\end{tabular*}
\caption{NIST/NRC Series, Cable B, Tests 13 and 16.}
\label{NIST_NRC_B_13_and_16}
\end{figure}

\begin{figure}[h]
\begin{tabular*}{\textwidth}{l@{\extracolsep{\fill}}r}
\includegraphics[width=2.6in]{FIGURES/NIST_NRC_03_v5_B_Cable_Gas_Temp_4-8} &
\includegraphics[width=2.6in]{FIGURES/NIST_NRC_09_v5_B_Cable_Gas_Temp_4-8} \\
\includegraphics[width=2.6in]{FIGURES/NIST_NRC_03_v5_B_Cable_Heat_Flux} &
\includegraphics[width=2.6in]{FIGURES/NIST_NRC_09_v5_B_Cable_Heat_Flux} \\
\includegraphics[width=2.6in]{FIGURES/NIST_NRC_03_v5_B_Cable_TC} &
\includegraphics[width=2.6in]{FIGURES/NIST_NRC_09_v5_B_Cable_TC}
\end{tabular*}
\caption{NIST/NRC Series, Cable B, Replicate Tests 3 and 9.}
\label{NIST_NRC_B_3_and_9}
\end{figure}

\begin{figure}[h]
\begin{tabular*}{\textwidth}{l@{\extracolsep{\fill}}r}
\includegraphics[width=2.6in]{FIGURES/NIST_NRC_05_v5_B_Cable_Gas_Temp_4-8} &
\includegraphics[width=2.6in]{FIGURES/NIST_NRC_14_v5_B_Cable_Gas_Temp_4-8} \\
\includegraphics[width=2.6in]{FIGURES/NIST_NRC_05_v5_B_Cable_Heat_Flux} &
\includegraphics[width=2.6in]{FIGURES/NIST_NRC_14_v5_B_Cable_Heat_Flux} \\
\includegraphics[width=2.6in]{FIGURES/NIST_NRC_05_v5_B_Cable_TC} &
\includegraphics[width=2.6in]{FIGURES/NIST_NRC_14_v5_B_Cable_TC}
\end{tabular*}
\caption{NIST/NRC Series, Cable B, Tests 5 and 14.}
\label{NIST_NRC_B_5_and_14}
\end{figure}

\begin{figure}[h]
\begin{tabular*}{\textwidth}{l@{\extracolsep{\fill}}r}
\includegraphics[width=2.6in]{FIGURES/NIST_NRC_15_v5_B_Cable_Gas_Temp_4-8} &
\includegraphics[width=2.6in]{FIGURES/NIST_NRC_18_v5_B_Cable_Gas_Temp_4-8} \\
\includegraphics[width=2.6in]{FIGURES/NIST_NRC_15_v5_B_Cable_Heat_Flux} &
\includegraphics[width=2.6in]{FIGURES/NIST_NRC_18_v5_B_Cable_Heat_Flux} \\
\includegraphics[width=2.6in]{FIGURES/NIST_NRC_15_v5_B_Cable_TC} &
\includegraphics[width=2.6in]{FIGURES/NIST_NRC_18_v5_B_Cable_TC}
\end{tabular*}
\caption{NIST/NRC Series, Cable B, Tests 15 and 18.}
\label{NIST_NRC_B_15_and_18}
\end{figure}


\clearpage




\subsection{Control Cable D in a Tray}

\vspace{1in}

\begin{figure}[h!]
\begin{tabular*}{\textwidth}{l@{\extracolsep{\fill}}r}
\includegraphics[width=2.6in]{FIGURES/NIST_NRC_01_v5_D_Cable_Gas_Temp_3-9} &
\includegraphics[width=2.6in]{FIGURES/NIST_NRC_07_v5_D_Cable_Gas_Temp_3-9} \\
\includegraphics[width=2.6in]{FIGURES/NIST_NRC_01_v5_D_Cable_Heat_Flux} &
\includegraphics[width=2.6in]{FIGURES/NIST_NRC_07_v5_D_Cable_Heat_Flux} \\
\includegraphics[width=2.6in]{FIGURES/NIST_NRC_01_v5_D_Cable_TC} &
\includegraphics[width=2.6in]{FIGURES/NIST_NRC_07_v5_D_Cable_TC}
\end{tabular*}
\caption{NIST/NRC Series, Cable D, Replicate Tests 1 and 7.}
\label{NIST_NRC_D_1_and_7}
\end{figure}

\begin{figure}[h]
\begin{tabular*}{\textwidth}{l@{\extracolsep{\fill}}r}
\includegraphics[width=2.6in]{FIGURES/NIST_NRC_02_v5_D_Cable_Gas_Temp_3-9} &
\includegraphics[width=2.6in]{FIGURES/NIST_NRC_08_v5_D_Cable_Gas_Temp_3-9} \\
\includegraphics[width=2.6in]{FIGURES/NIST_NRC_02_v5_D_Cable_Heat_Flux} &
\includegraphics[width=2.6in]{FIGURES/NIST_NRC_08_v5_D_Cable_Heat_Flux} \\
\includegraphics[width=2.6in]{FIGURES/NIST_NRC_02_v5_D_Cable_TC} &
\includegraphics[width=2.6in]{FIGURES/NIST_NRC_08_v5_D_Cable_TC}
\end{tabular*}
\caption{NIST/NRC Series, Cable D, Replicate Tests 2 and 8.}
\label{NIST_NRC_D_2_and_8}
\end{figure}

\begin{figure}[h]
\begin{tabular*}{\textwidth}{l@{\extracolsep{\fill}}r}
\includegraphics[width=2.6in]{FIGURES/NIST_NRC_04_v5_D_Cable_Gas_Temp_3-9} &
\includegraphics[width=2.6in]{FIGURES/NIST_NRC_10_v5_D_Cable_Gas_Temp_3-9} \\
\includegraphics[width=2.6in]{FIGURES/NIST_NRC_04_v5_D_Cable_Heat_Flux} &
\includegraphics[width=2.6in]{FIGURES/NIST_NRC_10_v5_D_Cable_Heat_Flux} \\
\includegraphics[width=2.6in]{FIGURES/NIST_NRC_04_v5_D_Cable_TC} &
\includegraphics[width=2.6in]{FIGURES/NIST_NRC_10_v5_D_Cable_TC}
\end{tabular*}
\caption{NIST/NRC Series, Cable D, Replicate Tests 4 and 10.}
\label{NIST_NRC_D_4_and_10}
\end{figure}

\begin{figure}[h]
\begin{tabular*}{\textwidth}{l@{\extracolsep{\fill}}r}
\includegraphics[width=2.6in]{FIGURES/NIST_NRC_13_v5_D_Cable_Gas_Temp_3-9} &
\includegraphics[width=2.6in]{FIGURES/NIST_NRC_16_v5_D_Cable_Gas_Temp_3-9} \\
\includegraphics[width=2.6in]{FIGURES/NIST_NRC_13_v5_D_Cable_Heat_Flux} &
\includegraphics[width=2.6in]{FIGURES/NIST_NRC_16_v5_D_Cable_Heat_Flux} \\
\includegraphics[width=2.6in]{FIGURES/NIST_NRC_13_v5_D_Cable_TC} &
\includegraphics[width=2.6in]{FIGURES/NIST_NRC_16_v5_D_Cable_TC}
\end{tabular*}
\caption{NIST/NRC Series, Cable D, Tests 13 and 16.}
\label{NIST_NRC_D_13_and_16}
\end{figure}

\begin{figure}[h]
\begin{tabular*}{\textwidth}{l@{\extracolsep{\fill}}r}
\includegraphics[width=2.6in]{FIGURES/NIST_NRC_03_v5_D_Cable_Gas_Temp_3-9} &
\includegraphics[width=2.6in]{FIGURES/NIST_NRC_09_v5_D_Cable_Gas_Temp_3-9} \\
\includegraphics[width=2.6in]{FIGURES/NIST_NRC_03_v5_D_Cable_Heat_Flux} &
\includegraphics[width=2.6in]{FIGURES/NIST_NRC_09_v5_D_Cable_Heat_Flux} \\
\includegraphics[width=2.6in]{FIGURES/NIST_NRC_03_v5_D_Cable_TC} &
\includegraphics[width=2.6in]{FIGURES/NIST_NRC_09_v5_D_Cable_TC}
\end{tabular*}
\caption{NIST/NRC Series, Cable D, Replicate Tests 3 and 9.}
\label{NIST_NRC_D_3_and_9}
\end{figure}

\begin{figure}[h]
\begin{tabular*}{\textwidth}{l@{\extracolsep{\fill}}r}
\includegraphics[width=2.6in]{FIGURES/NIST_NRC_05_v5_D_Cable_Gas_Temp_3-9} &
\includegraphics[width=2.6in]{FIGURES/NIST_NRC_14_v5_D_Cable_Gas_Temp_3-9} \\
\includegraphics[width=2.6in]{FIGURES/NIST_NRC_05_v5_D_Cable_Heat_Flux} &
\includegraphics[width=2.6in]{FIGURES/NIST_NRC_14_v5_D_Cable_Heat_Flux} \\
\includegraphics[width=2.6in]{FIGURES/NIST_NRC_05_v5_D_Cable_TC} &
\includegraphics[width=2.6in]{FIGURES/NIST_NRC_14_v5_D_Cable_TC}
\end{tabular*}
\caption{NIST/NRC Series, Cable D, Tests 5 and 14.}
\label{NIST_NRC_D_5_and_14}
\end{figure}

\begin{figure}[h]
\begin{tabular*}{\textwidth}{l@{\extracolsep{\fill}}r}
\includegraphics[width=2.6in]{FIGURES/NIST_NRC_15_v5_D_Cable_Gas_Temp_3-9} &
\includegraphics[width=2.6in]{FIGURES/NIST_NRC_18_v5_D_Cable_Gas_Temp_3-9} \\
\includegraphics[width=2.6in]{FIGURES/NIST_NRC_15_v5_D_Cable_Heat_Flux} &
\includegraphics[width=2.6in]{FIGURES/NIST_NRC_18_v5_D_Cable_Heat_Flux} \\
\includegraphics[width=2.6in]{FIGURES/NIST_NRC_15_v5_D_Cable_TC} &
\includegraphics[width=2.6in]{FIGURES/NIST_NRC_18_v5_D_Cable_TC}
\end{tabular*}
\caption{NIST/NRC Series, Cable D, Tests 15 and 18.}
\label{NIST_NRC_D_15_and_18}
\end{figure}


\clearpage



\subsection{Free-Hanging Power Cable F}

\vspace{1in}

\begin{figure}[h!]
\begin{tabular*}{\textwidth}{l@{\extracolsep{\fill}}r}
\includegraphics[width=2.6in]{FIGURES/NIST_NRC_01_v5_F_Cable_Gas_Temp_5-6} &
\includegraphics[width=2.6in]{FIGURES/NIST_NRC_07_v5_F_Cable_Gas_Temp_5-6} \\
\includegraphics[width=2.6in]{FIGURES/NIST_NRC_01_v5_F_Cable_Heat_Flux} &
\includegraphics[width=2.6in]{FIGURES/NIST_NRC_07_v5_F_Cable_Heat_Flux} \\
\includegraphics[width=2.6in]{FIGURES/NIST_NRC_01_v5_F_Cable_TC} &
\includegraphics[width=2.6in]{FIGURES/NIST_NRC_07_v5_F_Cable_TC}
\end{tabular*}
\caption{NIST/NRC Series, Cable F, Replicate Tests 1 and 7.}
\label{NIST_NRC_F_1_and_7}
\end{figure}

\begin{figure}[h]
\begin{tabular*}{\textwidth}{l@{\extracolsep{\fill}}r}
\includegraphics[width=2.6in]{FIGURES/NIST_NRC_02_v5_F_Cable_Gas_Temp_5-6} &
\includegraphics[width=2.6in]{FIGURES/NIST_NRC_08_v5_F_Cable_Gas_Temp_5-6} \\
\includegraphics[width=2.6in]{FIGURES/NIST_NRC_02_v5_F_Cable_Heat_Flux} &
\includegraphics[width=2.6in]{FIGURES/NIST_NRC_08_v5_F_Cable_Heat_Flux} \\
\includegraphics[width=2.6in]{FIGURES/NIST_NRC_02_v5_F_Cable_TC} &
\includegraphics[width=2.6in]{FIGURES/NIST_NRC_08_v5_F_Cable_TC}
\end{tabular*}
\caption{NIST/NRC Series, Cable F, Replicate Tests 2 and 8.}
\label{NIST_NRC_F_2_and_8}
\end{figure}

\begin{figure}[h]
\begin{tabular*}{\textwidth}{l@{\extracolsep{\fill}}r}
\includegraphics[width=2.6in]{FIGURES/NIST_NRC_04_v5_F_Cable_Gas_Temp_5-6} &
\includegraphics[width=2.6in]{FIGURES/NIST_NRC_10_v5_F_Cable_Gas_Temp_5-6} \\
\includegraphics[width=2.6in]{FIGURES/NIST_NRC_04_v5_F_Cable_Heat_Flux} &
\includegraphics[width=2.6in]{FIGURES/NIST_NRC_10_v5_F_Cable_Heat_Flux} \\
\includegraphics[width=2.6in]{FIGURES/NIST_NRC_04_v5_F_Cable_TC} &
\includegraphics[width=2.6in]{FIGURES/NIST_NRC_10_v5_F_Cable_TC}
\end{tabular*}
\caption{NIST/NRC Series, Cable F, Replicate Tests 4 and 10.}
\label{NIST_NRC_F_4_and_10}
\end{figure}

\begin{figure}[h]
\begin{tabular*}{\textwidth}{l@{\extracolsep{\fill}}r}
\includegraphics[width=2.6in]{FIGURES/NIST_NRC_13_v5_F_Cable_Gas_Temp_5-6} &
\includegraphics[width=2.6in]{FIGURES/NIST_NRC_16_v5_F_Cable_Gas_Temp_5-6} \\
\includegraphics[width=2.6in]{FIGURES/NIST_NRC_13_v5_F_Cable_Heat_Flux} &
\includegraphics[width=2.6in]{FIGURES/NIST_NRC_16_v5_F_Cable_Heat_Flux} \\
\includegraphics[width=2.6in]{FIGURES/NIST_NRC_13_v5_F_Cable_TC} &
\includegraphics[width=2.6in]{FIGURES/NIST_NRC_16_v5_F_Cable_TC}
\end{tabular*}
\caption{NIST/NRC Series, Cable F, Tests 13 and 16.}
\label{NIST_NRC_F_13_and_16}
\end{figure}

\begin{figure}[h]
\begin{tabular*}{\textwidth}{l@{\extracolsep{\fill}}r}
\includegraphics[width=2.6in]{FIGURES/NIST_NRC_03_v5_F_Cable_Gas_Temp_5-6} &
\includegraphics[width=2.6in]{FIGURES/NIST_NRC_09_v5_F_Cable_Gas_Temp_5-6} \\
\includegraphics[width=2.6in]{FIGURES/NIST_NRC_03_v5_F_Cable_Heat_Flux} &
\includegraphics[width=2.6in]{FIGURES/NIST_NRC_09_v5_F_Cable_Heat_Flux} \\
\includegraphics[width=2.6in]{FIGURES/NIST_NRC_03_v5_F_Cable_TC} &
\includegraphics[width=2.6in]{FIGURES/NIST_NRC_09_v5_F_Cable_TC}
\end{tabular*}
\caption{NIST/NRC Series, Cable F, Replicate Tests 3 and 9.}
\label{NIST_NRC_F_3_and_9}
\end{figure}

\begin{figure}[h]
\begin{tabular*}{\textwidth}{l@{\extracolsep{\fill}}r}
\includegraphics[width=2.6in]{FIGURES/NIST_NRC_05_v5_F_Cable_Gas_Temp_5-6} &
\includegraphics[width=2.6in]{FIGURES/NIST_NRC_14_v5_F_Cable_Gas_Temp_5-6} \\
\includegraphics[width=2.6in]{FIGURES/NIST_NRC_05_v5_F_Cable_Heat_Flux} &
\includegraphics[width=2.6in]{FIGURES/NIST_NRC_14_v5_F_Cable_Heat_Flux} \\
\includegraphics[width=2.6in]{FIGURES/NIST_NRC_05_v5_F_Cable_TC} &
\includegraphics[width=2.6in]{FIGURES/NIST_NRC_14_v5_F_Cable_TC}
\end{tabular*}
\caption{NIST/NRC Series, Cable F, Tests 5 and 14.}
\label{NIST_NRC_F_5_and_14}
\end{figure}

\begin{figure}[h]
\begin{tabular*}{\textwidth}{l@{\extracolsep{\fill}}r}
\includegraphics[width=2.6in]{FIGURES/NIST_NRC_15_v5_F_Cable_Gas_Temp_5-6} &
\includegraphics[width=2.6in]{FIGURES/NIST_NRC_18_v5_F_Cable_Gas_Temp_5-6} \\
\includegraphics[width=2.6in]{FIGURES/NIST_NRC_15_v5_F_Cable_Heat_Flux} &
\includegraphics[width=2.6in]{FIGURES/NIST_NRC_18_v5_F_Cable_Heat_Flux} \\
\includegraphics[width=2.6in]{FIGURES/NIST_NRC_15_v5_F_Cable_TC} &
\includegraphics[width=2.6in]{FIGURES/NIST_NRC_18_v5_F_Cable_TC}
\end{tabular*}
\caption{NIST/NRC Series, Cable F, Tests 15 and 18.}
\label{NIST_NRC_F_15_and_18}
\end{figure}


\clearpage




\subsection{Vertical Cable Tray G}

\vspace{1in}

\begin{figure}[h!]
\begin{tabular*}{\textwidth}{l@{\extracolsep{\fill}}r}
\includegraphics[width=2.6in]{FIGURES/NIST_NRC_01_v5_G_Cable_Gas_Temp_2-5} &
\includegraphics[width=2.6in]{FIGURES/NIST_NRC_07_v5_G_Cable_Gas_Temp_2-5} \\
\includegraphics[width=2.6in]{FIGURES/NIST_NRC_01_v5_G_Cable_Heat_Flux} &
\includegraphics[width=2.6in]{FIGURES/NIST_NRC_07_v5_G_Cable_Heat_Flux} \\
\includegraphics[width=2.6in]{FIGURES/NIST_NRC_01_v5_G_Cable_TC} &
\includegraphics[width=2.6in]{FIGURES/NIST_NRC_07_v5_G_Cable_TC}
\end{tabular*}
\caption{NIST/NRC Series, Cable G, Replicate Tests 1 and 7.}
\label{NIST_NRC_G_1_and_7}
\end{figure}

\begin{figure}[h]
\begin{tabular*}{\textwidth}{l@{\extracolsep{\fill}}r}
\includegraphics[width=2.6in]{FIGURES/NIST_NRC_02_v5_G_Cable_Gas_Temp_2-5} &
\includegraphics[width=2.6in]{FIGURES/NIST_NRC_08_v5_G_Cable_Gas_Temp_2-5} \\
\includegraphics[width=2.6in]{FIGURES/NIST_NRC_02_v5_G_Cable_Heat_Flux} &
\includegraphics[width=2.6in]{FIGURES/NIST_NRC_08_v5_G_Cable_Heat_Flux} \\
\includegraphics[width=2.6in]{FIGURES/NIST_NRC_02_v5_G_Cable_TC} &
\includegraphics[width=2.6in]{FIGURES/NIST_NRC_08_v5_G_Cable_TC}
\end{tabular*}
\caption{NIST/NRC Series, Cable G, Replicate Tests 2 and 8.}
\label{NIST_NRC_G_2_and_8}
\end{figure}

\begin{figure}[h]
\begin{tabular*}{\textwidth}{l@{\extracolsep{\fill}}r}
\includegraphics[width=2.6in]{FIGURES/NIST_NRC_04_v5_G_Cable_Gas_Temp_2-5} &
\includegraphics[width=2.6in]{FIGURES/NIST_NRC_10_v5_G_Cable_Gas_Temp_2-5} \\
\includegraphics[width=2.6in]{FIGURES/NIST_NRC_04_v5_G_Cable_Heat_Flux} &
\includegraphics[width=2.6in]{FIGURES/NIST_NRC_10_v5_G_Cable_Heat_Flux} \\
\includegraphics[width=2.6in]{FIGURES/NIST_NRC_04_v5_G_Cable_TC} &
\includegraphics[width=2.6in]{FIGURES/NIST_NRC_10_v5_G_Cable_TC}
\end{tabular*}
\caption{NIST/NRC Series, Cable G, Replicate Tests 4 and 10.}
\label{NIST_NRC_G_4_and_10}
\end{figure}

\begin{figure}[h]
\begin{tabular*}{\textwidth}{l@{\extracolsep{\fill}}r}
\includegraphics[width=2.6in]{FIGURES/NIST_NRC_13_v5_G_Cable_Gas_Temp_2-5} &
\includegraphics[width=2.6in]{FIGURES/NIST_NRC_16_v5_G_Cable_Gas_Temp_2-5} \\
\includegraphics[width=2.6in]{FIGURES/NIST_NRC_13_v5_G_Cable_Heat_Flux} &
\includegraphics[width=2.6in]{FIGURES/NIST_NRC_16_v5_G_Cable_Heat_Flux} \\
\includegraphics[width=2.6in]{FIGURES/NIST_NRC_13_v5_G_Cable_TC} &
\includegraphics[width=2.6in]{FIGURES/NIST_NRC_16_v5_G_Cable_TC}
\end{tabular*}
\caption{NIST/NRC Series, Cable G, Tests 13 and 16.}
\label{NIST_NRC_G_13_and_16}
\end{figure}

\begin{figure}[h]
\begin{tabular*}{\textwidth}{l@{\extracolsep{\fill}}r}
\includegraphics[width=2.6in]{FIGURES/NIST_NRC_03_v5_G_Cable_Gas_Temp_2-5} &
\includegraphics[width=2.6in]{FIGURES/NIST_NRC_09_v5_G_Cable_Gas_Temp_2-5} \\
\includegraphics[width=2.6in]{FIGURES/NIST_NRC_03_v5_G_Cable_Heat_Flux} &
\includegraphics[width=2.6in]{FIGURES/NIST_NRC_09_v5_G_Cable_Heat_Flux} \\
\includegraphics[width=2.6in]{FIGURES/NIST_NRC_03_v5_G_Cable_TC} &
\includegraphics[width=2.6in]{FIGURES/NIST_NRC_09_v5_G_Cable_TC}
\end{tabular*}
\caption{NIST/NRC Series, Cable G, Replicate Tests 3 and 9.}
\label{NIST_NRC_G_3_and_9}
\end{figure}

\begin{figure}[h]
\begin{tabular*}{\textwidth}{l@{\extracolsep{\fill}}r}
\includegraphics[width=2.6in]{FIGURES/NIST_NRC_05_v5_G_Cable_Gas_Temp_2-5} &
\includegraphics[width=2.6in]{FIGURES/NIST_NRC_14_v5_G_Cable_Gas_Temp_2-5} \\
\includegraphics[width=2.6in]{FIGURES/NIST_NRC_05_v5_G_Cable_Heat_Flux} &
\includegraphics[width=2.6in]{FIGURES/NIST_NRC_14_v5_G_Cable_Heat_Flux} \\
\includegraphics[width=2.6in]{FIGURES/NIST_NRC_05_v5_G_Cable_TC} &
\includegraphics[width=2.6in]{FIGURES/NIST_NRC_14_v5_G_Cable_TC}
\end{tabular*}
\caption{NIST/NRC Series, Cable G, Tests 5 and 14.}
\label{NIST_NRC_G_5_and_14}
\end{figure}

\begin{figure}[h]
\begin{tabular*}{\textwidth}{l@{\extracolsep{\fill}}r}
\includegraphics[width=2.6in]{FIGURES/NIST_NRC_15_v5_G_Cable_Gas_Temp_2-5} &
\includegraphics[width=2.6in]{FIGURES/NIST_NRC_18_v5_G_Cable_Gas_Temp_2-5} \\
\includegraphics[width=2.6in]{FIGURES/NIST_NRC_15_v5_G_Cable_Heat_Flux} &
\includegraphics[width=2.6in]{FIGURES/NIST_NRC_18_v5_G_Cable_Heat_Flux} \\
\includegraphics[width=2.6in]{FIGURES/NIST_NRC_15_v5_G_Cable_TC} &
\includegraphics[width=2.6in]{FIGURES/NIST_NRC_18_v5_G_Cable_TC}
\end{tabular*}
\caption{NIST/NRC Series, Cable G, Tests 15 and 18.}
\label{NIST_NRC_G_15_and_18}
\end{figure}


\clearpage


\section{NIST/WTC Test Series, Heat Fluxes}

\begin{figure}[h]
\begin{tabular*}{\textwidth}{l@{\extracolsep{\fill}}r}
\includegraphics[width=2.6in]{FIGURES/WTC_01_v5_Floor_Heat_Flux} &
\includegraphics[width=2.6in]{FIGURES/WTC_02_v5_Floor_Heat_Flux} \\
\includegraphics[width=2.6in]{FIGURES/WTC_03_v5_Floor_Heat_Flux} &
\includegraphics[width=2.6in]{FIGURES/WTC_04_v5_Floor_Heat_Flux} \\
\includegraphics[width=2.6in]{FIGURES/WTC_05_v5_Floor_Heat_Flux} &
\includegraphics[width=2.6in]{FIGURES/WTC_06_v5_Floor_Heat_Flux}
\end{tabular*}
\caption{NIST/WTC Test Series, Heat Fluxes to the Floor.}
\label{NIST_WTC_Floor_Heat_Flux}
\end{figure}

\begin{figure}[h]
\begin{tabular*}{\textwidth}{l@{\extracolsep{\fill}}r}
\includegraphics[width=2.6in]{FIGURES/WTC_01_v5_High_Column_Heat_Flux} &
\includegraphics[width=2.6in]{FIGURES/WTC_02_v5_High_Column_Heat_Flux} \\
\includegraphics[width=2.6in]{FIGURES/WTC_03_v5_High_Column_Heat_Flux} &
\includegraphics[width=2.6in]{FIGURES/WTC_04_v5_High_Column_Heat_Flux} \\
\includegraphics[width=2.6in]{FIGURES/WTC_05_v5_High_Column_Heat_Flux} &
\includegraphics[width=2.6in]{FIGURES/WTC_06_v5_High_Column_Heat_Flux}
\end{tabular*}
\caption{NIST/WTC Test Series, Heat Fluxes to the Lower Column.}
\label{NIST_WTC_Low_Column_Heat_Flux}
\end{figure}

\begin{figure}[h]
\begin{tabular*}{\textwidth}{l@{\extracolsep{\fill}}r}
\includegraphics[width=2.6in]{FIGURES/WTC_01_v5_Low_Column_Heat_Flux} &
\includegraphics[width=2.6in]{FIGURES/WTC_02_v5_Low_Column_Heat_Flux} \\
\includegraphics[width=2.6in]{FIGURES/WTC_03_v5_Low_Column_Heat_Flux} &
\includegraphics[width=2.6in]{FIGURES/WTC_04_v5_Low_Column_Heat_Flux} \\
\includegraphics[width=2.6in]{FIGURES/WTC_05_v5_Low_Column_Heat_Flux} &
\includegraphics[width=2.6in]{FIGURES/WTC_06_v5_Low_Column_Heat_Flux}
\end{tabular*}
\caption{NIST/WTC Test Series, Heat Fluxes to the Upper Column.}
\label{NIST_WTC_High_Column_Heat_Flux}
\end{figure}

\begin{figure}[h]
\begin{tabular*}{\textwidth}{l@{\extracolsep{\fill}}r}
\includegraphics[width=2.6in]{FIGURES/WTC_01_v5_Flux_Station_2_Heat_Flux} &
\includegraphics[width=2.6in]{FIGURES/WTC_02_v5_Flux_Station_2_Heat_Flux} \\
\includegraphics[width=2.6in]{FIGURES/WTC_03_v5_Flux_Station_2_Heat_Flux} &
\includegraphics[width=2.6in]{FIGURES/WTC_04_v5_Flux_Station_2_Heat_Flux} \\
\includegraphics[width=2.6in]{FIGURES/WTC_05_v5_Flux_Station_2_Heat_Flux} &
\includegraphics[width=2.6in]{FIGURES/WTC_06_v5_Flux_Station_2_Heat_Flux}
\end{tabular*}
\caption{NIST/WTC Test Series, Heat Fluxes to Flux Station 2.}
\label{NIST_WTC_Flux_Station_2_Heat_Flux}
\end{figure}

\begin{figure}[h]
\begin{tabular*}{\textwidth}{l@{\extracolsep{\fill}}r}
\includegraphics[width=2.6in]{FIGURES/WTC_01_v5_Heat_Flux_to_Ceiling} &
\includegraphics[width=2.6in]{FIGURES/WTC_02_v5_Heat_Flux_to_Ceiling} \\
\includegraphics[width=2.6in]{FIGURES/WTC_03_v5_Heat_Flux_to_Ceiling} &
\includegraphics[width=2.6in]{FIGURES/WTC_04_v5_Heat_Flux_to_Ceiling} \\
\includegraphics[width=2.6in]{FIGURES/WTC_05_v5_Heat_Flux_to_Ceiling} &
\includegraphics[width=2.6in]{FIGURES/WTC_06_v5_Heat_Flux_to_Ceiling}
\end{tabular*}
\caption{NIST/WTC Test Series, Heat Fluxes to Ceiling.}
\label{NIST_WTC_Flux_Heat_Flux_to_Ceiling}
\end{figure}


\clearpage


\section{NIST/WTC Test Series, Ceiling and Wall Temperatures}


\begin{figure}[h!]
\begin{tabular*}{\textwidth}{l@{\extracolsep{\fill}}r}
\includegraphics[width=2.6in]{FIGURES/WTC_01_v5_North_Ceiling_Temperature} &
\includegraphics[width=2.6in]{FIGURES/WTC_02_v5_North_Ceiling_Temperature} \\
\includegraphics[width=2.6in]{FIGURES/WTC_03_v5_North_Ceiling_Temperature} &
\includegraphics[width=2.6in]{FIGURES/WTC_04_v5_North_Ceiling_Temperature} \\
\includegraphics[width=2.6in]{FIGURES/WTC_05_v5_North_Ceiling_Temperature} &
\includegraphics[width=2.6in]{FIGURES/WTC_06_v5_North_Ceiling_Temperature}
\end{tabular*}
\caption{North Ceiling Temperatures for the NIST/WTC Test Series.}
\label{NIST_WTC North_Ceiling_Temp}
\end{figure}

\begin{figure}[p]
\begin{tabular*}{\textwidth}{l@{\extracolsep{\fill}}r}
\includegraphics[width=2.6in]{FIGURES/WTC_01_v5_East_Ceiling_Temperature} &
\includegraphics[width=2.6in]{FIGURES/WTC_02_v5_East_Ceiling_Temperature} \\
\includegraphics[width=2.6in]{FIGURES/WTC_03_v5_East_Ceiling_Temperature} &
\includegraphics[width=2.6in]{FIGURES/WTC_04_v5_East_Ceiling_Temperature} \\
\includegraphics[width=2.6in]{FIGURES/WTC_05_v5_East_Ceiling_Temperature} &
\includegraphics[width=2.6in]{FIGURES/WTC_06_v5_East_Ceiling_Temperature}
\end{tabular*}
\caption{East Ceiling Temperatures for the NIST/WTC Test Series.}
\label{NIST_WTC East_Ceiling_Temp}
\end{figure}

\begin{figure}[p]
\begin{tabular*}{\textwidth}{l@{\extracolsep{\fill}}r}
\includegraphics[width=2.6in]{FIGURES/WTC_01_v5_West_Ceiling_Temperature} &
\includegraphics[width=2.6in]{FIGURES/WTC_02_v5_West_Ceiling_Temperature} \\
\includegraphics[width=2.6in]{FIGURES/WTC_03_v5_West_Ceiling_Temperature} &
\includegraphics[width=2.6in]{FIGURES/WTC_04_v5_West_Ceiling_Temperature} \\
\includegraphics[width=2.6in]{FIGURES/WTC_05_v5_West_Ceiling_Temperature} &
\includegraphics[width=2.6in]{FIGURES/WTC_06_v5_West_Ceiling_Temperature}
\end{tabular*}
\caption{West Ceiling Temperatures for the NIST/WTC Test Series.}
\label{NIST_WTC West_Ceiling_Temp}
\end{figure}

\begin{figure}[p]
\begin{tabular*}{\textwidth}{l@{\extracolsep{\fill}}r}
\includegraphics[width=2.6in]{FIGURES/WTC_01_v5_Inner_Ceiling_Temperature} &
\includegraphics[width=2.6in]{FIGURES/WTC_02_v5_Inner_Ceiling_Temperature} \\
\includegraphics[width=2.6in]{FIGURES/WTC_03_v5_Inner_Ceiling_Temperature} &
\includegraphics[width=2.6in]{FIGURES/WTC_04_v5_Inner_Ceiling_Temperature} \\
\includegraphics[width=2.6in]{FIGURES/WTC_05_v5_Inner_Ceiling_Temperature} &
\includegraphics[width=2.6in]{FIGURES/WTC_06_v5_Inner_Ceiling_Temperature}
\end{tabular*}
\caption{Inner Ceiling Temperatures for the NIST/WTC Test Series.}
\label{NIST_WTC Inner_Ceiling_Temp}
\end{figure}

\begin{figure}[p]
\begin{tabular*}{\textwidth}{l@{\extracolsep{\fill}}r}
\includegraphics[width=2.6in]{FIGURES/WTC_01_v5_Inner_Ceiling_Temperature_2} &
\includegraphics[width=2.6in]{FIGURES/WTC_02_v5_Inner_Ceiling_Temperature_2} \\
\includegraphics[width=2.6in]{FIGURES/WTC_03_v5_Inner_Ceiling_Temperature_2} &
\includegraphics[width=2.6in]{FIGURES/WTC_04_v5_Inner_Ceiling_Temperature_2} \\
\includegraphics[width=2.6in]{FIGURES/WTC_05_v5_Inner_Ceiling_Temperature_2} &
\includegraphics[width=2.6in]{FIGURES/WTC_06_v5_Inner_Ceiling_Temperature_2}
\end{tabular*}
\caption{Inner Ceiling Temperatures for the NIST/WTC Test Series.}
\label{NIST_WTC Inner_Ceiling_Temp_2}
\end{figure}

\begin{figure}[p]
\begin{tabular*}{\textwidth}{l@{\extracolsep{\fill}}r}
\includegraphics[width=2.6in]{FIGURES/WTC_01_v5_North_Wall_Temperature} &
\includegraphics[width=2.6in]{FIGURES/WTC_02_v5_North_Wall_Temperature} \\
\includegraphics[width=2.6in]{FIGURES/WTC_03_v5_North_Wall_Temperature} &
\includegraphics[width=2.6in]{FIGURES/WTC_04_v5_North_Wall_Temperature} \\
\includegraphics[width=2.6in]{FIGURES/WTC_05_v5_North_wall_Temperature} &
\includegraphics[width=2.6in]{FIGURES/WTC_06_v5_North_Wall_Temperature}
\end{tabular*}
\caption{North Wall Temperatures for the NIST/WTC Test Series.}
\label{NIST_WTC North_Wall_Temp}
\end{figure}

\begin{figure}[p]
\begin{tabular*}{\textwidth}{l@{\extracolsep{\fill}}r}
\includegraphics[width=2.6in]{FIGURES/WTC_01_v5_North_Wall_Temperature_2} &
\includegraphics[width=2.6in]{FIGURES/WTC_02_v5_North_Wall_Temperature_2} \\
\includegraphics[width=2.6in]{FIGURES/WTC_03_v5_North_Wall_Temperature_2} &
\includegraphics[width=2.6in]{FIGURES/WTC_04_v5_North_Wall_Temperature_2} \\
\includegraphics[width=2.6in]{FIGURES/WTC_05_v5_North_wall_Temperature_2} &
\includegraphics[width=2.6in]{FIGURES/WTC_06_v5_North_Wall_Temperature_2}
\end{tabular*}
\caption{North Wall Temperatures for the NIST/WTC Test Series.}
\label{NIST_WTC North_Wall_Temp_2}
\end{figure}

\begin{figure}[p]
\begin{tabular*}{\textwidth}{l@{\extracolsep{\fill}}r}
\includegraphics[width=2.6in]{FIGURES/WTC_01_v5_Inner_North_Wall_Temperature} &
\includegraphics[width=2.6in]{FIGURES/WTC_02_v5_Inner_North_Wall_Temperature} \\
\includegraphics[width=2.6in]{FIGURES/WTC_03_v5_Inner_North_Wall_Temperature} &
\includegraphics[width=2.6in]{FIGURES/WTC_04_v5_Inner_North_Wall_Temperature} \\
\includegraphics[width=2.6in]{FIGURES/WTC_05_v5_Inner_North_wall_Temperature} &
\includegraphics[width=2.6in]{FIGURES/WTC_06_v5_Inner_North_Wall_Temperature}
\end{tabular*}
\caption{Inner North Wall Temperatures for the NIST/WTC Test Series.}
\label{NIST_WTC Inner North_Wall_Temp}
\end{figure}

\clearpage




\section{NIST/NRC Test Series, Compartment Walls, Floor and Ceiling}

Thirty-six heat flux gauges were positioned at various locations on all four walls of the compartment,
plus the ceiling and floor.  Comparisons between measured and predicted heat fluxes and surface temperatures are shown
on the following pages for a selected number of locations.
Over half of the measurement points are in roughly the same relative location to the fire and hence
the measurements and predictions are similar.  For this reason, data for the east and north walls are shown
because the data from the south and west walls are comparable.  Data from the south wall is used in cases where
the corresponding instrument on the north wall failed, or in cases where the fire is positioned close to the south wall.
For each test, eight locations are used for comparison, two on the long (mainly north) wall,
two on the short (east) wall, two on the floor, and two on the ceiling.  Of the two locations for each panel,
one is considered in the far-field, relatively remote from the fire; one is in the near-field,
relatively close to the fire.  How close or far varies from test to test, depending on the availability of working flux gauges.
The two short wall locations are equally remote from the fire; thus, one location is in the lower layer, one in the upper.
Table lists the locations for each test.
The heat flux gauges used on the compartment walls measured the net, not total, heat flux.
FDS predicts the net heat flux, but this prediction cannot be compared directly with the measured net heat
flux because the predicted and measured wall temperatures can differ, and this affects the net heat flux.
In a sense, the net heat flux and surface temperature are coupled, and it is difficult to assess the accuracy of the models
if the two quantities cannot be decoupled.  For the purpose of comparing prediction and measurement,
the following correction has been applied to both the measured and predicted net heat fluxes:
\be  \dq_{\hbox{\tiny total}}'' = \dq_{\hbox{\tiny net}}'' + \sigma (T_s^4-T_\infty^4) + h (T_s - T_\infty) \ee
where $T_s$ is the temperature of the surface.  A constant convective heat transfer coefficient is assumed
(5 W/m$^2$/K) and an emissivity of 1.
After applying the correction, it is easier to compare total heat fluxes that are independent of the surface temperature.

\clearpage

\subsection{Long Wall}

\vspace{2in}


\begin{figure}[h!]
\begin{tabular*}{\textwidth}{l@{\extracolsep{\fill}}r}
\includegraphics[width=2.6in]{FIGURES/NIST_NRC_01_v5_Long_Wall_Flux_Gauges} &
\includegraphics[width=2.6in]{FIGURES/NIST_NRC_01_v5_Long_Wall_TC} \\
\includegraphics[width=2.6in]{FIGURES/NIST_NRC_07_v5_Long_Wall_Flux_Gauges} &
\includegraphics[width=2.6in]{FIGURES/NIST_NRC_07_v5_Long_Wall_TC}

\end{tabular*}
\caption{NIST/NRC Series, Long Wall Heat Flux and Temperature, Tests 1 and 7.}
\label{NIST_NRC_Long_1}
\end{figure}

\begin{figure}[p]
\begin{tabular*}{\textwidth}{l@{\extracolsep{\fill}}r}
\includegraphics[width=2.6in]{FIGURES/NIST_NRC_02_v5_Long_Wall_Flux_Gauges} &
\includegraphics[width=2.6in]{FIGURES/NIST_NRC_02_v5_Long_Wall_TC} \\
\includegraphics[width=2.6in]{FIGURES/NIST_NRC_08_v5_Long_Wall_Flux_Gauges} &
\includegraphics[width=2.6in]{FIGURES/NIST_NRC_08_v5_Long_Wall_TC} \\
\includegraphics[width=2.6in]{FIGURES/NIST_NRC_04_v5_Long_Wall_Flux_Gauges} &
\includegraphics[width=2.6in]{FIGURES/NIST_NRC_04_v5_Long_Wall_TC} \\
\includegraphics[width=2.6in]{FIGURES/NIST_NRC_10_v5_Long_Wall_Flux_Gauges} &
\includegraphics[width=2.6in]{FIGURES/NIST_NRC_10_v5_Long_Wall_TC}

\end{tabular*}
\caption{NIST/NRC Series, Long Wall Heat Flux and Temperature, Tests 2, 8, 4 and 10.}
\label{NIST_NRC_Long_2}
\end{figure}

\begin{figure}[p]
\begin{tabular*}{\textwidth}{l@{\extracolsep{\fill}}r}
\includegraphics[width=2.6in]{FIGURES/NIST_NRC_13_v5_Long_Wall_Flux_Gauges} &
\includegraphics[width=2.6in]{FIGURES/NIST_NRC_13_v5_Long_Wall_TC} \\
\includegraphics[width=2.6in]{FIGURES/NIST_NRC_16_v5_Long_Wall_Flux_Gauges} &
\includegraphics[width=2.6in]{FIGURES/NIST_NRC_16_v5_Long_Wall_TC} \\
\includegraphics[width=2.6in]{FIGURES/NIST_NRC_03_v5_Long_Wall_Flux_Gauges} &
\includegraphics[width=2.6in]{FIGURES/NIST_NRC_03_v5_Long_Wall_TC} \\
\includegraphics[width=2.6in]{FIGURES/NIST_NRC_09_v5_Long_Wall_Flux_Gauges} &
\includegraphics[width=2.6in]{FIGURES/NIST_NRC_09_v5_Long_Wall_TC}

\end{tabular*}
\caption{NIST/NRC Series, Long Wall Heat Flux and Temperature, Test 13, 16, 3 and 9.}
\label{NIST_NRC_Long_3}
\end{figure}

\begin{figure}[p]
\begin{tabular*}{\textwidth}{l@{\extracolsep{\fill}}r}
\includegraphics[width=2.6in]{FIGURES/NIST_NRC_05_v5_Long_Wall_Flux_Gauges} &
\includegraphics[width=2.6in]{FIGURES/NIST_NRC_05_v5_Long_Wall_TC} \\
\includegraphics[width=2.6in]{FIGURES/NIST_NRC_14_v5_Long_Wall_Flux_Gauges} &
\includegraphics[width=2.6in]{FIGURES/NIST_NRC_14_v5_Long_Wall_TC} \\
\includegraphics[width=2.6in]{FIGURES/NIST_NRC_15_v5_Long_Wall_Flux_Gauges} &
\includegraphics[width=2.6in]{FIGURES/NIST_NRC_15_v5_Long_Wall_TC} \\
\includegraphics[width=2.6in]{FIGURES/NIST_NRC_18_v5_Long_Wall_Flux_Gauges} &
\includegraphics[width=2.6in]{FIGURES/NIST_NRC_18_v5_Long_Wall_TC}
\end{tabular*}
\caption{NIST/NRC Series, Long Wall Heat Flux and Temperature, Tests 5, 14, 15 and 18.}
\label{NIST_NRC_Long_4}
\end{figure}

\clearpage



\subsection{Short Wall}

\vspace{2in}


\begin{figure}[h!]
\begin{tabular*}{\textwidth}{l@{\extracolsep{\fill}}r}
\includegraphics[width=2.6in]{FIGURES/NIST_NRC_01_v5_Short_Wall_Flux_Gauges} &
\includegraphics[width=2.6in]{FIGURES/NIST_NRC_01_v5_Short_Wall_TC} \\
\includegraphics[width=2.6in]{FIGURES/NIST_NRC_07_v5_Short_Wall_Flux_Gauges} &
\includegraphics[width=2.6in]{FIGURES/NIST_NRC_07_v5_Short_Wall_TC}

\end{tabular*}
\caption{NIST/NRC Series, Short Wall Heat Flux and Temperature, Tests 1 and 7.}
\label{NIST_NRC_Short_1}
\end{figure}

\begin{figure}[p]
\begin{tabular*}{\textwidth}{l@{\extracolsep{\fill}}r}
\includegraphics[width=2.6in]{FIGURES/NIST_NRC_02_v5_Short_Wall_Flux_Gauges} &
\includegraphics[width=2.6in]{FIGURES/NIST_NRC_02_v5_Short_Wall_TC} \\
\includegraphics[width=2.6in]{FIGURES/NIST_NRC_08_v5_Short_Wall_Flux_Gauges} &
\includegraphics[width=2.6in]{FIGURES/NIST_NRC_08_v5_Short_Wall_TC} \\
\includegraphics[width=2.6in]{FIGURES/NIST_NRC_04_v5_Short_Wall_Flux_Gauges} &
\includegraphics[width=2.6in]{FIGURES/NIST_NRC_04_v5_Short_Wall_TC} \\
\includegraphics[width=2.6in]{FIGURES/NIST_NRC_10_v5_Short_Wall_Flux_Gauges} &
\includegraphics[width=2.6in]{FIGURES/NIST_NRC_10_v5_Short_Wall_TC}

\end{tabular*}
\caption{NIST/NRC Series, Short Wall Heat Flux and Temperature, Tests 2, 8, 4 and 10.}
\label{NIST_NRC_Short_2}
\end{figure}

\begin{figure}[p]
\begin{tabular*}{\textwidth}{l@{\extracolsep{\fill}}r}
\includegraphics[width=2.6in]{FIGURES/NIST_NRC_13_v5_Short_Wall_Flux_Gauges} &
\includegraphics[width=2.6in]{FIGURES/NIST_NRC_13_v5_Short_Wall_TC} \\
\includegraphics[width=2.6in]{FIGURES/NIST_NRC_16_v5_Short_Wall_Flux_Gauges} &
\includegraphics[width=2.6in]{FIGURES/NIST_NRC_16_v5_Short_Wall_TC} \\
\includegraphics[width=2.6in]{FIGURES/NIST_NRC_03_v5_Short_Wall_Flux_Gauges} &
\includegraphics[width=2.6in]{FIGURES/NIST_NRC_03_v5_Short_Wall_TC} \\
\includegraphics[width=2.6in]{FIGURES/NIST_NRC_09_v5_Short_Wall_Flux_Gauges} &
\includegraphics[width=2.6in]{FIGURES/NIST_NRC_09_v5_Short_Wall_TC}

\end{tabular*}
\caption{NIST/NRC Series, Short Wall Heat Flux and Temperature, Test 13, 16, 3 and 9.}
\label{NIST_NRC_Short_3}
\end{figure}

\begin{figure}[p]
\begin{tabular*}{\textwidth}{l@{\extracolsep{\fill}}r}
\includegraphics[width=2.6in]{FIGURES/NIST_NRC_05_v5_Short_Wall_Flux_Gauges} &
\includegraphics[width=2.6in]{FIGURES/NIST_NRC_05_v5_Short_Wall_TC} \\
\includegraphics[width=2.6in]{FIGURES/NIST_NRC_14_v5_Short_Wall_Flux_Gauges} &
\includegraphics[width=2.6in]{FIGURES/NIST_NRC_14_v5_Short_Wall_TC} \\
\includegraphics[width=2.6in]{FIGURES/NIST_NRC_15_v5_Short_Wall_Flux_Gauges} &
\includegraphics[width=2.6in]{FIGURES/NIST_NRC_15_v5_Short_Wall_TC} \\
\includegraphics[width=2.6in]{FIGURES/NIST_NRC_18_v5_Short_Wall_Flux_Gauges} &
\includegraphics[width=2.6in]{FIGURES/NIST_NRC_18_v5_Short_Wall_TC}
\end{tabular*}
\caption{NIST/NRC Series, Short Wall Heat Flux and Temperature, Tests 5, 14, 15 and 18.}
\label{NIST_NRC_Short_4}
\end{figure}

\clearpage



\subsection{Ceiling}

\vspace{2in}


\begin{figure}[h!]
\begin{tabular*}{\textwidth}{l@{\extracolsep{\fill}}r}
\includegraphics[width=2.6in]{FIGURES/NIST_NRC_01_v5_Ceiling_Flux_Gauges} &
\includegraphics[width=2.6in]{FIGURES/NIST_NRC_01_v5_Ceiling_TC} \\
\includegraphics[width=2.6in]{FIGURES/NIST_NRC_07_v5_Ceiling_Flux_Gauges} &
\includegraphics[width=2.6in]{FIGURES/NIST_NRC_07_v5_Ceiling_TC}

\end{tabular*}
\caption{NIST/NRC Series, Ceiling Heat Flux and Temperature, Tests 1 and 7.}
\label{NIST_NRC_Ceiling_1}
\end{figure}

\begin{figure}[p]
\begin{tabular*}{\textwidth}{l@{\extracolsep{\fill}}r}
\includegraphics[width=2.6in]{FIGURES/NIST_NRC_02_v5_Ceiling_Flux_Gauges} &
\includegraphics[width=2.6in]{FIGURES/NIST_NRC_02_v5_Ceiling_TC} \\
\includegraphics[width=2.6in]{FIGURES/NIST_NRC_08_v5_Ceiling_Flux_Gauges} &
\includegraphics[width=2.6in]{FIGURES/NIST_NRC_08_v5_Ceiling_TC} \\
\includegraphics[width=2.6in]{FIGURES/NIST_NRC_04_v5_Ceiling_Flux_Gauges} &
\includegraphics[width=2.6in]{FIGURES/NIST_NRC_04_v5_Ceiling_TC} \\
\includegraphics[width=2.6in]{FIGURES/NIST_NRC_10_v5_Ceiling_Flux_Gauges} &
\includegraphics[width=2.6in]{FIGURES/NIST_NRC_10_v5_Ceiling_TC}

\end{tabular*}
\caption{NIST/NRC Series, Ceiling Heat Flux and Temperature, Tests 2, 8, 4 and 10.}
\label{NIST_NRC_Ceiling_2}
\end{figure}

\begin{figure}[p]
\begin{tabular*}{\textwidth}{l@{\extracolsep{\fill}}r}
\includegraphics[width=2.6in]{FIGURES/NIST_NRC_13_v5_Ceiling_Flux_Gauges} &
\includegraphics[width=2.6in]{FIGURES/NIST_NRC_13_v5_Ceiling_TC} \\
\includegraphics[width=2.6in]{FIGURES/NIST_NRC_16_v5_Ceiling_Flux_Gauges} &
\includegraphics[width=2.6in]{FIGURES/NIST_NRC_16_v5_Ceiling_TC} \\
\includegraphics[width=2.6in]{FIGURES/NIST_NRC_03_v5_Ceiling_Flux_Gauges} &
\includegraphics[width=2.6in]{FIGURES/NIST_NRC_03_v5_Ceiling_TC} \\
\includegraphics[width=2.6in]{FIGURES/NIST_NRC_09_v5_Ceiling_Flux_Gauges} &
\includegraphics[width=2.6in]{FIGURES/NIST_NRC_09_v5_Ceiling_TC}

\end{tabular*}
\caption{NIST/NRC Series, Ceiling Heat Flux and Temperature, Test 13, 16, 3 and 9.}
\label{NIST_NRC_Ceiling_3}
\end{figure}

\begin{figure}[p]
\begin{tabular*}{\textwidth}{l@{\extracolsep{\fill}}r}
\includegraphics[width=2.6in]{FIGURES/NIST_NRC_05_v5_Ceiling_Flux_Gauges} &
\includegraphics[width=2.6in]{FIGURES/NIST_NRC_05_v5_Ceiling_TC} \\
\includegraphics[width=2.6in]{FIGURES/NIST_NRC_14_v5_Ceiling_Flux_Gauges} &
\includegraphics[width=2.6in]{FIGURES/NIST_NRC_14_v5_Ceiling_TC} \\
\includegraphics[width=2.6in]{FIGURES/NIST_NRC_15_v5_Ceiling_Flux_Gauges} &
\includegraphics[width=2.6in]{FIGURES/NIST_NRC_15_v5_Ceiling_TC} \\
\includegraphics[width=2.6in]{FIGURES/NIST_NRC_18_v5_Ceiling_Flux_Gauges} &
\includegraphics[width=2.6in]{FIGURES/NIST_NRC_18_v5_Ceiling_TC}
\end{tabular*}
\caption{NIST/NRC Series, Ceiling Heat Flux and Temperature, Tests 5, 14, 15 and 18.}
\label{NIST_NRC_Ceiling_4}
\end{figure}

\clearpage



\subsection{Floor}

\vspace{2in}


\begin{figure}[h!]
\begin{tabular*}{\textwidth}{l@{\extracolsep{\fill}}r}
\includegraphics[width=2.6in]{FIGURES/NIST_NRC_01_v5_Floor_Flux_Gauges} &
\includegraphics[width=2.6in]{FIGURES/NIST_NRC_01_v5_Floor_TC} \\
\includegraphics[width=2.6in]{FIGURES/NIST_NRC_07_v5_Floor_Flux_Gauges} &
\includegraphics[width=2.6in]{FIGURES/NIST_NRC_07_v5_Floor_TC}

\end{tabular*}
\caption{NIST/NRC Series, Floor Heat Flux and Temperature, Tests 1 and 7.}
\label{NIST_NRC_Floor_1}
\end{figure}

\begin{figure}[p]
\begin{tabular*}{\textwidth}{l@{\extracolsep{\fill}}r}
\includegraphics[width=2.6in]{FIGURES/NIST_NRC_02_v5_Floor_Flux_Gauges} &
\includegraphics[width=2.6in]{FIGURES/NIST_NRC_02_v5_Floor_TC} \\
\includegraphics[width=2.6in]{FIGURES/NIST_NRC_08_v5_Floor_Flux_Gauges} &
\includegraphics[width=2.6in]{FIGURES/NIST_NRC_08_v5_Floor_TC} \\
\includegraphics[width=2.6in]{FIGURES/NIST_NRC_04_v5_Floor_Flux_Gauges} &
\includegraphics[width=2.6in]{FIGURES/NIST_NRC_04_v5_Floor_TC} \\
\includegraphics[width=2.6in]{FIGURES/NIST_NRC_10_v5_Floor_Flux_Gauges} &
\includegraphics[width=2.6in]{FIGURES/NIST_NRC_10_v5_Floor_TC}

\end{tabular*}
\caption{NIST/NRC Series, Floor Heat Flux and Temperature, Tests 2, 8, 4 and 10.}
\label{NIST_NRC_Floor_2}
\end{figure}

\begin{figure}[p]
\begin{tabular*}{\textwidth}{l@{\extracolsep{\fill}}r}
\includegraphics[width=2.6in]{FIGURES/NIST_NRC_13_v5_Floor_Flux_Gauges} &
\includegraphics[width=2.6in]{FIGURES/NIST_NRC_13_v5_Floor_TC} \\
\includegraphics[width=2.6in]{FIGURES/NIST_NRC_16_v5_Floor_Flux_Gauges} &
\includegraphics[width=2.6in]{FIGURES/NIST_NRC_16_v5_Floor_TC} \\
\includegraphics[width=2.6in]{FIGURES/NIST_NRC_03_v5_Floor_Flux_Gauges} &
\includegraphics[width=2.6in]{FIGURES/NIST_NRC_03_v5_Floor_TC} \\
\includegraphics[width=2.6in]{FIGURES/NIST_NRC_09_v5_Floor_Flux_Gauges} &
\includegraphics[width=2.6in]{FIGURES/NIST_NRC_09_v5_Floor_TC}

\end{tabular*}
\caption{NIST/NRC Series, Floor Heat Flux and Temperature, Test 13, 16, 3 and 9.}
\label{NIST_NRC_Floor_3}
\end{figure}

\begin{figure}[p]
\begin{tabular*}{\textwidth}{l@{\extracolsep{\fill}}r}
\includegraphics[width=2.6in]{FIGURES/NIST_NRC_05_v5_Floor_Flux_Gauges} &
\includegraphics[width=2.6in]{FIGURES/NIST_NRC_05_v5_Floor_TC} \\
\includegraphics[width=2.6in]{FIGURES/NIST_NRC_14_v5_Floor_Flux_Gauges} &
\includegraphics[width=2.6in]{FIGURES/NIST_NRC_14_v5_Floor_TC} \\
\includegraphics[width=2.6in]{FIGURES/NIST_NRC_15_v5_Floor_Flux_Gauges} &
\includegraphics[width=2.6in]{FIGURES/NIST_NRC_15_v5_Floor_TC} \\
\includegraphics[width=2.6in]{FIGURES/NIST_NRC_18_v5_Floor_Flux_Gauges} &
\includegraphics[width=2.6in]{FIGURES/NIST_NRC_18_v5_Floor_TC}
\end{tabular*}
\caption{NIST/NRC Series, Floor Heat Flux and Temperature, Tests 5, 14, 15 and 18.}
\label{NIST_NRC_Floor_4}
\end{figure}

\clearpage



\chapter{Suppression}

This chapter looks at validation exercises where the aim is to predict the time of extinguishment of a fire.


\section{USCG/HAI Water Mist Suppression Tests}

The following pages contain comparisons of the predicted heat release rates for fires that are suppressed with a water mist system. In all cases, the flow rate of liquid
fuel is specified in the model, but the decrease in HRR due to the extinguishing system is predicted by the model. Table~\ref{USCG_HAI_Times} reports the observed extinguishment
times. Figure~\ref{USCG_Scatter} compares the measured versus predicted extinguishment times. For the simulations, the extinguishment time is taken to be when the HRR drops to
half of its specified value.

\begin{table}[h!]
\caption[USCG/HAI water mist suppression extinguishment times.]{Recorded extinguishment times for the USCG/HAI water mist suppression tests in a small shipboard machinery space. ``No''
means that the fire was not extinguished within 600 s of nozzle activation.}
\begin{center}
\begin{tabular}{|l|c|c|c|c|c|c|}
\hline
\multicolumn{2}{|l|}{System}                            & Navy  & Grinnell  & Fogtec    & Chemetron & Fike   \\ \hline  \hline
\multicolumn{2}{|l|}{Number of Nozzles}                 & 6     & 6         & 6         & 15        & 6      \\ \hline
\multicolumn{2}{|l|}{Operating Pressure (bar)}          & 70    & 13        & 100       & 12        & 21     \\ \hline
\multicolumn{2}{|l|}{Flow Rate (L/min)}                 & 68    & 75        & 22        & 70        & 48     \\ \hline
\multicolumn{2}{|l|}{Assumed Median Drop Size ($\mu$m)} & 175   & 225       & 100       &           & 200    \\ \hline
\multicolumn{2}{|l|}{Assumed Initial Velocity (m/s)}    & 75    & 32        & 90        &           & 41     \\ \hline
\multicolumn{2}{|l|}{Assumed Spray Angle (deg.)}        & 120   & 90        & 120       &           & 90     \\ \hline \hline
Fire Scenario       & Ventilation                       & \multicolumn{5}{c|}{Extinguishment Time (s)}      \\ \hline \hline
1.0 MW Spray        & Closed                            & 15    & 26        & 21        & 27        & 21     \\ \hline
1.0 MW Spray        & Natural                           & 15    & 40        & 32        & 43        & 35     \\ \hline
1.0 MW Spray        & Forced                            & 17    & 55        & 76        & 357       & 133    \\ \hline
0.5 MW Spray        & Closed                            & 34    & 70        & 39        & 53        & 56     \\ \hline
0.5 MW Spray        & Natural                           & 41    & 117       & 67        & 158       & 140    \\ \hline
0.5 MW Spray        & Forced                            & 124   & No        & No        & No        & No     \\ \hline
0.25 MW Spray       & Closed                            & 157   & 360       & 169       & 314       & 277    \\ \hline
0.25 MW Spray       & Natural                           & 206   & No        & 290       & 525       & 566    \\ \hline
0.25 MW Spray       & Forced                            & No    & No        & No        & No        & No     \\ \hline
\end{tabular}
\end{center}
\label{USCG_HAI_Times}
\end{table}

\begin{figure}[h!]
\begin{center}
\includegraphics[height=4in]{FIGURES/ScatterPlots/USCG_HAI_Extinction}
\caption[Comparison of measured and predicted extinguishment times for the USCG/HAI water mist suppression tests.]{Comparison of measured and predicted extinguishment times for the USCG/HAI water mist suppression tests.}
\label{USCG_Scatter}
\end{center}
\end{figure}




\begin{figure}[p]
\begin{tabular*}{\textwidth}{l@{\extracolsep{\fill}}r}
\includegraphics[height=2.2in]{FIGURES/USCG_HAI/USCG_HAI_HRR_1000_Closed_Grinnell} &
\includegraphics[height=2.2in]{FIGURES/USCG_HAI/USCG_HAI_HRR_1000_Closed_Navy} \\
\includegraphics[height=2.2in]{FIGURES/USCG_HAI/USCG_HAI_HRR_1000_Closed_Fogtec} &
\includegraphics[height=2.2in]{FIGURES/USCG_HAI/USCG_HAI_HRR_1000_Closed_Fike}
\end{tabular*}
\label{USCG_HAI_1}
\end{figure}

\begin{figure}[p]
\begin{tabular*}{\textwidth}{l@{\extracolsep{\fill}}r}
\includegraphics[height=2.2in]{FIGURES/USCG_HAI/USCG_HAI_HRR_1000_Natural_Grinnell} &
\includegraphics[height=2.2in]{FIGURES/USCG_HAI/USCG_HAI_HRR_1000_Natural_Navy} \\
\includegraphics[height=2.2in]{FIGURES/USCG_HAI/USCG_HAI_HRR_1000_Natural_Fogtec} &
\includegraphics[height=2.2in]{FIGURES/USCG_HAI/USCG_HAI_HRR_1000_Natural_Fike}
\end{tabular*}
\label{USCG_HAI_2}
\end{figure}

\begin{figure}[p]
\begin{tabular*}{\textwidth}{l@{\extracolsep{\fill}}r}
\includegraphics[height=2.2in]{FIGURES/USCG_HAI/USCG_HAI_HRR_1000_Forced_Grinnell} &
\includegraphics[height=2.2in]{FIGURES/USCG_HAI/USCG_HAI_HRR_1000_Forced_Navy} \\
\includegraphics[height=2.2in]{FIGURES/USCG_HAI/USCG_HAI_HRR_1000_Forced_Fogtec} &
\includegraphics[height=2.2in]{FIGURES/USCG_HAI/USCG_HAI_HRR_1000_Forced_Fike}
\end{tabular*}
\label{USCG_HAI_3}
\end{figure}

\begin{figure}[p]
\begin{tabular*}{\textwidth}{l@{\extracolsep{\fill}}r}
\includegraphics[height=2.2in]{FIGURES/USCG_HAI/USCG_HAI_HRR_500_Closed_Grinnell} &
\includegraphics[height=2.2in]{FIGURES/USCG_HAI/USCG_HAI_HRR_500_Closed_Navy} \\
\includegraphics[height=2.2in]{FIGURES/USCG_HAI/USCG_HAI_HRR_500_Closed_Fogtec} &
\includegraphics[height=2.2in]{FIGURES/USCG_HAI/USCG_HAI_HRR_500_Closed_Fike}
\end{tabular*}
\label{USCG_HAI_4}
\end{figure}

\begin{figure}[p]
\begin{tabular*}{\textwidth}{l@{\extracolsep{\fill}}r}
\includegraphics[height=2.2in]{FIGURES/USCG_HAI/USCG_HAI_HRR_500_Natural_Grinnell} &
\includegraphics[height=2.2in]{FIGURES/USCG_HAI/USCG_HAI_HRR_500_Natural_Navy} \\
\includegraphics[height=2.2in]{FIGURES/USCG_HAI/USCG_HAI_HRR_500_Natural_Fogtec} &
\includegraphics[height=2.2in]{FIGURES/USCG_HAI/USCG_HAI_HRR_500_Natural_Fike}
\end{tabular*}
\label{USCG_HAI_5}
\end{figure}

\begin{figure}[p]
\begin{tabular*}{\textwidth}{l@{\extracolsep{\fill}}r}
\includegraphics[height=2.2in]{FIGURES/USCG_HAI/USCG_HAI_HRR_500_Forced_Grinnell} &
\includegraphics[height=2.2in]{FIGURES/USCG_HAI/USCG_HAI_HRR_500_Forced_Navy} \\
\includegraphics[height=2.2in]{FIGURES/USCG_HAI/USCG_HAI_HRR_500_Forced_Fogtec} &
\includegraphics[height=2.2in]{FIGURES/USCG_HAI/USCG_HAI_HRR_500_Forced_Fike}
\end{tabular*}
\label{USCG_HAI_6}
\end{figure}

\begin{figure}[p]
\begin{tabular*}{\textwidth}{l@{\extracolsep{\fill}}r}
\includegraphics[height=2.2in]{FIGURES/USCG_HAI/USCG_HAI_HRR_250_Closed_Grinnell} &
\includegraphics[height=2.2in]{FIGURES/USCG_HAI/USCG_HAI_HRR_250_Closed_Navy} \\
\includegraphics[height=2.2in]{FIGURES/USCG_HAI/USCG_HAI_HRR_250_Closed_Fogtec} &
\includegraphics[height=2.2in]{FIGURES/USCG_HAI/USCG_HAI_HRR_250_Closed_Fike}
\end{tabular*}
\label{USCG_HAI_7}
\end{figure}

\begin{figure}[p]
\begin{tabular*}{\textwidth}{l@{\extracolsep{\fill}}r}
\includegraphics[height=2.2in]{FIGURES/USCG_HAI/USCG_HAI_HRR_250_Natural_Grinnell} &
\includegraphics[height=2.2in]{FIGURES/USCG_HAI/USCG_HAI_HRR_250_Natural_Navy} \\
\includegraphics[height=2.2in]{FIGURES/USCG_HAI/USCG_HAI_HRR_250_Natural_Fogtec} &
\includegraphics[height=2.2in]{FIGURES/USCG_HAI/USCG_HAI_HRR_250_Natural_Fike}
\end{tabular*}
\label{USCG_HAI_8}
\end{figure}


\begin{figure}[p]
\begin{tabular*}{\textwidth}{l@{\extracolsep{\fill}}r}
\includegraphics[height=2.2in]{FIGURES/USCG_HAI/USCG_HAI_HRR_250_Forced_Grinnell} &
\includegraphics[height=2.2in]{FIGURES/USCG_HAI/USCG_HAI_HRR_250_Forced_Navy} \\
\includegraphics[height=2.2in]{FIGURES/USCG_HAI/USCG_HAI_HRR_250_Forced_Fogtec} &
\includegraphics[height=2.2in]{FIGURES/USCG_HAI/USCG_HAI_HRR_250_Forced_Fike}
\end{tabular*}
\label{USCG_HAI_9}
\end{figure}



\chapter{Burning Rate}

This chapter looks at validation exercises where the aim is to {\em predict} the burning rate of the fuel. Most of the simulations included in the previous chapters involved
a {\em specified} burning or heat release rate.


\section{FAA Polymers}

A non-charring polymer is considered one of the easier solids to model because it typically involves only a single, first order reaction that converts solid plastic to fuel vapor. 
No residue is formed and the plastic is completely pyrolyzed. Table~\ref{FAA_Properties}
lists nine parameters for each polymer studied. These values have been input directly into FDS, and the predicted burning rates are compared with measured values from the NIST
Gasification Apparatus, a device that pyrolyzes the solid in a nitrogen environment to prevent combustion of fuel gases. The results are shown in Fig.~\ref{FAA_Polymers}. The exposing
heat flux was 52~kW/m$^2$. A 1~cm layer of insulation was placed under the sample. Its properties are given in Ref.~\cite{Stoliarov:CF2009}.

\begin{table}[h!]
\caption[FAA Polymer Properties]{Input parameters for FAA Polymers non-charring examples. Courtesy S.~Stoliarov, M.~McKinnon and J.~Li, University of Maryland.}
\begin{tabular}{|l|c|c|c|c|c|l|l|}
\hline
Property                    & Units         & HDPE                  & HIPS                  & PMMA                  & Unc. (\%)  & Method                &  Ref.                    \\ \hline \hline
Density                     & kg/m$^3$      & 860                   & 950                   & 1100                  & 5     & Direct                &  \cite{Stoliarov:CF2009}  \\ \hline
Conductivity                & W/m/K         & 0.29                  & 0.22                  & 0.20                  & 15    & Thermoflixer          &  \cite{Stoliarov:CF2009}  \\ \hline
Specific Heat               & kJ/kg/K       & 3.5                   & 2.0                   & 2.2                   & 15    & DSC                   &  \cite{Stoliarov:PDS2008}  \\ \hline
Emissivity                  &               & 0.92                  & 0.86                  & 0.85                  & 20    & Sphere                &  \cite{Hallman:PES1974}  \\ \hline
Absorption Coef.            & m$^{-1}$      & 1300                  & 2700                  & 2700                  & 50    & FTIR                  &  \cite{Tsilingiris:ECM2003}  \\ \hline
Pre-Exp.~Factor             & s$^{-1}$      & $4.8 \times 10^{22}$  & $1.2 \times 10^{16}$  & $8.5 \times 10^{12}$  & 50    & TGA                   &  \cite{Stoliarov:CF2009}  \\ \hline
Activation Energy           & kJ/kmol       & $3.49 \times 10^{5}$  & $2.47 \times 10^{5}$  & $1.88 \times 10^{5}$  & 3     & TGA                   &  \cite{Stoliarov:CF2009}  \\ \hline
Heat of Reaction            & kJ/kg         & 920                   & 1000                  & 870                   & 15    & DSC                   &  \cite{Stoliarov:PDS2008}  \\ \hline
\end{tabular}
\label{FAA_Properties}
\end{table}

\begin{tabbing}
Direct  \hspace{0.5in}     \= Direct measurement of mass and volume \\
Thermoflixer               \> Transient line source method \\
DSC                        \> Differential Scanning Calorimetry \\
Sphere                     \> Integrating sphere \\
FTIR                       \> Fourier Transform Infrared Spectroscopy \\
TGA                        \> Thermogravimetric Analysis
\end{tabbing}

\newpage

\begin{figure}[p]
\begin{center}
\begin{tabular}{c}
\includegraphics[height=2.2in]{FIGURES/FAA_Polymers/FAA_Polymers_HDPE} \\
\includegraphics[height=2.2in]{FIGURES/FAA_Polymers/FAA_Polymers_HIPS} \\
\includegraphics[height=2.2in]{FIGURES/FAA_Polymers/FAA_Polymers_PMMA}
\end{tabular}
\end{center}
\caption[Results of FAA Polymers comparison]{Comparison of predicted and measured mass loss rates for three non-charring polymers exposed to a heat flux of 52~kW/m$^2$ in a
nitrogen environment.}
\label{FAA_Polymers}
\end{figure}



% !TEX root = FDS_Validation_Guide.tex

\chapter{Wind Engineering and Atmospheric Dispersion}

This chapter presents results of simulations of wind over structures and atmospheric dispersion, all involving a simplified atmospheric boundary layer model in FDS.


\section{UWO Wind Tunnel Experiments}

A description of the UWO Wind Tunnel experiments is included in Section~\ref{UWO_Wind_Tunnel_Description}. Schematic drawings of the wind tunnel models are shown in Fig.~\ref{UWO_Drawings}.

Figures~\ref{UWO_Test_7_pressure_coefficients_180_1} through \ref{UWO_Test_7_pressure_coefficients_270_2} show comparisons of measured and predicted mean, rms, minimum and maximum values of the pressure coefficients on the surface of a 1:100 scale model of a building in a wind tunnel. The model building is shown at the top of Fig.~\ref{UWO_Drawings}. Figures~\ref{UWO_SS21_Test_6_pressure_coefficients_0_1} through \ref{UWO_SS21_Test_6_pressure_coefficients_45_4} show similar results for the model shown at the bottom of Fig.~\ref{UWO_Drawings}. The comparisons are made for two wind directions for each model. For SS20-Test~7, the 180$^\circ$ wind direction is perpendicular to the model's shorter side. The 270$^\circ$ wind direction is perpendicular to the model's longer side. For the SS21-Test~6 model, the wind directions are 0$^\circ$ (perpendicular to short side) and 45$^\circ$.

The diagrams in Fig.~\ref{UWO_Drawings} indicate the location of the ``lines'' where the data is compared. The discontinuities in the lines represent the transition from the windward side, to the roof, to the leeward side. The side plots do not include windward or leeward side data.

The simulations are run for approximately one-tenth the time as that of the experiments, which were run for 100~s. The minimum and maximum values for the simulations are extrapolated so that they may be compared to the measured min and max for the 100~s experiment. The procedure is described in the FDS User's Guide~\cite{FDS_Users_Guide}, in the section describing the {\ct TEMPORAL\_STATISTIC} {\ct 'MIN'} and {\ct 'MAX'}.

\begin{figure}[!ht]
\centering
\includegraphics[width=5.5in]{FIGURES/UWO_Wind_Tunnel/UWO-BLT-SS20-Test7}
\includegraphics[width=5.5in]{FIGURES/UWO_Wind_Tunnel/UWO-BLWT-SS21-Test6-40ft}
\caption[UWO Wind Tunnel schematic drawings]{UWO Wind Tunnel schematic drawings. (Top) SS20-Test 7. (Bottom) SS21-Test 6. The numbers at the base of the models denote the starting points of the lines over which the measurements and predictions are compared.}
\label{UWO_Drawings}
\end{figure}

\begin{figure}[p]
\begin{tabular*}{\textwidth}{l@{\extracolsep{\fill}}r}
\includegraphics[height=2.1in]{SCRIPT_FIGURES/UWO_Wind_Tunnel/UWO_Cp_mean_180_line_1} &
\includegraphics[height=2.1in]{SCRIPT_FIGURES/UWO_Wind_Tunnel/UWO_Cp_rms_180_line_1} \\
\includegraphics[height=2.1in]{SCRIPT_FIGURES/UWO_Wind_Tunnel/UWO_Cp_min_180_line_1} &
\includegraphics[height=2.1in]{SCRIPT_FIGURES/UWO_Wind_Tunnel/UWO_Cp_max_180_line_1} \\
\includegraphics[height=2.1in]{SCRIPT_FIGURES/UWO_Wind_Tunnel/UWO_Cp_mean_180_line_2} &
\includegraphics[height=2.1in]{SCRIPT_FIGURES/UWO_Wind_Tunnel/UWO_Cp_rms_180_line_2} \\
\includegraphics[height=2.1in]{SCRIPT_FIGURES/UWO_Wind_Tunnel/UWO_Cp_min_180_line_2} &
\includegraphics[height=2.1in]{SCRIPT_FIGURES/UWO_Wind_Tunnel/UWO_Cp_max_180_line_2}
\end{tabular*}
\caption[UWO Wind Tunnel, SS20-Test 7 pressure coefficients, 180\si{\degree}]{UWO Wind Tunnel, SS20-Test 7 pressure coefficients, 180\si{\degree} wind direction.}
\label{UWO_Test_7_pressure_coefficients_180_1}
\end{figure}

\begin{figure}[p]
\begin{tabular*}{\textwidth}{l@{\extracolsep{\fill}}r}
\includegraphics[height=2.1in]{SCRIPT_FIGURES/UWO_Wind_Tunnel/UWO_Cp_mean_180_line_3} &
\includegraphics[height=2.1in]{SCRIPT_FIGURES/UWO_Wind_Tunnel/UWO_Cp_rms_180_line_3} \\
\includegraphics[height=2.1in]{SCRIPT_FIGURES/UWO_Wind_Tunnel/UWO_Cp_min_180_line_3} &
\includegraphics[height=2.1in]{SCRIPT_FIGURES/UWO_Wind_Tunnel/UWO_Cp_max_180_line_3} \\
\includegraphics[height=2.1in]{SCRIPT_FIGURES/UWO_Wind_Tunnel/UWO_Cp_mean_180_line_4} &
\includegraphics[height=2.1in]{SCRIPT_FIGURES/UWO_Wind_Tunnel/UWO_Cp_rms_180_line_4}  \\
\includegraphics[height=2.1in]{SCRIPT_FIGURES/UWO_Wind_Tunnel/UWO_Cp_min_180_line_4} &
\includegraphics[height=2.1in]{SCRIPT_FIGURES/UWO_Wind_Tunnel/UWO_Cp_max_180_line_4}
\end{tabular*}
\caption[UWO Wind Tunnel, SS20-Test 7 pressure coefficients, 180\si{\degree}]{UWO Wind Tunnel, SS20-Test 7 pressure coefficients, 180\si{\degree} wind direction.}
\label{UWO_Test_7_pressure_coefficients_180_2}
\end{figure}

\begin{figure}[p]
\begin{tabular*}{\textwidth}{l@{\extracolsep{\fill}}r}
\includegraphics[height=2.1in]{SCRIPT_FIGURES/UWO_Wind_Tunnel/UWO_Cp_mean_270_line_5} &
\includegraphics[height=2.1in]{SCRIPT_FIGURES/UWO_Wind_Tunnel/UWO_Cp_rms_270_line_5} \\
\includegraphics[height=2.1in]{SCRIPT_FIGURES/UWO_Wind_Tunnel/UWO_Cp_min_270_line_5} &
\includegraphics[height=2.1in]{SCRIPT_FIGURES/UWO_Wind_Tunnel/UWO_Cp_max_270_line_5} \\
\includegraphics[height=2.1in]{SCRIPT_FIGURES/UWO_Wind_Tunnel/UWO_Cp_mean_270_line_6} &
\includegraphics[height=2.1in]{SCRIPT_FIGURES/UWO_Wind_Tunnel/UWO_Cp_rms_270_line_6} \\
\includegraphics[height=2.1in]{SCRIPT_FIGURES/UWO_Wind_Tunnel/UWO_Cp_min_270_line_6} &
\includegraphics[height=2.1in]{SCRIPT_FIGURES/UWO_Wind_Tunnel/UWO_Cp_max_270_line_6}
\end{tabular*}
\caption[UWO Wind Tunnel, SS20-Test 7 pressure coefficients, 270\si{\degree}]{UWO Wind Tunnel, SS20-Test 7 pressure coefficients, 270\si{\degree} wind direction.}
\label{UWO_Test_7_pressure_coefficients_270_1}
\end{figure}

\begin{figure}[p]
\begin{tabular*}{\textwidth}{l@{\extracolsep{\fill}}r}
\includegraphics[height=2.1in]{SCRIPT_FIGURES/UWO_Wind_Tunnel/UWO_Cp_mean_270_line_7} &
\includegraphics[height=2.1in]{SCRIPT_FIGURES/UWO_Wind_Tunnel/UWO_Cp_rms_270_line_7} \\
\includegraphics[height=2.1in]{SCRIPT_FIGURES/UWO_Wind_Tunnel/UWO_Cp_min_270_line_7} &
\includegraphics[height=2.1in]{SCRIPT_FIGURES/UWO_Wind_Tunnel/UWO_Cp_max_270_line_7} \\
\includegraphics[height=2.1in]{SCRIPT_FIGURES/UWO_Wind_Tunnel/UWO_Cp_mean_270_line_8} &
\includegraphics[height=2.1in]{SCRIPT_FIGURES/UWO_Wind_Tunnel/UWO_Cp_rms_270_line_8}  \\
\includegraphics[height=2.1in]{SCRIPT_FIGURES/UWO_Wind_Tunnel/UWO_Cp_min_270_line_8} &
\includegraphics[height=2.1in]{SCRIPT_FIGURES/UWO_Wind_Tunnel/UWO_Cp_max_270_line_8}
\end{tabular*}
\caption[UWO Wind Tunnel, SS20-Test 7 pressure coefficients, 270\si{\degree}]{UWO Wind Tunnel, SS20-Test 7 pressure coefficients, 270\si{\degree} wind direction.}
\label{UWO_Test_7_pressure_coefficients_270_2}
\end{figure}


\begin{figure}[p]
\begin{tabular*}{\textwidth}{l@{\extracolsep{\fill}}r}
\includegraphics[height=2.1in]{SCRIPT_FIGURES/UWO_Wind_Tunnel/UWO_SS21_Test_6_Cp_mean_0_line_1} &
\includegraphics[height=2.1in]{SCRIPT_FIGURES/UWO_Wind_Tunnel/UWO_SS21_Test_6_Cp_rms_0_line_1} \\
\includegraphics[height=2.1in]{SCRIPT_FIGURES/UWO_Wind_Tunnel/UWO_SS21_Test_6_Cp_min_0_line_1} &
\includegraphics[height=2.1in]{SCRIPT_FIGURES/UWO_Wind_Tunnel/UWO_SS21_Test_6_Cp_max_0_line_1} \\
\includegraphics[height=2.1in]{SCRIPT_FIGURES/UWO_Wind_Tunnel/UWO_SS21_Test_6_Cp_mean_0_line_2} &
\includegraphics[height=2.1in]{SCRIPT_FIGURES/UWO_Wind_Tunnel/UWO_SS21_Test_6_Cp_rms_0_line_2} \\
\includegraphics[height=2.1in]{SCRIPT_FIGURES/UWO_Wind_Tunnel/UWO_SS21_Test_6_Cp_min_0_line_2} &
\includegraphics[height=2.1in]{SCRIPT_FIGURES/UWO_Wind_Tunnel/UWO_SS21_Test_6_Cp_max_0_line_2}
\end{tabular*}
\caption[UWO Wind Tunnel, SS21-Test 6 pressure coefficients, 0\si{\degree}]{UWO Wind Tunnel, SS21-Test 6 pressure coefficients, 0\si{\degree} wind direction.}
\label{UWO_SS21_Test_6_pressure_coefficients_0_1}
\end{figure}

\begin{figure}[p]
\begin{tabular*}{\textwidth}{l@{\extracolsep{\fill}}r}
\includegraphics[height=2.1in]{SCRIPT_FIGURES/UWO_Wind_Tunnel/UWO_SS21_Test_6_Cp_mean_0_line_3} &
\includegraphics[height=2.1in]{SCRIPT_FIGURES/UWO_Wind_Tunnel/UWO_SS21_Test_6_Cp_rms_0_line_3} \\
\includegraphics[height=2.1in]{SCRIPT_FIGURES/UWO_Wind_Tunnel/UWO_SS21_Test_6_Cp_min_0_line_3} &
\includegraphics[height=2.1in]{SCRIPT_FIGURES/UWO_Wind_Tunnel/UWO_SS21_Test_6_Cp_max_0_line_3} \\
\includegraphics[height=2.1in]{SCRIPT_FIGURES/UWO_Wind_Tunnel/UWO_SS21_Test_6_Cp_mean_0_line_4} &
\includegraphics[height=2.1in]{SCRIPT_FIGURES/UWO_Wind_Tunnel/UWO_SS21_Test_6_Cp_rms_0_line_4} \\
\includegraphics[height=2.1in]{SCRIPT_FIGURES/UWO_Wind_Tunnel/UWO_SS21_Test_6_Cp_min_0_line_4} &
\includegraphics[height=2.1in]{SCRIPT_FIGURES/UWO_Wind_Tunnel/UWO_SS21_Test_6_Cp_max_0_line_4}
\end{tabular*}
\caption[UWO Wind Tunnel, SS21-Test 6 pressure coefficients, 0\si{\degree}]{UWO Wind Tunnel, SS21-Test 6 pressure coefficients, 0\si{\degree} wind direction.}
\label{UWO_SS21_Test_6_pressure_coefficients_0_2}
\end{figure}

\begin{figure}[p]
\begin{tabular*}{\textwidth}{l@{\extracolsep{\fill}}r}
\includegraphics[height=2.1in]{SCRIPT_FIGURES/UWO_Wind_Tunnel/UWO_SS21_Test_6_Cp_mean_45_line_1} &
\includegraphics[height=2.1in]{SCRIPT_FIGURES/UWO_Wind_Tunnel/UWO_SS21_Test_6_Cp_rms_45_line_1} \\
\includegraphics[height=2.1in]{SCRIPT_FIGURES/UWO_Wind_Tunnel/UWO_SS21_Test_6_Cp_min_45_line_1} &
\includegraphics[height=2.1in]{SCRIPT_FIGURES/UWO_Wind_Tunnel/UWO_SS21_Test_6_Cp_max_45_line_1} \\
\includegraphics[height=2.1in]{SCRIPT_FIGURES/UWO_Wind_Tunnel/UWO_SS21_Test_6_Cp_mean_45_line_2} &
\includegraphics[height=2.1in]{SCRIPT_FIGURES/UWO_Wind_Tunnel/UWO_SS21_Test_6_Cp_rms_45_line_2} \\
\includegraphics[height=2.1in]{SCRIPT_FIGURES/UWO_Wind_Tunnel/UWO_SS21_Test_6_Cp_min_45_line_2} &
\includegraphics[height=2.1in]{SCRIPT_FIGURES/UWO_Wind_Tunnel/UWO_SS21_Test_6_Cp_max_45_line_2}
\end{tabular*}
\caption[UWO Wind Tunnel, SS21-Test 6 pressure coefficients, 45\si{\degree}]{UWO Wind Tunnel, SS21-Test 6 pressure coefficients, 45\si{\degree} wind direction.}
\label{UWO_SS21_Test_6_pressure_coefficients_45_3}
\end{figure}

\begin{figure}[p]
\begin{tabular*}{\textwidth}{l@{\extracolsep{\fill}}r}
\includegraphics[height=2.1in]{SCRIPT_FIGURES/UWO_Wind_Tunnel/UWO_SS21_Test_6_Cp_mean_45_line_3} &
\includegraphics[height=2.1in]{SCRIPT_FIGURES/UWO_Wind_Tunnel/UWO_SS21_Test_6_Cp_rms_45_line_3} \\
\includegraphics[height=2.1in]{SCRIPT_FIGURES/UWO_Wind_Tunnel/UWO_SS21_Test_6_Cp_min_45_line_3} &
\includegraphics[height=2.1in]{SCRIPT_FIGURES/UWO_Wind_Tunnel/UWO_SS21_Test_6_Cp_max_45_line_3} \\
\includegraphics[height=2.1in]{SCRIPT_FIGURES/UWO_Wind_Tunnel/UWO_SS21_Test_6_Cp_mean_45_line_4} &
\includegraphics[height=2.1in]{SCRIPT_FIGURES/UWO_Wind_Tunnel/UWO_SS21_Test_6_Cp_rms_45_line_4} \\
\includegraphics[height=2.1in]{SCRIPT_FIGURES/UWO_Wind_Tunnel/UWO_SS21_Test_6_Cp_min_45_line_4} &
\includegraphics[height=2.1in]{SCRIPT_FIGURES/UWO_Wind_Tunnel/UWO_SS21_Test_6_Cp_max_45_line_4}
\end{tabular*}
\caption[UWO Wind Tunnel, SS21-Test 6 pressure coefficients, 45\si{\degree}]{UWO Wind Tunnel, SS21-Test 6 pressure coefficients, 45\si{\degree} wind direction.}
\label{UWO_SS21_Test_6_pressure_coefficients_45_4}
\end{figure}






\section{LNG Dispersion Experiments}
\label{Atmospheric Dispersion}

Details of the numerical modeling of these experiments is found in Section~\ref{LNG_Dispersion_Description}.

Figure~\ref{LNG_Dispersion_Burro_profiles} through Fig.~\ref{LNG_Dispersion_MaplinSands_profiles} display the measured velocity and temperature profiles, the corresponding Monin-Obukhov profiles that serve as initial and boundary conditions for FDS, and the resulting time-averaged profiles from the FDS simulations.

Figures~\ref{LNG_Dispersion_1}--\ref{LNG_Dispersion_2} compare measured and predicted downwind concentrations of natural gas originating from spills of liquefied natural gas (LNG) on water. In each case, the measured values are short-time (1~s to 3~s) averages of sensors positioned in arcs at discrete distances downwind of the spill site. For each arc, the maximum value is chosen. The processing of the FDS results follows the same procedure. The sensors were generally located a few meters off the relatively dry, flat terrain.

\newpage

\begin{figure}[p]
\begin{tabular*}{\textwidth}{l@{\extracolsep{\fill}}r}
\includegraphics[height=2.1in]{SCRIPT_FIGURES/LNG_Dispersion/Burro3_vel} &
\includegraphics[height=2.1in]{SCRIPT_FIGURES/LNG_Dispersion/Burro3_tmp} \\
\includegraphics[height=2.1in]{SCRIPT_FIGURES/LNG_Dispersion/Burro7_vel} &
\includegraphics[height=2.1in]{SCRIPT_FIGURES/LNG_Dispersion/Burro7_tmp} \\
\includegraphics[height=2.1in]{SCRIPT_FIGURES/LNG_Dispersion/Burro8_vel} &
\includegraphics[height=2.1in]{SCRIPT_FIGURES/LNG_Dispersion/Burro8_tmp} \\
\includegraphics[height=2.1in]{SCRIPT_FIGURES/LNG_Dispersion/Burro9_vel} &
\includegraphics[height=2.1in]{SCRIPT_FIGURES/LNG_Dispersion/Burro9_tmp}
\end{tabular*}
\caption[LNG Dispersion experiments, Burro velocity and temperature profiles]{LNG Dispersion experiments, Burro velocity and temperature profiles.}
\label{LNG_Dispersion_Burro_profiles}
\end{figure}

\begin{figure}[p]
\begin{tabular*}{\textwidth}{l@{\extracolsep{\fill}}r}
\includegraphics[height=2.1in]{SCRIPT_FIGURES/LNG_Dispersion/Coyote3_vel} &
\includegraphics[height=2.1in]{SCRIPT_FIGURES/LNG_Dispersion/Coyote3_tmp} \\
\includegraphics[height=2.1in]{SCRIPT_FIGURES/LNG_Dispersion/Coyote5_vel} &
\includegraphics[height=2.1in]{SCRIPT_FIGURES/LNG_Dispersion/Coyote5_tmp} \\
\includegraphics[height=2.1in]{SCRIPT_FIGURES/LNG_Dispersion/Coyote6_vel} &
\includegraphics[height=2.1in]{SCRIPT_FIGURES/LNG_Dispersion/Coyote6_tmp}
\end{tabular*}
\caption[LNG Dispersion experiments, Coyote velocity and temperature profiles]{LNG Dispersion experiments, Coyote velocity and temperature profiles.}
\label{LNG_Dispersion_Coyote_profiles}
\end{figure}

\begin{figure}[p]
\begin{tabular*}{\textwidth}{l@{\extracolsep{\fill}}r}
\includegraphics[height=2.1in]{SCRIPT_FIGURES/LNG_Dispersion/Falcon1_vel} &
\includegraphics[height=2.1in]{SCRIPT_FIGURES/LNG_Dispersion/Falcon1_tmp} \\
\includegraphics[height=2.1in]{SCRIPT_FIGURES/LNG_Dispersion/Falcon3_vel} &
\includegraphics[height=2.1in]{SCRIPT_FIGURES/LNG_Dispersion/Falcon3_tmp} \\
\includegraphics[height=2.1in]{SCRIPT_FIGURES/LNG_Dispersion/Falcon4_vel} &
\includegraphics[height=2.1in]{SCRIPT_FIGURES/LNG_Dispersion/Falcon4_tmp}
\end{tabular*}
\caption[LNG Dispersion experiments, Falcon velocity and temperature profiles]{LNG Dispersion experiments, Falcon velocity and temperature profiles.}
\label{LNG_Dispersion_Falcon_profiles}
\end{figure}

\begin{figure}[p]
\begin{tabular*}{\textwidth}{l@{\extracolsep{\fill}}r}
\includegraphics[height=2.1in]{SCRIPT_FIGURES/LNG_Dispersion/MaplinSands27_vel} &
\includegraphics[height=2.1in]{SCRIPT_FIGURES/LNG_Dispersion/MaplinSands27_tmp} \\
\includegraphics[height=2.1in]{SCRIPT_FIGURES/LNG_Dispersion/MaplinSands34_vel} &
\includegraphics[height=2.1in]{SCRIPT_FIGURES/LNG_Dispersion/MaplinSands34_tmp} \\
\includegraphics[height=2.1in]{SCRIPT_FIGURES/LNG_Dispersion/MaplinSands35_vel} &
\includegraphics[height=2.1in]{SCRIPT_FIGURES/LNG_Dispersion/MaplinSands35_tmp}
\end{tabular*}
\caption[LNG Dispersion experiments, Maplin Sands velocity and temperature profiles]{LNG Dispersion experiments, Maplin Sands velocity and temperature profiles.}
\label{LNG_Dispersion_MaplinSands_profiles}
\end{figure}


\begin{figure}[p]
\begin{tabular*}{\textwidth}{l@{\extracolsep{\fill}}r}
\includegraphics[height=2.1in]{SCRIPT_FIGURES/LNG_Dispersion/Burro3} &
\includegraphics[height=2.1in]{SCRIPT_FIGURES/LNG_Dispersion/Burro7} \\
\includegraphics[height=2.1in]{SCRIPT_FIGURES/LNG_Dispersion/Burro8} &
\includegraphics[height=2.1in]{SCRIPT_FIGURES/LNG_Dispersion/Burro9} \\
\includegraphics[height=2.1in]{SCRIPT_FIGURES/LNG_Dispersion/Coyote3} &
\includegraphics[height=2.1in]{SCRIPT_FIGURES/LNG_Dispersion/Coyote5} \\
\multicolumn{2}{c}{\includegraphics[height=2.1in]{SCRIPT_FIGURES/LNG_Dispersion/Coyote6}}
\end{tabular*}
\caption[LNG Dispersion experiments, Burro and Coyote]{LNG Dispersion experiments, Burro and Coyote.}
\label{LNG_Dispersion_1}
\end{figure}

\begin{figure}[p]
\begin{tabular*}{\textwidth}{l@{\extracolsep{\fill}}r}
\includegraphics[height=2.1in]{SCRIPT_FIGURES/LNG_Dispersion/Falcon1} &
\includegraphics[height=2.1in]{SCRIPT_FIGURES/LNG_Dispersion/Falcon3} \\
\multicolumn{2}{c}{\includegraphics[height=2.1in]{SCRIPT_FIGURES/LNG_Dispersion/Falcon4}} \\
\includegraphics[height=2.1in]{SCRIPT_FIGURES/LNG_Dispersion/MaplinSands27} &
\includegraphics[height=2.1in]{SCRIPT_FIGURES/LNG_Dispersion/MaplinSands34} \\
\multicolumn{2}{c}{\includegraphics[height=2.1in]{SCRIPT_FIGURES/LNG_Dispersion/MaplinSands35}}
\end{tabular*}
\caption[LNG Dispersion experiments, Falson and Maplin Sands]{LNG Dispersion experiments, Falcon and Maplin Sands.}
\label{LNG_Dispersion_2}
\end{figure}

\begin{figure}[p]
\begin{center}
\begin{tabular}{c}
\includegraphics[width=6.0in]{SCRIPT_FIGURES/ScatterPlots/FDS_Atmospheric_Dispersion}
\end{tabular}
\end{center}
\caption[Summary of LNG Dispersion predictions]{Summary of LNG Dispersion predictions. Note that the dashed black lines denote plus/minus a factor of two from the measured values. The red dashed lines represent two relative standard deviations about the solid red line, the model average.}
\label{Summary_LNG_Dispersion}
\end{figure}
















\chapter{Conclusion}


\section{Summary of FDS Model Uncertainty Statistics}

Table~\ref{summary_stats} lists the summary statistics for the different quantities examined in this Guide. This is, for each quantity of interest, Table~\ref{summary_stats} lists the bias and relative standard deviation of the predicted values. It also lists the total number of experimental data sets on which these statistics are based, as well as the total number of point to point comparisons. Obviously, the more data sets and the more points, the more reliable the statistics.

For further details about model uncertainty and the meaning of these statistics, see Chapter~\ref{Error_Chapter}.

\IfFileExists{SCRIPT_FIGURES/Scatterplots/validation_statistics.tex}{\input{SCRIPT_FIGURES/Scatterplots/validation_statistics.tex}}{\typeout{Error: Missing file SCRIPT_FIGURES/Scatterplots/validation_statistics.tex}}


\section{Normality Tests}
\label{normality_tests}

The histograms on the following pages display the distribution of the quantity $\ln(M/E)$, where $M$ is a random variable representing the \underline{M}odel predictions and $E$ is a random variable representing the \underline{E}xperimental measurements for each of the quantities of interest listed in Table~\ref{summary_stats}. Recall from Chapter~\ref{Error_Chapter} that $\ln(M/E)$ is assumed to be normally distributed. Formally testing the assumption of normality is difficult because most normality tests tend to reject the assumption of normality when the number of samples is relatively large. As can be seen in some of the histograms on the following pages, some fairly ``normal'' looking distributions fail normality tests while decidedly non-normal distributions pass. Rather than relying on a formal statistical test, it is better to simply judge if a histogram conforms to the typical bell-shaped curve, and if so this adds confidence to the statistical treatment of the data. If the histogram is not bell-shaped, this might cast doubt on the statistical treatment for that particular quantity.

\IfFileExists{SCRIPT_FIGURES/Scatterplots/validation_histograms.tex}{\input{SCRIPT_FIGURES/Scatterplots/validation_histograms.tex}}{\typeout{Error: Missing file SCRIPT_FIGURES/Scatterplots/validation_histograms.tex}}


\clearpage


\section{Summary of FDS Validation Git Statistics}

Table~\ref{validation_git_stats} shows the Git repository statistics for all of the validation datasets. For each dataset, the corresponding last changed date and Git revision string are shown. This indicates the Git revision string and date for which the most recent validation results for a given dataset were committed to the repository.

\IfFileExists{SCRIPT_FIGURES/Scatterplots/validation_git_stats.tex}{\input{SCRIPT_FIGURES/Scatterplots/validation_git_stats.tex}}{\typeout{Error: Missing file SCRIPT_FIGURES/Scatterplots/validation_git_stats.tex}}


\printbibliography[heading=bibintoc,title={References}]

\end{document}
