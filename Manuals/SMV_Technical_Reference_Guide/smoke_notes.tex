\section{Radiance Equation}
\newcommand{\siga}{ \sigma_a(x) }
\newcommand{\sigt}{ \sigma_t(x) }
\newcommand{\sigs}{ \sigma_s(x) }
\newcommand{\sigts}{ \sigma_t(s) }
\newcommand{\Le}{ L_e(x) }
\newcommand{\Lexo}{ L_e(x,\omega) }
\newcommand{\Lxo}{ L(x,\omega) }
\newcommand{\dLdx}{ \frac{dL}{dx}(x)}
\newcommand{\intf}[2]{ \exp\left({\int_#1^#2 \sigts ds}\right) }
\newcommand{\intff}[2]{ {\int_#1^#2 \sigts ds} }
\newcommand{\intmf}[2]{ \exp\left({-\int_#1^#2 \sigts ds}\right) }
\newcommand{\intmff}[2]{ {-\int_#1^#2 \sigts ds} }

The change in radiance along a ray with direction $\omega$ may be
expressed by the radiation transfer equation given by

\begin{eqnarray}
\label{eq:fullrte}
 \left(\omega\cdot\nabla\right)\Lxo =
-\underbrace{\siga\Lxo}_\mathrm{absorption}-\underbrace{\sigs\Lxo}_\mathrm{out-scattering}
+ \underbrace{\siga\Lexo}_\mathrm{emission} +
\underbrace{\sigs\int_{4\pi}p(x,\omega,\omega')L_i(x,\omega')d\omega'}_\mathrm{in-scattering}
\end{eqnarray}

\noindent where, as illustrated in Figure \ref{figRadiance},
$\Lxo$ is the radiance at $x$ along a direction $\omega$, $\siga$
is the absorption coefficient, $\sigs$ is the scattering
coefficient, $\Lexo$ is the radiance emitted at $x$ in direction
$\omega$ and $p(x,\omega,\omega')$ is the fraction of light moving
along direction $\omega'$ scattered along direction $\omega$.
Absorption, and out-scattering causes radiance to decrease while
emission and in-scattering causes radiance to increase.
\begin{figure}[\figoptions]
\begin{center}
\includegraphics[width=6.0in]{figures/rte_setup}
\end{center}
\caption{xxx.} \label{figRadiance}
\end{figure}

Equation (\ref{eq:fullrte}) may be simplified by neglecting the
in-scattering and emission terms and combining the absorption and
out-scattering terms using $\sigt=\siga+\sigs$ to obtain

\begin{eqnarray*}
\dLdx&=&-\sigt L(x)\\
L(0)&=&L_0
\end{eqnarray*}
which has solution (known as Beer's law)
\begin{eqnarray*}
\frac{L(D)}{L_0}&=&\intmf{0}{D}
\end{eqnarray*}

Alternatively, if the emission term is also included when
simplifying equation (\ref{eq:fullrte})  the following
differential equation results.

\begin{eqnarray}
\label{eq:simple_rte}
\dLdx&=&-\sigt L(x) + \siga L_e(x)\\
 L(0)&=&L_0\nonumber
\end{eqnarray}

This equation may be solved by rearranging terms and applying the
integrating factor $\exp(\int_0^x \sigts ds)$ to both sides to
obtain
\begin{eqnarray*}
\intf{0}{x}\left(\dLdx+\sigt L(x)\right)&=&  \intf{0}{x}\siga \Le\\
\frac{d}{dx}\left(\intf{0}{x} L(x)\right)&=& \intf{0}{x}\siga \Le
\end{eqnarray*}
Integrating both sides and substituting the integration limits
results in
\begin{eqnarray*}
\left.\intf{0}{x} L(x)\right|_0^D&=& \int_0^D\intf{0}{x}\siga \Le dx \\
\intf{0}{D} L(D)-L_0&=& \int_0^D\intf{0}{x}\siga \Le dx
\end{eqnarray*}
Solving for $L(D)$ after noting that
$\intff{0}{x}-\intff{0}{D}=-\intff{x}{D}$ results in
\begin{eqnarray*}
L(D)&=&\intmf{0}{D} L_0+ \int_0^D\intmf{x}{D}\siga \Le dx\\
\end{eqnarray*}
The radiance at $D$ is then given by
\begin{equation}
 L(D)=\tau(0)L_0 + \int_0^D\tau(x)\siga\Le dx
\end{equation}
where $\tau(x)=\intmf{x}{D}$ .
