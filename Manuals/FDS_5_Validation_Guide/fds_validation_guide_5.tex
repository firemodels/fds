\documentclass[11pt]{book}
\usepackage{mathptm,times}
\usepackage[pdftex]{graphicx}
\usepackage{hyperref}

%\usepackage{eso-pic}
%\usepackage{graphicx}
%\usepackage{color}
%\usepackage{type1cm}

%\makeatletter
%   \AddToShipoutPicture{%
%     \setlength{\@tempdimb}{.5\paperwidth}%
%    \setlength{\@tempdimc}{.5\paperheight}%
%   \setlength{\unitlength}{1pt}%
%  \put(\strip@pt\@tempdimb,\strip@pt\@tempdimc){%
%     \makebox(0,0){\rotatebox{45}{\textcolor[gray]{0.75}{\fontsize{8cm}{8cm}\selectfont{DRAFT}}}}}}
%\makeatother

\setlength{\textwidth}{6.5in}
\setlength{\textheight}{9.0in}
\setlength{\topmargin}{0.in}
\setlength{\headheight}{0.in}
\setlength{\headsep}{0.in}
\setlength{\parindent}{0.25in}
\setlength{\oddsidemargin}{0.0in}
\setlength{\evensidemargin}{0.0in}

\begin{document}

\bibliographystyle{unsrt}

\newcommand{\dod}[2]{\frac{\partial #1}{\partial #2}}
\newcommand{\DoD}[2]{\frac{D #1}{D #2}}
\newcommand{\dsods}[2]{\frac{\partial^2 #1}{\partial #2^2}}
\newcommand{\dx}{\delta x}
\newcommand{\dy}{\delta y}
\newcommand{\dz}{\delta z}
\newcommand{\x}{x}
\newcommand{\y}{y}
\newcommand{\z}{z}
\newcommand{\dt}{\delta t}
\newcommand{\dn}{\delta n}
\newcommand{\cH}{{\cal H}}
\newcommand{\hu}{u}
\newcommand{\hv}{v}
\newcommand{\hw}{w}
\newcommand{\la}{\lambda}
%\newcommand{\bO}{\mbox{\boldmath $\Omega$}}
\newcommand{\bO}{{\Omega}}
\newcommand{\bo}{{\bf \omega}}
%\newcommand{\btau}{\mbox{\boldmath $\tau$}}
\newcommand{\btau}{{\bf \tau}}
\newcommand{\bdelta}{{\bf \delta}}
\newcommand{\sumym}{\sum (Y_i/W_i)}
\newcommand{\oW}{\overline{W}}
\newcommand{\om}{\omega}
\newcommand{\omx}{\omega_x}
\newcommand{\omy}{\omega_y}
\newcommand{\omz}{\omega_z}
\newcommand{\erf}{\hbox{erf}}
\newcommand{\bF}{{\bf F}}
\newcommand{\bof}{{\bf f}}
\newcommand{\bq}{{\bf q}}
\newcommand{\br}{{\bf r}}
\newcommand{\bu}{{\bf u}}
\newcommand{\bx}{{\bf x}}
\newcommand{\bk}{{\bf k}}
\newcommand{\bv}{{\bf v}}
\newcommand{\bg}{{\bf g}}
\newcommand{\bn}{{\bf n}}
\newcommand{\bS}{{\bf S}}
\newcommand{\dS}{d{\bf S}}
\newcommand{\bs}{{\bf s}}
\newcommand{\bI}{{\bf I}}
\newcommand{\hp}{{\cal H}}
\newcommand{\trho}{\tilde{\rho}}
\newcommand{\dph}{{\delta\phi}}
\newcommand{\dth}{{\delta\theta}}
\newcommand{\tp}{\tilde{p}}
\newcommand{\dQ}{\dot{Q}}
\newcommand{\dq}{\dot{q}}
\newcommand{\dm}{\dot{m}}
\newcommand{\ha}{\frac{1}{2}}
\newcommand{\ft}{\frac{4}{3}}
\newcommand{\ot}{\frac{1}{3}}
\newcommand{\fofi}{\frac{4}{5}}
\newcommand{\of}{\frac{1}{4}}
\newcommand{\twth}{\frac{2}{3}}
\newcommand{\R}{{\cal R}}
\newcommand{\be}{\begin{equation}}
\newcommand{\ee}{\end{equation}}
\newcommand{\RE}{\hbox{Re}}
\newcommand{\LE}{\hbox{Le}}
\newcommand{\PR}{\hbox{Pr}}
\newcommand{\PE}{\hbox{Pe}}
\newcommand{\NU}{\hbox{Nu}}
\newcommand{\SC}{\hbox{Sc}}
\newcommand{\SH}{\hbox{Sh}}
\newcommand{\WE}{\hbox{We}}
\newcommand{\COTWO}{{\tiny \hbox{CO}_2}}
\newcommand{\OTWO}{{\tiny \hbox{O}_2}}
\newcommand{\CO}{{\tiny \hbox{CO}}}
\newcommand{\F}{{\tiny \hbox{F}}}

\pagestyle{empty}

\begin{minipage}[t][9in][s]{6.5in}

\huge
\flushright{NIST Special Publication XXXX}

\vspace{1in}

\Huge
\flushright{Fire Dynamics Simulator (Version 5) \\ Verification \&
Validation Guide \\ \LARGE Volume 2: Validation Guide } 

\vspace{.5in}

\large
\flushright{
Kevin McGrattan \\
Anthony Hamins \\
Simo Hostikka \\
Jason Floyd \\
Bryan Klein }

\vfill

\includegraphics[width=\textwidth]{FIGURES/nistlogo_1line}

\end{minipage}

\newpage

\hspace{5in}

\newpage

\begin{minipage}[t][9in][s]{6.5in}

\huge
\flushright{NIST Special Publication XXXX}

\vspace{1.in}

\Huge
\flushright{Fire Dynamics Simulator (Version 5) \\ Verification \&
Validation Guide \\ \LARGE Volume 2: Validation Guide } 

\vspace{.5in}

\normalsize
\flushright{
%\includegraphics[width=1in]{FIGURES/bfrl}  \\
Kevin McGrattan \\
Anthony Hamins \\
Bryan Klein \\
{\em Fire Research Division} \\
{\em Building and Fire Research Laboratory}  \\
%\includegraphics[width=1in]{FIGURES/VTT_GREY_L}  \\
Simo Hostikka \\
{\em VTT Technical Research Centre of Finland}\\
Espoo, Finland \\
Jason Floyd \\
{\em Hughes Associates, Inc. }  \\
Baltimore, Maryland, USA }

\vspace{.25in}

\flushright{September 2006}

\vfill

\flushright{\includegraphics[width=1in]{FIGURES/doc} }

\small
\flushright{U.S. Department of Commerce \\
{\em Donald L. Evans, Secretary} \\
\hspace{1in} \\
Technology Administration \\
{\em Phillip J.~Bond, Under Secretary for Technology}  \\
\hspace{1in} \\
National Institute of Standards and Technology \\
{\em Arden L. Bement, Jr., Director} }


\end{minipage}

\newpage

\begin{minipage}[t][9in][s]{6.5in}

\flushright{Certain commercial entities, equipment, or materials may be identified in this \\
document in order to describe an experimental procedure or concept adequately. Such \\
identification is not intended to imply recommendation or endorsement by the \\
National Institute of Standards and Technology, nor is it intended to imply that the \\
entities, materials, or equipment are necessarily the best available for the purpose.
}

\vspace{3in}

\large
\flushright{\bf National Institute of Standards and Technology Special Publication XXXX \\
Natl.~Inst.~Stand.~Technol.~Spec.~Publ.~XXXX, 94 pages (March 2007) \\
CODEN: NSPUE2 }

\vfill

\flushright{U.S. GOVERNMENT PRINTING OFFICE \\
WASHINGTON: 2004 \\
\rule{3.5in}{0.01in} \\
For sale by the Superintendent of Documents, U.S. Government Printing Office \\
Internet: bookstore.gpo.gov -- Phone: (202) 512-1800 -- Fax: (202) 512-2250 \\
Mail: Stop SSOP, Washington, DC 20402-0001 }

\end{minipage}



\newpage

\frontmatter

\pagestyle{plain}


\chapter{Preface}


This guide is based in part on the ``Standard Guide for
Evaluating the Predictive Capability of Deterministic Fire Models,'' ASTM~E~1355~\cite{ASTM:E1355}.
ASTM~E~1355 defines {\em model evaluation} as ``the process of quantifying
the accuracy of chosen results from a model when applied for a specific use.''
The model evaluation process consists of two main components: verification and validation.
{\em Verification} is a process to check the correctness of the solution of the
governing equations. Verification does not imply that the governing equations are
appropriate; only that the equations are being solved correctly.
{\em Validation} is a process to determine the appropriateness of the governing equations as a mathematical
model of the physical phenomena of interest. Typically, validation involves comparing
model results with experimental measurement. Differences that cannot be explained in terms of
numerical errors in the model or uncertainty in the measurements
are attributed to the assumptions and simplifications of the physical model.

Evaluation is critical to establishing both the acceptable uses
and limitations of a model. Throughout its development, FDS has undergone various forms of evaluation,
both at NIST and beyond. This guide provides a survey of work conducted to date to evaluate FDS.



\chapter{Disclaimer}

The US Department of Commerce makes no warranty, expressed or implied,
to users of the Fire Dynamics Simulator (FDS), and accepts no responsibility for its use.
Users of FDS assume sole responsibility under Federal law for determining
the appropriateness of its use in any particular application;
for any conclusions drawn from the results of its use; and for any
actions taken or not taken as a result of analysis performed using
these tools.

Users are warned that FDS is intended for use only by those competent
in the fields of fluid dynamics, thermodynamics, heat transfer,
combustion, and fire science, and is intended only to supplement the
informed judgment of the qualified user. The software package is a
computer model that may or may not have predictive capability when
applied to a specific set of factual circumstances. Lack of accurate
predictions by the model could lead to erroneous conclusions with
regard to fire safety. All results should be evaluated by an informed
user.

Throughout this document, the mention of computer hardware or commercial
software does not constitute endorsement by NIST, nor does it indicate that
the products are necessarily those best suited for the intended purpose.


\chapter{Acknowledgments}

\label{acksection}

Support for validation experiments and the preparation of the FDS
manuals has been provided by the Office of Nuclear Regulatory Research
of the US Nuclear Regulatory Commission (US NRC). Special thanks to
Mark Salley and Jason Dreisbach for their efforts.

Alex Maranghides oversees the Large Fire Laboratory at NIST where
these tests were conducted, and has helped to design the
experiments. Thanks also to technicians Laurean Delauter, Jay McElroy,
and Jack Lee who constructed and conducted the validation experiments
at NIST.

Rick Peacock of NIST assisted in the interpretation of results from
the ``NBS Multi-Room Test Series,'' a set of three room
fire experiments conducted at the National Bureau of Standards (now
NIST) in the mid-1980's.

Thanks to VTT, Finland, for their contribution of experimental data,
referred to in this document as the ``VTT Large Hall Experiments.''

At the University of Maryland, Professor Fred~Mowrer and Phil Friday
were the first to apply FDS to the NRC-sponsored experiments referred
to in this document as the ``FM/SNL Test Series'' (Factory Mutual and
Sandia National Laboratories conducted these experiments).

Also at NIST, Dan Madrzykowski, Doug Walton, Bob Vettori, Dave Stroup,
Steve Kerber, and Nelson Bryner have applied FDS to the reconstruction
of several real fire incidents that led to the death of fire fighters.

At VTT, Finland, Jukka Hietaniemi, Jukka Vaari and Timo Korhonen have
performed validation studies of various sub-models. The VTT Fire
Research group continues to work on model development and
validation of the model for various applications.

At the University of Canterbury, Christchurch, New Zealand,
Prof.~Charles Fleischmann and his students have provided an extensive
set of material properties of upholstered furnishings, plus performed
validation studies. Doctoral student Jason Clement spent a semester at
NIST validating the hydrodynamics of FDS against salt water data.

Several postdoctoral fellows at NIST have made contributions to model
development. Francine Battaglia studied the extension of the model to
look at fire whirls, Javier Trelles extended the large eddy simulation
technique to large, wind-blown fire plumes, Amy Musser considered
indoor air quality applications along with her advisor Steve Emmerich
of the Building Environment Division of BFRL.




\tableofcontents

\mainmatter








\chapter{Overview}

Because FDS is used both  for research and for practical applications,
there  are  some  routines  within   the  model  that  have  not  been
comprehensively validated. Even  those parts of the model  for which a
substantial  amount of  validation  work has  been  performed will  be
pushed beyond these limits by  the need to perform increasingly comlex
analyses.  Consequently,  it is  the  responsibility  of  the user  to
demonstrate the applicability of the model for scenarios that have not
yet been validated.

FDS  is  suited  for  a  wide range  of  thermally-driven  fluid  flow
scenarios,  including fire, both  in the  open ({\em  e.g.} unconfined
fire plumes) as  well as within the built  environment. To date, about
half of  the applications of  FDS have been  for design, and  half for
forensic reconstruction.

Design  applications  typically  involve  an existing  building  or  a
building  under  design. A  so-called  ``design  fire'' is  prescribed
either by  a regulatory authority  or by the engineers  performing the
analysis. Because the  fire's heat release rate is  known, the role of
the model is to predict  the transport of heat and combustion products
throughout  the room or  rooms of  interest. Ventilation  equipment is
often included  in the simulation, like fans,  blowers, exhaust hoods,
HVAC ducts,  smoke management systems,  {\em etc.} Sprinkler  and heat
and smoke detector activation are also of interest.  The effect of the
sprinkler spray on the fire is usually less of interest since the fire
is  prescribed rather  than  predicted. Detailed  descriptions of  the
contents of the building are usually not necessary because these items
are not assumed to be burning,  and even if they are, the burning rate
will be  fixed, not predicted.  Sometimes, it is necessary  to predict
the heat  flux from the fire  to a nearby ``target,''  and even though
the target  may heat up  to some prescribed ignition  temperature, the
subsequent spread  of the  fire usually goes  beyond the scope  of the
analysis because of the uncertainty  inherent in object to object fire
spread.

Forensic reconstructions require the  model to simulate an actual fire
based on  information that is collected  after the event,  such as eye
witness accounts, unburned materials,  burn signatures, {\em etc.} The
purpose  of  the simulation  is  to  connect  a sequence  of  discrete
observations  with  a continuous  description  of  the fire  dynamics.
Usually,  reconstructions  involve  more gas/solid  phase  interaction
because  virtually  all  objects  in  a  given  room  are  potentially
ignitable, especially when flashover  occurs. Thus, there is much more
emphasis on  such phenomena as  heat transfer to  surfaces, pyrolysis,
flame  spread, and suppression.  In general,  forensic reconstructions
are more challenging simulations  to perform because they require more
detailed  information  about the  room  contents,  and  there is  much
greater uncertainty in the total heat release rate as the fire spreads
from object to object.

Validation  studies  of FDS  to  date  have  focussed more  on  design
applications   than  reconstructions.  The   reason  is   that  design
applications usually involve prescribed  fires and demand a minimum of
thermophysical properties  of real materials.  Transport  of smoke and
heat  is  the  primary  focus,  and measurements  can  be  limited  to
well-placed thermocouples, a few  heat flux gauges, gas samplers, {\em
etc.} Phenomena of importance in forensic reconstructions, like second
item  ignition, flame  spread, vitiation  effects and  extinction, are
more   difficult  to   model  and   more  difficult   to   study  with
well-controlled experiments. Uncertainties  in material properties and
measurements, as  well as simplifying assumptions in  the model, often
force the  comparison between model and measurement  to be qualitative
at best.  Nevertheless, current validation  efforts are moving  in the
direction of these more difficult issues.




\subsection{Model Accuracy}

The degree of  accuracy for each output variable  required by the user
is  highly  dependent on  the  technical  issues  associated with  the
analysis.  The user  must ask: How accurate does  the analysis have to
be  to  answer  the  technical  question posed?  Thus,  a  generalized
definition of the  accuracy required for each quantity  with no regard
as  to the specifics  of a  particular analysis  is not  practical and
would be limited in its usefulness.

Returning   to    the   earlier   definitions    of   ``design''   and
``reconstruction,''  design applications  typically are  more accurate
because the heat release rate is prescribed rather than predicted, and
the    initial    and    boundary    conditions   are    far    better
characterized. Mathematically, a design calculation is an example of a
``well-posed''  problem  in  which   the  solution  of  the  governing
equations is  advanced in  time starting from  a known set  of initial
conditions and constrained by a known set of boundary conditions.  The
accuracy of the results is a function of the fidelity of the numerical
solution, which is  mainly dependent on the size  of the computational
grid. The FDS Validation Guide~\cite{FDS_Validation_Guide_5} describes
efforts to date involving well-characterized geometries and prescribed
fires. These studies show that  FDS predictions vary from being within
experimental   uncertainty  to  being   about  20~\%   different  than
measurements of temperature, heat flux, gas concentration, {\em etc}.

A reconstruction is an example of an ``ill-posed'' problem because the
outcome  is known  whereas  the initial  and  boundary conditions  are
not. There is  no single, unique solution to the  problem, that is, it
is possible to simulate numerous fires that produce the given outcome.
There is no right or wrong answer, but rather a small set of plausible
fire scenarios that are  consistent with the collected evidence. These
simulations are then used to demonstrate to fire service personnel why
the fire behaved as it did  based on the current understanding of fire
physics  incorporated in  the model.  Most  often, the  result of  the
analysis is only  qualitative. If there is any  quantification at all,
it could be in the time to reach critical events, like a roof collapse
or room flashover.


\chapter{Survey of Past Validation Work}


In this  chapter, a survey of  FDS validation work  will be presented.
Some of the work has been  performed at NIST, some by its grantees and
some by  engineering firms using the model.  Because each organization
has its  own reasons for  validating the model, the  referenced papers
and reports do not follow any particular guidelines. Some of the works
only provide  a qualitative assessment  of the model,  concluding that
the  model  agreement with  a  particular  experiment  is ``good''  or
``reasonable.'' Sometimes, the conclusion is that the model works well
in certain cases, not as well in others. These studies are included in
the survey because the references  are useful to other model users who
may have a similar application  and are interested in even qualitative
assessment. It is important to note  that some of the papers point out
flaws in early releases of FDS that have been corrected or improved in
more recent  releases. Some of  the issues raised, however,  are still
subjects of  active research. The  research agenda for FDS  is greatly
influenced  by   the  feedback   provided  by  users,   often  through
publication of validation efforts.


\section{Validation Work with Pre-Release Versions of FDS}

FDS was officially released in  2000. However, for two decades various
CFD codes using the basic FDS hydrodynamic framework were developed at
NIST for  different applications and  for research. In the  mid 1990s,
many of  these different codes were consolidated  into what eventually
became FDS.  Before FDS, the various  models were referred  to as LES,
NIST-LES, LES3D,  IFS (Industrial Fire Simulator), and  ALOFT (A Large
Outdoor Fire Plume Trajectory).

The  NIST LES model  describes the  transport of  smoke and  hot gases
during  a fire  in an  enclosure using  the  Boussinesq approximation,
where it is assumed that the density and temperature variations in the
flow                           are                          relatively
small~\cite{Rehm:1,Rehm:SIAM83,Rehm:ANM85,Rehm:IAFSS3}.     Such    an
approximation  can be  applied  to a  fire  plume away  from the  fire
itself.   Much of  the early  work  with this  form of  the model  was
devoted  to  the  formulation of  the  low  Mach  number form  of  the
Navier-Stokes  equations and  the development  of the  basic numerical
algorithm.   Early validation  efforts  compared the  model with  salt
water   experiments~\cite{Baum:1,McGrattan:1,Rehm:IAFSS5},   and  fire
plumes~\cite{Baum:IAFSS5,Baum:2,Baum:3,Baum:4}.  Clement validated the
hydrodynamic model  in FDS by  measuring salt water flows  using Laser
Induced   dye   Fluorescence~(LIF)~\cite{Clement:1}.  An   interesting
finding  of this  work was  that the  transition from  a laminar  to a
turbulent plume is very difficult  to predict with any technique other
than DNS.

Eventually, the  Boussinesq approximation was  dropped and simulations
began  to   include  more  fire-specific   phenomena.  Simulations  of
enclosure   fires   were   compared   to  experiments   performed   by
Steckler~\cite{McGrattan:4}.  Mell~{\em  et al.}~\cite{Mell:1} studied
small helium  plumes, with particular attention to  the relative roles
of  baroclinic torque and  buoyancy as  sources of  vorticity.  Cleary
{\em  et  al.}~\cite{LES:6}  used   the  LES  model  to  simulate  the
environment  seen by  multi-sensor fire  detectors and  performed some
simple validation work to check the model before using it.  Large fire
experiments were performed by NIST  at the FRI test facility in Japan,
and at US Naval aircraft hangars in Hawaii and Iceland~\cite{Davis:1}.
Room   airflow   applications   were   considered  by   Emmerich   and
McGrattan~\cite{Emmerich:1,Emmerich:2}.

These early validation efforts were encouraging, but still pointed out
the  need  to  improve  the  hydrodynamic  model  by  introducing  the
Smagorinsky form of large eddy simulation.  This addition improved the
stability  of the  model  because of  the  relatively simple  relation
between the  local strain rate  and the turbulent viscosity.  There is
both   a   physical  and   numerical   benefit   to  the   Smagorinsky
model. Physically,  the viscous term used  in the model  has the right
functional  form to describe  sub-grid mixing  processes. Numerically,
local oscillations in the computed  flow quantities are damped if they
become  large   enough  to  threaten  the  stability   of  the  entire
calculation.

During the 1980s and 1990s,  the Building and Fire Research Laboratory
at NIST studied the burning of  crude oil under the sponsorship of the
US Minerals Management Service.  The aim of the work was to assess the
feasibility of using burning as a means to remove spilled oil from the
sea surface. As part of the  effort, Rehm and Baum developed a special
application of the LES model called ALOFT. The model was a spin-off of
the two-dimensional LES enclosure  model, in which a three-dimensional
steady-state plume was computed  as a two-dimensional evolution of the
lateral wind field  generated by a large fire blown  in a steady wind.
The ALOFT model is based on  large eddy simulation in that it attempts
to resolve the relevant scales of a large, bent-over plume. Validation
work  was  performed  by  simulating  the plumes  from  several  large
experimental burns of crude oil in which aerial and ground sampling of
smoke       particulate       was       performed~\cite{McGrattan:4a}.
Yamada~\cite{ALOFT:2} performed  a validation  of the ALOFT  model for
10~m oil  tank fire. The results  indicate that the  prediction of the
plume  cross  section  500~m  from   the  fire  agree  well  with  the
experimental observations.




\section{Validation of FDS since 2000}

There is an  on-going effort at NIST and elsewhere  to validate FDS as
new capabilities are  added. To date, most of  the validation work has
evaluated the  model's ability  to predict the  transport of  heat and
exhaust products from  a fire through an enclosure.  In these studies,
the heat release rate is usually prescribed, along with the production
rates of  various products  of combustion.  More  recently, validation
efforts  have moved  beyond  just transport  issues  to consider  fire
growth, flame spread,  suppression, sprinkler/detector activation, and
other fire-specific phenomena.

The  validation work  discussed below  can be  organized  into several
categories: Comparisons with full-scale tests conducted especially for
the   chosen  evaluation,   comparisons   with  previously   published
full-scale  test data,  comparisons with  standard  tests, comparisons
with  documented  fire experience,  and  comparisons with  engineering
correlations.  There is no single  method by which the predictions and
measurements are compared.   Formal, rigorous validation exercises are
time-consuming  and  expensive.  Most  validation exercises  are  done
simply to assess if the model can be used for a very specific purpose.
While  not  comprehensive on  their  own,  these studies  collectively
constitute a valuable assessment of the model.


\subsection{Comparison with Full-Scale Tests Conducted Specifically for the Chosen Evaluation}

As part of the NIST investigation  of the World Trade Center fires and
collapse,  a series  of large  scale fire  experiments  were performed
specifically  to  validate  FDS~\cite{Hamins:WTC1}.   The  tests  were
performed in  a rectangular  compartment 7.2~m long  by 3.6~m  wide by
3.8~m tall.   The fires were  fueled by heptane  for some tests  and a
heptane/toluene mixture  for the others.  The fuel was sprayed  from a
nozzle into a steel pan. The compartment was heavily instrumented.

FDS  simulations were  performed  before testing  began  to guide  the
design  of the  compartment  and also  to  provide a  baseline set  of
"blind" predictions\footnote{A blind  prediction simply means that the
calculation is performed before  the experiment is conducted. At NIST,
the  results  of  a  blind  prediction  are  given  to  the  engineers
conducting  the experiment  so  that  they can  be  compared with  the
measurements immediately after the experiment is completed. This is an
effective  test the  model, and  also a  good way  to ensure  that the
measurements make  sense.}.  A uniform numerical grid  whose cells are
10 cm on a side was chosen based on a grid resolution study.  The heat
release rate (HRR)  of the simulated burner was set  to that which was
measured in  the experiments. No attempt  was made to  model the spray
burner.

To  quantify  the  accuracy  of  the  predictions,  the  measured  and
predicted gas temperatures  of the upper layer were  compared for each
test. Because the  uncertainty of the measured HRR  is often the major
source of  disagreement between model and experiment,  an analysis was
performed to  assess how sensitive  the measured temperatures  were to
changes  in  the  HRR.   According  to  an  empirical  correlation  by
McCaffrey,  Quintiere and  Harkleroad~\cite{SFPE:Walton}, the  rise in
the  upper layer  gas temperature  $\Delta  T_g$ in  a compartment  is
related to  the overall HRR by the  relation \be \Delta T_g  = 6.85 \;
\left(    \frac{\dQ^2}{A_0   \,    \sqrt{H_0}   \,    h_k    \,   A_T}
\right)^\frac{1}{3} \label{MQH} \ee where $\dQ$ is the HRR (kW), $A_0$
is the area of the opening (m$^2$), $H_0$ is the height of the opening
(m),  $h_k$ is  the thermal  conductivity of  the walls  (kW/m/K), and
$A_T$ is the total area of  the compartment walls (m$^2$).  What is of
importance here is the fact  that the temperature rise is proportional
to the  HRR to  the 2/3  power. The reported  uncertainty in  the heat
release rate  measurement was 5~\% (one standard  deviation). The 5~\%
uncertainty  in  the  HRR  corresponds  to  a  2/3  x  5~\%  =  3.3~\%
uncertainty in the temperature  rise.  For upper layer temperatures of
approximately 600~$^\circ$C, this  translates to roughly 20~$^\circ$C.
The  difference in  measured  and predicted  upper layer  temperatures
ranged from 5~$^\circ$C to  20~$^\circ$C, meaning that the predictions
were within the  uncertainty range of the HRR.   Even though there are
uncertainties  in  the  measurement  of the  temperature  itself,  the
discrepancy between measurement and prediction can be explained solely
in terms of the uncertainty in the HRR measurement.

To extend  this analysis further, it  was observed that  the heat flux
onto surfaces in the upper layer was very nearly given by $\sigma T^4$
where  $\sigma=5.67  \times 10^{-11}$~kW/m$^2$/K$^4$  and  $T$ is  the
temperature in degrees K. In  other words, the emissivity of the upper
layer  gases is  nearly  1, not  surprising  given the  high level  of
soot. Given  the uncertainty  in the upper  layer temperature  rise of
3.3~\%, the  uncertainty in  absolute temperature at  600~$^\circ$C is
2.3~\%, leading  to an  estimate for the  uncertainty in heat  flux of
4~$\times$~2.3~\% = 9.2~\%.  In most instances, the difference between
measurement and prediction was within 10~\%, confirming that the model
is within experimental uncertainty.

In the  discussion above,  it was shown  that the FDS  predicted upper
layer temperatures  and heat fluxes were within  the uncertainty range
of  the  experiment.  For  this  exercise  it  was  assumed  that  the
uncertainty in the  experiment was based solely on  the uncertainty of
the heat  release rate measurement.  Thus, it was shown  how sensitive
the upper  layer temperature  and heat flux  measurements were  to the
heat release  rate. In the  numerical simulations there are  dozens of
input parameters prescribed by the  model user.  Often there is no way
to  assess the  sensitivity of  these parameters  except  by numerical
experiment; that is, running the  model with small changes to the base
parameters to see  what effect these have on  the predictions. To this
end, the  data set was exploited  to assess the sensitivity  of FDS to
grid cell  size, wall thermal  properties, fire soot  yield, radiation
fraction,  and  various other  quantities.  Details  can  be found  in
Reference~\cite{Hamins:WTC1}.

A second set of experiments to validate FDS for use in the World Trade
Center  investigation is  documented  in Ref.~\cite{Hamins:WTC2}.  The
intent of  these tests  was to  evaluate the ability  of the  model to
simulate the growth  of a fire burning 3  office workstations within a
compartment of dimensions 11~m by 7~m by 4~m, open at one end to mimic
the ventilation  of 5  windows similar to  those in  WTC 1 and  2. Six
tests  were performed  with various  initial conditions  exploring the
effect of jet fuel spray and ceiling tiles covering the surface of the
desks and carpet. Measurements were  made of the heat release rate and
compartment  gas   temperatures  at  four   locations  using  vertical
thermocouple arrays. Six different material samples were tested in the
NIST  cone calorimeter:  desk,  chair, paper,  computer case,  privacy
panel, and  carpet. Data for the  carpet, desk and  privacy panel were
input directly into FDS, with the other 3 materials lumped together to
form an  idealized fuel type.  Open burns of single  workstations were
used to  calibrate the simplified fuel  package. Then FDS  was used to
make  blind  predictions  of   the  3  workstation  fires  within  the
compartment. Peak  heat release rates and  temperatures were predicted
to within 20~\% for all tests.


\subsection{Comparison with Engineering Correlations}

There are  several examples of  fire flows that have  been extensively
studied, so much  so that a set of  engineering correlations combining
the   results  of   many  experiments   have  been   developed.  These
correlations are  useful to modelers because of  their simplicity. The
most studied  phenomena include fire  plumes, ceiling jets,  and flame
heights.

Although much  of the  early validation work  before FDS  was released
involved fire plumes, it remains an active area of interest. One study
by  Chow  and Yin~\cite{Chow:1}  surveys  the  performance of  various
models in predicting plume temperatures and entrainment.  They compare
various correlations, a  RANS (Reynolds-Averaged Navier-Stokes) model,
and FDS.  Simulations  were carried out that replicated  a 470~kW fire
with a  diameter of  1~m and an  unbounded ceiling.  A  numerical grid
size of  96 by  96 by  96 cells was  used in  the FDS  calculation and
provided results  that agreed  well with those  predicted by  the RANS
model and the various correlations.

Battaglia~{\em  et al.}~\cite{Battaglia:1} used  FDS to  simulate fire
whirls.   First,  the  model  was  shown to  reproduce  the  McCaffrey
correlation  of  a  fire  plume,   then  it  was  shown  to  reproduce
qualitatively certain features  of fire whirls. At the  time, FDS used
Lagrangian elements to introduce heat  from the fire (no longer used),
and this  combustion model could not replicate  the extreme stretching
of the core of the flame zone.

Quintiere and Ma~\cite{Ma:2,Ma:3} compared predicted flame heights and
plume  centerline temperatures to  empirical correlations.   For plume
temperature,   the  Heskestad   correlation~\cite{SFPE:Heskestad}  was
chosen.  Favorable  agreement was found  in the plume region,  but the
results  near  the  flame  region  were found  to  be  grid-dependent,
especially for  low $Q^*$  fires.  At this  same time,  researchers at
NIST were  reaching similar  conclusions, and it  was noticed  by both
teams  that a  critical parameter  for the  model is  $D^*/\dx$, where
$D^*$ is the  characteristic fire diameter and $\dx$  is the grid cell
size.  If  this parameter  is  sufficiently  large,  the fire  can  be
considered  well-resolved  and  agreement  with various  flame  height
correlations was found. If the parameter is not large enough, the fire
is not  well-resolved and adjustments  must be made to  the combustion
routine to account for it.



\subsection{Comparisons with Previously Published Full-Scale Test Data}
\label{prevpub}

Experiments  conducted  solely   for  model  validation  are  somewhat
rare.  More common  are validation  studies  that use  data from  past
experiments.  This   section  contains  brief   descriptions  of  work
published  comparing  FDS with  past  experiments  or correlations  of
experimental data.

\subsubsection{Pool Fires}


Xin~{\em et  al.}~\cite{Xin:JSS2005} used FDS to model  a 1~m diameter
methane pool  fire.  The  computational domain was  2~m by 2~m  by 4~m
with  a uniform  grid  size of  2.5  cm.  The  predicted results  were
compared  to   experimental  data  and  found   to  qualitatively  and
quantitatively  reproduce   the  velocity  field.   The  same  authors
performed a similar study of a 7.1~cm methane burner~\cite{Xin:CF2005}
and a helium plume~\cite{Xin:CS2002}.

Hostikka~{\em  et al.}~\cite{Hostikka:3} modeled  small pool  fires of
methane and methanol to test  the FDS radiation solver for low-sooting
fires.   They conclude that  the predicted  radiative fluxes  for both
fuels  are  higher than  measured  values,  especially  at small  heat
release rates, due to an over-prediction of the gas temperature.

Hietaniemi,  Hostikka and  Vaari~\cite{Hietaniemi:1}  consider heptane
pool fires of various diameters.  Predictions of the burning rate as a
function  of  diameter  follow  the  trend observed  in  a  number  of
experimental studies.  Their results show an improvement  in the model
over the earlier work with  methanol fires, due to improvements in the
radiation  routine  and the  fact  that  heptane  is more  sooty  than
methanol, simplifying  the treatment of radiation.   The authors point
out  that reliable  predictions of  the burning  rate of  liquid fuels
require roughly twice as fine a grid spanning the burner than would be
necessary to predict plume velocities and temperatures. The reason for
this is  the prediction  of the heat  feedback to the  burning surface
necessary to {\em predict} rather  than to {\em prescribe} the burning
rate.


\subsubsection{Airflows in Non-Fire Compartments}

The low Mach number assumption in FDS is appropriate not only to fire,
but to  most building  ventilation scenarios.  An  example of  how the
model  can  be used  to  assess indoor  air  quality  is presented  by
Musser~{\em  et  al.}~\cite{Musser:1}.   The  test compartment  was  a
displacement   ventilation  test   room   that  contained   computers,
furniture, and  lighting fixtures as well as  heated rectangular boxes
intended to  represent occupants.  A detailed description  of the test
configuration is  given by Yuan~{\em et  al.}~\cite{Yuan:1}.  The room
is ventilated with  cool supply air introduced via  a diffuser that is
mounted on a side  wall near the floor. The air rises  as it is warmed
by heat sources  and exits through a return duct  located in the upper
portion  of  the  room.  The   flow  pattern  is  intended  to  remove
contaminants by sweeping  them upward at the source  and removing them
from the room.  Sulphur  hexafluoride, SF$_6$, was introduced into the
compartment during the  experiment as a tracer gas  near the breathing
zone  of  the   occupants.   Temperature,  tracer  concentration,  and
velocity were  measured during the experiments.   For temperature, the
two finest grids (50 by 36 by  24 and 64 by 45 by 30) produced results
in  which the  agreement between  the measurement  and  prediction was
considered  acceptable.  The agreement  for the  tracer concentrations
were  not as  good.  It  was suggested  that the  difference  could be
related to  the way  the source  of the tracer  gas was  modeled.  The
comparison  of   velocity  data  was  deemed   reasonable,  given  the
limitations of the velocity probes at low velocities.

In another  study, Musser and  Tan~\cite{Musser:2} used FDS  to assess
the  design  of ventilation  systems  for  facilities  in which  train
locomotives  operate.  Although there  is  only  a  limited amount  of
validation, the  study is useful  in demonstrating a practical  use of
FDS for a non-fire scenario.

Mniszewski~\cite{Mniszewski:1}  used  FDS  to  model  the  release  of
flammable gases in simple enclosures and open areas. In this work, the
gases were not ignited.

Kerber  and Walton  provided a  comparison between  FDS version  1 and
experiments on positive pressure ventilation in a full-scale enclosure
without a fire.   The model predictions of velocity  were within 10~\%
to 20~\% of the experimental values~\cite{Kerber:1}.


\subsubsection{Wind Engineering}

Most applications  of FDS involve fires within  buildings. However, it
can be used to model thermal  plumes in the open and wind impinging on
the   exterior   of   a   building.    Rehm,   McGrattan,   Baum   and
Simiu~\cite{LES:4} use the LES solver to estimate surface pressures on
simple rectangular blocks in  a crosswind, and compare these estimates
to experimental measurements.  In a subsequent paper~\cite{Rehm:WS02},
they consider the qualitative  effects of multiple buildings and trees
on a wind field.

A   different    approach   to   wind    is   taken   by    Wang   and
Joulain~\cite{Wang:IAFSS2002}. They  consider a  small fire in  a wind
tunnel  0.4~m wide  and  0.7~m tall  with  flow speeds  of 0.5~m/s  to
2.5~m/s.  Much  of  the  comparison with  experiment  is  qualitative,
including  flame shape,  lean,  length.  They also  use  the model  to
determine  the  predominant  modes  of  heat  transfer  for  different
operating  conditions. To  assess  the combustion,  they implement  an
``Eddy Break-up''  combustion model~\cite{Magnussen:1} and  compare it
to the mixture  fraction approach used by FDS.  The two models perform
better or  worse, depending on  the operating conditions. Some  of the
weaknesses of the mixture fraction model as implemented in FDS version
2 are addressed in subsequent versions. The Eddy Break-up approach has
not been implemented in the official version of FDS.

Chang and Meroney~\cite{ChangJWE2003} compared the results of FDS with
the  commercial CFD  package  FLUENT in  simulating  the transport  of
pollutants   from  steady   point  sources   in  an   idealized  urban
environment.  FLUENT  employs a  variety  of  RANS (Reynolds  Averaged
Navier-Stokes)  closure  methods,   whereas  FDS  employs  large  eddy
simulation (LES).   The results of the numerical  models were compared
with wind tunnel measurements within a 1:50 scale physical model of an
urban street "canyon". The authors conclude:
\begin{quote}
The CFD  programs reproduced the  overall flow fields  observed during
the  measurement   program,  but  it  is   evident  that  steady-state
calculations  are  not  reproducing  the intermittent  nature  of  the
penetration  of elevated flows  down into  the canyons  [areas between
buildings].  This   results  in   situations  where  the   FLUENT  CFD
concentrations overpredict magnitudes along  canyon walls. The FDS CFD
program  is inherently  a time-dependent  calculation; however,  it is
found that wall  magnitudes can be very sensitive  to the rather crude
wall boundary conditions incorporated in the program.
\end{quote}


\subsubsection{Growing Fires}
\label{growingfires}

Vettori~\cite{Vettori:1} modeled two different fire growth rates in an
obstructed ceiling geometry.  The rectangular compartment was 9.2~m by
5.6~m by 2.4~m  with a hollow steel door to  the outside that remained
closed during the tests. An open wooden stairway led to an upper floor
with the same dimensions as the fire compartment below.  Wooden joists
measuring 0.038~m by 0.24~m were spaced at 0.41~m intervals across the
ceiling and  were supported  by a single  steel beam that  spanned the
width of the  room.  A rectangular methane gas  burner measuring 0.7~m
by 1.0~m by 0.31~m was placed  in the corner of the chamber.  Slow and
fast  burning  fires  that  reached  1055~kW  in  600  s  and  150  s,
respectively,  were  monitored.   Four   vertical  arrays  of  Type  K
thermocouples were used to measure temperatures during the tests.  The
FDS model used four grid refinements and piecewise linear grid spacing
for each fire growth rate (slow  and fast). For the fast growing fire,
the predicted  temperatures were within  20~\% of the  measured values
and within  10~\% for the slow  growing fire. In  general, finer grids
produced better agreement.

In a follow-up report,  Vettori~\cite{Vettori:2} extended his study to
include sloped ceilings, with  and without obstructions. He found that
the  difference between  predicted and  measured  sprinkler activation
times varied  between 4~\%  and 26~\% for  all cases studied.  He also
noted that FDS was able to predict the first activation of a sprinkler
twice  as far  from  the fire  as  another; caused  presumably by  the
re-direction of smoke by the beams on the ceiling.

Floyd~\cite{Floyd:5,Floyd:6} validated  FDS by comparing  the modeling
results with  measurements from fire tests at  the Heiss-Dampf Reaktor
(HDR) facility.  The structure was originally the containment building
for a nuclear power reactor  in Germany. The cylindrical structure was
20~m in  diameter and  50~m in height  topped by a  hemispherical dome
10~m  in radius.   The building  was divided  into eight  levels.  The
total  volume of  the building  was approximately  11,000~m$^3$.  From
1984  to 1991, four  fire test  series were  performed within  the HDR
facility.  The T51  test series consisted of eleven  propane gas tests
and three  wood crib  tests.  To avoid  permanently damaging  the test
facility, a special set of  test rooms were constructed, consisting of
a fire  room with a narrow  door, a long corridor  wrapping around the
reactor vessel  shield wall, and  a curtained area centered  beneath a
maintenance  hatch.   The  fire   room  walls  were  lined  with  fire
brick. The doorway and corridor walls had the same construction as the
test chamber. Six gas burners were mounted in the fire room.  The fuel
source was propane gas mixed with  10~\% air fed at a constant rate to
one of the  six burners.  For comparison with the  FDS model, only the
fire room, hallway and curtained  region was input into the model, for
a total of 450,000 grid cells.  The burners were defined within FDS as
separate vents  with a constant  inlet velocity.  Two sets  of burners
were created, the first set at the physical location of the burners as
the source  of fuel and second set  directly above the first  set as a
source for  ambient air.   The data was  presented at  fifteen minutes
into  the fire.   The  results  comparing the  measured  data and  the
predicted data are presented in Table~\ref{T51}.
\begin{table}
\begin{center}
\caption{T51 Test Results \cite{Floyd:5}}
\label{T51}
\vspace{0.1in}
\begin{tabular}{|*{3}{c|}} \hline
Quantity & Experimental Data &  FDS \\ \hline \hline Upper Temperature
$^\circ$C & 730  & 808 \\ \hline  Upper Velocity (m/s) & 4.5  & 2.8 \\
\hline Lower Velocity  (m/s) & 2.3 & 1.1 \\ \hline  Layer Height (m) &
1.0 & 0.9 \\ \hline $O_{2}$ Concentration (\%) & 11.6 & 7.6 \\ \hline
\end{tabular}
\end{center}
\end{table}
The FDS model predicted the  layer height and temperature of the space
to within 10~\% of the experimental values~\cite{Floyd:5}.

FDS predictions of fire growth and smoke movement in large spaces were
presented by  Kashef~\cite{Kashef:1}.  The experiments  were conducted
at the National Research Council  Canada.  The tests were performed in
a compartment with  dimensions of 9~m by 6~m by  5.5~m with 32 exhaust
inlets and a  single supply fan.  A burner  generated fires ranging in
size  from 15~kW  to 1000~kW.   FDS produced  good predictions  of the
experimental layer  temperatures and interface heights,  but there was
some disagreement in the shape of the temperature profiles.


\subsubsection{Flame Spread}
\label{flame spread}

FDS was  evaluated to predict  the heat transfer  to the wall  from an
adjacent pool fire.   The experimental results were based  on the work
by Beck  {\em et al.}  The  predicted heat flux was  in agreement with
the experimental  results.  The temperatures  are within 30~\%  of the
measured values  near the base of  the wall but  decrease more rapidly
than  the  experimental  measurements.   The  difference  between  the
experimental and predicted values  can be attributed to the combustion
model within FDS.

The flame spread  calculations from FDS were compared  to the vertical
flame  spread over  a 5~m  slab of  PMMA performed  by  Factory Mutual
Research  Corporation (FMRC).   The  predicted flame  spread rate  was
within  0.3~m/s  for any  point  in  time  during the  analysis.   The
comparison at  the quasi-steady  burning rate once  the full  slab was
burning     shows    that     FDS    over-estimated     the    burning
rate~\cite{Ma:2,Ma:3}.

A   charring  model   was   implemented  in   FDS   by  Hostikka   and
McGrattan~\cite{Hostikka:2}.  The model  is a  simplification  of work
done at  NIST by Ritchie  {\em et al.}~\cite{Ritchie:1}.  The charring
model was  first used to  predict the burning  rate of a  small wooden
sample in the  cone calorimeter.  The results were  more favorable for
higher imposed heat fluxes. For  low imposed fluxes, the heat transfer
at the edge  of the sample was more pronounced,  and more difficult to
model  accurately.   Full-scale room  tests  with  wood paneling  were
modeled, but  the results were  judged to be grid-dependent.  This was
likely a consequence of the  gas phase spatial resolution, rather than
the solid phase. The authors concluded that it is difficult to predict
the growth rate of a fire  in a wood-lined room without ``tuning'' the
pyrolysis rate  coefficients. For real  wood products, it  is unlikely
that all  of the  necessary properties can  be obtained  easily. Thus,
grid sensitivity  and uncertain  material properties make  {\em blind}
predictions of fire  growth on real materials beyond  the reach of the
current version of the model. However, the model can still be used for
a qualitative assessment  of fire behavior as long  as the uncertainty
in the flame spread rate is recognized.


\subsubsection{Response of Active and Passive Fire Protection}

A   significant  validation  effort   for  sprinkler   activation  and
suppression was  a project entitled the  International Fire Sprinkler,
Smoke and Heat Vent, Draft  Curtain Fire Test Project organized by the
National   Fire  Protection   Research  Foundation~\cite{McGrattan:5}.
Thirty-nine  large scale  fire  tests were  conducted at  Underwriters
Laboratories in  Northbrook, IL.  The  tests were aimed  at evaluating
the performance of various  fire protection systems in large buildings
with  flat ceilings, like  warehouses and  ``big box''  retail stores.
All the  tests were conducted  under a 30~m by  30~m adjustable-height
platform in a 37~m by 37~m by 15~m high test bay. At the time, FDS had
not been publicly released and  was referred to as the Industrial Fire
Simulator (IFS), but it was essentially the same as FDS version 1.

For  model  validation  of  sprinkler activation,  the  most  valuable
experiments performed were a series  of heptane spray fires.  With the
spray burner in different  locations, with and without draft curtains,
with  and  without  vertical  vents,  the model  made  predictions  of
sprinkler activation and upper layer temperatures.  For all tests, the
first ring  of sprinklers surrounding the fire  activated within 15~\%
of the experimental  times; within 25~\% for the  second ring. The gas
temperatures near the ceiling were  predicted to within about 15~\% of
the measured values.

Most of the full-scale experiments performed during the project used a
heptane  spray  burner  to   generate  controlled  fires  of  1~MW  to
10~MW.  However, 5  experiments  were performed  with  6~m high  racks
containing the  Factory Mutual Standard Plastic Commodity,  or Group A
Plastic. To model these  fires, bench scale experiments were performed
to characterize the burning behavior of the commodity, and larger test
fires  provided   validation  data  with  which  to   test  the  model
predictions    of    the     burning    rate    and    flame    spread
behavior~\cite{Hamins:1,Hamins:IAFSS2002}.     Two   to    four   tier
configurations  were  evaluated.  For  the  period  of  time prior  to
application of water, the simulated heat release rate was within 20~\%
of the experimental  heat release rates.  It should  be noted that the
model was very  sensitive to the thermal parameters  and the numerical
grid when used to model the fire growth in the piled commodity tests.


High rack storage fires of pool chemicals were modeled by Olenick~{\em
et  al.}~\cite{Olenick:1}  to  determine  the  validity  of  sprinkler
activation predictions  of FDS.  The model was  compared to full-scale
fires conducted  in January, 2000  at Southwest Research  Institute in
San Antonio,  Texas.  The results indicated that  the model accurately
predicted sprinkler activation and the over-pressurization of the test
compartment.

FDS  has  been  used  to  study  the behavior  of  a  fire  undergoing
suppression     by     a    water     mist     system.     Kim     and
Ryou~\cite{Kim:BE2003,Kim:IJACR2004}   compared  FDS   predictions  to
results of compartment fire tests  with and without the application of
a water mist. The cooling and oxygen dilution were predicted to within
about 10~\% of the measurements, but the simulations failed to predict
the complete extinguishment of a hexane pool fire. The authors suggest
that this is a result of the combustion model rather than the spray or
droplet model.

Another study  of water  mist suppression using  FDS was  conducted by
Hume   at   the    University   of   Canterbury,   Christchurch,   New
Zealand~\cite{Hume:Masters}. Full-scale  experiments were performed in
which a fine  water mist was combined with  a displacement ventilation
system to protect occupants and electrical equipment in the event of a
fire.  Simulations of  these experiments  with FDS  showed qualitative
agreement, but the version of the  model used in the study (version 3)
was not able  to predict accurately the decrease  in heat release rate
of the fire.

Hostikka    and   McGrattan~\cite{Hostikka:FSJ2006}    evaluated   the
absorption of  thermal radiation by water sprays.  They considered two
sets of  experimental data and concluded  that FDS has  the ability to
predict the  attenuation of thermal radiation  ``when the hydrodynamic
interaction  between   the  droplets  is   weak.''  However,  modeling
interacting sprays would require a more costly coalescence model. They
also note that  the results of the model were  sensitive to grid size,
angular discretization, and droplet sampling.

\subsubsection{Airflows in Fire Compartments}

Cochard~\cite{Cochard:1} used  FDS to  study the ventilation  within a
tunnel. He  compared the model  results with a full-scale  tunnel fire
experiment conducted  as part of the  Massachusetts Highway Department
Memorial Tunnel Fire Ventilation  Test Program.  The test consisted of
a single  point supply of  fresh air through  a 28~m$^2$ opening  in a
135~m tunnel.  The ventilation was started 2 min after the ignition of
a  40~MW  fire.  Fifteen  temperature  measurement  trees were  placed
within  the  tunnel and  replicated  within  the  model. Depending  on
location,  the difference between  predicted and  measured temperature
rise ranged from 10~\% to 20~\%.

McGrattan and Hamins~\cite{McGrattan:HST} also applied FDS to simulate
two of the Memorial Tunnel Fire Tests as validation for the use of the
model  in  studying  an  actual  fire in  the  Howard  Street  Tunnel,
Baltimore,  Maryland,  July  2001.  The  experiments  chosen  for  the
comparison  were unventilated. One  experiment was  a 20~MW  fire; the
other a 50~MW fire.  FDS predictions of peak near-ceiling temperatures
were within 50~$^\circ$C of the measured peak temperatures, which were
600~$^\circ$C and 800~$^\circ$C, respectively.

Friday studied the  use of FDS in large  scale mechanically ventilated
spaces.   The ventilated  enclosure  was provided  with air  injection
rates of  1 to 12 air  changes per hour  and a fire with  heat release
rates ranging  from 0.5~MW to  2~MW.  The test measurements  and model
output were compared to assess the accuracy of FDS~\cite{Friday:1}.

Zhang {\em  et al.}~\cite{Zhang:2} utilized  the FDS model  to predict
turbulence characteristics  of the flow and temperature  fields due to
fire  in a compartment.   The experimental  data was  acquired through
tests that replicated a half-scale ISO Room Fire Test.  Two cases were
explored:  the heat  source in  the center  of the  room and  the heat
source adjacent to a wall.  The heat source was a heating element with
an output of 12~kW/m$^2$ and was  assumed stable after 300 s.  For the
first case,  the predicted  average velocity and  temperature profiles
were found to ``agree reasonably  well.''  Near the ceiling, the model
under-predicted   temperature   and   over-predicted  velocity.    The
predicted  intensity of  the temperature  fluctuation  ``agree[d] very
well'' at  all points  except those directly  adjacent to  the burner.
The turbulent heat flux was found to be larger in the region above the
heat source.

The  second case also  used a  burner with  a 12~kW/m$^2$  heat source
located  at the  wall.  As  with the  first case,  the  predicted mean
velocities  agree  with  the  experimental  results  except  near  the
ceiling.   The  temperatures  near   the  ceiling  were  found  to  be
over-predicted by FDS.  The  intensity of the velocity fluctuation was
found to  ``agree well''  with the experimental  data except  near the
ceiling.   The  predicted  intensity  of the  temperature  fluctuation
agrees ``very well''  with the experimental data except  in the region
near the  middle of the room.  This  might be due to  the influence of
the door  sill.  Overall, in  both cases, the predicted  values agreed
well  with the  experimental values  in  all regions  except near  the
ceiling.


The ability of  version 1 of FDS to  accurately predict smoke detector
activation was studied by D'Souza~\cite{DSouza:1}. The smoke transport
model within FDS  was tested and compared with UL  217 test data.  The
second step  in this research was  to further validate  the model with
full-scale  multi-compartment fire tests.  The results  indicated that
FDS is capable of predicting  smoke detector activation when used with
smoke  detector  lag correlations  that  correct  for  the time  delay
associated with smoke having to penetrate the detector housing.

Another study of smoke detector  activation was carried out by Brammer
at  the University  of Canterbury,  New  Zealand~\cite{Brammer:1}. Two
fire  tests from  a series  performed  in a  two-story residence  were
simulated, and  smoke detector  activation times were  predicted using
three different methods. The methods consisted of either a temperature
correlation,  a time-lagged  function  of the  optical  density, or  a
thermal device much like a heat detector.  The purpose was to identify
ways to reliably predict smoke detector activation using typical model
output like temperature and  smoke concentration. It was remarked that
simulating  the  early stage  of  the  fire  is critical  to  reliable
prediction.

Cleary~\cite{Cleary:1} also provided a comparison between FDS computed
gas  velocity,  temperature  and  concentrations at  various  detector
locations.   The research concluded  that multi-room  fire simulations
with the FDS model can accurately predict the conditions that a sensor
might experience during a real fire  event.  The FDS model was able to
predict the smoke and gas concentrations, heat, and flow velocities at
various detector locations to within 15~\% of measurements.

Piergoirgio~{\em  et al.}~\cite{Piergiorgio:1} provided  a qualitative
analysis of FDS applied to a  truck fire within a tunnel.  The goal of
their analysis  was to describe the  spread of the  toxic gases within
the tunnels, to determine the  places not involved in the spreading of
combustion products  and to quantify  the oxygen, carbon  monoxide and
hydrochloric acid concentrations during the fire.

Edwards~{\em  et al.}~\cite{Edwards:SME2005,Edwards:FSJ}  used  FDS to
determine the critical air velocity  for smoke reversal in a tunnel as
a function of  the fire intensity, and his  results compared favorably
with   experimental  results.   In  a   further  study,   Edwards  and
Hwang~\cite{Edwards:SME2006}  applied FDS to  study fire  spread along
combustibles in  a ventilated mine  entry. Analyses such as  these are
intended for planning and implementation of ventilation changes during
mine fire fighting and rescue operations.


\subsubsection{Combustion Model}

Floyd~{\em et al.}~\cite{Floyd:1,Floyd:6} compared the radiation model
of  FDS version 2  with full-scale  data from  the Virginia  Tech Fire
Research Laboratory (VTFRL).  The  test compartment was outfitted with
equipment   capable   of  taking   temperature,   air  velocity,   gas
concentrations, unburned hydrocarbon  and heat flux measurements.  The
test facility consisted of  a single compartment geometrically similar
to the ISO 9705 standard compartment with dimensions of 1.2~m by 1.8~m
by  1.2~m  in height.   The  ceiling  and  walls were  constructed  of
fiberboard  over  a steel  shell  with  a  floor of  concrete.   Three
baseline experiments  were completed with  fires ranging in  size from
90~kW  to 440~kW.   A  sample of  the  test results  are presented  in
Table~\ref{grid3}.

\begin{table}
\begin{center}
\caption{Comparison of FDS and measured quantities in a half-scale ISO
9705 compartment \cite{Floyd:1}}
\label{grid3}
\vspace{0.1in}
\begin{tabular}{|*{3}{c|}} \hline
Parameter & VTFRL &  FDS \\ \hline \hline Room TC 1  & 191 $^\circ$C &
134 $^\circ$C \\  \hline Door TC 5 & 348 $^\circ$C  & 365 $^\circ$C \\
\hline Room TC 5 & 585 $^\circ$C & 529 $^\circ$C \\ \hline Door TC 8 &
227 $^\circ$C  & 409 $^\circ$C \\ \hline  Room TC 8 &  625 $^\circ$C &
559 $^\circ$C \\ \hline Door Vel 1  & 4.4 m/s & 3.3 m/s \\ \hline Door
$O_{2}$ 1 & 14.6 \% & 13.9 \% \\ \hline Door Vel 5 & -0.54 m/s & -0.75
m/s \\ \hline Door $O_{2}$ 5 & 21.0  \% & 21.0 \% \\ \hline Room HF1 &
29 kW/m$^2$ & 26  kW/m$^2$ \\ \hline Door $CO_{2}$1 & 3.5  \% & 3.8 \%
\\  \hline Door  HF2 &  3.6  kW/m$^2$ &  4.1 kW/m$^2$  \\ \hline  Door
$CO_{2}$4 & 0.2 \% & 2.2 \% \\ \hline
\end{tabular}
\end{center}
\end{table}

Overall,  FDS  predicted  the  temperatures  to within  15~\%  of  the
measured  temperatures.  The  FDS velocity  measurements  followed the
trend  of  the test  data  but did  not  replicate  it.  The  outgoing
velocities  were under-predicted by  30~\% to  40~\% and  the incoming
velocities were  over-predicted by 40~\%. FDS predicted  the heat flux
gauge response to within 10~\%  of the measured values.  The radiation
model  in FDS  predicted the  measured  fluxes to  within 15~\%.   The
radiation to  and from the  compartment wall was estimated  well.  The
mixture  fraction  model was  also  successful.   For well  ventilated
tests,  the  predictions  were  excellent.   The quality  of  the  FDS
prediction decreased with the under-ventilated cases.  The main source
of error with the model  predictions in the under-ventilated cases was
the over-prediction  of compartment gas  temperatures and the  size of
the upper layer~\cite{Floyd:1,Floyd:6}.

Xin   and  Gore~\cite{Xin:JSS2003}   compared   FDS  predictions   and
measurements  of the  spectral radiation  intensities of  small fires.
The fuel flow rates for  methane and ethylene burners were selected so
that the  Froude numbers  matched that of  liquid toluene  pool fires.
The heat release rate was 4.2~kW  for the methane flame and 3.4~kW for
the ethylene flame.  Line of sight spectral radiation intensities were
measured  at   six  downstream  locations.    The  spectral  radiation
intensity calculations were performed by post-processing the transient
scalar  distributions provided  by FDS.   The calculated  and measured
spectral  radiation  intensities were  found  to  be in  ``excellent''
agreement for the gas radiation bands.

Zhang~{\em et al.}~\cite{Zhang:1} compared the experimental results of
a  circular methane  gas burner  to predictions  computed by  FDS. The
compartment was 2.8~m by 2.8~m  by 2.2~m high with natural ventilation
from a  standard door.  Good agreement was  found for  the temperature
prediction at the doorway where  the radiation model was used. The FDS
model predicted the temperatures at the corner of the room better than
other models  compared by the group.  It was found  that, overall, FDS
predicted temperatures  well but the prescribed  turbulent Prandtl and
Schmidt numbers play an important  role in determining the accuracy of
the model.


Bundy,  Dillon and  Hamins~\cite{Dillon:1,Hamins:FPE2005}  studied the
use of FDS  in providing data and correlations  for fire investigators
to  support their investigations.   A paraffin  wax candle  was placed
within  a  0.61~m by  0.61~m  by  0.76~m  plexi-glass enclosure.   The
chamber was raised 20~mm off the surface to reveal 44 uniformly spaced
6~mm diameter holes.   The holes provided oxygen to  the flame without
subjecting the  flame to a draft.   A 150~mm hole was  provided at the
top of the enclosure to allow  for the heat and combustion products to
exit the space.  The heat flux  from the candle flame was modeled with
FDS.  The model  provides a prediction of the heat  flux of the candle
at a height of  56~mm above the base of the flame  with an accuracy of
5~\%. The flux is under predicted  by 16~\% at 76~mm above the base of
the  flame. The remainder  of the  predictions show  flux measurements
were under-predicted by 15~\% to 40~\% of the measured values.



\subsection{Comparison with Standard Tests}

Standard fire tests are  performed at various testing laboratories and
universities  around  the world.   While  most  were  not designed  as
validation tools, they nevertheless  can be used as relatively simple,
well characterized fire experiments.

An extensive  amount of  validation work with  FDS version 4  has been
performed    by   Hietaniemi,    Hostikka,   and    Vaari    at   VTT,
Finland~\cite{Hietaniemi:1}.  The  case studies are  comprised of fire
experiments ranging  in scale  from the cone  calorimeter (ISO~5660-1,
2002) to full-scale fire tests such as the room corner test (ISO~9705,
1993).  Comparisons are  also  made  between FDS  4  results and  data
obtained  in the  SBI (Single  Burning Item)  Euro-classification test
apparatus (EN  13823, 2002) as  well as data  obtained in two  {\em ad
hoc} experimental  configurations: one is  similar to the  room corner
test but  has only partial linings and  the other is a  space to study
fires in building cavities. In the study of upholstered furniture, the
experimental configurations  are the cone  and furniture calorimeters,
and the  ISO room. For liquid  pool fires, comparison is  made to data
obtained  by  numerous  researchers.   The burning  materials  include
spruce  timber, MDF  (Medium Density  Fiber) board,  PVC  wall carpet,
upholstered furniture, cables with plastic sheathing, and heptane. The
cases studied are summarized in Table~\ref{VTTwork}.

\begin{table}[t]
\begin{center}
\caption{Summary of materials and test methods considered in VTT study \cite{Hietaniemi:1}}
\label{VTTwork}
\vspace{0.1in}
\begin{tabular}{|l|l|}
\hline
Burning material          & Experimental Configuration  \\ \hline \hline
10 mm thick spruce timber &  Cone calorimeter \\
                          &  SBI test         \\
                          &  Room corner test  \\ \hline
22 mm thick spruce timber & Modified room corner test \\
                          & 6 m by 1.1 m by 0.5 m cavity  \\ \hline
12 mm thick MDF board     & Cone calorimeter \\
                          & SBI test \\
                          & Room corner test \\  \hline
PVC wall carpet on        &  Cone calorimeter \\
gypsum plasterboard       &  SBI test \\
                          &  Room corner test \\ \hline
Upholstered furniture:    &  Cone calorimeter \\
a chair with PU padding   &  Furniture calorimeter \\
and PP fabric             &  ISO room test \\ \hline
Cables with plastic       &  6 m by 1.2 m by 0.6 m cavity \\
sheathing                 &  lined  with non-combustible board \\ \hline
Heptane                   & Pool fires of various sizes \\ \hline
\end{tabular}
\end{center}
\end{table}

The scope of the VTT work is considerable.
Assessing the accuracy of the model must be done on a case by case basis. In some cases,
predictions of the burning rate of the material were based solely on its fundamental properties, as
in the heptane pool fire simulations.
In other cases, some properties of the material are unknown, as in the spruce timber simulations. Thus,
some of the simulations are true predictions, some are calibrations. The intent of the authors was to
provide guidance to engineers using the model as to appropriate grid sizes and material properties.
In many cases, the numerical grid was made fairly coarse to account for the fact that in practice, FDS is
used to model large spaces of which the fuel may only comprise a small fraction.


\subsection{Comparison with Documented Fire Experience}

Documented fire experience includes known behavior of materials in fires,
eyewitness accounts of real fires, observed post fire conditions, and other means.
To date, several actual fires have been reconstructed using FDS. One case study performed by NIST
is documented in Ref.~\cite{Madrzykowski:1}.
Two fire fighters were killed and one severely injured in a townhouse fire in
Washington, D.C. during the evening of May 30, 1999.
Questions arose about the injuries the fire fighters had sustained,
the lack of thermal damage in the living room where a fallen
fire fighter was found and why the fire fighters never opened their hose
lines to protect themselves or to extinguish the fire.

To answer some of the questions, a rectangular volume of 10~m by 6~m by 5.1~m was
divided into 76,500 cells in the FDS model. The FDS results that best replicated the observed fire behavior
indicated that the opening of the basement sliding glass door provided oxygen to a pre-heated,
under-ventilated fire. Flashover was estimated to occur in less than 60 s following the entry of
fire fighters into the basement. The resulting fire gases flowed up the basement stairs and
moved across the living room ceiling towards the back wall of the townhouse.
These hot gases came in direct contact with the fire fighters who were killed.
The hot gases traversed the townhouse in less than 2 s, giving the fire fighters little time to respond.
The model showed that the oxygen level was too low to support flaming and, therefore,
the fire fighters did not have a visual cue of the thermal conditions until it was too late.
Results of the FDS study were shared with the D.C. fire department and have been made available via
a multi-media CD-ROM to other fire departments across the country.

Another case study performed at NIST involved a fire in a Houston restaurant~\cite{Texas}.
On the morning of February 14, 2000, a fire started in the office area of a fast food
restaurant. Two fire fighters died when the roof collapsed.
The FDS model was used to simulate the fire. The fuel was assumed to be the
contents of a typical office, and the fire was assumed to have a slowly growing heat release rate peaking at 6~MW.
Multiple vents were modeled and the time at which they opened replicated the
fire fighters' actions after arrival.  The model provided a visual representation of the
fire during the initial phases until the collapse of the roof.

NIST also performed a case study on a fire that killed three children and
three fire fighters on the morning of December 22, 1999~\cite{Iowa}. The fire started
on top of a stove in a two-story residence.
FDS was used to simulate the fire.  The fuel packages consisted of several
furniture items in the kitchen and living room with heat release rates reaching 5.2~MW.
The model results indicated the critical event in the fire was flashover of the
kitchen. The fire became a multi-room event after
flashover with temperatures increasing to over 600~$^\circ$C. The hot gases
spread quickly from the living room to the stairway on the second floor trapping the
fire fighters.

Outside of NIST, FDS has been used to investigate many actual fires, but very few of these studies are
documented in the literature. Exceptions include a study by Rein~{\em et al.}~\cite{Rein:Interflam2004}
looking at several fire events using an analytical fire growth model, the NIST zone model CFAST, and FDS.
A similar study was performed several years earlier by Spearpoint {\em et al.}~\cite{Spearpoint:ICFRE3} as
a class exercise at the University of Maryland.
During the SFPE Professional Development Week in the fall of 2001, a workshop was held in which
several engineers related their experiences using FDS as a forensic tool~\cite{Carpenter:SFPE2001}.
The role of carbon monoxide in the deaths of three fire fighters was studied by Christensen and Icove~\cite{Christensen:JFS}.
There is little quantitative validation of the model afforded by these studies. However, the degree to
which the model is able to reproduce observed behavior can be used as an indicator of the model's strengths and



\chapter{Description of Experiments}


\section{VTT Large Hall Tests}

The experiments are described in Ref.~\cite{Hostikka:Hall}. The series consisted of 8
experiments, but because of replicates only three unique fire scenarios. The experiments
were undertaken to study the movement of smoke in a large hall with a sloped ceiling.
The tests were conducted inside the VTT Fire Test Hall, with dimensions
of 19 m (62 ft) high by 27 m (89 ft) long by 14 m (46 ft) wide. Figure shows detailed plan,
side and perspective schematic diagrams of the experimental arrangement. Each test involved a
single heptane pool fire, ranging from 2~MW to 4~MW. Figure is a photo of a 2~MW fire.
Four types of measurements were used in the present evaluation -- the HGL temperature and depth, average
flame height and the plume temperature. Three vertical arrays of thermocouples, plus two
thermocouples in the plume, were compared to model simulation results. The HGL temperature
and height were reduced from an average of the three TC trees using the standard algorithm.
The ceiling jet temperature was not considered, because the ceiling in the test hall is not flat, and
the standard model algorithm is not appropriate for these conditions.

The VTT test report lacks some information needed to model the experiments, so some
information was based on private communications with the principal investigator, Simo Hostikka. The
information used to conduct the model simulations is presented in Table 2-3, including
information on the fire, the compartment, and the ventilation.

Surface Materials: The walls and ceiling of the test hall consist of a 1 mm (0.039 in) thick layer
of sheet metal on top of a 5 cm (2 in) layer of mineral wool. The floor was constructed of
concrete. The report does not provide thermal properties of these materials. Thermophysical
properties of the materials that were used in the simulations are given in Chapter 3.

Natural Ventilation: In Cases 1 and 2, all doors were closed, and ventilation was restricted to
infiltration through the building envelope. Precise information on air infiltration during these tests
is not available. The scientists who conducted the experiments recommend a leakage area of
about 2 m2 (20 ft2), distributed uniformly throughout the enclosure. By contrast, in Case 3, the
doors located in each end wall (Doors 1 and 2, respectively) were open to the external ambient
environment. These doors are each 0.8 m (2.6 ft) wide by 4 m (5 ft) high, and are located such that
their centers are 9.3 m (30.5 ft) from the south wall.

Mechanical Ventilation: The test hall had a single mechanical exhaust duct, located in the roof
space, running along the center of the building. This duct had a circular section with a diameter of
1 m (40 in), and opened horizontally to the hall at a distance of 12 m (39 ft) from the floor and 10.5
m (34.4 ft) from the west wall. Mechanical exhaust ventilation was operational for Case 3, with a
constant volume flow rate of 11 m3/s drawn through the 1 m (40 in) diameter exhaust duct.
2-14
Heat Release Rate: Each test used a single fire source with its center located 16 m (52 ft) from the west wall and
7.4 m (24.3 ft) from the south wall. For all tests, the fuel was heptane in a circular steel pan that
was partially filled with water. The pan had a diameter of 1.17 m (46.0 in) for Case 1 and 1.6 m
(63 in) for Cases 2 and 3. In each case, the fuel surface was 1 m (40 in) above the floor. The
trays were placed on load cells, and the HRR was calculated from the mass loss rate (see definition
in Chapter 3). For the three cases, the fuel mass loss rate was averaged from individual replicate
tests. In the HRR estimation, the heat of combustion (taken as 44.6 kJ/g) and the combustion efficiency
for n-heptane was used. Hostikka suggests a value of 0.8 for the combustion
efficiency. Bundy [Ref. 23] estimates the efficiency of a 500 kW heptane pool fire to be equal
to 0.97. Tewarson reports a value of 0.93 for a 10 cm pool [Ref. 17]. The magnitude of the
combustion efficiency is a complicated function of fire size, ventilation, and other effects.
Consideration of the chemical structure of a fire suggests that the combustion efficiency should
decrease as the fire size grows. Available data confirms this [Ref. 24]. The size of a
compartment may also impact this parameter, but there is little information in the fire literature
that addresses this point. In summary, there is little certainty in the actual value of the
combustion efficiency in this experiment. In this report, a combustion efficiency of 0.85 � 0.12
(or � 14 %) is recommended for the BE #2 pool fire tests, based on engineering judgment. Due
to the relatively large value of the uncertainty associated with .a, the uncertainty in HRR is
dominated by the uncertainty in the combustion efficiency. Uncertainty in the mass loss rate
measurement also contributed to the overall uncertainty, and the uncertainty in HRR was estimated
as 15 %. Figures 2-9 to 2-11 show the prescribed HRR as a function of time during Cases 1 to 3,
respectively. Tables 2-4 to 2-6 represent the mass loss and estimated HRR associated with Figures
2-9 to 2-11, respectively.

Radiative Fraction: The radiative fraction was assigned a value of 0.35, similar to many smoky hydrocarbons
[Ref. 19]. The relative combined expanded (2s) uncertainty in this parameter was assigned a
value of �20 %, which is typical of uncertainty values reported in the literature for this
parameter.



\clearpage

\section{UL/NFPRF Sprinkler, Vent, and Draft Curtain Study}
\label{UL_NFPRF_Description}

In January, 1997, a series of 22
heptane spray burner experiments was conducted at the
Large Scale Fire Test Facility at Underwriters Laboratories (UL) in
Northbrook, Illinois~\cite{Sheppard:1}. The objective of the experiments was
to characterize the temperature and flow field for fire
scenarios with a controlled heat release rate in the presence of
sprinklers, draft curtains and a single smoke \& heat vent.

\subsubsection{Test Facility}

The Large Scale Fire Test Facility at UL
contains a 37~m by 37~m (120~ft by 120~ft) main fire test cell,
equipped with a 30.5~m by 30.5~m (100~ft by 100~ft)
adjustable height ceiling.
The layout of the experiments is shown in Fig.~\ref{layout}. One
1.2~m by 2.4~m (4~ft by 8~ft) vent
was installed among 49 upright sprinklers on a 3~m by 3~m
(10~ft by 10~ft) spacing.

\begin{figure}[p]
\begin{center}
\setlength{\unitlength}{.05416667in}
\begin{picture}(120,120)

\linethickness{1.mm}
\put(0,0){\framebox(120,120)[tc]{North Wall}}
\linethickness{.5mm}
\put(10,10){\framebox(100,100)[tc]{Adjustable Height Ceiling}}

\thinlines
\put(117,67){\vector(0,-1){67}}
\put(117,73){\vector(0, 1){47}}
\put(117,70){\makebox(0,0){$120'$}}
\put(113,57){\vector(0,-1){47}}
\put(111,110){\line(1,0){4.}}
\put(111, 10){\line(1,0){4.}}
\put(113,63){\vector(0, 1){47}}
\put(113,60){\makebox(0,0){$100'$}}
\put(30.9,12.83){\dashbox{1}(67.1,71.17)[tc]{Draft Curtains}}
\put(27.9,40){\vector(0,-1){27.17}}
\put(27.9,46){\vector(0, 1){38.0}}
\put(25.9,84.){\line(1,0){4.}}
\put(25.9,12.83){\line(1,0){4.}}
\put(27.9,43){\makebox(0,0){$71'2''$}}
\put(64.0,87.){\vector(-1,0){33.1}}
\put(72.0,87.){\vector( 1,0){26.0}}
\put(30.9,85.){\line(0,1){4.}}
\put(98.0,85.){\line(0,1){4.}}
\put(68.0,87.){\makebox(0,0){$67'1''$}}

\put(16.0,87.){\vector(-1,0){6.}}
\put(24.0,87.){\vector( 1,0){6.92}}
\put(20.0,87.){\makebox(0,0){$20'11''$}}
\put(101.,87.){\vector(-1,0){3.}}
\put(107.,87.){\vector( 1,0){3.}}
\put(104.,87.){\makebox(0,0){$12'$}}

\put(27.9,100){\vector(0,1){10.}}
\put(27.9,94){\vector(0,-1){10.}}
\put(27.9,97){\makebox(0,0){$26'$}}

\put(27.9,8){\vector(0,1){2.}}
\put(27.9,8){\line(1,0){3.}}
\put(30.9,8){\makebox(0,0)[l]{$2'10''$}}

\put(55.08,14.83){\line(-1,0){2.}}
\put(54.08,16.83){\vector(0,-1){2.}}
\put(54.08,7.83){\vector(0,1){5.}}
\put(54.08,7.83){\line(1,0){3.}}
\put(57.08,7.83){\makebox(0,0)[l]{$2'$}}

\put(85.08,24.83){\line(0,-1){2.}}
\put(95.08,24.83){\line(0,-1){2.}}
\put(93.08,23.83){\vector(1,0){2.}}
\put(87.08,23.83){\vector(-1,0){2.}}
\put(103.00,23.83){\vector(-1,0){5.}}
\put(103.00,23.83){\line(0,-1){3.}}
\put(103.00,20.83){\makebox(0,0)[ct]{$2'11''$}}
\put(90.08,23.83){\makebox(0,0)[c]{$10'$}}

\thicklines
\put(78.08,55.83){\framebox(4,8){ }}

\put(78.58,58.33){\framebox(3,3)[c]{A}}
\put(78.58,68.33){\framebox(3,3)[c]{B}}
\put(88.58,58.33){\framebox(3,3)[c]{C}}
\put(58.58,38.33){\framebox(3,3)[c]{D}}

\thinlines

\multiput(35.08,14.83)(0,10){7}{\circle*{.8}}
\multiput(45.08,14.83)(0,10){7}{\circle*{.8}}
\multiput(55.08,14.83)(0,10){7}{\circle*{.8}}
\multiput(65.08,14.83)(0,10){7}{\circle*{.8}}
\multiput(75.08,14.83)(0,10){7}{\circle*{.8}}
\multiput(85.08,14.83)(0,10){7}{\circle*{.8}}
\multiput(95.08,14.83)(0,10){7}{\circle*{.8}}
\tiny
\put(35.48,15.23){98}
\put(45.48,15.23){91}
\put(55.48,15.23){84}
\put(65.48,15.23){81}
\put(75.48,15.23){78}
\put(85.48,15.23){75}
\put(95.48,15.23){72}
\put(35.48,25.23){99}
\put(45.48,25.23){92}
\put(55.48,25.23){85}
\put(65.48,25.23){82}
\put(75.48,25.23){79}
\put(85.48,25.23){76}
\put(95.48,25.23){73}
\put(35.48,35.23){100}
\put(45.48,35.23){93}
\put(55.48,35.23){86}
\put(65.48,35.23){83}
\put(75.48,35.23){80}
\put(85.48,35.23){77}
\put(95.48,35.23){74}
\put(35.48,45.23){101}
\put(45.48,45.23){94}
\put(55.48,45.23){87}
\put(65.48,45.23){62}
\put(75.48,45.23){58}
\put(85.48,45.23){54}
\put(95.48,45.23){50}
\put(35.48,55.23){102}
\put(45.48,55.23){95}
\put(55.48,55.23){88}
\put(65.48,55.23){63}
\put(75.48,55.23){59}
\put(85.48,55.23){55}
\put(95.48,55.23){51}
\put(35.48,65.23){103}
\put(45.48,65.23){96}
\put(55.48,65.23){89}
\put(65.48,65.23){64}
\put(75.48,65.23){60}
\put(85.48,65.23){56}
\put(95.48,65.23){52}
\put(35.48,75.23){104}
\put(45.48,75.23){97}
\put(55.48,75.23){90}
\put(65.48,75.23){65}
\put(75.48,75.23){61}
\put(85.48,75.23){57}
\put(95.48,75.23){53}
\put(70.08,49.83){\makebox(0,0)[c]{68}}
\put(70.08,59.83){\makebox(0,0)[c]{69}}
\put(70.08,69.83){\makebox(0,0)[c]{70}}
\put(80.08,49.83){\makebox(0,0)[c]{67}}
\put(90.08,49.83){\makebox(0,0)[c]{66}}
\put(90.08,69.83){\makebox(0,0)[c]{71}}

\multiput(80.08,56.83)(0,1){7}{\circle*{.2}}
\put(80.58,62.83){\line(1,0){22.5}}
\put(104.,62.83){\makebox(0,0)[l]{43}}
\put(104.,61.33){\makebox(0,0)[l]{44}}
\put(104.,59.83){\makebox(0,0)[l]{45}}
\put(104.,58.33){\makebox(0,0)[l]{46}}
\put(104.,56.83){\makebox(0,0)[l]{47}}
\put(104.,55.33){\makebox(0,0)[l]{48}}
\put(104.,53.83){\makebox(0,0)[l]{49}}

\normalsize

\end{picture}
\end{center}
\caption[Plan view of the UL/NFPRF Experiments.]
{\bf Plan view of the UL/NFPRF Experiments.
The sprinklers are indicated
by the solid circles and are spaced 3~m (10~ft) apart. The number beside each
sprinkler location indicates the channel number of the nearest
thermocouple. The vent dimensions
are 4~ft by 8~ft. The boxed letters A, B, C and D indicate burner positions.
Corresponding to each burner position is a vertical array of thermocouples.
Thermocouples 1--9 hang 7, 22, 36, 50, 64, 78, 92, 106 and 120~in from the
ceiling, respectively, above Position A. Thermocouples 10 and 11 are
positioned above and below the ceiling tile directly above Position B, followed
by 12--20 that hang at the same levels below the ceiling as 1--9.
The same pattern is followed at Positions C and D, with thermocouples
21--31 at C and 32--42 at D.}
\label{layout}
\end{figure}

The ceiling was raised to a height of 7.6~m (25~ft) and
instrumented with thermocouples and other measurement devices.
The ceiling was constructed of 0.6~m by 1.2~m by 1.6~cm
(2~ft by 4~ft by 5/8~in)
UL fire rated Armstrong Ceramaguard (Item 602B) ceiling tiles.
The manufacturer reported the thermal properties of the material to
be: specific heat 753 J/kg$\cdot$K,
thermal diffusivity $2.6 \times 10^{-7}$~m$^2$/s, conductivity
0.0611~W/m$\cdot$K, and density 313~kg/m$^3$.

Draft curtains 1.8~m (6~ft) deep were installed for 16 of the
22 tests, enclosing an area of about 450~m$^2$ (4,800~ft$^2$).
The curtains were constructed of 1.4~m (54~in) wide sheets of 18
gauge sheet metal.

The sprinklers used were Central ELO-231 (Extra Large Orifice) uprights.
The orifice diameter of this sprinkler is reported by the
manufacturer to be nominally 0.64~in, the
reference actuation temperature is reported by the manufacturer to be
74$^\circ$C (165$^\circ$F). The RTI (Response Time Index) and
C-factor (Conductivity factor) were reported by UL
to be 148~(m$\cdot$s)$^\ha$ (268~(ft$\cdot$s)$^\ha$) and
0.7~(m/s)$^\ha$ (1.3~(ft/s)$^\ha$), respectively~\cite{Sheppard:1}.
When installed, the sprinkler deflector was located 8~cm (3~in)
below the ceiling. The thermal element of the sprinkler
was located 11~cm (4.25~in)
below the ceiling. The sprinklers were installed with 3~m by 3~m
(10~ft by 10~ft) spacing in a system designed
to deliver a constant 0.34~L/(s$\cdot$m$^2$) (0.50 gpm/ft$^2$)
discharge density when supplied by a 131~kPa (19~psi) discharge pressure

A single UL listed
double leaf fire vent with steel covers and steel curb
was installed in the adjustable height ceiling in the position shown
in Fig.~\ref{layout}.
The vent is designed to open manually or automatically.
The vent doors were recessed into the ceiling about 0.3~m (1~ft).

\subsubsection{Fire and Heat Release Rate}

The heptane spray
burner consisted of a 1~m by 1~m (40~in by 40~in) square of
1/2~in pipe supported by four cement blocks
0.6~m (2~ft) off the floor.
Four atomizing spray nozzles were used to provide a free spray of heptane
that was then ignited.
For all but one of the tests, the total heat release rate from the
fire was manually ramped up following the curve
$$ \dot{Q} = \dot{Q}_0 \; \left( \frac{t}{\tau} \right)^2 $$
where $\dot{Q}_0=10$~MW and $\tau=75$~s ($\tau=150$~s was used in Test I-16).
The fire growth curve was followed
until a specified fire size was reached or the first sprinkler
activated. After either of these events, the fire size was maintained
at that level until conditions reached roughly a steady state,
{\em i.e.} the temperatures recorded near the ceilings remained
steady and no more sprinkler activations occurred.

The heat release rate from the burner was confirmed by placing it under
the large product calorimeter at UL, ramping up the flow of heptane in the
same manner as in the tests, and measuring the total and
convective heat release rates. It was found that
the convective heat release rate was 0.65$\pm$0.02 of the total.

\subsubsection{Instrumentation}

The instrumentation for the tests consisted of thermocouples, gas
analysis equipment, and pressure transducers. The locations of the
instrumentation are referenced in the plan view of the facility
(Fig.~\ref{layout}).

Temperature measurements were recorded at 104 locations.
Type K 0.0625~in diameter Inconel sheathed thermocouples
were positioned to measure (i) temperatures near the ceiling, (ii)
temperatures of the ceiling jet, and (iii) temperatures near the vent.
The thermocouples numbered 50--65 were positioned near the sprinklers,
10~cm (4~in) below the ceiling. These were intended to measure near-sprinkler
gas temperatures as well as to detect sprinkler
activation when wetted. Thermocouples 66--104 were placed 5~cm (2~in)
below the ceiling. Thermocouples 43--49 ran down the centerline of the
vent at the level of the ceiling, and were spaced 0.3~m (1~ft) apart.
Thermocouples 1--42 were mounted on arrays hanging above each fire
location. The positions are noted in the caption to Fig.~\ref{layout}.

Oxygen, carbon dioxide and carbon monoxide
sampling probes were placed
at the ground (5~cm (3~in) from the floor, 2~m (6~ft) from the burner), and at
the vent (15~cm (6~in) below the ceiling, vent center).




\section{NIST/NRC Test Series}

These experiments, sponsored by the US NRC and conducted at NIST, consisted of 15 large-scale experiments
performed in June 2003. All 15 tests were considered in this study. The experiments are documented
in Ref. [6]. The fire sizes ranged from 350 kW to 2.2 MW in a compartment with dimensions
21.7 m by 7.1 m by 3.8 m high, designed to represent a compartment in a NPP containing power and
control cables. A photo of the fire seen through the compartment doorway is shown in Fig. 2-12.
Walls and ceiling were covered with two layers of marinate boards, each layer 0.0125 m thick.
The floor was covered with one layer of 0.0125  thick gypsum board on top of a
0.0183 m layer of plywood. Thermophysical and optical properties of the marinate and
other materials used in the compartment are given in Chapter 3 and Ref. [6]. The room had one
door and a mechanical air injection and extraction system. Ventilation conditions, the fire size,
and fire location were varied. Numerous measurements (approximately 350 per test) were made
including gas and surface temperatures, heat fluxes and gas velocities. Detailed schematic
diagrams of the experimental arrangement are shown in Figure 2-13. Table 2-7 lists information
associated with the fuel including the fuel type, the steady heat release rate, the pan position and
duration of the ramp-up, ramp-down and steady burn periods. Other information used to conduct
the model simulations is presented in Table 2-8, including information on the fire, the
compartment, the ventilation, targets, and ambient conditions.
Figure 2-12. Photograph of a 1 MW heptane fire seen through the open doorway. Photo
provided by Anthony Hamins, NIST.

Natural Ventilation: The compartment had a 2 m by 2 m door in the middle of the west wall.
Some of the tests had a closed door and no mechanical ventilation (Tests 2, 7, 8, 13, and 17), and
in those tests the measured compartment leakage was an important consideration. Ref. [6]
reports leakage area based on measurements performed prior to Tests 1, 2, 7, 8, and 13. For the
closed door tests, the leakage area used in the simulations ought to be based on the last available
measurement. It should be noted that the chronological order of the tests differed from the
numerical order [Ref. 6]. For Test 4, it is recommended that the leakage area measured before
Test 2 be used. For Tests 10 and 16, it is recommended that the leakage area measured before
Test 7 be used.

Mechanical Ventilation: The mechanical ventilation and exhaust was used during Tests 4, 5, 10,
and 16, providing about 5 air changes per hour. The door was closed during Test 4 and open
during Tests 5, 10, and 16. The supply duct was positioned on the south wall, about 2 m off the
floor. An exhaust duct of equal area to the supply duct was positioned on the opposite wall at a
comparable location. The flow rates through the supply and exhaust ducts were measured in
detail during breaks in the testing, in the absence of a fire.
During the tests, the flows were monitored with single bi-directional probes during the tests
themselves. A bi-directional probe was positioned in the center of the exhaust duct, and its
velocity was recorded under the column header ~SBP Exhaust Vent~T in the experimental
data sets. Its value varied between 3 m/s (10 ft/s) and 4 m/s (13 ft/s) for the four ventilated tests.
The supply and exhaust volume flow rates and other pertinent information can be found in
Ref. [6]. Usually this is expressed as the vent area times an average velocity. Another bidirectional
probe was positioned in the supply duct, 30 cm (1 ft) from the bottom of the duct
during Tests 4 and 5, and 15 cm from the bottom of the duct for Tests 10 and 16. In the data
sets, this measurement is listed under the column header ~SBP Supply Vent-16~T. Its value was
between 3 m/s (10 ft/s) and 4 m/s (13 ft/s) for Tests 4 and 5, and was as high as 10 m/s (33 ft/s)
during Tests 10 and 16.
The exhaust duct profile was relatively uniform, whereas the supply was not. Most of the air
was blown out of the bottom third of the supply duct. The single point measurements during the
fire tests indicated that the flow field changed from its ambient values. The measured supply
volume flow rate of 1.06 m3/s (37.4 ft3/s) pre-test decreased to 0.9 m3/s (31 ft3/s) during testing.
For the exhaust, the measured volume flow rate of 1.03 m3/s (36.4 ft3/s) pretest increased to
about 1.7 m3/s (60 ft3/s) during testing. The uncertainties during the fires are substantially higher
than the uncertainties in the ambient measurements (�0.2 m3/s or 7 ft3/s). Doubling this value is
appropriate.
The ventilation system affected the compartment pressure, HGL temperature, and the surface
temperature of various cable targets. The cable surface TCs were just outside of the direct path
of the supply fan. In the absence of a fire, blowing was observed to flow upwards at about a 35�
angle.
Fire: The fire was located at floor level in the center of the compartment for most of the tests
(Tests 1 - 13, 16, and 17). In Test 14, the fire was centered 1.8 m (72 in) from the North wall. In
Test 15, the fire was centered 1.25 m (50 in) from the South wall. In Test 18, the fire was
centered 1.55 m (62 in) from the South wall. Physically, the fuel pan was 2 m (80 in) long x 1 m
2-26

Heat Release Rate: A single nozzle was used to spray liquid hydrocarbon fuels
onto a 1 m by 2 m fire pan that was about 0.02 m deep. The test plan
originally called for the use of two nozzles to provide the fuel spray. Experimental observation
suggested that the fire was less unsteady with the use of a single nozzle. In addition, it was
observed that the actual extent of the liquid pool was well-approximated by a 1 m (40 in) circle
in the center of the pan.
For safety reasons, the fuel flow was terminated when the lower-layer oxygen concentration
dropped to approximately 15% by volume.

Heat Release Rate:
The fuel used in 14 of the tests was heptane, while toluene was used for one test (see Test 17 in
Table 2-7). The HRR was determined using oxygen consumption calorimetry. The uncertainty in
the measurement was documented in Ref. [6]. The recommended uncertainty values were
17 \% for all of the tests (also see Table 3-1). Figure 2-18 shows the measured and prescribed Q
as a function of time during Test 3. Ref. [6] discusses the shape of the prescribed HRR curve in
detail.

Radiative Fraction:
The radiative fraction was measured in an independent study for the same fuels using the same
spray burner as used in the test series [Ref. 18]. The value of the radiative fraction and its
uncertainty were reported as 0.44 � 16 % and 0.40 � 23 % for heptane and toluene, respectively
(also see Table 3-1).









\chapter{Hot Gas Layer Temperature and Depth}

FDS, like any CFD model, does not perform a direct calculation of the HGL temperature or height.
These are constructs unique to two-zone models, like CFAST and MAGIC.
Nevertheless, FDS does make predictions of gas temperature at the same locations as the thermocouples in the experiments,
and these values can be reduced in the same manner as the experimental measurements to produce an ``average''
HGL temperature and height.  Regardless of the validity of the reduction method,
the FDS predictions of the HGL temperature and height ought to be representative of the accuracy of its predictions
of the individual thermocouple measurements that are used in the HGL reduction.
The temperature measurements from all six test series are used to compute an HGL temperature and height with
which to compare to FDS.  The same layer reduction method is used for five of the six test series.
Only the NBS Multi-Room series uses another method.

A brief description of each test series is included below, followed by graphs comparing the
predicted and measured HGL temperature and layer height.
A summary table is provided at the end of the section that displays the relative differences between
predictions and measurements for all six test series.  Note that the calculation of relative difference
is based on the temperature rise above ambient, and the layer depth, that is, the distance from the
ceiling to where the hot gas layer descends.  Where the model over-predicts the HGL temperature or the depth of the HGL,
the relative difference is a positive number.
This convention is used throughout this report where the model over-predicts the severity of the fire,
the relative difference is positive; where it under-predicts, the difference is negative.

\clearpage


\section{NIST/WTC Test Series}


\begin{figure}[p]
\begin{tabular*}{\textwidth}{l@{\extracolsep{\fill}}r}
\includegraphics[width=2.6in]{FIGURES/WTC_01_v5_HGL_Temperature} &
\includegraphics[width=2.6in]{FIGURES/WTC_02_v5_HGL_Temperature} \\
\includegraphics[width=2.6in]{FIGURES/WTC_03_v5_HGL_Temperature} &
\includegraphics[width=2.6in]{FIGURES/WTC_04_v5_HGL_Temperature} \\
\includegraphics[width=2.6in]{FIGURES/WTC_05_v5_HGL_Temperature} &
\includegraphics[width=2.6in]{FIGURES/WTC_06_v5_HGL_Temperature}
\end{tabular*}
\caption{Hot Gas Layer Temperatures for the NIST/WTC Test Series.}
\label{NIST_WTC_HGL_Temp}
\end{figure}

\begin{figure}[p]
\begin{tabular*}{\textwidth}{l@{\extracolsep{\fill}}r}
\includegraphics[width=2.6in]{FIGURES/WTC_01_v5_HGL_Height} &
\includegraphics[width=2.6in]{FIGURES/WTC_02_v5_HGL_Height} \\
\includegraphics[width=2.6in]{FIGURES/WTC_03_v5_HGL_Height} &
\includegraphics[width=2.6in]{FIGURES/WTC_04_v5_HGL_Height} \\
\includegraphics[width=2.6in]{FIGURES/WTC_05_v5_HGL_Height} &
\includegraphics[width=2.6in]{FIGURES/WTC_06_v5_HGL_Height}
\end{tabular*}
\caption{Hot Gas Layer Height for the NIST/WTC Test Series.}
\label{NIST_WTC_HGL_Temp}
\end{figure}

\clearpage

\section{VTT Large Hall Test Series}

The HGL temperature and depth are calculated from the averaged gas temperatures from three vertical
thermocouple arrays using the standard reduction method.
There are 10 thermocouples in each vertical array, spaced 2 m apart in the lower two-thirds of the hall,
and 1 m apart near the ceiling. Figure~\ref{VTT_Overview} presents a snapshot from one of the simulations.
Note in the figure that all of the obstructions, including the slanted roof and exhaust duct,
are approximated in the model as rectangular to conform with the rectilinear grid.

\begin{figure}[p]
\begin{tabular*}{\textwidth}{l@{\extracolsep{\fill}}r}
\includegraphics[width=2.6in]{FIGURES/VTT_01_v5_HGL_Temp} &
\includegraphics[width=2.6in]{FIGURES/VTT_01_v5_HGL_Height} \\
\includegraphics[width=2.6in]{FIGURES/VTT_01_v5_HGL_Temp} &
\includegraphics[width=2.6in]{FIGURES/VTT_01_v5_HGL_Height} \\
\includegraphics[width=2.6in]{FIGURES/VTT_01_v5_HGL_Temp} &
\includegraphics[width=2.6in]{FIGURES/VTT_01_v5_HGL_Height}
\end{tabular*}
\caption{Predicted HGL Temperature and Height for the VTT Large Hall Tests.}
\label{VTT_HGL}
\end{figure}



\clearpage

\section{NIST/NRC Test Series}

The NIST/NRC series consisted of 15 liquid spray fire tests with different heat release rates, pan locations, and ventilation conditions.
The basic geometry, including the numerical grid, is shown in Figure~\ref{NIST_NRC_Overview}.
Gas temperatures were measured using seven floor-to-ceiling thermocouple arrays (or ``trees'')
distributed throughout the compartment.  The average hot gas layer temperature and height are calculated using
thermocouple Trees 1, 2, 3, 5, 6 and 7. Tree 4 was not used because one of its thermocouples (4-9)
malfunctioned during most of the experiments.

A few observations about the simulations:
\begin{itemize}
\item During Tests 4, 5, 10 and 16 a fan blew air into the compartment through a vent in the south wall.
The measured velocity profile of the fan is not uniform, with the bulk of the air blowing from the lower third of
the duct towards the ceiling at a roughly 45? angle.  The exact flow pattern is
difficult to replicate in the model, thus, the results for Tests 4, 5, 10 and 16 should be evaluated with this in mind.
The effect of the fan on the hot gas layer is small, but it does have a some effect on target temperatures near the vent.
\item For all of the tests involving a fan, the predicted HGL height rises after the fire is extinguished,
while the measured HGL drops.  This appears to be a curious artifact of the layer reduction algorithm.
It is not included in the calculation of the relative difference.
\item In the closed door tests, the hot gas layer descends all the way to the floor.
However, the reduction method, used on both the measured and predicted temperatures,
does not account for the formation of a single layer, and therefore does not indicate that the layer drops all the way to the floor.
This is neither a flaw in the measurements nor in FDS, but rather in the layer reduction method.
\item The HGL reduction method produces spurious results in the first few minutes of each test because no clear layer has yet formed.
These early times are not included in the relative difference calculation.
\end{itemize}

\begin{figure}[p]
\begin{tabular*}{\textwidth}{l@{\extracolsep{\fill}}r}
\includegraphics[width=2.6in]{FIGURES/NIST_NRC_01_v5_HGL_Temperature} &
\includegraphics[width=2.6in]{FIGURES/NIST_NRC_01_v5_HGL_Height} \\
\includegraphics[width=2.6in]{FIGURES/NIST_NRC_07_v5_HGL_Temperature} &
\includegraphics[width=2.6in]{FIGURES/NIST_NRC_07_v5_HGL_Height} \\
\includegraphics[width=2.6in]{FIGURES/NIST_NRC_02_v5_HGL_Temperature} &
\includegraphics[width=2.6in]{FIGURES/NIST_NRC_02_v5_HGL_Height} \\
\includegraphics[width=2.6in]{FIGURES/NIST_NRC_08_v5_HGL_Temperature} &
\includegraphics[width=2.6in]{FIGURES/NIST_NRC_08_v5_HGL_Height}
\end{tabular*}
\caption{Predicted HGL Temperature and Height for the NIST/NRC Tests 1, 7, 2 and 8.}
\label{NIST_NRC_HGL_Closed_1}
\end{figure}

\begin{figure}[p]
\begin{tabular*}{\textwidth}{l@{\extracolsep{\fill}}r}
\includegraphics[width=2.6in]{FIGURES/NIST_NRC_04_v5_HGL_Temperature} &
\includegraphics[width=2.6in]{FIGURES/NIST_NRC_04_v5_HGL_Height} \\
\includegraphics[width=2.6in]{FIGURES/NIST_NRC_10_v5_HGL_Temperature} &
\includegraphics[width=2.6in]{FIGURES/NIST_NRC_10_v5_HGL_Height} \\
\includegraphics[width=2.6in]{FIGURES/NIST_NRC_13_v5_HGL_Temperature} &
\includegraphics[width=2.6in]{FIGURES/NIST_NRC_13_v5_HGL_Height} \\
\includegraphics[width=2.6in]{FIGURES/NIST_NRC_16_v5_HGL_Temperature} &
\includegraphics[width=2.6in]{FIGURES/NIST_NRC_16_v5_HGL_Height}
\end{tabular*}
\caption{Predicted HGL Temperature and Height for the NIST/NRC Tests 4, 10, 13 and 16.}
\label{NIST_NRC_HGL_Closed_2}
\end{figure}

\begin{figure}[p]
\begin{tabular*}{\textwidth}{l@{\extracolsep{\fill}}r}
\includegraphics[width=2.6in]{FIGURES/NIST_NRC_17_v5_HGL_Temperature} &
\includegraphics[width=2.6in]{FIGURES/NIST_NRC_17_v5_HGL_Height} \\
\multicolumn{2}{c}{Open door tests to follow} \\
\includegraphics[width=2.6in]{FIGURES/NIST_NRC_03_v5_HGL_Temperature} &
\includegraphics[width=2.6in]{FIGURES/NIST_NRC_03_v5_HGL_Height} \\
\includegraphics[width=2.6in]{FIGURES/NIST_NRC_09_v5_HGL_Temperature} &
\includegraphics[width=2.6in]{FIGURES/NIST_NRC_09_v5_HGL_Height}
\end{tabular*}
\caption{Predicted HGL Temperature and Height for the NIST/NRC Tests 17, 3 and 9.}
\label{NIST_NRC_HGL_Open_1}
\end{figure}

\begin{figure}[p]
\begin{tabular*}{\textwidth}{l@{\extracolsep{\fill}}r}
\includegraphics[width=2.6in]{FIGURES/NIST_NRC_05_v5_HGL_Temperature} &
\includegraphics[width=2.6in]{FIGURES/NIST_NRC_05_v5_HGL_Height} \\
\includegraphics[width=2.6in]{FIGURES/NIST_NRC_14_v5_HGL_Temperature} &
\includegraphics[width=2.6in]{FIGURES/NIST_NRC_14_v5_HGL_Height} \\
\includegraphics[width=2.6in]{FIGURES/NIST_NRC_15_v5_HGL_Temperature} &
\includegraphics[width=2.6in]{FIGURES/NIST_NRC_15_v5_HGL_Height} \\
\includegraphics[width=2.6in]{FIGURES/NIST_NRC_18_v5_HGL_Temperature} &
\includegraphics[width=2.6in]{FIGURES/NIST_NRC_18_v5_HGL_Height}
\end{tabular*}
\caption{Predicted HGL Temperature and Height for the NIST/NRC Tests 5, 14, 15 and 18.}
\label{NIST_NRC_HGL_Open_2}
\end{figure}


\clearpage

\section{FM/SNL Test Series}

Tests 4, 5, and 21 from the FM/SNL test series are selected for comparison.
The hot gas layer temperature and height are calculated using the standard method.
The thermocouple arrays that are referred to as Sectors 1, 2 and 3 are averaged (with an equal weighting for each) for Tests 4 and 5.
For Test 21, only Sectors 1 and 3 are used, as Sector 2 falls within the smoke plume.

Note the following:
\begin{itemize}
\item The HGL heights, both the measured and predicted, are somewhat noisy due to the effect of ventilation ducts in the upper layer.
\item The ventilation was turned off after 9 min in Test 5,
the effect of which was a slight increase in both the measured and predicted HGL temperature.
\item The measured HGL temperature is noticeably greater than the prediction in Test 21.
This is possibly due to an increase in the HRR towards the end of the test.  The simulations all used fixed HRRs after the 4 min ramp up.
\end{itemize}

\begin{figure}[p]
\begin{tabular*}{\textwidth}{l@{\extracolsep{\fill}}r}
\includegraphics[width=2.6in]{FIGURES/FM_SNL_04_v5_HGL_Temp} &
\includegraphics[width=2.6in]{FIGURES/FM_SNL_04_v5_HGL_Height} \\
\includegraphics[width=2.6in]{FIGURES/FM_SNL_05_v5_HGL_Temp} &
\includegraphics[width=2.6in]{FIGURES/FM_SNL_05_v5_HGL_Height} \\
\includegraphics[width=2.6in]{FIGURES/FM_SNL_21_v5_HGL_Temp} &
\includegraphics[width=2.6in]{FIGURES/FM_SNL_21_v5_HGL_Height}
\end{tabular*}
\caption{Hot Gas Layer Temperature and Height for the FM/SNL Tests.}
\label{FM_SNL_HGL}
\end{figure}

\clearpage

\section{NBS Multi-Room Test Series}

This series of experiments consists of two relatively small rooms connected by a long corridor.
The fire is located in one of the rooms.  Eight vertical arrays of thermocouples are positioned
throughout the test space: one in the burn room, one near the door of the burn room, three in the corridor,
one in the exit to the outside at the far end of the corridor, one near the door of the other or ``target'' room,
and one inside the target room.  Four of the eight arrays have been selected for comparison with model prediction:
the array in the burn room (BR), the array in the middle of the
corridor (18 ft from the BR), the array at the far end of the corridor (38 ft from the BR),
and the array in the target room (TR).  In Tests 100A and 100O, the target room is closed,
in which case the array in the exit (EXI) doorway is used.
The test director reduced the layer information individually for the eight thermocouple arrays using an alternative method.
These results are included in the original data sets.
However, for the current validation study, the selected TC trees were reduced using the conventional method common
to all the experiments considered.  The results are presented below.


\begin{figure}[p]
\begin{tabular*}{\textwidth}{l@{\extracolsep{\fill}}r}
\includegraphics[width=2.6in]{FIGURES/NBS_100A_v5_Tree_1_HGL_Temp} &
\includegraphics[width=2.6in]{FIGURES/NBS_100A_v5_Tree_1_HGL_Height} \\
\includegraphics[width=2.6in]{FIGURES/NBS_100A_v5_Tree_3_HGL_Temp} &
\includegraphics[width=2.6in]{FIGURES/NBS_100A_v5_Tree_3_HGL_Height} \\
\includegraphics[width=2.6in]{FIGURES/NBS_100A_v5_Tree_4_HGL_Temp} &
\includegraphics[width=2.6in]{FIGURES/NBS_100A_v5_Tree_4_HGL_Height} \\
\includegraphics[width=2.6in]{FIGURES/NBS_100A_v5_Tree_5_HGL_Temp} &
\includegraphics[width=2.6in]{FIGURES/NBS_100A_v5_Tree_5_HGL_Height}
\end{tabular*}
\caption{Hot Gas Layer Temperature and Height for the NBS Multi-Room Test 100A.}
\label{NBS_100A_HGL}
\end{figure}

\begin{figure}[p]
\begin{tabular*}{\textwidth}{l@{\extracolsep{\fill}}r}
\includegraphics[width=2.6in]{FIGURES/NBS_100O_v5_Tree_1_HGL_Temp} &
\includegraphics[width=2.6in]{FIGURES/NBS_100O_v5_Tree_1_HGL_Height} \\
\includegraphics[width=2.6in]{FIGURES/NBS_100O_v5_Tree_3_HGL_Temp} &
\includegraphics[width=2.6in]{FIGURES/NBS_100O_v5_Tree_3_HGL_Height} \\
\includegraphics[width=2.6in]{FIGURES/NBS_100O_v5_Tree_4_HGL_Temp} &
\includegraphics[width=2.6in]{FIGURES/NBS_100O_v5_Tree_4_HGL_Height} \\
\includegraphics[width=2.6in]{FIGURES/NBS_100O_v5_Tree_5_HGL_Temp} &
\includegraphics[width=2.6in]{FIGURES/NBS_100O_v5_Tree_5_HGL_Height}
\end{tabular*}
\caption{Hot Gas Layer Temperature and Height for the NBS Multi-Room Test 100O.}
\label{NBS_100O_HGL}
\end{figure}

\begin{figure}[p]
\begin{tabular*}{\textwidth}{l@{\extracolsep{\fill}}r}
\includegraphics[width=2.6in]{FIGURES/NBS_100Z_v5_Tree_1_HGL_Temp} &
\includegraphics[width=2.6in]{FIGURES/NBS_100Z_v5_Tree_1_HGL_Height} \\
\includegraphics[width=2.6in]{FIGURES/NBS_100Z_v5_Tree_3_HGL_Temp} &
\includegraphics[width=2.6in]{FIGURES/NBS_100Z_v5_Tree_3_HGL_Height} \\
\includegraphics[width=2.6in]{FIGURES/NBS_100Z_v5_Tree_4_HGL_Temp} &
\includegraphics[width=2.6in]{FIGURES/NBS_100Z_v5_Tree_4_HGL_Height} \\
\includegraphics[width=2.6in]{FIGURES/NBS_100Z_v5_Tree_5_HGL_Temp} &
\includegraphics[width=2.6in]{FIGURES/NBS_100Z_v5_Tree_5_HGL_Height}
\end{tabular*}
\caption{Hot Gas Layer Temperature and Height for the NBS Multi-Room Test 100Z.}
\label{NBS_100Z_HGL}
\end{figure}

\clearpage





\chapter{Fire Plumes}

Plume temperature measurements are available from the VTT Large Hall and the FM/SNL series.
For all the other series of experiments, the temperature above the fire is not reported, or the fire plume
leans because of the flow pattern within the compartment, or the fire is positioned against a wall.
Only for the VTT and the FM/SNL series are the plumes relatively free from perturbations.

\section{VTT Large Hall Test Series}

These experiments consist of liquid fuel pan fires conducted in the middle of a large fire test hall.
Plume temperatures are measured at two heights above the fire, 6 m and 12 m.
The flames extend to about 4 m above the fire pan.

Photographs from the VTT tests are available. It is difficult to precisely measure the flame height,
but the photos and videos allow one to make estimates accurate to within a pan diameter.
Similarly, flame height in FDS is assessed using the visualization program Smokeview.
There are various ways to render the fire in Smokeview.  The
most direct method is to show, via three dimensional surface plots, the volume within which the energy from the fire is being released.
The other method is to show the stoichiometric iso-surface of the mixture fraction.
FDS tracks the fuel and oxygen via a single scalar variable called the mixture fraction.
The stoichiometric iso-surface is essentially a sheet on which combustion occurs.  The average vertical extent of either the volume
in which energy is being released or the stoichiometric mixture fraction iso-surface is the FDS predicted flame height.
Shown in Figure~\ref{FDS_Flame} are snapshots from the simulation of the 1.6 m diameter heptane pan fire.
The pan has been approximated as a square because of the requirement by FDS of rectangular geometry.
Figure~\ref{Simo_Photos} contains photographs of the actual fire.
The height of the visible flame in the photographs has been estimated to be between 2.4 and 3 pan diameters (3.8 m to 4.8 m).
The height of the simulated fire fluctuates from 5 m to 6 m during the peak heat release rate phase.


\section{FM/SNL Test Series}

In Tests 4 and 5, thermocouples are positioned near the ceiling directly over the fire pan.
In Test 21, the fire is located within an empty electrical cabinet, and the closest near ceiling thermocouple
is used to assess the plume temperature.  Note that in Test 5, the FDS plume temperature curve has been smoothed
to better assess the relative difference between the peak values of the model and the measurement.


\begin{figure}[p]
\begin{tabular*}{\textwidth}{l@{\extracolsep{\fill}}r}
\includegraphics[width=2.6in]{FIGURES/VTT_01_v5_Plume_Temperature} &
\includegraphics[width=2.6in]{FIGURES/FM_SNL_04_v5_Plume_Temperature} \\
\includegraphics[width=2.6in]{FIGURES/VTT_02_v5_Plume_Temperature} &
\includegraphics[width=2.6in]{FIGURES/FM_SNL_05_v5_Plume_Temperature} \\
\includegraphics[width=2.6in]{FIGURES/VTT_03_v5_Plume_Temperature} &
\includegraphics[width=2.6in]{FIGURES/FM_SNL_21_v5_Plume_Temperature}
\end{tabular*}
\caption{Plume Temperatures from the VTT Large Hall (Left) and FM/SNL Tests (Right).}
\label{VTT_FM_SNL_Plume}
\end{figure}

\clearpage





\chapter{Ceiling Jets and Device Activation}

FDS is a computational fluid dynamics (CFD) model and has no explicit ceiling jet model.
Rather, temperatures throughout the fire compartment are computed directly from the governing conservation equations.
Nevertheless, temperature measurements near the ceiling can be used to evaluate the model's ability to predict the flow of
hot gases across a relatively flat ceiling. Measurements for this category are available from the NIST/NRC and the FM/SNL series.

\section{NIST/NRC Test Series}

The thermocouple nearest the ceiling in Tree 7, located towards the back of the compartment,
has been chosen as a surrogate for the ceiling jet temperature.
Curiously, the difference between measured and predicted temperatures is noticeably greater for the open door tests.
Certainly, the open door changes the flow pattern of the exhaust gases.
However, the predicted HGL heights for the open door tests, shown in the previous section,
do not show a noticeable difference from their closed door counterparts.
The predicted HGL temperatures are only slightly less than those measured in the open door tests,
due in large part to the contribution of Tree 7 in the layer reduction calculation.

\begin{figure}[p]
\begin{tabular*}{\textwidth}{l@{\extracolsep{\fill}}r}
\includegraphics[width=2.6in]{FIGURES/NIST_NRC_01_v5_Ceiling_Jet} &
\includegraphics[width=2.6in]{FIGURES/NIST_NRC_07_v5_Ceiling_Jet} \\
\includegraphics[width=2.6in]{FIGURES/NIST_NRC_02_v5_Ceiling_Jet} &
\includegraphics[width=2.6in]{FIGURES/NIST_NRC_08_v5_Ceiling_Jet} \\
\includegraphics[width=2.6in]{FIGURES/NIST_NRC_04_v5_Ceiling_Jet} &
\includegraphics[width=2.6in]{FIGURES/NIST_NRC_10_v5_Ceiling_Jet} \\
\includegraphics[width=2.6in]{FIGURES/NIST_NRC_13_v5_Ceiling_Jet} &
\includegraphics[width=2.6in]{FIGURES/NIST_NRC_16_v5_Ceiling_Jet}
\end{tabular*}
\caption{Ceiling Jet Temperature for the NIST/NRC Series, closed door tests.}
\label{NIST_NRC_Jet_Closed}
\end{figure}

\begin{figure}[p]
\begin{tabular*}{\textwidth}{l@{\extracolsep{\fill}}r}
\includegraphics[width=2.6in]{FIGURES/NIST_NRC_17_v5_Ceiling_Jet} &
 \\
\includegraphics[width=2.6in]{FIGURES/NIST_NRC_03_v5_Ceiling_Jet} &
\includegraphics[width=2.6in]{FIGURES/NIST_NRC_09_v5_Ceiling_Jet} \\
\includegraphics[width=2.6in]{FIGURES/NIST_NRC_05_v5_Ceiling_Jet} &
\includegraphics[width=2.6in]{FIGURES/NIST_NRC_14_v5_Ceiling_Jet} \\
\includegraphics[width=2.6in]{FIGURES/NIST_NRC_15_v5_Ceiling_Jet} &
\includegraphics[width=2.6in]{FIGURES/NIST_NRC_18_v5_Ceiling_Jet}
\end{tabular*}
\caption{Ceiling Jet Temperature for the NIST/NRC Series, open door tests.}
\label{NIST_NRC_Jet_Open}
\end{figure}

\clearpage


\section{FM/SNL Test Series}

The near-ceiling thermocouples in Sectors 1 and 3 have been chosen as surrogates for the ceiling jet temperature.
The results are shown below.  The only noticeable discrepancy is in Test 21, and it is the same pattern that
was observed in the HGL temperature comparison for this test.





\section{UL/NFPRF Sprinkler, Vent, and Draft Curtain Experiments}
\label{UL_NFPRF:Results}

The ceiling jet is an important fire phenomenon because of the presence of automatic fire protection devices at the ceiling, like
sprinklers and smoke/heat vents. The results of the UL/NFPRF experiments provide useful data to assess the accuracy of FDS in predicting
the velocity and temperature near the ceiling, and consequently the resulting activation of sprinklers.
The UL/NFPRF test results (Series I) are summarized in Table~\ref{ULmatrix}, along with the predictions of FDS.

\begin{table}[h]
\begin{center}
\begin{tabular}{|c||c|c|c|c|c|c|c|c|}
\hline
\multicolumn{9}{|c|}{\bf Heptane Spray Burner Test Series I}  \\ \hline \hline
Test & Burner & Vent                    & \multicolumn{2}{|c|}{First Act. (s) } & \multicolumn{2}{|c|}{Total Acts.}  & Draft    & Heat Release Rate \\ \cline{4-7}
No.  & Pos.   & Operation               & Exp. & FDS                            & Exp.  & FDS                        & Curtains & MW @ s \\
\hline \hline
I-1   & B  & Closed                     & 65   & 53                             & 11   & 12     & Yes  & 4.4 @ 50  \\ \hline
I-2   & B  & Manual (0:40)              & 66   & 52                             & 12   & 8      & Yes  & 4.4 @ 50  \\ \hline
I-3   & B  & Manual (1:30)              & 64   & 53                             & 12   & 9      & Yes  & 4.4 @ 50  \\ \hline
I-4   & C  & Closed                     & 60   & 52                             & 10   & 11     & Yes  & 4.4 @ 50  \\ \hline
I-5   & C  & Manual (0:40)              & 72   & 52                             & 9    & 8      & Yes  & 4.4 @ 50  \\ \hline
I-6   & C  & Manual (1:30)              & 62   & 52                             & 8    & 8      & Yes  & 4.4 @ 50  \\ \hline
I-7   & C  & 74$^\circ$C link (DNO)     & 70   & 52                             & 10   & 11     & Yes  & 4.4 @ 50  \\ \hline
I-8   & B  & 74$^\circ$C link (9:26)    & 60   & 53                             & 11   & 12     & Yes  & 4.4 @ 50  \\ \hline
I-9   & D  & 74$^\circ$C link (DNO)     & 70   & 55                             & 12   & 15     & Yes  & 4.4 @ 50  \\ \hline
I-10  & D  & Manual (0:40)              & 72   & 54                             & 13   & 15     & Yes  & 4.4 @ 50  \\ \hline
I-11  & D  & 74$^\circ$C link (4:48)    & N/A  & N/A                            & N/A  & N/A    & Yes  & 4.4 @ 50  \\ \hline
I-12  & A  & Closed                     & 68   & 62                             & 14   & 14     & Yes  & 4.4 @ 50  \\ \hline
I-13  & A  & 74$^\circ$C link (1:04)    & 69   & 62                             & 5    & 13     & Yes  & 6.0 @ 60  \\ \hline
I-14  & A  & Manual (0:40)              & 74   & 136                            & 7    & 10     & Yes  & 5.8 @ 60  \\ \hline
I-15  & A  & Manual (1:30)              & 64   & 60                             & 5    & 9      & Yes  & 5.8 @ 60  \\ \hline
I-16  & A  & 74$^\circ$C link (1:46)    & 106  & 97                             & 4    & 7      & Yes  & 5.0 @ 110 \\ \hline
\hline
I-17  & B  & 100$^\circ$C link (DNO)    & 58   & 54                             & 4    & 4      & No   & 4.6 @ 50 \\ \hline
I-18  & C  & 100$^\circ$C link (DNO)    & 58   & 57                             & 4    & 4      & No   & 3.7 @ 50 \\ \hline
I-19  & A  & 100$^\circ$C link (10:00)  & 56   & 61                             & 10   & 5      & No   & 4.6 @ 50 \\ \hline
I-20  & A  & 74$^\circ$C link (1:20)    & 54   & 64                             & 4    & 4      & No   & 4.2 @ 50 \\ \hline
I-21  & C  & 74$^\circ$C link (7:00)    & 58   & 52                             & 10   & 4      & No   & 4.6 @ 50 \\ \hline
I-22  & D  & 100$^\circ$C link (DNO)    & 60   & 54                             & 6    & 9      & No   & 4.6 @ 50 \\ \hline
\end{tabular}
\end{center}
\caption[Results of the UL/NFPRF Experiments.]
{\bf Results of the UL/NFPRF Experiments. Note that DNO means
``Did Not Open''. Also note, the fires grew at a rate proportional
to the square of the time until a certain flow rate of fuel was achieved
at which time the flow rate was held steady. Thus, the ``Heat Release Rate''
was the size of the fire at the time when the fuel supply was leveled off.}
\label{ULmatrix}
\end{table}


\begin{figure}[ht]
\includegraphics[width=\textwidth]{FIGURES/UL_NFPRF_Scatter_Plot}
\caption{Measured vs. Predicted sprinkler activation times for the UL/NFPRF Test Series.}
\label{UL_NFPRF_Scatter_Plot}
\end{figure}











\chapter{Gas Velocity}

Gas velocity is often measured at compartment inlets and outlets as part of a global assessment of mass and
energy conservation.  This chapter contains measurements of gas velocity and related quantities.



\section{NIST/WTC Test Series}


\begin{figure}[p]
\begin{tabular*}{\textwidth}{l@{\extracolsep{\fill}}r}
\includegraphics[width=2.6in]{FIGURES/WTC_01_v5_Inlet_Velocity} &
\includegraphics[width=2.6in]{FIGURES/WTC_01_v5_Outlet_Velocity} \\
\includegraphics[width=2.6in]{FIGURES/WTC_02_v5_Inlet_Velocity} &
\includegraphics[width=2.6in]{FIGURES/WTC_02_v5_Outlet_Velocity} \\
\includegraphics[width=2.6in]{FIGURES/WTC_03_v5_Inlet_Velocity} &
\includegraphics[width=2.6in]{FIGURES/WTC_03_v5_Outlet_Velocity}
\end{tabular*}
\caption{Inlet and outlet velocity for NIST/WTC Tests 1, 2 and 3.}
\label{NIST_WTC_Velocity_1}
\end{figure}


\begin{figure}[p]
\begin{tabular*}{\textwidth}{l@{\extracolsep{\fill}}r}
\includegraphics[width=2.6in]{FIGURES/WTC_04_v5_Inlet_Velocity} &
\includegraphics[width=2.6in]{FIGURES/WTC_04_v5_Outlet_Velocity} \\
\includegraphics[width=2.6in]{FIGURES/WTC_05_v5_Inlet_Velocity} &
\includegraphics[width=2.6in]{FIGURES/WTC_05_v5_Outlet_Velocity} \\
\includegraphics[width=2.6in]{FIGURES/WTC_06_v5_Inlet_Velocity} &
\includegraphics[width=2.6in]{FIGURES/WTC_06_v5_Outlet_Velocity}
\end{tabular*}
\caption{Inlet and outlet velocity for NIST/WTC Tests 4, 5 and 6.}
\label{NIST_WTC_Velocity_2}
\end{figure}

\clearpage










\chapter{Gas Species and Smoke}

FDS uses a mixture fraction combustion model, meaning that all gas species within the compartment are
assumed to be functions of a single scalar variable.  FDS solves only one transport equation for this variable,
and reports gas concentrations at any given point at any given time by extracting
its value from a pre-computed ``look-up'' table.  For the major species, like carbon dioxide and oxygen,
the predictions are essentially an indicator of how well FDS is predicting the bulk transport of combustion products throughout the space.
For minor species, like carbon monoxide and soot, FDS version 4 does not account for changes in combustion efficiency,
relying only on fixed yields of CO and soot from the combustion process.
In reality, the generation rate of CO and soot change depending on the ventilation conditions in the compartment.


\section{NIST/WTC Test Series, Oxygen and CO$_2$}


\begin{figure}[p]
\begin{tabular*}{\textwidth}{l@{\extracolsep{\fill}}r}
\includegraphics[width=2.6in]{FIGURES/WTC_01_v5_Gas_Concentration} &
\includegraphics[width=2.6in]{FIGURES/WTC_02_v5_Gas_Concentration} \\
\includegraphics[width=2.6in]{FIGURES/WTC_03_v5_Gas_Concentration} &
\includegraphics[width=2.6in]{FIGURES/WTC_04_v5_Gas_Concentration} \\
\includegraphics[width=2.6in]{FIGURES/WTC_05_v5_Gas_Concentration} &
\includegraphics[width=2.6in]{FIGURES/WTC_06_v5_Gas_Concentration}
\end{tabular*}
\caption{Oxygen and CO$_2$ for the NIST/WTC Test Series.}
\label{NIST_WTC_Gas}
\end{figure}

\clearpage

\section{NIST/NRC Test Series, Oxygen and CO$_2$}

The following pages present comparisons of oxygen and carbon dioxide concentration predictions with measurement for the
NIST/NRC series. There were two oxygen measurements, one in the upper layer, one in the lower.  There was only one carbon
dioxide measurement in the upper layer.

\begin{figure}[p]
\begin{tabular*}{\textwidth}{l@{\extracolsep{\fill}}r}
\includegraphics[width=2.6in]{FIGURES/NIST_NRC_01_v5_Oxygen_Concentration} &
\includegraphics[width=2.6in]{FIGURES/NIST_NRC_07_v5_Oxygen_Concentration} \\
\includegraphics[width=2.6in]{FIGURES/NIST_NRC_02_v5_Oxygen_Concentration} &
\includegraphics[width=2.6in]{FIGURES/NIST_NRC_08_v5_Oxygen_Concentration} \\
\includegraphics[width=2.6in]{FIGURES/NIST_NRC_04_v5_Oxygen_Concentration} &
\includegraphics[width=2.6in]{FIGURES/NIST_NRC_10_v5_Oxygen_Concentration} \\
\includegraphics[width=2.6in]{FIGURES/NIST_NRC_13_v5_Oxygen_Concentration} &
\includegraphics[width=2.6in]{FIGURES/NIST_NRC_16_v5_Oxygen_Concentration}
\end{tabular*}
\caption{Oxygen and CO$_2$ for the NIST/NRC Series, closed door tests.}
\label{NIST_NRC_Gas_Closed}
\end{figure}

\begin{figure}[p]
\begin{tabular*}{\textwidth}{l@{\extracolsep{\fill}}r}
\includegraphics[width=2.6in]{FIGURES/NIST_NRC_17_v5_Oxygen_Concentration} &
 \\
\includegraphics[width=2.6in]{FIGURES/NIST_NRC_03_v5_Oxygen_Concentration} &
\includegraphics[width=2.6in]{FIGURES/NIST_NRC_09_v5_Oxygen_Concentration} \\
\includegraphics[width=2.6in]{FIGURES/NIST_NRC_05_v5_Oxygen_Concentration} &
\includegraphics[width=2.6in]{FIGURES/NIST_NRC_14_v5_Oxygen_Concentration} \\
\includegraphics[width=2.6in]{FIGURES/NIST_NRC_15_v5_Oxygen_Concentration} &
\includegraphics[width=2.6in]{FIGURES/NIST_NRC_18_v5_Oxygen_Concentration}
\end{tabular*}
\caption{Oxygen and CO$_2$ for the NIST/NRC Series, open door tests.}
\label{NIST_NRC_Gas_Open}
\end{figure}

\clearpage


\section{NIST/NRC Test Series, Smoke}

FDS treats smoke like all other combustion products, basically a tracer gas whose mass fraction is a function of the mixture fraction.
To model smoke movement, the user need only prescribe the smoke yield, that is, the fraction of the fuel mass that is
converted to smoke particulate.  For the simulations of the NIST/NRC tests, the smoke yield is specified as one of the test parameters.
Figure and Figure contain comparisons of measured and predicted smoke concentration at one measuring station in the upper layer.
There are two obvious trends in the figures: first, the predicted concentrations are about 50 \% higher than the measured
in the open door tests.  Second,
the predicted concentrations are roughly three times the measured concentrations in the closed door tests.
As a contrast, Figure displays the time history of CO concentration for 6 of the NIST/NRC tests.
Like smoke, the CO is specified in FDS via a fixed yield, measured along with smoke and reported in the test document.
The large differences between model and measurement seen in the smoke data do not appear in the CO data.

\begin{figure}[p]
\begin{tabular*}{\textwidth}{l@{\extracolsep{\fill}}r}
\includegraphics[width=2.6in]{FIGURES/NIST_NRC_01_v5_Smoke_Concentration} &
\includegraphics[width=2.6in]{FIGURES/NIST_NRC_07_v5_Smoke_Concentration} \\
\includegraphics[width=2.6in]{FIGURES/NIST_NRC_02_v5_Smoke_Concentration} &
\includegraphics[width=2.6in]{FIGURES/NIST_NRC_08_v5_Smoke_Concentration} \\
\includegraphics[width=2.6in]{FIGURES/NIST_NRC_04_v5_Smoke_Concentration} &
\includegraphics[width=2.6in]{FIGURES/NIST_NRC_10_v5_Smoke_Concentration} \\
\includegraphics[width=2.6in]{FIGURES/NIST_NRC_13_v5_Smoke_Concentration} &
\includegraphics[width=2.6in]{FIGURES/NIST_NRC_16_v5_Smoke_Concentration}
\end{tabular*}
\caption{Smoke concentration for the NIST/NRC Series, closed door tests.}
\label{NIST_NRC_Smoke_Closed}
\end{figure}

\begin{figure}[p]
\begin{tabular*}{\textwidth}{l@{\extracolsep{\fill}}r}
\includegraphics[width=2.6in]{FIGURES/NIST_NRC_17_v5_Smoke_Concentration} &
 \\
\includegraphics[width=2.6in]{FIGURES/NIST_NRC_03_v5_Smoke_Concentration} &
\includegraphics[width=2.6in]{FIGURES/NIST_NRC_09_v5_Smoke_Concentration} \\
\includegraphics[width=2.6in]{FIGURES/NIST_NRC_05_v5_Smoke_Concentration} &
\includegraphics[width=2.6in]{FIGURES/NIST_NRC_14_v5_Smoke_Concentration} \\
\includegraphics[width=2.6in]{FIGURES/NIST_NRC_15_v5_Smoke_Concentration} &
\includegraphics[width=2.6in]{FIGURES/NIST_NRC_18_v5_Smoke_Concentration}
\end{tabular*}
\caption{Smoke concentration for the NIST/NRC Series, open door tests.}
\label{NIST_NRC_Smoke_Open}
\end{figure}

\clearpage




\chapter{Pressure}

The pressure within the compartment was measured at a single point, near the floor.
In the simulations of the closed door tests, the compartment is assumed to leak via a small uniform flow spread
over the walls and ceiling.  The flow rate is calculated based on the assumption that the leakage rate is proportional
to the measured leakage area times the square root of compartment over-pressure.


\section{NIST/NRC Test Series}

Comparisons between measured and predicted pressures for the NIST/NRC Test Series are shown
in Figs.~\ref{NIST_NRC_Pressure_Closed} and \ref{NIST_NRC_Pressure_Open}.
For those tests in which the door to the compartment is
open, the over-pressures are only a few Pascals, whereas when the door is closed, the over-pressures are several hundred Pascals.
Note that in the closed door tests, there is often a dramatic drop in the predicted compartment pressure.
This is the result of the assumption in FDS that the heat release rate is decreased to zero in one second at the time
in the experiment when the fuel flow was stopped for safety reasons.  In reality, the fire did not extinguish
immediately because there was an excess of fuel in the pan following the flow stoppage.
For the purpose of model comparison, the peak over-pressures are differenced in the closed door tests,
and the peak (albeit small) under-pressures are compared in the open door tests.



\begin{figure}[p]
\begin{tabular*}{\textwidth}{l@{\extracolsep{\fill}}r}
\includegraphics[width=2.6in]{FIGURES/NIST_NRC_01_v5_Compartment_Pressure} &
\includegraphics[width=2.6in]{FIGURES/NIST_NRC_07_v5_Compartment_Pressure} \\
\includegraphics[width=2.6in]{FIGURES/NIST_NRC_02_v5_Compartment_Pressure} &
\includegraphics[width=2.6in]{FIGURES/NIST_NRC_08_v5_Compartment_Pressure} \\
\includegraphics[width=2.6in]{FIGURES/NIST_NRC_04_v5_Compartment_Pressure} &
\includegraphics[width=2.6in]{FIGURES/NIST_NRC_10_v5_Compartment_Pressure} \\
\includegraphics[width=2.6in]{FIGURES/NIST_NRC_13_v5_Compartment_Pressure} &
\includegraphics[width=2.6in]{FIGURES/NIST_NRC_16_v5_Compartment_Pressure}
\end{tabular*}
\caption{Compartment pressures for the NIST/NRC Series, closed door tests.}
\label{NIST_NRC_Pressure_Closed}
\end{figure}

\begin{figure}[p]
\begin{tabular*}{\textwidth}{l@{\extracolsep{\fill}}r}
\includegraphics[width=2.6in]{FIGURES/NIST_NRC_17_v5_Compartment_Pressure} &
   \\
\includegraphics[width=2.6in]{FIGURES/NIST_NRC_03_v5_Compartment_Pressure} &
\includegraphics[width=2.6in]{FIGURES/NIST_NRC_09_v5_Compartment_Pressure} \\
\includegraphics[width=2.6in]{FIGURES/NIST_NRC_05_v5_Compartment_Pressure} &
\includegraphics[width=2.6in]{FIGURES/NIST_NRC_14_v5_Compartment_Pressure} \\
\includegraphics[width=2.6in]{FIGURES/NIST_NRC_15_v5_Compartment_Pressure} &
\includegraphics[width=2.6in]{FIGURES/NIST_NRC_18_v5_Compartment_Pressure}
\end{tabular*}
\caption{Compartment pressures for the NIST/NRC Series, open door tests.}
\label{NIST_NRC_Pressure_Open}
\end{figure}

\clearpage






\chapter{Heat Flux and Surface Temperature}



\clearpage

\section{NIST/WTC Test Series, Steel Structural Members}

A single box column was installed in the test compartment, about 1~m away from the fire. The column was instrumented near its base (about
0.5~m from the floor, middle (1.5~m), and upper (2.5~m). Four measurements of steel (and insulation) temperatures were made at each location, for
each of its four sides.

\clearpage

\subsection{NIST/WTC Test Series, Column Steel Temperatures}

\vspace{1in}

\begin{figure}[h!]
\begin{tabular*}{\textwidth}{l@{\extracolsep{\fill}}r}
\includegraphics[width=2.6in]{FIGURES/WTC_01_v5_Upper_Column_Steel_Temp} &
\includegraphics[width=2.6in]{FIGURES/WTC_02_v5_Upper_Column_Steel_Temp} \\
\includegraphics[width=2.6in]{FIGURES/WTC_03_v5_Upper_Column_Steel_Temp} &
\includegraphics[width=2.6in]{FIGURES/WTC_04_v5_Upper_Column_Steel_Temp} \\
\includegraphics[width=2.6in]{FIGURES/WTC_05_v5_Upper_Column_Steel_Temp} &
\includegraphics[width=2.6in]{FIGURES/WTC_06_v5_Upper_Column_Steel_Temp}
\end{tabular*}
\caption{Upper Column Steel Temperatures for the NIST/WTC Test Series.}
\label{NIST_WTC_Upper_Column_Steel}
\end{figure}

\begin{figure}[p]
\begin{tabular*}{\textwidth}{l@{\extracolsep{\fill}}r}
\includegraphics[width=2.6in]{FIGURES/WTC_01_v5_Middle_Column_Steel_Temp} &
\includegraphics[width=2.6in]{FIGURES/WTC_02_v5_Middle_Column_Steel_Temp} \\
\includegraphics[width=2.6in]{FIGURES/WTC_03_v5_Middle_Column_Steel_Temp} &
\includegraphics[width=2.6in]{FIGURES/WTC_04_v5_Middle_Column_Steel_Temp} \\
\includegraphics[width=2.6in]{FIGURES/WTC_05_v5_Middle_Column_Steel_Temp} &
\includegraphics[width=2.6in]{FIGURES/WTC_06_v5_Middle_Column_Steel_Temp}
\end{tabular*}
\caption{Middle Column Steel Temperatures for the NIST/WTC Test Series.}
\label{NIST_WTC_Middle_Column_Steel}
\end{figure}

\begin{figure}[p]
\begin{tabular*}{\textwidth}{l@{\extracolsep{\fill}}r}
\includegraphics[width=2.6in]{FIGURES/WTC_01_v5_Lower_Column_Steel_Temp} &
\includegraphics[width=2.6in]{FIGURES/WTC_02_v5_Lower_Column_Steel_Temp} \\
\includegraphics[width=2.6in]{FIGURES/WTC_03_v5_Lower_Column_Steel_Temp} &
\includegraphics[width=2.6in]{FIGURES/WTC_04_v5_Lower_Column_Steel_Temp} \\
\includegraphics[width=2.6in]{FIGURES/WTC_05_v5_Lower_Column_Steel_Temp} &
\includegraphics[width=2.6in]{FIGURES/WTC_06_v5_Lower_Column_Steel_Temp}
\end{tabular*}
\caption{Lower Column Steel Temperatures for the NIST/WTC Test Series.}
\label{NIST_WTC_Lower_Column_Steel}
\end{figure}

\clearpage



\subsection{NIST/WTC Test Series, Truss A Steel Temperatures}

\vspace{1in}

\begin{figure}[h!]
\begin{tabular*}{\textwidth}{l@{\extracolsep{\fill}}r}
\includegraphics[width=2.6in]{FIGURES/WTC_01_v5_Truss_A_Upper_Steel_Temp} &
\includegraphics[width=2.6in]{FIGURES/WTC_02_v5_Truss_A_Upper_Steel_Temp} \\
\includegraphics[width=2.6in]{FIGURES/WTC_03_v5_Truss_A_Upper_Steel_Temp} &
\includegraphics[width=2.6in]{FIGURES/WTC_04_v5_Truss_A_Upper_Steel_Temp} \\
\includegraphics[width=2.6in]{FIGURES/WTC_05_v5_Truss_A_Upper_Steel_Temp} &
\includegraphics[width=2.6in]{FIGURES/WTC_06_v5_Truss_A_Upper_Steel_Temp}
\end{tabular*}
\caption{Truss A Upper Flange Steel Temperatures for the NIST/WTC Test Series.}
\label{NIST_WTC_Truss_A_Upper_Steel}
\end{figure}

\begin{figure}[p]
\begin{tabular*}{\textwidth}{l@{\extracolsep{\fill}}r}
\includegraphics[width=2.6in]{FIGURES/WTC_01_v5_Truss_A_Middle_Steel_Temp} &
\includegraphics[width=2.6in]{FIGURES/WTC_02_v5_Truss_A_Middle_Steel_Temp} \\
\includegraphics[width=2.6in]{FIGURES/WTC_03_v5_Truss_A_Middle_Steel_Temp} &
\includegraphics[width=2.6in]{FIGURES/WTC_04_v5_Truss_A_Middle_Steel_Temp} \\
\includegraphics[width=2.6in]{FIGURES/WTC_05_v5_Truss_A_Middle_Steel_Temp} &
\includegraphics[width=2.6in]{FIGURES/WTC_06_v5_Truss_A_Middle_Steel_Temp}
\end{tabular*}
\caption{Truss A Web Steel Temperatures for the NIST/WTC Test Series.}
\label{NIST_WTC_Truss_A_Middle_Steel}
\end{figure}

\begin{figure}[p]
\begin{tabular*}{\textwidth}{l@{\extracolsep{\fill}}r}
\includegraphics[width=2.6in]{FIGURES/WTC_01_v5_Truss_A_Lower_Steel_Temp} &
\includegraphics[width=2.6in]{FIGURES/WTC_02_v5_Truss_A_Lower_Steel_Temp} \\
\includegraphics[width=2.6in]{FIGURES/WTC_03_v5_Truss_A_Lower_Steel_Temp} &
\includegraphics[width=2.6in]{FIGURES/WTC_04_v5_Truss_A_Lower_Steel_Temp} \\
\includegraphics[width=2.6in]{FIGURES/WTC_05_v5_Truss_A_Lower_Steel_Temp} &
\includegraphics[width=2.6in]{FIGURES/WTC_06_v5_Truss_A_Lower_Steel_Temp}
\end{tabular*}
\caption{Truss A Lower Flange Steel Temperatures for the NIST/WTC Test Series.}
\label{NIST_WTC_Truss_A_Lower_Steel}
\end{figure}

\clearpage


\subsection{NIST/WTC Test Series, Truss B Steel Temperatures}

\vspace{1in}

\begin{figure}[h!]
\begin{tabular*}{\textwidth}{l@{\extracolsep{\fill}}r}
\includegraphics[width=2.6in]{FIGURES/WTC_01_v5_Truss_B_Upper_Steel_Temp} &
\includegraphics[width=2.6in]{FIGURES/WTC_02_v5_Truss_B_Upper_Steel_Temp} \\
\includegraphics[width=2.6in]{FIGURES/WTC_03_v5_Truss_B_Upper_Steel_Temp} &
\includegraphics[width=2.6in]{FIGURES/WTC_04_v5_Truss_B_Upper_Steel_Temp} \\
\includegraphics[width=2.6in]{FIGURES/WTC_05_v5_Truss_B_Upper_Steel_Temp} &
\includegraphics[width=2.6in]{FIGURES/WTC_06_v5_Truss_B_Upper_Steel_Temp}
\end{tabular*}
\caption{Truss B Upper Flange Steel Temperatures for the NIST/WTC Test Series.}
\label{NIST_WTC_Truss_B_Upper_Steel}
\end{figure}

\begin{figure}[p]
\begin{tabular*}{\textwidth}{l@{\extracolsep{\fill}}r}
\includegraphics[width=2.6in]{FIGURES/WTC_01_v5_Truss_B_Middle_Steel_Temp} &
\includegraphics[width=2.6in]{FIGURES/WTC_02_v5_Truss_B_Middle_Steel_Temp} \\
\includegraphics[width=2.6in]{FIGURES/WTC_03_v5_Truss_B_Middle_Steel_Temp} &
\includegraphics[width=2.6in]{FIGURES/WTC_04_v5_Truss_B_Middle_Steel_Temp} \\
\includegraphics[width=2.6in]{FIGURES/WTC_05_v5_Truss_B_Middle_Steel_Temp} &
\includegraphics[width=2.6in]{FIGURES/WTC_06_v5_Truss_B_Middle_Steel_Temp}
\end{tabular*}
\caption{Truss B Web Steel Temperatures for the NIST/WTC Test Series.}
\label{NIST_WTC_Truss_B_Middle_Steel}
\end{figure}

\begin{figure}[p]
\begin{tabular*}{\textwidth}{l@{\extracolsep{\fill}}r}
\includegraphics[width=2.6in]{FIGURES/WTC_01_v5_Truss_B_Lower_Steel_Temp} &
\includegraphics[width=2.6in]{FIGURES/WTC_02_v5_Truss_B_Lower_Steel_Temp} \\
\includegraphics[width=2.6in]{FIGURES/WTC_03_v5_Truss_B_Lower_Steel_Temp} &
\includegraphics[width=2.6in]{FIGURES/WTC_04_v5_Truss_B_Lower_Steel_Temp} \\
\includegraphics[width=2.6in]{FIGURES/WTC_05_v5_Truss_B_Lower_Steel_Temp} &
\includegraphics[width=2.6in]{FIGURES/WTC_06_v5_Truss_B_Lower_Steel_Temp}
\end{tabular*}
\caption{Truss B Lower Flange Steel Temperatures for the NIST/WTC Test Series.}
\label{NIST_WTC_Truss_B_Lower_Steel}
\end{figure}

\clearpage


\subsection{NIST/WTC Test Series, Steel Bar Temperatures}

\vspace{1in}

\begin{figure}[h!]
\begin{tabular*}{\textwidth}{l@{\extracolsep{\fill}}r}
\includegraphics[width=2.6in]{FIGURES/WTC_01_v5_Bar_1_Steel_Temp} &
\includegraphics[width=2.6in]{FIGURES/WTC_02_v5_Bar_1_Steel_Temp} \\
\includegraphics[width=2.6in]{FIGURES/WTC_03_v5_Bar_1_Steel_Temp} &
\includegraphics[width=2.6in]{FIGURES/WTC_04_v5_Bar_1_Steel_Temp} \\
\includegraphics[width=2.6in]{FIGURES/WTC_05_v5_Bar_1_Steel_Temp} &
\includegraphics[width=2.6in]{FIGURES/WTC_06_v5_Bar_1_Steel_Temp}
\end{tabular*}
\caption{Bar 1 Steel Temperatures for the NIST/WTC Test Series.}
\label{NIST_WTC Bar_1_Steel}
\end{figure}

\begin{figure}[h!]
\begin{tabular*}{\textwidth}{l@{\extracolsep{\fill}}r}
\includegraphics[width=2.6in]{FIGURES/WTC_01_v5_Bar_2_Steel_Temp} &
\includegraphics[width=2.6in]{FIGURES/WTC_02_v5_Bar_2_Steel_Temp} \\
\includegraphics[width=2.6in]{FIGURES/WTC_03_v5_Bar_2_Steel_Temp} &
 \\
\end{tabular*}
\caption{Bar 2 Steel Temperatures for NIST/WTC Tests 1, 2 and 3.}
\label{NIST_WTC Bar_2_Steel}
\end{figure}



\clearpage


\subsection{NIST/WTC Test Series, Nickel ``Slug'' Temperatures}

\vspace{1in}

\begin{figure}[h!]
\begin{tabular*}{\textwidth}{l@{\extracolsep{\fill}}r}
\includegraphics[width=2.6in]{FIGURES/WTC_01_v5_Slug_1_Temp} &
\includegraphics[width=2.6in]{FIGURES/WTC_02_v5_Slug_1_Temp} \\
\includegraphics[width=2.6in]{FIGURES/WTC_03_v5_Slug_1_Temp} &
\includegraphics[width=2.6in]{FIGURES/WTC_04_v5_Slug_1_Temp} \\
\includegraphics[width=2.6in]{FIGURES/WTC_05_v5_Slug_1_Temp} &
\includegraphics[width=2.6in]{FIGURES/WTC_06_v5_Slug_1_Temp}
\end{tabular*}
\caption{Nickel ``Slug'' 1 Temperatures for the NIST/WTC Test Series.}
\label{NIST_WTC Slug_1}
\end{figure}

\begin{figure}[h!]
\begin{tabular*}{\textwidth}{l@{\extracolsep{\fill}}r}
\includegraphics[width=2.6in]{FIGURES/WTC_01_v5_Slug_2_Temp} &
\includegraphics[width=2.6in]{FIGURES/WTC_02_v5_Slug_2_Temp} \\
\includegraphics[width=2.6in]{FIGURES/WTC_03_v5_Slug_2_Temp} &
\includegraphics[width=2.6in]{FIGURES/WTC_04_v5_Slug_2_Temp} \\
\includegraphics[width=2.6in]{FIGURES/WTC_05_v5_Slug_2_Temp} &
\includegraphics[width=2.6in]{FIGURES/WTC_06_v5_Slug_2_Temp}
\end{tabular*}
\caption{Nickel ``Slug'' 2 Temperatures for the NIST/WTC Test Series.}
\label{NIST_WTC Slug_2}
\end{figure}

\begin{figure}[h!]
\begin{tabular*}{\textwidth}{l@{\extracolsep{\fill}}r}
\includegraphics[width=2.6in]{FIGURES/WTC_01_v5_Slug_3_Temp} &
\includegraphics[width=2.6in]{FIGURES/WTC_02_v5_Slug_3_Temp} \\
\includegraphics[width=2.6in]{FIGURES/WTC_03_v5_Slug_3_Temp} &
\includegraphics[width=2.6in]{FIGURES/WTC_04_v5_Slug_3_Temp} \\
\includegraphics[width=2.6in]{FIGURES/WTC_05_v5_Slug_3_Temp} &
\includegraphics[width=2.6in]{FIGURES/WTC_06_v5_Slug_3_Temp}
\end{tabular*}
\caption{Nickel ``Slug'' 3 Temperatures for the NIST/WTC Test Series.}
\label{NIST_WTC Slug_3}
\end{figure}

\begin{figure}[h!]
\begin{tabular*}{\textwidth}{l@{\extracolsep{\fill}}r}
\includegraphics[width=2.6in]{FIGURES/WTC_01_v5_Slug_4_Temp} &
\includegraphics[width=2.6in]{FIGURES/WTC_02_v5_Slug_4_Temp} \\
\includegraphics[width=2.6in]{FIGURES/WTC_03_v5_Slug_4_Temp} &
\includegraphics[width=2.6in]{FIGURES/WTC_04_v5_Slug_4_Temp} \\
\includegraphics[width=2.6in]{FIGURES/WTC_05_v5_Slug_4_Temp} &
\includegraphics[width=2.6in]{FIGURES/WTC_06_v5_Slug_4_Temp}
\end{tabular*}
\caption{Nickel ``Slug'' 4 Temperatures for the NIST/WTC Test Series.}
\label{NIST_WTC Slug_4}
\end{figure}

\begin{figure}[h!]
\begin{tabular*}{\textwidth}{l@{\extracolsep{\fill}}r}
\includegraphics[width=2.6in]{FIGURES/WTC_01_v5_Slug_5_Temp} &
\includegraphics[width=2.6in]{FIGURES/WTC_02_v5_Slug_5_Temp} \\
\includegraphics[width=2.6in]{FIGURES/WTC_03_v5_Slug_5_Temp} &
\includegraphics[width=2.6in]{FIGURES/WTC_04_v5_Slug_5_Temp} \\
\includegraphics[width=2.6in]{FIGURES/WTC_05_v5_Slug_5_Temp} &
\includegraphics[width=2.6in]{FIGURES/WTC_06_v5_Slug_5_Temp}
\end{tabular*}
\caption{Nickel ``Slug'' 5 Temperatures for the NIST/WTC Test Series.}
\label{NIST_WTC Slug_5}
\end{figure}




\clearpage

\section{NIST/NRC Test Series, Cables}

Cables in various types (power and control), and configurations (horizontal, vertical, in trays or free-hanging), were installed in
the test compartment.
For each of the four cable targets considered, measurements of the local gas temperature, surface temperature, radiative heat flux,
and total heat flux are available.  The following pages display comparisons of these quantities for
Control Cable B, Horizontal Cable Tray D, Power Cable F and Vertical Cable Tray G.
FDS does not have a detailed solid phase model that can account for the heat transfer within the bundled,
cylindrical, non-homogenous cables.  For the bundled cables within horizontal and vertical trays (Targets D and G),
FDS assumes them to be rectangular slabs of thickness comparable to the diameter of the individual cables.
For the free-hanging cables B and F, FDS assumes them to be cylinders of uniform composition into which it
computes the radial heat transfer as a function of the heat flux to a designated location.
The superposition of gas temperature, heat flux and surface temperature in the figures on the following pages
provides information about how cables heat up in fires.  Favorable or unfavorable predictions of cable surface
temperatures can often be explained in terms of comparable errors in the prediction of the thermal environment in the vicinity of the cable.

\clearpage

\subsection{Free-Hanging Control Cable B}

\vspace{1in}

\begin{figure}[h!]
\begin{tabular*}{\textwidth}{l@{\extracolsep{\fill}}r}
\includegraphics[width=2.6in]{FIGURES/NIST_NRC_01_v5_B_Cable_Gas_Temp_4-8} &
\includegraphics[width=2.6in]{FIGURES/NIST_NRC_07_v5_B_Cable_Gas_Temp_4-8} \\
\includegraphics[width=2.6in]{FIGURES/NIST_NRC_01_v5_B_Cable_Heat_Flux} &
\includegraphics[width=2.6in]{FIGURES/NIST_NRC_07_v5_B_Cable_Heat_Flux} \\
\includegraphics[width=2.6in]{FIGURES/NIST_NRC_01_v5_B_Cable_TC} &
\includegraphics[width=2.6in]{FIGURES/NIST_NRC_07_v5_B_Cable_TC}
\end{tabular*}
\caption{NIST/NRC Series, Cable B, Replicate Tests 1 and 7.}
\label{NIST_NRC_B_1_and_7}
\end{figure}

\begin{figure}[h]
\begin{tabular*}{\textwidth}{l@{\extracolsep{\fill}}r}
\includegraphics[width=2.6in]{FIGURES/NIST_NRC_02_v5_B_Cable_Gas_Temp_4-8} &
\includegraphics[width=2.6in]{FIGURES/NIST_NRC_08_v5_B_Cable_Gas_Temp_4-8} \\
\includegraphics[width=2.6in]{FIGURES/NIST_NRC_02_v5_B_Cable_Heat_Flux} &
\includegraphics[width=2.6in]{FIGURES/NIST_NRC_08_v5_B_Cable_Heat_Flux} \\
\includegraphics[width=2.6in]{FIGURES/NIST_NRC_02_v5_B_Cable_TC} &
\includegraphics[width=2.6in]{FIGURES/NIST_NRC_08_v5_B_Cable_TC}
\end{tabular*}
\caption{NIST/NRC Series, Cable B, Replicate Tests 2 and 8.}
\label{NIST_NRC_B_2_and_8}
\end{figure}

\begin{figure}[h]
\begin{tabular*}{\textwidth}{l@{\extracolsep{\fill}}r}
\includegraphics[width=2.6in]{FIGURES/NIST_NRC_04_v5_B_Cable_Gas_Temp_4-8} &
\includegraphics[width=2.6in]{FIGURES/NIST_NRC_10_v5_B_Cable_Gas_Temp_4-8} \\
\includegraphics[width=2.6in]{FIGURES/NIST_NRC_04_v5_B_Cable_Heat_Flux} &
\includegraphics[width=2.6in]{FIGURES/NIST_NRC_10_v5_B_Cable_Heat_Flux} \\
\includegraphics[width=2.6in]{FIGURES/NIST_NRC_04_v5_B_Cable_TC} &
\includegraphics[width=2.6in]{FIGURES/NIST_NRC_10_v5_B_Cable_TC}
\end{tabular*}
\caption{NIST/NRC Series, Cable B, Replicate Tests 4 and 10.}
\label{NIST_NRC_B_4_and_10}
\end{figure}

\begin{figure}[h]
\begin{tabular*}{\textwidth}{l@{\extracolsep{\fill}}r}
\includegraphics[width=2.6in]{FIGURES/NIST_NRC_13_v5_B_Cable_Gas_Temp_4-8} &
\includegraphics[width=2.6in]{FIGURES/NIST_NRC_16_v5_B_Cable_Gas_Temp_4-8} \\
\includegraphics[width=2.6in]{FIGURES/NIST_NRC_13_v5_B_Cable_Heat_Flux} &
\includegraphics[width=2.6in]{FIGURES/NIST_NRC_16_v5_B_Cable_Heat_Flux} \\
\includegraphics[width=2.6in]{FIGURES/NIST_NRC_13_v5_B_Cable_TC} &
\includegraphics[width=2.6in]{FIGURES/NIST_NRC_16_v5_B_Cable_TC}
\end{tabular*}
\caption{NIST/NRC Series, Cable B, Tests 13 and 16.}
\label{NIST_NRC_B_13_and_16}
\end{figure}

\begin{figure}[h]
\begin{tabular*}{\textwidth}{l@{\extracolsep{\fill}}r}
\includegraphics[width=2.6in]{FIGURES/NIST_NRC_03_v5_B_Cable_Gas_Temp_4-8} &
\includegraphics[width=2.6in]{FIGURES/NIST_NRC_09_v5_B_Cable_Gas_Temp_4-8} \\
\includegraphics[width=2.6in]{FIGURES/NIST_NRC_03_v5_B_Cable_Heat_Flux} &
\includegraphics[width=2.6in]{FIGURES/NIST_NRC_09_v5_B_Cable_Heat_Flux} \\
\includegraphics[width=2.6in]{FIGURES/NIST_NRC_03_v5_B_Cable_TC} &
\includegraphics[width=2.6in]{FIGURES/NIST_NRC_09_v5_B_Cable_TC}
\end{tabular*}
\caption{NIST/NRC Series, Cable B, Replicate Tests 3 and 9.}
\label{NIST_NRC_B_3_and_9}
\end{figure}

\begin{figure}[h]
\begin{tabular*}{\textwidth}{l@{\extracolsep{\fill}}r}
\includegraphics[width=2.6in]{FIGURES/NIST_NRC_05_v5_B_Cable_Gas_Temp_4-8} &
\includegraphics[width=2.6in]{FIGURES/NIST_NRC_14_v5_B_Cable_Gas_Temp_4-8} \\
\includegraphics[width=2.6in]{FIGURES/NIST_NRC_05_v5_B_Cable_Heat_Flux} &
\includegraphics[width=2.6in]{FIGURES/NIST_NRC_14_v5_B_Cable_Heat_Flux} \\
\includegraphics[width=2.6in]{FIGURES/NIST_NRC_05_v5_B_Cable_TC} &
\includegraphics[width=2.6in]{FIGURES/NIST_NRC_14_v5_B_Cable_TC}
\end{tabular*}
\caption{NIST/NRC Series, Cable B, Tests 5 and 14.}
\label{NIST_NRC_B_5_and_14}
\end{figure}

\begin{figure}[h]
\begin{tabular*}{\textwidth}{l@{\extracolsep{\fill}}r}
\includegraphics[width=2.6in]{FIGURES/NIST_NRC_15_v5_B_Cable_Gas_Temp_4-8} &
\includegraphics[width=2.6in]{FIGURES/NIST_NRC_18_v5_B_Cable_Gas_Temp_4-8} \\
\includegraphics[width=2.6in]{FIGURES/NIST_NRC_15_v5_B_Cable_Heat_Flux} &
\includegraphics[width=2.6in]{FIGURES/NIST_NRC_18_v5_B_Cable_Heat_Flux} \\
\includegraphics[width=2.6in]{FIGURES/NIST_NRC_15_v5_B_Cable_TC} &
\includegraphics[width=2.6in]{FIGURES/NIST_NRC_18_v5_B_Cable_TC}
\end{tabular*}
\caption{NIST/NRC Series, Cable B, Tests 15 and 18.}
\label{NIST_NRC_B_15_and_18}
\end{figure}


\clearpage




\subsection{Control Cable D in a Tray}

\vspace{1in}

\begin{figure}[h!]
\begin{tabular*}{\textwidth}{l@{\extracolsep{\fill}}r}
\includegraphics[width=2.6in]{FIGURES/NIST_NRC_01_v5_D_Cable_Gas_Temp_3-9} &
\includegraphics[width=2.6in]{FIGURES/NIST_NRC_07_v5_D_Cable_Gas_Temp_3-9} \\
\includegraphics[width=2.6in]{FIGURES/NIST_NRC_01_v5_D_Cable_Heat_Flux} &
\includegraphics[width=2.6in]{FIGURES/NIST_NRC_07_v5_D_Cable_Heat_Flux} \\
\includegraphics[width=2.6in]{FIGURES/NIST_NRC_01_v5_D_Cable_TC} &
\includegraphics[width=2.6in]{FIGURES/NIST_NRC_07_v5_D_Cable_TC}
\end{tabular*}
\caption{NIST/NRC Series, Cable D, Replicate Tests 1 and 7.}
\label{NIST_NRC_D_1_and_7}
\end{figure}

\begin{figure}[h]
\begin{tabular*}{\textwidth}{l@{\extracolsep{\fill}}r}
\includegraphics[width=2.6in]{FIGURES/NIST_NRC_02_v5_D_Cable_Gas_Temp_3-9} &
\includegraphics[width=2.6in]{FIGURES/NIST_NRC_08_v5_D_Cable_Gas_Temp_3-9} \\
\includegraphics[width=2.6in]{FIGURES/NIST_NRC_02_v5_D_Cable_Heat_Flux} &
\includegraphics[width=2.6in]{FIGURES/NIST_NRC_08_v5_D_Cable_Heat_Flux} \\
\includegraphics[width=2.6in]{FIGURES/NIST_NRC_02_v5_D_Cable_TC} &
\includegraphics[width=2.6in]{FIGURES/NIST_NRC_08_v5_D_Cable_TC}
\end{tabular*}
\caption{NIST/NRC Series, Cable D, Replicate Tests 2 and 8.}
\label{NIST_NRC_D_2_and_8}
\end{figure}

\begin{figure}[h]
\begin{tabular*}{\textwidth}{l@{\extracolsep{\fill}}r}
\includegraphics[width=2.6in]{FIGURES/NIST_NRC_04_v5_D_Cable_Gas_Temp_3-9} &
\includegraphics[width=2.6in]{FIGURES/NIST_NRC_10_v5_D_Cable_Gas_Temp_3-9} \\
\includegraphics[width=2.6in]{FIGURES/NIST_NRC_04_v5_D_Cable_Heat_Flux} &
\includegraphics[width=2.6in]{FIGURES/NIST_NRC_10_v5_D_Cable_Heat_Flux} \\
\includegraphics[width=2.6in]{FIGURES/NIST_NRC_04_v5_D_Cable_TC} &
\includegraphics[width=2.6in]{FIGURES/NIST_NRC_10_v5_D_Cable_TC}
\end{tabular*}
\caption{NIST/NRC Series, Cable D, Replicate Tests 4 and 10.}
\label{NIST_NRC_D_4_and_10}
\end{figure}

\begin{figure}[h]
\begin{tabular*}{\textwidth}{l@{\extracolsep{\fill}}r}
\includegraphics[width=2.6in]{FIGURES/NIST_NRC_13_v5_D_Cable_Gas_Temp_3-9} &
\includegraphics[width=2.6in]{FIGURES/NIST_NRC_16_v5_D_Cable_Gas_Temp_3-9} \\
\includegraphics[width=2.6in]{FIGURES/NIST_NRC_13_v5_D_Cable_Heat_Flux} &
\includegraphics[width=2.6in]{FIGURES/NIST_NRC_16_v5_D_Cable_Heat_Flux} \\
\includegraphics[width=2.6in]{FIGURES/NIST_NRC_13_v5_D_Cable_TC} &
\includegraphics[width=2.6in]{FIGURES/NIST_NRC_16_v5_D_Cable_TC}
\end{tabular*}
\caption{NIST/NRC Series, Cable D, Tests 13 and 16.}
\label{NIST_NRC_D_13_and_16}
\end{figure}

\begin{figure}[h]
\begin{tabular*}{\textwidth}{l@{\extracolsep{\fill}}r}
\includegraphics[width=2.6in]{FIGURES/NIST_NRC_03_v5_D_Cable_Gas_Temp_3-9} &
\includegraphics[width=2.6in]{FIGURES/NIST_NRC_09_v5_D_Cable_Gas_Temp_3-9} \\
\includegraphics[width=2.6in]{FIGURES/NIST_NRC_03_v5_D_Cable_Heat_Flux} &
\includegraphics[width=2.6in]{FIGURES/NIST_NRC_09_v5_D_Cable_Heat_Flux} \\
\includegraphics[width=2.6in]{FIGURES/NIST_NRC_03_v5_D_Cable_TC} &
\includegraphics[width=2.6in]{FIGURES/NIST_NRC_09_v5_D_Cable_TC}
\end{tabular*}
\caption{NIST/NRC Series, Cable D, Replicate Tests 3 and 9.}
\label{NIST_NRC_D_3_and_9}
\end{figure}

\begin{figure}[h]
\begin{tabular*}{\textwidth}{l@{\extracolsep{\fill}}r}
\includegraphics[width=2.6in]{FIGURES/NIST_NRC_05_v5_D_Cable_Gas_Temp_3-9} &
\includegraphics[width=2.6in]{FIGURES/NIST_NRC_14_v5_D_Cable_Gas_Temp_3-9} \\
\includegraphics[width=2.6in]{FIGURES/NIST_NRC_05_v5_D_Cable_Heat_Flux} &
\includegraphics[width=2.6in]{FIGURES/NIST_NRC_14_v5_D_Cable_Heat_Flux} \\
\includegraphics[width=2.6in]{FIGURES/NIST_NRC_05_v5_D_Cable_TC} &
\includegraphics[width=2.6in]{FIGURES/NIST_NRC_14_v5_D_Cable_TC}
\end{tabular*}
\caption{NIST/NRC Series, Cable D, Tests 5 and 14.}
\label{NIST_NRC_D_5_and_14}
\end{figure}

\begin{figure}[h]
\begin{tabular*}{\textwidth}{l@{\extracolsep{\fill}}r}
\includegraphics[width=2.6in]{FIGURES/NIST_NRC_15_v5_D_Cable_Gas_Temp_3-9} &
\includegraphics[width=2.6in]{FIGURES/NIST_NRC_18_v5_D_Cable_Gas_Temp_3-9} \\
\includegraphics[width=2.6in]{FIGURES/NIST_NRC_15_v5_D_Cable_Heat_Flux} &
\includegraphics[width=2.6in]{FIGURES/NIST_NRC_18_v5_D_Cable_Heat_Flux} \\
\includegraphics[width=2.6in]{FIGURES/NIST_NRC_15_v5_D_Cable_TC} &
\includegraphics[width=2.6in]{FIGURES/NIST_NRC_18_v5_D_Cable_TC}
\end{tabular*}
\caption{NIST/NRC Series, Cable D, Tests 15 and 18.}
\label{NIST_NRC_D_15_and_18}
\end{figure}


\clearpage



\subsection{Free-Hanging Power Cable F}

\vspace{1in}

\begin{figure}[h!]
\begin{tabular*}{\textwidth}{l@{\extracolsep{\fill}}r}
\includegraphics[width=2.6in]{FIGURES/NIST_NRC_01_v5_F_Cable_Gas_Temp_5-6} &
\includegraphics[width=2.6in]{FIGURES/NIST_NRC_07_v5_F_Cable_Gas_Temp_5-6} \\
\includegraphics[width=2.6in]{FIGURES/NIST_NRC_01_v5_F_Cable_Heat_Flux} &
\includegraphics[width=2.6in]{FIGURES/NIST_NRC_07_v5_F_Cable_Heat_Flux} \\
\includegraphics[width=2.6in]{FIGURES/NIST_NRC_01_v5_F_Cable_TC} &
\includegraphics[width=2.6in]{FIGURES/NIST_NRC_07_v5_F_Cable_TC}
\end{tabular*}
\caption{NIST/NRC Series, Cable F, Replicate Tests 1 and 7.}
\label{NIST_NRC_F_1_and_7}
\end{figure}

\begin{figure}[h]
\begin{tabular*}{\textwidth}{l@{\extracolsep{\fill}}r}
\includegraphics[width=2.6in]{FIGURES/NIST_NRC_02_v5_F_Cable_Gas_Temp_5-6} &
\includegraphics[width=2.6in]{FIGURES/NIST_NRC_08_v5_F_Cable_Gas_Temp_5-6} \\
\includegraphics[width=2.6in]{FIGURES/NIST_NRC_02_v5_F_Cable_Heat_Flux} &
\includegraphics[width=2.6in]{FIGURES/NIST_NRC_08_v5_F_Cable_Heat_Flux} \\
\includegraphics[width=2.6in]{FIGURES/NIST_NRC_02_v5_F_Cable_TC} &
\includegraphics[width=2.6in]{FIGURES/NIST_NRC_08_v5_F_Cable_TC}
\end{tabular*}
\caption{NIST/NRC Series, Cable F, Replicate Tests 2 and 8.}
\label{NIST_NRC_F_2_and_8}
\end{figure}

\begin{figure}[h]
\begin{tabular*}{\textwidth}{l@{\extracolsep{\fill}}r}
\includegraphics[width=2.6in]{FIGURES/NIST_NRC_04_v5_F_Cable_Gas_Temp_5-6} &
\includegraphics[width=2.6in]{FIGURES/NIST_NRC_10_v5_F_Cable_Gas_Temp_5-6} \\
\includegraphics[width=2.6in]{FIGURES/NIST_NRC_04_v5_F_Cable_Heat_Flux} &
\includegraphics[width=2.6in]{FIGURES/NIST_NRC_10_v5_F_Cable_Heat_Flux} \\
\includegraphics[width=2.6in]{FIGURES/NIST_NRC_04_v5_F_Cable_TC} &
\includegraphics[width=2.6in]{FIGURES/NIST_NRC_10_v5_F_Cable_TC}
\end{tabular*}
\caption{NIST/NRC Series, Cable F, Replicate Tests 4 and 10.}
\label{NIST_NRC_F_4_and_10}
\end{figure}

\begin{figure}[h]
\begin{tabular*}{\textwidth}{l@{\extracolsep{\fill}}r}
\includegraphics[width=2.6in]{FIGURES/NIST_NRC_13_v5_F_Cable_Gas_Temp_5-6} &
\includegraphics[width=2.6in]{FIGURES/NIST_NRC_16_v5_F_Cable_Gas_Temp_5-6} \\
\includegraphics[width=2.6in]{FIGURES/NIST_NRC_13_v5_F_Cable_Heat_Flux} &
\includegraphics[width=2.6in]{FIGURES/NIST_NRC_16_v5_F_Cable_Heat_Flux} \\
\includegraphics[width=2.6in]{FIGURES/NIST_NRC_13_v5_F_Cable_TC} &
\includegraphics[width=2.6in]{FIGURES/NIST_NRC_16_v5_F_Cable_TC}
\end{tabular*}
\caption{NIST/NRC Series, Cable F, Tests 13 and 16.}
\label{NIST_NRC_F_13_and_16}
\end{figure}

\begin{figure}[h]
\begin{tabular*}{\textwidth}{l@{\extracolsep{\fill}}r}
\includegraphics[width=2.6in]{FIGURES/NIST_NRC_03_v5_F_Cable_Gas_Temp_5-6} &
\includegraphics[width=2.6in]{FIGURES/NIST_NRC_09_v5_F_Cable_Gas_Temp_5-6} \\
\includegraphics[width=2.6in]{FIGURES/NIST_NRC_03_v5_F_Cable_Heat_Flux} &
\includegraphics[width=2.6in]{FIGURES/NIST_NRC_09_v5_F_Cable_Heat_Flux} \\
\includegraphics[width=2.6in]{FIGURES/NIST_NRC_03_v5_F_Cable_TC} &
\includegraphics[width=2.6in]{FIGURES/NIST_NRC_09_v5_F_Cable_TC}
\end{tabular*}
\caption{NIST/NRC Series, Cable F, Replicate Tests 3 and 9.}
\label{NIST_NRC_F_3_and_9}
\end{figure}

\begin{figure}[h]
\begin{tabular*}{\textwidth}{l@{\extracolsep{\fill}}r}
\includegraphics[width=2.6in]{FIGURES/NIST_NRC_05_v5_F_Cable_Gas_Temp_5-6} &
\includegraphics[width=2.6in]{FIGURES/NIST_NRC_14_v5_F_Cable_Gas_Temp_5-6} \\
\includegraphics[width=2.6in]{FIGURES/NIST_NRC_05_v5_F_Cable_Heat_Flux} &
\includegraphics[width=2.6in]{FIGURES/NIST_NRC_14_v5_F_Cable_Heat_Flux} \\
\includegraphics[width=2.6in]{FIGURES/NIST_NRC_05_v5_F_Cable_TC} &
\includegraphics[width=2.6in]{FIGURES/NIST_NRC_14_v5_F_Cable_TC}
\end{tabular*}
\caption{NIST/NRC Series, Cable F, Tests 5 and 14.}
\label{NIST_NRC_F_5_and_14}
\end{figure}

\begin{figure}[h]
\begin{tabular*}{\textwidth}{l@{\extracolsep{\fill}}r}
\includegraphics[width=2.6in]{FIGURES/NIST_NRC_15_v5_F_Cable_Gas_Temp_5-6} &
\includegraphics[width=2.6in]{FIGURES/NIST_NRC_18_v5_F_Cable_Gas_Temp_5-6} \\
\includegraphics[width=2.6in]{FIGURES/NIST_NRC_15_v5_F_Cable_Heat_Flux} &
\includegraphics[width=2.6in]{FIGURES/NIST_NRC_18_v5_F_Cable_Heat_Flux} \\
\includegraphics[width=2.6in]{FIGURES/NIST_NRC_15_v5_F_Cable_TC} &
\includegraphics[width=2.6in]{FIGURES/NIST_NRC_18_v5_F_Cable_TC}
\end{tabular*}
\caption{NIST/NRC Series, Cable F, Tests 15 and 18.}
\label{NIST_NRC_F_15_and_18}
\end{figure}


\clearpage




\subsection{Vertical Cable Tray G}

\vspace{1in}

\begin{figure}[h!]
\begin{tabular*}{\textwidth}{l@{\extracolsep{\fill}}r}
\includegraphics[width=2.6in]{FIGURES/NIST_NRC_01_v5_G_Cable_Gas_Temp_2-5} &
\includegraphics[width=2.6in]{FIGURES/NIST_NRC_07_v5_G_Cable_Gas_Temp_2-5} \\
\includegraphics[width=2.6in]{FIGURES/NIST_NRC_01_v5_G_Cable_Heat_Flux} &
\includegraphics[width=2.6in]{FIGURES/NIST_NRC_07_v5_G_Cable_Heat_Flux} \\
\includegraphics[width=2.6in]{FIGURES/NIST_NRC_01_v5_G_Cable_TC} &
\includegraphics[width=2.6in]{FIGURES/NIST_NRC_07_v5_G_Cable_TC}
\end{tabular*}
\caption{NIST/NRC Series, Cable G, Replicate Tests 1 and 7.}
\label{NIST_NRC_G_1_and_7}
\end{figure}

\begin{figure}[h]
\begin{tabular*}{\textwidth}{l@{\extracolsep{\fill}}r}
\includegraphics[width=2.6in]{FIGURES/NIST_NRC_02_v5_G_Cable_Gas_Temp_2-5} &
\includegraphics[width=2.6in]{FIGURES/NIST_NRC_08_v5_G_Cable_Gas_Temp_2-5} \\
\includegraphics[width=2.6in]{FIGURES/NIST_NRC_02_v5_G_Cable_Heat_Flux} &
\includegraphics[width=2.6in]{FIGURES/NIST_NRC_08_v5_G_Cable_Heat_Flux} \\
\includegraphics[width=2.6in]{FIGURES/NIST_NRC_02_v5_G_Cable_TC} &
\includegraphics[width=2.6in]{FIGURES/NIST_NRC_08_v5_G_Cable_TC}
\end{tabular*}
\caption{NIST/NRC Series, Cable G, Replicate Tests 2 and 8.}
\label{NIST_NRC_G_2_and_8}
\end{figure}

\begin{figure}[h]
\begin{tabular*}{\textwidth}{l@{\extracolsep{\fill}}r}
\includegraphics[width=2.6in]{FIGURES/NIST_NRC_04_v5_G_Cable_Gas_Temp_2-5} &
\includegraphics[width=2.6in]{FIGURES/NIST_NRC_10_v5_G_Cable_Gas_Temp_2-5} \\
\includegraphics[width=2.6in]{FIGURES/NIST_NRC_04_v5_G_Cable_Heat_Flux} &
\includegraphics[width=2.6in]{FIGURES/NIST_NRC_10_v5_G_Cable_Heat_Flux} \\
\includegraphics[width=2.6in]{FIGURES/NIST_NRC_04_v5_G_Cable_TC} &
\includegraphics[width=2.6in]{FIGURES/NIST_NRC_10_v5_G_Cable_TC}
\end{tabular*}
\caption{NIST/NRC Series, Cable G, Replicate Tests 4 and 10.}
\label{NIST_NRC_G_4_and_10}
\end{figure}

\begin{figure}[h]
\begin{tabular*}{\textwidth}{l@{\extracolsep{\fill}}r}
\includegraphics[width=2.6in]{FIGURES/NIST_NRC_13_v5_G_Cable_Gas_Temp_2-5} &
\includegraphics[width=2.6in]{FIGURES/NIST_NRC_16_v5_G_Cable_Gas_Temp_2-5} \\
\includegraphics[width=2.6in]{FIGURES/NIST_NRC_13_v5_G_Cable_Heat_Flux} &
\includegraphics[width=2.6in]{FIGURES/NIST_NRC_16_v5_G_Cable_Heat_Flux} \\
\includegraphics[width=2.6in]{FIGURES/NIST_NRC_13_v5_G_Cable_TC} &
\includegraphics[width=2.6in]{FIGURES/NIST_NRC_16_v5_G_Cable_TC}
\end{tabular*}
\caption{NIST/NRC Series, Cable G, Tests 13 and 16.}
\label{NIST_NRC_G_13_and_16}
\end{figure}

\begin{figure}[h]
\begin{tabular*}{\textwidth}{l@{\extracolsep{\fill}}r}
\includegraphics[width=2.6in]{FIGURES/NIST_NRC_03_v5_G_Cable_Gas_Temp_2-5} &
\includegraphics[width=2.6in]{FIGURES/NIST_NRC_09_v5_G_Cable_Gas_Temp_2-5} \\
\includegraphics[width=2.6in]{FIGURES/NIST_NRC_03_v5_G_Cable_Heat_Flux} &
\includegraphics[width=2.6in]{FIGURES/NIST_NRC_09_v5_G_Cable_Heat_Flux} \\
\includegraphics[width=2.6in]{FIGURES/NIST_NRC_03_v5_G_Cable_TC} &
\includegraphics[width=2.6in]{FIGURES/NIST_NRC_09_v5_G_Cable_TC}
\end{tabular*}
\caption{NIST/NRC Series, Cable G, Replicate Tests 3 and 9.}
\label{NIST_NRC_G_3_and_9}
\end{figure}

\begin{figure}[h]
\begin{tabular*}{\textwidth}{l@{\extracolsep{\fill}}r}
\includegraphics[width=2.6in]{FIGURES/NIST_NRC_05_v5_G_Cable_Gas_Temp_2-5} &
\includegraphics[width=2.6in]{FIGURES/NIST_NRC_14_v5_G_Cable_Gas_Temp_2-5} \\
\includegraphics[width=2.6in]{FIGURES/NIST_NRC_05_v5_G_Cable_Heat_Flux} &
\includegraphics[width=2.6in]{FIGURES/NIST_NRC_14_v5_G_Cable_Heat_Flux} \\
\includegraphics[width=2.6in]{FIGURES/NIST_NRC_05_v5_G_Cable_TC} &
\includegraphics[width=2.6in]{FIGURES/NIST_NRC_14_v5_G_Cable_TC}
\end{tabular*}
\caption{NIST/NRC Series, Cable G, Tests 5 and 14.}
\label{NIST_NRC_G_5_and_14}
\end{figure}

\begin{figure}[h]
\begin{tabular*}{\textwidth}{l@{\extracolsep{\fill}}r}
\includegraphics[width=2.6in]{FIGURES/NIST_NRC_15_v5_G_Cable_Gas_Temp_2-5} &
\includegraphics[width=2.6in]{FIGURES/NIST_NRC_18_v5_G_Cable_Gas_Temp_2-5} \\
\includegraphics[width=2.6in]{FIGURES/NIST_NRC_15_v5_G_Cable_Heat_Flux} &
\includegraphics[width=2.6in]{FIGURES/NIST_NRC_18_v5_G_Cable_Heat_Flux} \\
\includegraphics[width=2.6in]{FIGURES/NIST_NRC_15_v5_G_Cable_TC} &
\includegraphics[width=2.6in]{FIGURES/NIST_NRC_18_v5_G_Cable_TC}
\end{tabular*}
\caption{NIST/NRC Series, Cable G, Tests 15 and 18.}
\label{NIST_NRC_G_15_and_18}
\end{figure}


\clearpage


\section{NIST/WTC Test Series, Heat Fluxes}

\begin{figure}[h]
\begin{tabular*}{\textwidth}{l@{\extracolsep{\fill}}r}
\includegraphics[width=2.6in]{FIGURES/WTC_01_v5_Floor_Heat_Flux} &
\includegraphics[width=2.6in]{FIGURES/WTC_02_v5_Floor_Heat_Flux} \\
\includegraphics[width=2.6in]{FIGURES/WTC_03_v5_Floor_Heat_Flux} &
\includegraphics[width=2.6in]{FIGURES/WTC_04_v5_Floor_Heat_Flux} \\
\includegraphics[width=2.6in]{FIGURES/WTC_05_v5_Floor_Heat_Flux} &
\includegraphics[width=2.6in]{FIGURES/WTC_06_v5_Floor_Heat_Flux}
\end{tabular*}
\caption{NIST/WTC Test Series, Heat Fluxes to the Floor.}
\label{NIST_WTC_Floor_Heat_Flux}
\end{figure}

\begin{figure}[h]
\begin{tabular*}{\textwidth}{l@{\extracolsep{\fill}}r}
\includegraphics[width=2.6in]{FIGURES/WTC_01_v5_High_Column_Heat_Flux} &
\includegraphics[width=2.6in]{FIGURES/WTC_02_v5_High_Column_Heat_Flux} \\
\includegraphics[width=2.6in]{FIGURES/WTC_03_v5_High_Column_Heat_Flux} &
\includegraphics[width=2.6in]{FIGURES/WTC_04_v5_High_Column_Heat_Flux} \\
\includegraphics[width=2.6in]{FIGURES/WTC_05_v5_High_Column_Heat_Flux} &
\includegraphics[width=2.6in]{FIGURES/WTC_06_v5_High_Column_Heat_Flux}
\end{tabular*}
\caption{NIST/WTC Test Series, Heat Fluxes to the Lower Column.}
\label{NIST_WTC_Low_Column_Heat_Flux}
\end{figure}

\begin{figure}[h]
\begin{tabular*}{\textwidth}{l@{\extracolsep{\fill}}r}
\includegraphics[width=2.6in]{FIGURES/WTC_01_v5_Low_Column_Heat_Flux} &
\includegraphics[width=2.6in]{FIGURES/WTC_02_v5_Low_Column_Heat_Flux} \\
\includegraphics[width=2.6in]{FIGURES/WTC_03_v5_Low_Column_Heat_Flux} &
\includegraphics[width=2.6in]{FIGURES/WTC_04_v5_Low_Column_Heat_Flux} \\
\includegraphics[width=2.6in]{FIGURES/WTC_05_v5_Low_Column_Heat_Flux} &
\includegraphics[width=2.6in]{FIGURES/WTC_06_v5_Low_Column_Heat_Flux}
\end{tabular*}
\caption{NIST/WTC Test Series, Heat Fluxes to the Upper Column.}
\label{NIST_WTC_High_Column_Heat_Flux}
\end{figure}

\begin{figure}[h]
\begin{tabular*}{\textwidth}{l@{\extracolsep{\fill}}r}
\includegraphics[width=2.6in]{FIGURES/WTC_01_v5_Flux_Station_2_Heat_Flux} &
\includegraphics[width=2.6in]{FIGURES/WTC_02_v5_Flux_Station_2_Heat_Flux} \\
\includegraphics[width=2.6in]{FIGURES/WTC_03_v5_Flux_Station_2_Heat_Flux} &
\includegraphics[width=2.6in]{FIGURES/WTC_04_v5_Flux_Station_2_Heat_Flux} \\
\includegraphics[width=2.6in]{FIGURES/WTC_05_v5_Flux_Station_2_Heat_Flux} &
\includegraphics[width=2.6in]{FIGURES/WTC_06_v5_Flux_Station_2_Heat_Flux}
\end{tabular*}
\caption{NIST/WTC Test Series, Heat Fluxes to Flux Station 2.}
\label{NIST_WTC_Flux_Station_2_Heat_Flux}
\end{figure}

\begin{figure}[h]
\begin{tabular*}{\textwidth}{l@{\extracolsep{\fill}}r}
\includegraphics[width=2.6in]{FIGURES/WTC_01_v5_Heat_Flux_to_Ceiling} &
\includegraphics[width=2.6in]{FIGURES/WTC_02_v5_Heat_Flux_to_Ceiling} \\
\includegraphics[width=2.6in]{FIGURES/WTC_03_v5_Heat_Flux_to_Ceiling} &
\includegraphics[width=2.6in]{FIGURES/WTC_04_v5_Heat_Flux_to_Ceiling} \\
\includegraphics[width=2.6in]{FIGURES/WTC_05_v5_Heat_Flux_to_Ceiling} &
\includegraphics[width=2.6in]{FIGURES/WTC_06_v5_Heat_Flux_to_Ceiling}
\end{tabular*}
\caption{NIST/WTC Test Series, Heat Fluxes to Ceiling.}
\label{NIST_WTC_Flux_Heat_Flux_to_Ceiling}
\end{figure}


\clearpage


\section{NIST/WTC Test Series, Ceiling and Wall Temperatures}


\begin{figure}[h!]
\begin{tabular*}{\textwidth}{l@{\extracolsep{\fill}}r}
\includegraphics[width=2.6in]{FIGURES/WTC_01_v5_North_Ceiling_Temperature} &
\includegraphics[width=2.6in]{FIGURES/WTC_02_v5_North_Ceiling_Temperature} \\
\includegraphics[width=2.6in]{FIGURES/WTC_03_v5_North_Ceiling_Temperature} &
\includegraphics[width=2.6in]{FIGURES/WTC_04_v5_North_Ceiling_Temperature} \\
\includegraphics[width=2.6in]{FIGURES/WTC_05_v5_North_Ceiling_Temperature} &
\includegraphics[width=2.6in]{FIGURES/WTC_06_v5_North_Ceiling_Temperature}
\end{tabular*}
\caption{North Ceiling Temperatures for the NIST/WTC Test Series.}
\label{NIST_WTC North_Ceiling_Temp}
\end{figure}

\begin{figure}[p]
\begin{tabular*}{\textwidth}{l@{\extracolsep{\fill}}r}
\includegraphics[width=2.6in]{FIGURES/WTC_01_v5_East_Ceiling_Temperature} &
\includegraphics[width=2.6in]{FIGURES/WTC_02_v5_East_Ceiling_Temperature} \\
\includegraphics[width=2.6in]{FIGURES/WTC_03_v5_East_Ceiling_Temperature} &
\includegraphics[width=2.6in]{FIGURES/WTC_04_v5_East_Ceiling_Temperature} \\
\includegraphics[width=2.6in]{FIGURES/WTC_05_v5_East_Ceiling_Temperature} &
\includegraphics[width=2.6in]{FIGURES/WTC_06_v5_East_Ceiling_Temperature}
\end{tabular*}
\caption{East Ceiling Temperatures for the NIST/WTC Test Series.}
\label{NIST_WTC East_Ceiling_Temp}
\end{figure}

\begin{figure}[p]
\begin{tabular*}{\textwidth}{l@{\extracolsep{\fill}}r}
\includegraphics[width=2.6in]{FIGURES/WTC_01_v5_West_Ceiling_Temperature} &
\includegraphics[width=2.6in]{FIGURES/WTC_02_v5_West_Ceiling_Temperature} \\
\includegraphics[width=2.6in]{FIGURES/WTC_03_v5_West_Ceiling_Temperature} &
\includegraphics[width=2.6in]{FIGURES/WTC_04_v5_West_Ceiling_Temperature} \\
\includegraphics[width=2.6in]{FIGURES/WTC_05_v5_West_Ceiling_Temperature} &
\includegraphics[width=2.6in]{FIGURES/WTC_06_v5_West_Ceiling_Temperature}
\end{tabular*}
\caption{West Ceiling Temperatures for the NIST/WTC Test Series.}
\label{NIST_WTC West_Ceiling_Temp}
\end{figure}

\begin{figure}[p]
\begin{tabular*}{\textwidth}{l@{\extracolsep{\fill}}r}
\includegraphics[width=2.6in]{FIGURES/WTC_01_v5_Inner_Ceiling_Temperature} &
\includegraphics[width=2.6in]{FIGURES/WTC_02_v5_Inner_Ceiling_Temperature} \\
\includegraphics[width=2.6in]{FIGURES/WTC_03_v5_Inner_Ceiling_Temperature} &
\includegraphics[width=2.6in]{FIGURES/WTC_04_v5_Inner_Ceiling_Temperature} \\
\includegraphics[width=2.6in]{FIGURES/WTC_05_v5_Inner_Ceiling_Temperature} &
\includegraphics[width=2.6in]{FIGURES/WTC_06_v5_Inner_Ceiling_Temperature}
\end{tabular*}
\caption{Inner Ceiling Temperatures for the NIST/WTC Test Series.}
\label{NIST_WTC Inner_Ceiling_Temp}
\end{figure}

\begin{figure}[p]
\begin{tabular*}{\textwidth}{l@{\extracolsep{\fill}}r}
\includegraphics[width=2.6in]{FIGURES/WTC_01_v5_Inner_Ceiling_Temperature_2} &
\includegraphics[width=2.6in]{FIGURES/WTC_02_v5_Inner_Ceiling_Temperature_2} \\
\includegraphics[width=2.6in]{FIGURES/WTC_03_v5_Inner_Ceiling_Temperature_2} &
\includegraphics[width=2.6in]{FIGURES/WTC_04_v5_Inner_Ceiling_Temperature_2} \\
\includegraphics[width=2.6in]{FIGURES/WTC_05_v5_Inner_Ceiling_Temperature_2} &
\includegraphics[width=2.6in]{FIGURES/WTC_06_v5_Inner_Ceiling_Temperature_2}
\end{tabular*}
\caption{Inner Ceiling Temperatures for the NIST/WTC Test Series.}
\label{NIST_WTC Inner_Ceiling_Temp_2}
\end{figure}

\begin{figure}[p]
\begin{tabular*}{\textwidth}{l@{\extracolsep{\fill}}r}
\includegraphics[width=2.6in]{FIGURES/WTC_01_v5_North_Wall_Temperature} &
\includegraphics[width=2.6in]{FIGURES/WTC_02_v5_North_Wall_Temperature} \\
\includegraphics[width=2.6in]{FIGURES/WTC_03_v5_North_Wall_Temperature} &
\includegraphics[width=2.6in]{FIGURES/WTC_04_v5_North_Wall_Temperature} \\
\includegraphics[width=2.6in]{FIGURES/WTC_05_v5_North_wall_Temperature} &
\includegraphics[width=2.6in]{FIGURES/WTC_06_v5_North_Wall_Temperature}
\end{tabular*}
\caption{North Wall Temperatures for the NIST/WTC Test Series.}
\label{NIST_WTC North_Wall_Temp}
\end{figure}

\begin{figure}[p]
\begin{tabular*}{\textwidth}{l@{\extracolsep{\fill}}r}
\includegraphics[width=2.6in]{FIGURES/WTC_01_v5_North_Wall_Temperature_2} &
\includegraphics[width=2.6in]{FIGURES/WTC_02_v5_North_Wall_Temperature_2} \\
\includegraphics[width=2.6in]{FIGURES/WTC_03_v5_North_Wall_Temperature_2} &
\includegraphics[width=2.6in]{FIGURES/WTC_04_v5_North_Wall_Temperature_2} \\
\includegraphics[width=2.6in]{FIGURES/WTC_05_v5_North_wall_Temperature_2} &
\includegraphics[width=2.6in]{FIGURES/WTC_06_v5_North_Wall_Temperature_2}
\end{tabular*}
\caption{North Wall Temperatures for the NIST/WTC Test Series.}
\label{NIST_WTC North_Wall_Temp_2}
\end{figure}

\begin{figure}[p]
\begin{tabular*}{\textwidth}{l@{\extracolsep{\fill}}r}
\includegraphics[width=2.6in]{FIGURES/WTC_01_v5_Inner_North_Wall_Temperature} &
\includegraphics[width=2.6in]{FIGURES/WTC_02_v5_Inner_North_Wall_Temperature} \\
\includegraphics[width=2.6in]{FIGURES/WTC_03_v5_Inner_North_Wall_Temperature} &
\includegraphics[width=2.6in]{FIGURES/WTC_04_v5_Inner_North_Wall_Temperature} \\
\includegraphics[width=2.6in]{FIGURES/WTC_05_v5_Inner_North_wall_Temperature} &
\includegraphics[width=2.6in]{FIGURES/WTC_06_v5_Inner_North_Wall_Temperature}
\end{tabular*}
\caption{Inner North Wall Temperatures for the NIST/WTC Test Series.}
\label{NIST_WTC Inner North_Wall_Temp}
\end{figure}

\clearpage




\section{NIST/NRC Test Series, Compartment Walls, Floor and Ceiling}

Thirty-six heat flux gauges were positioned at various locations on all four walls of the compartment,
plus the ceiling and floor.  Comparisons between measured and predicted heat fluxes and surface temperatures are shown
on the following pages for a selected number of locations.
Over half of the measurement points are in roughly the same relative location to the fire and hence
the measurements and predictions are similar.  For this reason, data for the east and north walls are shown
because the data from the south and west walls are comparable.  Data from the south wall is used in cases where
the corresponding instrument on the north wall failed, or in cases where the fire is positioned close to the south wall.
For each test, eight locations are used for comparison, two on the long (mainly north) wall,
two on the short (east) wall, two on the floor, and two on the ceiling.  Of the two locations for each panel,
one is considered in the far-field, relatively remote from the fire; one is in the near-field,
relatively close to the fire.  How close or far varies from test to test, depending on the availability of working flux gauges.
The two short wall locations are equally remote from the fire; thus, one location is in the lower layer, one in the upper.
Table lists the locations for each test.
The heat flux gauges used on the compartment walls measured the net, not total, heat flux.
FDS predicts the net heat flux, but this prediction cannot be compared directly with the measured net heat
flux because the predicted and measured wall temperatures can differ, and this affects the net heat flux.
In a sense, the net heat flux and surface temperature are coupled, and it is difficult to assess the accuracy of the models
if the two quantities cannot be decoupled.  For the purpose of comparing prediction and measurement,
the following correction has been applied to both the measured and predicted net heat fluxes:
\be  \dq_{\hbox{\tiny total}}'' = \dq_{\hbox{\tiny net}}'' + \sigma (T_s^4-T_\infty^4) + h (T_s - T_\infty) \ee
where $T_s$ is the temperature of the surface.  A constant convective heat transfer coefficient is assumed
(5 W/m$^2$/K) and an emissivity of 1.
After applying the correction, it is easier to compare total heat fluxes that are independent of the surface temperature.

\clearpage

\subsection{Long Wall}

\vspace{2in}


\begin{figure}[h!]
\begin{tabular*}{\textwidth}{l@{\extracolsep{\fill}}r}
\includegraphics[width=2.6in]{FIGURES/NIST_NRC_01_v5_Long_Wall_Flux_Gauges} &
\includegraphics[width=2.6in]{FIGURES/NIST_NRC_01_v5_Long_Wall_TC} \\
\includegraphics[width=2.6in]{FIGURES/NIST_NRC_07_v5_Long_Wall_Flux_Gauges} &
\includegraphics[width=2.6in]{FIGURES/NIST_NRC_07_v5_Long_Wall_TC}

\end{tabular*}
\caption{NIST/NRC Series, Long Wall Heat Flux and Temperature, Tests 1 and 7.}
\label{NIST_NRC_Long_1}
\end{figure}

\begin{figure}[p]
\begin{tabular*}{\textwidth}{l@{\extracolsep{\fill}}r}
\includegraphics[width=2.6in]{FIGURES/NIST_NRC_02_v5_Long_Wall_Flux_Gauges} &
\includegraphics[width=2.6in]{FIGURES/NIST_NRC_02_v5_Long_Wall_TC} \\
\includegraphics[width=2.6in]{FIGURES/NIST_NRC_08_v5_Long_Wall_Flux_Gauges} &
\includegraphics[width=2.6in]{FIGURES/NIST_NRC_08_v5_Long_Wall_TC} \\
\includegraphics[width=2.6in]{FIGURES/NIST_NRC_04_v5_Long_Wall_Flux_Gauges} &
\includegraphics[width=2.6in]{FIGURES/NIST_NRC_04_v5_Long_Wall_TC} \\
\includegraphics[width=2.6in]{FIGURES/NIST_NRC_10_v5_Long_Wall_Flux_Gauges} &
\includegraphics[width=2.6in]{FIGURES/NIST_NRC_10_v5_Long_Wall_TC}

\end{tabular*}
\caption{NIST/NRC Series, Long Wall Heat Flux and Temperature, Tests 2, 8, 4 and 10.}
\label{NIST_NRC_Long_2}
\end{figure}

\begin{figure}[p]
\begin{tabular*}{\textwidth}{l@{\extracolsep{\fill}}r}
\includegraphics[width=2.6in]{FIGURES/NIST_NRC_13_v5_Long_Wall_Flux_Gauges} &
\includegraphics[width=2.6in]{FIGURES/NIST_NRC_13_v5_Long_Wall_TC} \\
\includegraphics[width=2.6in]{FIGURES/NIST_NRC_16_v5_Long_Wall_Flux_Gauges} &
\includegraphics[width=2.6in]{FIGURES/NIST_NRC_16_v5_Long_Wall_TC} \\
\includegraphics[width=2.6in]{FIGURES/NIST_NRC_03_v5_Long_Wall_Flux_Gauges} &
\includegraphics[width=2.6in]{FIGURES/NIST_NRC_03_v5_Long_Wall_TC} \\
\includegraphics[width=2.6in]{FIGURES/NIST_NRC_09_v5_Long_Wall_Flux_Gauges} &
\includegraphics[width=2.6in]{FIGURES/NIST_NRC_09_v5_Long_Wall_TC}

\end{tabular*}
\caption{NIST/NRC Series, Long Wall Heat Flux and Temperature, Test 13, 16, 3 and 9.}
\label{NIST_NRC_Long_3}
\end{figure}

\begin{figure}[p]
\begin{tabular*}{\textwidth}{l@{\extracolsep{\fill}}r}
\includegraphics[width=2.6in]{FIGURES/NIST_NRC_05_v5_Long_Wall_Flux_Gauges} &
\includegraphics[width=2.6in]{FIGURES/NIST_NRC_05_v5_Long_Wall_TC} \\
\includegraphics[width=2.6in]{FIGURES/NIST_NRC_14_v5_Long_Wall_Flux_Gauges} &
\includegraphics[width=2.6in]{FIGURES/NIST_NRC_14_v5_Long_Wall_TC} \\
\includegraphics[width=2.6in]{FIGURES/NIST_NRC_15_v5_Long_Wall_Flux_Gauges} &
\includegraphics[width=2.6in]{FIGURES/NIST_NRC_15_v5_Long_Wall_TC} \\
\includegraphics[width=2.6in]{FIGURES/NIST_NRC_18_v5_Long_Wall_Flux_Gauges} &
\includegraphics[width=2.6in]{FIGURES/NIST_NRC_18_v5_Long_Wall_TC}
\end{tabular*}
\caption{NIST/NRC Series, Long Wall Heat Flux and Temperature, Tests 5, 14, 15 and 18.}
\label{NIST_NRC_Long_4}
\end{figure}

\clearpage



\subsection{Short Wall}

\vspace{2in}


\begin{figure}[h!]
\begin{tabular*}{\textwidth}{l@{\extracolsep{\fill}}r}
\includegraphics[width=2.6in]{FIGURES/NIST_NRC_01_v5_Short_Wall_Flux_Gauges} &
\includegraphics[width=2.6in]{FIGURES/NIST_NRC_01_v5_Short_Wall_TC} \\
\includegraphics[width=2.6in]{FIGURES/NIST_NRC_07_v5_Short_Wall_Flux_Gauges} &
\includegraphics[width=2.6in]{FIGURES/NIST_NRC_07_v5_Short_Wall_TC}

\end{tabular*}
\caption{NIST/NRC Series, Short Wall Heat Flux and Temperature, Tests 1 and 7.}
\label{NIST_NRC_Short_1}
\end{figure}

\begin{figure}[p]
\begin{tabular*}{\textwidth}{l@{\extracolsep{\fill}}r}
\includegraphics[width=2.6in]{FIGURES/NIST_NRC_02_v5_Short_Wall_Flux_Gauges} &
\includegraphics[width=2.6in]{FIGURES/NIST_NRC_02_v5_Short_Wall_TC} \\
\includegraphics[width=2.6in]{FIGURES/NIST_NRC_08_v5_Short_Wall_Flux_Gauges} &
\includegraphics[width=2.6in]{FIGURES/NIST_NRC_08_v5_Short_Wall_TC} \\
\includegraphics[width=2.6in]{FIGURES/NIST_NRC_04_v5_Short_Wall_Flux_Gauges} &
\includegraphics[width=2.6in]{FIGURES/NIST_NRC_04_v5_Short_Wall_TC} \\
\includegraphics[width=2.6in]{FIGURES/NIST_NRC_10_v5_Short_Wall_Flux_Gauges} &
\includegraphics[width=2.6in]{FIGURES/NIST_NRC_10_v5_Short_Wall_TC}

\end{tabular*}
\caption{NIST/NRC Series, Short Wall Heat Flux and Temperature, Tests 2, 8, 4 and 10.}
\label{NIST_NRC_Short_2}
\end{figure}

\begin{figure}[p]
\begin{tabular*}{\textwidth}{l@{\extracolsep{\fill}}r}
\includegraphics[width=2.6in]{FIGURES/NIST_NRC_13_v5_Short_Wall_Flux_Gauges} &
\includegraphics[width=2.6in]{FIGURES/NIST_NRC_13_v5_Short_Wall_TC} \\
\includegraphics[width=2.6in]{FIGURES/NIST_NRC_16_v5_Short_Wall_Flux_Gauges} &
\includegraphics[width=2.6in]{FIGURES/NIST_NRC_16_v5_Short_Wall_TC} \\
\includegraphics[width=2.6in]{FIGURES/NIST_NRC_03_v5_Short_Wall_Flux_Gauges} &
\includegraphics[width=2.6in]{FIGURES/NIST_NRC_03_v5_Short_Wall_TC} \\
\includegraphics[width=2.6in]{FIGURES/NIST_NRC_09_v5_Short_Wall_Flux_Gauges} &
\includegraphics[width=2.6in]{FIGURES/NIST_NRC_09_v5_Short_Wall_TC}

\end{tabular*}
\caption{NIST/NRC Series, Short Wall Heat Flux and Temperature, Test 13, 16, 3 and 9.}
\label{NIST_NRC_Short_3}
\end{figure}

\begin{figure}[p]
\begin{tabular*}{\textwidth}{l@{\extracolsep{\fill}}r}
\includegraphics[width=2.6in]{FIGURES/NIST_NRC_05_v5_Short_Wall_Flux_Gauges} &
\includegraphics[width=2.6in]{FIGURES/NIST_NRC_05_v5_Short_Wall_TC} \\
\includegraphics[width=2.6in]{FIGURES/NIST_NRC_14_v5_Short_Wall_Flux_Gauges} &
\includegraphics[width=2.6in]{FIGURES/NIST_NRC_14_v5_Short_Wall_TC} \\
\includegraphics[width=2.6in]{FIGURES/NIST_NRC_15_v5_Short_Wall_Flux_Gauges} &
\includegraphics[width=2.6in]{FIGURES/NIST_NRC_15_v5_Short_Wall_TC} \\
\includegraphics[width=2.6in]{FIGURES/NIST_NRC_18_v5_Short_Wall_Flux_Gauges} &
\includegraphics[width=2.6in]{FIGURES/NIST_NRC_18_v5_Short_Wall_TC}
\end{tabular*}
\caption{NIST/NRC Series, Short Wall Heat Flux and Temperature, Tests 5, 14, 15 and 18.}
\label{NIST_NRC_Short_4}
\end{figure}

\clearpage



\subsection{Ceiling}

\vspace{2in}


\begin{figure}[h!]
\begin{tabular*}{\textwidth}{l@{\extracolsep{\fill}}r}
\includegraphics[width=2.6in]{FIGURES/NIST_NRC_01_v5_Ceiling_Flux_Gauges} &
\includegraphics[width=2.6in]{FIGURES/NIST_NRC_01_v5_Ceiling_TC} \\
\includegraphics[width=2.6in]{FIGURES/NIST_NRC_07_v5_Ceiling_Flux_Gauges} &
\includegraphics[width=2.6in]{FIGURES/NIST_NRC_07_v5_Ceiling_TC}

\end{tabular*}
\caption{NIST/NRC Series, Ceiling Heat Flux and Temperature, Tests 1 and 7.}
\label{NIST_NRC_Ceiling_1}
\end{figure}

\begin{figure}[p]
\begin{tabular*}{\textwidth}{l@{\extracolsep{\fill}}r}
\includegraphics[width=2.6in]{FIGURES/NIST_NRC_02_v5_Ceiling_Flux_Gauges} &
\includegraphics[width=2.6in]{FIGURES/NIST_NRC_02_v5_Ceiling_TC} \\
\includegraphics[width=2.6in]{FIGURES/NIST_NRC_08_v5_Ceiling_Flux_Gauges} &
\includegraphics[width=2.6in]{FIGURES/NIST_NRC_08_v5_Ceiling_TC} \\
\includegraphics[width=2.6in]{FIGURES/NIST_NRC_04_v5_Ceiling_Flux_Gauges} &
\includegraphics[width=2.6in]{FIGURES/NIST_NRC_04_v5_Ceiling_TC} \\
\includegraphics[width=2.6in]{FIGURES/NIST_NRC_10_v5_Ceiling_Flux_Gauges} &
\includegraphics[width=2.6in]{FIGURES/NIST_NRC_10_v5_Ceiling_TC}

\end{tabular*}
\caption{NIST/NRC Series, Ceiling Heat Flux and Temperature, Tests 2, 8, 4 and 10.}
\label{NIST_NRC_Ceiling_2}
\end{figure}

\begin{figure}[p]
\begin{tabular*}{\textwidth}{l@{\extracolsep{\fill}}r}
\includegraphics[width=2.6in]{FIGURES/NIST_NRC_13_v5_Ceiling_Flux_Gauges} &
\includegraphics[width=2.6in]{FIGURES/NIST_NRC_13_v5_Ceiling_TC} \\
\includegraphics[width=2.6in]{FIGURES/NIST_NRC_16_v5_Ceiling_Flux_Gauges} &
\includegraphics[width=2.6in]{FIGURES/NIST_NRC_16_v5_Ceiling_TC} \\
\includegraphics[width=2.6in]{FIGURES/NIST_NRC_03_v5_Ceiling_Flux_Gauges} &
\includegraphics[width=2.6in]{FIGURES/NIST_NRC_03_v5_Ceiling_TC} \\
\includegraphics[width=2.6in]{FIGURES/NIST_NRC_09_v5_Ceiling_Flux_Gauges} &
\includegraphics[width=2.6in]{FIGURES/NIST_NRC_09_v5_Ceiling_TC}

\end{tabular*}
\caption{NIST/NRC Series, Ceiling Heat Flux and Temperature, Test 13, 16, 3 and 9.}
\label{NIST_NRC_Ceiling_3}
\end{figure}

\begin{figure}[p]
\begin{tabular*}{\textwidth}{l@{\extracolsep{\fill}}r}
\includegraphics[width=2.6in]{FIGURES/NIST_NRC_05_v5_Ceiling_Flux_Gauges} &
\includegraphics[width=2.6in]{FIGURES/NIST_NRC_05_v5_Ceiling_TC} \\
\includegraphics[width=2.6in]{FIGURES/NIST_NRC_14_v5_Ceiling_Flux_Gauges} &
\includegraphics[width=2.6in]{FIGURES/NIST_NRC_14_v5_Ceiling_TC} \\
\includegraphics[width=2.6in]{FIGURES/NIST_NRC_15_v5_Ceiling_Flux_Gauges} &
\includegraphics[width=2.6in]{FIGURES/NIST_NRC_15_v5_Ceiling_TC} \\
\includegraphics[width=2.6in]{FIGURES/NIST_NRC_18_v5_Ceiling_Flux_Gauges} &
\includegraphics[width=2.6in]{FIGURES/NIST_NRC_18_v5_Ceiling_TC}
\end{tabular*}
\caption{NIST/NRC Series, Ceiling Heat Flux and Temperature, Tests 5, 14, 15 and 18.}
\label{NIST_NRC_Ceiling_4}
\end{figure}

\clearpage



\subsection{Floor}

\vspace{2in}


\begin{figure}[h!]
\begin{tabular*}{\textwidth}{l@{\extracolsep{\fill}}r}
\includegraphics[width=2.6in]{FIGURES/NIST_NRC_01_v5_Floor_Flux_Gauges} &
\includegraphics[width=2.6in]{FIGURES/NIST_NRC_01_v5_Floor_TC} \\
\includegraphics[width=2.6in]{FIGURES/NIST_NRC_07_v5_Floor_Flux_Gauges} &
\includegraphics[width=2.6in]{FIGURES/NIST_NRC_07_v5_Floor_TC}

\end{tabular*}
\caption{NIST/NRC Series, Floor Heat Flux and Temperature, Tests 1 and 7.}
\label{NIST_NRC_Floor_1}
\end{figure}

\begin{figure}[p]
\begin{tabular*}{\textwidth}{l@{\extracolsep{\fill}}r}
\includegraphics[width=2.6in]{FIGURES/NIST_NRC_02_v5_Floor_Flux_Gauges} &
\includegraphics[width=2.6in]{FIGURES/NIST_NRC_02_v5_Floor_TC} \\
\includegraphics[width=2.6in]{FIGURES/NIST_NRC_08_v5_Floor_Flux_Gauges} &
\includegraphics[width=2.6in]{FIGURES/NIST_NRC_08_v5_Floor_TC} \\
\includegraphics[width=2.6in]{FIGURES/NIST_NRC_04_v5_Floor_Flux_Gauges} &
\includegraphics[width=2.6in]{FIGURES/NIST_NRC_04_v5_Floor_TC} \\
\includegraphics[width=2.6in]{FIGURES/NIST_NRC_10_v5_Floor_Flux_Gauges} &
\includegraphics[width=2.6in]{FIGURES/NIST_NRC_10_v5_Floor_TC}

\end{tabular*}
\caption{NIST/NRC Series, Floor Heat Flux and Temperature, Tests 2, 8, 4 and 10.}
\label{NIST_NRC_Floor_2}
\end{figure}

\begin{figure}[p]
\begin{tabular*}{\textwidth}{l@{\extracolsep{\fill}}r}
\includegraphics[width=2.6in]{FIGURES/NIST_NRC_13_v5_Floor_Flux_Gauges} &
\includegraphics[width=2.6in]{FIGURES/NIST_NRC_13_v5_Floor_TC} \\
\includegraphics[width=2.6in]{FIGURES/NIST_NRC_16_v5_Floor_Flux_Gauges} &
\includegraphics[width=2.6in]{FIGURES/NIST_NRC_16_v5_Floor_TC} \\
\includegraphics[width=2.6in]{FIGURES/NIST_NRC_03_v5_Floor_Flux_Gauges} &
\includegraphics[width=2.6in]{FIGURES/NIST_NRC_03_v5_Floor_TC} \\
\includegraphics[width=2.6in]{FIGURES/NIST_NRC_09_v5_Floor_Flux_Gauges} &
\includegraphics[width=2.6in]{FIGURES/NIST_NRC_09_v5_Floor_TC}

\end{tabular*}
\caption{NIST/NRC Series, Floor Heat Flux and Temperature, Test 13, 16, 3 and 9.}
\label{NIST_NRC_Floor_3}
\end{figure}

\begin{figure}[p]
\begin{tabular*}{\textwidth}{l@{\extracolsep{\fill}}r}
\includegraphics[width=2.6in]{FIGURES/NIST_NRC_05_v5_Floor_Flux_Gauges} &
\includegraphics[width=2.6in]{FIGURES/NIST_NRC_05_v5_Floor_TC} \\
\includegraphics[width=2.6in]{FIGURES/NIST_NRC_14_v5_Floor_Flux_Gauges} &
\includegraphics[width=2.6in]{FIGURES/NIST_NRC_14_v5_Floor_TC} \\
\includegraphics[width=2.6in]{FIGURES/NIST_NRC_15_v5_Floor_Flux_Gauges} &
\includegraphics[width=2.6in]{FIGURES/NIST_NRC_15_v5_Floor_TC} \\
\includegraphics[width=2.6in]{FIGURES/NIST_NRC_18_v5_Floor_Flux_Gauges} &
\includegraphics[width=2.6in]{FIGURES/NIST_NRC_18_v5_Floor_TC}
\end{tabular*}
\caption{NIST/NRC Series, Floor Heat Flux and Temperature, Tests 5, 14, 15 and 18.}
\label{NIST_NRC_Floor_4}
\end{figure}

\clearpage









\chapter{Conclusion}



\backmatter


\bibliography{../Bibliography/FDSVVBiB}

\addcontentsline{toc}{chapter}{References}

\end{document}
