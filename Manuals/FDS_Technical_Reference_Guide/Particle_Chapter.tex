% !TEX root = FDS_Technical_Reference_Guide.tex

\chapter{Lagrangian Particles}
\label{chapter:lagrangian_particles}

Lagrangian particles are used to represent a wide variety of objects that cannot be resolved on
the numerical grid. Liquid droplets are the most common example. This chapter outlines the treatment of the transport, size
distribution, and mass, momentum and energy transfer to and from Lagrangian particles.

\section{Particle Transport in the Gas Phase}

In the gas phase momentum equation, Eq.~(\ref{momentum}), the force term $\bof_{\rm b}$ represents the momentum transferred from particles to the gas.
It is obtained by summing the force transferred from each particle
in a grid cell and dividing by the cell volume, $V$:
\be
    {\bof_{\rm b}} = \frac{1}{V} \sum \left[ \ha \rho C_{\rm d} \, \pi \, r_{\rm p}^2 \, (\bu_{\rm p}-\bu) |\bu_{\rm p}-\bu| - \frac{\d m_{\rm p}}{\d t}(\bu_{\rm p}-\bu) \right]
\ee
where $C_{\rm d}$ is the drag coefficient, $r_{\rm p}$ is the particle radius, $\bu_{\rm p}$ is the particle velocity, $\bu$ is the
gas velocity, and $\rho$ is the gas density. The particle acceleration is given by
\be
    \frac{\d \bu_{\rm p}}{\d t} = \bg - \ha \frac{\rho\,C_{\rm d} \, \pi  r_{\rm p}^2}{m_{\rm p}} \,
    (\bu_{\rm p}-\bu) |\bu_{\rm p}-\bu|
\ee
where $m_{\rm p}$ is the particle mass.  The particle position, $\bx_{\rm p}$, is determined from the equation
\be
    \frac{\d \bx_{\rm p}}{\d t} = \bu_{\rm p}
\ee
The drag coefficient (default based on a sphere) is a function of the local Reynolds number that is based on the particle diameter, $D$
\begin{eqnarray}
 C_{\rm d} &=& \left\{ \begin{array}{ll}
     24/\RE_D                                          & \RE_D < 1    \\ [0.1in]
     24\left(0.85+0.15 \, \RE_D^{0.687} \right)/\RE_D  & 1 < \RE_D < 1000 \\ [0.1in]
     0.44                                              & 1000 < \RE_D
     \end{array} \right.  \\ [0.2in]
\RE_D &=& \frac{\rho \, |\bu_{\rm p}-\bu| \, 2 r_{\rm p}}{\mu(T)} \end{eqnarray}
where $\mu(T)$ is the dynamic viscosity of air at temperature $T$. Additional corrections are made to account for drag reduction due to the wake effect~\cite{Ramirez:1} and deformation of the droplet~\cite{Loth:1}.


\section{Liquid Droplet Size Distribution}

A liquid spray consists of a sampled set of spherical liquid droplets. The size distribution is
expressed in terms of its Cumulative Volume Fraction (CVF), a function
that relates the fraction of the liquid volume (mass) transported by
droplets less than a given diameter. Size distribution is used to specify the initial distribution of droplets that are introduced to the
computational domain from a nozzle or surface, or that are initially present in the simulation.

In FDS, the the size distributions can be described using analytical distributions, such as the lognormal and Rosin-Rammler distribution.
Researchers at Factory Mutual have suggested that the CVF for a sprinkler may be represented by a combination of lognormal and
Rosin-Rammler distributions~\cite{Chan:1}
\be F_{\rm v}(D) = \left\{ \begin{array}{ll}
   \frac{1}{\sqrt{2\pi}} {\displaystyle \int_0^D} \, \frac{1}{\sigma\, D'} \,
   \exp \left( -\frac{[\ln(D'/D_{\rm v,0.5})]^2}{2\sigma^2} \right) \; \d D'            & (D \le D_{\rm v,0.5}) \\ [0.2in]
   1 - \exp \left( -0.693 \left(\frac{D}{D_{\rm v,0.5}}\right)^\gamma \right)           & (D_{\rm v,0.5} < D)
   \end{array} \right.  \ee
where $D_{\rm v,0.5}$ is the median volumetric droplet diameter (i.e., half the mass
is carried by droplets with diameters of $D_{\rm v,0.5}$ or less), and $\gamma$ and
$\sigma$ are empirical constants equal to approximately 2.4 and 0.6, respectively.\footnote{The Rosin-Rammler and
lognormal distributions are smoothly joined if $\sigma=2/(\sqrt{2\pi} \, (\ln\,2) \; \gamma)=1.15/\gamma$ .}
Alternatively, the user may specify any form of size distribution using the tabulated input data.

The median droplet diameter is a function of the sprinkler orifice diameter, operating pressure, and geometry. Research at Factory Mutual
has yielded a correlation for the median droplet diameter~\cite{Yu:2}
\be \frac{D_{\rm v,0.5}}{d} \propto \WE^{-\ot}  \label{dropcor} \ee
where $d$ is the orifice diameter of the nozzle.
The orifice Weber number, the ratio of inertial forces to surface tension forces, is given by
\be \WE = \frac{\rho_{\rm p} \, u_{\rm p}^2 \, d}{\sigma}  \label{Weber} \ee
where $\rho_{\rm p}$ is the liquid density, $u_{\rm p}$ is the discharge
velocity, and $\sigma$ is the liquid surface tension ($72.8 \times 10^{-3}$~N/m at 20~$^\circ$C for water).
The discharge velocity can be computed from the mass flow rate, a function of the operating pressure and orifice coefficient known as the K-factor.
FM reports that the constant of proportionality in Eq.~(\ref{dropcor}) appears to be independent of flow
rate and operating pressure. Three different sprinklers were tested in
their study with orifice diameters of 16.3~mm, 13.5~mm, 12.7~mm and
the constants were approximately 4.3, 2.9, 2.3, respectively. The strike
plates of the two smaller sprinklers were notched, while that of the
largest sprinkler was not~\cite{Yu:2}.

In real sprinkler systems, the operating pressure is affected by the number of open nozzles. Typically, the pressure
in the piping is high when the first sprinkler activates, and
decreases when more and more sprinkler heads are activated. The pipe pressure has an effect on
flow rate, droplet velocity and droplet size distribution. FDS tries
not to predict the variation of pipe pressure; it should be specified
by the user; but the following dependencies are used to update the
droplet boundary conditions for mass flow, droplet speed, and median diameter:
\be
    \dot{m}_{\rm p} \propto p^{1/2} \quad ; \quad u_{\rm p} \propto p^{1/2} \quad ; \quad D_{\rm v,0.5}  \propto  p^{-1/3}
\ee
The droplet diameters are randomly chosen from the given size distribution.
The cumulative number fraction (CNF), $F_{\rm n}$, is determined from the cumulative volume fraction, $F_{\rm v}$, as follows
\be F_{\rm n}(D) = \int_0^D \frac{F'_{\rm v}(D')}{D'^3} \, \d D'  \left/ \int_0^\infty \, \frac{F'_{\rm v}(D')}{D'^3}
     \, \d D' \right. \quad ; \quad F_{\rm v}' \equiv \frac{\d F_{\rm v}}{\d D}   \ee
Figure~\ref{rosin} displays the Rosin-Rammler/lognormal function and the resulting cumulative number fraction.

The selection of droplet diameters is performed using a stratified sampling technique. First, the diameter range is divided
into $N_{\rm s}$ bins of equal width $\Delta D_{\rm s}= D_{\rm max}/N_{\rm s}$ where $D_{\rm max}$ is the maximum diameter considered.
The upper and lower bounds of bin $n_{\rm s}$ are denoted by $F_{\rm n,0}(n_{\rm s})$ and $F_{\rm n,1}(n_{\rm s})$, respectively.
The diameter is then chosen from the inverse of the CNF acting on the uniform random variable, ${\cal U}(0,1)$, inside a random bin
\be 
   D = F^{-1}_{\rm n} \Big[ F_{\rm n,0}(n_{\rm s}) + {\cal U}(0,1)\left(F_{\rm n,1}(n_{\rm s})-F_{\rm n,0}(n_{\rm s})\right) \Big] \label{Ud_strat}  
\ee
The purpose of the the stratified sampling is to ensure that droplets are chosen from all parts of the size distribution, even with a small number of statistical samples. The default number of $N_{\rm s}$ is 7. Without the stratification, the tails of the distribution could be poorly represented.
\begin{figure}[t]
\begin{center}
\includegraphics[width=4.5in]{FIGURES/rosin}
\caption[Liquid droplet size distribution]{Cumulative Volume Fraction and Cumulative Number
Fraction functions of the droplet size distribution from
a typical industrial-scale sprinkler. The median volumetric diameter, $D_{\rm v,0.5}$, is
1~mm, $\sigma=0.6$ and $\gamma=2.4$.}
\label{rosin}
\end{center}
\end{figure}
In most cases, a sampled set of the droplets or particles is explicitly tracked in the model. The procedure for selecting droplet sizes is as follows:
\begin{enumerate}
\item Suppose the mass flow rate of the liquid is $\dm$. Suppose also that the time interval for droplet insertion into the numerical simulation is $\dt$, and the number of droplets inserted each time interval is $N$. Suppose that the number of size distribution bins is $N_{\rm s}$.
\item Choose $N$ uniformly distributed random integers $n_{\rm s}$ between 1 and $N_{\rm s}$.
\item Choose $N$ uniformly distributed random numbers between 0 and 1 and calculate $N$ random droplet diameters, $D_i$, from Eq.~(\ref{Ud_strat}).
\item Compute weighting constants $C_i = F_{\rm n,1}(n_{\rm s}) - F_{\rm n,0}(n_{\rm s})$.
\item Compute a weighting constant C from the mass balance
    \be \dm \; \dt = C \, \sum_{i=1}^N \; C_i \; \frac{4}{3} \pi \rho_{\rm p}
      \left( \frac{D_i}{2} \right)^3 
    \ee
    The mass and heat transferred from each droplet will be multiplied by the weighting factor $C$.
\end{enumerate}


\section{Heating and Evaporation of Liquid Droplets}

Liquid droplets are represented either as discrete airborne spheres or as elements of the thin liquid film that forms on wetted solids. These film ``droplets'' are still individually tracked as Lagrangian particles, but the heat and mass transfer coefficients are different. In the discussion to follow, the term ``droplets'' will be used to describe either form.

Over the course of a time step of the gas phase solver, the droplets in a given grid cell evaporate as a function of the liquid equilibrium vapor mass fraction, $Y_{\rm l}$, the local gas phase vapor mass fraction, $Y_{\rm g}$, the (assumed uniform) liquid temperature, $T_{\rm l}$, and the local gas temperature, $T_{\rm g}$. If the droplet is attached to a solid, $T_{\rm s}$ is the surface temperature. The mass and energy transfer between the gas and the liquid can be described by the following set of equations~\cite{Cheremisinoff:1}
\begin{align}
\frac{\d m_{\rm l}}{\d t}               & =  - A \; h_m \, \rho \, (Y_{\rm l} - Y_{\rm g}) \\
m_{\rm l} \; c_{\rm l} \; \frac{\d T_{\rm l}}{\d t} & =    A \, h  \, (T_{\rm g}-T_{\rm l}) + A \, h_s \, (T_{\rm s}-T_{\rm l}) + \dq_{\rm r} + \frac{\d m_{\rm l}}{\d t} \; h_{\rm v}  \label{droplet_temp}   \end{align}
Here, $m_{\rm l}$ is the mass of the liquid droplet (or that fraction of the surface film associated with the formerly airborne droplet), $A$ is the surface area of the liquid droplet (or that fraction of the film exposed to the gas and to the wall), $h_m$ is the mass transfer coefficient to be discussed below,
$\rho$ is the gas density, $c_{\rm l}$ is the liquid specific heat, $h$ is the heat transfer coefficient between the liquid and the gas, $h_{\rm s}$ is the heat transfer coefficient between the liquid and the solid surface, $\dq_{\rm r}$ is the rate of radiative heating of the droplet, and $h_{\rm v}$ is the latent heat of vaporization of the liquid. The vapor mass fraction of the gas, $Y_{\rm g}$, is obtained from the gas phase mass conservation equations, and the liquid equilibrium vapor mass fraction is obtained from the Clausius-Clapeyron equation
\be X_{\rm l} = \frac{p_{\rm 1 \; atm}}{\bp_m} \exp \left[ \frac{h_{\rm v} \, W_{\rm l}}{\cal R}
      \left( \frac{1}{T_{\rm b}}-\frac{1}{T_{\rm l}} \right) \right]  \quad ; \quad
      Y_{\rm l} = \frac{X_{\rm l}}{X_{\rm l} \, (1-W_{\rm a}/W_{\rm l}) + W_{\rm a}/W_{\rm l}}  \label{clausius_clapeyron} \ee
where $X_{\rm l}$ is the equilibrium vapor {\em volume} fraction, $W_{\rm l}$ is the molecular weight of the evaporated liquid, $W_{\rm a}$ is the molecular weight of air, $\cal R$ is the universal gas constant, and $T_{\rm b}$ is the boiling temperature of the liquid at standard atmospheric pressure.

Mass and heat transfer between liquid and gas are described with analogous empirical correlations.
The mass transfer coefficient, $h_m$, is described by the empirical relationships~\cite{Incropera:1}:
\be
   h_m = \frac{\SH \; D_{\rm lg}}{L} \quad ; \quad \SH = \left\{ \begin{array}{ll} 2 + 0.6 \; \RE_D^\ha \;           \SC^\ot & \hbox{droplet} \\ [0.1in]
                                                                                 0.037 \;   \RE_L^{\frac{4}{5}} \; \SC^\ot & \hbox{film}     \end{array} \right.
\ee
$\SH$ is the Sherwood number, $D_{\rm lg}$ is the binary diffusion coefficient between the liquid vapor and the surrounding gas (usually assumed air), $L$ is a length scale equal to either the droplet diameter or 1~m for a surface film, $\RE_D$ is the Reynolds number of the droplet (based on the diameter, $D$, and the relative air-droplet velocity), $\RE_L$ is the Reynolds number based on the length scale $L$, and $\SC$ is the Schmidt number ($\nu/D_{\rm lg}$, assumed 0.6 for all cases).

An analogous relationship exists for the heat transfer coefficient:
\be
   h  = \frac{\NU \; k}{L} \quad ; \quad \NU = \left\{ \begin{array}{ll} 2 + 0.6 \; \RE_D^\ha \; \PR^\ot           & \hbox{droplet} \\ [0.1in]
                                                                         0.037 \;   \RE_L^{\frac{4}{5}} \; \PR^\ot & \hbox{film}     \end{array} \right.
\ee
$\NU$ is the Nusselt number, $k$ is the thermal conductivity of the gas, and $\PR$ is the Prandtl number (assumed 0.7 for all cases).


The exchange of mass and energy between liquid droplets and the surrounding gases (or solid surfaces) is computed droplet by droplet.
After the temperature of each droplet is computed, the
appropriate amount of vaporized liquid is added to the given mesh cell, and the cell
gas temperature is reduced slightly based on the energy lost to the droplet.

Equation~(\ref{droplet_temp}) is solved semi-implicitly over the course of a gas phase time step as follows.
Note that a few terms have been left out to make the algorithm more clear.
\be
   \frac{T_{\rm l}^{n+1}-T_{\rm l}^n}{\dt} = \frac{1}{m_{\rm l} \, c_{\rm l}} \left[ A \, h \, \left( T_{\rm g}-\frac{T_{\rm l}^{n+1}+T_{\rm l}^n}{2} \right) -
   A \, h_m \, \rho \left( \frac{Y_{\rm l}^{n+1}+Y_{\rm l}^n}{2} - Y_{\rm g} \right) h_{\rm v}  \right]
\ee
The equilibrium vapor mass fraction, $Y_{\rm l}^n$, is a function of $T_{\rm l}^n$ via
Eq.~(\ref{clausius_clapeyron}), and its value at the next time step is approximated via
\be
   Y_{\rm l}^{n+1} \approx Y_{\rm l}^n + \left( \frac{\d Y_{\rm l}}{\d T_{\rm l}} \right)^n \; \Big( T_{\rm l}^{n+1}-T_{\rm l}^n \Big)
\ee
where the derivative of $Y_{\rm l}$ with respect to temperature is obtained via the chain rule:
\be
   \frac{\d Y_{\rm l}}{\d T_{\rm l}} = \frac{\d Y_{\rm l}}{\d X_{\rm l}} \, \frac{\d X_{\rm l}}{\d T_{\rm l}}  = \frac{W_{\rm a}/W_{\rm l}}{ (X_{\rm l} (1-W_{\rm a}/W_{\rm l}) + W_{\rm a}/W_{\rm l})^2 } \;
     \frac{p_{\rm 1 \; atm}}{\bp_m} \frac{h_{\rm v} W_{\rm l}}{{\cal R} \, T_{\rm l}^2} \, \exp \left[ \frac{h_{\rm v} \, W_{\rm l}}{\cal R} \left( \frac{1}{T_{\rm b}}-\frac{1}{T_{\rm l}} \right) \right]
\ee
The amount of evaporated liquid is given by
\be
   \delta m_{\rm l} = -\dt \, A \, h_m \, \rho  \left[  Y_{\rm l}^n + \ha \left( \frac{\d Y_{\rm l}}{\d T_{\rm l}} \right)^n \;
   \Big( T_{\rm l}^{n+1}-T_{\rm l}^n \Big)  - Y_{\rm g} \right]
\ee
The amount of heat extracted from the gas is
\be
   \delta q = \dt \, A \, h \, \left( T_{\rm g} - \frac{T_{\rm l}^n+T_{\rm l}^{n+1}}{2} \right)
\ee


\section{Fire Suppression by Water}

The previous sections describe heat transfer from a liquid droplet to a gas, a solid, or both. Although there is some
uncertainty in the values of the respective heat transfer coefficients,
the fundamental physics are fairly well understood. However, when
the droplets encounter burning surfaces,
simple heat transfer correlations become more difficult to apply.
The reason for this is that the water is not only cooling the surface
and the surrounding gas, but it is also changing the pyrolysis rate
of the fuel. If the surface of the fuel is planar, it is possible
to characterize the decrease in the pyrolysis rate as a function of
the decrease in the total heat feedback to the surface. Unfortunately,
most fuels of interest in fire applications are multi-component solids
with complex geometry at scales unresolvable by the computational grid.

\subsection{Droplet Transport on a Surface}

When a liquid droplet hits a solid horizontal surface, it is assigned a
random horizontal direction and moves at a fixed velocity until it
reaches the edge, at which point it drops straight down at the same
fixed velocity. This ``dripping'' velocity has been measured for water to be on
the order of 0.5~m/s~\cite{Hamins:1,Hamins:IAFSS2002}.
While attached to a surface, the ``droplet'' is assumed to form a thin film of liquid that
transfers heat to the solid, and heat and mass to
the gas. The film thickness, $\delta$, is given by
\be
   \delta = \max \left( \delta_{\min} , \sum \frac{4}{3} \, \frac{\pi \, r_{\rm p}^3}{A} \right)
\ee
where $A$ is the area of the wall cell to which the droplet is attached. It is assumed that the minimum film thickness, $\delta_{\min}$, is $1 \times 10^{-5}$~m. This prevents a very small amount of liquid from spreading across the entire cell width. It is also assumed that the liquid is opaque with regard to thermal radiation.

\subsection{Reduction of Pyrolysis Rate due to Water}

To date, most of the work in this area has been
performed at Factory Mutual. An important paper on the subject is
by Yu {\em et al.}~\cite{Yu:1}. The authors consider dozens of
rack storage commodity fires of different geometries and water
application rates, and characterize the suppression rates in terms of
a few global parameters. Their analysis yields an
expression for the total heat release rate from a rack storage fire
after sprinkler activation
\be \dQ = \dQ_0 \; \mathrm{e}^{-k (t-t_0)}  \label{fmexting} \ee
where $\dQ_0$ is the total heat release rate at the time of application
$t_0$, and $k$ is a fuel-dependent constant.
For the FMRC Standard Plastic commodity $k$ is given as
\be k = 0.716 \; \dot{m}_{\rm w}'' - 0.0131 \quad  \hbox{s}^{-1} \ee
where $\dot{m}_{\rm w}''$ is the flow rate of water impinging on the
box tops, divided by the area of exposed surface (top and sides). It is
expressed in units of kg/(m$^2 \cdot$s). For the Class II commodity, $k$ is
given as
\be k = 0.536 \; \dot{m}_{\rm w}'' - 0.0040 \quad  \hbox{s}^{-1} \ee
Unfortunately, this analysis is based on global water flow and
burning rates. Equation~(\ref{fmexting})
accounts for both the cooling of non-burning surfaces as well as the
decrease in heat release rate of burning surfaces. In the FDS model,
the cooling of unburned surfaces and the reduction in the heat
release rate are computed locally. Thus, it is awkward to apply a
global suppression rule.
However, the exponential nature of suppression by water is observed
both locally and globally, thus it is assumed that the local burning rate
of the fuel can be expressed in the form~\cite{Hamins:1,Hamins:IAFSS2002}
\be \dm_{\rm f}''(t) = \dm_{\rm f,0}''(t) \; \mathrm{e}^{-\int k(t) \, \d t}
\label{nistexting} \ee
Here $\dm_{\rm f,0}''(t)$ is the burning rate per unit area of the fuel
when no water is applied and $k(t)$ is a linear function of the local water
mass per unit area, $m_{\rm w}''$, expressed in units of kg/m$^2$,
\be k(t) = a \; m_{\rm w}''(t) \quad   \hbox{s}^{-1} \ee
Note that $a$ is an empirical constant.


\section{Using Lagrangian Particles to Model Complex Objects}

There are many real objects that participate in a fire that cannot be modeled easily as solid obstructions that conform
to the rectilinear mesh. For example, electrical cables, dry brush, tree branches, and so on, are potential fuels that cannot
be well-represented as solid cubes, not only because the geometry is wrong, but also because the solid restricts the
movement of hot gases through the complex collection of objects.  Additionally objects such as window screens also impose flow restrictions but are typically not resolvable in an engineering calculation. As a potential remedy for the problem, these objects can
be modeled as discrete particles that are either spheres, cylinders or small sheets. Each particle can be assigned a surface
type in much the same way as is done for solid obstructions that conform to the numerical grid. The particle is assumed to be
thermally-thick, but for simplicity the heat conduction within the particle is assumed to be one-dimensional in either
a cylindrical, spherical or cartesian coordinate system.

It is assumed that the particles interact with the surrounding gas via an additional source term in the energy conservation
equation. For a grid cell with indices $ijk$, the source term is:
\be \dq_{{\rm r},ijk}''' \equiv (-\nabla \cdot \dot{\bq}_{\rm r}'')_{ijk} = \sum \kappa_{\rm p} \left( U_{ijk} - 4 \sigma \, T_{\rm p}^4 \right) \ee
where the summation is over all the particles within the cell. The effective absorption coefficient for a single particle is given by
\be \kappa_{\rm p} = \frac{A}{4 \, \dx \, \dy \, \dz} \ee
where $A$ is the surface area of the particle and $\dx \, \dy \, \dz$ is the volume of the cell.
The net radiative heat flux onto the surface of the particle is
\be \dq_{\rm r,p}'' = \epsilon \left( \frac{U_{ijk}}{4} - \sigma T_{\rm p}^4 \right) \ee

\subsection{Porous Media (Screens, Metal Meshes, and Similar Materials)}

Catwalk grating, expanded metal mesh, window screens, and similar thin porous materials are commonly found in buildings.  For a typical calculation, the dimensions of the holes in the porous material and the material surrounding them are going to be much smaller than the grid size.  There is, therefore, no easy way to model these materials using solid obstructions. Lagrangian particles can; however, be used to represent both the drag and the mass of these materials. Lagrangian particles arranged in a plane can represent a window screen, for example. By modifying the particle drag law, the particles can impose drag appropriate for a thin porous material.  The pressure drop across a screen of thickness $l$ is given by~\cite{Bartzanas:1}:
\be
   \frac {\Delta p}{l} =  \frac{\mu}{K} u + \rho \frac{Y}{\sqrt{K}} \, u^2
\ee
where $K$ is a permeability constant, $Y$ is an inertial constant, $u$ is the velocity normal to the screen, $\rho$ is the density, and $\mu$ is the viscosity of the gas.  $K$ and $Y$ are functions the screen porosity (free area/total area), $\varepsilon$:
\be
   K = 3.44 \times 10^{-9} \; \varepsilon^{1.6} \; \; \hbox{m}^2 \quad ; \quad Y = 0.043 \, \varepsilon^{2.13}
\ee
The force vector $\bof_{\rm b}$ in Eq.~(\ref{momentum}) represents the momentum transferred from the screen to the gas:
\be
   \bof_{\rm b} = l \; \left( \frac{\mu}{K} + \rho \frac{Y}{\sqrt{K}} |\bu| \right) \left( \frac{u}{\dx} , \frac{v}{\dy} , \frac{w}{\dz} \right)
\ee
This force term essentially spreads the pressure drop over the width of a grid cell.




\section{Turbulent Dispersion}

The effect of subgrid-scale turbulent fluid motion on the velocity and position of a Lagrangian particle may be accounted for using a random walk model \cite{Raman:CF}.  The position of a tracer particle obeys the stochastic differential equation
\be
\mbox{d}\mathbf{x}^* = \left[ \tilde{\mathbf{u}} + \frac{1}{\bar{\rho}}\nabla (\bar{\rho}\,D_t) \right] \,\mbox{d}t + \sqrt{2D_t} \,\mbox{d}\mbox{\textbf{W}}
\ee
where $\mathbf{x}^*$ denotes the particle position (an asterisk signifies a particle property), $\tilde{\mathbf{u}}$ is the resolved LES velocity, $D_t$ is the turbulent diffusivity (taken from an eddy viscosity model, for example), and $\mbox{\textbf{W}}$ is an independent Wiener process.  Notice that if no turbulent diffusion exists the particle follows the resolved flow.  The term added to the resolved velocity accounts for the deterministic mean drift and the random walk term (Wiener process) accounts for the reorientation effect of unresolved turbulent motion.

For those unfamiliar with stochastic differential equations, the Wiener process may be understood numerically as $\mbox{dW}(t) = (\delta t)^{1/2} \, \zeta(t)$ in the limit $\delta t \rightarrow 0$, where $\zeta(t)$ is an independent standardized Gaussian random variable \cite{Pope:2000}.  In FDS, $\zeta(t)$ are generated from a Box-Muller transform \cite{Box-Muller:1958}.






















