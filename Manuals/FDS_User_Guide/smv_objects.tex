
\newcommand{\devicewidth}{1.5in}
\newcommand{\boxwidth}{3.0in}
\newcommand{\incgraphics}[1]{
\parbox[c]{\devicewidth}{
\vspace{0.01in}
\includegraphics[width=\devicewidth]{#1}
\vspace{0.01in}
}
}

\section{Visualizing FDS Devices Using Smokeview Objects}

Smokeview generates visual representations of FDS devices using instructions found in a data file named
{\ct objects.svo}, which is located in the same directory where Smokeview is installed.  These instructions correspond to OpenGL library calls, the same type of calls Smokeview
uses to visualize FDS cases.  This section gives an overview
of Smokeview objects detailing what objects are available and how to modify them.  Further documentation
giving underlying technical details may be found in the Smokeview User's Guide~\cite{Smokeview_Users_Guide}.

Smokeview objects may be static or dynamic.  A static object is defined entirely in terms of data and instructions
found in the {\ct objects.svo}\ file.  Its appearance remains the same regardless of how an FDS input file is set up.  A
dynamic object is also defined using instructions found in {\ct objects.svo}, but in addition uses data specified on
the {\ct PROP}  line and/or data contained in a particle file.  As a result, the appearance of dynamic
objects depends on the particular FDS case that is run.


\subsection{Static Smokeview Objects}
\label{info:SMOKEVIEW_ID}

Smokeview objects consist of one or more frames or views.  Smokeview then displays FDS devices in a normal/inactive
state or in an active state.  A sprinkler, for example, is drawn differently depending on whether it has activated
or not.  When FDS determines that a device has activated it places a message in the {\ct .smv} file indicating the
object number, the activation time and the state (0 for inactive or 1 for active).  Smokeview then draws the corresponding
frame.  Tables~\ref{tab:devices_static} and \ref{tab:devices_mstatic} give a list of various static objects.  Each entry shows
an image of the object in its normal/inactive state and in its active state if it has one.  The intersection of black tubes
indicate the origin, the part of the device displayed at the $(x,y,z)$ coordinate specified on the {\ct DEVC} line.

The {\ct SMOKEVIEW\_ID} keyword found on the {\ct PROP} line is used to associate an FDS device with a Smokeview object.
The following FDS input file lines
were used to display the target device in Table~\ref{tab:devices_static}.

\footnotesize
\begin{verbatim}
&PROP ID='target' SMOKEVIEW_ID='target' /
&DEVC XYZ=0.5,0.8,0.6, QUANTITY='TEMPERATURE' PROP_ID='target' /
\end{verbatim} \normalsize

\newpage

\begin{longtable}[t!]{|l|c|}
\caption{Single frame static objects}
\label{tab:devices_static}
\\ \hline
{\ct SMOKEVIEW\_ID} & Image  \\ \hline \hline
\endfirsthead
\caption{Single frame static objects (continued)} \\ \hline
{\ct SMOKEVIEW\_ID} & Image  \\ \hline \hline
\endhead

{\ct sensor} & \incgraphics{SCRIPT_FIGURES/sensor} \\ \hline
{\ct target} & \incgraphics{SCRIPT_FIGURES/target} \\ \hline

\end{longtable}

\begin{longtable}[ht]{|l|c|c|}
\caption{Dual frame static objects}
\label{tab:devices_mstatic}
\\ \hline
\multirow{2}{*}{{\ct SMOKEVIEW\_ID}} &\multicolumn{2}{|c|}{Image}\\ \cline{2-3}
& inactive & active  \\ \hline \hline
\endfirsthead
\caption{Dual frame static objects (continued)}
\\ \hline
\multirow{2}{*}{{\ct SMOKEVIEW\_ID}} &\multicolumn{2}{|c|}{Image}\\ \cline{2-3}
& inactive & active  \\ \hline \hline
\endhead

{\ct heat\_detector}      & \incgraphics{SCRIPT_FIGURES/heat_detector_0}     & \incgraphics{SCRIPT_FIGURES/heat_detector_1} \\ \hline
{\ct nozzle}              & \incgraphics{SCRIPT_FIGURES/nozzle_0}            & \incgraphics{SCRIPT_FIGURES/nozzle_1} \\ \hline
{\ct smoke\_detector}     & \incgraphics{SCRIPT_FIGURES/smoke_detector_0}    & \incgraphics{SCRIPT_FIGURES/smoke_detector_1} \\ \hline
{\ct sprinkler\_upright}  & \incgraphics{SCRIPT_FIGURES/sprinkler_upright_0} & \incgraphics{SCRIPT_FIGURES/sprinkler_upright_1} \\ \hline
{\ct sprinkler\_pendent}  & \incgraphics{SCRIPT_FIGURES/sprinkler_pendent_0} & \incgraphics{SCRIPT_FIGURES/sprinkler_pendent_1} \\ \hline

\end{longtable}


\subsection{Dynamic Smokeview Objects - Customized Using \&PROP Parameters}
\label{info:SMOKEVIEW_PARAMETERS}

The appearance of several Smokeview objects may be modified using data specified on the {\ct \&PROP} line in an FDS input
file.  Objects may also be customized using data stored in a particle file.  This is discussed in the next section.

The {\ct SMOKEVIEW\_PARAMETERS} keyword on the {\ct PROP}
namelist statement is used to customize the appearance of Smokeview objects.  For example, the {\ct \&DEVC}\ and {\ct \&PROP}\ statements:

\footnotesize
\begin{verbatim}
&PROP ID='sphere' SMOKEVIEW_PARAMETERS(1:4)='R=0','G=255','B=0',
                   'D=0.5' SMOKEVIEW_ID='sphere' /
&DEVC XYZ=0.5,0.8,1.5, QUANTITY='TEMPERATURE' PROP_ID='sphere' /
\end{verbatim} \normalsize

\noindent
create an FDS device drawn as a sphere colored green with diameter 0.5~m. Each parameter specified using the
{\ct SMOKEVIEW\_PARAMETERS} keyword
is a text string enclosed in single quotes.  The text string is of the form {\ct 'keyword=value'} where possible
keywords are found in the {\ct objects.svo} file (labels beginning with `:').  For example, {\ct R}, {\ct G},
{\ct B} and {\ct D} may be used as keywords to customize the following {\ct sphere} object:

\footnotesize
\begin{verbatim}
OBJECTDEF // object for particle file sphere
 sphere
 :R=0 :G=0 :B=0 :D=0.1
 $R $G $B setrgb
 $D drawsphere
\end{verbatim} \normalsize

\noindent Another, Smokeview object, the {\ct tsphere}, uses a texture map or picture to alter the appearance of the object.
The texture map is specified using {\ct SMOKEVIEW\_PARAMETERS} keyword by placing the characters {\ct t\%}
before the texture file name ({\em e.g.}\ {\ct t\%texturefile.jpg}).

Table~\ref{tab:devices_dynamic} gives a list of dynamic objects and the keyword/parameter pairs used to specify them.
Each entry shows an image of the object and the parameters used to customize its appearance.

\begin{longtable}[ht]{|l|l|c|}
\caption{Dynamic Smokeview objects}
\label{tab:devices_dynamic}
\\ \hline
{\ct SMOKEVIEW\_ID}  & {\ct SMOKEVIEW\_PARAMETERS} & Image  \\ \hline \hline
\endfirsthead
\caption{Dynamic Smokeview objects (continued)}
\\ \hline
{\ct SMOKEVIEW\_ID}  & {\ct SMOKEVIEW\_PARAMETERS} & Image  \\ \hline \hline
\endhead

{\ct ball} &
\parbox[c]{\boxwidth}{
\hspace{1in} \\
{\ct SMOKEVIEW\_PARAMETERS(1:6)=}\\
{\ct 'R=128','G=192','B=255',}\\
{\ct 'DX=0.5','DY=.75','DZ=1.0'}\\  \\
R, G, B - color components (0 to 255) \\
DX, DY, DZ - amount ball is stretched along x, y, z axis (m) \\
\hspace{1in} } &
\incgraphics{SCRIPT_FIGURES/ball} \\ \hline

{\ct cone} &
\parbox[c]{\boxwidth}{
\hspace{1in} \\
{\ct SMOKEVIEW\_PARAMETERS(1:5)=}\\
{\ct 'R=128','G=255','B=192',}\\
{\ct 'D=0.4','H=0.6'}\\ \\
R, G, B - color components ranging (0 to 255) \\
D, H - diameter and height (m) \\
\hspace{1in}
} &
\incgraphics{SCRIPT_FIGURES/cone} \\ \hline

{\ct fan} &
\parbox[c]{\boxwidth}{
\hspace{1in} \\
{\ct SMOKEVIEW\_PARAMETERS(1:11)=}\\
{\ct 'HUB\_R=0','HUB\_G=0','HUB\_B=0',}\\
{\ct 'HUB\_D=0.1','HUB\_L=0.12',}\\
{\ct 'BLADE\_R=128','BLADE\_G=64',}\\
{\ct 'BLADE\_B=32','BLADE\_ANGLE=60.0',}\\
{\ct 'BLADE\_D=0.5','BLADE\_H=0.09'}\\  \\
HUB\_R, HUB\_G, HUB\_B - color components of fan hub (0 to 255) \\
HUB\_D, HUB\_L - diameter and length of fan hub (m) \\
BLADE\_R, BLADE\_G, BLADE\_B - color components of fan blades (0 to 255) \\
BLADE\_ANGLE, BLADE\_D, BLADE\_H - angle, diameter and height of a fan blade \\
\hspace{1in}
} &
\incgraphics{SCRIPT_FIGURES/fan} \\ \hline

{\ct tsphere} &
\parbox[c]{\boxwidth}{
\hspace{1in} \\
    {\ct SMOKEVIEW\_PARAMETERS(1:9)=}\\
    {\ct 'R=255','G=255','B=255',}\\
    {\ct 'AX0=0.0','ELEV0=90.0',}\\
    {\ct 'ROT0=0.0','ROTATION\_RATE=10.0',}\\
    {\ct 'D=1.0',}\\
    {\ct 'tfile="t\%sphere\_cover\_04.png"'}\\ \\
R, G, B - color components (0 to 255) \\
AX0, ELEV0, ROT0 - initial azimuth, elevation and rotation angle (deg) \\
ROTATION\_RATE - rotation rate about z axis (deg/s) \\
D - diameter (m) \\
tfile - name of texture map file \\
\hspace{1in}
} &
\incgraphics{SCRIPT_FIGURES/tsphere} \\ \hline

{\ct vent} &
\parbox[c]{\boxwidth}{
\hspace{1in} \\
{\ct SMOKEVIEW\_PARAMETERS(1:6)=}\\
{\ct 'R=192','G=192','B=128',}\\
{\ct 'W=0.5','H=1.0', 'ROT=90.0'}\\ \\
R, G, B - color components (0 to 255) \\
W, H - width and height (m) \\
ROT - rotation angle (deg) \\
\hspace{1in}
} &
\parbox[c]{\devicewidth}{
\vspace{0.01in}
\includegraphics[width=\devicewidth]{SCRIPT_FIGURES/vent1}
inactive vent\\
\vspace{0.01in}
\includegraphics[width=\devicewidth]{SCRIPT_FIGURES/vent2}
active vent\\
\vspace{0.01in}
}
\\ \hline
\end{longtable}

\subsection{Dynamic Smokeview Objects - Customized Using \&PROP Parameters and Particle File Data}
\label{info:SMOKEVIEW_PART}


Particle file data may be used to customize the appearance of Smokeview objects.  Any objects that have color
labels named {\ct R}, {\ct G}, {\ct B} (including those objects in Table \ref{tab:devices_dynamic}) may be
colored using particle file data.  In addition, objects that use variable names that match shortened
particle file quantity names\footnote{Short forms of particle file quantity names appear in the Smokeview colorbar label
and in the Smokeview File/Bounds dialog box.}
may be customized.  This data may be used to alter the geometry or structure of the object
using particle file data.  For example, {\ct U-VEL}, {\ct V-VEL}, {\ct W-VEL} and {\ct temp} are shortened
quantity names that correspond to the full names {\ct U-VELOCITY}, {\ct V-VELOCITY}, {\ct W-VELOCITY} and
{\ct TEMPERATURE}.  These full names are documented in Table~\ref{tab:output} in this guide.

The first three lines of the {\ct velegg}\ object definition are:

\footnotesize
\begin{verbatim}
OBJECTDEF // color with FDS quantity, stretch with velocity
 velegg
 :R=0 :G=0 :B=0 :D :U-VEL=1.0 :V-VEL=1.0 :W-VEL=1.0
 \end{verbatim} \normalsize

\noindent The variables {\ct U-VEL}, {\ct V-VEL}, and {\ct W-VEL} in the above line are also particle file quantities (shortened version) that may be
selected in an FDS input file.  If they are selected, then the {\ct velegg} object may be used to display
particle file information. This  object colors the sphere using the currently selected Smokeview particle
variable and stretches a sphere in the x, y and z directions using the {\ct U-VEL}, {\ct V-VEL}, and {\ct W-VEL}
velocity particle data respectively.

Table \ref{tab:devices_dynamic2} documents those objects that can be customized
using particle file data.  These objects may be customized as before using data specified with the {\ct SMOKEVIEW\_PARAMETERS}
keyword or using particle file data.

\begin{longtable}[ht]{|l|l|c|}
\caption{Dynamic Smokeview objects for Lagrangian particles}
\label{tab:devices_dynamic2}
\\ \hline
{\ct SMOKEVIEW\_ID}  & {\ct SMOKEVIEW\_PARAMETERS} & Image  \\ \hline \hline
\endfirsthead
\caption{Dynamic Smokeview objects for Lagrangian particles (continued)}
\\ \hline
{\ct SMOKEVIEW\_ID}  & {\ct SMOKEVIEW\_PARAMETERS} & Image  \\ \hline \hline
\endhead

{\ct box} &
\parbox[c]{\boxwidth}{
\hspace{1in} \\
{\ct SMOKEVIEW\_PARAMETERS(1:6)=}\\
{\ct 'R=192','G=255','B=128',}\\
{\ct 'DX=0.25','DY=.5','DZ=0.125'}\\  \\
R, G, B - color components (0 to 255) \\
DX, DY, DZ - amount box is stretched along x, y, z axis \\
\hspace{1in}
} &
\incgraphics{SCRIPT_FIGURES/box} \\ \hline

{\ct tube} &
\parbox[c]{\boxwidth}{
\hspace{1in} \\
{\ct SMOKEVIEW\_PARAMETERS(1:5)=}\\
{\ct 'R=255','G=0','B=0',}\\
{\ct 'D=0.2','L=0.6'}\\ \\
R, G, B - color components (0 to 255) \\
D, L - diameter and length (m) \\
\hspace{1in}
} &
\incgraphics{SCRIPT_FIGURES/tube} \\ \hline

{\ct velegg} &
\parbox[c]{\boxwidth}{
\hspace{1in} \\
{\ct SMOKEVIEW\_PARAMETERS(1:9)=} \\
{\ct 'R=0', 'G=0', 'B=0'} \\
{\ct 'U-VEL=1.', 'V-VEL=1.', 'W-VEL=1.'}  \\
{\ct 'VELMIN=0.01', 'VELMAX=0.2', 'D=1.0'} \\  \\
R, G, B - color components (0 to 255) \\
U-VEL, V-VEL, W-VEL - velocity components (m/s) \\
VELMIN, VELMAX - minimum and maximum velocity\\
D - diameter of egg at maximum velocity (m) \\
\hspace{1in}
} &
\incgraphics{SCRIPT_FIGURES/velegg} \\ \hline

{\ct veltube} &
\parbox[c]{\boxwidth}{
\hspace{1in} \\
{\ct SMOKEVIEW\_PARAMETERS(1:9)=}\\
{\ct 'R=0', 'G=0',  'B=0'} \\
{\ct 'U-VEL=1.', 'V-VEL=1.', 'W-VEL=1.' }  \\
{\ct 'VELMIN=0.01', 'VELMAX=0.2', 'D=0.1'} \\  \\
R, G, B - color components (0 to 255) \\
U-VEL, V-VEL, W-VEL - velocity components (m/s) \\
VELMIN, VELMAX - minimum and maximum velocity \\
D - diameter of tube at VELMAX (m) \\
\hspace{1in}
} &
\incgraphics{SCRIPT_FIGURES/veltube} \\ \hline
\end{longtable}
