\chapter{Description of Experiments}

\label{Experiments_Chapter}

This chapter contains a brief description of the experiments that were used for model validation. Only enough detail is included here to provide a
general understanding of the model simulations. Anyone wishing to use the experimental measurements for validation ought to consult the cited test reports for a
comprehensive description.



\section{VTT Large Hall Tests}

The experiments are described in Ref.~\cite{Hostikka:VTT2104}. The series consisted of 8 experiments, but because of replicates only three unique fire
scenarios. The experiments were undertaken to study the movement of smoke in a large hall with a sloped ceiling. The tests were conducted inside the
VTT Fire Test Hall, with dimensions of 19~m high by 27~m long by 14~m wide. Each test involved a single heptane pool fire, ranging from 2~MW to 4~MW.
Four types of predicted output were used in the present evaluation -- the HGL temperature and depth, average flame height and the plume
temperature. Three vertical arrays of thermocouples (TC), plus two thermocouples in the plume, were compared to FDS predictions. The HGL
temperature and height were reduced from an average of the three TC
arrays using the standard algorithm described in
Chapter~\ref{HGL:Chapter}. The ceiling jet temperature was not
considered, because the ceiling in the test hall is not flat, and the
standard model algorithm is not appropriate for this geometry.

The VTT test report lacks some information needed to model the experiments, which is why some information was based on private communications with the
principal investigator, Simo Hostikka.
\begin{description}
\item[Surface Materials:] The walls and ceiling of the test hall consist of a 1~mm thick layer of sheet metal on top of a 5~cm layer of
mineral wool. The floor was constructed of concrete. The report does not provide thermal properties of these materials.
\item[Natural Ventilation:] In Cases~1 and 2, all doors were closed, and ventilation was restricted to infiltration through the building envelope. Precise
information on air infiltration during these tests is not available. The scientists who conducted the experiments recommend a leakage area of about 2~m$^2$,
distributed uniformly throughout the enclosure. By contrast, in Case~3, the doors located in each end wall (Doors 1 and 2, respectively)
were open to the external ambient environment. These doors are each 0.8~m wide by 4~m high, and are located such that their centers
are 9.3~m from the south wall.
\item[Mechanical Ventilation:] The test hall has a single mechanical exhaust duct, located in the roof space, running along the center of the building. This
duct had a circular section with a diameter of 1~m, and opened horizontally to the hall at a distance of 12~m from the floor and 10.5~m from the west wall.
Mechanical exhaust ventilation was operational for Case~3, with a constant volume flow rate of 11~m$^3$/s drawn through
the exhaust duct.
\item[Heat Release Rate:] Each test used a single liquid fuel pan with its center located 16~m from the west
wall and 7.4~m from the south wall. For all tests, the fuel was heptane in a circular steel pan that was partially filled with water. The
pan had a diameter of 1.17~m for Case~1 and 1.6~m for Cases~2 and 3. In each case, the fuel surface was 1~m above the
floor. The trays were placed on load cells, and the HRR was calculated from the mass loss rate. For the three cases,
the fuel mass loss rate was averaged from individual replicate tests. In the HRR estimation, the heat of combustion (taken as 44,600~kJ/kg) and the
combustion efficiency for n-heptane was used. Hostikka suggests a value of 0.8 for the combustion efficiency.
Tewarson reports a value of 0.93 for a 10~cm pool fire~\cite{SFPE:Tewarson}. For the calculations reported in the current
study, a combustion efficiency of 0.85 is assumed. In general, an uncertainty of 15~\% has been assumed for the reported HRR of most of the large
scale fire experiments used.
\item[Radiative Fraction:] The radiative fraction was assumed to be 0.35, similar to many smoky hydrocarbons.
\end{description}

\begin{sidewaysfigure}[p]
\begin{center}
\includegraphics[height=6.5in]{FIGURES/VTT/VTT_Drawing}
\end{center}
\caption{Geometry of the VTT Large Fire Test Hall.}
\label{VTT_Drawing}
\end{sidewaysfigure}


\clearpage


\section{UL/NFPRF Sprinkler, Vent, and Draft Curtain Study}
\label{UL_NFPRF_Description}

In January, 1997, a series of 22 heptane spray burner experiments was conducted at the Large Scale Fire Test Facility at Underwriters Laboratories
(UL) in Northbrook, Illinois~\cite{Sheppard:1}. The objective of the experiments was to characterize the temperature and flow field for fire
scenarios with a controlled heat release rate in the presence of sprinklers, draft curtains and a single smoke \& heat vent.
The Large Scale Fire Test Facility at UL contains a 37~m by 37~m (120~ft by 120~ft) main fire test cell, equipped with a 30.5~m by 30.5~m (100~ft by
100~ft) adjustable height ceiling. The layout of the experiments is shown in Fig.~\ref{layout}. One 1.2~m by 2.4~m (4~ft by 8~ft) vent was installed
among 49 upright sprinklers on a 3~m by 3~m (10~ft by 10~ft) spacing.

\begin{figure}[p]
\begin{center}
\setlength{\unitlength}{.05416667in}
\begin{picture}(120,120)

\linethickness{1.mm} \put(0,0){\framebox(120,120)[tc]{North Wall}} \linethickness{.5mm} \put(10,10){\framebox(100,100)[tc]{Adjustable Height
Ceiling}}

\thinlines \put(117,67){\vector(0,-1){67}} \put(117,73){\vector(0, 1){47}} \put(117,70){\makebox(0,0){$120'$}} \put(113,57){\vector(0,-1){47}}
\put(111,110){\line(1,0){4.}} \put(111, 10){\line(1,0){4.}} \put(113,63){\vector(0, 1){47}} \put(113,60){\makebox(0,0){$100'$}}
\put(30.9,12.83){\dashbox{1}(67.1,71.17)[tc]{Draft Curtains}} \put(27.9,40){\vector(0,-1){27.17}} \put(27.9,46){\vector(0, 1){38.0}}
\put(25.9,84.){\line(1,0){4.}} \put(25.9,12.83){\line(1,0){4.}} \put(27.9,43){\makebox(0,0){$71'2''$}} \put(64.0,87.){\vector(-1,0){33.1}}
\put(72.0,87.){\vector( 1,0){26.0}} \put(30.9,85.){\line(0,1){4.}} \put(98.0,85.){\line(0,1){4.}} \put(68.0,87.){\makebox(0,0){$67'1''$}}

\put(16.0,87.){\vector(-1,0){6.}} \put(24.0,87.){\vector( 1,0){6.92}} \put(20.0,87.){\makebox(0,0){$20'11''$}} \put(101.,87.){\vector(-1,0){3.}}
\put(107.,87.){\vector( 1,0){3.}} \put(104.,87.){\makebox(0,0){$12'$}}

\put(27.9,100){\vector(0,1){10.}} \put(27.9,94){\vector(0,-1){10.}} \put(27.9,97){\makebox(0,0){$26'$}}

\put(27.9,8){\vector(0,1){2.}} \put(27.9,8){\line(1,0){3.}} \put(30.9,8){\makebox(0,0)[l]{$2'10''$}}

\put(55.08,14.83){\line(-1,0){2.}} \put(54.08,16.83){\vector(0,-1){2.}} \put(54.08,7.83){\vector(0,1){5.}} \put(54.08,7.83){\line(1,0){3.}}
\put(57.08,7.83){\makebox(0,0)[l]{$2'$}}

\put(85.08,24.83){\line(0,-1){2.}} \put(95.08,24.83){\line(0,-1){2.}} \put(93.08,23.83){\vector(1,0){2.}} \put(87.08,23.83){\vector(-1,0){2.}}
\put(103.00,23.83){\vector(-1,0){5.}} \put(103.00,23.83){\line(0,-1){3.}} \put(103.00,20.83){\makebox(0,0)[ct]{$2'11''$}}
\put(90.08,23.83){\makebox(0,0)[c]{$10'$}}

\thicklines \put(78.08,55.83){\framebox(4,8){ }}

\put(78.58,58.33){\framebox(3,3)[c]{A}} \put(78.58,68.33){\framebox(3,3)[c]{B}} \put(88.58,58.33){\framebox(3,3)[c]{C}}
\put(58.58,38.33){\framebox(3,3)[c]{D}}

\thinlines

\multiput(35.08,14.83)(0,10){7}{\circle*{.8}} \multiput(45.08,14.83)(0,10){7}{\circle*{.8}} \multiput(55.08,14.83)(0,10){7}{\circle*{.8}}
\multiput(65.08,14.83)(0,10){7}{\circle*{.8}} \multiput(75.08,14.83)(0,10){7}{\circle*{.8}} \multiput(85.08,14.83)(0,10){7}{\circle*{.8}}
\multiput(95.08,14.83)(0,10){7}{\circle*{.8}} \tiny \put(35.48,15.23){98} \put(45.48,15.23){91} \put(55.48,15.23){84} \put(65.48,15.23){81}
\put(75.48,15.23){78} \put(85.48,15.23){75} \put(95.48,15.23){72} \put(35.48,25.23){99} \put(45.48,25.23){92} \put(55.48,25.23){85}
\put(65.48,25.23){82} \put(75.48,25.23){79} \put(85.48,25.23){76} \put(95.48,25.23){73} \put(35.48,35.23){100} \put(45.48,35.23){93}
\put(55.48,35.23){86} \put(65.48,35.23){83} \put(75.48,35.23){80} \put(85.48,35.23){77} \put(95.48,35.23){74} \put(35.48,45.23){101}
\put(45.48,45.23){94} \put(55.48,45.23){87} \put(65.48,45.23){62} \put(75.48,45.23){58} \put(85.48,45.23){54} \put(95.48,45.23){50}
\put(35.48,55.23){102} \put(45.48,55.23){95} \put(55.48,55.23){88} \put(65.48,55.23){63} \put(75.48,55.23){59} \put(85.48,55.23){55}
\put(95.48,55.23){51} \put(35.48,65.23){103} \put(45.48,65.23){96} \put(55.48,65.23){89} \put(65.48,65.23){64} \put(75.48,65.23){60}
\put(85.48,65.23){56} \put(95.48,65.23){52} \put(35.48,75.23){104} \put(45.48,75.23){97} \put(55.48,75.23){90} \put(65.48,75.23){65}
\put(75.48,75.23){61} \put(85.48,75.23){57} \put(95.48,75.23){53} \put(70.08,49.83){\makebox(0,0)[c]{68}} \put(70.08,59.83){\makebox(0,0)[c]{69}}
\put(70.08,69.83){\makebox(0,0)[c]{70}} \put(80.08,49.83){\makebox(0,0)[c]{67}} \put(90.08,49.83){\makebox(0,0)[c]{66}}
\put(90.08,69.83){\makebox(0,0)[c]{71}}

\multiput(80.08,56.83)(0,1){7}{\circle*{.2}} \put(80.58,62.83){\line(1,0){22.5}} \put(104.,62.83){\makebox(0,0)[l]{43}}
\put(104.,61.33){\makebox(0,0)[l]{44}} \put(104.,59.83){\makebox(0,0)[l]{45}} \put(104.,58.33){\makebox(0,0)[l]{46}}
\put(104.,56.83){\makebox(0,0)[l]{47}} \put(104.,55.33){\makebox(0,0)[l]{48}} \put(104.,53.83){\makebox(0,0)[l]{49}}

\normalsize

\end{picture}
\end{center}
\caption[Plan view of the UL/NFPRF Experiments.] {\bf Plan view of the UL/NFPRF Experiments. The sprinklers are indicated by the solid circles and
are spaced 3~m apart. The number beside each sprinkler location indicates the channel number of the nearest thermocouple. The vent dimensions
are 4~ft by 8~ft. The boxed letters A, B, C and D indicate burner positions. Corresponding to each burner position is a vertical array of
thermocouples. Thermocouples 1--9 hang 7, 22, 36, 50, 64, 78, 92, 106 and 120~in from the ceiling, respectively, above Position A. Thermocouples 10
and 11 are positioned above and below the ceiling tile directly above Position B, followed by 12--20 that hang at the same levels below the ceiling
as 1--9. The same pattern is followed at Positions C and D, with thermocouples 21--31 at C and 32--42 at D.} \label{layout}
\end{figure}

\begin{description}
\item[Ceiling:] The ceiling was raised to a height of 7.6~m and instrumented with thermocouples and other measurement devices. The ceiling was constructed of
0.6~m by 1.2~m by 1.6~cm UL fire-rated Armstrong Ceramaguard (Item 602B) ceiling tiles. The manufacturer reported the
thermal properties of the material to be: specific heat 753 J/(kg$\cdot$K), thermal conductivity
0.0611~W/(m$\cdot$K), and density 313~kg/m$^3$.
\item[Draft Curtains:] Sheet metal, 1.2~mm thick and 1.8~m deep, was suspended from the ceiling for 16 of the 22 tests, enclosing an area of about 450~m$^2$ and 49 sprinklers.
\item[Sprinklers:] Central ELO-231 (Extra Large Orifice) uprights were used for all the tests. The orifice diameter of this sprinkler is reported by the manufacturer to be
nominally 1.6~cm (0.64~in), the reference actuation temperature is reported by the manufacturer to be 74$^\circ$C (165$^\circ$F). The RTI (Response Time
Index) and C-factor (Conductivity factor) were reported by UL to be 148~(m$\cdot$s)$^\ha$ and 0.7~(m/s)$^\ha$, respectively~\cite{Sheppard:1}.
When installed, the sprinkler deflector was located 8~cm below the ceiling. The thermal
element of the sprinkler was located 11~cm below the ceiling. The sprinklers were installed with nominal 3~m by 3~m (exact 10~ft by 10~ft) spacing in a
system designed to deliver a constant 0.34~L/(s$\cdot$m$^2$) (0.50 gpm/ft$^2$) discharge density when supplied by a 131~kPa (19~psi) discharge
pressure
\item[Vent:] A single UL listed double leaf fire vent with steel covers and steel curb was installed in the adjustable height ceiling in the position shown in
Fig.~\ref{layout}. The vent is designed to open manually or automatically. The vent doors were recessed into the ceiling about 0.3~m (1~ft).
\item[Heat Release Rate:] The heptane spray burner consisted of a 1~m by 1~m square of 1.3~cm pipe supported by four cement blocks 0.6~m off the floor.
Four atomizing spray nozzles were used to provide a free spray of heptane that was then ignited. For all but one of the tests, the total heat release
rate from the fire was manually ramped up following a ``t-squared'' curve to a steady-state in 75~s
(150~s was used in Test I-16). The fire growth curve was followed until a specified fire size was
reached or the first sprinkler activated. After either of these events, the fire size was maintained at that level until conditions reached roughly a
steady state, {\em i.e.} the temperatures recorded near the ceilings remained steady and no more sprinkler activations occurred.
The heat release rate from the burner was confirmed by placing it under the large product calorimeter at UL, ramping up the flow of heptane in the
same manner as in the tests, and measuring the total and convective heat release rates. It was found that the convective heat release rate was
0.65$\pm$0.02 of the total.
\item[Instrumentation:] The instrumentation for the tests consisted of thermocouples, gas analysis equipment, and pressure transducers. The locations of the instrumentation
are referenced in the plan view of the facility (Fig.~\ref{layout}).
Temperature measurements were recorded at 104 locations. Type K 0.0625~in diameter Inconel sheathed thermocouples were positioned to measure (i)
temperatures near the ceiling, (ii) temperatures of the ceiling jet, and (iii) temperatures near the vent.
\end{description}

The UL/NFPRF test results (Series I) are summarized in Table~\ref{ULmatrix}.


\begin{table}[h!]
\begin{center}
\begin{tabular}{|c||c|c|c|c|c|c|}
\hline
\multicolumn{7}{|c|}{\bf Heptane Spray Burner Test Series I}  \\ \hline \hline
Test & Burner & Vent                    & First         & Total      & Draft    & Heat Release Rate \\
No.  & Pos.   & Operation               & Actuation (s) & Actuations & Curtains & MW @ s \\
\hline \hline
I-1   & B  & Closed                     & 65            & 11        & Yes  & 4.4 @ 50  \\ \hline
I-2   & B  & Manual (0:40)              & 66            & 12        & Yes  & 4.4 @ 50  \\ \hline
I-3   & B  & Manual (1:30)              & 64            & 12        & Yes  & 4.4 @ 50  \\ \hline
I-4   & C  & Closed                     & 60            & 10        & Yes  & 4.4 @ 50  \\ \hline
I-5   & C  & Manual (0:40)              & 72            & 9         & Yes  & 4.4 @ 50  \\ \hline
I-6   & C  & Manual (1:30)              & 62            & 8         & Yes  & 4.4 @ 50  \\ \hline
I-7   & C  & 74$^\circ$C link (DNO)     & 70            & 10        & Yes  & 4.4 @ 50  \\ \hline
I-8   & B  & 74$^\circ$C link (9:26)    & 60            & 11        & Yes  & 4.4 @ 50  \\ \hline
I-9   & D  & 74$^\circ$C link (DNO)     & 70            & 12        & Yes  & 4.4 @ 50  \\ \hline
I-10  & D  & Manual (0:40)              & 72            & 13        & Yes  & 4.4 @ 50  \\ \hline
I-11  & D  & 74$^\circ$C link (4:48)    & N/A           & N/A       & Yes  & 4.4 @ 50  \\ \hline
I-12  & A  & Closed                     & 68            & 14        & Yes  & 4.4 @ 50  \\ \hline
I-13  & A  & 74$^\circ$C link (1:04)    & 69            & 5         & Yes  & 6.0 @ 60  \\ \hline
I-14  & A  & Manual (0:40)              & 74            & 7         & Yes  & 5.8 @ 60  \\ \hline
I-15  & A  & Manual (1:30)              & 64            & 5         & Yes  & 5.8 @ 60  \\ \hline
I-16  & A  & 74$^\circ$C link (1:46)    & 106           & 4         & Yes  & 5.0 @ 110 \\ \hline \hline
I-17  & B  & 100$^\circ$C link (DNO)    & 58            & 4         & No   & 4.6 @ 50 \\ \hline
I-18  & C  & 100$^\circ$C link (DNO)    & 58            & 4         & No   & 3.7 @ 50 \\ \hline
I-19  & A  & 100$^\circ$C link (10:00)  & 56            & 10        & No   & 4.6 @ 50 \\ \hline
I-20  & A  & 74$^\circ$C link (1:20)    & 54            & 4         & No   & 4.2 @ 50 \\ \hline
I-21  & C  & 74$^\circ$C link (7:00)    & 58            & 10        & No   & 4.6 @ 50 \\ \hline
I-22  & D  & 100$^\circ$C link (DNO)    & 60            & 6         & No   & 4.6 @ 50 \\ \hline
\end{tabular}
\end{center}
\caption[Results of the UL/NFPRF Experiments.]
{\bf Results of the UL/NFPRF Experiments. Note that DNO means
``Did Not Open''. Also note, the fires grew at a rate proportional
to the square of the time until a certain flow rate of fuel was achieved
at which time the flow rate was held steady. Thus, the ``Heat Release Rate''
was the size of the fire at the time when the fuel supply was leveled off.}
\label{ULmatrix}
\end{table}


\clearpage

\section{NIST/NRC Test Series}

These experiments, sponsored by the US NRC and conducted at NIST, consisted of 15 large-scale experiments performed in June 2003. All 15 tests were
included in the validation study. The experiments are documented in Ref.~\cite{Hamins:SP1013-1}. The fire sizes ranged from 350 kW to 2.2 MW in a compartment with dimensions
21.7~m by 7.1~m by 3.8~m high, designed to represent a compartment in a nuclear power plant containing power and control cables.
The walls and ceiling were covered with two layers of marinate boards, each layer 0.0125~m thick. The floor
was covered with one layer of gypsum board on top of a layer of plywood. Thermo-physical and optical properties of the marinate
and other materials used in the compartment are given in Ref.~\cite{Hamins:SP1013-1}. The room had one door and a mechanical air injection and extraction
system. Ventilation conditions, the fire size, and fire location were varied. Numerous measurements (approximately 350 per test) were made including
gas and surface temperatures, heat fluxes and gas velocities.

Following are some notes provided by Anthony Hamins, who conducted the experiments:
\begin{description}
\item[Natural Ventilation:] The compartment had a 2~m by 2~m door in the middle of the west wall. Some of the tests had a closed door and no mechanical
ventilation (Tests 2, 7, 8, 13, and 17), and in those tests the measured compartment leakage was an important consideration. The test report lists leakage
areas based on measurements performed prior to Tests 1, 2, 7, 8, and 13. For the closed door tests, the leakage area used in the simulations was
based on the last available measurement. The chronological order of the tests differed from the numerical order.
For Test 4, the leakage area measured before Test 2 was used. For Tests 10 and 16, the leakage area
measured before Test 7 was used.
\item[Mechanical Ventilation:] The mechanical ventilation and exhaust was used during Tests 4, 5, 10, and 16, providing about 5 air changes per hour. The
door was closed during Test 4 and open during Tests 5, 10, and 16. The supply duct was positioned on the south wall, about 2~m off the floor. An
exhaust duct of equal area to the supply duct was positioned on the opposite wall at a comparable location. The flow rates through the supply and
exhaust ducts were measured in detail during breaks in the testing, in the absence of a fire. During the tests, the flows were monitored with single
bi-directional probes during the tests themselves.
\item[Heat Release Rate:] A single nozzle was used to spray liquid hydrocarbon fuels onto a 1~m by 2~m fire pan that was about 0.1~m deep. The test plan
originally called for the use of two nozzles to provide the fuel spray. Experimental observation suggested that the fire was less unsteady with the
use of a single nozzle. In addition, it was observed that the actual extent of the liquid pool was well-approximated by a 1~m circle in the
center of the pan. For safety reasons, the fuel flow was terminated when the lower-layer oxygen concentration
dropped to approximately 15~\% by volume.
The fuel used in 14 of the tests was heptane, while toluene was used for one test. The HRR was
determined using oxygen consumption calorimetry. The recommended uncertainty values
were 17~\% for all of the tests.
\item[Radiative Fraction:]  The value of the radiative fraction and its
uncertainty were reported as 0.44 and 0.40 for heptane and toluene, respectively.
\end{description}

\begin{sidewaysfigure}[p]
\begin{center}
\includegraphics[height=6.5in]{FIGURES/NIST_NRC/NIST_NRC_Drawing}
\end{center}
\caption{Geometry of the NIST/NRC Experiments.}
\label{NIST_NRC_Drawing}
\end{sidewaysfigure}





\clearpage


\section{WTC Spray Burner Test Series}

As part of its investigation of the World Trade Center disaster, the Building and Fire Research Laboratory at NIST conducted several series of fire experiments to both gain insight into the
observed fire behavior and also to validate FDS for use in reconstructing the fires. The first series of experiments involved a relatively simple compartment with a liquid spray burner and
various structural elements with varying amounts of sprayed fire-resistive materials (SFRM).
A complete description of the experiments can be found in the NIST WTC report NCSTAR~1-5B~\cite{NIST_NCSTAR_1-5B}.
The overall enclosure was rectangular, as were the vents and most of the obstructions. The compartment walls and ceiling were made of 2.54~cm thick marinite. The manufacturer provided the thermal properties of the material used in the calculation. The density was 737~kg/m$^3$, conductivity 0.12~W/m/K. The specific heat ranged from 1.17~kJ/kg/K at 93~$^\circ$C to
1.42~kJ/kg/K at 425~$^\circ$C. This value was assumed for higher temperatures.
The steel used to construct the column and truss flanges was 0.64~cm thick.  The density of the steel was assumed to be 7,860~kg/m$^3$; its specific heat 0.45~kJ/kg/K.

Two fuels were used in the tests. The properties of the fuels were obtained from measurements made on a series of unconfined burns that are referenced in the test report.
The first fuel was a blend of heptane isomers, C$_7$H$_{16}$. Its soot yield was set at a constant 1.5~\%. The second fuel was a mixture (40~\% - 60~\% by volume) of toluene, C$_7$H$_8$,
and heptane. Because FDS only considers the burning of a single hydrocarbon fuel, the mixture was taken to be C$_7$H$_{12}$ with a soot yield of 11.2~\%.
The radiative fraction for the heptane blend was 0.44; for the heptane/toluene mixture it was 0.39.
The heat release rate of the simulated burner was set to that which was measured in the experiments. The spray burner was modeled using reported properties of the nozzle and
liquid fuel droplets.





\section{FM/SNL Test Series}

The Factory Mutual and Sandia National Laboratories (FM/SNL) test series was a series of 25 fire
tests conducted in 1985 for the U.S. Nuclear Regulatory Commission (NRC) by Factory Mutual Research Corporation (FMRC), under
the direction of Sandia National Laboratories (SNL)~\cite{Nowlen:NUREG4681,Nowlen:NUREG4527}. The primary purpose of these tests was to
provide data with which to validate computer models for various types of compartments typical of nuclear power plants. The
experiments were conducted in an enclosure measuring approximately 18~m long x 12~m wide x 6~m high, constructed at the FMRC fire test facility in Rhode Island.
All of the tests involved forced ventilation to simulate typical power plant operations. Four of
the tests were conducted with a full-scale control room mockup in place. Parameters varied
during the experiments included fire intensity, enclosure ventilation rate, and fire location.

The current study used data from three experiments (Tests 4, 5, and 21). In these tests, the fire
source was a propylene gas burner with a diameter of approximately 0.9 m, with its rim
located approximately 0.1 m above the floor. For Tests 4 and 5, the burner was positioned on the longitudinal axis centerline, 6.1 m from
the nearest wall. For Test 21, the fire source was placed within a simulated electrical
cabinet.

Following is supplemental information provided by the test director, Steve Nowlen of Sandia National Laboratory:
\begin{description}
\item[Heat Release Rate:] The HRR was determined using oxygen consumption calorimetry in the exhaust stack with
a correction applied for the carbon dioxide in the upper layer of the compartment. The
uncertainty of the fuel mass flow was not documented. All three tests selected for this study had
the same target peak heat release rate of 516~kW following a 4~min ``t-squared'' growth
profile. The test report contains time histories of the measured HRR, for which the average,
sustained HRR following the ramp up for Tests 4, 5, and 21 have been estimated as 510~kW, 480~kW, and 470~kW, respectively.
Once reached, the peak HRR was maintained essentially constant
during a steady-burn period of 6~min in Tests~4 and 5, and 16~min in Test~21. Note that in Test 21, Nowlen reports a
``significant'' loss of effluent from the exhaust hood that could lead to an under-estimate of the HRR towards the end of the experiment.
\item[Radiative Fraction:] The radiative fraction was not measured during the experiment, but
in this study it is assumed to equal 0.35, which is typical for a smoky hydrocarbons.
It was further assumed that the radiative fraction was about the same in
Test~21 as the other tests, as fuel burning must have occurred outside of the electrical cabinet in
which the burner was placed.
\item[Measurements:] Four types of measurements were conducted during the FM/SNL test series that are used in the
current model evaluation study, including the HGL temperature and depth, and the ceiling jet and
plume temperatures. Aspirated thermocouples (TCs) were used to make all of the temperature
measurements. Generally, aspirated TC measurements are preferable to bare-bead TC measurements,
as systematic radiative exchange measurement error is reduced.
\item[HGL Depth and Temperature:] Data from all of the vertical TC trees were used when reducing
the HGL height and temperature. For the FM/SNL Tests 4 and 5, Sectors 1, 2, and 3 were used,
all weighted evenly. For Test 21, Sectors 1 and 3 were used, evenly weighted. Sector 2 was
partially within the fire plume in Test 21.
\end{description}

\begin{sidewaysfigure}[p]
\begin{center}
\includegraphics[height=6.5in]{FIGURES/FM_SNL/FM_SNL_Drawing}
\end{center}
\caption{Geometry of the FM/SNL Experiments.}
\label{FM_SNL_Drawing}
\end{sidewaysfigure}

\clearpage

\section{NBS Multi-Room Test Series}

The National Bureau of Standards (NBS, which is now called the National Institute of Standards
and Technology, NIST) Multi-Room Test Series consisted of 45 fire tests representing
9 different sets of conditions were conducted in a three-room suite. The experiments were
conducted in 1985 and are described in detail in Ref.~\cite{Peacock:NBS_Multi-Room}. The suite consisted of two relatively
small rooms, connected via a relatively long corridor. The fire source, a gas burner, was located
against the rear wall of one of the small compartments.
Fire tests of 100~kW, 300~kW and 500~kW were conducted. For the current study, only three 100~kW fire experiments have been used,
including Test~100A from Set~1, Test~100O from Set~2, and Test~100Z from Set~4. These tests
were selected because they had been used in prior validation studies, and because these tests had the
steadiest values of measured heat release rate during the steady-burn period.

Following is additional information provided by the test director, Richard Peacock of NIST:
\begin{description}
\item[Heat Release Rate:] In the two tests for which
the door was open, the HRR during the steady-burn period measured via oxygen consumption
calorimetry was 110~kW with an uncertainty of about 15~\%, consistent with the replicate
measurements made during the experimental series and the uncertainty typical of oxygen
consumption calorimetry. It was assumed that the closed door test (Test~100O) had the same HRR as the open
door tests.
\item[Radiative Fraction:] Natural gas was used as the fuel in
Test~100A. In Tests~100O and 100Z, acetylene was added to the natural gas to increase the
smoke yield, and as a consequence, the radiative fraction increased. The radiative fraction of
natural gas has been studied previously, whereas the radiative fraction of the acetylene/natural
gas mixture has not been studied. The radiative fraction for the natural gas fire was assigned a
value of 0.20, whereas a value of 0.30 was assigned for the natural gas/acetylene fires.
\item[Measurements:] Only two types of measurements conducted during the NBS test series were used in the
evaluation considered here, because there was less confidence in the other measurements.
The measurements considered here were the HGL temperature and depth, in which bare bead
TCs were used to make these measurements. Single point measurements of temperature within
the burn room were not used in the evaluation of plume or ceiling jet algorithms. This is because
the geometry was not consistent in either case with the assumptions used in the model algorithms
of plumes or jets. Specifically, the burner was mounted against a wall, and the room width-to-height
ratio was less than that assumed by the various ceiling jet correlations.
\end{description}

\begin{sidewaysfigure}[p]
\begin{center}
\includegraphics[height=6.5in]{FIGURES/NBS/NBS_Drawing}
\end{center}
\caption{Geometry of the NBS Multi-Room Experiments.}
\label{NBS_Drawing}
\end{sidewaysfigure}




\clearpage

\section{Steckler Compartment Experiments}

Steckler, Quintiere and Rinkinen performed a set of 55 compartment fire tests at NBS in 1979. The compartment was 2.8~m by 2.8~m by 2.13~m high\footnote{The test report
gives the height of the compartment as 2.18~m. This is a misprint. The compartment was 2.13~m high.}, with a single door of
various widths, or alternatively a single window with various heights. A 30~cm diameter methane burner was used to generate fires with heat release rates of
31.6~kW, 62.9~kW, 105.3~kW and 158~kW. Vertical profiles of velocity and temperature were measured in the doorway, along with a single vertical profile of temperature
within the compartment.
A full description and results are reported in Reference~\cite{Steckler:NBSIR_82-2520}. The basic test matrix is listed in Table~\ref{Steckler_Table}. Note that the
test report does not include a detailed description of the compartment. However, an internal report\footnote{ {\em Technical Research Report, Fire Induced Flows
Through Room Openings - Flow Coefficients}, Project 203005-003, Armstrong Cork Company, Lancaster, Pennsylvania, May, 1981.} by the test sponsor, Armstrong Cork Company,
reports that the compartment floor was composed of 19~mm calcium silicate board on top of 12.7~mm plywood on wood joists. The walls and ceiling consisted of
12.7~mm ceramic fiber insulation board over 0.66~mm aluminum sheet attached to wood studs.

\begin{table}[h!]
\caption{Summary of Steckler compartment experiments.}
\begin{center}
\begin{tabular}{|c|c|c|c|c||c|c|c|c|c|}
\hline
        & Opening   & Opening       &  HRR       & Burner       &       & Opening   & Opening     &  HRR         & Burner        \\
Test    & Width     & Height        & $\dot{Q}$  & Location     & Test  & Width     & Height      & $\dot{Q}$    & Location      \\
        & (m)       & (m)           & (kW)       &              &       & (m)       &  (m)        & (kW)         &                \\ \hline \hline
10      & 0.24      & 1.83          &  62.9      & Center       & 224   & 0.74      & 0.92        &  62.9         & Back Corner         \\ \hline
11      & 0.36      & 1.83          &  62.9      & Center       & 324   & 0.74      & 0.92        &  62.9         & Back Corner         \\ \hline
12      & 0.49      & 1.83          &  62.9      & Center       & 220   & 0.74      & 1.83        &  31.6         & Back Corner         \\ \hline
612     & 0.49      & 1.83          &  62.9      & Center       & 221   & 0.74      & 1.83        &  105.3        & Back Corner         \\ \hline
13      & 0.62      & 1.83          &  62.9      & Center       & 514   & 0.24      & 1.83        &  62.9         & Back Wall           \\ \hline
14      & 0.74      & 1.83          &  62.9      & Center       & 544   & 0.36      & 1.83        &  62.9         & Back Wall           \\ \hline
18      & 0.74      & 1.83          &  62.9      & Center       & 512   & 0.49      & 1.83        &  62.9         & Back Wall           \\ \hline
710     & 0.74      & 1.83          &  62.9      & Center       & 542   & 0.62      & 1.83        &  62.9         & Back Wall           \\ \hline
810     & 0.74      & 1.83          &  62.9      & Center       & 610   & 0.74      & 1.83        &  62.9         & Back Wall           \\ \hline
16      & 0.86      & 1.83          &  62.9      & Center       & 510   & 0.74      & 1.83        &  62.9         & Back Wall           \\ \hline
17      & 0.99      & 1.83          &  62.9      & Center       & 540   & 0.86      & 1.83        &  62.9         & Back Wall           \\ \hline
22      & 0.74      & 1.38          &  62.9      & Center       & 517   & 0.99      & 1.83        &  62.9         & Back Wall           \\ \hline
23      & 0.74      & 0.92          &  62.9      & Center       & 622   & 0.74      & 1.38        &  62.9         & Back Wall           \\ \hline
30      & 0.74      & 0.92          &  62.9      & Center       & 522   & 0.74      & 1.38        &  62.9         & Back Wall           \\ \hline
41      & 0.74      & 0.46          &  62.9      & Center       & 524   & 0.74      & 0.92        &  62.9         & Back Wall           \\ \hline
19      & 0.74      & 1.83          &  31.6      & Center       & 541   & 0.74      & 0.46        &  62.9         & Back Wall           \\ \hline
20      & 0.74      & 1.83          &  105.3     & Center       & 520   & 0.74      & 1.83        &  31.6         & Back Wall           \\ \hline
21      & 0.74      & 1.83          &  158.0     & Center       & 521   & 0.74      & 1.83        &  105.3        & Back Wall           \\ \hline
114     & 0.24      & 1.83          &  62.9      & Back Corner  & 513   & 0.74      & 1.83        &  158.0        & Back Wall           \\ \hline
144     & 0.36      & 1.83          &  62.9      & Back Corner  & 160   & 0.74      & 1.83        &  62.9         & Center$^*$          \\ \hline
212     & 0.49      & 1.83          &  62.9      & Back Corner  & 163   & 0.74      & 1.83        &  62.9         & Back Corner$^*$     \\ \hline
242     & 0.62      & 1.83          &  62.9      & Back Corner  & 164   & 0.74      & 1.83        &  62.9         & Back Wall$^*$       \\ \hline
410     & 0.74      & 1.83          &  62.9      & Back Corner  & 165   & 0.74      & 1.83        &  62.9         & Left Wall$^*$       \\ \hline
210     & 0.74      & 1.83          &  62.9      & Back Corner  & 162   & 0.74      & 1.83        &  62.9         & Right Wall$^*$      \\ \hline
310     & 0.74      & 1.83          &  62.9      & Back Corner  & 167   & 0.74      & 1.83        &  62.9         & Front Center$^*$    \\ \hline
240     & 0.86      & 1.83          &  62.9      & Back Corner  & 161   & 0.74      & 1.83        &  62.9         & Doorway$^*$         \\ \hline
116     & 0.99      & 1.83          &  62.9      & Back Corner  & 166   & 0.74      & 1.83        &  62.9         & Front Corner$^*$    \\ \hline
122     & 0.74      & 1.38          &  62.9      & Back Corner  &  \multicolumn{5}{r|}{$^*$ Raised burner}                   \\ \hline
\end{tabular}
\end{center}
\label{Steckler_Table}
\end{table}


\clearpage

\section{Bryant Doorway Velocity Measurements}

Rodney Bryant of the Fire Research Division at NIST performed a series of velocity measurements of the gas
velocity within the doorway of a standard ISO~9705 compartment for fires ranging from
34~kW to 511~kW~\cite{Bryant:FSJ2009}. A doorway
served as the only vent for the enclosure. It included a jamb of 30~cm extending outward
to facilitate the laser measurements. The entire compartment was elevated 0.3~m off the floor of the laboratory.

The measurements were made using both bi-directional probes and PIV (Particle Image Velocimetry). The PIV measurements only cover the
lower two-thirds of the doorway because of difficulties in seeding the hot outflow gases. The bi-directional probe measurements span the
entire height of the doorway, but Bryant reports that
these measurements were up to 20~\% greater than the PIV measurements in certain regions of
the flow. Consequently, only the PIV data was used for comparison to the
model.


\section{SP Adiabatic Surface Temperature Experiments}

In 2008, three compartment experiments were performed at SP Technical Research Institute of Sweden under the sponsorship of Brandforsk, the Swedish Fire Research Board~\cite{Wickstrom_AST}. The
objective of the experiments was to demonstrate how plate thermometer measurements in the vicinity of a simple steel beam can be used to supply the boundary conditions
for a multi-dimensional heat conduction calculation for the beam. The adiabatic surface temperature was derived from the plate temperatures and used by TASEF, a finite-element
thermal-structural program.

The experiments were performed inside a standard compartment designed for corner fire testing (ISO 9705).
The compartment is 3.6~m deep, 2.4~m wide and 2.4~m high and includes a door opening 0.8~m by 2.0~m. The room was constructed of 20~cm thick light weight concrete
blocks with a density of 600~kg/m$^3$ $\pm 100$~kg/m$^3$.
The heat source was a gas burner run at a constant power of 450~kW. The top of the burner, with a square opening 30~cm by 30~cm, was placed 65~cm above the floor, 2.5~cm from the walls.
A single steel beam was suspended 20~cm below the ceiling
along the centerline of the compartment. There were three measurement stations along the beam at lengths of 0.9~m (Position A), 1.8~m (Position B), and
2.7~m (Position C) from the far wall where the fire was either positioned in the corner (Tests 1 and 2), or the center (Test 3). The beam in Test 1 was
a rectangular steel tube filled with an insulation material. The beam in Tests 2 and 3 was an I-beam.


\section{ATF Corridors Experiments}

A series of eighteen experiments were conducted in a two-story structure with long hallways and a connecting stairway
in the large burn room of the ATF Fire Research Laboratory in Ammendale, Maryland, in 2008~\cite{Sheppard:Corridors}.
The test enclosure consisted of two 17.0~m long hallways connected by a
stairway consisting of two staircases and an intermediary landing.
There was a door at the opposite end of the first floor hallway, which was closed during all tests.
The end of the second floor hallway was open with a soffit near the ceiling.

The walls and ceilings of the test structure were constructed of 1.2~cm gypsum wallboard.
The flooring throughout the structure, including the stairwell landing floor, consisted of one layer of 1.3~cm thick cement board on one
layer of 1.9~cm thick plywood supported by wood joists. The first set of stairs, which had eight risers, led from the first floor up to the landing area.
The second set of stairs, which had nine risers, led from the landing area up to the second floor.
The stairs were constructed of 2.5~cm thick clear pine lumber. The two set of stairs were separated by an approximately 0.42~m wide gap in the middle of the stairwell.
This gap was separated from the stairs by a 0.91~m tall barrier constructed of a single piece of gypsum board.
The flue space was open to the first floor.  The flue space was separated from the second floor by a 0.9~m tall barrier constructed of gypsum board.
There was a metal exterior type door at the end of the first floor near the burner.  The door was closed during all experiments.

The fire source was a natural gas diffusion burner.  The burner surface was horizontal, square and 0.45~m on each side, its surface was 0.37~m above the floor, and
it was filled with gravel.  The burner was located near the end of the first floor away from the stairs.



\section{McCaffrey Plume Experiments}

In 1979, at the National Bureau of Standards (now NIST), Bernard McCaffrey measured centerline temperature and velocity profiles above a porous,
refractory burner. There were five distinct heat release rates, ranging from 14~kW to 57~kW. The fuel was natural gas. The burner was square, 0.3~m on each side.
The results of the experiments are reported in Reference~\cite{McCaffrey:NBSIR_79-1910}.


\section{Smyth Slot Burner Experiment}

Kermit Smyth {\em et al.} conducted diffusion flame experiments at NIST using a methane/air Wolfhard-Parker slot burner.  The experiments are described in
detail in Refs.~\cite{Norton:1,Smyth:1}.  The Wolfhard-Parker slot burner consists of an 8~mm wide
central slot flowing fuel surrounded by two 16~mm wide slots flowing dry air with 1~mm separations between the slots.
The slots are 41~mm in length.  Measurements were made of all major species and a number of minor species along with temperature
and velocity.  Experimental uncertainties have been reported as 5~\% for temperature  and 10~\% to 20~\%
for the major species.


\section{Beyler Hood Experiments}

Craig Beyler performed a large number of experiments involving a variety of fuels, fire sizes, burner diameters, and
burner distances beneath a hood~\cite{Beyler:Hood}.  The hood consisted of concentric cylinders separated
by an air gap.  The inner cylinder was shorter than the outer and this allowed combustion products to be removed
uniformly from the hood perimeter.  The exhaust gases were then analyzed to determine species concentrations.
The burner could be raised and lowered with respect to the bottom edge of the hood.  Based on the published
measurement uncertainties, species errors are estimated at 6~\%.


\section{NIST Reduced Scale Enclosure Experiments}

The CO production test series used the NIST Reduced Scale Enclosure (RSE)~\cite{Bryner:1}.  The RSE is a
40~\% scaled version of the ISO 9705 compartment.
It measures 0.98 m wide by 1.46 m deep by 0.98 m tall.  The compartment contains a door centered on the small
face that measures 0.48 m wide by 0.81 m tall.  A 15 cm diameter natural gas burner was positioned in the
center of the compartment.  The burner was on a stand so that its top was 15~cm above the floor.
Species measurements were made inside the upper layer of the compartment at the front near the door
and near the rear of the compartment.


\section{Hamins Methane Burner Experiments}

Anthony Hamins {\em et al.} performed a series of tests on circular gas
burners measuring the radial and vertical radiative heat flux profiles
outside the flame region. The tests are described
in~\cite{Hostikka:3}. Tests at three burner diameters, 0.10 m, 0.38 m
and 1.0 m are used for validation.


\section{Restivo Compartment Air Flow Experiment}

Velocity measurements for forced airflow within a 9~m by 3~m by 3~m high compartment were made by Restivo~\cite{Restivo:1979}. These measurements
have been widely used to validate CFD models designed for indoor air quality applications. It was also used to assess early versions of
FDS~\cite{Emmerich:1,Emmerich:2,Musser:1}. In the experiment, air was forced into the compartment through a 16.8~cm vertical slot along the ceiling
running the width of the compartment with a velocity of 0.455~m/s. A passive exhaust was located near the floor on the opposite wall, with
conditions specified such that there was no buildup of pressure in the enclosure. The component
of velocity in the lengthwise direction was measured in four arrays: two vertical arrays located 3~m and 6~m  from the inlet along the
centerline of the room, and two horizontal arrays located 8.4~cm above the floor and below the ceiling, respectively.
These measurements were taken using hot-wire anemometers. While data on the specific
instrumentation used are not readily available, hot-wire systems tend to have limitations at low velocities,
with typical thresholds of approximately 0.1~m/s.




\section{NRL/HAI Wall Heat Flux Measurements}

Back, Beyler, DiNenno and Tatem~\cite{Back:IAFSS4} measured the heat flux from 9 different sized propane fires set up against a wall composed
of gypsum board. The experiments were sponsored by the Naval Research Laboratory and conducted by Hughes Associates, Inc., of Baltimore, Maryland. The
square sand burner ranged in size from 0.28~m to 0.70~m, and the fires ranged in size from 50~kW to 520~kW.

\section{Ulster SBI Corner Heat Flux Measurements}

Zhang {\em et al.}~\cite{Zhang:IAFSS9} measured the heat flux and flame heights from
fires in the single burning item (SBI) enclosure at the University of Ulster, Northern Ireland.
Thin steel plate probes were used to measure the surface heat flux, and flame
heights were determined by analyzing the instantaneous images extracted from the videos of the
experiments by a CCD camera. Three heat release rates were used -- 30~kW, 45~kW, and 60~kW.

\section{FM Parallel Panel Experiments}

Patricia Beaulieu made heat flux measurements within a set of vertical parallel panels as part of a cooperative research
program between Worcester Polytechnic Institute and FM Global (Factory Mutual)~\cite{Beaulieu:FM}. The experimental
apparatus consisted of two vertical parallel
panels, 2.4~m high and 0.6~m wide, with a sand burner at the base. The objective of the project was to measure the flame spread
rate over various composite wall lining materials, but there were also experiments conducted with inert walls for the purpose of measuring the
heat flux from two fuels, propane and propylene, at heat release rates of 30~kW, 60~kW, and 100~kW.

%\section{INERIS Hydrogen Dispersion Experiment}
%
%A series of experiments was conducted by the Institut National de l'Environnement Industriel et des Risques (INERIS)~\cite{INERIS} to study the release of small amounts of hydrogen in
%a confined space. The experimental setup consisted of a 1~g/s vertical hydrogen release for 240~s from an orifice of 20~mm
%diameter in a rectangular room (garage) of dimensions 3.78~m by 7.2~m by 2.88~m. Two small openings (50~mm diameter) at the front
%and bottom side of the room assured constant pressure conditions. Hydrogen concentrations were measured as a function of time
%during both the release phase and the subsequent diffusion phase (5160~s) at various heights in the garage.


\section{Helium Release in a Reduced Scale Garage Geometry}

FDS simulations were performed to predict the helium release and dispersion in a reduced scale garage geometry. An
approximately quarter scale two-car residential garage with interior dimensions of 1.5~m by 1.5~m by 0.745~m was
constructed with 1.25~cm thick plexiglas. Helium gas (used as a surrogate for hydrogen, due to safety concerns) was release
through a burner that 20.7~cm tall and the exit diameter was 3.6~cm. Helium flow through the burner was controlled through a  mass
flow controller and the specified flow rate for 1~hour and 4~hour releases were 14.95~L/min and 3.74~L/min,
respectively (flow rate was scaled to represent the emptying of a 5~kg of hydrogen fuel tank in 1 or 4 hours).
Idealized leaks were chosen to have areas that provide minimum ventilation
requirements for residential garages. The leaks consisted of either a single 2.4~cm square opening in the center of one face, or
two equal openings (each opening was 2.15~cm square) centered horizontally, at the top and bottom of the front face.



\section{Cable Response to Live Fire -- CAROLFIRE}

CAROLFIRE was a project sponsored by the U.S. Nuclear Regulatory Commission to study the fire-induced thermal
response and functional behavior of electrical cables~\cite{CAROLFIRE}.
The primary objective of CAROLFIRE was to characterize the various modes of electrical
failure ({\em e.g.} hot shorts, shorts to ground) within bundles of power, control and instrument cables.
A secondary objective of the project was to test a simple model to predict \underline{th}ermally-\underline{i}nduced
\underline{e}lectrical \underline{f}ailure (THIEF). The measurements used for these purposes were conducted at Sandia National Laboratories and are described in
Volume II of the CAROLFIRE test report. In brief, there were two series of experiments. The first were conducted within
a heated cylindrical enclosure known as the Penlight apparatus. Single and bundled cables were exposed to various heat fluxes and the
electrical failure modes recorded. The second series of experiments involved cables within trays in a semi-enclosed space under which a gas-fueled burner
created a hot layer to force cable failure.

Petra Andersson and Patrick Van Hees of the Swedish National Testing and Research Institute
(SP) proposed that a cable's thermally-induced electrical failure can be predicted
via a one-dimensional heat transfer calculation, under the assumption that the cable can
be treated as a homogenous cylinder~\cite{Andersson:2005}. Their results for PVC
cables were encouraging and suggested that the simplification of the analysis is reasonable and
that it should extend to other types of cables.
The assumptions underlying the THIEF model are as follows:
\begin{enumerate}
\item The heat penetration into a cable of circular cross section is largely in the radial direction.
This greatly simplifies the analysis, and it is also conservative because it is assumed that
the cable is completely surrounded by the heat source.
\item The cable is homogenous in composition. In reality, a cable is constructed of several
different types of polymeric materials, cellulosic fillers, and a conducting metal, most
often copper.
\item The thermal properties � conductivity, specific heat, and density � of the assumed
homogenous cable are independent of temperature. In reality, both the thermal
conductivity and specific heat of polymers are temperature-dependent, but this
information is very difficult to obtain from manufacturers. More discussion of this
assumption is found below.
\item It is assumed that no decomposition reactions occur within the cable during its heating,
and ignition and burning are not considered in the model. In fact, thermoplastic cables
melt, thermosets form a char layer, and both off-gas volatiles up to and beyond the point
of electrical failure.
\item Electrical failure occurs when the temperature just inside the cable jacket reaches an
experimentally determined value.
\end{enumerate}
Because the CAROLFIRE Penlight experiments tested single cables that were heated uniformly
on all sides, the one-dimensional THIEF model accurately predicted the times for the
temperature inside the cable jacket to reach ``threshold'' values that are typically observed when
the cable fails electrically.





\clearpage

\section{Summary of Experiments}

This section presents a summary of all the experiments described in this chapter in terms of quantities commonly used in fire protection engineering. This ``parameter space''
outlines the range of applicability of the validation performed to date. The parameters are explained below:

\begin{description}
\item[Heat Release Rate, $\dot{Q}$,] is the range of peak heat release rates of the fires in the test series.
\item[Fire Plume Parameter, $Q^*$,] is a useful non-dimensional quantity for plume correlations and flame height estimates.
\be Q^* = \frac{\dot{Q}}{\rho_\infty c_p T_\infty \sqrt{gD} D^2} \ee
\item[Global Equivalence Ratio, $\phi$,] is the ratio of the mass flux of fuel to the mass flux of oxygen into the compartment, divided by the stoichiometric value.
\item[Fire Diameter, $D$,] is the equivalent diameter of the base of the fire.
\item[Characteristic Fire Diameter, $D^*$,] is a useful length scale that incorporates the heat release rate of the fire.
\be D^* = \left( \frac{\dot{Q}}{\rho_\infty c_p T_\infty \sqrt{g}} \right)^{2/5}  \ee
\item[Non-dimensional Ceiling Height, $H/D^*$,] reports the actual ceiling height, $H$, in terms of the fire's characteristic length scale. This is a useful
parameter in assessing the role that the fire plume plays in the simulation. A large value represents a relatively small fire in a relatively high space.
\item[Non-dimensional Ceiling Jet Radius, $r_{cj}/H$,] is mainly used in empirical correlations. It is not as useful for CFD.
\item[Non-dimensional Radiation Radius, $r_{rad}/D$,] is an appropriate measure of whether the radiation measurement is {\em near-field} or {\em far-field}.
\end{description}


\begin{table}[ht!]
\caption{Summary of the major experimental parameters. }
\begin{center}
\begin{tabular}{|l|c|c|c|c|c|c|c|c|}
\hline
Test Series     & $\dot{Q}$     & $Q^*$     & $\phi$    & $D$       & $D^*$     & $H/D*$    & $r_{cj}/H$    & $r_{rad}/D$       \\
                & (kW)          &           &           & (m)       & (m)       &           &               &                   \\ \hline \hline
ATF Corridors   & 50-500        & 0.25-2.5  &  N/A      & 0.5       & 0.3-0.7   &  8.0-3.4  &   7           &  N/A              \\ \hline
FM/SNL          & 500           & 0.6       &  0.04-0.4 & 0.9       & 1.0-1.3   &  8.4      &   1.2-1.6     &  N/A              \\ \hline
NBS Multi-Room  & 100           & 1.4       &  0.04     & 0.34      & 1.0-1.3   &  5.6      &   N/A         &  N/A              \\ \hline
NIST/NRC        & 400-2300      & 0.4-2.1   &  0.1-0.6  & 1.0       & 1.0-1.3   &  2.9-5.7  &   1.3-1.7     &  2.2-5.7          \\ \hline
NIST RSE        & 600           & 62        &  0.83     & 0.15      & 0.8       &  1.3      &   N/A         &  N/A              \\ \hline
Steckler        & 63            & 1.1       &  0.02     & 0.30      & 0.32      &  6.7      &   N/A         &  N/A              \\ \hline
UL/NFPRF        & 4400          & 3.9       &  0.11     & 1.0       & 1.7       &  4.4      &   0.86        &  N/A              \\ \hline
VTT             & 1800-3600     & 1.0       &  0.10     & 1.2-1.6   & 1.0-1.3   &  12-16    &   N/A         &  N/A              \\ \hline
WTC             & 3000          & 2.7       &  0.33     & 1.0       & 1.5       &  2.6      &   0.52        &  1.0              \\ \hline
\end{tabular}
\end{center}
\label{Test_Parameters}
\end{table}
