\chapter{Fire Plumes}

For FDS simulations involving buoyant plumes, a measure of how well the flow field is resolved is given by the
non-dimensional expression $D^*/\dx$, where $D^*$ is a characteristic
fire diameter
\be D^* = \left(
     \frac{\dQ}{\rho_\infty \, c_p \, T_\infty \, \sqrt{g} }
     \right)^\frac{2}{5}  \ee
and $\dx$ is the nominal size of a mesh cell\footnote{The characteristic
fire diameter is related to the characteristic fire size via the
relation $Q^* = (D^*/D)^{5/2}$, where $D$ is the physical diameter of the
fire.}. The quantity $D^*/\dx$ can be thought of as the number of computational cells
spanning the characteristic (not necessarily the physical) diameter of the fire.
The more cells spanning the fire, the better the resolution of the
calculation. It is better to assess the quality of the mesh in terms
of this non-dimensional parameter, rather than an absolute mesh cell size.
For example, a cell size of 10~cm may be ``adequate,'' in some sense,
for evaluating the spread of smoke and heat through a building from a
sizable fire, but may not be appropriate to study a very small, smoldering source.



\section{McCaffrey's Plume Correlation}

The following pages show the results of simulations of McCaffrey's five fires with a grid resolution
such that $D^*/\dx=10$. The mesh cells were all cubes, and no stretching was used.


\begin{figure}[p]
\begin{tabular*}{\textwidth}{l@{\extracolsep{\fill}}r}
\includegraphics[height=2.2in]{FIGURES/McCaffrey_Plume/McCaffrey_Plume_Temperature_14_kW} &
\includegraphics[height=2.2in]{FIGURES/McCaffrey_Plume/McCaffrey_Plume_Velocity_14_kW} \\
\includegraphics[height=2.2in]{FIGURES/McCaffrey_Plume/McCaffrey_Plume_Temperature_22_kW} &
\includegraphics[height=2.2in]{FIGURES/McCaffrey_Plume/McCaffrey_Plume_Velocity_22_kW} \\
\includegraphics[height=2.2in]{FIGURES/McCaffrey_Plume/McCaffrey_Plume_Temperature_33_kW} &
\includegraphics[height=2.2in]{FIGURES/McCaffrey_Plume/McCaffrey_Plume_Velocity_33_kW}
\end{tabular*}
\label{McCaffrey_Plume_1}
\end{figure}

\begin{figure}[p]
\begin{tabular*}{\textwidth}{l@{\extracolsep{\fill}}r}
\includegraphics[height=2.2in]{FIGURES/McCaffrey_Plume/McCaffrey_Plume_Temperature_45_kW} &
\includegraphics[height=2.2in]{FIGURES/McCaffrey_Plume/McCaffrey_Plume_Velocity_45_kW} \\
\includegraphics[height=2.2in]{FIGURES/McCaffrey_Plume/McCaffrey_Plume_Temperature_57_kW} &
\includegraphics[height=2.2in]{FIGURES/McCaffrey_Plume/McCaffrey_Plume_Velocity_57_kW}
\end{tabular*}
\label{McCaffrey_Plume_2}
\end{figure}



\clearpage

\section{VTT Large Hall and FM/SNL Test Series}

Plume temperature measurements are available from the VTT Large Hall and the FM/SNL series.
For all the other full-scale experiments, the temperature above the fire has not been reported, or the fire plume
leans because of the flow pattern within the compartment, or the fire is positioned against a wall.
Only for the VTT and the FM/SNL series are the plumes relatively free from perturbations.

The VTT experiments consist of liquid fuel pan fires positioned in the middle of a large fire test hall.
Plume temperatures are measured at two heights above the fire, 6 m and 12 m.
The flames were observed to extend to about 4 m above the fire pan.


%Photographs from the VTT tests are available. It is difficult to precisely measure the flame height,
%but the photos and videos allow one to make estimates accurate to within a pan diameter.
%Similarly, flame height in FDS is assessed using the visualization program Smokeview.
%There are various ways to render the fire in Smokeview.  The
%most direct method is to show, via three dimensional surface plots, the volume within which the energy from the fire is being released.
%The other method is to show the stoichiometric iso-surface of the mixture fraction.
%FDS tracks the fuel and oxygen via a single scalar variable called the mixture fraction.
%The stoichiometric iso-surface is essentially a sheet on which combustion occurs.  The average vertical extent of either the volume
%in which energy is being released or the stoichiometric mixture fraction iso-surface is the FDS predicted flame height.
%Shown in Figure~\ref{FDS_Flame} are snapshots from the simulation of the 1.6 m diameter heptane pan fire.
%The pan has been approximated as a square because of the requirement by FDS of rectangular geometry.
%Figure~\ref{Simo_Photos} contains photographs of the actual fire.
%The height of the visible flame in the photographs has been estimated to be between 2.4 and 3 pan diameters (3.8 m to 4.8 m).
%The height of the simulated fire fluctuates from 5 m to 6 m during the peak heat release rate phase.


In the FM/SNL experiments, in Tests 4 and 5, thermocouples were positioned near the ceiling directly over the fire pan.
In Test 21, the fire was positioned within an empty electrical cabinet, and the closest near ceiling thermocouple
was used to assess the plume temperature prediction.

Comparisons of the predicted and measured plume temperatures for the VTT and FM/SNL tests are found on the following pages, including a
summary plot at the end of the chapter.

\begin{figure}[p]
\begin{tabular*}{\textwidth}{l@{\extracolsep{\fill}}r}
\includegraphics[height=2.2in]{FIGURES/VTT/VTT_01_v5_Plume_Temperature} &
\includegraphics[height=2.2in]{FIGURES/FM_SNL/FM_SNL_04_v5_Plume_Temperature} \\
\includegraphics[height=2.2in]{FIGURES/VTT/VTT_02_v5_Plume_Temperature} &
\includegraphics[height=2.2in]{FIGURES/FM_SNL/FM_SNL_05_v5_Plume_Temperature} \\
\includegraphics[height=2.2in]{FIGURES/VTT/VTT_03_v5_Plume_Temperature} &
\includegraphics[height=2.2in]{FIGURES/FM_SNL/FM_SNL_21_v5_Plume_Temperature}
\end{tabular*}
\label{VTT_FM_SNL_Plume}
\end{figure}





\begin{figure}[p]
\begin{center}
\begin{tabular}{c}
\includegraphics[width=4.0in]{FIGURES/ScatterPlots/Plume_Temperature} \\
\vspace{0.25in} \\
\end{tabular}
\caption{Summary of Plume Temperature Comparisons.}
\end{center}
\label{Plume_Summary}
\end{figure}

\clearpage
