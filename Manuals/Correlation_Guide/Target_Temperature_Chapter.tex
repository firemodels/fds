% !TEX root = Correlation_Guide.tex

\chapter{Target Temperature}
\label{Target_Temperature_Chapter}

The calculation of target temperature is a common objective of fire modeling analyses. The targets in this validation study include electrical cables as well as unprotected and protected steel members.

\section{Cable Failure Time}

\subsection*{Description}

Even though an electrical cable is considered a ``target'', the cable failure time quantity is included in this study to assess the models' ability to predict the time to cable failure. This is an indirect way of assessing the model prediction of temperature. The model only predicts the interior temperature of the cable, and the failure time is considered as the time at which the predicted temperature rises above an experimentally determined value.

The thermally-induced electrical failure (THIEF) of a cable can be predicted via a simple one-dimensional heat transfer calculation, under the assumption that the cable can be treated as a homogeneous cylinder~\cite{CAROLFIRE}. The governing equation for the cable temperature,
$T(r,t)$, is given by
\be
\rho c \left( \frac{\partial T}{\partial t} \right) = \frac{1}{r} \frac{\partial}{\partial r} k r \left( \frac{\partial T}{\partial r} \right)
\label{eq:cable_temp}
\ee
where $\rho$, $c$ and $k$ are the effective density, specific heat, and thermal conductivity of the solid, respectively, and $r$ is the radius of the cable~(\si{m}).
The boundary condition is defined as
\be
\dot q'' = k \left( \frac{\partial T}{\partial r} \right) (r,t)
\ee
A finite difference approximation to Eq.~\ref{eq:cable_temp} is given by
\be
\rho c \left[ \frac{T_i^{n+1} - T_i^n}{\delta t} \right] = \frac{2 k}{(r_{i+1} + r_i)} \frac{1}{2 \delta r} \left[ r_i \frac{T_{i+1}^n - T_i^n}{\delta r} - r_{i-1} \frac{T_{i}^n - T_{i-1}^n}{\delta r} + r_i \frac{T_{i+1}^{n+1} - T_i^{n+1}}{\delta r} - r_{i-1} \frac{T_{i}^{n+1} - T_{i-1}^{n+1}}{\delta r} \right]
\ee
where the time step $\delta t$ is given by
\be
\delta t = \frac{c \rho \delta r^2}{2 k}
\ee


\clearpage


\subsection*{Verification}

This example case is based on Penlight Test 7 from the Cable Response to Live Fire (CAROLFIRE) series. This test involved a cable inside of conduit that was located in a heated cylindrical enclosure.

\begin{table}[!ht]
\caption[Verification case, cable failure time]
{Verification case, cable failure time.}
\begin{center}
\begin{tabular}{|c|c|c|c|}
\hline
\multicolumn{4}{|c|}{}                                                                                   \\
\multicolumn{4}{|c|}{User-Specified Input}                                                               \\
\multicolumn{4}{|c|}{}                                                                                   \\ \hline
\multicolumn{2}{|c|}{}                             &  \multicolumn{2}{c|}{}                              \\
\multicolumn{2}{|l|}{\rb{Parameter}}               &  \multicolumn{2}{l|}{\rb{Value}}                    \\ \hline \hline
\multicolumn{2}{|l|}{Time Ramp}                    &  \multicolumn{2}{l|}{0, 80, 820, 1240, 1800, 1900}  \\ \hline
\multicolumn{2}{|l|}{Temperature Ramp}             &  \multicolumn{2}{l|}{24, 460, 460, 460, 460, 0}     \\ \hline
\multicolumn{2}{|l|}{Cable Diameter (mm)}          &  \multicolumn{2}{l|}{16.3}                          \\ \hline
\multicolumn{2}{|l|}{Mass per Unit Length (kg/m)}  &  \multicolumn{2}{l|}{0.529}                         \\ \hline
\multicolumn{2}{|l|}{Jacket Thickness (mm)}        &  \multicolumn{2}{l|}{1.5}                           \\ \hline
\multicolumn{2}{|l|}{Conduit Diameter (mm)}        &  \multicolumn{2}{l|}{50}                            \\ \hline
\multicolumn{2}{|l|}{Conduit Thickness (mm)}       &  \multicolumn{2}{l|}{4.9}                           \\ \hline
\multicolumn{2}{|l|}{$T_\infty$ ($^\circ$C)}       &  \multicolumn{2}{l|}{24}                            \\ \hline
\multicolumn{2}{c}{}                                                                                     \\ \hline
\multicolumn{4}{|c|}{}                                                                                   \\
\multicolumn{4}{|c|}{Expected Output}                                                                    \\
\multicolumn{4}{|c|}{}                                                                                   \\ \hline
           &                    &                    &                                                   \\
           &  \rb{Exposing}     &  \rb{Cable}        &  \rb{Conduit}                                     \\
\rb{Time}  &  \rb{Temperature}  &  \rb{Temperature}  &  \rb{Temperature}                                 \\
\rb{(s)}   &  \rb{($^\circ$C)}  &  \rb{($^\circ$C)}  &  \rb{($^\circ$C)}                                 \\ \hline \hline
50         &  296.3             &  24.1              &  32.4                                             \\ \hline
80         &  460.0             &  24.6              &  52.3                                             \\ \hline
1473       &  460.0             &  400               &  440.3                                            \\ \hline
\end{tabular}
\end{center}
\end{table}


\clearpage


\subsection*{Validation}

A summary of the comparisons between predicted and measured cable failure times (the time at which the cable reaches its threshold failure temperature) is shown in Fig.~\ref{Surface_Temperature_THIEF_Summary}.

\begin{figure}[!ht]
\begin{center}
\begin{tabular}{l}
\includegraphics[width=4.0in]{SCRIPT_FIGURES/Scatterplots/Cable_Failure_Time}
\end{tabular}
\end{center}
\caption[Summary of cable failure time predictions]
{Summary of cable failure time predictions.}
\label{Surface_Temperature_THIEF_Summary}
\end{figure}

\clearpage


\section{Unprotected Steel Temperature}

\subsection*{Description}

The temperature rise, $\Delta T\sb{s}$, of an unprotected steel member exposed to fire can be predicted using~\cite{SFPE:Milke2}
\be
\Delta T\sb{s} = \frac{F}{V} \frac{1}{\rho\sb{s} c\sb{s}} \left[ h\sb{c} (T\sb{f} - T\sb{s}) + \sigma \epsilon (T\sb{f}^4 - T\sb{s}^4) \right] \Delta t
\label{eq:unprotected_steel}
\ee
where $F/V$ is the ratio of heated surface area to volume~(\si{m^{-1}}), $\rho\sb{s}$ is the density of steel~(\si{kg/m^3}), $c\sb{s}$ is the specific heat of steel~(\si{J/(kg.K)}), $h\sb{c}$ is the convective heat transfer coefficient~(\si{W/(m^2.K)}), $T\sb{f}$ is the exposing fire temperature~(\si{K}), $T\sb{s}$ is the steel temperature~(\si{K}), $\sigma$ is the Stefan-Boltzmann constant (\si{W/(m^2.K^4)}), $\epsilon$ is the flame emissivity~(-), and $\Delta t$ is the time step~(\si{s}). Note that the HGL temperature, plume temperature, or other exposing temperature can be used as the fire temperature, $T\sb{f}$.


\clearpage


\subsection*{Verification}

This example case is based on the 1.1 m Diesel Fire Test from the SP AST Column series. This test involved a large test hall with a steel column located in the middle of a diesel pool fire.

\begin{table}[!ht]
\caption[Verification case, unprotected steel temperature]
{Verification case, unprotected steel temperature.}
\begin{center}
\begin{tabular}{|c|c|c|}
\hline
\multicolumn{3}{|c|}{}                                                                   \\
\multicolumn{3}{|c|}{User-Specified Input}                                               \\
\multicolumn{3}{|c|}{}                                                                   \\ \hline
\multicolumn{2}{|c|}{}                               &  \multicolumn{1}{c|}{}            \\
\multicolumn{2}{|l|}{\rb{Parameter}}                 &  \multicolumn{1}{c|}{\rb{Value}}  \\ \hline \hline
\multicolumn{2}{|l|}{$F/V$ (1/m)}                    &  \multicolumn{1}{c|}{205}         \\ \hline
\multicolumn{2}{|l|}{$\rho\sb{s}$ (kg/m$^3$)}        &  \multicolumn{1}{c|}{7833}        \\ \hline
\multicolumn{2}{|l|}{$c\sb{s}$ (\si{kJ/(kg.K)})}     &  \multicolumn{1}{c|}{0}           \\ \hline
\multicolumn{2}{|l|}{$\epsilon$ (-)}                 &  \multicolumn{1}{c|}{0}           \\ \hline
\multicolumn{2}{|l|}{$h\sb{c}$ (\si{W/(m^2.K)})}     &  \multicolumn{1}{c|}{25}          \\ \hline \hline
\multicolumn{2}{|l|}{Correlation for $T\sb{f}$ (-)}  &  \multicolumn{1}{c|}{McCaffrey}   \\ \hline \hline
\multicolumn{2}{|l|}{$\dot Q$ (kW)}                  &  \multicolumn{1}{c|}{1434}        \\ \hline
\multicolumn{2}{|l|}{Height (m)}                     &  \multicolumn{1}{c|}{1}           \\ \hline
\multicolumn{2}{|l|}{$T_\infty$ ($^\circ$C)}         &  \multicolumn{1}{c|}{20}          \\ \hline
\multicolumn{2}{c}{}                                                                     \\ \hline
\multicolumn{3}{|c|}{}                                                                   \\
\multicolumn{3}{|c|}{Expected Output}                                                    \\
\multicolumn{3}{|c|}{}                                                                   \\ \hline
           &                    &                                                        \\
           &  \rb{Fire}         &  \rb{Steel}                                            \\
\rb{Time}  &  \rb{Temperature}  &  \rb{Temperature}                                      \\
\rb{(s)}   &  \rb{($^\circ$C)}  &  \rb{($^\circ$C)}                                      \\ \hline \hline
15         &  872.81            &  89.7                                                  \\ \hline
30         &  872.81            &  162.4                                                 \\ \hline
45         &  872.81            &  232.8                                                 \\ \hline
\end{tabular}
\end{center}
\end{table}


\clearpage


\section{Protected Steel Temperature}
\label{info:protected_steel_temperature}

\subsection*{Description}

The temperature rise, $\Delta T\sb{s}$, of a protected steel member exposed to fire can be predicted, but we must first determine if the thermal capacity of the insulation layer should be accounted for or if it can be neglected.
\be
\Delta T\sb{s} = \left\{ \begin{array}{cl}
   k\sb{i} \left( \frac{T\sb{f} - T\sb{s}}{c\sb{s} h \frac{W}{D}} \right) \Delta t        &  c\sb{s} \frac{W}{D} > 2 c\sb{i} \rho\sb{i} h \\[0.1in]
   \frac{k\sb{i}}{h} \left( \frac{T\sb{f} - T\sb{s}}{c\sb{s} \frac{W}{D} + \frac{1}{2} c\sb{i} \rho\sb{i} h} \right) \Delta t  &  c\sb{s} \frac{W}{D} < 2 c\sb{i} \rho\sb{i} h
   \end{array} \right.
\label{eq:protected_steel}
\ee
where $k\sb{i}$ is the thermal conductivity of the insulation material~(\si{W/(m.K)}), $T\sb{f}$ is the exposing fire temperature~(\si{K}), $T\sb{s}$ is the steel temperature~(\si{K}), $c\sb{s}$ is the specific heat of steel~(\si{J/(kg.K)}), $c\sb{i}$ is the specific heat of the insulation material~(\si{J/(kg.K)}), $h$ is the thickness of the insulation~(\si{m}), $W/D$ is the ratio of the weight of steel section per unit length to the heated perimeter~(\si{kg/m^2}), $\rho\sb{i}$ is the density of the insulating material~(\si{kg/m^3}), and $\Delta t$ is the time step~(\si{s}). Note that the HGL temperature, plume temperature, or other exposing temperature can be used as the fire temperature, $T\sb{f}$.


\clearpage


\subsection*{Verification}

This example case examines the temperature of a bar structural element in Test 4 of the World Trade Center (WTC) series. This test involved a simple compartment with a heptane spray burner and various structural elements with varying amounts of sprayed fire-resistive materials.

\begin{table}[!ht]
\caption[Verification case, protected steel temperature]
{Verification case, protected steel temperature.}
\begin{center}
\begin{tabular}{|c|c|c|}
\hline
\multicolumn{3}{|c|}{}                                                                   \\
\multicolumn{3}{|c|}{User-Specified Input}                                               \\
\multicolumn{3}{|c|}{}                                                                   \\ \hline
\multicolumn{2}{|c|}{}                               &  \multicolumn{1}{c|}{}            \\
\multicolumn{2}{|l|}{\rb{Parameter}}                 &  \multicolumn{1}{c|}{\rb{Value}}  \\ \hline \hline
\multicolumn{2}{|l|}{$c\sb{s}$ (\si{kJ/(kg.K)})}     &  \multicolumn{1}{c|}{0.450}       \\ \hline
\multicolumn{2}{|l|}{$W/D$ (kg/m$^2$)}               &  \multicolumn{1}{c|}{50.1}        \\ \hline
\multicolumn{2}{|l|}{$k\sb{i}$ (\si{W/(m.K)})}       &  \multicolumn{1}{c|}{0.10}        \\ \hline
\multicolumn{2}{|l|}{$\rho\sb{i}$ (kg/m$^3$)}        &  \multicolumn{1}{c|}{208}         \\ \hline
\multicolumn{2}{|l|}{$c\sb{i}$ (\si{kJ/(kg.K)})}     &  \multicolumn{1}{c|}{2.0}         \\ \hline
\multicolumn{2}{|l|}{$h\sb{i}$ (m)}                  &  \multicolumn{1}{c|}{0.0191}      \\ \hline \hline
\multicolumn{2}{|l|}{Correlation for $T\sb{f}$ (-)}  &  \multicolumn{1}{c|}{MQH}         \\ \hline \hline
\multicolumn{2}{|l|}{$\dot Q$ (kW)}                  &  \multicolumn{1}{c|}{3200}        \\ \hline
\multicolumn{2}{|l|}{$L$ (m)}                        &  \multicolumn{1}{c|}{7.04}        \\ \hline
\multicolumn{2}{|l|}{$W$ (m)}                        &  \multicolumn{1}{c|}{3.60}        \\ \hline
\multicolumn{2}{|l|}{$H$ (m)}                        &  \multicolumn{1}{c|}{3.82}        \\ \hline
\multicolumn{2}{|l|}{$H\sb{o}$ (m)}                  &  \multicolumn{1}{c|}{2.82}        \\ \hline
\multicolumn{2}{|l|}{$W\sb{o}$ (m)}                  &  \multicolumn{1}{c|}{2.4}         \\ \hline
\multicolumn{2}{|l|}{$k$ (\si{kW/(m.K)})}            &  \multicolumn{1}{c|}{0.00012}     \\ \hline
\multicolumn{2}{|l|}{$\rho$ (kg/m$^3$)}              &  \multicolumn{1}{c|}{737}         \\ \hline
\multicolumn{2}{|l|}{$c$ (\si{kJ/(kg.K)})}           &  \multicolumn{1}{c|}{1.42}        \\ \hline
\multicolumn{2}{|l|}{$\delta$ (m)}                   &  \multicolumn{1}{c|}{0.0254}      \\ \hline
\multicolumn{2}{|l|}{$T_\infty$ ($^\circ$C)}         &  \multicolumn{1}{c|}{20}          \\ \hline
\multicolumn{2}{c}{}                                                                     \\ \hline
\multicolumn{3}{|c|}{}                                                                   \\
\multicolumn{3}{|c|}{Expected Output}                                                    \\
\multicolumn{3}{|c|}{}                                                                   \\ \hline
           &                    &                                                        \\
           &  \rb{Fire}         &  \rb{Steel}                                            \\
\rb{Time}  &  \rb{Temperature}  &  \rb{Temperature}                                      \\
\rb{(s)}   &  \rb{($^\circ$C)}  &  \rb{($^\circ$C)}                                      \\ \hline \hline
15         &  313.5             &  21.0                                                  \\ \hline
30         &  349.4             &  22.3                                                  \\ \hline
45         &  372.4             &  23.66                                                 \\ \hline
\end{tabular}
\end{center}
\end{table}


\clearpage


\subsection*{Validation}

For the unprotected and protected steel cases, a summary of the comparisons between peak predicted and measured target temperatures is shown in Fig.~\ref{Surface_Temperature_Steel_Summary}.

\begin{figure}[!ht]
\begin{center}
\begin{tabular}{l}
\includegraphics[width=4.0in]{SCRIPT_FIGURES/Scatterplots/Target_Temperature}
\end{tabular}
\end{center}
\caption[Summary of target temperature predictions]
{Summary of target temperature predictions.}
\label{Surface_Temperature_Steel_Summary}
\end{figure}

