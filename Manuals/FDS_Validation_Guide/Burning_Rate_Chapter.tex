
\chapter{Burning Rate}

This chapter contains a series of validation exercises where the aim is to {\em predict} the burning rate of the fuel. Most of the simulations included in the previous chapters involved
a {\em specified} burning or heat release rate. Here, the objective is to apply measured thermophysical properties of the material and predict its burning rate, either with a
specified heat flux or as a free burn.


\section{FAA Polymers}

The U.S.~Federal Aviation Administration (FAA) has studied various plastics that are commonly used aboard commercial aircraft. 

This section presents measured properties of various polymers and the numerical predictions of their mass loss and/or burning rates under constant heat heating. 
Two types of experiments are considered. First, the NIST Gasification Apparatus is used to measure the mass loss rate of non-burning samples in a nitrogen atmosphere. 
Second, the standard Cone Calorimeter~\cite{conecal} is used to measure the heat release rate of materials in a normal atmosphere. 
When just the mass loss rate of a non-burning sample has been measured, FDS is run in ``solid phase only'' mode; that is, a 1-D heat conduction calculation is performed in a
single grid cell. The result is the predicted mass loss rate as a function of time. To simulate a cone calorimeter experiment, FDS simulates the burning of a 10~cm by 10~cm sample with
a specified heat flux to represent the effect of the cone heater. The cone itself is not included in the simulation. As the sample burns, FDS predicts the additional radiative and
convective heating of the sample as a result of the fire.

In general, the burning/gasification rate of a charring polymer is more difficult to predict than a non-charring one because there are more parameters that need to be measured and
more complicated behavior, like intumescence, need to be considered.


\newpage

\subsection{Non-Charring Polymers in the Gasification Apparatus}

A non-charring polymer is considered one of the easier solids to model because it typically involves only a single, first order reaction that converts solid plastic to fuel vapor.
No residue is formed and the plastic is completely pyrolyzed. Table~\ref{FAA_Properties}
lists nine parameters for each polymer studied. These values have been input directly into FDS, and the predicted mass loss rates are compared with measured values from the NIST
Gasification Apparatus, a device that pyrolyzes the solid in a nitrogen environment to prevent combustion of fuel gases. The results are shown in Fig.~\ref{FAA_Polymers}. The exposing
heat flux was 52~kW/m$^2$. A 1~cm layer of insulation was placed under the sample. Its properties are given in Ref.~\cite{Stoliarov:CF2009}.


\begin{table}[h!]
\caption[FAA non-charring polymer properties.]{Input parameters for FAA Polymers non-charring samples. Courtesy S.~Stoliarov, M.~McKinnon and J.~Li, University of Maryland.}
\begin{center}
\begin{tabular}{|l|c|c|c|c|c|l|l|}
\hline
Property                    & Units         & HDPE                  & HIPS                  & PMMA                  & Unc. (\%)  & Method                &  Ref.                    \\ \hline \hline
Density                     & kg/m$^3$      & 860                   & 950                   & 1100                   & 5     & Direct                &  \cite{Stoliarov:CF2009}  \\ \hline
Conductivity                & W/m/K         & 0.29                  & 0.22                  & 0.20                 & 15    & Thermoflixer          &  \cite{Stoliarov:CF2009}  \\ \hline
Specific Heat               & kJ/kg/K       & 3.5                   & 2.0                   & 2.2                    & 15    & DSC                   &  \cite{Stoliarov:PDS2008}  \\ \hline
Emissivity                  &               & 0.92                  & 0.86                  & 0.85                  & 20    & Sphere                &  \cite{Hallman:PES1974}  \\ \hline
Absorption Coef.            & m$^{-1}$      & 1300                  & 2700                  & 2700                  & 50    & FTIR                  &  \cite{Tsilingiris:ECM2003}  \\ \hline
Pre-Exp.~Factor             & s$^{-1}$      & $4.8 \times 10^{22}$  & $1.2 \times 10^{16}$  & $8.5 \times 10^{12}$    & 50    & TGA                   &  \cite{Stoliarov:CF2009}  \\ \hline
Activation Energy           & kJ/kmol       & $3.49 \times 10^{5}$  & $2.47 \times 10^{5}$  & $1.88 \times 10^{5}$  & 3     & TGA                   &  \cite{Stoliarov:CF2009}  \\ \hline
Heat of Reaction            & kJ/kg         & 920                   & 1000                  & 870                 & 15    & DSC                   &  \cite{Stoliarov:PDS2008}  \\ \hline
\end{tabular}
\end{center}
\label{FAA_Properties}
\end{table}

\begin{tabbing}
Direct  \hspace{0.5in}     \= Direct measurement of mass and volume \\
Thermoflixer               \> Transient line source method \\
DSC                        \> Differential Scanning Calorimetry \\
Sphere                     \> Integrating sphere \\
FTIR                       \> Fourier Transform Infrared Spectroscopy \\
TGA                        \> Thermogravimetric Analysis
\end{tabbing}


\begin{figure}[h!]
\begin{tabular*}{\textwidth}{l@{\extracolsep{\fill}}r}
\includegraphics[height=2.2in]{FIGURES/FAA_Polymers/FAA_Polymers_HDPE} &
\includegraphics[height=2.2in]{FIGURES/FAA_Polymers/FAA_Polymers_HIPS} \\
\includegraphics[height=2.2in]{FIGURES/FAA_Polymers/FAA_Polymers_PMMA}&
\end{tabular*}
\caption[Results of FAA Polymers, non-charring, comparison]{Comparison of predicted and measured mass loss rates for three non-charring polymers exposed to a heat flux of 52~kW/m$^2$ in a
nitrogen environment.}
\label{FAA_Polymers}
\end{figure}

\clearpage

\subsection{Polycarbonate (PC) in the Cone Calorimeter}

Table~\ref{Properties_PC} lists the measured properties of polycarbonate. These values have been input directly into FDS, and the predicted heat release rates 
are compared with measured values from the Cone
Calorimeter. The results for samples of various thicknesses and imposed heat fluxes are shown in Fig.~\ref{HRR_PC}.
A 1~cm layer of Kaowool insulation was placed under the sample. Its properties are given in Ref.~\cite{Stoliarov:CF2010}.


\begin{table}[h!]
\caption[Properties of polycarbonate (PC).]{Properties of polycarbonate (PC). Courtesy S.~Stoliarov, University of Maryland.}
\begin{center}
\begin{tabular}{|l|c|c|c|c|c|l|l|}
\hline
Property                    & Units         & Value                             & Method                &  Reference                    \\ \hline \hline
Polymer Density             & kg/m$^3$      & 1180 $\pm$ 60                     & Direct Measurement    &  \cite{Stoliarov:CF2010}      \\ \hline
Polymer Conductivity        & W/m/K         & 0.22 $\pm$ 0.03                   & Literature            &  \cite{Stoliarov:CF2010}      \\ \hline
Polymer Specific Heat       & kJ/kg/K       & 1.9 $\pm$ 0.3                     & DSC                   &  \cite{Stoliarov:PDS2008}     \\ \hline
Polymer Emissivity          &               & 0.90 $\pm$ 0.05                   &Integrating  Sphere                &  \cite{Hallman:PES1974}       \\ \hline
Polymer Absorption Coef.    & m$^{-1}$      & 1770 $\pm$ 590                    & FTIR                  &  \cite{Tsilingiris:ECM2003}   \\ \hline
Char Density                & kg/m$^3$      & 248                               & Cone Calorimeter      &  \cite{Stoliarov:CF2010}      \\ \hline
Char Conductivity           & W/m/K         & 0.37                              & Cone Calorimeter      &  \cite{Stoliarov:CF2010}      \\ \hline
Char Specific Heat          & kJ/kg/K       & 1.72 $\pm$ 0.17                   & Pulsed Current        &  \cite{Stoliarov:CF2010,Matsumoto:1996}  \\ \hline
Char Emissivity             &               & 0.85 $\pm$ 0.05                   & Pulsed Current        &  \cite{Stoliarov:CF2010,Matsumoto:1996}  \\ \hline
Char Absorption Coef.       & m$^{-1}$      & Opaque                            & Assumption            &  \cite{Stoliarov:CF2010}      \\ \hline
Pre-Exp.~Factor             & s$^{-1}$      & $(1.9 \pm 1.1) \times 10^{18}$    & TGA                   &  \cite{Stoliarov:CF2010}      \\ \hline
Activation Energy           & kJ/kmol       & $(2.95 \pm 0.06) \times 10^{5}$   & TGA                   &  \cite{Stoliarov:CF2010}      \\ \hline
Heat of Reaction            & kJ/kg         & 830 $\pm$ 140                     & DSC                   &  \cite{Stoliarov:PDS2008}     \\ \hline
Heat of Combustion          & kJ/kg         & 25600 $\pm$ 130                   & MCC                   &  \cite{Stoliarov:CF2010}      \\ \hline
Combustion Efficiency       &               & 0.84 $\pm$ 0.03                   & Cone Calorimeter      &  \cite{Stoliarov:CF2010}      \\ \hline
\end{tabular}
\end{center}
\label{Properties_PC}
\end{table}

\begin{figure}[p]
\begin{tabular*}{\textwidth}{l@{\extracolsep{\fill}}r}
\includegraphics[height=2.2in]{FIGURES/FAA_Polymers/FAA_Polymers_PC_6_75} &
\includegraphics[height=2.2in]{FIGURES/FAA_Polymers/FAA_Polymers_PC_6_92} \\
\includegraphics[height=2.2in]{FIGURES/FAA_Polymers/FAA_Polymers_PC_6_50} &
\includegraphics[height=2.2in]{FIGURES/FAA_Polymers/FAA_Polymers_PC_3_75} \\
\includegraphics[height=2.2in]{FIGURES/FAA_Polymers/FAA_Polymers_PC_9_75} &
\end{tabular*}
\caption[Heat release rate of polycarbonate (PC).]{Comparison of predicted and measured heat release rates for polycarbonate (PC).}
\label{HRR_PC}
\end{figure}

\clearpage


\subsection{Poly(vinyl chloride) (PVC) in the Cone Calorimeter}

Table~\ref{Properties_PVC} lists the measured properties of poly(vinyl chloride). These values have been input directly into FDS, and the predicted heat release rates
are compared with measured values from the Cone
Calorimeter. The results for samples of various thicknesses and imposed heat fluxes are shown in Fig.~\ref{HRR_PVC}.
A 1~cm layer of Kaowool insulation was placed under the sample. Its properties are given in Ref.~\cite{Stoliarov:CF2010}.

It is assumed that the polymer decomposes via a two-step reaction:
\begin{eqnarray}
   \hbox{Polymer} &\to& \hbox{Char 1} + \hbox{Gas 1}  \\
   \hbox{Char 1}  &\to& \hbox{Char 2} + \hbox{Gas 2}  
\end{eqnarray}


\begin{table}[h!]
\caption[Properties of poly(vinyl chloride) (PVC).]{Properties of poly(vinyl chloride) (PVC). Courtesy S.~Stoliarov, University of Maryland.}
\begin{center}
\begin{tabular}{|l|c|c|c|c|c|l|l|}
\hline
Property                    & Units         & Value                             & Method                    &  Reference                    \\ \hline \hline
Polymer Density             & kg/m$^3$      & 1430 $\pm$ 70                     & Direct Measurement        &  \cite{Stoliarov:CF2010}      \\ \hline
Polymer Conductivity        & W/m/K         & 0.17 $\pm$ 0.01                   & Literature                &  \cite{Stoliarov:CF2010}      \\ \hline
Polymer Specific Heat       & kJ/kg/K       & 1.55 $\pm$ 0.25                   & DSC                       &  \cite{Stoliarov:PDS2008}     \\ \hline
Polymer Emissivity          &               & 0.90 $\pm$ 0.05                   & Integrating Sphere        &  \cite{Hallman:PES1974}       \\ \hline
Polymer Absorption Coef.    & m$^{-1}$      & 2145 $\pm$ 715                    & FTIR                      &  \cite{Tsilingiris:ECM2003}   \\ \hline
Char 1 Density              & kg/m$^3$      & 629                               & Assumed constant volume   &  \cite{Stoliarov:CF2010}      \\ \hline
Char 1 Conductivity         & W/m/K         & 0.17                              & Assumed same as polymer   &  \cite{Stoliarov:CF2010}      \\ \hline
Char 1 Specific Heat        & kJ/kg/K       & 1.55 $\pm$ 0.25                   & Assumed same as polymer   &  \cite{Stoliarov:CF2010}      \\ \hline
Char 1 Emissivity           &               & 0.90 $\pm$ 0.05                   & Assumed same as polymer   &  \cite{Stoliarov:CF2010}      \\ \hline
Char 1 Absorption Coef.     & m$^{-1}$      & 2453                              & Fit to cone data          &  \cite{Stoliarov:CF2010}      \\ \hline
Char 2 Density              & kg/m$^3$      & 296                               & Assumed constant volume   &  \cite{Stoliarov:CF2010}      \\ \hline
Char 2 Conductivity         & W/m/K         & 0.26                              & Fit to cone data          &  \cite{Stoliarov:CF2010}      \\ \hline
Char 2 Specific Heat        & kJ/kg/K       & 1.72 $\pm$ 0.17                   & Pulsed Current   &  \cite{Stoliarov:CF2010,Matsumoto:1996}  \\ \hline
Char 2 Emissivity           &               & 0.85 $\pm$ 0.05                   & Pulsed Current    &  \cite{Stoliarov:CF2010,Matsumoto:1996}  \\ \hline
Char 2 Absorption Coef.     & m$^{-1}$      & Opaque                            & Assumed                   &  \cite{Stoliarov:CF2010}      \\ \hline
Reac 1 Pre-Exp.~Factor      & s$^{-1}$      & $(1.4 \pm 0.8) \times 10^{33}$    & TGA                       &  \cite{Stoliarov:CF2010}      \\ \hline
Reac 1 Activation Energy    & kJ/kmol       & $(3.67 \pm 0.07) \times 10^{5}$   & TGA                       &  \cite{Stoliarov:CF2010}      \\ \hline
Reac 1 Char Yield           &               & $0.44 \pm 0.01$                   & TGA                       &  \cite{Stoliarov:CF2010}      \\ \hline
Reac 1 Heat of Reaction     & kJ/kg         & 170 $\pm$ 17                      & DSC                       &  \cite{Stoliarov:PDS2008}     \\ \hline
Gas 1 Heat of Combustion    & kJ/kg         & 2700 $\pm$ 300                    & MCC                       &  \cite{Stoliarov:CF2010}      \\ \hline
Gas 1 Combustion Efficiency &               & 0.75 $\pm$ 0.03                   & Cone Calorimeter          &  \cite{Stoliarov:CF2010}      \\ \hline
Reac 2 Pre-Exp.~Factor      & s$^{-1}$      & $(3.5 \pm 2.1) \times 10^{12}$    & TGA                       &  \cite{Stoliarov:CF2010}      \\ \hline
Reac 2 Activation Energy    & kJ/kmol       & $(2.07 \pm 0.04) \times 10^{5}$   & TGA                       &  \cite{Stoliarov:CF2010}      \\ \hline
Reac 2 Char Yield           &               & $0.47 \pm 0.01$                   & TGA                       &  \cite{Stoliarov:CF2010}      \\ \hline
Reac 2 Heat of Reaction     & kJ/kg         & 1200 $\pm$ 900                    & DSC                       &  \cite{Stoliarov:PDS2008}     \\ \hline
Gas 2 Heat of Combustion    & kJ/kg         & 36500 $\pm$ 1800                  & MCC                       &  \cite{Stoliarov:CF2010}      \\ \hline
Gas 2 Combustion Efficiency &               & 0.75 $\pm$ 0.03                   & Cone Calorimeter          &  \cite{Stoliarov:CF2010}      \\ \hline
\end{tabular}
\end{center}
\label{Properties_PVC}
\end{table}

\begin{figure}[p]
\begin{tabular*}{\textwidth}{l@{\extracolsep{\fill}}r}
\includegraphics[height=2.2in]{FIGURES/FAA_Polymers/FAA_Polymers_PVC_6_75} &
\includegraphics[height=2.2in]{FIGURES/FAA_Polymers/FAA_Polymers_PVC_6_92} \\
\includegraphics[height=2.2in]{FIGURES/FAA_Polymers/FAA_Polymers_PVC_6_50} &
\includegraphics[height=2.2in]{FIGURES/FAA_Polymers/FAA_Polymers_PVC_3_75} \\
\includegraphics[height=2.2in]{FIGURES/FAA_Polymers/FAA_Polymers_PVC_9_75} &
\end{tabular*}
\caption[Heat release rate of poly(vinyl chloride) (PVC).]{Comparison of predicted and measured heat release rates for poly(vinyl chloride) (PVC).}
\label{HRR_PVC}
\end{figure}

\clearpage



\subsection{Poly(butylene terephtalate) (PBT) in the Gasification Apparatus}

Poly(butylene terephtalate) (PBT) is heated in the Gasification Apparatus at a heat flux of
50~kW/m$^2$. The properties of PBT are listed in Table~\ref{Properties_PBT}. 

\begin{table}[h!]
\caption[Properties of poly(butylene terephtalate) (PBT).]{Properties of poly(butylene terephtalate) (PBT). Courtesy S.~Stoliarov, University of Maryland.}
\begin{center}
\begin{tabular}{|l|c|c|c|c|}
\hline
Property                & Units     & Value                             & Method                & Reference         \\ \hline \hline
Density                 & kg/m$^3$  & 1300 $\pm$ 70                     & Direct Measurement    & \cite{Kempel:1}   \\ \hline
Specific Heat           & kJ/kg/K   & 2.23 $\pm$ 0.34                   & DSC                   & \cite{Kempel:1}   \\ \hline
Conductivity            & W/m/K     & 0.29 $\pm$ 0.05                   & Transient Line Source & \cite{Kempel:1}   \\ \hline
Emissivity              &           & 0.88 $\pm$ 0.05                   & FTIR                  & \cite{Linteris:2} \\ \hline
Absorbtion Coefficient  & m$^{-1}$  & 2561 $\pm$ 140                    & FTIR                  & \cite{Linteris:2} \\ \hline
Pre-Exp. Factor         & s$^{-1}$  & $(2.49 \pm 0.62) \times 10^{14}$  & TGA                   & \cite{Kempel:1}   \\ \hline
Activation Energy       & kJ/kmol   & $(2.12 \pm 0.53) \times 10^{5}$   & TGA                   & \cite{Kempel:1}   \\ \hline
Heat of Reaction        & kJ/kg     & 507                               & DSC, Literature       & \cite{Kempel:1,Lyon:Ency2005}   \\ \hline
Heat of Combustion      & kJ/kg     & 19500                             & Cone Calorimeter      & \cite{Kempel:1}   \\ \hline
Combustion Efficiency   &           & 1                                 & Assumption            & \cite{Kempel:1}   \\ \hline
\end{tabular}
\end{center}
\label{Properties_PBT}
\end{table}


\begin{figure}[h!]
\begin{center}
\includegraphics[height=2.2in]{FIGURES/FAA_Polymers/FAA_Polymers_PBT}
\caption[Heat release rate of poly(butylene terephtalate) (PBT).]{Comparison of predicted and measured heat release rates for poly(butylene terephtalate) (PBT).}
\label{HRR_PBT}
\end{center}
\end{figure}


\clearpage


\subsection{PBT with Glass Fibers (PBT-GF) in the Gasification Apparatus}

Poly(butylene terephtalate) (PBT), blended with 30~\% by mass glass fibers, is heated in the Gasification Apparatus at a heat flux of
50~kW/m$^2$. The properties of PBT-GF are listed in Table~\ref{Properties_PBT-GF}.

\begin{table}[h!]
\caption[Properties of poly(butylene terephtalate) with glass fibers (PBT-GF).]{Properties of poly(butylene terephtalate) with glass fibers (PBT-GF).
Courtesy S.~Stoliarov, University of Maryland.}
\begin{center}
\begin{tabular}{|l|c|c|c|c|}
\hline
Property                &      Units    &      Value                        & Method                    & Reference         \\ \hline \hline
Polymer Density         &     kg/m$^3$  &  1520 $\pm$ 80                    & Direct Measurement        & \cite{Kempel:1}   \\ \hline
Polymer Specific Heat   &    kJ/kg/K    & 1.68 $\pm$ 0.26                   & DSC                       & \cite{Kempel:1}   \\ \hline
Polymer Conductivity    &      W/m/K    & 0.36 $\pm$ 0.06                   & Transient Line Source     & \cite{Kempel:1}   \\ \hline
Polymer Emissivity      &               & 0.87 $\pm$ 0.05                   & FTIR                      & \cite{Linteris:2} \\ \hline
Polymer Absorbtion Coef.&  m$^{-1}$     & 2860 $\pm$ 150                    & FTIR                      & \cite{Linteris:2} \\ \hline
Char Density            &     kg/m$^3$  &        482                        & Assumed Constant Volume   & \cite{Kempel:1}   \\ \hline
Char Specific Heat      &    kJ/kg/K    &        0.85                       & Literature                & \cite{SCHOTT}     \\ \hline
Char Conductivity       &      W/m/K    & 0.07 $\pm$ 0.02                   & Laser Flash               & \cite{Kempel:1}   \\ \hline
Char Emissivity         &               &       0.85                        & Literature                & \cite{Braeuer:1}  \\ \hline
Char Absorbtion Coef.   &      m$^{-1}$ &       10000                       & Estimated                 & \cite{Kempel:1}   \\ \hline
Pre-Exp. Factor         &      s$^{-1}$ & $(2.49 \pm 0.63) \times 10^{14}$  & TGA                       & \cite{Kempel:1}   \\ \hline
Activation Energy       &    kJ/kmol    & $(2.12 \pm 0.53) \times 10^{5}$   & TGA                       & \cite{Kempel:1}   \\ \hline
Heat of Reaction        &      kJ/kg    &        355                        & DSC, Literature           & \cite{Kempel:1,Lyon:Ency2005} \\ \hline
Heat of Combustion      &      kJ/kg    & 19500                             & Cone Calorimeter          & \cite{Kempel:1}   \\ \hline
Char Yield              &               & 0.32 $\pm$ 0.05                   & Gasification Device       & \cite{Kempel:1}   \\ \hline
Combustion Efficiency   &               &          1                        & Assumed                   & \cite{Kempel:1}   \\ \hline
\end{tabular}
\end{center}
\label{Properties_PBT-GF}
\end{table}

\begin{figure}[h!]
\begin{center}
\includegraphics[height=2.2in]{FIGURES/FAA_Polymers/FAA_Polymers_PBTGF}
\caption[Heat release rate of poly(butylene terephtalate) with glass fibers (PBT-GF).]
{Comparison of predicted and measured heat release rates for poly(butylene terephtalate) with glass fibers (PBT-GF).}
\label{HRR_PBTGF}
\end{center}
\end{figure}

\subsection{Method Glossary}
\begin{description}
\item[Assumption]  Characteristics were assumed from known properties in similar materials.
\item[Cone Calorimeter] (ASTM E 1354) The Cone Calorimeter exposes a small sample to a constant external radiant heat flux simulating exposure of the sample to a large scale fire. Mass loss data along with Oxygen Consumption Calorimetry allows the determination of the Heat of Combustion of the sample. 
\item[Direct]  Direct measurement of densities is performed by measuring the dimensions and mass of the sample.
\item[DSC] (ASTM E2070) A Differential Scanning Calorimeter precisely raises the temperature of a small sample of material at a constant rate. This coupled with knowledge of heat absorbed by the sample allows for the calculation of the specific heat function of a material.
\item[Estimated] Characteristics were approximated based on know properties in similar materials.
\item[Fit to Cone Data] Property was established by fitting  equations to experimental values obtained in the Cone Calorimeter
\item[FTIR] Fourier Transform Infrared Spectroscopy uses a spectrometer to simultaneously characterize the absorption of all frequencies of infrared light. In testing a sample is exposed to infrared light and a detector records light that has passed through the sample. A Fourier transform of detector measurement is then translated into absorption information.
\item[Gasification Device] Similar to the Cone Calorimeter except the experiments are conducted in a sealed cylindrical chamber. The conical heating element is held at a constant temperature and heat flux is controlled by varying sample position relative to the heater. The walls are painted black and water cooled to room temperature to minimize background radiation. Purge gases are introduced through the bottom of the chamber.
\item[Integrating Sphere] (ASTM E1175 ) An Integrating Sphere, or an Ulbricht Sphere, is a hollow cavity whose interior has a high diffuse reflectivity. A sample placed inside the sphere is exposed to incident radiation and reflectivity measured. Emissivity can be determined from this information. The standard above is for measurement of Solar reflectivity, and was not necessarily precisely followed.
\item[Laser Flash] (ASTM E1461) In the Laser Flash Method one surface of a sample is rapidly heated using a single pulse from a laser. Heat sensors on the opposite side of the sample record the arrival of the resulting temperature disturbance. From this thermal diffusivity/thermal conductivity can be calculated.
\item[Literature] Results were found within previously published literature
\item[MCC] (ASTM D7309 ) The Microscale Combustion Calorimeter (MCC)  rapidly pyrolizes a milligram size sample in an inert atmosphere. The pyrolyzate is then exposed to an abundance of oxygen.  Heat of combustion is obtained from oxygen consumption.
\item[Pulsed Current] A sample is positioned between two electrodes in a sealed chamber with an inert atmosphere. The sample is heated through pulses of current. A radiation thermometer at the bottom of the chamber records temperature and spectral emissivity from the sample.
\item[TGA] (ASTM E1131) In Thermal Gravimetric Analysis (TGA)  a small sample is heated at uniform rate, generally in an Nitrogen (N2) atmosphere. The percentage weight loss of the sample is recorded relative to the sample’s temperature. Rate constants can then be fitted to the data.
\item[Thermoflixer] The Thermoflixer apparatus is based on the Transient Line Source method. A thin probe containing a heater wire and thermocouple are inserted into the sample Thermal conductivity is determined from a small change in probe temperature occurring briefly after applying power to the probe heater.
\item[Transient Line Source] (ASTM D5930) The Transient Line Source method records temperature of a single point at a fixed distance in a sample over time using a probe. Given knowledge of the heat exposure of the sample the thermal conductivity can be found from the slope of the recorded data.
\end{description}

