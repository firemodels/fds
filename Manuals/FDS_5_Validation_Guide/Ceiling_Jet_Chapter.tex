\chapter{Ceiling Jets and Device Activation}

FDS is a computational fluid dynamics (CFD) model and has no explicit ceiling jet model.
Rather, temperatures throughout the fire compartment are computed directly from the governing conservation equations.
Nevertheless, temperature measurements near the ceiling can be used to evaluate the model's ability to predict the flow of
hot gases across a relatively flat ceiling. Measurements for this category are available from the NIST/NRC and the FM/SNL series.

\section{WTC Test Series}

Aspirated thermocouples were positioned 3~m to the west (TTRW1) and 2~m to the east (TTRE1) of the fire pan, 18~cm below the ceiling.

\begin{figure}[p]
\begin{tabular*}{\textwidth}{l@{\extracolsep{\fill}}r}
\includegraphics[height=2.2in]{FIGURES/WTC/WTC_01_v5_Ceiling_Jet} &
\includegraphics[height=2.2in]{FIGURES/WTC/WTC_02_v5_Ceiling_Jet} \\
\includegraphics[height=2.2in]{FIGURES/WTC/WTC_03_v5_Ceiling_Jet} &
\includegraphics[height=2.2in]{FIGURES/WTC/WTC_04_v5_Ceiling_Jet} \\
\includegraphics[height=2.2in]{FIGURES/WTC/WTC_05_v5_Ceiling_Jet} &
\includegraphics[height=2.2in]{FIGURES/WTC/WTC_06_v5_Ceiling_Jet}
\end{tabular*}
\label{WTC_Jet}
\end{figure}

\section{NIST/NRC Test Series}

The thermocouple nearest the ceiling in Tree 7, located towards the back of the compartment,
has been chosen as a surrogate for the ceiling jet temperature.
Curiously, the difference between measured and predicted temperatures is noticeably greater for the open door tests.
Certainly, the open door changes the flow pattern of the exhaust gases.
However, the predicted HGL heights for the open door tests, shown in the previous section,
do not show a noticeable difference from their closed door counterparts.
The predicted HGL temperatures are only slightly less than those measured in the open door tests,
due in large part to the contribution of Tree 7 in the layer reduction calculation.


\begin{figure}[p]
\begin{tabular*}{\textwidth}{l@{\extracolsep{\fill}}r}
\includegraphics[height=2.2in]{FIGURES/NIST_NRC/NIST_NRC_01_v5_Ceiling_Jet} &
\includegraphics[height=2.2in]{FIGURES/NIST_NRC/NIST_NRC_07_v5_Ceiling_Jet} \\
\includegraphics[height=2.2in]{FIGURES/NIST_NRC/NIST_NRC_02_v5_Ceiling_Jet} &
\includegraphics[height=2.2in]{FIGURES/NIST_NRC/NIST_NRC_08_v5_Ceiling_Jet} \\
\includegraphics[height=2.2in]{FIGURES/NIST_NRC/NIST_NRC_04_v5_Ceiling_Jet} &
\includegraphics[height=2.2in]{FIGURES/NIST_NRC/NIST_NRC_10_v5_Ceiling_Jet} \\
\includegraphics[height=2.2in]{FIGURES/NIST_NRC/NIST_NRC_13_v5_Ceiling_Jet} &
\includegraphics[height=2.2in]{FIGURES/NIST_NRC/NIST_NRC_16_v5_Ceiling_Jet}
\end{tabular*}
\label{NIST_NRC_Jet_Closed}
\end{figure}

\begin{figure}[p]
\begin{tabular*}{\textwidth}{l@{\extracolsep{\fill}}r}
\includegraphics[height=2.2in]{FIGURES/NIST_NRC/NIST_NRC_17_v5_Ceiling_Jet} &
 \\
\includegraphics[height=2.2in]{FIGURES/NIST_NRC/NIST_NRC_03_v5_Ceiling_Jet} &
\includegraphics[height=2.2in]{FIGURES/NIST_NRC/NIST_NRC_09_v5_Ceiling_Jet} \\
\includegraphics[height=2.2in]{FIGURES/NIST_NRC/NIST_NRC_05_v5_Ceiling_Jet} &
\includegraphics[height=2.2in]{FIGURES/NIST_NRC/NIST_NRC_14_v5_Ceiling_Jet} \\
\includegraphics[height=2.2in]{FIGURES/NIST_NRC/NIST_NRC_15_v5_Ceiling_Jet} &
\includegraphics[height=2.2in]{FIGURES/NIST_NRC/NIST_NRC_18_v5_Ceiling_Jet}
\end{tabular*}
\label{NIST_NRC_Jet_Open}
\end{figure}


\section{FM/SNL Test Series}

The near-ceiling thermocouples in Sectors 1 and 3 have been chosen as surrogates for the ceiling jet temperature.
The results are shown below.  The only noticeable discrepancy is in Test 21, and it is the same pattern that
was observed in the HGL temperature comparison for this test.

\begin{figure}[p]
\begin{tabular*}{\textwidth}{l@{\extracolsep{\fill}}r}
\includegraphics[height=2.2in]{FIGURES/FM_SNL/FM_SNL_04_v5_Ceiling_Jet} &
\includegraphics[height=2.2in]{FIGURES/FM_SNL/FM_SNL_05_v5_Ceiling_Jet} \\
\includegraphics[height=2.2in]{FIGURES/FM_SNL/FM_SNL_21_v5_Ceiling_Jet} &
\end{tabular*}
\label{FM_SNL_Ceiling_Jet}
\end{figure}




\begin{figure}[p]
\begin{center}
\begin{tabular}{c}
\includegraphics[width=4.0in]{FIGURES/ScatterPlots/Ceiling_Jet_Temperature} \\
\vspace{0.25in}
\end{tabular}
\caption{Summary of Ceiling Jet Temperature Results.}
\end{center}
\end{figure}



\clearpage

\section{UL/NFPRF Sprinkler, Vent, and Draft Curtain Experiments}
\label{UL_NFPRF:Results}

The ceiling jet is an important fire phenomenon because of the presence of automatic fire protection devices at the ceiling, like
sprinklers and smoke/heat vents. The results of the UL/NFPRF experiments provide useful data to assess the accuracy of FDS in predicting
the velocity and temperature near the ceiling, and consequently the resulting activation of sprinklers.
The UL/NFPRF test results (Series I) are summarized in Table~\ref{ULmatrix}, along with the predictions of FDS.

\begin{table}[h!]
\begin{center}
\begin{tabular}{|c||c|c|c|c|c|c|c|c|}
\hline
\multicolumn{9}{|c|}{\bf Heptane Spray Burner Test Series I}  \\ \hline \hline
Test & Burner & Vent                    & \multicolumn{2}{|c|}{First Act. (s) } & \multicolumn{2}{|c|}{Total Acts.}  & Draft    & Heat Release Rate \\ \cline{4-7}
No.  & Pos.   & Operation               & Exp. & FDS                            & Exp.  & FDS                        & Curtains & MW @ s \\
\hline \hline
I-1   & B  & Closed                     & 65   & 53                             & 11   & 11     & Yes  & 4.4 @ 50  \\ \hline
I-2   & B  & Manual (0:40)              & 66   & 53                             & 12   & 9      & Yes  & 4.4 @ 50  \\ \hline
I-3   & B  & Manual (1:30)              & 64   & 53                             & 12   & 9      & Yes  & 4.4 @ 50  \\ \hline
I-4   & C  & Closed                     & 60   & 52                             & 10   & 11     & Yes  & 4.4 @ 50  \\ \hline
I-5   & C  & Manual (0:40)              & 72   & 52                             & 9    & 8      & Yes  & 4.4 @ 50  \\ \hline
I-6   & C  & Manual (1:30)              & 62   & 52                             & 8    & 8      & Yes  & 4.4 @ 50  \\ \hline
I-7   & C  & 74$^\circ$C link (DNO)     & 70   & 52                             & 10   & 11     & Yes  & 4.4 @ 50  \\ \hline
I-8   & B  & 74$^\circ$C link (9:26)    & 60   & 53                             & 11   & 11     & Yes  & 4.4 @ 50  \\ \hline
I-9   & D  & 74$^\circ$C link (DNO)     & 70   & 55                             & 12   & 16     & Yes  & 4.4 @ 50  \\ \hline
I-10  & D  & Manual (0:40)              & 72   & 55                             & 13   & 14     & Yes  & 4.4 @ 50  \\ \hline
I-11  & D  & 74$^\circ$C link (4:48)    & N/A  & N/A                            & N/A  & N/A    & Yes  & 4.4 @ 50  \\ \hline
I-12  & A  & Closed                     & 68   & 58                             & 14   & 13     & Yes  & 4.4 @ 50  \\ \hline
I-13  & A  & 74$^\circ$C link (1:04)    & 69   & 59                             & 5    & 11     & Yes  & 6.0 @ 60  \\ \hline
I-14  & A  & Manual (0:40)              & 74   & 127                            & 7    & 7      & Yes  & 5.8 @ 60  \\ \hline
I-15  & A  & Manual (1:30)              & 64   & 59                             & 5    & 13     & Yes  & 5.8 @ 60  \\ \hline
I-16  & A  & 74$^\circ$C link (1:46)    & 106  & 99                             & 4    & 6      & Yes  & 5.0 @ 110 \\ \hline \hline
I-17  & B  & 100$^\circ$C link (DNO)    & 58   & 54                             & 4    & 6      & No   & 4.6 @ 50 \\ \hline
I-18  & C  & 100$^\circ$C link (DNO)    & 58   & 58                             & 4    & 4      & No   & 3.7 @ 50 \\ \hline
I-19  & A  & 100$^\circ$C link (10:00)  & 56   & 60                             & 10   & 4      & No   & 4.6 @ 50 \\ \hline
I-20  & A  & 74$^\circ$C link (1:20)    & 54   & 64                             & 4    & 4      & No   & 4.2 @ 50 \\ \hline
I-21  & C  & 74$^\circ$C link (7:00)    & 58   & 52                             & 10   & 4      & No   & 4.6 @ 50 \\ \hline
I-22  & D  & 100$^\circ$C link (DNO)    & 60   & 54                             & 6    & 5      & No   & 4.6 @ 50 \\ \hline
\end{tabular}
\end{center}
\caption[Results of the UL/NFPRF Experiments.]
{\bf Results of the UL/NFPRF Experiments. Note that DNO means
``Did Not Open''. Also note, the fires grew at a rate proportional
to the square of the time until a certain flow rate of fuel was achieved
at which time the flow rate was held steady. Thus, the ``Heat Release Rate''
was the size of the fire at the time when the fuel supply was leveled off.}
\label{ULmatrix}
\end{table}

Figure~\ref{UL_NFPRF} displays graphically the difference between predicted and measured sprinkler activation times as a function of 
burner position. Note that there are no experimental uncertainty bounds on the plot because it is difficult to estimate the 
combined uncertainty related to the various parameters that are input into the model. For example, changing the median volumetric droplet 
size from 1000~$\mu$m to 750~$\mu$m led to a reduction of approximately 50~\% in the number of predicted sprinkler activations due to the
increased cooling of the smaller droplets. Consequently, three replicate experiments are compared to show how much variation one can expect in
sprinkler activation times in repeat experiments.

\begin{figure}[p]
\label{UL_NFPRF}
\begin{center}
\begin{tabular}{c}
\includegraphics[width=4in]{FIGURES/ScatterPlots/UL_NFPRF_Activation_Times} \\
\includegraphics[width=4in]{FIGURES/ScatterPlots/UL_NFPRF_Repeatability}
\vspace{0.25in}
\end{tabular}
\caption{Above: Comparison of predicted and measured sprinkler activation times for the UL/NFPRF Series. Below: Comparison of activation
times for three pairs of replicate experiments.}
\end{center}
\end{figure}


