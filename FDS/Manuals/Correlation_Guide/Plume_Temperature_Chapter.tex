% !TEX root = Correlation_Guide.tex

\chapter{Plume Temperature}
\label{Plume_Chapter}

The fire plume transports hot gases into the HGL. Its temperature is greater than the ceiling jet and HGL temperature. It is particularly important scenarios that involve targets directly above a potential fire.

\section{Heskestad Method}
\label{sec:Heskestad}

\subsection*{Description}

For a fire plume, the correlation by Heskestad~\cite{SFPE:Heskestad} predicts that the increase in centerline temperature, $\Delta T_0$~(\si{\celsius}), is given by
\be
\Delta T_0 = \frac{9.1 \left( \frac{T_\infty}{g c\sb{p}^2 \rho_{\infty}^2} \right)^{1/3} \dot Q\sb{c}^{2/3}}{(z-z_0)^{5/3}}
\label{eq:Heskestad}
\ee
where $T_\infty$ is the ambient air temperature~(\si{\celsius}), $g$ is the acceleration of gravity~(m/s$^2$), $c\sb{p}$ is the specific heat of air~(\si{kJ/(kg.K)}), $\rho_{\infty}$ is the ambient air density~(\si{kg/m^3}), and $z$ is the elevation above the fire source~(\si{m}). The convective HRR, $\dot Q\sb{c}$ (\si{kW}), is given by
\be
\dot Q\sb{c} = \dot Q (1 - \chi\sb{r})
\label{eq:Heskestad_Qc}
\ee
where $\dot Q$ is the total HRR~(\si{kW}), and $\chi\sb{r}$ is the radiative fraction. Note that the total HRR $\dot Q$ is the actual HRR, not the idealized HRR. The hypothetical virtual origin of the fire, $z_0$ (\si{m}), is given by
\be
z_0 = -1.02 D + 0.083 \dot Q^{2/5}
\label{eq:Heskestad_z0}
\ee
where $D$ is the diameter of the fire source~(\si{m}) and is given by
\be
D = \sqrt{\frac{4 A}{\pi}}
\label{eq:Heskestad_D}
\ee
where $A$ is the area of the fire source~(\si{m^2}).
Note that this plume temperature correlation is only valid above the mean flame height.


\clearpage


\subsection*{Verification}

This example case is based on Test 1 from the VTT~\cite{Hostikka:VTT2104} series. This test involved a large test hall with closed doors, a heptane pool fire, and no ventilation.

\begin{table}[!ht]
\caption[Verification case, plume temperature]
{Verification case, plume temperature.}
\begin{center}
\begin{tabular}{|l|c|}
\hline
\multicolumn{2}{|c|}{}                        \\
\multicolumn{2}{|c|}{User-Specified Input}    \\
\multicolumn{2}{|c|}{}                        \\ \hline
                            &                 \\
\rb{Parameter}              &  \rb{Value}     \\ \hline \hline
$\dot Q$ (m)                &  1245           \\ \hline
$c\sb{p}$ (\si{kJ/(kg.K)})  &  1.0            \\ \hline
$z$ (m)                     &  6              \\ \hline
$A$ (m$^2$)                 &  1.075          \\ \hline
$\chi\sb{r}$                &  0.40           \\ \hline
$T_\infty$ (\si{\celsius})  &  22             \\ \hline
\multicolumn{2}{c}{}                          \\ \hline
\multicolumn{2}{|c|}{}                        \\
\multicolumn{2}{|c|}{Calculated Output}       \\
\multicolumn{2}{|c|}{}                        \\ \hline
\multicolumn{2}{|c|}{}                        \\
\multicolumn{2}{|c|}{\rb{Plume Temperature}}  \\
\multicolumn{2}{|c|}{\rb{(\si{\celsius})}}    \\ \hline \hline
\multicolumn{2}{|c|}{133.78}                  \\ \hline
\end{tabular}
\end{center}
\end{table}


\clearpage


\subsection*{Validation}

A summary of the comparisons between peak predicted and measured plume temperatures is shown in Fig.~\ref{Plume Temperature (Heskestad)}.

\begin{figure}[!ht]
\begin{center}
\begin{tabular}{l}
\includegraphics[width=4.0in]{SCRIPT_FIGURES/Scatterplots/Plume_Temperature_Heskestad}
\end{tabular}
\end{center}
\caption[Summary of plume temperature predictions (Heskestad)]
{Summary of plume temperature predictions using the Heskestad method.}
\label{Plume Temperature (Heskestad)}
\end{figure}


\clearpage


\section{McCaffrey Method}
\label{sec:McCaffrey}

\subsection*{Description}

For a fire plume, the correlation by McCaffrey~\cite{McCaffrey:NBSIR_79-1910} predicts that the increase in centerline temperature, $\Delta T_0$~(\si{\celsius}), is given by
\be
\Delta T_0 = \left[ \left( \frac{\kappa}{0.9 \sqrt{2 g}} \right)^2 \left( \frac{z}{\dot Q^{2/5}} \right)^{2 \eta - 1} \right] T_\infty
\label{eq:McCaffrey}
\ee
where $g$ is the acceleration of gravity~(\si{m/s^2}), $z$ is the elevation above the fire source~(\si{m}), $\dot Q$ is the HRR~(\si{kW}), and $T_\infty$ is the ambient air temperature~(\si{\celsius}). The constants $\eta$ and $\kappa$ are a function of the height $z$ within the plume and are listed in Table~\ref{tbl:McCaffrey_constants}.

\vspace{\baselineskip}
\begin{table}[!ht]
\begin{center}
\caption[Constants used in McCaffrey plume temperature correlation]
{Constants used in McCaffrey plume temperature correlation.}
\label{tbl:McCaffrey_constants}
\begin{tabular}{|c|c|c|c|}
\hline
Region        &  $z/\dot Q^{2/5}$  &   $\eta$  &  $\kappa$  \\ \hline \hline
Continuous    &  < 0.08            &   1/2     &  6.8       \\ \hline
Intermittent  &  < 0.08 -- 0.2     &   0       &  1.9       \\ \hline
Plume         &  > 0.2             &   -1/3    &  1.1       \\
\hline
\end{tabular}
\end{center}
\end{table}


\clearpage


\subsection*{Verification}

This example case is based on Test 1 from the VTT~\cite{Hostikka:VTT2104} series. This test involved a large test hall with closed doors, a heptane pool fire, and no ventilation.

\begin{table}[!ht]
\caption[Verification case, plume temperature]
{Verification case, plume temperature.}
\begin{center}
\begin{tabular}{|l|c|}
\hline
\multicolumn{2}{|c|}{}                        \\
\multicolumn{2}{|c|}{User-Specified Input}    \\
\multicolumn{2}{|c|}{}                        \\ \hline
                            &                 \\
\rb{Parameter}              &  \rb{Value}     \\ \hline \hline
$\dot Q$ (m)                &  1245           \\ \hline
$z$ (m)                     &  6              \\ \hline
$T_\infty$ (\si{\celsius})  &  22             \\ \hline
\multicolumn{2}{c}{}                          \\ \hline
\multicolumn{2}{|c|}{}                        \\
\multicolumn{2}{|c|}{Calculated Output}       \\
\multicolumn{2}{|c|}{}                        \\ \hline
\multicolumn{2}{|c|}{}                        \\
\multicolumn{2}{|c|}{\rb{Plume Temperature}}  \\
\multicolumn{2}{|c|}{\rb{(\si{\celsius})}}    \\ \hline \hline
\multicolumn{2}{|c|}{153.21}                  \\ \hline
\end{tabular}
\end{center}
\end{table}


\clearpage


\subsection*{Validation}

A summary of the comparisons between peak predicted and measured plume temperatures is shown in Fig.~\ref{Plume Temperature (McCaffrey)}.

\begin{figure}[!ht]
\begin{center}
\begin{tabular}{l}
\includegraphics[width=4.0in]{SCRIPT_FIGURES/Scatterplots/Plume_Temperature_McCaffrey}
\end{tabular}
\end{center}
\caption[Summary of plume temperature predictions (McCaffrey)]
{Summary of plume temperature predictions using the McCaffrey method.}
\label{Plume Temperature (McCaffrey)}
\end{figure}
