\chapter{Combustion (Chemically Reacting Flows)}

\label{combustionsection}
In FDS, the combustion routine is responsible for the chemical mass production rate of species $\alpha$ per unit volume, $\dot{m}^{\prime\prime\prime}_{\alpha}$ Eq.~(\ref{eq:m_tprime_alpha}), and appears in the species transport equation, Eq.~(\ref{species}). The heat release for per unit volume, $\dot{q}^{\prime\prime\prime}$, is also obtained from the combustion routine, found by multiplying the sum of the lumped species mass production rates by their respective heat of formation, Eq.~(\ref{eq:q_tprime}).

\begin{equation}\label{eq:m_tprime_alpha}
\dot{m}^{\prime\prime\prime}_{\alpha}=\rho \,\frac{\mbox{d}Y_{\alpha}}{\mbox{d}t}
\end{equation}

\begin{equation}\label{eq:q_tprime}
\dot{q}^{\prime\prime\prime} \equiv -\displaystyle \sum_{\alpha} \dot{m}^{\prime\prime\prime} \Delta H_{f,\alpha}
\end{equation}


\section{Lumped Species Approach}
In the typical FDS problem we consider the simple reaction:
\begin{equation}\label{eq:simple}
\mathrm{Fuel + Air \rightarrow Products}
\end{equation}
Here fuel, air, and products are lumped species. The traditional approach for tracking species is to use a mixture fraction and progress variable \cite{fox2003}. A limitation in the progress variable approach is that the variable is not physically bounded and initial and boundary conditions can be ill-defined. The lumped species approach, which evolved out a need to track additional species, is similar to the progress variable approach for reactions in the form of Eq:~(\ref{eq:simple}). An advantage of the lumped species approach is that the variables are mass fractions and are therefore physically bounded between 0 and 1. 

In a simple hydrocarbon reaction, the typical reactants are fuel, oxygen, and nitrogen with products being carbon dioxide, water vapor, and nitrogen. In FDS, each of these individual $\alpha$ species are known as primitive species and their mass fractions are represented by $Y_{\alpha}$.
The key assumption made in lumping the primitive species is that the new species groups transport (implying equal diffusivities) and react together. We can expand Eq.~(\ref{eq:simple}), to show the components of the lumped species for a simple, one-step methane reaction. Here air is assumed to be 21\% O$_2$ and 79\% N$_2$ by volume.
\begin{equation}\label{eq:lumped_expand}
\mathrm{2\underbrace{(\mbox{O}_2+(.79/.21)\mbox{N}_2)}_\text{Background(Oxidant),~Z$_0$}+\underbrace{\mbox{CH}_4}_\text{Fuel,~Z$_1$} \rightarrow 1\underbrace{(\mbox{CO}_2+2\mbox{H}_2\mbox{O}+2(.79/.21)\mbox{N}_2)}_\text{Products,~Z$_2$}}
\end{equation}
As described by Eq.~(\ref{eq:lumped_expand}), two moles of the lumped species `Background' react with one mole of lumped species `Fuel', to produce one mole of lumped species `Products'. To convert from lumped species to primitive species, we need to find the mass fractions of the primitive species within a particular lumped species if the lumped species mass fraction were one. Consider, Z$_0$, the background species:

\begin{alignat}{4}\label{eq:backgroud}
Y_\mathrm{O_{2}} &= \frac{W_\mathrm{O_{2}}}{W_\mathrm{O_{2}}+(.79/.21)W_\mathrm{N_{2}}} &=& \frac{W_\mathrm{O_{2}}}{W_{\mathrm{Z_0}}} &= 0.2330 \\
\nonumber Y_\mathrm{N_{2}} &= \frac{(.79/.21)W_\mathrm{N_{2}}}{W_\mathrm{O_{2}}+(.79/.21)W_\mathrm{N_{2}}} &=& \frac{(.79/.21)W_\mathrm{N_{2}}}{W_{\mathrm{Z_0}}} &= 0.7670
\end{alignat}
Lumped species are defined by the vector $\textbf{Z}$ which is indexed from $0:N_{z}$, since the background species will not be transported. This means that $N_{z}$ transport equations are solved. Each lumped species is a combination of primitive species which we can define by the vector $\textbf{Y}$ which is indexed from $1:N_{y}$. Therefore, the transformation matrix, $A$ is $N_{y} \times (N_{z}+1)$ (rows $\times$ columns) since it includes the background species. Transformation from lumped species to primitive species is defined by: 
\begin{equation}\label{eq:transform}
\textbf{Y}=A\textbf{Z} 
\end{equation}
The elements are $A$ are the mass fractions for each primitive species in a given lumped species:
\begin{equation}\label{eq:A_def}
A_{\alpha i} = \frac{\upsilon_{\alpha i}W_{\alpha}}{\displaystyle \sum_{\alpha}\upsilon_{\alpha i}W_{\alpha}}
\end{equation}
where $\upsilon_{\alpha i}$ are the volume ratios of primitive species $\alpha$ in lumped species {i}.

If we look at the primitive species in Eq.~(\ref{eq:lumped_expand}) and assume that $\mathbf{Z} = [0.3\, \, 0.2\, \, 0.5]^\mathrm{T}$, we can transform from lumped species to primitive species:

\begin{displaymath}
\mathbf{Y}=\left[\begin{array}{c}
       Y_{O_2} \\
       Y_{N_2} \\
       Y_{CH_4} \\
       Y_{CO_2} \\
       Y_{H_2O} \\
     \end{array}\right]
     =\left[\begin{array}{ccc}
     0.2330 & 0 & 0 \\
     0.7670 & 0 & 0.7248 \\
     0 & 1 & 0 \\
     0 & 0 & 0.1514 \\
     0 & 0 & 0.1238 \\
     \end{array}\right]
     \left[\begin{array}{c}
     0.3 \\
     0.2 \\
     0.5 \\
     \end{array}\right]
     =\left[\begin{array}{c}
     0.0699\\
     0.5925\\
     0.2000\\
     0.0757\\
     0.0619\\
     \end{array}\right]
\end{displaymath}
To convert back to lumped species from primitive species we can perform linear algebra operations to get:
\begin{equation}\label{eq:transform_back}
\textbf{Z}=B\textbf{Y} \, \, \, \text{where} \, \, \, \, B=(A^TA)^{-1}A^T
\end{equation}
Now we revisit Eq.~(\ref{eq:lumped_expand}) but decompose the reaction into all primitive species, we get:
\begin{equation}\label{eq:prim}
\underbrace{\mbox{CH}_4}_\text{Fuel}+\underbrace{2\mbox{O}_2}_\text{Oxidizer}+\underbrace{7.52\mbox{N}_2}_\text	{Background}\rightarrow \underbrace{\mbox{CO}_2}_\text{P1}+\underbrace{2\mbox{H}_2\mbox{O}}_\text{P2}+\underbrace{7.52\mbox{N}_2}_\text{Background}
\end{equation}
In this formulation, we would have to solve transport equations for four species compared to two when the species were lumped. Note that in cases with all primitive species, $A$ becomes an identity matrix. With complex reactions, tracking all of the primitive species could lead to solving 10-20 equations, each with an additional 5~\% computational cost.

\subsection{Specified CO and Soot Yield}
The default reaction system in FDS is that fuel plus oxidizer goes to products, where the products are carbon dioxide, water vapor, nitrogen, carbon monoxide, and soot. The carbon monoxide and soot yields are by default zero and can be specified by the user on the {\ct REAC} line. For a typical hydrocarbon fuel, the reaction in FDS follows:

\begin{eqnarray}
\nu_{0}\underbrace{(\upsilon_{\mathrm{O}_{2},0}\mathrm{O}_2+\upsilon_{\mathrm{H}_{2}\mathrm{O},0}\mathrm{H}_2\mathrm{O}+\upsilon_{\mathrm{CO}_{2},0}\mathrm{CO}_2+\upsilon_{\mathrm{N}_{2},0}\mathrm{N}_2)}_\text{Background(Oxidant),~Z$_0$}+\nu_{1}\underbrace{\mbox{C}_m\mbox{H}_n\mbox{O}_a\mbox{N}_b}_\text{Fuel,~Z$_1$} \rightarrow \\
\nonumber \nu_{2}\underbrace{(\upsilon_{\mathrm{CO}_{2},2}\mathrm{CO}_2+\upsilon_{\mathrm{H}_{2}\mathrm{O},0}\mathrm{H}_2\mathrm{O}+\upsilon_{\mathrm{N}_{2},0}\mathrm{N}_2+\upsilon_{\mathrm{CO},2}\mathrm{CO}+\upsilon_{\mathrm{S},2}\mathrm{S})}_\text{Products,~Z$_2$}
\end{eqnarray}
Here the volume fraction of primitive species $\alpha$ in lumped species $i$ is denoted by $\upsilon_{\alpha i}$ and the stoichiometric coefficients for the lumped $i$ are denoted by $\nu_{i}$. CO and soot yields ($y_{CO}$ and $y_{S}$ respectively) specified by the user determine the stoichiometric coefficients on the product side of the reaction by:
\begin{equation}\label{eq:co_yield}
\nu_{2}\upsilon_{\mathrm{CO},2}=-\nu_{1}\frac{W_1}{W_{\mathrm{CO}}}y_{\mathrm{CO}}
\end{equation}
\begin{equation}\label{eq:soot_yield}
\nu_{2}\upsilon_{\mathrm{S},2}=-\nu_{1}\frac{W_1}{W_{\mathrm{S}}}y_{\mathrm{S}}
\end{equation}
Here, $W_{1}$ is the molecular weight of lumped species 1. The remaining coefficients come from an atom balance.


\section{Eddy Dissipation Concept}
FDS can model multiple, simultaneous, reversible chemical reactions.  In FDS only the species that have mixed have the potential to react. Once mixed, FDS uses one of two approaches to compute the reaction rate for a chemical reaction.  The first approach is an infinite fast chemical reaction rate. That is the reactants (e.g. fuel and air) react infinitely fast at the rate that the underlying gas transport mixes them together.  For the second model, individual gas species react according to specified Arrhenius reaction parameters. The Arrhenius rate can still be limited by the amount of mixing within a computational cell. This latter model is most often used in a direct numerical simulation (DNS) where the diffusion of fuel and oxygen can be modeled directly. However, most often for large eddy simulations (LES), where the grid is not fine enough to resolve the diffusion of fuel and oxygen,
the mixing-controlled combustion model is assumed. A single FDS simulation can contain a mix of reaction rate types.

In the eddy dissipation concept model (EDC) each computational cell can be thought of as a batch reactor. The species contained within this reactor exist in one of two states: unmixed or mixed. The proportion of the cell in the mixed state is governed by $\zeta$, the unmixed fraction. A cell can range from completely unmixed ($\zeta=1$) to completely mixed ($\zeta=0$). In a combustion sense, $\zeta=1$  represents a diffusion flame and $\zeta=0$ represents a premixed flame. $\zeta(t;\tau_{mix})$, a function of time (t) and parameterized by the mixing time ($\tau_{mix}$), is governed by the following ordinary differential equation:

\begin{equation}\label{eq:zeta}
\frac{\mbox{d}\zeta}{\mbox{d}t}=-\frac{\zeta}{\tau_{mix}}
\end{equation}     
Only those species in the mixed state can react, therefore the proportion of the cell (reactor) composition in the mixed state must be known as function of time. 

The mixing time, $\tau_{mix}$, is found by examining characteristics of the local state of the flow field by \cite{McDermott:2011}:
\begin{equation}\label{eq:tau_mix}
\tau_{mix}=\mbox{max}(\tau_{chem},\mbox{min}(\tau_{d},\tau_{u},\tau_{g},\tau_{flame}))
\end{equation}
where $\tau_{chem}$ is the chemical time scale, $\tau_{d}$ is associated with molecular diffusion, $\tau_{g}$ is the gravitational time school, $\tau_{u}$ is the time scale associated with turbulent advection, and $\tau_{flame}$ is upper limit of the time scale since all fuel must be consumed within a computational cell. The mathematical description of the sub models are:
\begin{equation}\label{eq:tau_d}
\tau_{d}=\frac{\mbox{Sc}_{t}\bar{\rho}\Delta^2}{\mu+\mu_{t}}
\end{equation}
\begin{equation}\label{eq:tau_u}
\tau_{u}=\frac{\Delta}{\sqrt{2k_{sgs}}}
\end{equation}
\begin{equation}\label{eq:tau_g}
\tau_{g}=\sqrt{2\Delta/g}
\end{equation}
In FDS, $\Delta$ is the filter width and is equivalent to the local cell size $\delta x$. Fig.~\ref{fig:reac_time_scale} shows the impact of the components of Eq.~(\ref{eq:tau_mix}) on the value of $\tau_{mix}$. $\tau_{mix}$ is represented by the bold solid line.

\begin{figure}[h!]
\begin{center}
\includegraphics[height=2.2in]{FIGURES/reaction_time_scale}
\caption{\label{fig:reac_time_scale} Reaction time scale model.}
\end{center}
\end{figure}

The following equations define the cell composition:
\begin{eqnarray}\label{eq:mixunmix}
\rho V &= U(t) + M(t) \\
U(t) &= \zeta(t)\,\rho V \\
M(t) &= (1-\zeta(t))\,\rho V
\end{eqnarray} 
where $\rho V$ is the total mass in the reactor, $U(t)$ is unmixed portion of the mass, and $M(t)$ is the mixed portion of the mass. $\rho V$ is not a function of time as the total mass in a cell remains constant during a single FDS time interval. From the portion of mass that is mixed, $M(t)$, the local mixed composition ($\phi_{\alpha}(t)$) is found by:

\begin{equation}\label{eq:phi}
\phi_{\alpha}(t)=\frac{m_{\alpha}(t)}{M(t)}
\end{equation}
where $m_\alpha(t)$ is the mixed mass of each of the $\alpha$ species in the simulation. The mixed mass of each species as a function of time is governed by:

\begin{equation}\label{eq:mixmass}
\frac{\mbox{d}m_{\alpha}}{\mbox{d}t}=\frac{M(t) \, \nu_{\alpha} \, W_{\alpha}}{W_F}\displaystyle \sum_{i}^{\text{N\,Reactions}}r_{i}+Y_{\alpha,0}\frac{\mbox{d}M}{\mbox{d}t} 
\end{equation}
where $Y_{\alpha,0}$ is the mass fraction of each $\alpha$ species in the unmixed portion of the cell. The mass fractions of each species in the unmixed portion remain constant over an integration time step therefore the proportion of species which mix remains constant. $r_{i}$, the reaction rate, is discussed in detail in section \ref{Reaction_Rate_Model}. The complete composition of the computational cell is found by a combination of the unmixed and mixed portions:

\begin{equation}\label{eq:final_comp}
Y_{\alpha}(t)=\zeta(t)Y_{\alpha,0}+(1-\zeta(t))\phi_{\alpha}(t)
\end{equation}
Finally the chemical mass production rate, $\dot{m}^{\prime\prime\prime}_{\alpha}$, is found:
\begin{equation}\label{mass_prod_rate}
\dot{m}^{\prime\prime\prime}_{\alpha}=\rho \,\frac{\mbox{d}Y_{\alpha}}{\mbox{d}t}
\end{equation}

\section{Reaction Rate} 

\label{Reaction_Rate_Model}

The rate expression (on a fuel basis) for both infinitely fast chemistry and Arrhenius rate chemistry follow the same general form:
\begin{equation}\label{eq:Arrheniusrateeqn}
\frac{\mbox{d}[X_\F]}{\mbox{d}t} = -A \; \prod \left([X_{\alpha}]^{a_{\alpha}} \right) T^n \; e^{-E/RT} 
\end{equation}
Note that the rate expression can be dependent on temperature and that the rate can be dependent upon species that do not directly participate in the reaction. Since FDS solves for mass fractions of lumped species rather than mole fractions of individual species, the only "species" that can be consumed or created are the lumped species.  However, any of the primitive species components of those lumped species could participate as a collision species.  Therefore, rather than $\frac{\mbox{d}[X_\F]}{\mbox{d}t}$,  FDS must solve for $\frac{\mbox{d}\phi_\F}{\mbox{d}t}$.  Additionally since it is quicker to determine $\phi_i$ rather than $X_i$, the relation $[X_i]=\frac{\phi_i \rho}{W_i}$ is used.  Combined, these result in a modified rate equation:

\begin{equation}\label{eq:Arrheniusratemode}
\frac{\mbox{d}\phi_\F}{\mbox{d}t} = -A_{mod} \rho^{\sum (a_{\alpha}) -1} \; \prod \left(Y_{\alpha}^{a_{\alpha}} \right) T^n \; e^{-E/RT} \;\;\; A_{mod} = A \prod \left(W_{\alpha}^{-a_{\alpha}} \right)  
\end{equation}

\subsection{Infinitely Fast Reaction}
For fast chemistry reactions, $E$ and $n$ in Eq.~(\ref{eq:Arrheniusratemode}) are set to $0$ which removes the temperature dependence from the rate expression. The species exponents, $a_{\alpha}$, are also set to zero to remove concentration dependence. Finally, the pre-exponential, $A$, is set to $1 \times 10^{16}$. In this formulation, $\frac{\mbox{d}\phi_\F}{\mbox{d}t}$ becomes sufficiently large such that all of the fuel in a computational cell is consumed in a single time step, effectively making the rate infinite. However, the rate is limited by the minimum reactant (fuel or oxidizer) in the mixed composition of the cell which makes this formulation a mixing-controlled reaction.   

\subsection{Arrhenius Reaction}
There are conditions under which the local temperature and species concentrations support the computation of the reaction rate.  One set of conditions is where the reacting regions of the flow can be considered a well-stirred reactor.  The other is during a DNS calculation.  In the first case rather than a reacting surface, there is a reacting volume.  In the second case, the fine grid resolution enables the direct modeling of the diffusion of chemical species (fuel,
oxygen, and combustion products).  This allows the flame to be resolved in a DNS calculation.  Under conditions where the local gas
temperatures can be used to determine the reaction kinetics, Arrhenius rates can be used.


\subsection{Local Extinction}

\label{extinction}

The physical limitation of the mixing-controlled reaction model described in the previous section is that it assumes that fuel and oxygen burn instantaneously when mixed. For large-scale, well-ventilated
fires, this is a good assumption. However, if a fire is in an
under-ventilated compartment, or if a suppression agent like water
mist or CO$_2$ is introduced, or if the shear layer between fuel and oxidizing streams
has a sufficiently large local strain rate,
fuel and oxygen may mix but may not burn.
The physical mechanisms underlying these phenomena are complex, and
even simplified models still rely on an accurate prediction
of the flame temperature and local strain rate.
Subgrid-scale modeling of gas phase suppression and
extinction is still an area of active research in the combustion
community.

Simple empirical rules can be used to predict local
extinction based on the species and temperature present in the flame sheet.  The FDS extinction model consists of two parts. The first part, checks to see if the local temperature is above an auto-ignition temperature for the fuel.  If the temperature is too low, then combustion will not occur.  Note that this temperature is by default set to absolute zero so that typical users do not need to specify an ignition source.  The second part uses the concept of a limiting flame temperature.  If the local combustion cannot raise the local temperature above the limiting flame temperature, then combustion will not occur.  This is done in the following steps:

\begin{enumerate}
\item Search over all reactant species to find the limiting species and express that species in terms of the reaction's fuel:

\be \Delta Z_\F = \min_{i \; = \; reactant} \left(\frac{Z_i W_\F}{W_i \nu_i} \right) \ee
\item Remove that amount of fuel from the local gas, the resulting gas is the "air" for the reaction.
\item Search over all the non-fuel reactant species in the "air" to determine the limiting reactant.  This defines how much air is required to burn the fuel:

\be \Delta Z_{Air} = \min_{i \; = \; non-fuel \; reactant} \left(\frac{\Delta Z_\F W_i \nu_i}{Z_i W_\F} \right) \ee

\item Compute the enthalpy for the fuel and the air at both the current temperature and the limiting flame temperature.
\item Combustion is allowed if 

\be \Delta Z_{Air} h_{air}(T)+\Delta Z_F \left( h_\F(T)+\Delta H_\F \right) > \Delta Z_{Air} h_{air}(T_{LFT})+ \Delta Z_\F h_\F(T_{LFT}) \ee

\end{enumerate}

\section{Time Integration of Chemical Reactions}

\subsection{Single Step, Mixing-Controlled Reaction Rates}
For single-step chemical reactions with mixing-controlled reaction rates, an explicit Euler scheme is used to solve the system of ODEs that determine the chemical mass production rate, $\dot{m}^{\prime\prime\prime}_{\alpha}$.   


\subsection{Other Chemistry Models}

For anything other than single step chemistry with mixing-controlled reaction rates a more complex ODE solver is employed.  A fourth-order explicit integrator with error control is used.  Since chemical timescales are often fast with respect to the hydrodynamic timescales, the solver uses subtimesteps during time integration of the chemical reactions to maintain the specified error tolerance.  Time integration is halted once no reactants are present or if the local heat release rate exceeds maximum allowable values.

More information on the numerical methods of the integrators can be found in Appendix~\ref{chemistry_integration}.


\section{Heat Release Rate}

Each chemical reaction in FDS must be defined with a fuel and sufficient information such that a heat of formation is known for each participating species. The heat release per unit volume is found by summing the species mass production rates by the respective species heat of formation:
\begin{equation}\label{eq:vol_heat_gen}
\dot{q}^{\prime\prime\prime} \equiv -\displaystyle \sum_{\alpha} \dot{m}^{\prime\prime\prime} \Delta H_{f,\alpha}
\end{equation}
While the mixing controlled reaction rate is an easily computed and robust subgrid-scale model of and the Arrhenius rate is certainly a robust model when applicable, there is still a need, in certain situations, to put an upper bound on the local heat release rate per unit volume. The reason for
this is that FDS is applied over length scales ranging from millimeters to tens of meters, and the resolution of the numerical grid
is sometimes too coarse to expect the simple mixing time model to work effectively.
A scaling analysis of pool fires by Orloff and De Ris~\cite{Orloff:19th_Symposium} suggests that the spatial average of the
heat release rate of a fire is approximately 1200~kW/m$^3$. FDS uses by default a value of 2500~kW/m$^3$ as an upper bound\footnote{Note that
for DNS, FDS imposes a less restrictive upper bound on the local heat release rate per unit volume. It is
\be \dq_{\max}''' = 200/\dx + 2500 \quad \hbox{kW/m}^3 \ee
The value of 200~kW/m$^2$ is an upper bound on the heat release rate per unit area of flame sheet.}
on the local value of the heat release rate per unit volume. Typically, this bound only affects fires whose value of $Q^*$
is less than one\footnote{The non-dimensional quantity, $Q^*$, is a measure of the fire's heat release rate divided by the
area of its base. It is expressed as $Q^*=\dQ/(\rho_\infty c_p T_\infty \sqrt{g D} D^2)$.}.

