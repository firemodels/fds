% !TEX root = FDTs_Validation_Guide.tex

\chapter{Compartment Pressure}
\label{Pressure_Chapter}

In FDS, the pressure is decomposed into a temporally-varying background pressure and a spatially-varying perturbation that drives the flow. The former can
be thought of as the ``over-pressure,'' and it is essentially a check on global mass and energy balances; whereas the latter has most to do with momentum conservation.
In real buildings, leakage and ventilation affect the compartment ``over-pressure'' along with the fire, which also affects the pressure perturbation.

\section{NIST/NRC Test Series}

Comparisons between measured and predicted pressures for the NIST/NRC Test Series are shown
in on the following pages.
For those tests in which the door to the compartment is
open, the over-pressures are only a few Pascals, whereas when the door is closed, the over-pressures are several hundred Pascals.
The pressure within the compartment was measured at a single point, near the floor.
In the simulations of the closed door tests, the compartment was assumed to leak via a small uniform flow spread
over the walls and ceiling.  The flow rate was calculated based on the assumption that the leakage rate is proportional
to the measured leakage area times the square root of compartment over-pressure.

Note that in the closed door tests, there is often a dramatic drop in the predicted compartment pressure.
This is the result of the assumption in FDS that the heat release rate is decreased to zero in one second at the time
in the experiment when the fuel flow was stopped for safety reasons.  In reality, the fire did not extinguish
immediately because there was an excess of fuel in the pan following the flow stoppage.
For the purpose of model comparison, the peak over-pressures are differenced in the closed door tests,
and the peak (albeit small) under-pressures are compared in the open door tests.

\newpage

\begin{figure}[p]
\begin{tabular*}{\textwidth}{l@{\extracolsep{\fill}}r}
\includegraphics[height=2.2in]{SCRIPT_FIGURES/NIST_NRC/NIST_NRC_01_Pressure} &
\includegraphics[height=2.2in]{SCRIPT_FIGURES/NIST_NRC/NIST_NRC_07_Pressure} \\
\includegraphics[height=2.2in]{SCRIPT_FIGURES/NIST_NRC/NIST_NRC_02_Pressure} &
\includegraphics[height=2.2in]{SCRIPT_FIGURES/NIST_NRC/NIST_NRC_08_Pressure} \\
\includegraphics[height=2.2in]{SCRIPT_FIGURES/NIST_NRC/NIST_NRC_04_Pressure} &
\includegraphics[height=2.2in]{SCRIPT_FIGURES/NIST_NRC/NIST_NRC_10_Pressure} \\
\includegraphics[height=2.2in]{SCRIPT_FIGURES/NIST_NRC/NIST_NRC_13_Pressure} &
\includegraphics[height=2.2in]{SCRIPT_FIGURES/NIST_NRC/NIST_NRC_16_Pressure}
\end{tabular*}
\label{NIST_NRC_Pressure_Closed}
\end{figure}

\begin{figure}[p]
\begin{tabular*}{\textwidth}{l@{\extracolsep{\fill}}r}
\includegraphics[height=2.2in]{SCRIPT_FIGURES/NIST_NRC/NIST_NRC_17_Pressure} &
   \\
\includegraphics[height=2.2in]{SCRIPT_FIGURES/NIST_NRC/NIST_NRC_03_Pressure} &
\includegraphics[height=2.2in]{SCRIPT_FIGURES/NIST_NRC/NIST_NRC_09_Pressure} \\
\includegraphics[height=2.2in]{SCRIPT_FIGURES/NIST_NRC/NIST_NRC_05_Pressure} &
\includegraphics[height=2.2in]{SCRIPT_FIGURES/NIST_NRC/NIST_NRC_14_Pressure} \\
\includegraphics[height=2.2in]{SCRIPT_FIGURES/NIST_NRC/NIST_NRC_15_Pressure} &
\includegraphics[height=2.2in]{SCRIPT_FIGURES/NIST_NRC/NIST_NRC_18_Pressure}
\end{tabular*}
\label{NIST_NRC_Pressure_Open}
\end{figure}

\begin{figure}[p]
\begin{center}
\begin{tabular}{c}
\includegraphics[width=4.0in]{SCRIPT_FIGURES/ScatterPlots/Compartment_Pressure} \\
\includegraphics[width=4.0in]{SCRIPT_FIGURES/ScatterPlots/Open_Compartment_Pressure}
\end{tabular}
\end{center}
\caption[Summary of pressure predictions, NIST/NRC test series.]
{Summary of Pressure Results. The top graph shows the compartment ``over-pressure'' in closed door tests; the lower graph
shows the small pressure perturbation in the open door tests.}
\end{figure}



\clearpage

\section{LLNL Enclosure Series}

The test report of the LLNL Enclosure experiments lists the mass flow rate, $\dot{m}$, through the exhaust duct at different times
during the tests. It also lists the compartment over-pressures, $\Delta p$, at these same times. From the simple leak formula:
\be \frac{\dot{m}}{\rho_0} = A \, \sqrt{\frac{2 \, \Delta p}{\rho_0}} \ee
the leakage area, $A$, is estimated to be 0.018~m$^2$, based on the initial exhaust rate and pressure. 
For modeling purposes, the ``leakage area'' is assumed to be the sum of the
inlet duct area plus any actual compartment leakage area. The mass flow rate through the exhaust duct is specified explicitly in
the model. There is not enough details of the ventilation system to model the fan and filtration system within the exhaust duct.

In the figures on the following pages, the open circles represent the measured pressure; the line represents the predicted pressure. The
predicted pressures are time-averaged over a time interval that is one-tenth the total simulation time. In general, the short-duration pressure
spike that is typical of fires within relatively tight compartments has been smoothed over in the reported test data. Depending on the simulation, it
often appears in the simulation data. The comparison of measurement and prediction is based on the final few pressure points, not the
initial spike. 

\begin{figure}[p]
\begin{tabular*}{\textwidth}{l@{\extracolsep{\fill}}r}
\includegraphics[height=2.2in]{SCRIPT_FIGURES/LLNL_Enclosure/LLNL_01_Pres} &
\includegraphics[height=2.2in]{SCRIPT_FIGURES/LLNL_Enclosure/LLNL_02_Pres} \\
\includegraphics[height=2.2in]{SCRIPT_FIGURES/LLNL_Enclosure/LLNL_03_Pres} &
\includegraphics[height=2.2in]{SCRIPT_FIGURES/LLNL_Enclosure/LLNL_04_Pres} \\
\includegraphics[height=2.2in]{SCRIPT_FIGURES/LLNL_Enclosure/LLNL_05_Pres} &
\includegraphics[height=2.2in]{SCRIPT_FIGURES/LLNL_Enclosure/LLNL_06_Pres} \\
\includegraphics[height=2.2in]{SCRIPT_FIGURES/LLNL_Enclosure/LLNL_07_Pres} &
\includegraphics[height=2.2in]{SCRIPT_FIGURES/LLNL_Enclosure/LLNL_08_Pres}
\end{tabular*}
\label{LLNL_Enclosure_Pres_1}
\end{figure}

\begin{figure}[p]
\begin{tabular*}{\textwidth}{l@{\extracolsep{\fill}}r}
\includegraphics[height=2.2in]{SCRIPT_FIGURES/LLNL_Enclosure/LLNL_09_Pres} &
\includegraphics[height=2.2in]{SCRIPT_FIGURES/LLNL_Enclosure/LLNL_10_Pres} \\
\includegraphics[height=2.2in]{SCRIPT_FIGURES/LLNL_Enclosure/LLNL_11_Pres} &
\includegraphics[height=2.2in]{SCRIPT_FIGURES/LLNL_Enclosure/LLNL_12_Pres} \\
\includegraphics[height=2.2in]{SCRIPT_FIGURES/LLNL_Enclosure/LLNL_13_Pres} &
\includegraphics[height=2.2in]{SCRIPT_FIGURES/LLNL_Enclosure/LLNL_14_Pres} \\
\includegraphics[height=2.2in]{SCRIPT_FIGURES/LLNL_Enclosure/LLNL_15_Pres} &
\includegraphics[height=2.2in]{SCRIPT_FIGURES/LLNL_Enclosure/LLNL_16_Pres}
\end{tabular*}
\label{LLNL_Enclosure_Pres_2}
\end{figure}

\begin{figure}[p]
\begin{tabular*}{\textwidth}{l@{\extracolsep{\fill}}r}
\includegraphics[height=2.2in]{SCRIPT_FIGURES/LLNL_Enclosure/LLNL_17_Pres} &
\includegraphics[height=2.2in]{SCRIPT_FIGURES/LLNL_Enclosure/LLNL_18_Pres} \\
\includegraphics[height=2.2in]{SCRIPT_FIGURES/LLNL_Enclosure/LLNL_19_Pres} &
\includegraphics[height=2.2in]{SCRIPT_FIGURES/LLNL_Enclosure/LLNL_20_Pres} \\
\includegraphics[height=2.2in]{SCRIPT_FIGURES/LLNL_Enclosure/LLNL_21_Pres} &
\includegraphics[height=2.2in]{SCRIPT_FIGURES/LLNL_Enclosure/LLNL_22_Pres} \\
\includegraphics[height=2.2in]{SCRIPT_FIGURES/LLNL_Enclosure/LLNL_23_Pres} &
\includegraphics[height=2.2in]{SCRIPT_FIGURES/LLNL_Enclosure/LLNL_24_Pres}
\end{tabular*}
\label{LLNL_Enclosure_Pres_3}
\end{figure}

\begin{figure}[p]
\begin{tabular*}{\textwidth}{l@{\extracolsep{\fill}}r}
\includegraphics[height=2.2in]{SCRIPT_FIGURES/LLNL_Enclosure/LLNL_25_Pres} &
\includegraphics[height=2.2in]{SCRIPT_FIGURES/LLNL_Enclosure/LLNL_26_Pres} \\
\includegraphics[height=2.2in]{SCRIPT_FIGURES/LLNL_Enclosure/LLNL_27_Pres} &
\includegraphics[height=2.2in]{SCRIPT_FIGURES/LLNL_Enclosure/LLNL_28_Pres} \\
\includegraphics[height=2.2in]{SCRIPT_FIGURES/LLNL_Enclosure/LLNL_29_Pres} &
\includegraphics[height=2.2in]{SCRIPT_FIGURES/LLNL_Enclosure/LLNL_30_Pres} \\
\includegraphics[height=2.2in]{SCRIPT_FIGURES/LLNL_Enclosure/LLNL_31_Pres} &
\includegraphics[height=2.2in]{SCRIPT_FIGURES/LLNL_Enclosure/LLNL_32_Pres}
\end{tabular*}
\label{LLNL_Enclosure_Pres_4}
\end{figure}

\begin{figure}[p]
\begin{tabular*}{\textwidth}{l@{\extracolsep{\fill}}r}
\includegraphics[height=2.2in]{SCRIPT_FIGURES/LLNL_Enclosure/LLNL_33_Pres} &
\includegraphics[height=2.2in]{SCRIPT_FIGURES/LLNL_Enclosure/LLNL_34_Pres} \\
\includegraphics[height=2.2in]{SCRIPT_FIGURES/LLNL_Enclosure/LLNL_35_Pres} &
\includegraphics[height=2.2in]{SCRIPT_FIGURES/LLNL_Enclosure/LLNL_36_Pres} \\
\includegraphics[height=2.2in]{SCRIPT_FIGURES/LLNL_Enclosure/LLNL_37_Pres} &
\includegraphics[height=2.2in]{SCRIPT_FIGURES/LLNL_Enclosure/LLNL_38_Pres} \\
\includegraphics[height=2.2in]{SCRIPT_FIGURES/LLNL_Enclosure/LLNL_39_Pres} &
\includegraphics[height=2.2in]{SCRIPT_FIGURES/LLNL_Enclosure/LLNL_40_Pres}
\end{tabular*}
\label{LLNL_Enclosure_Pres_5}
\end{figure}

\begin{figure}[p]
\begin{tabular*}{\textwidth}{l@{\extracolsep{\fill}}r}
\includegraphics[height=2.2in]{SCRIPT_FIGURES/LLNL_Enclosure/LLNL_41_Pres} &
\includegraphics[height=2.2in]{SCRIPT_FIGURES/LLNL_Enclosure/LLNL_42_Pres} \\
\includegraphics[height=2.2in]{SCRIPT_FIGURES/LLNL_Enclosure/LLNL_43_Pres} &
\includegraphics[height=2.2in]{SCRIPT_FIGURES/LLNL_Enclosure/LLNL_44_Pres} \\
\includegraphics[height=2.2in]{SCRIPT_FIGURES/LLNL_Enclosure/LLNL_45_Pres} &
\includegraphics[height=2.2in]{SCRIPT_FIGURES/LLNL_Enclosure/LLNL_46_Pres} \\
\includegraphics[height=2.2in]{SCRIPT_FIGURES/LLNL_Enclosure/LLNL_47_Pres} &
\includegraphics[height=2.2in]{SCRIPT_FIGURES/LLNL_Enclosure/LLNL_48_Pres}
\end{tabular*}
\label{LLNL_Enclosure_Pres_6}
\end{figure}

\begin{figure}[p]
\begin{tabular*}{\textwidth}{l@{\extracolsep{\fill}}r}
\includegraphics[height=2.2in]{SCRIPT_FIGURES/LLNL_Enclosure/LLNL_49_Pres} &
\includegraphics[height=2.2in]{SCRIPT_FIGURES/LLNL_Enclosure/LLNL_50_Pres} \\
\includegraphics[height=2.2in]{SCRIPT_FIGURES/LLNL_Enclosure/LLNL_51_Pres} &
\includegraphics[height=2.2in]{SCRIPT_FIGURES/LLNL_Enclosure/LLNL_52_Pres} \\
\includegraphics[height=2.2in]{SCRIPT_FIGURES/LLNL_Enclosure/LLNL_53_Pres} &
\includegraphics[height=2.2in]{SCRIPT_FIGURES/LLNL_Enclosure/LLNL_54_Pres} \\
\includegraphics[height=2.2in]{SCRIPT_FIGURES/LLNL_Enclosure/LLNL_55_Pres} &
\includegraphics[height=2.2in]{SCRIPT_FIGURES/LLNL_Enclosure/LLNL_56_Pres}
\end{tabular*}
\label{LLNL_Enclosure_Pres_7}
\end{figure}

\begin{figure}[p]
\begin{tabular*}{\textwidth}{l@{\extracolsep{\fill}}r}
\includegraphics[height=2.2in]{SCRIPT_FIGURES/LLNL_Enclosure/LLNL_57_Pres} &
\includegraphics[height=2.2in]{SCRIPT_FIGURES/LLNL_Enclosure/LLNL_58_Pres} \\
\includegraphics[height=2.2in]{SCRIPT_FIGURES/LLNL_Enclosure/LLNL_59_Pres} &
\includegraphics[height=2.2in]{SCRIPT_FIGURES/LLNL_Enclosure/LLNL_60_Pres} \\
\includegraphics[height=2.2in]{SCRIPT_FIGURES/LLNL_Enclosure/LLNL_61_Pres} &
\includegraphics[height=2.2in]{SCRIPT_FIGURES/LLNL_Enclosure/LLNL_62_Pres} \\
\includegraphics[height=2.2in]{SCRIPT_FIGURES/LLNL_Enclosure/LLNL_63_Pres} &
\includegraphics[height=2.2in]{SCRIPT_FIGURES/LLNL_Enclosure/LLNL_64_Pres}
\end{tabular*}
\label{LLNL_Enclosure_Pres_8}
\end{figure}

\begin{figure}[p]
\begin{center}
\includegraphics[width=4.0in]{SCRIPT_FIGURES/ScatterPlots/LLNL_Pressure} 
\end{center}
\caption[Summary of pressure predictions, LLNL Enclosure Experiments.]
{Comparison of measured and predicted pressure rise in the LLNL Enclosure Experiments.}
\end{figure}
