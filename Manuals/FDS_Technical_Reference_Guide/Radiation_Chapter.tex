% !TEX root = FDS_Technical_Reference_Guide.tex

\chapter{Thermal Radiation}
\label{chapter:radiation}

\section{Radiation Transport Equation}

Energy transport consists of convection, conduction and
radiation. Convection of heat is accomplished via the solution of the
basic conservation equations. Gains and losses of heat via gas phase conduction
and radiation are represented by the divergence of the heat flux
vector in the energy equation, $\nabla\!\cdot \dbq''$. This section
describes the equations associated with the radiative part, $\dbq''_r$.

The Radiative Transport Equation (RTE) for an absorbing, emitting, and scattering medium is
\begin{eqnarray}
\bs \cdot \nabla I_{\la}(\bx,\bs) &=&
\underbrace{ - \kappa(\bx,\la)   \; I_\la(\bx,\bs) }_{\textrm{Energy loss by absorption}} -
\underbrace{\sigma_s(\bx,\la) \; I_\la(\bx,\bs) }_{\textrm{Energy loss by diffusion}} +  \nonumber  \\ [0.25in]
& & \underbrace{   B(\bx,\la) }_{\textrm{Emission source term}} + \quad
\underbrace{   \frac{\sigma_s(\bx,\la)}{4\pi}
\int_{4\pi}\Phi(\bs',\bs) \; I_{\la}(\bx,\bs') \; d\bs'
 }_{\textrm{In-scattering term}}
\label{RTEbasic}
\end{eqnarray}
where $I_{\la}(\bx,\bs)$ is the radiation intensity at wavelength,
$\la$, $\bs$ is the direction vector of the intensity, and
$\kappa(\bx,\la)$ and $\sigma_s(\bx,\la)$ are the local absorption
and scattering coefficients,
respectively. $B(\bx,\la)$ is the emission source term, describing how much heat is emitted by the local mixture of gas, soot and droplets/particles.
The integral on the right hand side describes the in-scattering from other directions. The in-scattering and scattering terms are detailed in section~\ref{droplet-radiation}.

In practical simulations, the spectral ($\la$) dependence of the RTE cannot be resolved
accurately. Instead, the radiation spectrum is divided into a relatively small number of bands and a separate RTE is derived for
each band. For instance, the band specific RTE for a non-scattering gas is
\be   \bs \cdot \nabla I_n(\bx,\bs) = B_n(\bx)- \kappa_n(\bx) I_n(\bx,\bs),\;\; n = 1...N
\label{bandRTE} \ee
where $I_n$ is the intensity integrated over the band $n$, and $\kappa_n$
is the appropriate mean absorption coefficient for the band. When the intensities corresponding to the bands are known, the total
intensity is calculated by summing over all the bands
\be
   I(\bx,\bs) = \sum_{n=1}^N I_n(\bx,\bs)
\ee
Similarly, the total values of other radiation quantities can be calculated as sums of their band-specific values.

\section{Radiation in absorming and emitting media}

\subsection{Emission Source Term}

The emission term is a function of the refractive index, $n_r$, absorption coefficient, $\kappa$, and a source term, $I_{b,\la}$:
\be
B(\bx,\la) = n_r(\bx,\la) \kappa(\bx,\la) I_{b,\la}(\bx)
\ee
In most fire applications the radiation is emitted to air and we can assume $n_r(\bx,\la)=1$. 

The source term $I_{b,\la}(\bx)$ is the {\em Planck function} that represents the
intensity of blackbody radiation at temperature, $T=T(\bx)$, and wavelength, $\la$
\be
I_{b,\la}(T) = \dfrac{2 \, h \, c^2 }{\la^5 \left[\exp\left(\dfrac{h \, c }{\la \, k_B \, T}\right)-1\right]}
\ee
Here, $h$ is the Planck constant ($h$ = 6.626 10$^{\rm -34}$~J.s),
$c$ is the speed of light in vacuum ($c$ = 2.998$\cdot$10$^{\rm 8}$~m.s$^{\rm -1}$),
and $k_B$ is the Boltzmann constant ($k_B$ = 1.381 10$^{\rm -23}$~J.K$^{\rm -1}$).
The Planck function can also be written in terms of wavenumber, $\om=1/\la$,
\be
I_{b,\om}(T) = \dfrac{2 \, h \, c^2 \, \om^3}{\exp\left(\dfrac{h \, c \, \om}{k_B \, T}\right)-1}
\ee

The emission source term for radiation band $n$ is
\be
B_n(\bx) = \kappa_n(\bx) I_{b,n}(T(\bx))
\ee
where $I_{b,n}$ is can be calculated as a fraction of the blackbody radiation at temperature $T(\bx)$
\be I_{b,n} = F_n(\la_{\rm min},\la_{\rm max}) \; \sigma \; T^4/\pi \ee
where $\sigma$ is the Stefan-Boltzmann constant.
The calculation of factors $F_n$ is explained in Ref.~\cite{Siegel:1}.

\subsection{Limits of the radiation bands}

Even with a reasonably small number of bands, solving multiple RTEs is very time consuming. Fortunately, in most large-scale fire
scenarios soot is the most important combustion product controlling the thermal radiation from the fire and hot smoke. As the radiation spectrum of
soot is continuous, it is possible to assume that the gas behaves as a gray medium.  The spectral dependence is then lumped into one
absorption coefficient ($N=1$) and the source term is given by the
black body radiation intensity
\be 
I_b(\bx) = \frac{\sigma \, T(\bx)^4}{\pi} \label{emission_source_term} 
\ee
This is the default mode of FDS and appropriate for most problems of fire engineering. 

In optically thin flames, where the amount of soot is small compared to the amount of $\rm CO_2$ and water vapor, the gray gas
assumption may produce significant over-predictions of the emitted radiation. 
From a series of numerical experiments using methane as the fuel, it has been found
that six bands ($N=6$) are usually enough to improve the accuracy in these cases. The limits of the bands are selected to give an accurate
representation of the most important radiation bands of $\rm CO_2$ and water vapor. 
The limits of the bands for methane are shown in Table~\ref{banditos}.

If the absorption of the fuel is known to be important, separate bands can be reserved for fuel, increasing the total number of bands, $N$. The number of 
additional bands depends on the fuel, as discussed in section~\ref{gas_spectra}.

\begin{table}[ht]
%To BE UPDATED
\caption{Limits of the spectral bands.}
\vspace{0.1in}
\label{banditos}
\small
\begin{tabular}{|*{10}{c|}}
\hline
\hspace{0.5in} \underline{9 Band Model} \hspace{0.5in} & 1  & 2  & 3 & 4  & 5 & 6 & 7 & 8 & 9 \\ \cline{2-10}
                                     & Soot   & CO$_2$       & CH$_4$ & Soot & CO$_2$ & H$_2$O & H$_2$O       & Soot & Soot          \\
\raisebox{1.5ex}[0pt]{Major Species} &        & H$_2$O, Soot & Soot   &      & Soot   & Soot   & CH$_4$, Soot &      &      \\ \hline
\multicolumn{1}{c}{$\omega$ (1/cm)}
             & \multicolumn{1}{@{\hspace{-.2in}10000}c@{}}{ }
             & \multicolumn{1}{@{\hspace{-.2in} 3800}c@{}}{ }
             & \multicolumn{1}{@{\hspace{-.2in} 3400}c@{}}{ }
             & \multicolumn{1}{@{\hspace{-.2in} 2800}c@{}}{ }
             & \multicolumn{1}{@{\hspace{-.2in} 2400}c@{}}{ }
             & \multicolumn{1}{@{\hspace{-.2in} 2174}c@{}}{ }
             & \multicolumn{1}{@{\hspace{-.2in} 1429}c@{}}{ }
             & \multicolumn{1}{@{\hspace{-.2in} 1160}c@{}}{ }
             & \multicolumn{1}{@{\hspace{-.2in} 1000}c@{50}}{ } \\
\multicolumn{1}{c}{$\la$ ($\mu$m)}
             & \multicolumn{1}{@{\hspace{-.2in} 1.00}c@{}}{ }
             & \multicolumn{1}{@{\hspace{-.2in} 2.63}c@{}}{ }
             & \multicolumn{1}{@{\hspace{-.2in} 2.94}c@{}}{ }
             & \multicolumn{1}{@{\hspace{-.2in} 3.57}c@{}}{ }
             & \multicolumn{1}{@{\hspace{-.2in} 4.17}c@{}}{ }
             & \multicolumn{1}{@{\hspace{-.2in} 4.70}c@{}}{ }
             & \multicolumn{1}{@{\hspace{-.2in} 7.00}c@{}}{ }
             & \multicolumn{1}{@{\hspace{-.2in} 8.62}c@{}}{ }
             & \multicolumn{1}{@{\hspace{-.2in} 10.0}c@{200}}{ } \\ \hline
\underline{6 Band Model}  & 1  & 2  & \multicolumn{2}{|c|}{3} & 4  & \multicolumn{3}{|c|}{5} & 6  \\ \cline{2-10}
          & Soot   & CO$_2$       & \multicolumn{2}{|c|}{CH$_4$      } & CO$_2$ & \multicolumn{3}{|c|}{H$_2$O, CH$_4$, Soot} & Soot  \\
\raisebox{1.5ex}[0pt]{Major Species} &        & H$_2$O, Soot & \multicolumn{2}{|c|}{Soot} & Soot   & \multicolumn{3}{|c|}{  } &       \\
               \hline
\end{tabular}
\end{table}
\normalsize

\subsection{Absorption Coefficients}

For the calculation of the gray or band-mean absorption coefficients, $\kappa_n$, a narrow-band model, RADCAL~\cite{RadCal}, has been
implemented in FDS. RADCAL computes the spectral properties of the radiation participating species at discrete values of wavenumber and temperature,
and returns two alternative mean absorption coefficients for each spectral band, $n$. The first coefficient is the Planck mean coefficient
\begin{equation}
\kappa_n(P,T) = \dfrac{\pi}{\sigma T^4}
\displaystyle\int_{\om_{\min}}^{\om_{\max}}{I_{b,\om}(T)
\displaystyle\sum_{i}^{\rm species} \bar{\kappa}_i(\om,T) \, P_i \; \d\om}
\end{equation}
where $\om$ is the wavenumber, in units of cm$^{\rm -1}$,
$P_i$ is the partial pressure of the $i$th participating species, in units of ${\rm atm}$, and
$\bar{\kappa}_i$ is the spectral absorption coefficient of the $i$th participating species, in units of
${\rm atm^{-1}cm^{-1}}$. 
Note that the temperature used in the calculation of $\kappa_n$ is the local gas temperature; thus, $\kappa_n(P,T)$ is a function
of the gas-phase temperature and partial pressure, and is independent of the pathlength. Its units are in cm$^{\rm -1}$.

The second coefficient is so-called {\em path mean} or {\em effective} absorption coefficient,$\kappa_n(e,T)$ which is defined according to the following
equation
\be
\int_{\la_{\rm min}}^{\la_{\rm max}}I(\la,L,T,T_{\rm rad})d\la = \frac{\sigma}{\pi}
       \left[\left(1-e^{-\kappa_n(e,T)L}\right)T^4 + e^{-\kappa_n(e,T)L} T_{\rm rad}^4\right]
\ee
where $L$ is the path lenght and $T_{\rm rad}$ is the effective temperature of flame radiation. RADCAL calculates the left-hand-side 
integral by calculating the intensity leaving a uniform gas layer of thickness $L$, bounded by a black wall at temperature $T_{\rm rad}$, for 
a large number of narrow spectral bands.  By default, path length $L$ 
is five times the characteristic cell size of the simulation, limited by value $L = $ 10 m from above. It can also be set by the user. 

In cases with only one band ($N$=1), the smaller of the two absorption coefficients is used:
\be \kappa_n=\min(\kappa_n(P,T),\kappa_n(e,T))
\ee
If $N>1$ or $L=0$, $\kappa_n=\kappa_n(P,T)$. Note that the spectral data within RADCAL are used whenever the gas mixture contains water vapor, 
fuel or combustion products, regardless of the number of radiation bands $N$. 

\subsection{Spectral data for gases}
\label{gas_spectra}

The value of the spectral absorption coefficient, $\bar{\kappa}_i$, is averaged over a narrow band whose spectral width, $\Delta \om$, varies from
5~cm$^{-1}$ for $\om < $ 1100~cm$^{-1}$, to 25~cm$^{-1}$ for 1100~cm$^{-1} \leq \om < $ 5000~cm$^{\rm -1}$,
and to 50~cm$^{\rm -1}$ for 5000~cm$^{\rm -1}\leq \om$. 

A combination of molecular models and data tables are used to compute the spectral radiative
properties of the radiation participating species.
The original version of RADCAL includes spectral properties of $\rm CO_2$, $\rm H_2O$, $\rm CO$, and
$\rm CH_4$ that are either modeled through quantitative molecular spectroscopy derivations
or tabulated from the fitting of experimental data into appropriate statistical narrow band models.
The original RADCAL data have been supplemented with new tabulated experimental data
for the following fuels:
\begin{itemize}
  \item Ethylene:  $\rm C_2H_4$
  \item Ethane:    $\rm C_2H_6$
  \item Propylene: $\rm C_3H_6$
  \item Propane:   $\rm C_3H_8$
  \item Toluene:   $\rm C_7H_8$
  \item \textit{n}-Heptane: $\rm C_7H_{16}$
  \item Methanol:  $\rm CH_3OH$
  \item Methyl Methacrylate: $\rm C_5H_8O_2$
\end{itemize}
These new data have been obtained through FTIR measurements for wavenumbers between 700~cm$\rm ^{-1}$ and 4000~cm$\rm ^{-1}$.
A useful quantity to compare the relative importance of the different IR bands is provided by the integrated band intensity,
$\alpha_i$, defined for the $i$th participating species as:
\be
  \alpha_i(T) = \displaystyle\int_{\om_{\min}}^{\om_{\max}} \bar{\kappa}_i(T) \; \d \om
\ee
whose units are $\rm {atm^{-1} cm^{-2}}$. The subsections below briefly describe the molecular bands for each of the gas-phase radiative species, and provide for most of them the integrated band intensity of their most important bands at the indicated temperature.

At the start of a simulation, the absorption coefficient are calculated using RADCAL and then tabulated as a function of mixture fraction and temperature. During the simulation, the local absorption coefficient is interpolated from the table of values.

\subsubsection{Carbon Dioxide: $\rm CO_2$}

Carbon dioxide is a linear molecule and has four vibrational modes, but only two fundamental IR vibration frequencies. It has five distinct bands that are included in RADCAL, see Table \ref{Table::CO2}.
\begin{table}[h!]
    \centering
    \caption{Spectral bands of $\rm CO_2$ included in RADCAL.}
    \vspace{0.1in}
    \label{Table::CO2}
    \begin{tabular}{|c|c|c|c|}
      \hline
      Band \# & \multicolumn{2}{|l|}{Bounds (cm$\rm ^{-1}$) } & Method \\
      \cline{1-4}
      1 &  500 & 880  & tabulated \\
      2 &  880 & 1100 & modeled \\
      3 & 1975 & 2475 & modeled \\
      4 & 3050 & 3800 & modeled \\
      5 & 4550 & 5275 & modeled \\
      \hline
    \end{tabular}
\end{table}
The strongest band in the $\rm CO_2$ spectrum is Band~3. At 300~K, it has an integrated band intensity of 2963~atm$^{-1}$cm$^{-2}$. The tabulated data were obtained from experiments with temperatures ranging from 300~K to 2400~K using the Goody statistical narrow band model.

\subsubsection{Carbon Monoxide: $\rm CO$}

Carbon Monoxide is a diatomic molecule and as such, it has only one vibrational mode. RADCAL includes one distinct band, see Table \ref{Table::CO}.
\begin{table}[h!]
    \centering
    \caption{Spectral bands of $\rm CO$ included in RADCAL.}
    \vspace{0.1in}
    \label{Table::CO}
    \begin{tabular}{|c|c|c|c|}
      \hline
      Band \# & \multicolumn{2}{|l|}{Bounds (cm$\rm ^{-1}$) } & Method \\
      \cline{1-4}
      1 & 1600 & 2400 & modeled \\
      \hline
    \end{tabular}
\end{table}
It corresponds to the stretching of the triple bond $\rm C \equiv O$. The first overtone (centered at $ \om\approx 4260\;\rm {cm^{-1}}$) is not accounted for; its integrated band intensity is negligible at standard temperature and pressure. At 295~K, the integrated band intensity of Band~1 is $260\;\rm {atm^{-1}cm^{-2}}$.
The statistical narrow band model associated with $\rm CO$ is the Goody model. Recommended temperatures of use range from 295~K to 2500~K.

\subsubsection{Water Vapor: $\rm H_2O$}

Due to the non-linearity of its molecular structure, the IR spectrum of water vapor is complex and broad. In RADCAL, water vapor spectrum from 50~cm$\rm ^{-1}$ to 9300~cm$\rm ^{-1}$ is considered. Data in RADCAL are tabulated and provide from Ludwig \textit{et al.}~\cite{Ludwig:NASA}. Experimental data have been fitted using the statistical Goody narrow band model. The strongest bands at standard temperature and pressure are located in the ranges $\left[50-2100\right]$~cm$\rm ^{-1}$: $\alpha = 300\;\rm {atm^{-1}cm^{-2}}$, and $\left[3000-4000\right]$~cm$\rm ^{-1}$: $\alpha = 220\;\rm {atm^{-1}cm^{-2}}$.

\subsubsection{Methane: $\rm CH_4$}

Methane is a spherical top molecule of tetrahedral shape with the carbon atom occupying the center of the tetrahedron. It belongs to the point group $T_d$. The methane IR spectrum is the result of the vibration-rotation modes of the $\rm C-H$ groups. It has nine vibrational modes, but due to its symmetry, this translates into only two distinct IR active fundamental vibration frequencies. In RADCAL, the methane IR spectrum is divided into three distinct bands (fundamentals + degenerates), see Table \ref{Table::CH4}.
\begin{table}[ht]
      \centering
      \caption{Spectral bands of $\rm CH_4$ included in RADCAL.}
      \vspace{0.1in}
      \label{Table::CH4}
    \begin{tabular}{|c|c|c|c|c|c|}
    \hline
    Band \# & \multicolumn{2}{|l|}{Bounds (cm$\rm ^{-1}$) } & Method & Assignment & $\alpha(T=296 \; {\rm K})$\ \\
    \cline{1-6}
    1 & 1150 & 1600 & tabulated &  $\rm C-H$ Bend    & 237  \\
    2 & 2700 & 3250 & tabulated &  $\rm C-H$ Stretch & 212  \\
    3 & 3400 & 5000 & modeled   &  $\rm C-H$ Stretch &   \\
    \hline
   \end{tabular}
\end{table}
The strongest bands are Bands~1 and 2 which at standard temperature and pressure have an integrated band intensity of $237\;\rm {atm^{-1}cm^{-2}}$ and $212\;\rm {atm^{-1}cm^{-2}}$, respectively. The tabulated data were obtained from high resolution FTIR experiments with temperatures varying from 300~K to 1400~K. The spectral absorption coefficients were obtained assuming the FTIR measurements to be in the weak line regime and applying the Beer-Lambert Law to the experimental spectral transmissivity.

\subsubsection{Ethylene: $\rm C_2H_4$}

Ethylene is a molecule with a plane symmetrical form and belongs to the point group $D_{2h}$. The ethylene IR spectrum is the result of the vibration-rotation modes of the $\rm C=C$, $\rm CH$, and $\rm CH_2$ groups. It has 12 vibrational modes. In RADCAL, its IR spectrum is divided into four distinct bands, see Table \ref{Table::C2H4}.
\begin{table}[ht]
    \centering
    \caption{Spectral bands of $\rm C_2H_4$ included in RADCAL.}
    \vspace{0.1in}
    \label{Table::C2H4}
    \begin{tabular}{|c|c|c|c|c|c|}
      \hline
      Band \# & \multicolumn{2}{|l|}{Bounds (cm$\rm ^{-1}$) } & Method & Assignment & $\alpha(T=296 \; {\rm K})$\\
      \cline{1-6}
      1 & 780  & 1250 & tabulated &  $\rm CH_2$ Bend      & 366 \\
      2 & 1300 & 1600 & tabulated &  $\rm CH_2$ Bend      & 43\\
      3 & 1750 & 2075 & tabulated &  $\rm C=C$  Stretch   & 20 \\
      4 & 2800 & 3400 & tabulated &  $\rm C-H$  Stretch   & 183 \\
      \hline
    \end{tabular}
\end{table}
Band~1 is the strongest absorbing band. All the ethylene IR spectral absorption data were obtained from high resolution FTIR experiments with temperatures varying from 296~K to 801~K. The spectral absorption coefficients were obtained by fitting the experimental spectral transmissivity of a homogeneous column of isothermal ethylene with a total pressure of 1~atm using the Goody model.

\subsubsection{Ethane: $\rm C_2H_6$}

Ethane has a three-fold axis of symmetry and belongs to the point group $D_{3d}$. The ethane IR spectrum is the result of the vibration-rotation modes of the $\rm C-C$, $\rm CH$, and $\rm CH_2$ groups. It has 18 vibrational modes; its IR spectrum is divided into three distinct bands, see Table~\ref{Table::C2H6}.
\begin{table} [ht]
    \centering
    \caption{Spectral bands of $\rm C_2H_6$ included in RADCAL.}
    \vspace{0.1in}
    \label{Table::C2H6}
    \begin{tabular}{|c|c|c|c|c|c|}
      \hline
      Band \# & \multicolumn{2}{|l|}{Bounds (cm$\rm ^{-1}$) } & Method & Assignment & $\alpha(T=296 \; {\rm K})$\\
      \cline{1-6}
      1 & 730  & 1095 & tabulated &  $\rm CH_3$ Rock   &  29  \\
      2 & 1250 & 1700 & tabulated &  $\rm CH$  Bend    &  64  \\
      3 & 2550 & 3375 & tabulated &  $\rm CH$  Stretch &  761 \\
      \hline
    \end{tabular}
\end{table}
Band~3 corresponds to the stretching of $\rm CH$ and is the strongest absorbing band. At standard temperature and pressure, its integrated band intensity is more than 10 times the value of
Band~2, and more than 20 times the value of Band~1. All the ethane IR spectral absorption data were obtained from high resolution FTIR experiments with temperatures varying from 296~K to 1000~K. The spectral absorption coefficients were obtained by fitting the experimental spectral transmissivity of a homogeneous column of isothermal ethane with a total pressure of 1~atm using the Elsasser model.

\subsubsection{Propylene: $\rm C_3H_6$}

Propylene has only one plane of symmetry and belongs to the point group $C_s$. The propylene IR spectrum is the result of the vibration-rotation modes of the $\rm C-C$, $\rm C=C$, $\rm CH$, $\rm CH_2$, and $\rm CH_3$ groups. It has 21 vibrational modes; its IR spectrum is divided into three distinct bands, see Table \ref{Table::C3H6}.
\begin{table}[ht]
   \centering
   \caption{Spectral bands of $\rm C_3H_6$ included in RADCAL.}
   \vspace{0.1in}
   \label{Table::C3H6}
   \begin{tabular}{|c|c|c|c|c|c|}
    \hline
    Band \# & \multicolumn{2}{|l|}{Bounds (cm$\rm ^{-1}$) } & Method & Assignment & $\alpha(T=296 \; {\rm K})$\\
    \cline{1-6}
    1 & 775  & 1150 & tabulated &  $\rm C-C$ Stretch, $\rm CH_3$ Rock & 296 \\
    2 & 1225 & 1975 & tabulated &  $\rm C=C$ Stretch, $\rm CH$ Bend   & 271 \\
    3 & 2650 & 3275 & tabulated &  $\rm CH$ Stretch                   & 509 \\
    \hline
   \end{tabular}
\end{table}
Band~3 corresponds to the stretching of $\rm CH$ and is the strongest of all Propylene absorbing bands. All the propylene IR spectral absorption data were obtained from high resolution FTIR experiments with temperatures varying from 296~K to 1003~K. The spectral absorption coefficients were obtained by fitting the experimental spectral transmissivity of a homogeneous column of isothermal propylene with a total pressure of 1~atm using the Goody model.

\subsubsection{Propane: $\rm C_3H_8$}

Propane has two planes of symmetry and two axes of rotation. It belongs to the point group $C_{2v}$. The propane IR spectrum is the result of the vibration-rotation modes of the $\rm C-C$, $\rm CH_2$, $\rm CH_3$ groups. It has 27 vibrational modes; its IR spectrum is divided into two distinct bands, see Table \ref{Table::C3H8}.
\begin{table}[ht]
   \centering
   \caption{Spectral bands of $\rm C_3H_8$ included in RADCAL.}
   \vspace{0.1in}
   \label{Table::C3H8}
   \begin{tabular}{|c|c|c|c|c|c|}
    \hline
    Band \# & \multicolumn{2}{|l|}{Bounds (cm$\rm ^{-1}$) } & Method & Assignment & $\alpha(T=295 \; {\rm K})$ \\
    \cline{1-6}
    1 & 1175 & 1675 & tabulated &  $\rm CH_3$ Bending        & 121 \\
    2 & 2550 & 3375 & tabulated &  $\rm CH_3, CH_2$ Stretch  & 1186 \\
    \hline
   \end{tabular}
\end{table}
Band~2 corresponds to the stretching of $\rm CH_3$ and $\rm CH_2$ and is the strongest of all propane absorbing bands. For similar conditions, Band~1 has a much lower integrated band intensity. All the propane IR spectral absorption data were obtained from high resolution FTIR experiments with temperatures varying from 295~K to 1009~K. The spectral absorption coefficients were obtained by fitting the experimental spectral transmissivity of a homogeneous column of isothermal propane with a total pressure of 1~atm using the Goody model.

\subsubsection{Toluene: $\rm C_7H_8$}

Toluene has only one plane of symmetry. It belongs to the point group $C_{s}$. The toluene IR spectrum is the result of the vibration-rotation modes of the $\rm C=C$, $\rm CH$, and $\rm CH_3$ groups. It has 39 vibrational modes. For ease of modeling using statistical narrow band models, its IR spectrum has been divided into five distinct bands, see Table \ref{Table::C7H8}.
\begin{table}[ht]
   \centering
   \caption{Spectral bands of $\rm C_7H_8$ included in RADCAL.}
   \vspace{0.1in}
   \label{Table::C7H8}
   \begin{tabular}{|c|c|c|c|c|c|}
    \hline
    Band \# & \multicolumn{2}{|l|}{Bounds (cm$\rm ^{-1}$) } & Method & Assignment & $\alpha(T=300 \; {\rm K})$\\
    \cline{1-6}
    1 & 700  & 805  & tabulated &  $\rm CH$ Bending      & 237 \\
    2 & 975  & 1175 & tabulated &  $\rm CH$ Bending      & 40  \\
    3 & 1275 & 1650 & tabulated &  $\rm CH_3$ Bending    & 166 \\
    4 & 1650 & 2075 & tabulated &  $\rm C=C$ Stretching  & 101 \\
    5 & 2675 & 3225 & tabulated &  $\rm CH_3$, $\rm CH$  Stretching & 510  \\
    \hline
   \end{tabular}
\end{table}
Band~5 corresponds to the stretching of $\rm CH_3$ and $\rm CH$, and it is the strongest absorbing band. All the toluene IR spectral absorption data were obtained from high resolution FTIR experiments with temperatures varying from 300~K to 795~K. The spectral absorption coefficients were obtained by fitting the experimental spectral transmissivity of a homogeneous column of isothermal toluene with a total pressure of 1~atm using the Goody model.

\subsubsection{\textit{n}-Heptane: $\rm C_7H_{16}$}

\textit{n}-heptane has two planes of symmetry and two axes of rotation. It belongs to the point group $C_{2v}$. The \textit{n}-heptane IR spectrum results from the vibration-rotation modes of the $\rm C-C$, $\rm CH_2$, and $\rm CH_3$ groups. It has 63 vibrational modes. For ease of modeling using statistical narrow band models, its IR spectrum has been divided into two distinct bands, see Table \ref{Table::C7H16}.
\begin{table}[ht]
   \centering
   \caption{Spectral bands of $\rm C_7H_{16}$ included in RADCAL.}
   \vspace{0.1in}
   \label{Table::C7H16}
   \begin{tabular}{|c|c|c|c|c|c|}
    \hline
    Band \# & \multicolumn{2}{|l|}{Bounds (cm$\rm ^{-1}$) } & Method & Assignment & $\alpha(T=293 \; {\rm K})$\\
    \cline{1-6}
    1 & 1100  & 1800 & tabulated &  $\rm CH_2, CH_3$ Bending    & 298 \\
    2 & 2250  & 3275 & tabulated &  $\rm CH_2, CH_3$ Stretching & 3055 \\
    \hline
   \end{tabular}
\end{table}
Band~2 corresponds to the stretching of $\rm CH_3$ and $\rm CH_2$ groups, and is the strongest absorbing band. All the \textit{n}-heptane IR spectral absorption data were obtained from high resolution FTIR experiments with temperatures varying from 293~K to 794~K. The spectral absorption coefficients were obtained by fitting the experimental spectral transmissivity of a homogeneous column of isothermal \textit{n}-heptane with a total pressure of 1~atm using the Goody model.

\subsubsection{Methanol: $\rm CH_3OH$}

Methanol has only one plane of symmetry. It belongs to the point group $C_{s}$. The methanol IR spectrum results from the vibration-rotation modes of the $\rm C-O$, $\rm OH$, and $\rm CH_3$ groups. It has 12 vibrational modes. For ease of modeling using statistical narrow band models, its IR spectrum has been divided into four distinct bands, see Table \ref{Table::CH3OH}.
\begin{table}[ht]
   \centering
   \caption{Spectral bands of $\rm CH_3OH$ included in RADCAL.}
   \vspace{0.1in}
   \label{Table::CH3OH}
   \begin{tabular}{|c|c|c|c|c|c|}
    \hline
    Band \# & \multicolumn{2}{|l|}{Bounds (cm$\rm ^{-1}$) } & Method & Assignment &  $\alpha(T=293 \; {\rm K})$ \\
    \cline{1-6}
    1 & 825  & 1125 & tabulated & $\rm C-O$ Stretching   & 593 \\
    2 & 1125 & 1700 & tabulated & $\rm CH_3, OH$ Bending & 197 \\
    3 & 2600 & 3225 & tabulated & $\rm CH_3$ Stretching  & 684 \\
    4 & 3525 & 3850 & tabulated & $\rm OH$ Stretching    & 112 \\
    \hline
   \end{tabular}
\end{table}
Band~3 corresponds to the stretching of the $\rm CH_3$ group and is the strongest absorbing band. All the methanol IR spectral absorption data were obtained from high resolution FTIR experiments with temperatures varying from 293~K to 804~K. The spectral absorption coefficients were obtained by fitting the experimental spectral transmissivity of a homogeneous column of isothermal methanol with a total pressure of 1~atm using the Goody model.

\subsubsection{Methyl Methacrylate: $\rm C_5H_8O_2$}

Methyl Methacrylate or MMA has the most complex IR spectrum of all the fuels presented above. With 15 atoms, it has 39 vibrational modes. The MMA IR spectrum results from the vibration-rotation modes of the $\rm C-O$, $\rm C=O$, $\rm C=C$, $\rm CH_2$, and $\rm CH_3$ groups. For ease of modeling using statistical narrow band models, its IR spectrum has been divided into six distinct bands, see Table \ref{Table::C5H8O2}.
\begin{table}[ht]
   \centering
   \caption{Spectral bands of $\rm C_5H_8O_2$ included in RADCAL.}
   \vspace{0.1in}
   \label{Table::C5H8O2}
    \begin{tabular}{|c|c|c|c|c|c|}
    \hline
    Band \# & \multicolumn{2}{|l|}{Bounds (cm$\rm ^{-1}$) } & Method & Assignment &  $\alpha(T=396 \; {\rm K})$ \\
    \cline{1-6}
    1 & 750  & 875  & tabulated & $\rm CH_2$ Bending          & 42   \\
    2 & 875  & 1050 & tabulated & $\rm CH_2$ Bending          & 131  \\
    3 & 1050 & 1250 & tabulated & $\rm C-O$ Stretching        & 800  \\
    4 & 1250 & 1550 & tabulated & $\rm CH_3$ Bending          & 490  \\
    5 & 1550 & 1975 & tabulated & $\rm C=C, C=O$ Stretching   & 538  \\
    6 & 2650 & 3275 & tabulated & $\rm CH_2, CH_3$ Stretching & 294  \\
    \hline
   \end{tabular}
\end{table}
Band~3 corresponds to the stretching of the $\rm C-O$ group and has the highest integrated band intensity. All the MMA IR spectral absorption data were obtained from high resolution FTIR experiments with temperatures varying from 396~K to 803~K. The spectral absorption coefficients were obtained by fitting the experimental spectral transmissivity of a homogeneous column of isothermal MMA with a total pressure of 1~atm using the Goody model.

\subsubsection{Statistical Narrow Band Models}

This section briefly describes the statistical models used to obtain most of the
tabulated species IR spectral absorption coefficients at different temperatures,
$\bar{\kappa}_i(T)$. Narrow band models are used in lieu of line-by-line models to
represent the IR spectra of radiating species in engineering applications. In the narrow band approach, the whole spectrum is
divided into small spectral bands (typically several $\rm cm^{-1}$), and
different statistical approaches are used to compute the average radiative
properties over these narrow bands. Two main models are presented below: the
Elsasser model and the Goody model. Both models assume Lorentz lines.

The Elsasser model assumes all the lines to have the same shape, same strength,
and to be equally spaced from each other. In this model, the spectral
transmissivity, $\bar{\tau}_{\om}$, of a homogeneous isothermal column filled
with only some gas of species $i$,
at a total pressure $P_T$ and with an optical pathlength $U = P_i L$ ($L$
being the column physical length and $P_i$ the $i$th participating radiating
species partial pressure), is given by the expression:
\be\label{eq::Elsasser}
    \bar{\tau}_{\om} = 1- \erf \left( \frac{\sqrt{\pi} \, \bar{\kappa}_i \, U }{\displaystyle \sqrt{1 + \frac{\pi \, \bar{\kappa}_i \, U}{4 \, \bar{\gamma}_i \, P_T}}}  \right)
\ee
where $\bar{\gamma}_i$ is the spectral fine structure parameter of the narrow
band. Its units are in $\rm atm^{-1}$. $\bar{\kappa}_i$ is the spectral absorption coefficient of the narrow
band. Its units are in $\rm atm^{-1}cm^{-1}$. The Goody model assumes all the lines to have the same shape, but to be randomly
spaced within the narrow band, and their line strength follows an exponential
distribution. For this model, the spectral transmissivity is given by the expression:
\be\label{eq::Goody}
    \bar{\tau}_{\om} = \exp\left(-\frac{\bar{\kappa}_i \, U} {\displaystyle \sqrt{1+\frac{\bar{\kappa}_i \, U}{4 \, \bar{\gamma}_i \, P_T}}}\right)
\ee
For both models, the two narrow band spectral quantities of the $i$th species ($\bar{\kappa}_i$ and
$\bar{\gamma}_i$) are obtained either from line-by-line calculations or by fitting
experimental data.

\textbf{Note}: For all the tabulated data, a linear interpolation of
$\bar{\kappa}_i$ and $\bar{\gamma}_i$ in temperature and/or in wavenumber is
performed by RADCAL when necessary. If the temperature sought is out of the
tabulated data range, then the data at the nearest temperature are used.


\subsection{Radiation Contribution to Energy Equation}

The radiant heat flux vector $\dbq_r''$ is defined
\be \dbq_r''(\bx) = \int_{4\pi} \; \bs' \, I(\bx,\bs') \; \d \bs'   \ee
The gas phase contribution to the radiative loss term in the energy equation is
\be -\nabla\!\cdot \dbq_r''(\bx)(\mbox{gas}) =
    \kappa(\bx) \, \left[ U(\bx) - 4 \pi \, I_b(\bx) \right]  \quad ; \quad
    U(\bx) = \int_{4\pi} \, I(\bx,\bs') \, \d \bs'  \label{net_emission}
\ee
In words, the net radiant energy gained by a grid cell is the
difference between that which is absorbed and that which is emitted.

\subsection{Correction of the Emission Source Term}

In calculations of limited spatial resolution, the source term, $I_b$,
defined in Eq.~(\ref{emission_source_term}) requires special treatment in the flaming region of the fire. Typical FDS calculations
use grid cells that are tens of centimeters in size, and consequently the computed temperatures constitute a bulk average for a given grid cell and are considerably lower than
one would expect in a diffusion flame.
Because of its fourth-power dependence on the temperature,
the source term must be modeled in those grid cells where combustion occurs. Elsewhere, the computed temperature is used directly to compute the source term.
It is assumed that this ``flaming region'' is where the local heat release rate is non-zero, $\dq'''>0$. In this region, we
multiply the emission source term by a corrective factor, $C$:
\be I_{b,f}(\bx) = C \, \frac{\sigma \, T(\bx)^4}{\pi}   \quad ; \quad
    C = \frac{\sum_{\dq'''_{ijk}>0} \left( \chi_r \, \dq'''_{ijk} + \kappa_{ijk} \, U_{ijk} \right) \, \d V}{\sum_{\dq'''_{ijk}>0} \left( 4 \, \kappa_{ijk} \, \sigma \, T^4_{ijk} \right) \, \d V} \label{corrected_emission_source_term}
\ee
When the source term defined in Eq.~(\ref{corrected_emission_source_term}) is
substituted into Eq.~(\ref{net_emission}), the net radiative emission from the flaming region becomes the desired fraction of the total heat release rate.

The radiative
fraction, $\chi_r$, is a useful quantity in fire science. It is the nominal fraction of the heat release rate is emitted as thermal radiation. For most combustibles, $\chi_r$
is between 0.3 and 0.4. However, in Eq.~(\ref{corrected_emission_source_term}), $\chi_r$ is
interpreted as the fraction of energy radiated from the flaming
region.  For a small fire ($D<1$~m), the local $\chi_r$ is
approximately equal to its global counterpart. However, as the fire
increases in size, the global value will typically decrease due to a
net re-absorption of the thermal radiation by the increasing smoke
mantle.


\section{Numerical Method}
\label{radnumericalmethodsection}

This section describes how $\nabla\!\cdot \dbq_r''$ (the radiative loss
term) is computed at all gas-phase cells, and how the
the radiative heat flux $\dq_r''$ is computed at solid boundaries.

\subsection{Angular Discretization}
\label{radiation-discre}


To obtain the discretized form of the
RTE, the unit sphere is divided into a finite number of solid angles.
The coordinate system used to discretize the solid angle is
shown in Figure~\ref{Angular}.
\begin{figure}[ht]
\begin{center}
\includegraphics[height=2in]{FIGURES/RadCoord}
\caption{Coordinate system of the angular discretization.}
\label{Angular}
\end{center}
\end{figure}
The discretization of the solid angle is done by dividing first
the polar angle, $\theta$, into $N_{\theta}$ bands, where
$N_{\theta}$ is an even integer.
Each $\theta$-band is then divided into
$N_{\phi}(\theta)$ parts in the azimuthal ($\phi$) direction.
$N_{\phi}(\theta)$ must be divisible by 4.
The numbers $N_{\theta}$ and $N_{\phi}(\theta)$ are chosen
to give the total number of angles $N_{\Omega}$ as close to
the value defined by the user as possible.
$N_{\Omega}$ is calculated as
\be
 N_{\Omega} = \sum_{i=1}^{N_{\theta}} N_{\phi}(\theta_i)
\ee
The distribution of the angles is based on empirical rules that try
to produce equal solid angles $\delta \bO^l = 4\pi/N_{\Omega}$. The
number of $\theta$-bands is
\be
 N_{\theta} = 1.17 \; N_{\Omega}^{1/2.26}
\ee
rounded to the nearest even integer. The number of $\phi$-angles
on each band is
\be
 N_{\phi}(\theta) = \max\left\{4,
        0.5\,N_{\Omega}\,\left[\cos(\theta^-)-\cos(\theta^+)\right]\right\}
\ee
rounded to the nearest integer that is divisible by 4.
$\theta^-$ and $\theta^+$ are
the lower and upper bounds of the $\theta$-band, respectively.
The discretization is symmetric with respect to the planes $x=0$, $y=0$, and
$z=0$. This symmetry has three important benefits:
First, it avoids the problems caused by the fact that the first-order
upwind scheme, used to calculate intensities on the cell boundaries,
is more diffusive in non-axial directions than axial.
Second, the treatment of the mirror boundaries becomes very simple, as
will be shown later. Third,
it avoids so called
``overhang'' situations, where $\bs\cdot {\bf i}$, $\bs\cdot {\bf j}$
or $\bs\cdot {\bf k}$ changes sign inside
the control angle. These ``overhangs'' would make the resulting system of
linear equations more complicated.

In the axially symmetric case these ``overhangs'' cannot be avoided, and a
special treatment, developed by Murthy and Mathur~\cite{Murthy}, is
applied. In these cases $N_{\phi}(\theta_i)$ is kept constant, and
the total number of angles is $N_{\Omega} = N_{\theta} \times
N_{\phi}$. In addition, the angle of the vertical slice of the cylinder is
chosen to be same as $\delta\phi$.



\subsection{Spatial Discretization}

The grid used for the RTE solver is the same as for the fluid solver.
The radiative transport equation~(\ref{bandRTE}) is solved using
techniques similar to those for convective transport in finite volume
methods for fluid flow~\cite{Raithby}; thus, the name given to it is
the Finite Volume Method (FVM). More details of the model
implementation can be found from~\cite{Hostikka:2008}.
To obtain the discretized form of the
RTE, the computational domain, the unit sphere and the spectrum are divided as explained in section~\ref{radiation-discre}.
In each grid cell a discretized equation is derived by integrating
Eq.~(\ref{bandRTE}) over the volume of cell $ijk$ and the control
angle $\delta \Omega^l$, to obtain
\be
  \int_{\delta \Omega^l} \int_{V_{ijk}}
   \bs' \cdot \nabla I(\bx',\bs') \d\bx' \d\bs' =
   \int_{\delta \Omega^l} \int_{V_{ijk}} \kappa(\bx') \;
    \left[ I_{b}(\bx') - I(\bx',\bs') \right] \d \bx' \d \bs'
\ee
The volume integral on the left hand side is replaced by a surface
integral over the cell faces using the divergence theorem. Note that
the procedure outlined below is appropriate for each band of a wide
band model, thus the subscript $n$ has been removed for clarity.

Assuming that the radiation intensity $I(\bx,\bs)$ is constant on each
of the cell faces, the surface integral can be approximated by a sum
over the cell faces.  Assuming further that $I(\bx,\bs)$ is constant
within the volume $V_{ijk}$ and over the angle $\delta \bO^l$ we
obtain
\be  \sum_{m=1}^6 A_m \; I_m^l \;
      \int_{\Omega^l}(\bs' \cdot \bn_m) \d \bs'
   = \kappa_{ijk} \,
     \left[ I_{b,ijk} - I_{ijk}^l \right] \; V_{ijk} \,
     \delta \Omega^l   \label{RTEdiscrete2}
\ee
where
\begin{tabbing}
$I_{ijk}^l$ \hspace{1in}  \=  radiant intensity in direction $l$ \\
$I_m^l$                   \>  radiant intensity at cell face $m$ \\
$I_{b,ijk}$               \>  radiant black body Intensity in cell \\
$\delta \Omega^l$         \>  solid angle corresponding to direction $l$ \\
$V_{ijk}$                 \>  volume of cell $ijk$ \\
$A_m$                     \>  area of cell face $m$ \\
$\bn_m$                   \>  unit normal vector of the cell face $m$
\end{tabbing}
Note that while the intensity is assumed constant within
the angle $\delta \bO^l$, its direction covers the angle $\delta \bO^l$
exactly.The local incident radiation intensity is
\be
 U_{ijk} = \sum_{l=1}^{N_{\Omega}} I_{ijk}^l \delta\Omega^l
\ee

In Cartesian coordinates\footnote{In the axisymmetric case
equation~(\ref{RTEdiscrete2}) becomes
a little bit more complicated, as the cell face normal vectors $\bn_m$
are not always constant. However, the computational efficiency can still be
retained.},
the normal vectors $\bn_m$ are the base
vectors of the coordinate system and the integrals over the solid
angle do not depend on the physical coordinate, but the direction
only. The intensities on the cell boundaries, $I_m^l$, are calculated
using a first-order upwind scheme.  If the physical space is swept in
the direction $\bs^l$, the intensity $I_{ijk}^l$ can be directly obtained
from an algebraic equation. This makes the numerical solution of the
FVM very fast.  Iterations are needed only to account for the
reflective boundaries. However, this is seldom necessary in
practice, because the time step set by the flow solver is small.


The cell face intensities, $I_m^l$ appearing on the left hand side of
(\ref{RTEdiscrete2}) are calculated using a first order
upwind scheme. Consider, for example, a control angle having a
direction vector $\bs$. If the radiation is traveling in the positive
$x$-direction, i.e., $\bs\cdot {\bf i} \geq 0$, the intensity on the
upwind side, $I_{xu}^l$ is assumed to be
the intensity in the neighboring cell, $I_{i-1\,jk}^l$,
and the intensity on the downwind
side is the intensity in the cell itself $I_{ijk}^l$.

On a rectilinear grid, the normal vectors $\bn_m$ are the base vectors
of the coordinate system and the integrals over the solid angle can be
calculated analytically. Equation (\ref{RTEdiscrete2}) can be simplified
\be
  a_{ijk}^l I_{ijk}^l =
  a_{x}^l   I_{xu}^l +
  a_{y}^l   I_{yu}^l +
  a_{z}^l   I_{zu}^l +   b_{ijk}^l \label{RTEdiscrete3}
\ee
where
\begin{eqnarray}
  a_{ijk}^l & = & A_x |D_x^l| + A_y |D_y^l| +A_z |D_z^l| +
        \kappa_{ijk} \; V_{ijk} \delta\Omega^l \\
  \nonumber \\
  a_{x}^l & = & A_x |D_x^l| \\
  a_{y}^l & = & A_y |D_y^l| \\
  a_{z}^l & = & A_z |D_z^l| 
\end{eqnarray}

\begin{eqnarray}
   D_x^l & = & \int_{\Omega^l}(\bs^l\cdot {\bf i}) \, \d\Omega = \int_{\dph}\int_{\dth} (\bs^l\cdot{\bf i}) \sin\theta \; \d \theta \d\phi = \int_{\dph}\int_{\dth} \cos\phi\sin\theta\sin\theta\; \d\theta \d\phi \nonumber \\
         & = & \frac{1}{2}\left( \sin\phi^+ - \sin\phi^- \right) \left[\Delta\theta - \left(\cos\theta^+\sin\theta^+ - \cos\theta^-\sin\theta^-\right)\right]  \\
   D_y^l & = & \int_{\Omega^l}(\bs^l\cdot {\bf j}) \, \d\Omega  =  \int_{\dph}\int_{\dth} \sin\phi\sin\theta\sin\theta\; \d\theta \d\phi \nonumber \\
         & = & \frac{1}{2}\left( \cos\phi^- - \cos\phi^+ \right) \left[\Delta\theta - \left(\cos\theta^+\sin\theta^+ - \cos\theta^-\sin\theta^-\right)\right]  \\
   D_z^l & = & \int_{\Omega^l}(\bs^l\cdot {\bf k}) \d \Omega = \int_{\dph}\int_{\dth} \cos\theta\sin\theta\;  \d\theta \d\phi \nonumber \\
         & = & \frac{1}{2}\Delta\phi \left[\left(\sin\theta^+\right)^2 - \left(\sin\theta^-\right)^2\right] 
\end{eqnarray}

\begin{eqnarray}
  b_{ijk}^l & = &  \kappa_{ijk} \; I_{b,ijk} \; V_{ijk} \; \delta\Omega^l \\
  \nonumber \\
  \delta \Omega^l & = & \int_{\Omega^l}d\Omega = \int_{\dph}\int_{\dth} \sin\theta \; \d\theta \d\phi
\end{eqnarray}

\noindent
Here $\bf i$, $\bf j$ and $\bf k$ are the base vectors of the
Cartesian coordinate system. $\theta^+$, $\theta^-$, $\phi^+$ and
$\phi^-$ are the upper and lower boundaries of the control angle in
the polar and azimuthal directions, respectively, and $\Delta\theta =
\theta^+ - \theta^-$ and $\Delta\phi = \phi^+ - \phi^-$. In the cells
next to a wall, the areas $A_m$ of the faces, that are perpendicular to
the wall, are multiplied by 0.5.

The solution method of (\ref{RTEdiscrete3}) is based on an explicit
marching sequence~\cite{Kim}. The marching direction depends on the
propagation direction of the radiation intensity. As the marching is
done in the ``downwind'' direction, the ``upwind'' intensities in all
three spatial directions are known, and the intensity $I_{ijk}^l$ can
be solved directly. Iterations may be needed only with the reflective
walls and optically thick situations.  Currently, no iterations are
made.

\subsection{Boundary Conditions}

The boundary condition for the radiation intensity leaving
a gray diffuse wall is given as
\be I_w(\bs) = \frac{\epsilon \, \sigma \, T_w^4}{\pi} + \frac{1-\epsilon}{\pi}
 \int_{\bs'\cdot \bn_w < 0} I_w(\bs')\; |\bs'\cdot \bn_w | \; \d\bs'
 \label{RTEbc} \ee
where $I_w(\bs)$ is the intensity at the wall, $\epsilon$ is the
emissivity, and $T_{w}$ is the wall surface temperature.
In discretized form, the boundary condition on a solid wall is given as
\be I_w^l = \frac{\epsilon \, \sigma \, T_w^4}{\pi} + \frac{1-\epsilon}{\pi} \sum_{D_w^{l'}<0} I_w^{l'}\; |D_w^{l'} |  \ee
where $D_w^{l'}= \int_{\Omega^{l'}}(\bs\cdot \bn_w)\d\Omega$.
The constraint $D_w^{l'}<0$ means that only the ``incoming'' directions
are taken into account when calculating the reflection.
The {\em net} radiative heat flux on the wall is
\be \dq_r'' = \sum_{l=1}^{N_{\Omega}} I_w^l \int_{\delta \Omega^l} (\bs' \cdot \bn_w) \, \d\bs'
     = \sum_{l=1}^{N_{\Omega}} I_w^l D_n^l \label{qrdef} \ee
where the coefficients $D_n^l$ are equal to $\pm D_x^l$, $\pm D_y^l$ or
$\pm D_z^l$, and can be calculated for each wall element at the start of the
calculation.

The open boundaries are treated as black walls, where the incoming intensity is
the black body intensity of the ambient temperature. On mirror
boundaries the intensities leaving the wall
are calculated from the incoming intensities using a
predefined connection matrix:
\be  I_{w,ijk}^l = I^{l'} \ee
Computationally intensive integration over all the incoming directions
is avoided by keeping the solid angle discretization symmetric on the $x$, $y$ and $z$ planes.
The connection matrix associates one incoming direction $l'$ to each mirrored direction on each wall cell.


\section{Absorption and Scattering of Thermal Radiation by Droplets/Particles}
\label{droplet-radiation}

The attenuation of thermal radiation by liquid droplets and particles is an important consideration, especially for water mist
systems~\cite{Ravigururajan:1}.  Droplets and particles attenuate thermal radiation through a combination of scattering and
absorption~\cite{Tuntomo:1}.  The radiation-spray interaction must therefore be solved for both the accurate prediction of the radiation
field and for the particle energy balance.

If the gas phase absorption and emission in Eq.~(\ref{RTEbasic}) are temporarily neglected for simplicity, the radiative transport
equation becomes
\be \bs \cdot \nabla I_{\la}(\bx,\bs) = -\left[\kappa_d(\bx,\la) + \sigma_d(\bx,\la)\right]
I_{\la}(\bx,\bs) +\kappa_d(\bx,\la) \; I_{b,d}(\bx,\la) +
\frac{\sigma_d(\bx,\la)}{4\pi}
\int_{4\pi}\Phi(\bs',\bs) \; I_{\la}(\bx,\bs') \; \d\bs'
\label{RTEspray} \ee
where $\kappa_d$ is the particle absorption coefficient, $\sigma_d$ is the
particle scattering coefficient and $I_{b,d}$ is the emission
term of the particles. $\Phi(\bs',\bs)$ is a scattering phase function
that gives the scattered intensity fraction from direction $\bs'$ to $\bs$.\\

\subsection{Absorption and scattering coefficients}

The capability of an individual droplet or particle to absorb and scatter radiation depends on its cross sectional area and radiative material 
properties. For simplicity, we next assume that the particles are spherical in shape, in which case the cross sectional area of a particle
is $\pi r^2$, where $r$ is the particle radius. If the local number density distribution of droplets or particles at location $\bx$ is 
denoted by $n(r(\bx))$, the local absorption and scattering coefficients within a spray/particle cloud can be calculated
from 
\begin{align}
\kappa_d(\bx,\la) &= \int_0^\infty n(r(\bx)) \; Q_a(r,\la) \; \pi r^2 \; \d r \\ %N(\bx)\int_0^\infty f(r,d_m(\bx)) \; C_a(r,\la) \; dr \\
\sigma_d(\bx,\la) &= \int_0^\infty n(r(\bx)) \; Q_s(r,\la) \; \pi r^2 \; \d r    %N(\bx)\int_0^\infty f(r,d_m(\bx)) \; C_s(r,\la) \; dr
\end{align}
where $Q_a$ and $Q_s$ are the absorption and scattering efficiencies, respectively.  
For the computation of the spray/particle cloud radiative properties,
spherical particles are assumed and the radiative properties of the individual particles
are computed using Mie theory.

Based on \cite{Collin} and \cite{Maruyama}, the real particle size distribution inside a grid cell is modeled as a mono-disperse suspension
whose diameter corresponds to the Sauter mean ($d_{32}$) diameter of the poly-disperse spray.
This assumption leads to a simplified expression of the radiative coefficients
\begin{align}
\kappa_d(\bx,\la) &= A_d(\bx) \; Q_a(r_{32},\la) \\
\sigma_d(\bx,\la) &= A_d(\bx) \; Q_s(r_{32},\la)
\end{align}
These expressions are functions of the total cross sectional area per unit volume of the droplets $A_d$. $A_d$ is computed simply by summing the cross sectional areas of all the droplets within a cell and divided by the cell volume. For practical reasons, a relaxation factor of 0.5 is
used to smooth slightly the temporal variation of $A_d$.

\subsection{In-scattering term}

An accurate computation of the in-scattering integral on the right
hand side of Eq.~(\ref{RTEspray}) would be extremely time
consuming. It is here approximated by dividing the total $4\pi$ solid
angle to a ``forward angle'' $\delta\Omega^l$ and ``ambient angle''
$\delta\Omega^*=4\pi - \delta\Omega^l$.  For compatibility with the
FVM solver, $\delta\Omega^l$ is set equal to the control angle given
by the angular discretization.  However, it is assumed to be symmetric
around the center of the control angle.  Within $\delta\Omega^l$ the
intensity is $I_{\la}(\bx,\bs)$ and elsewhere it is approximated as
\be
U^*(\bx,\la) = \frac{U(\bx,\la) - \delta\Omega^l \, I_{\la}(\bx,\bs)}{\delta\Omega^*}
\ee
where $U(\bx)$ is the total intensity integrated over the unit sphere. The in-scattering
integral can now be written as
\begin{align}
\frac{\sigma_d(\bx,\la)}{4\pi}\int_{4\pi}\Phi(\bs,\bs') \; I_{\la}(\bx,\bs')
  \; \d\Omega'
  &=
\sigma_d(\bx,\la)~ \left( \chi_f I_{\la}(\bx,\bs) ~+~\frac{1}{\delta\Omega^{*}}~\left(1-\chi_f\right)~
\int\limits_{\delta\Omega^{*}} I_{\la}(\bx,\bs') \; \textrm{d} \Omega' \right)\\
  &=
\sigma_d(\bx,\la)~ \left( \chi_f \; I_{\la}(\bx,\bs) ~+~
(1-\chi_f)U^*(\bx,\la) \right)
\end{align}
where $\chi_f = \chi_f(r,\la)$ is a fraction of the total intensity
originally within the solid angle $\delta\Omega^l$ that is scattered
into the same angle $\delta\Omega^l$. The calculation of $chi_f$ is discussed in section~\ref{forward_fraction}.

An effective scattering coefficient is defined
\be
\overline{\sigma_d}(\bx,\la) = A_d(\bx) \frac{4\pi}{4\pi-\delta\Omega^l} (1-\chi_f) \; Q_s(r_{32},\la) 
\ee
gives
%SIMO is going HERE
\be
\bs \cdot \nabla I_{\la}(\bx,\bs) =
-\left[\kappa_d(\bx,\la) + \overline{\sigma_d}(\bx,\la)\right]
I_{\la}(\bx,\bs)
+\kappa_d(\bx,\la) \; I_{b,d}(\bx,\la)
+\frac{\overline{\sigma_d}(\bx,\la)}{4\pi}U(\bx,\la)
\ee
This equation can be integrated over the spectrum to get the band
specific RTE's. The procedure is exactly the same as that used for the
gas phase RTE. After the band integrations, the spray RTE for band $n$
becomes
\be
\bs \cdot \nabla I_{n}(\bx,\bs) =
-\left[\kappa_{d,n}(\bx) + \overline{\sigma_{d,n}}(\bx)\right] I_n(\bx,\bs)
+\kappa_{d,n}(\bx) \; I_{b,d,n}(\bx)
+\frac{\overline{\sigma_d}(\bx,\la)}{4\pi}U_n(\bx)
\ee
where the source function is based on the average particle temperature within a cell.

Before the actual simulation, both $\kappa_d$ and $\overline{\sigma_d}$ are averaged over the
infrared wavelengths (1-200 $\mu$m).  A constant ``radiation'' temperature, $T_{\rm rad}$, is used
in the wavelength weighting.  $T_{\rm rad}$ should be selected to
represent a typical radiating flame temperature. A value of 1173~K is
used by default.  The averaged quantities, now functions of the
droplet mean diameter only, are stored in one-dimensional arrays.
During the simulation, the local properties are found by table
look-up using the local Sauter mean droplet diameter.

\subsection{Forward fraction of scattering}
\label{forward_fraction}

The computation of $\chi_f$ for a similar but simpler situation has been derived in Ref.~\cite{Yang:3}. It can be shown that here
$\chi_f$ becomes
\be
\chi_f = \frac{1}{\delta\Omega^l}
\int_0^{\mu^l}\int_0^{\mu^l}\int_{\mu_{d,0}}^{\mu_{d,\pi}}
\frac{P_0(\theta_d)}
{\left[(1-\mu^2)(1-\mu'^2)-(\mu_d-\mu\mu')^2\right]^{1/2}}
\; \d\mu_d \, \d\mu \, \d\mu'
\ee
where $\mu_d$ is a cosine of the scattering angle $\theta_d$ and
$P_0(\theta_d)$ is a single droplet scattering phase function
\be
P_0(\theta_d) =
\frac{\la^2\left(|S_1(\theta_d)|^2+|S_2(\theta_d)|^2\right)}{2 \, C_s(r,\la)}
\ee
$S_1(\theta_d)$ and $S_2(\theta_d)$ are the scattering amplitudes, given by
Mie-theory. The integration limit $\mu^l$ is a cosine of the polar angle
defining the boundary of the symmetric control angle $\delta\Omega^l$
\be
\mu^l = \cos(\theta^l) = 1 - \frac{2}{N_\Omega}
\ee
The limits of the innermost integral are
\be
\mu_{d,0}   = \mu\mu' + \sqrt{1-\mu^2}\sqrt{1-\mu'^2}  \quad ; \quad
\mu_{d,\pi} = \mu\mu' - \sqrt{1-\mu^2}\sqrt{1-\mu'^2}
\ee
When $\chi_f$ is integrated over the particle size distribution to get
an averaged value, it is multiplied by $C_s(r,\la)$. It is therefore
$|S_1|^2+|S_2|^2$, not $P_0(\theta_d)$, that is integrated. Physically,
this means that intensities are added, not
probabilities~\cite{Wiscombe}.\\


The absorption and scattering efficiencies and the scattering phase
function are calculated using the MieV code developed by Wiscombe~\cite{Wiscombe}.
The spectral properties of the particles can be specified in terms of wavelength dependent complex refractive index.
Pre-compiled data are included for water and a generic hydrocarbon fuel based on diesel and heptane properties.
In case of water, the values of the imaginary part of the complex refractive index (related to absorption coefficient) are taken from
Ref.~\cite{Hale:1}, and value 1.33 is used for the real part.
For fuel, the droplet spectral properties are taken from~\cite{Dombrovsky:1}, who measured the refractive index
of a diesel fuel and compared the values with those of heptane. The diesel properties are used for the real part of the refractive index. For the complex
part, the heptane properties are used to avoid the uncertainty associated with different types of diesel oils. Usually, the radiative properties
of the particle cloud are much more sensitive to the particle size and concentration than to values of the refractive index.




\subsection{Heat absorbed by droplets}

The droplet contribution to the radiative
loss term is
\be -\nabla\!\cdot \dbq_r''(\bx)(\mbox{droplets}) =
    \kappa_d(\bx) \, \left[ U(\bx) - 4 \pi \, I_{b,d}(\bx) \right]
\ee
For each individual droplet, the radiative heating/cooling power is
computed as
\be
\dq_r = \frac{m_d}{\rho_d(\bx)}
    \kappa_d(\bx) \, \left[ U(\bx) - 4\pi \, I_{b,d}(\bx) \right]
\ee
where $m_d$ is the mass of the droplet and $\rho_d(\bx)$ is the total
density of droplets in the cell.\\





