% !TEX root = Correlation_Guide.tex

\chapter{Smoke Detector Activation Time}
\label{Smoke_Detector_Activation_Time_Chapter}

Smoke detectors can be modeled in a variety of ways. A common method is to assume that the detector behaves like a very sensitive sprinkler with a low activation temperature and RTI. Other empirical correlations have been developed to predict the detector activation time as a function of the HRR and the location of the smoke detector relative to the fire.

\section{Temperature Rise Method}

\subsection*{Description}

In this method, the prediction of smoke detector activation time is identical to that for a sprinkler (as described in Chapter~\ref{Sprinkler_Activation_Time_Chapter}). Bukowski and Averill~\cite{Bukowski:2} suggest an activation temperature that corresponds to a temperature rise above ambient of 5~$^\circ$C to be typical of many residential smoke alarms. It is assumed that the smoke detectors are low-RTI devices ($\textrm{RTI}=5~\si{(m.s)^{1/2}}$).

Note that some of these cases assume a quasi-steady approach for a fire source $\dot Q$ that follows a specified t-squared growth, which was specified as $\dot Q = \alpha t^2$ up to a cutoff time of $t\sb{fire}$. After the time $t\sb{fire}$, the fire HRR was steady.


\clearpage


\subsection*{Verification}

This example case is based on Test SDC02 from the NIST Smoke Alarms series. This test involved ionization and photoelectric smoke alarms located in a single-story manufactured home with closed door, an upholstered chair fuel source, and no ventilation.

\begin{table}[!ht]
\caption[Verification case, smoke detector activation time]
{Verification case, smoke detector activation time.}
\begin{center}
\begin{tabular}{|c|c|c|}
\hline
\multicolumn{3}{|c|}{}                                                                  \\
\multicolumn{3}{|c|}{User-Specified Input}                                              \\
\multicolumn{3}{|c|}{}                                                                  \\ \hline
\multicolumn{2}{|c|}{}                              &  \multicolumn{1}{c|}{}            \\
\multicolumn{2}{|l|}{\rb{Parameter}}                &  \multicolumn{1}{c|}{\rb{Value}}  \\ \hline \hline
\multicolumn{2}{|l|}{$\alpha$ (kW/s$^2$)}           &  \multicolumn{1}{c|}{0.00463}     \\ \hline
\multicolumn{2}{|l|}{Location Factor (-)}           &  \multicolumn{1}{c|}{1}           \\ \hline
\multicolumn{2}{|l|}{$t\sb{fire}$ (s)}              &  \multicolumn{1}{c|}{300}         \\ \hline
\multicolumn{2}{|l|}{$r$ (m)}                       &  \multicolumn{1}{c|}{1.3}         \\ \hline
\multicolumn{2}{|l|}{$H$ (m)}                       &  \multicolumn{1}{c|}{2.1}         \\ \hline
\multicolumn{2}{|l|}{$\Delta T\sb{c}$ ($^\circ$C)}  &  \multicolumn{1}{c|}{5}           \\ \hline
\multicolumn{2}{|l|}{RTI (\si{(m.s)^{1/2}})}        &  \multicolumn{1}{c|}{5}           \\ \hline
\multicolumn{2}{|l|}{$T_\infty$ ($^\circ$C)}        &  \multicolumn{1}{c|}{21}          \\ \hline
\multicolumn{2}{c}{}                                                                    \\ \hline
\multicolumn{3}{|c|}{}                                                                  \\
\multicolumn{3}{|c|}{Expected Output}                                                   \\
\multicolumn{3}{|c|}{}                                                                  \\ \hline
           &             &                                                              \\
\rb{Time}  &  \rb{HRR}   &  \rb{Activation Time}                                        \\
\rb{(s)}   &  \rb{(kW)}  &  \rb{(s)}                                                    \\ \hline \hline
28         &  3.63       &  35.0                                                        \\ \hline
\end{tabular}
\end{center}
\end{table}

% Note that the FDTs spreadsheet uses incorrect coefficients in Alpert's ceiling jet correlation, which gives an activation time of 35.12 s.


\clearpage


\subsection*{Validation}

A summary of the comparisons between predicted and measured smoke detector activation times is shown in Fig.~\ref{Smoke_Detector_Activation_Summary_Temperature_Rise}.

\begin{figure}[!ht]
\begin{center}
\begin{tabular}{l}
\includegraphics[width=4.0in]{SCRIPT_FIGURES/Scatterplots/Smoke_Detector_Activation_Time_Temperature_Rise}
\end{tabular}
\end{center}
\caption[Summary of smoke detector activation time predictions (Temperature Rise)]
{Summary of smoke detector activation time predictions using the Temperature Rise method.}
\label{Smoke_Detector_Activation_Summary_Temperature_Rise}
\end{figure}


\clearpage


\section{Milke Method}

\subsection*{Description}

The correlation of Milke~\cite{Milke:1} predicts that the time of smoke detector activation, $t\sb{act}$, is given by
\be
t\sb{act} = \frac{X H^{4/3}}{\dot Q^{1/3}}
\label{eq:Milke}
\ee
where $H$ is the height of the ceiling above the top of the fuel~(\si{ft}), and $\dot Q$ is the HRR~(\si{kW}). The constant $X$ is given by
\be
X = 4.6 \times 10^{-4} (Y^2) + 2.7 \times 10^{-15} (Y^6)
\label{eq:Milke_X}
\ee
and the constant $Y$ is given by
\be
Y = \frac{\Delta T\sb{c} H^{5/3}}{\dot Q^{2/3}}
\label{eq:Milke_Y}
\ee
where $\Delta T\sb{c}$ is the temperature rise of gases under the ceiling required for the smoke detector to activate~($^\circ$F).

Note that some of these cases assume a quasi-steady approach for a fire source $\dot Q$ that follows a specified t-squared growth, which was specified as $\dot Q = \alpha t^2$ up to a cutoff time of $t\sb{fire}$. After the time $t\sb{fire}$, the fire HRR was steady.

\subsection*{Verification}

This example case is based on Test SDC02 from the NIST Smoke Alarms series. This test involved ionization and photoelectric smoke alarms located in a single-story manufactured home with closed door, an upholstered chair fuel source, and no ventilation.

\begin{table}[!ht]
\caption[Verification case, smoke detector activation time]
{Verification case, smoke detector activation time.}
\begin{center}
\begin{tabular}{|c|c|c|}
\hline
\multicolumn{3}{|c|}{}                                                                  \\
\multicolumn{3}{|c|}{User-Specified Input}                                              \\
\multicolumn{3}{|c|}{}                                                                  \\ \hline
\multicolumn{2}{|c|}{}                              &  \multicolumn{1}{c|}{}            \\
\multicolumn{2}{|l|}{\rb{Parameter}}                &  \multicolumn{1}{c|}{\rb{Value}}  \\ \hline \hline
\multicolumn{2}{|l|}{$\alpha$ (kW/s$^2$)}           &  \multicolumn{1}{c|}{0.00463}     \\ \hline
\multicolumn{2}{|l|}{$t\sb{fire}$ (s)}              &  \multicolumn{1}{c|}{300}         \\ \hline
\multicolumn{2}{|l|}{$H$ (m)}                       &  \multicolumn{1}{c|}{2.1}         \\ \hline
\multicolumn{2}{|l|}{$\Delta T\sb{c}$ ($^\circ$C)}  &  \multicolumn{1}{c|}{5}           \\ \hline
\multicolumn{2}{c}{}                                                                    \\ \hline
\multicolumn{3}{|c|}{}                                                                  \\
\multicolumn{3}{|c|}{Expected Output}                                                   \\
\multicolumn{3}{|c|}{}                                                                  \\ \hline
           &             &                                                              \\
\rb{Time}  &  \rb{HRR}   &  \rb{Activation Time}                                        \\
\rb{(s)}   &  \rb{(kW)}  &  \rb{(s)}                                                    \\ \hline \hline
30         &  4.2        &  30.8                                                        \\ \hline
\end{tabular}
\end{center}
\end{table}

% Note: The FDTs spreadsheets use an incorrect unit conversion factor from kW to Btu/s, which gives an activation time of 25.4~s.


\clearpage


\subsection*{Validation}

A summary of the comparisons between predicted and measured smoke detector activation times is shown in Fig.~\ref{Smoke_Detector_Activation_Summary_Milke}.

\begin{figure}[!ht]
\begin{center}
\begin{tabular}{l}
\includegraphics[width=4.0in]{SCRIPT_FIGURES/Scatterplots/Smoke_Detector_Activation_Time_Milke}
\end{tabular}
\end{center}
\caption[Summary of smoke detector activation time predictions (Milke)]
{Summary of smoke detector activation time predictions using the Milke method.}
\label{Smoke_Detector_Activation_Summary_Milke}
\end{figure}


\clearpage


\section{Mowrer Method}

\subsection*{Description}

The correlation of Mowrer~\cite{Mowrer:1} predicts that the time of smoke detector activation, $t\sb{act}$, is given by
\be
t\sb{act} = t\sb{pl} + t\sb{cj}
\label{eq:Mowrer}
\ee
where the transport lag time of the plume, $t\sb{pl}$, is given by
\be
t\sb{pl} = C\sb{pl} \frac{H^{4/3}}{\dot Q^{1/3}}
\label{eq:Mowrer_tpl}
\ee
where $C\sb{pl}$ is the plume lag time constant~(0.67), $H$ is the height of the ceiling above the fuel~(\si{m}), and $\dot Q$ is the HRR~(\si{kW}).
The transport lag time of the ceiling jet, $t\sb{cj}$, is given by
\be
t\sb{cj} = \frac{1}{C\sb{cj}} \frac{r^{11/6}}{\dot Q^{1/3} H^{1/2}}
\label{eq:Mowrer_tcj}
\ee
where $C\sb{cj}$ is the ceiling jet time lag time constant~(1.2), and $r$ is the radial distance to the detector~(\si{m}).

Note that some of these cases assume a quasi-steady approach for a fire source $\dot Q$ that follows a specified t-squared growth, which was specified as $\dot Q = \alpha t^2$ up to a cutoff time of $t\sb{fire}$. After the time $t\sb{fire}$, the fire HRR was steady.


\clearpage


\subsection*{Verification}

This example case is based on Test SDC02 from the NIST Smoke Alarms series. This test involved ionization and photoelectric smoke alarms located in a single-story manufactured home with closed door, an upholstered chair fuel source, and no ventilation.

\begin{table}[!ht]
\caption[Verification case, smoke detector activation time]
{Verification case, smoke detector activation time.}
\begin{center}
\begin{tabular}{|c|c|c|}
\hline
\multicolumn{3}{|c|}{}                                                         \\
\multicolumn{3}{|c|}{User-Specified Input}                                     \\
\multicolumn{3}{|c|}{}                                                         \\ \hline
\multicolumn{2}{|c|}{}                     &  \multicolumn{1}{c|}{}            \\
\multicolumn{2}{|l|}{\rb{Parameter}}       &  \multicolumn{1}{c|}{\rb{Value}}  \\ \hline \hline
\multicolumn{2}{|l|}{$\alpha$ (kW/s$^2$)}  &  \multicolumn{1}{c|}{0.00463}     \\ \hline
\multicolumn{2}{|l|}{$t\sb{fire}$ (s)}     &  \multicolumn{1}{c|}{300}         \\ \hline
\multicolumn{2}{|l|}{$C\sb{pl}$ (-)}       &  \multicolumn{1}{c|}{0.67}        \\ \hline
\multicolumn{2}{|l|}{$C\sb{cj}$ (-)}       &  \multicolumn{1}{c|}{1.2}         \\ \hline
\multicolumn{2}{|l|}{$r$ (m)}              &  \multicolumn{1}{c|}{1.3}         \\ \hline
\multicolumn{2}{|l|}{$H$ (m)}              &  \multicolumn{1}{c|}{2.1}         \\ \hline
\multicolumn{2}{c}{}                                                           \\ \hline
\multicolumn{3}{|c|}{}                                                         \\
\multicolumn{3}{|c|}{Expected Output}                                          \\
\multicolumn{3}{|c|}{}                                                         \\ \hline
           &             &                                                     \\
\rb{Time}  &  \rb{HRR}   &  \rb{Activation Time}                               \\
\rb{(s)}   &  \rb{(kW)}  &  \rb{(s)}                                           \\ \hline \hline
5          &  0.116      &  5.6                                                \\ \hline
\end{tabular}
\end{center}
\end{table}


\clearpage


\subsection*{Validation}

A summary of the comparisons between predicted and measured smoke detector activation times is shown in Fig.~\ref{Smoke_Detector_Activation_Summary_Mowrer}.

\begin{figure}[!ht]
\begin{center}
\begin{tabular}{l}
\includegraphics[width=4.0in]{SCRIPT_FIGURES/Scatterplots/Smoke_Detector_Activation_Time_Mowrer}
\end{tabular}
\end{center}
\caption[Summary of smoke detector activation time predictions (Mowrer)]
{Summary of smoke detector activation time predictions using the Mowrer method (bottom).}
\label{Smoke_Detector_Activation_Summary_Mowrer}
\end{figure}