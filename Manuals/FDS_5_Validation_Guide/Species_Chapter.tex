\chapter{Gas Species and Smoke}


\section{NIST/WTC Test Series, Oxygen and CO$_2$}




\section{NIST/NRC Test Series, Oxygen and CO$_2$}

The following pages present comparisons of oxygen and carbon dioxide concentration predictions with measurement for the
NIST/NRC series. There were two oxygen measurements, one in the upper layer, one in the lower.  There was only one carbon
dioxide measurement in the upper layer.




\section{NIST/NRC Test Series, Smoke}

FDS treats smoke like all other combustion products, basically a tracer gas whose mass fraction is a function of the mixture fraction.
To model smoke movement, the user need only prescribe the smoke yield, that is, the fraction of the fuel mass that is
converted to smoke particulate.  For the simulations of the NIST/NRC tests, the smoke yield is specified as one of the test parameters.
Figure and Figure contain comparisons of measured and predicted smoke concentration at one measuring station in the upper layer.
There are two obvious trends in the figures: first, the predicted concentrations are about 50 \% higher than the measured
in the open door tests.  Second,
the predicted concentrations are roughly three times the measured concentrations in the closed door tests.
As a contrast, Figure displays the time history of CO concentration for 6 of the NIST/NRC tests.
Like smoke, the CO is specified in FDS via a fixed yield, measured along with smoke and reported in the test document.
The large differences between model and measurement seen in the smoke data do not appear in the CO data.

\begin{figure}[p]
\begin{center}
\begin{tabular}{c}
\includegraphics[width=4.0in]{FIGURES/ScatterPlots/Carbon_Dioxide_Concentration} \\
\vspace{0.25in} \\
\includegraphics[width=4.0in]{FIGURES/ScatterPlots/Oxygen_Concentration}\\
\vspace{0.25in}
\end{tabular}
\caption{Summary of Gas Species Results.}
\end{center}
\end{figure}



\clearpage

\section{NIST Diffusion Flame Test Series}

The CO production model was used to simulate the methane/air slot burner flame.  The Fig.~\ref{WP_CH4_Temp} shows predicted and
measured temperatures at three elevations above the burner.  The model predicts a flame that is slightly narrower
(5.5 mm vs. 6.5 mm or a 15 \% error) and cooler (1700 �C vs. 1800 �C or a 5 \% error) than measured.  The model
also predicts higher centerline temperatures.  These results are not surprising.  The new combustion model considers
the first step, Fuel to CO, to be infinitely fast, assuming that the local oxygen concentration satisfies the
flammability criteria.  This is true in the vicinity of the lip of the burner.  In reality, the cold fuel and air
streams do not react infinitely fast there and some oxygen penetrates the flame at the base, resulting in cooler
gases being entrained into the core of the flame with a resulting drop in the centerline temperature.

Fig.~\ref{WP_CH4_Species} shows predicted and
measured values of $CH_4$ + $O_2$, CO,  $CO_2$ at three elevations above the burner along with a contour plot of CO near the
burner lip (black rectangle is burner).  Note,as discussed above, that the test data shows that a small quantity of oxygen exists along
the burner centerline which is not echoed in the predictions.  Along the centerline, the model predicts higher values of fuel and higher values of
products than measured.  The species profiles are also slightly narrower than measured, echoing the temperature plot \
in Fig.~\ref{WP_CH4_temp}.  The peak $CH_4$ values along the burner centerline have errors ranging from 6 \% to 13 \%,
the peak $CO_2$ values in the flame have errors ranging from 3 \% to 10 \%, and the peak CO values in the flame have
errors ranging from 9 \% to 25 \%.  Given the reported 10 \% to 20 \% species measurement uncertainty, the new
combustion model predicts well both the magnitude and shape of the mixture fraction profiles.

The contour of CO illustrates the combustion model quite well.  At the burner lip, fuel and oxygen first meet.
Since ambient levels of oxygen exist on the air side of the slot and the temperatures are relatively low at the
burner lip, combustion occurs and converts $Z_1$ to $Z_2$.  This consumes the oxygen present and the CO does not have an
opportunity to fully convert to $CO_2$.  As the CO rises in the flame, oxygen is entrained and the temperatures are high
enough that the CO reacts with the entrained oxygen to form $CO_2$.  Eventually, at a point high enough in the flame,
all of the fuel is consumed and all of the CO is oxidized.

\begin{figure}
\includegraphics[width=6in]{FIGURES/WP/WP_CH4_Temp}
\caption{Predicted and measured temperature at three elevations (7 mm, 9mm and 11 mm) above a methane-air slot burner}
\label{WP_CH4_Temp}
\end{figure}

\begin{figure}
\includegraphics[width=6in]{FIGURES/WP/WP_CH4_species}
\caption{Predicted and measured mole fractions at three elevations (7 mm, 9mm and 11 mm) above a methane-air slot burner,
along with contours of CO near the burner lip}
\label{WP_CH4_Species}
\end{figure}

\clearpage

\section{Beyler Hood Experiments}

Fig.~\ref{Beyler_Species} shows species predictions made by the two step model compared with measured data for a
range of fire sizes and burner positions.  The dotted lines indicate the estimated measurement uncertainty.  The
model predicts the time-averaged species concentration at the hood exhaust vent.  $CO_2$ predictions are within the
measurement uncertainty for all but one of the simulations performed.  For the well-ventilated fires (-10 cm), CO,
$CO_2$, and unburned fuel predictions match the data.  As the fires become under-ventilated, CO is over-predicted while
fuel and $O_2$ are under-predicted.  The most likely explanation for the discrepancy is that the model assumes fuel and
oxygen react infinitely fast in the vicinity of near ambient conditions.  This occurs at the lower edge of the hood
were the vitiated layer is adjacent to the ambient air below the hood, and as a result layer burning is occurring in
the model which depletes the fuel and O2 and creates CO.  This is not unexpected, and indicates that more work is
required to establish the conditions under which combustion in the first step, conversion of fuel to CO, will be
allowed.

\begin{figure}[p]
\includegraphics[width=\textwidth]{FIGURES/Beyler_Hood/Beyler_species}
\caption{Comparison of measured and predicted species concentrations in the Beyler hood experiments,
dotted lines show experimental uncertainty }
\label{Beyler_Species}
\end{figure}

\clearpage

\section{NIST Reduced Scale Compartment CO Production Test Series}

The RSE natural gas experiments were selected as currently the interest is in modeling CO production rather than
CO and soot production. Nine fire sizes were simulated: 50 kW, 75 kW, 100 kW, 150 kW, 200 kW, 300 kW, 400 kW,
500 kW, and 600 kW.  The tests were modeled using properties of the natural gas supplied to the test facility.
The model geometry included the compartment interior along with a 0.6 m deep region outside the door.
Fig.~\ref{RSE_Pitts_Species} shows the measured and predicted $CO_2$ and CO concentrations.  The measured values are
from the test series performed Bryner, Johnsson, and Pitts12.  The model matches the data up to a fire size of
300 kW, including the location of the peak $CO_2$ concentration at 200 kW.  For larger fires the model predicts more
CO surviving in the upper layer than measured, along with correspondingly lower $CO_2$ levels.  At 300 kW the
model predicts front and rear concentrations of 7.7 \% and 7.4 \% $CO_2$ vs. 7.5 \% and 7.7 \% in the data.  At
600 kW the respective CO2 values are 6.5 \% and 5.3 \% vs. 6.1 \% and 6.8 \% in the data.  For CO at 300 kW the
model predicts 2.2 \% and 2.4 \% vs. 1.8 \% and 2.0 \% in the data.  For CO at 600 kW the model predicts 3.2 \%
and 3.8 \% vs. 2.9 \% and 2.1 \% in the data.  As the compartment becomes under-ventilated, the model under-predicts
CO2 and over-predicts CO. The relative error increases as the compartment becomes more under-ventilated.  However,
note that the absolute model under-prediction of CO2 in the rear (1.5 \%) is equivalent to the absolute
over-prediction of CO (1.7 \%).  This implies that part of the model error results from the assumption of
infinitely fast chemistry for the first step.  Furthermore, since the model as implemented performs the two steps
sequentially, higher CO is predicted where the first step consumes all of the available oxygen.

\begin{figure}
\includegraphics[width=6in]{FIGURES/RSE_Pitts/RSE_Pitts_Species}
\caption{Predicted and measured CO2 and CO concentrations for the NIST RSE experiments of Bryner, Johnsson, and Pitts}
\label{RSE_Pitts_Species}
\end{figure}
