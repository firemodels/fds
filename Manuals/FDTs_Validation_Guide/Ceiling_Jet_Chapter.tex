% !TEX root = FDTs_Validation_Guide.tex

\chapter{Ceiling Jets and Device Activation}
\label{Ceiling_Jet_Chapter}

\section{Ceiling Jet Temperature}

\subsubsection*{Description}

For a steady-state fire, the correlation of Alpert~\cite{SFPE:Alpert} predicts that the ceiling jet temperature rise, $\Delta T_{jet}$, from a fire plume is given by
\be
\Delta T_{jet} = \left\{ \begin{array}{cl}
   \frac{16.9 \dot Q^{2/3}}{H^{5/3}}  &  r/H <= 0.18 \\[0.1in]
   \frac{5.38 (\dot Q / r)^{2/3}}{H}  &  r/H >  0.18 
   \end{array} \right. \quad ^\circ{\rm C}
\label{eq:Alpert_Tjet}
\ee
where $\dot Q$ is the total HRR~(\si{kW}), $H$ is the height of the ceiling above the fuel~(\si{m}), and $r$~is the radial distance to the detector~(\si{m}).

Note that some of these cases assume a quasi-steady approach for a fire source $\dot Q$ that follows a specified t-squared growth, which is given by
\be
\dot Q = \alpha t^2
\label{eq:t_squared}
\ee
where $\alpha$ is the t-squared growth parameter~(\si{kW/s^2}), and $t$ is time~(\si{s}).

For cases in which the fire was located against a wall or corner, these correlations are adjusted based on the method of reflection. For a fire adjacent to a flat wall, 2$\dot Q$ is substituted for $\dot Q$; and for a fire in a 90-degree corner, 4$\dot Q$ is substituted for $\dot Q$. This adjustment is denoted in the input parameters as the location factor. For a given case, the location factor has a value of 1, 2, or 4, which corresponds to a fire away from walls or corners, a fire adjacent to a wall, or a fire located in a corner, respectively.

It is important to note that this ceiling jet temperature correlation was developed using data from tests that were conducted in a large facility in which the distant walls and large compartment size did not allow for the development of a significant hot gas layer (HGL). In a more typical fire scenario (i.e., a smaller compartment), the HGL develops relatively quickly, and temperatures at the ceiling are affected by the ceiling jet as well as the accumulating HGL. Thus, when compared to experimentally measured ceiling temperatures in a compartment fire, this correlation tends to underpredict the temperatures because it is not accounting for the development of the HGL. This is an important consideration when using this correlation to predict detector or sprinkler activations.

For the reasons stated above, two scatter plot comparisons are shown in Fig.~\ref{Ceiling_Jet_Temperature_Summary}. One scatter plot shows underpredicted temperature comparisons for compartment fire tests. The other scatter plot shows results for unconfined tests that were conducted under a false ceiling in which the hot plume gases did not accumulate to form an HGL, but were allowed to spill out from under a false ceiling. In the unconfined cases, the ceiling temperature predictions are in better agreement with experimental data because this scenario is more representative of a temperature rise due to only ceiling jet flow from the fire plume rather than heat buildup from a HGL.

\subsubsection*{Verification}

Test: NIST/NRC Test 1

\begin{table}[!ht]
\caption[Input parameters, ceiling jet temperature (Alpert)]
{Input parameters, ceiling jet temperature (Alpert).}
\begin{center}
\begin{tabular}{|l|c|}
\hline
                          &              \\
\rb{Input Parameter}      &  \rb{Value}  \\ \hline \hline
$\dot Q$ (kW)             &  410         \\ \hline
Location Factor (-)       &  1           \\ \hline
$r$ (m)                   &  5.90        \\ \hline
$H$ (m)                   &  3.72        \\ \hline
$T_{\infty}$ ($^\circ$C)  &  22          \\ \hline
\end{tabular}
\end{center}
\end{table}

\noindent Expected result: The ceiling jet temperature $T_{jet}$ is 46.45~$^\circ$C.

\subsubsection*{Validation}

\begin{figure}[p]
\begin{center}
\begin{tabular}{l}
\includegraphics[width=4.0in]{SCRIPT_FIGURES/Scatterplots/Ceiling_Jet_Temperature_Compartment} \\
\includegraphics[width=4.0in]{SCRIPT_FIGURES/Scatterplots/Ceiling_Jet_Temperature_Unconfined}
\end{tabular}
\end{center}
\caption[Summary of ceiling jet temperature predictions]
{Summary of compartment (top) and unconfined (bottom) ceiling jet temperature predictions.}
\label{Ceiling_Jet_Temperature_Summary}
\end{figure}

\clearpage


\section{Sprinkler Activation Time}
\label{sec:sprinkler_alpert}

\subsubsection*{Description}

For a steady-state fire, the correlation of Alpert~\cite{SFPE:Alpert} predicts that the activation time of a sprinkler, $t_{activation}$, is given by (Budnick, Evans, and Nelson, 1997)
\be
t_{activation} =  \frac{RTI}{\sqrt{u_{jet}}} \ln \left( \frac{T_{jet} - T_\infty}{T_{jet} - T_{activation}} \right)\label{eq:Alpert}
\ee
where $RTI$ is the response time index of the sprinkler~(m-s)$^{1/2}$, $T_\infty$ is the ambient air temperature~(\si{\celsius}), and $T_{activation}$ is the activation temperature of the sprinkler~(\si{\celsius}). The ceiling jet temperature, $T_{jet}$ (\si{\celsius}), is given by
\be
T_{jet} = \left\{ \begin{array}{cl}
   \frac{16.9 \dot Q^{2/3}}{H^{5/3}} + T_\infty  &  r/H <= 0.18 \\[0.1in]
   \frac{5.38 (\dot Q / r)^{2/3}}{H} + T_\infty  &  r/H >  0.18
   \end{array} \right.
\label{eq:sprinkler_Tjet}
\ee
where $\dot Q$ is the total HRR~(kW), $H$ is the height of the ceiling above the fuel~(m), and $r$~is the radial distance to the detector~(m).
The ceiling jet velocity, $u_{jet}$ (\si{m/s}), is given by
\be
u_{jet} = \left\{ \begin{array}{cl}
   0.947 \left( \frac{\dot Q}{H} \right)^{1/3}  &  r/H <= 0.15 \\[0.1in]
   \frac{0.197 \dot Q^{1/3} H^{1/2}}{r^{5/6}}   &  r/H >  0.15
   \end{array} \right.
\label{eq:sprinkler_ujet}
\ee

Note that some of these cases assume a quasi-steady approach for a fire source $\dot Q$ that follows a specified t-squared growth, which is given by Eq.~\ref{eq:t_squared}.

For cases in which the fire was located against a wall or corner, these correlations are adjusted based on the method of reflection. For a fire adjacent to a flat wall, 2$\dot Q$ is substituted for $\dot Q$; and for a fire in a 90-degree corner, 4$\dot Q$ is substituted for $\dot Q$~\cite{SFPE:Alpert}. This adjustment is denoted in the input parameters as the location factor. For a given case, the location factor has a value of 1, 2, or 4, which corresponds to a fire away from walls or corners, a fire adjacent to a wall, or a fire located in a corner, respectively.

\subsubsection*{Verification}

Test: Vettori Flat Ceiling Test 1

\begin{table}[!ht]
\caption[Input parameters, sprinkler activation time]
{Input parameters, sprinkler activation time.}
\begin{center}
\begin{tabular}{|l|c|}
\hline
                              &              \\
\rb{Input Parameter}          &  \rb{Value}  \\ \hline \hline
$\alpha$ (kW/s$^2$)           &  0.105       \\ \hline
Location Factor (-)           &  1           \\ \hline
RTI (m-s)$^{1/2}$             &  55          \\ \hline
$T_{activation}$ ($^\circ$C)  &  68          \\ \hline
$r$ (m)                       &  2.20        \\ \hline
$H$ (m)                       &  2.09        \\ \hline
$T_\infty$ ($^\circ$C)        &  16.6        \\ \hline
\end{tabular}
\end{center}
\end{table}

\noindent Expected result: At 50~s, the HRR $\dot Q$ is 262.5~kW, and the activation time $t_{activation}$ is 98.74~s.


\clearpage


\subsubsection*{Validation}

\begin{figure}[!ht]
\begin{center}
\begin{tabular}{l}
\includegraphics[width=4.0in]{SCRIPT_FIGURES/Scatterplots/Sprinkler_Activation_Time}
\end{tabular}
\end{center}
\caption[Summary of sprinkler activation time predictions]
{Summary of sprinkler activation time predictions.}
\label{Sprinkler_Activation_Summary}
\end{figure}


\clearpage


\section{Smoke Detector Activation Time}

\subsection{Method of Alpert}

\subsubsection*{Description}

In this method, the prediction of smoke detector activation time is identical to that for a sprinkler (as described in Section~\ref{sec:sprinkler_alpert}). Heskestad and Delichatsios~\cite{Heskestad:4} correlated smoke detector activation to a smoke temperature change of 10~$^\circ$C (18~$^\circ$F) from typical fuels. It is assumed that the smoke detectors are low-RTI devices ($\textrm{RTI}=5$).

Note that some of these cases assume a quasi-steady approach for a fire source $\dot Q$ that follows a specified t-squared growth, which is given by Eq.~\ref{eq:t_squared}.

\subsubsection*{Verification}

Test: NIST Dunes 2000, Test SDC02

\begin{table}[!ht]
\caption[Input parameters, smoke detector activation time (Alpert)]
{Input parameters, smoke detector activation time (Alpert).}
\begin{center}
\begin{tabular}{|l|c|}
\hline
                          &              \\
\rb{Input Parameter}      &  \rb{Value}  \\ \hline \hline
Location Factor (-)       &  1           \\ \hline
$\alpha$ (kW/s$^2$)       &  0.00463     \\ \hline
$t_{fire}$ (s)            &  300         \\ \hline
$r$ (m)                   &  1.3         \\ \hline
$H$ (m)                   &  2.1         \\ \hline
$\Delta T_c$ ($^\circ$C)  &  10          \\ \hline
RTI (m-s)$^{1/2}$         &  5           \\ \hline
$T_\infty$ ($^\circ$C)    &  21          \\ \hline
\end{tabular}
\end{center}
\end{table}

\noindent Expected result: At 47~s, the HRR $\dot Q$ is 10.23~kW, and the activation time $t_{activation}$ is 30.8~s.


\clearpage


\subsubsection*{Validation}

\begin{figure}[!ht]
\begin{center}
\begin{tabular}{l}
\includegraphics[width=4.0in]{SCRIPT_FIGURES/Scatterplots/Smoke_Detector_Activation_Time_Alpert}
\end{tabular}
\end{center}
\caption[Summary of smoke detector activation time predictions]
{Summary of smoke detector activation time predictions using the method of Alpert.}
\label{Smoke_Detector_Activation_Summary_Alpert}
\end{figure}


\clearpage


\subsection{Method of Milke}

\subsubsection*{Description}

The correlation of Milke~\cite{Milke:1} predicts that the time of smoke detector activation, $t_{activation}$, is given by
\be
t_{activation} = \frac{X H^{4/3}}{\dot Q^{1/3}}
\label{eq:Milke}
\ee
where $H$ is the height of the ceiling above the top of the fuel~(\si{ft}), and $\dot Q$ is the HRR~(\si{kW}). The constant $X$ is given by
\be
X = 4.6 \times 10^{-4} (Y^2) + 2.7 \times 10^{-15} (Y^6)
\label{eq:Milke_X}
\ee
and the constant $Y$ is given by
\be
Y = \frac{\Delta T_c H^{5/3}}{\dot Q^{2/3}}
\label{eq:Milke_Y}
\ee
where $\Delta T_c$ is the temperature rise of gases under the ceiling required for the smoke detector to activate~($^\circ$F).

Note that some of these cases assume a quasi-steady approach for a fire source $\dot Q$ that follows a specified t-squared growth, which is given by Eq.~\ref{eq:t_squared}.

\subsubsection*{Verification}

Test: NIST Dunes 2000, Test SDC02

\begin{table}[!ht]
\caption[Input parameters, smoke detector activation time (Milke)]
{Input parameters, smoke detector activation time (Milke).}
\begin{center}
\begin{tabular}{|l|c|}
\hline
                          &              \\
\rb{Input Parameter}      &  \rb{Value}  \\ \hline \hline
$\alpha$ (kW/s$^2$)       &  0.00463     \\ \hline
$t_{fire}$ (s)            &  300         \\ \hline
$H$ (m)                   &  2.1         \\ \hline
$\Delta T_c$ ($^\circ$C)  &  10          \\ \hline
\end{tabular}
\end{center}
\end{table}

\noindent Expected result: At 40~s, the HRR $\dot Q$ is 7.41~kW, and the activation time $t_{activation}$ is 47.3~s.
\\ \\
Note: The FDTs spreadsheets use an incorrect unit conversion factor from kW to Btu/s, which gives an activation time of 39.5~s.


\clearpage


\subsubsection*{Validation}

\begin{figure}[!ht]
\begin{center}
\begin{tabular}{l}
\includegraphics[width=4.0in]{SCRIPT_FIGURES/Scatterplots/Smoke_Detector_Activation_Time_Milke}
\end{tabular}
\end{center}
\caption[Summary of smoke detector activation time predictions]
{Summary of smoke detector activation time predictions using the method of Milke.}
\label{Smoke_Detector_Activation_Summary_Milke}
\end{figure}


\clearpage


\subsection{Method of Mowrer}

\subsubsection*{Description}

The correlation of Mowrer~\cite{Mowrer:1} predicts that the time of smoke detector activation, $t_{activation}$, is given by
\be
t_{activation} = t_{pl} + t_{cj}
\label{eq:Mowrer}
\ee
where the transport lag time of the plume, $t_{pl}$, is given by
\be
t_{pl} = C_{pl} \frac{H^{4/3}}{\dot Q^{1/3}}
\label{eq:Mowrer_tpl}
\ee
where $C_{pl}$ is the plume lag time constant~(0.67), $H$ is the height of the ceiling above the fuel~(\si{m}), and $\dot Q$ is the HRR~(\si{kW}).
The transport lag time of the ceiling jet, $t_{cj}$, is given by
\be
t_{cj} = \frac{1}{C_{cj}} \frac{r^{11/6}}{\dot Q^{1/3} H^{1/2}}
\label{eq:Mowrer_tcj}
\ee
where $C_{cj}$ is the ceiling jet time lag time constant~(1.2), and $r$ is the radial distance to the detector~(\si{m}).

Note that some of these cases assume a quasi-steady approach for a fire source $\dot Q$ that follows a specified t-squared growth, which is given by Eq.~\ref{eq:t_squared}.

\subsubsection*{Verification}

Test: NIST Dunes 2000, Test SDC02

\begin{table}[!ht]
\caption[Input parameters, smoke detector activation time (Mowrer)]
{Input parameters, smoke detector activation time (Mowrer).}
\begin{center}
\begin{tabular}{|l|c|}
\hline
                      &              \\
\rb{Input Parameter}  &  \rb{Value}  \\ \hline \hline
$\alpha$ (kW/s$^2$)   &  0.00463     \\ \hline
$t_{fire}$ (s)        &  300         \\ \hline
$C_{pl}$ (-)          &  0.67        \\ \hline
$C_{cj}$ (-)          &  1.2         \\ \hline
$r$ (m)               &  1.3         \\ \hline
$H$ (m)               &  2.1         \\ \hline
\end{tabular}
\end{center}
\end{table}

\noindent Expected result: At 5~s, the HRR $\dot Q$ is 0.116~kW, and the activation time $t_{activation}$ is 5.6~s.


\clearpage


\subsubsection*{Validation}

\begin{figure}[!ht]
\begin{center}
\begin{tabular}{l}
\includegraphics[width=4.0in]{SCRIPT_FIGURES/Scatterplots/Smoke_Detector_Activation_Time_Mowrer}
\end{tabular}
\end{center}
\caption[Summary of smoke detector activation time predictions]
{Summary of smoke detector activation time predictions using the method of Mowrer (bottom).}
\label{Smoke_Detector_Activation_Summary_Mowrer}
\end{figure}