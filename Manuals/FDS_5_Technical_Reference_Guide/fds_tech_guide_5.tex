\documentclass[11pt]{book}
\usepackage{mathptm,times}
\usepackage[pdftex]{graphicx}
\usepackage{hyperref}
\usepackage{lastpage} % Automatic last page number reference.

%\usepackage{array,eqnarray}

%\usepackage{eso-pic}
%\usepackage{graphicx}
%\usepackage{color}
%\usepackage{type1cm}

%\makeatletter
%   \AddToShipoutPicture{%
%     \setlength{\@tempdimb}{.5\paperwidth}%
%    \setlength{\@tempdimc}{.5\paperheight}%
%   \setlength{\unitlength}{1pt}%
%  \put(\strip@pt\@tempdimb,\strip@pt\@tempdimc){%
%     \makebox(0,0){\rotatebox{45}{\textcolor[gray]{0.75}{\fontsize{8cm}{8cm}\selectfont{DRAFT}}}}}}
%\makeatother

\setlength{\textwidth}{6.5in}
\setlength{\textheight}{9.0in}
\setlength{\topmargin}{0.in}
\setlength{\headheight}{0.in}
\setlength{\headsep}{0.in}
\setlength{\parindent}{0.25in}
\setlength{\oddsidemargin}{0.0in}
\setlength{\evensidemargin}{0.0in}

\begin{document}

\bibliographystyle{unsrt}

\newcommand{\dod}[2]{\frac{\partial #1}{\partial #2}}
\newcommand{\DoD}[2]{\frac{D #1}{D #2}}
\newcommand{\dsods}[2]{\frac{\partial^2 #1}{\partial #2^2}}
\newcommand{\dx}{\delta x}
\newcommand{\dy}{\delta y}
\newcommand{\dz}{\delta z}
\newcommand{\x}{x}
\newcommand{\y}{y}
\newcommand{\z}{z}
\newcommand{\dt}{\delta t}
\newcommand{\dn}{\delta n}
\newcommand{\cH}{{\cal H}}
\newcommand{\hu}{u}
\newcommand{\hv}{v}
\newcommand{\hw}{w}
\newcommand{\la}{\lambda}
%\newcommand{\bO}{\mbox{\boldmath $\Omega$}}
\newcommand{\bO}{{\Omega}}
\newcommand{\bo}{{\bf \omega}}
%\newcommand{\btau}{\mbox{\boldmath $\tau$}}
\newcommand{\btau}{{\bf \tau}}
\newcommand{\bdelta}{{\bf \delta}}
\newcommand{\sumyw}{\sum (Y_\alpha/W_\alpha)}
\newcommand{\oW}{\overline{W}}
\newcommand{\om}{\omega}
\newcommand{\omx}{\omega_x}
\newcommand{\omy}{\omega_y}
\newcommand{\omz}{\omega_z}
\newcommand{\erf}{\hbox{erf}}
\newcommand{\bF}{{\bf F}}
\newcommand{\bof}{{\bf f}}
\newcommand{\bq}{{\bf q}}
\newcommand{\br}{{\bf r}}
\newcommand{\bu}{{\bf u}}
\newcommand{\bx}{{\bf x}}
\newcommand{\bk}{{\bf k}}
\newcommand{\bv}{{\bf v}}
\newcommand{\bg}{{\bf g}}
\newcommand{\bn}{{\bf n}}
\newcommand{\bS}{{\bf S}}
\newcommand{\bW}{\overline{W}}
\newcommand{\dS}{d{\bf S}}
\newcommand{\bs}{{\bf s}}
\newcommand{\bI}{{\bf I}}
\newcommand{\hp}{{\cal H}}
\newcommand{\trho}{\tilde{\rho}}
\newcommand{\dph}{{\delta\phi}}
\newcommand{\dth}{{\delta\theta}}
\newcommand{\tp}{\tilde{p}}
\newcommand{\bp}{\overline{p}}
\newcommand{\dQ}{\dot{Q}}
\newcommand{\dq}{\dot{q}}
\newcommand{\dbq}{\dot{\bf q}}
\newcommand{\dm}{\dot{m}}
\newcommand{\ha}{\frac{1}{2}}
\newcommand{\ft}{\frac{4}{3}}
\newcommand{\ot}{\frac{1}{3}}
\newcommand{\fofi}{\frac{4}{5}}
\newcommand{\of}{\frac{1}{4}}
\newcommand{\twth}{\frac{2}{3}}
\newcommand{\R}{{\cal R}}
\newcommand{\be}{\begin{equation}}
\newcommand{\ee}{\end{equation}}
\newcommand{\RE}{\hbox{Re}}
\newcommand{\LE}{\hbox{Le}}
\newcommand{\PR}{\hbox{Pr}}
\newcommand{\PE}{\hbox{Pe}}
\newcommand{\NU}{\hbox{Nu}}
\newcommand{\SC}{\hbox{Sc}}
\newcommand{\SH}{\hbox{Sh}}
\newcommand{\WE}{\hbox{We}}
\newcommand{\COTWO}{{\tiny \hbox{CO}_2}}
\newcommand{\HTWOO}{{\tiny \hbox{H}_2\hbox{O}}}
\newcommand{\OTWO}{{\tiny \hbox{O}_2}}
\newcommand{\NTWO}{{\tiny \hbox{N}_2}}
\newcommand{\CO}{{\tiny \hbox{CO}}}
\newcommand{\F}{{\tiny \hbox{F}}}
\newcommand{\C}{{\tiny \hbox{C}}}
\newcommand{\Hy}{{\tiny \hbox{H}}}
\newcommand{\So}{{\tiny \hbox{S}}}
\newcommand{\M}{{\tiny \hbox{M}}}
\newcommand{\xx}{{\tiny \hbox{x}}}
\newcommand{\yy}{{\tiny \hbox{y}}}
\newcommand{\zz}{{\tiny \hbox{z}}}

\newcommand{\dif}{\mathrm{d}}
\newcommand{\Div}{\nabla\cdot}
\newcommand{\mhalf}{\mbox{$\frac{1}{2}$}}
\newcommand{\tripleprime}{{\prime\prime\prime}}

\pagestyle{empty}

\begin{minipage}[t][9in][s]{6.5in}

\huge \flushright{NIST Special Publication 1018-5}

\vspace{1in}

\Huge \flushright{Fire Dynamics Simulator (Version 5) \\ Technical Reference Guide }

\vspace{.5in}

\large
\flushright{
Kevin McGrattan \\
Simo Hostikka \\
Jason Floyd \\
Howard Baum \\
Ronald Rehm}

\vspace{0.5in}

\flushright{In cooperation with: \\
VTT Technical Research Centre of Finland  }



\vfill


\flushright{\includegraphics[width=2.5in]{FIGURES/nistident_flright_300ppi}}


\end{minipage}

\newpage

\hspace{5in}

\newpage

\begin{minipage}[t][9in][s]{6.5in}

\huge \flushright{NIST Special Publication 1018-5}

\vspace{.75in}

\Huge \flushright{Fire Dynamics Simulator (Version 5) \\ Technical Reference Guide}

\vspace{.25in}

\normalsize
\flushright{
Kevin McGrattan \\
Howard Baum \\
Ronald Rehm \\
{\em Fire Research Division} \\
{\em Building and Fire Research Laboratory}  \\
\hspace{1.in} \\
Simo Hostikka \\
{\em VTT Technical Research Centre of Finland} \\
{\em Espoo, Finland}  \\
\hspace{1.in} \\
Jason Floyd \\
{\em Hughes Associates, Inc.}  \\
{\em Baltimore, Maryland, USA}}

\vspace{.25in}

\flushright{\today \\
$SVN Repository$~$Revision$}

\vfill

\flushright{\includegraphics[width=1in]{FIGURES/doc} }

\small
\flushright{U.S. Department of Commerce \\
{\em Carlos M. Gutierrez, Secretary} \\
\hspace{1in} \\
Technology Administration \\
{\em Robert Cresanti, Under Secretary for Technology}  \\
\hspace{1in} \\
National Institute of Standards and Technology \\
{\em William A. Jeffrey, Director} }

\end{minipage}

\newpage

\begin{minipage}[t][9in][s]{6.5in}

\flushright{Certain commercial entities, equipment, or materials may be identified in this \\
document in order to describe an experimental procedure or concept adequately. Such \\
identification is not intended to imply recommendation or endorsement by the \\
National Institute of Standards and Technology, nor is it intended to imply that the \\
entities, materials, or equipment are necessarily the best available for the purpose.
}

\vspace{3in}

\large
\flushright{\bf National Institute of Standards and Technology Special Publication 1018-5 \\
Natl.~Inst.~Stand.~Technol.~Spec.~Publ.~1018-5, \pageref{LastPage} pages (August 2007) \\
CODEN: NSPUE2 }

\vfill

\flushright{U.S. GOVERNMENT PRINTING OFFICE \\
WASHINGTON: 2007 \\
\rule{3.5in}{0.01in} \\
For sale by the Superintendent of Documents, U.S. Government Printing Office \\
Internet: bookstore.gpo.gov -- Phone: (202) 512-1800 -- Fax: (202) 512-2250 \\
Mail: Stop SSOP, Washington, DC 20402-0001 }

\end{minipage}



\newpage

\frontmatter

\pagestyle{plain}


\chapter{Preface}

The use of fire models currently extends beyond the fire research laboratories
and into the engineering, fire service and legal communities.
Surveys~\cite{Olenick:2,Friedman:1} of available fire models
show a significant increase in number
over the last decade. Sufficient evaluation of any model is necessary to
ensure that users can judge the adequacy of its technical basis, appropriateness of its
use, and confidence level of its predictions. This document
provides the theoretical basis for the Fire Dynamics Simulator (FDS), following the general framework set forth in
the ``Standard Guide for
Evaluating the Predictive Capability of Deterministic Fire Models,'' ASTM~E~1355~\cite{ASTM:E1355}.

A separate document, {\em Fire Dynamics Simulator, User's Guide}~\cite{FDS_Users_Guide_5} describes how the FDS software is
actually used.


\chapter{Disclaimer}

The US Department of Commerce makes no warranty, expressed or implied,
to users of the Fire Dynamics Simulator (FDS), and accepts no responsibility for its use.
Users of FDS assume sole responsibility under Federal law for determining
the appropriateness of its use in any particular application;
for any conclusions drawn from the results of its use; and for any
actions taken or not taken as a result of analysis performed using these tools.

Users are warned that FDS is intended for use only by those competent
in the fields of fluid dynamics, thermodynamics, heat transfer, combustion, and fire science,
and is intended only to supplement the informed judgment of the qualified user.
The software package is a computer model that may or may not have predictive
capability when applied to a specific set of factual circumstances.
Lack of accurate predictions by the model could lead to erroneous
conclusions with regard to fire safety. All results should be evaluated by an informed user.

Throughout this document, the mention of computer hardware or commercial
software does not constitute endorsement by NIST, nor does it indicate that
the products are necessarily those best suited for the intended purpose.


\chapter{About the Authors}

\begin{description}
\item[Kevin McGrattan] is a mathematician in the Building and Fire Research Laboratory of NIST. He received a bachelors of science degree from
the School of Engineering and Applied Science of Columbia University in 1987 and a doctorate at the Courant Institute
of New York University in 1991. He joined the NIST staff in 1992 and has since worked on the development of fire models, most notably the
Fire Dynamics Simulator.
\item[Simo Hostikka] is a Senior Research Scientist at VTT Technical
Research Centre of Finland.  He is the principal developer of the
radiation and solid phase sub-models within FDS
\item[Jason Floyd] is a Senior Engineer at Hughes Associates, Inc., in Baltimore, Maryland. He received a bachelors of science and Ph.D. in the Nuclear Engineering
Program of the University of Maryland. After graduating, he won a National Research Council Post-Doctoral Fellowship at the Building and Fire
Research Laboratory of NIST, where he developed the combustion algorithm within FDS. He is currently funded by NIST under grant 60NANB5D1205 from the Fire Research Grants Program (15 USC 278f).
He is the principal developer of the multi-parameter mixture fraction combustion model and control logic within FDS.
\item[Howard Baum] is a NIST Fellow {\em emeritus}. He received a bachelors of science degree from Brooklyn Polytechnic Institute in 1957 and a
doctorate from Harvard University in 1964. After joining the staff of the National Bureau of Standards (now NIST), he, along with Ronald Rehm,
developed the low Mach number formulation of the Navier-Stokes equations that is still the core transport algorithm within FDS.
\item[Ronald Rehm] is a former NIST fellow, currently retired but still active in a number of projects at NIST. He received a bachelors of science
degree from Purdue University in 1960 and a doctorate from the Massachusetts Institute of Technology in 1965. Along with Howard Baum, he
formulated the original numerical scheme that evolved over 25 years into the current FDS solver.
\end{description}


\chapter{Acknowledgments}

\label{acksection}

The development and maintenance of the Fire Dynamics Simulator has been made possible through
a partnership of public and private organizations, both in the United States and abroad. Following
is a list of contributors from the various sectors of the fire research, fire protection engineering and
fire services communities:

FDS is supported financially via internal funding at both NIST and
VTT, Finland. In addition, support is provided by other agencies of
the US Federal Government, most notably the Nuclear Regulatory Agency
Office of Research. The US NRC Office of Research has funded key
validation experiments, the preparation of the FDS manuals, and the
development of various sub-models that are of importance in the area
of nuclear power plant safety. Special thanks to Mark Salley and Jason
Dreisbach for their efforts and support.  The Office of Nuclear
Material Safety and Safeguards, another branch of the NRC, has
supported modeling studies of tunnel fires under the direction of
Chris Bajwa and Allen Hansen.

Another source of support for FDS development has been the Microgravity Combustion Program of the National Aeronautics and Space
Administration (NASA).

Originally, the basic hydrodynamic solver was designed by Ronald Rehm
and Howard Baum with programming help from Darcy Barnett, Dan Lozier
and Hai Tang of the Computing and Applied Mathematics Laboratory
(CAML) at NIST, and Dan Corley of the Building and Fire Research
Laboratory (BFRL). Jim Sims of CAML developed the original
visualization software.  The direct pressure solver was written by
Roland Sweet of the National Center for Atmospheric Research (NCAR),
Boulder, Colorado.  Kuldeep Prasad added the multiple-mesh data
structures, paving the way for parallel processing.  William (Ruddy)
Mell has added special routines to extend the model into areas such as
microgravity combustion and wildland fire spread. Charles Bouldin
devised the basic framework of the parallel version of the code.

At NIST, Glenn Forney developed the visualization tool Smokeview that
not only made the public release possible, but it also serves as the
principal diagnostic tool for the continuing development of
FDS.

William Grosshandler and Tom Cleary, both currently at NIST, developed
an enhancement to the smoke detector activation algorithm, originally
conceived by Gunnar Heskestad of Factory Mutual. Steve
Olenick of Combustion Science and Engineering (CSE) implemented the
smoke detector model into FDS.

William Grosshandler is also the developer of RadCal, a library of
subroutines that have been incorporated in FDS to provide the
radiative properties of gases and smoke.

Professor Fred Mowrer of the University of Maryland provided a simple
of model of gas phase extinction to FDS. Chris Lautenburger of the University of California, Berkeley, provided valuable insight in the development of the
solid phase model.



Finally, thanks to Randall McDermott for a very thorough review of the manuscript and suggestions for improvements
to the various routines.









\tableofcontents

\mainmatter


\chapter{Introduction}
\subsubsection{Howard Baum, NIST Fellow Emeritus}

The idea that the dynamics of a fire might be studied numerically dates back
to the beginning of the computer age. Indeed, the fundamental
conservation equations governing fluid dynamics, heat transfer, and
combustion were first written down over a century ago.
Despite this, practical mathematical models of fire
(as distinct from controlled combustion) are relatively recent due
to the inherent complexity of the problem.
Indeed, in his brief history of the early days of fire research,
Hoyt Hottel noted ``A case can be made for fire being,
next to the life processes, the most complex of phenomena to understand''~\cite{Hottel:1}.

The difficulties revolve about three issues:
First, there are an enormous number of possible fire scenarios
to consider due to their accidental nature. Second, the physical
insight and computing power required to perform all the necessary calculations
for most fire scenarios are limited. Any fundamentally based study of fires
must consider at least some aspects of bluff body aerodynamics, multi-phase flow,
turbulent mixing and combustion, radiative transport, and conjugate heat transfer;
all of which are active research areas in their own right.
Finally, the ``fuel'' in most fires was never intended as such.
Thus, the mathematical models and the data needed to characterize the
degradation of the condensed phase materials that supply the fuel may not be available.
Indeed, the mathematical modeling of the physical and
chemical transformations of real materials as they burn is still in its infancy.

In order to make progress, the questions that are asked have to be greatly simplified.
To begin with, instead of seeking a methodology that can be applied to all fire problems,
we begin by looking at a few scenarios that seem to be most amenable to analysis.
Hopefully, the methods developed to study these ``simple'' problems can be generalized
over time so that more complex scenarios can be analyzed.
Second, we must learn to live with idealized descriptions of fires and approximate
solutions to our idealized equations. Finally, the methods should be capable of systematic improvement.
As our physical insight and computing power grow more powerful, the methods of analysis can
grow with them.

To date, three distinct approaches to the simulation of fires have emerged.
Each of these treats the fire as an inherently three dimensional process evolving in time.
The first to reach maturity, the ``zone'' models, describe compartment fires.
Each compartment is divided into two spatially homogeneous volumes, a hot upper layer and a cooler lower layer.
Mass and energy balances are enforced for each layer, with additional models describing other
physical processes appended as differential or algebraic equations as appropriate.
Examples of such phenomena include fire plumes, flows through doors, windows and other vents,
radiative and convective heat transfer, and solid fuel pyrolysis.
Descriptions of the physical and mathematical assumptions
behind the zone modeling concept are given in separate papers by Jones~\cite{Jones:1} and Quintiere~\cite{Quintiere:1},
who chronicle developments through 1983.
Model development since then has progressed to the point where
documented and supported software implementing these models are widely available~\cite{Forney:1}.

The relative physical and computational simplicity of the zone models has led to their
widespread use in the analysis of fire scenarios. So long as detailed spatial
distributions of physical properties are not required, and the two layer description
reasonably approximates reality, these models are quite reliable.
However, by their very nature, there is no way to systematically improve them.
The rapid growth of computing power and the corresponding maturing of computational
fluid dynamics (CFD), has led to the development of CFD based ``field'' models applied to fire research problems.
Virtually all this work is based on the conceptual framework provided by the Reynolds-averaged form of the
Navier-Stokes equations (RANS), in particular the $k -\epsilon$ turbulence model pioneered by
Patankar and Spalding~\cite{Patankar:1}. The use of CFD models has allowed
the description of fires in complex geometries, and the incorporation of a wide variety of
physical phenomena. However, these models have a fundamental limitation for fire applications --
the averaging procedure at the root of the model equations.

RANS models were developed as a time-averaged approximation to the conservation equations of fluid dynamics.
While the precise nature of the averaging time is not specified, it is clearly long enough to
require the introduction of large eddy transport coefficients to describe the unresolved fluxes of mass,
momentum and energy. This is the root cause of the smoothed appearance of the results of even the most
highly resolved fire simulations. The smallest resolvable length scales are determined by the product
of the local velocity and the averaging time rather than the spatial resolution of the underlying computational grid.
This property of RANS models is typically exploited in numerical computations by using implicit
numerical techniques to take large time steps.

Unfortunately, the evolution of large eddy structures characteristic of most fire plumes is lost
with such an approach, as is the prediction of local transient events. It is sometimes
argued that the averaging process used to define the equations is an ``ensemble average'' over many
replicates of the same experiment or postulated scenario. However, this is a moot point in
fire research since neither experiments nor real scenarios are replicated in the sense required
by that interpretation of the equations. The application of ``Large Eddy Simulation'' (LES)
techniques to fire is aimed at extracting greater temporal and spatial fidelity from simulations
of fire performed on the more finely meshed grids allowed by ever faster computers.

The phrase LES refers to the description of turbulent mixing of the gaseous fuel and combustion
products with the local atmosphere surrounding the fire. This process, which determines the burning
rate in most fires and controls the spread of smoke and hot gases, is extremely difficult
to predict accurately. This is true not only in fire research but in almost all phenomena
involving turbulent fluid motion. The basic idea behind the LES technique is that the eddies
that account for most of the mixing are large enough to be calculated with reasonable
accuracy from the equations of fluid dynamics. The hope (which must ultimately be justified
by comparison to experiments) is that small-scale eddy motion can either be crudely accounted for or ignored.

The equations describing the transport of mass, momentum, and energy by the fire-induced flows must
be simplified so that they can be efficiently solved for the fire scenarios of interest.
The general equations of fluid dynamics describe a rich variety of physical processes,
many of which have nothing to do with fires. Retaining this generality would lead to an
enormously complex computational task that would shed very little additional insight on fire dynamics.
The simplified equations, developed by Rehm and Baum~\cite{Rehm:1}, have been widely adopted
by the larger combustion research community, where they are referred to as the ``low Mach number''
combustion equations. They describe the low speed motion of a gas driven by chemical heat release and buoyancy forces.
Oran and Boris provide a useful discussion of the technique as applied to various reactive flow regimes in the chapter
entitled ``Coupled Continuity Equations for Fast and Slow Flows'' in Ref.~\cite{Oran:1}.
They comment that ``There is generally a heavy price for being able to use a single algorithm for both
fast and slow flows, a price that translates into many computer operations per time step often spent in
solving multiple and complicated matrix operations.''

The low Mach number equations are solved numerically by dividing the physical space where
the fire is to be simulated into a large number of rectangular cells. Within each cell the gas
velocity, temperature, {\em etc.}, are assumed to be uniform; changing only with time.
The accuracy with which the fire dynamics can be simulated depends on the number of cells
that can be incorporated into the simulation. This number is ultimately limited
by the computing power available. Present day, single processor desktop computers limit the number of
such cells to at most a few million. This means that the ratio of largest to smallest eddy length
scales that can be resolved by the computation (the ``dynamic range'' of the simulation) is on the order of 100.
Parallel processing can be used to extend this range to some extent, but
the range of length scales that need to be accounted for if all relevant
fire processes are to be simulated is roughly $10^4$ to $10^5$ because combustion processes take place at
length scales of 1~mm or less, while the length scales associated with building fires are of the order of
tens of meters. The form of the numerical equations discussed below depends on which end of the
spectrum one wants to capture directly, and which end is to be ignored or approximated.





\chapter{Model Overview}

This chapter presents general information about the Fire Dynamics Simulator, following the basic framework set forth in
ASTM E 1355~\cite{ASTM:E1355}. This chapter also provides information about how the FDS software is developed and maintained, often referred to as a
{\em Configuration Management Plan}. Formal guidelines for such a plan are detailed in IEEE Standard~828-1998.


\section{Basic Description of FDS}


\subsection{Type of Model}

FDS is a Computational Fluid Dynamics (CFD) model of fire-driven fluid flow.
The model solves numerically a form of the Navier-Stokes equations appropriate
for low-speed, thermally-driven flow with an emphasis on smoke and heat transport
from fires. The partial derivatives of the conservation equations of mass, momentum and energy are approximated
as finite differences, and the solution is updated in time on a three-dimensional, rectilinear grid.
Thermal radiation is computed using a finite volume technique on the same grid as the flow solver.
Lagrangian particles are used to simulate smoke movement, sprinkler discharge, and fuel sprays.

Smokeview is a companion program to FDS that produces images and animations of the results. In recent years, its developer, Glenn Forney, has
added to Smokeview the ability to visualize fire and smoke in a fairly realistic way. In a sense, Smokeview now is, via its three-dimensional
renderings, an integral part of the physical model, as it allows one to assess the visibility within a fire compartment in ways that ordinary
scientific visualization software cannot.

Although not part of the FDS/Smokeview suite maintained at NIST, there are several third-party and proprietary ``add-ons'' to FDS either available
commercially or privately maintained by individual users. Most notably, there are several Graphical User Interfaces (GUIs) that can be used
to create the input file containing all the necessary information needed to perform a simulation.

\subsection{Version History}

Version 1 of FDS was publicly released in February 2000, version 2
in December 2001, version 3 in November 2002, and version 4 in July 2004.
The present version of FDS is 5, first released in September, 2007.

Starting with FDS 5, a formal revision management system has been implemented to track changes to the FDS source
code. The open-source program development tools are provided by an Internet-based organization known as Google Code (code.google.com).

The version number for FDS has three parts.  For example, FDS 5.2.12
indicates that this is FDS 5, the fifth major release. The 2 indicates
a significant upgrade, but still within the framework of FDS 5.  The
12 indicates the twelveth minor upgrade of 5.2, mostly bug fixes and
minor user requests.


\subsection{Model Developers}


Currently, FDS is maintained by the Building and Fire Research Laboratory (BFRL) of National Institute of Standards and Technology. The developers
at NIST have formed a loose collaboration of interested stakeholders, including:
\begin{itemize}
\item VTT Technical Research Centre of Finland, a research and testing
laboratory similar to NIST
\item The Society of Fire Protection Engineers (SFPE) who conduct training classes on the use of FDS
\item Fire protection engineering firms that use the software
\item Engineering departments at various universities with a particular emphasis on fire
\end{itemize}
BFRL awards grants on a competitive basis to external organizations who conduct research in fire science and engineering. Some of these grants have
been used to assist the development of FDS. The role of the grantee in supporting day to day development varies. Not all of the developers outside
of NIST are grantees.

Starting with Version 5, the FDS development team uses
an Internet-based development
environment called GoogleCode, a free service of the search engine company, Google. GoogleCode is a widely used service designed to assist
open source software development by providing a repository for source
code, revision control, program distribution, bug tracking, and
various other very useful services.

Each member of the FDS development team has an account and password
access to the FDS repository. In
addition, anonymous access is available to all interested users, who
can receive the latest versions of the source code, manuals, and other
items. Anonymous users simply do not have the power to commit changes
to any of these items. The power to commit changes to FDS or its
manuals can be granted to anyone on a case by case basis.

The FDS manuals are typeset using \LaTeX, specifically, PDF \LaTeX. The \LaTeX files are essentially text files that are under
SVN (Subversion) control. The figures are either in the form of PDF or jpeg files, depending on whether they are vector or
raster format. There are a variety of \LaTeX packages available, including MiKTeX. The FDS developers edit the manuals as part of the
day to day upkeep of the model. Different editions of the manuals are distinguished by date.


\subsection{Development Process}

Changes are made to the FDS source code daily, and tracked via
revision control software. However, these
daily changes do not constitute a change to the version number. After
the developers determine that enough changes have been made to the
source, they release a new minor upgrade, 5.2.12 to 5.2.13, for
example. This happens every few weeks. A change from 5.2 to 5.3 might
happen only a few times a year, when significant improvements have
been made to the model physics.

There is no formal process by which FDS is updated. Each developer
works on various routines, and makes changes as warranted. Minor bugs
are fixed without any communication (the developers are in different
locations), but more significant changes are discussed via email or
telephone calls. A suite of simple verification calculations (included
in this document) are routinely run to ensure that the daily bug fixes
have not altered any of the important algorithms. A suite of
validation calculations (also included here) are run with each
significant upgrade.
Significant changes to FDS are made based on the following criteria, in no particular order:
\begin{description}
\item[Better Physics:] The goal of any model is to be {\em predictive}, but it also must be reliable. FDS is a blend of empirical and
deterministic sub-models, chosen based on their robustness, consistency, and reliability. Any new sub-model must demonstrate that it is
of comparable or superior accuracy to its empirical counterpart.
\item[Modest CPU Increase:] If a proposed algorithm doubles the calculation time but yields only a marginal improvement in accuracy, it is
likely to be rejected. Also, the various routines in FDS are expected to consume CPU time in proportion to their overall importance. For example,
the radiation transport algorithm consumes about 25~\% of the CPU time, consistent with the fact that about one-fourth to one-third of the
fire's energy is emitted as thermal radiation.
\item[Simpler Algorithm:] If a new algorithm does what the old one did using less lines of code, it is almost always accepted, so long as
it does not decrease functionality.
\item[Increased or Comparable Accuracy:] The validation experiments that are part of this guide serve as the metric for new routines. It is
not enough for a new algorithm to perform well in a few cases. It must show clear improvement across the suite of experiments. If the
accuracy is only comparable to the previous version, then some other criteria must be satisfied.
\item[Acceptance by the Fire Protection Community:] Especially in regard to fire-specific devices, like sprinklers and smoke detectors, the
algorithms in FDS often are based on their acceptance among the practicing engineers.
\end{description}



\subsection{Intended Uses of FDS}

Throughout its development, FDS has been aimed at solving practical
fire problems in fire protection engineering, while at the same time
providing a tool to study fundamental fire dynamics and combustion.
FDS can be used to model the following phenomena:
\begin{itemize}
\setlength{\itemsep}{0.0in}
\item Low speed transport of heat and combustion products from fire
\item Radiative and convective heat transfer between the gas and solid surfaces
\item Pyrolysis
\item Flame spread and fire growth
\item Sprinkler, heat detector, and smoke detector activation
\item Sprinkler sprays and suppression by water
\end{itemize}
Although FDS was designed specifically for fire simulations,
it can be used for other low-speed fluid flow simulations that do not necessarily
include fire or thermal effects. To date, about half of the
applications of the model have been for design of smoke control
systems and sprinkler/detector activation studies.
The other half consist of residential and industrial fire reconstructions.


\subsection{Required Input for FDS}

All of the input parameters required by FDS to describe a particular
scenario are conveyed via a single text file created by the user.
The file contains information about the numerical grid, ambient environment, building geometry, material
properties, combustion kinetics, and desired output quantities.
The numerical grid consists of one or more rectilinear meshes with (usually) uniform cells. All geometric features of the
scenario must conform to this numerical grid. Objects smaller than a single grid cell are either approximated
as a single cell, or rejected. The building geometry is input as a series of rectangular blocks. Boundary conditions are
applied to solid surfaces as rectangular patches. Materials are defined by their thermal conductivity, specific heat,
density, thickness, and burning behavior. There are various ways that this information is conveyed, depending on the
desired level of detail.

Any simulation of a real fire scenario involves specifying material properties for the walls, floor, ceiling,
and furnishings. FDS treats all of these objects as multi-layered solids, thus the physical parameters for many real
objects can only be viewed as approximations to the actual properties. Describing these materials in the input file is
the single most challenging task for the user. Thermal properties such as conductivity, specific heat,
density, and thickness can be found in various handbooks, or in manufacturers literature, or from bench-scale measurements.
The burning behavior of materials at different heat fluxes is more difficult to describe, and the properties more difficult
to obtain. Even though entire books are devoted to the
subject~\cite{Babrauskas:2}, it is still difficult to find information on a particular item.

A significant part of the FDS input file directs the code to output various quantities in various ways. Much like in an
actual experiment, the user must decide before the calculation begins what information to save. There is no way to
recover information after the calculation is over if it was not requested at the start.

A complete description of the input parameters required by FDS can be found in the FDS User's Guide~\cite{FDS_Users_Guide_5}.




\subsection{FDS Output Quantities}

FDS computes the temperature, density, pressure, velocity and chemical composition within each numerical
grid cell at each discrete time step. There are typically hundreds of thousands to millions of grid
cells and thousands to hundreds of thousands of time steps. In addition, FDS computes at solid surfaces
the temperature, heat flux, mass loss rate, and various other quantities. The user must carefully select what
data to save, much like one would do in designing an actual experiment. Even though only a small fraction of
the computed information can be saved, the output typically consists of fairly large data files. Typical
output quantities for the gas phase include:
\begin{itemize}
\setlength{\itemsep}{0.0in}
\item Gas temperature
\item Gas velocity
\item Gas species concentration (water vapor, CO$_2$, CO, N$_2$)
\item Smoke concentration and visibility estimates
\item Pressure
\item Heat release rate per unit volume
\item Mixture fraction (or air/fuel ratio)
\item Gas density
\item Water droplet mass per unit volume
\end{itemize}
On solid surfaces, FDS predicts additional quantities associated with the energy balance between
gas and solid phase, including
\begin{itemize}
\setlength{\itemsep}{0.0in}
\item Surface and interior temperature
\item Heat flux, both radiative and convective
\item Burning rate
\item Water droplet mass per unit area
\end{itemize}
Global quantities recorded by the program include:
\begin{itemize}
\setlength{\itemsep}{0.0in}
\item Total Heat Release Rate (HRR)
\item Sprinkler and detector activation times
\item Mass and energy fluxes through openings or solids
\end{itemize}
Time histories of various quantities at a single point in space or global
quantities like the fire's heat release rate (HRR) are saved in simple, comma-delimited text files that
can be plotted using a spreadsheet program.
However, most field or surface data are visualized with a program called Smokeview, a tool specifically
designed to analyze data generated by FDS. FDS and Smokeview are used in concert to model and visualize fire phenomena.
Smokeview performs this visualization by presenting animated tracer particle flow,
animated contour slices of computed gas variables and animated surface data.
Smokeview also presents contours and vector plots of static data anywhere
within a scene at a fixed time.

A complete list of FDS output quantities and formats is given in Ref.~\cite{FDS_Users_Guide_5}.
Details on the use of Smokeview are found in Ref.~\cite{Smokeview_Users_Guide_5}.




\subsection{FDS Governing Equations, Assumptions and Numerics}

Following is a brief description of
the major components of FDS. Detailed information regarding the assumptions and governing equations associated
with the model is provided in Section~\ref{govequations}.
\begin{description}
\item[Hydrodynamic Model] FDS
solves numerically a form of the Navier-Stokes equations appropriate
for low-speed, thermally-driven flow with an emphasis on
smoke and heat transport from fires. The core algorithm is an
explicit predictor-corrector scheme that is second order accurate in space
and time. Turbulence is treated by means of the Smagorinsky form of
Large Eddy Simulation (LES). It is possible to perform a Direct
Numerical Simulation (DNS) if the underlying numerical grid is fine
enough. LES is the default mode of operation.
\item[Combustion Model]
For most applications, FDS uses a combustion model based on the mixture fraction concept.
The mixture fraction is a conserved scalar quantity
that is defined as the
fraction of gas at a given point in the flow field that originates as fuel.
Unlike versions of FDS prior to 5, the reaction of fuel and oxygen is not necessarily instantaneous and complete, and there are
several optional schemes that are designed to predict the extent of combustion in under-ventilated spaces.
The mass fractions of all of the major reactants and products can
be derived from the mixture fraction by means of ``state relations,''
expressions arrived at by a
combination of simplified analysis and measurement.
\item[Radiation Transport] Radiative heat transfer is included in the
model via the solution of the radiation transport equation for a gray
gas. In a limited number of cases, a wide band model can be used in
place of the gray gas model to provide a better spectral accuracy. The
radiation equation is solved using a technique similar to a finite
volume method for convective transport, thus the name given to it is
the Finite Volume Method (FVM). Using approximately 100 discrete
angles, the finite volume solver requires about 20~\% of the total CPU
time of a calculation, a modest cost given the complexity of radiation
heat transfer.  Water droplets can absorb and scatter thermal
radiation. This is important in cases involving mist sprinklers, but
also plays a role in all sprinkler cases. The absorption and
scattering coefficients are based on Mie theory. The scattering from
the gaseous species and soot is not included in the model.
\item[Geometry]
FDS approximates the governing equations on one or more rectilinear grids. The
user prescribes rectangular obstructions that are forced to conform
with the underlying grid.
\item[Boundary Conditions]
All solid surfaces are assigned thermal boundary conditions, plus
information about the burning behavior of the material. Usually,
material properties are stored in a database and invoked by name.
Heat and mass transfer to and from solid surfaces is
usually handled with empirical correlations, although it is possible
to compute directly the heat and mass transfer when performing a
Direct Numerical Simulation (DNS).
\item[Sprinklers and Detectors] The activation of sprinklers and heat and smoke detectors
is modeled using fairly simple correlations of thermal inertia for
sprinklers and heat detectors, and transport lag for smoke detectors.
Sprinkler sprays are modeled by Lagrangian particles that represent a sampling of the
water droplets ejected from the sprinkler.
\end{description}


\subsection{Limitations of FDS}

Although FDS can address most fire scenarios, there are limitations in all of its various
algorithms. Some of the more prominent limitations of the model are listed here. More
specific limitations are discussed as part of the description of the governing equations
in Section~\ref{govequations}.
\begin{description}
\item[Low Speed Flow Assumption] The use of FDS is limited to low-speed\footnote{Mach numbers less than about 0.3} flow
with an emphasis on smoke and heat transport from fires. This assumption rules out using the model for any scenario
involving flow speeds approaching the speed of sound, such as explosions, choke flow at nozzles, and detonations.
\item[Rectilinear Geometry] The efficiency of FDS is due to the simplicity of its rectilinear numerical grid and the
use of a fast, direct solver for the pressure field.
This can be a limitation in some situations where certain geometric features
do not conform to the rectangular grid, although most building components do. There are techniques in FDS to
lessen the effect of ``sawtooth'' obstructions used to represent non-rectangular objects, but these cannot be expected
to produce good results if, for example, the intent of the calculation is to study boundary layer effects. For most
practical large-scale simulations, the increased grid resolution afforded by the fast pressure solver offsets the
approximation of a curved boundary by small rectangular grid cells.
\item[Fire Growth and Spread]
Because the model was originally designed to analyze industrial-scale fires,
it can be used reliably when the heat release rate (HRR) of the fire is specified and the
transport of heat and exhaust products is the principal aim of the simulation.
In these cases, the model predicts flow velocities and temperatures to an accuracy within
10~\% to 20~\% of experimental measurements, depending on the resolution of the numerical grid
\footnote{It is extremely rare to
find measurements of local velocities and/or temperatures from fire experiments that
have reported error estimates that are less than 10~\%. Thus, the most accurate
calculations using FDS do not introduce significantly greater errors in these quantities
than the vast majority of fire experiments.}.
However, for fire scenarios where the heat release rate is {\em predicted} rather than {\em prescribed},
the uncertainty of the model is higher.
There are several reasons for this: (1) properties of real materials
and real fuels are often unknown or difficult to obtain, (2) the physical processes of combustion,
radiation and solid phase heat transfer are more complicated than their mathematical representations
in FDS, (3) the results of calculations are sensitive to both the numerical and physical parameters.
Current research is aimed at improving this situation, but it is safe to say that
modeling fire growth and spread will always require a higher level of
user skill and judgment than that required for modeling the transport of smoke and heat from prescribed fires.
\item[Combustion]
For most applications, FDS uses a mixture fraction-based combustion model.
The mixture fraction is a conserved scalar quantity that is defined as the
fraction of gas at a given point in the flow field that originated as fuel.
In its simplest form, the model assumes that combustion is mixing-controlled, and that the
reaction of fuel and oxygen is infinitely fast, regardless of the temperature.
For large-scale, well-ventilated
fires, this is a good assumption. However, if a fire is in an
under-ventilated compartment, or if a suppression agent like water
mist or CO$_2$ is introduced, fuel and oxygen are allowed to mix and not burn, according to a few empirically-based criteria.
The physical mechanisms underlying these phenomena are complex, and are tied closely to the flame temperature and local strain rate, neither of
which are readily-available in a large scale fire simulation.
Subgrid-scale modeling of gas phase suppression and
extinction is still an area of active research in the combustion
community. Until reliable models can be developed for building-scale
fire simulations, simple empirical rules can be used that
prevent burning from taking place when the atmosphere immediately
surrounding the fire cannot sustain the combustion. Details are found in
Section~\ref{combustionsection}.
\item[Radiation] Radiative heat transfer is included in the model via
the solution of the radiation transport equation (RTE) for a gray gas, and
in some limited cases using a wide band model.  The RTE is solved
using a technique similar to finite volume methods for convective
transport, thus the name given to it is the Finite Volume Method
(FVM). There are several limitations of the model. First, the
absorption coefficient for the smoke-laden gas is a complex function
of its composition and temperature. Because of the simplified
combustion model, the chemical composition of the smokey gases,
especially the soot content, can effect both the absorption and
emission of thermal radiation.  Second, the radiation transport is
discretized via approximately 100 solid angles, although the user may
choose to use more angles. For targets far away from a localized
source of radiation, like a growing fire, the discretization can lead
to a non-uniform distribution of the radiant energy. This error is
called ``ray effect'' and can be seen in the visualization of surface
temperatures, where ``hot spots'' show the effect of the finite number
of solid angles. The problem can be lessened by the inclusion of more
solid angles, but at a price of longer computing times. In most cases,
the radiative flux to far-field targets is not as important as those
in the near-field, where coverage by the default number of angles is
much better.
\end{description}



\clearpage
\section{Peer Review Process}

FDS is reviewed both internally and externally. All documents issued by the
National Institute of Standards and Technology are formally reviewed internally by members of
the staff. The theoretical basis of FDS is laid out in the present document, and is
subject to internal review by staff members who are not active participants in the development
of the model, but who are members of the Fire Research Division and are considered experts in
the fields of fire and combustion. Externally, papers detailing various parts of FDS are
regularly published in peer-reviewed journals and conference proceedings. In addition, FDS
is used world-wide by fire protection engineering firms who review the technical details of
the model related to their particular application. Some of these firms also publish in the
open literature reports documenting internal efforts to validate the model for a particular
use. Many of these studies are referenced in the FDS Validation Guide~\cite{FDS_VV_Guide_5}.


\subsection{Survey of the Relevant Fire and Combustion Literature}

\label{Relevantdocs}

FDS has three separate manuals --
the FDS Technical Reference Guide, the FDS Validation Guide~\cite{FDS_VV_Guide_5},
and the FDS User's Guide~\cite{FDS_Users_Guide_5}.
Smokeview has its own User's Guide~\cite{Smokeview_Users_Guide_5}. The User Guides only describe the mechanics of using the
computer programs. The
Technical Reference Guide provides the theory, algorithm details, and verification work. The Validation Guide is new with FDS~5.
It describes validation studies of FDS, both those performed by NIST and by others.

There are numerous sources that describe various parts of the
model. The basic set of equations solved in FDS was formulated by Rehm
and Baum in the {\em Journal of Research of the National Bureau of
Standards}~\cite{Rehm:1}.  The basic hydrodynamic algorithm evolved at
NIST through the 1980s and 1990s, incorporating fairly well-known
numerical schemes that are documented in books by Anderson, Tannehill
and Pletcher~\cite{Anderson:1}, Peyret and Taylor~\cite{Peyret:1}, and
Ferziger and Peri\'{c}~\cite{Ferziger:1}. This last book provides a
good description of the large eddy simulation technique and provides
references to many current publications on the subject.  Numerical
techniques appropriate for combustion systems are described by Oran
and Boris~\cite{Oran:1}.  The mixture fraction combustion model is
described in a review article by Bilger~\cite{Bilger:AnnRev}. Basic
heat transfer theory is provided by Holman~\cite{Holman:1} and
Incropera~\cite{Incropera:1}. Thermal radiation is described in Siegel
and Howell~\cite{Siegel:1}.

Much of the current knowledge of fire science and engineering
is found in the {\em SFPE Handbook of Fire Protection Engineering}~\cite{SFPE}. Popular textbooks in fire protection
engineering include those by Drysdale~\cite{Drysdale:1} and Quintiere~\cite{Quintiere:2}. On-going research in
fire and combustion is documented in several periodicals and conference proceedings.
The International Association of Fire Safety Science (IAFSS)
organizes a conference every two years, the proceedings of which are frequently referenced by fire researchers.
Interscience Communications, a London-based publisher of several fire-related journals, hosts a conference known as Interflam roughly
every three years in the United Kingdom.
The Combustion Institute hosts an international symposium on combustion every two years, and in addition to the
proceedings of this symposium, the organization publishes its own journal, {\em Combustion and Flame}.
The papers appearing in the IAFSS conference proceedings,
the Combustion Symposium proceedings, and {\em Combustion and Flame} are all peer-reviewed, while those appearing in the
Interflam proceedings are selected based on the submission of a short abstract.
Both the Society for Fire Protection Engineers (SFPE) and the National Fire Protection Association (NFPA) publish
peer-reviewed technical journals entitled the {\em Journal of Fire Protection Engineering} and {\em Fire Technology}.
Other often-cited, peer-reviewed technical journals include the {\em Fire Safety Journal}, {\em Fire and Materials}, {\em Combustion
Science and Technology}, {\em Combustion Theory and Modeling} and the {\em Journal of Heat Transfer}.

Research at NIST is documented in various ways beyond contributions made by staff to external journals and conferences.
NIST publishes several forms of internal reports, special publications, and its own journal called the {\em Journal
of Research of NIST}. An internal report, referred to as a NISTIR (NIST Inter-agency Report), is a convenient means to disseminate information,
especially when the quantity of data exceeds what could normally be accepted by a journal. Often parts of a NISTIR are
published externally, with the NISTIR itself serving as the complete record of the work performed. Previous versions of the
FDS Technical Reference Guide and User's Guide were published as NISTIRs. The current FDS and Smokeview manuals are
being published as NIST Special Publications, distinguished from NISTIRs by the fact that they are permanently archived.
Work performed by an outside person or organization working
under a NIST grant or contract is published in the form of a NIST Grant/Contract Report (GCR).
All work performed by the staff of the Building and Fire Research Laboratory at NIST beyond 1993 is permanently stored in
electronic form and made freely available via the Internet and yearly-released compact disks (CDs) or other electronic media.




\subsection{Review of the Theoretical Basis of the Model}

\label{JustAA}
The technical approach and assumptions of the model have been presented in
the peer-reviewed scientific literature and at technical conferences cited in the previous section.
The major assumptions of the model, for example the large eddy simulation technique and the mixture fraction combustion
model, have undergone a roughly 40 year development and are now documented in popular introductory text books.
More specific sub-models, like the sprinkler spray routine or the various pyrolysis models, have yet to be developed to
this extent. As a consequence, all documents produced by NIST staff are required to go
through an internal editorial review and approval process.
This process is designed to ensure compliance with the technical requirements,
policy, and editorial quality required by NIST.
The technical review includes a critical evaluation of the technical content and
methodology, statistical treatment of data, uncertainty analysis, use of appropriate
reference data and units, and bibliographic references.
The FDS and Smokeview manuals are first reviewed by a member of the Fire Research Division,
then by the immediate supervisor of the author of the document,
then by the chief of the Fire Research Division, and finally by a reader from
outside the division. Both the immediate supervisor and the division chief are
technical experts in the field. Once the document has been reviewed, it is
then brought before the Editorial Review Board (ERB),
a body of representatives from all the NIST laboratories.
At least one reader is designated by the Board for each document that it accepts for
review. This last reader is selected based on technical competence and impartiality.
The reader is usually from outside the division producing the document and is
responsible for checking that the document conforms with NIST policy on units, uncertainty
and scope. He/she does not need to be a technical expert in fire or combustion.

Recently, the US Nuclear Regulatory Commission (US NRC) published a seven-volume report on its own verification and validation
study of five different fire models used for nuclear power plant applications~\cite{NUREG_1824}. Two of the models are essentially a set
of empirically-based correlations in the form of engineering ``spread sheets.'' Two of the models are classic two-zone fire models, one of which
is the NIST developed CFAST. FDS is the sole CFD model in the study. More on the study and its results can be found in the
FDS Validation Guide~\cite{FDS_VV_Guide_5}.

Besides formal internal and peer review, FDS is subjected to continuous scrutiny because
it is available free of charge to the general public and is used
internationally by those involved in fire safety design and post-fire reconstruction.
The quality of the FDS and Smokeview User Guides is checked implicitly by the fact that the
majority of model users have not taken a formal training course in the actual use of the model, but
are able to read the supporting documents, perform a few sample simulations, and then systematically build
up a level of expertise appropriate for their applications. The developers receive daily feedback from
users on the clarity of the documentation and add clarifications
when needed. Before new versions of the model are released, there is a several month ``beta test'' period
in which users test the new version using the updated documentation. This process is similar,
although less formal, to that which most computer software programs undergo.
Also, the source code for FDS is released publicly, and has been used at
various universities world-wide, both in the classroom as a teaching tool as well as for research.
As a result, flaws in the theoretical development and the computer program itself
have been identified and corrected. As FDS continues to evolve, the user base will continue to
serve as a means to evaluate the model. We consider this process as important to the development of FDS as the formal
internal and external peer-review processes.








\chapter{Governing Equations}

\label{basisformodel}

This chapter presents the governing equations of FDS. The numerical algorithm is presented in the next chapter.
The governing equations are presented as a set of
partial differential equations, with appropriate simplifications and approximations
noted. The numerical method essentially consists of a finite difference approximation of the governing equations and a procedure for
updating these equations in time.

\section{Hydrodynamic Model}
\label{govequations}

An approximate form of the Navier-Stokes equations appropriate for
low Mach number applications is used in the model.
The approximation involves the filtering out of acoustic waves while
allowing for large variations in temperature and density~\cite{Rehm:1}.
This gives the equations an elliptic character, consistent with low speed,
thermal convective processes. The computation can either be treated as a
Direct Numerical Simulation (DNS), in which the dissipative terms are computed directly,
or as a Large Eddy Simulation (LES), in which the large-scale eddies are computed
directly and the subgrid-scale dissipative processes are modeled. The numerical algorithm is
designed so that LES becomes DNS as the grid is refined.
Most applications of FDS are LES. For example, in simulating the flow of smoke through a large, multi-room enclosure, it
is not possible to resolve the combustion and transport processes directly.
However, for small-scale combustion experiments,
it is possible to compute the transport and combustion processes directly.

\subsection{The Fundamental Conservation Equations}

\label{basicequations}

The conservation equations for mass, momentum and energy for a
Newtonian fluid are presented here. These are the same equations that can
be found in almost any textbook on fluid dynamcs or CFD. A particularly useful
reference for a description of the equations, the notation used,
and the various approximations employed is Anderson {\em et al.}~\cite{Anderson:1}.

\vspace{\baselineskip}
\leftline{\underline{Conservation of Mass}}
\be \dod{\rho}{t} + \nabla \cdot \rho \bu  =  \dm_b'''  \label{mass} \ee
\leftline{\underline{Conservation of Momentum (Newton's Second Law)}}
\be \dod{}{t} (\rho \bu) + \nabla \cdot \rho \bu \bu
+ \nabla p = \rho \bg + \bof_b
+ \nabla \cdot \btau_{ij}   \label{momentum} \ee
\leftline{\underline{Conservation of Energy (First Law of Thermodynamics)} }
\be \dod{ }{t}(\rho h) + \nabla \cdot \rho h \bu = \frac{Dp}{Dt}  + \dq''' - \dq_b'''
        - \nabla \cdot \dbq''  + \epsilon   \label{energy} \ee
\leftline{\underline{Equation of State for a Perfect Gas}}
\be p = \frac{\rho \R T}{\bW}  \label{basicstate} \ee
Note that this is a set of partial differential equations consisting of
six equations for six unknowns, all functions of three spatial dimensions and time:
the density $\rho$, the three components
of velocity $\bu=(u,v,w)$, the temperature $T$, and the pressure $p$.
The sensible enthalpy $h$ is a function of the temperature: $h=\int_{T_0}^T c_p(T') \, dT'$.

The notation adopted above is intended to render the equations in as
simple and compact a form as possible. However, when all of the terms are expanded, the
equations appear unwieldy. For the model developers, this is unavoidable. For the model
user, it is sufficient to understand the basic conservation laws to appreciate what the
model does, and does not, represent. Following is a brief description of the
terms. Note that boldfaced quantities represent vectors. Boldfaced quantities with the subscripts $ij$
represent tensors, which can be best understood as 3$\times$3 matrices.

The mass conservation equation is often written in terms of the mass fractions of the individual gaseous species, $Y_\alpha$:
\be \dod{ }{t}(\rho Y_\alpha) + \nabla \cdot \rho Y_\alpha \bu   = \nabla \cdot \rho D_\alpha \nabla Y_\alpha + \dm_\alpha''' + \dm_{b,\alpha}'''  \label{species} \ee
Summing these equations over all species yields the original mass conservation equation because
$\sum Y_\alpha=1$ and $\sum \dm_\alpha''' = 0$ and $\sum \dm_{b,\alpha}'''=\dm_b'''$, by definition, and because it is assumed that $\sum \rho D_\alpha \nabla Y_\alpha = 0$. This last
assertion is not true, in general. However, transport equations are solved for total mass and all but one of the species, implying
that the diffusion coefficient of the implicit species is chosen so that the sum of all the diffusive fluxes is zero.

In the momentum equation,
the term $\bu \bu$ is a diadic tensor formed by multiplying (in the sense of two 3$\times$1 matrices) the vectors
$\bu^T$ and $\bu$. The term $\nabla \cdot \rho \bu \bu$ is thus a vector formed by applying
the vector operator $\nabla = (\dod{}{x},\dod{}{y},\dod{}{z})$ to the tensor.
The force term $\bof_b$ in the momentum equation represents external forces such as
the drag exerted by liquid droplets. The stress tensor $\btau_{ij}$ is defined:
\be \btau_{ij} = \mu \left( 2 \; \bS_{ij}
   - \frac{2}{3} \bdelta_{ij} (\nabla \cdot \bu) \right) \quad ; \quad
   \bdelta_{ij}=\left\{ \begin{array}{ll} 1 & i=j \\ 0 & i\ne j \end{array} \right.   \quad ; \quad
   \bS_{ij} = \frac{1}{2} \left( \dod{u_i}{x_j}+\dod{u_j}{x_i} \right) \quad i,j=1,2,3   \ee
The term $\bS_{ij}$ is the strain tensor, written using conventional tensor notation.
The symbol $\mu$ is the dynamic viscosity of the fluid.

In the energy equation,
note the use of the material derivative, $D(\,)/Dt=\partial(\,) /\partial t + \bu \cdot \nabla (\,)$. The term
$\dq'''$ is the heat release rate per unit volume from a chemical reaction. The term $\dq_b'''$ is the energy transferred to the evaporating droplets.
The term
$\dbq''$ represents the conductive and radiative heat fluxes:
\be \dbq'' = -k \nabla T - \sum_\alpha h_\alpha \rho D_\alpha \nabla Y_\alpha + \dbq_r'' \ee
where $k$ is the thermal conductivity.

The term $\epsilon$ in the energy equation is known
as the {\em dissipation rate}. It is the rate at which kinetic energy is transferred to thermal energy due to the
viscosity of the fluid:
\begin{eqnarray}
\epsilon &\equiv& \btau_{ij} \cdot \nabla \bu \quad =  \quad
   \mu \left( 2 \; \bS_{ij} \cdot \bS_{ij}
                  - \frac{2}{3} (\nabla \cdot \bu)^2 \right) \\
  &=& \mu \left[ 2 \left(\dod{u}{x}\right)^2
 + 2 \left(\dod{v}{y}\right)^2 + 2 \left(\dod{w}{z}\right)^2 + \right. \\
& & \left.
  \left(\dod{v}{x}+\dod{u}{y}\right)^2 + \left(\dod{w}{y}+\dod{v}{z}\right)^2
 + \left(\dod{u}{z}+\dod{w}{x}\right)^2 - \frac{2}{3}
   \left(\dod{u}{x}+\dod{v}{y} + \dod{w}{z} \right)^2  \right]  \label{dissipation} \end{eqnarray}
This term is usually neglected because it is very small relative to the heat release rate of
the fire. However, it is included here because it plays a role in the discussion of large eddy simulation.

The equations presented here form the basis for a wide variety of engineering applications, but not without
the further application of simplifying assumptions unique to each field.
The only assumptions made thus far are that the fluid is a perfect gas, that the stress is
linearly dependent on the strain, that heat is conducted according to Fourier's Law, and that gas species diffuse according to
Fick's Law. In the sections to follow, additional assumptions will be
imposed on the governing equations to apply them to fire and other low speed thermal processes.
The most important of these assumptions involves the treatment of the diffusion and source
terms that differentiates one type of CFD model from
another.




\subsection{The Low Mach Number Assumption and the Equation of State}

A distinguishing feature of a CFD model is the regime of
flow speeds (relative to the speed of sound) for which it is designed. High
speed flow codes involve compressibility effects and shock waves. Low speed
solvers, however, explicitly eliminate compressibity effects that give rise
to acoustic (sound) waves. As written in Section~\ref{basicequations}, the Navier-Stokes equations describe the
propagation of information at speeds comparable to that of the fluid flow (for fire, 10-20~m/s),
but also at speeds comparable to that of sound waves (for still air,
300~m/s). Solving a discretized form of these equations would require extremely small
time steps in order to account for information traveling at the speed of sound, making
practical simulations difficult.

Following the work of Rehm and Baum~\cite{Rehm:1}, an approximation to the equation of state~(\ref{basicstate}) is made by decomposing the pressure
into a ``background'' component and a perturbation. The original version of FDS assumed that the background component of the pressure
applied to the entire computational domain, most often a single compartment. Starting in FDS version 5, it is now assumed that
the background component of pressure can differ from compartment to compartment. If
a volume within the computational domain is isolated from other volumes, except via leak paths or ventilation ducts, it is referred to as a ``pressure
zone'' and assigned its own background pressure. The pressure within the $m$th zone, for example, is a linear combination
of its background component and the flow-induced perturbation:
\be p(\bx,t) = \bp_m(z,t) + \tp(\bx,t) \ee
Note that the background pressure is a function of $z$, the vertical spatial coordinate, and time. For most
compartment fire applications, $\bp_m$ changes very little with height or time. However, for situations where the pressure
increases due to a fire in a tightly sealed enclosure, or when the height of the domain is significant, $\bp_m$ takes these effects into
account~\cite{Baum:5}. The ambient pressure field is denoted $\bp_0(z)$. Note that the subscript 0 denotes the exterior of the computational domain, not
time 0. This is the assumed atmospheric pressure stratification that serves as both
the initial and boundary condition for the governing equations.

The purpose of decomposing the pressure is that for low-Mach number flows, it can be assumed that the temperature and density are inversely
proportional, and thus the equation of state (in the $m$th pressure zone) can be approximated
\be \bp_m  =  \rho T \R \sumyw = \rho T \R / \bW  \label{state} \ee
The pressure $p$ in the state and energy equations is replaced by the background pressure $\bp_m$ to filter out sound waves
that travel at speeds that are much faster
than typical flow speeds expected in fire applications. The low Mach number assumption serves two purposes. First, the filtering of acoustic waves
means that the time step in the numerical algorithm is bound only by the flow speed as opposed to the speed of sound, and second, the modified state
equation leads to a reduction in the number of dependent variables in the system of equations by one. The energy equation (\ref{energy}) is never
explicitly solved, but its source terms are included in the expression for the flow divergence, to be derived presently.

The stratification of the atmosphere is derived from the relation
\be \frac{d \bp_0}{dz} = - \rho_0(z) \, g  \ee
where $\rho_0$ is the background
density. Using Eq.~(\ref{state}), the background pressure can be written as a function of the background temperature, $T_0(z)$
\be \bp_0(z) = p_\infty \; \exp \; \left( -\int^z_{z_\infty} \frac{\bW \, g}{\R \, T_0} dz \right)  \label{pstrat} \ee
where the subscript infinity generally refers to the ground. Usually, the temperature stratification of the atmosphere is
specified by the user, in which case $\bp_0$ and $\rho_0$ are derived from Eqs.~(\ref{pstrat}) and (\ref{state}), respectively.


\subsection{Combination of the Mass and Energy Equations via the Divergence}

Because of the low Mach number assumption, the divergence of the flow, $\nabla \cdot \bu$, plays a very important role in the overall
solution scheme. The divergence is obtained by taking the material (total) derivative
of the modified Equation of State~(\ref{state}), and then substituting terms from the mass
and energy conservation equations. A few assumptions are needed. First, define the constant-pressure
specific heat of the mixture: $c_p = \sum_\alpha c_{p,\alpha} \, Y_\alpha$ where
$c_{p,\alpha}$ is the temperature-dependent specific heat of species $\alpha$.
Next, define the enthalpy $h = \sum_\alpha h_\alpha \, Y_\alpha$ where
\be
   h_\alpha = \int_{T_{\hbox{\tiny ref}}}^T c_{p,\alpha}(T') \; dT'
\ee
Finally, note that the material derivative of the background pressure, $\bp_m(z,t)$, is written:
\be \frac{D\bp_m}{Dt} = \dod{\bp_m}{t} + w \, \dod{\bp_m}{z} = \dod{\bp_m}{t} - w \rho_m g   \ee
The divergence can now be written:
\be \nabla \cdot \bu = {\cal D} + {\cal P} \; \dod{\bp_m}{t}  \label{phi} \ee
where
\begin{eqnarray*}
\cal{D}          &=& \frac{\dm_b'''}{\rho} \frac{\bW}{W_b} +  \frac{\R}{\bW c_p \bp_m} \; \Big( \dq''' - \dq_b''' - \nabla \cdot \dbq'' \Big) - {\cal P} \, w \rho_m g  \quad + \nonumber \\
                 & & \frac{\bW}{\rho} \sum_\alpha \nabla \cdot \rho D_\alpha \nabla (Y_\alpha/W_\alpha) -
\frac{\R}{\bW c_p \bp_m} \; \sum_\alpha h_\alpha \nabla \cdot \rho D_\alpha \nabla Y_\alpha +
 \frac{1}{\rho} \sum_\alpha \left( \frac{\bW}{W_\alpha} - \frac{h_\alpha}{c_p T} \right) \; \dm_\alpha''' \\
\cal{P}          &=& \frac{1}{\bp_m} \; \left( \frac{\R}{\bW c_p} - 1 \right)
\end{eqnarray*}
Contributions to the divergence of the flow include the heat release rate of the fire, $\dq'''$, heat losses to evaporating droplets, $\dm_b'''$,
the net heat flux from thermal conduction and
radiation, $\nabla \cdot \dbq''$, updrafts of air over considerable heights of the atmosphere,
the net mass flux from gas species diffusion and production, and global pressure changes.
The change in the background pressure with time, $\partial \bp_m/\partial t$,
is non-zero only if it assumed that the compartment is tightly sealed,
in which case the background pressure, $\bp_m$, can no longer be assumed constant due to the
increase (or decrease) in mass and thermal energy within
the enclosure. The time derivative of the background pressure of the $m$th pressure zone, $\Omega_m$,
is found by integrating Eq.~(\ref{phi}) over the
zone volume:
\be \dod{\bp_m}{t} = \left( \int_{\Omega_m} {\cal D} \, dV - \int_{\partial \Omega_m}
 \bu \cdot d\bS \right) \Big/ \int_{\Omega_m} {\cal P} \, dV  \label{concon} \ee
Equation~(\ref{concon}) is essentially a consistency condition, ensuring that blowing air or starting a fire within a sealed
compartment leads to an appropriate decrease in the divergence within the volume.

\subsection{The Momentum Equation}

The momentum equation is simplified to make it easier to solve
numerically. First, we start with the non-conservative form of the momentum equation introduced above
\be \rho \left( \dod{\bu}{t} + (\bu \cdot \nabla)\bu  \right) + \nabla p = \rho \bg + \bof_b + \nabla \cdot \btau_{ij}  \ee
Next, we make the following substitutions:
\begin{enumerate}
\item Subtract the hydrostatic pressure gradient of the $n$th pressure zone, $\rho_n(z,t) \bg$, from both sides. Note that
$\nabla p=\rho_n \bg + \nabla \tp$.
\item Apply the vector identity: $(\bu \cdot \nabla) \bu = \nabla|\bu|^2/2 - \bu\times\bo $
\item Divide all terms by the density, $\rho$
\item Decompose the pressure term, noting that $\rho_\infty$ is the constant density at the ground:
   $$ \frac{\nabla \tp}{\rho} = \frac{\nabla \tp}{\rho_\infty} + \left( \frac{1}{\rho} -
   \frac{1}{\rho_\infty} \right) \nabla \tp  $$
\item Define ${\cal H} \equiv |\bu|^2/2 + \tp/\rho_\infty $
\end{enumerate}
Now the momentum equation can be written
\be \dod{\bu}{t} - \bu\times\bo + \nabla {\cal H} + \left( \frac{1}{\rho} -
   \frac{1}{\rho_\infty} \right) \nabla \tp = \frac{1}{\rho} \Big[ (\rho-\rho_n) \bg
+ \bof_b + \nabla \cdot \btau_{ij} \Big]  \label{momeq} \ee
The numerical solution of the pressure equation obtained by taking the divergence of Eq.~(\ref{momeq})
is greatly simplified by either neglecting the last term on the left hand side, or
in cases where it cannot be neglected, treating it with some care.
The decision to either neglect the extra pressure term or to approximate it depends on its
relative contribution to the creation of vorticity. An evolution equation for the circulation, $\Gamma=\oint \bu \cdot d\bx$, reveals the
sources of vorticity in the absence of any external forces, $\bof_b$:
\be \frac{d\Gamma}{dt} =
    \oint \left(\frac{1}{\rho_\infty} - \frac{1}{\rho} \right) \nabla \tp \cdot d\bx
  + \oint \frac{\rho-\rho_n}{\rho} \bg \cdot d\bx
  + \oint \frac{ \nabla \cdot \btau_{ij}}{\rho}  \cdot d\bx \ee
The first term on the right hand side represents the baroclinic torque.
The second term is buoyancy-induced vorticity.
The third term represents the vorticity generated by molecular and subgrid-scale stresses, as in boundary and shear layers.
In most practical large scale fire simulations, the fire itself occupies a small part of
the computational domain. Hence, the fire is often not well resolved by
the numerical grid, in which case the vorticity generated in regions where there are
large deviations in density is not captured
directly. The mixing of air and combustion products occurs in the plume above the fire where
buoyancy is the dominant source of vorticity. In these calculations, the baroclinic torque
can be neglected to simplify the numerical solution. In simulations where detailed flame dynamics are
resolvable, the pressure term responsible for the baroclinic torque cannot be neglected, but for reasons
to be made clear below, must be treated differently than the other pressure term.
In neither of these cases is the Boussinesq approximation
invoked -- the fluid is still considered thermally-expandable; the divergence is non-zero; and the mass and
energy equations are not modified.

\subsection{The Equation for Pressure (Poisson's Equation)}

The reason for either neglecting the baroclinic torque or decomposing the pressure term in the
momentum equation is to simplify the
elliptic partial differential equation obtained by
taking the divergence of the momentum equation
\be \nabla^2 {\cal H} =
     -\dod{(\nabla \cdot \bu)}{t} - \nabla \cdot \bF
    \quad ; \quad \bF = - \bu\times\bo + \left( \frac{1}{\rho} -
   \frac{1}{\rho_\infty} \right) \nabla \tp - \frac{1}{\rho}
    \Big( (\rho-\rho_0) \bg + \bof_b + \nabla \cdot \btau_{ij} \Big)
   \label{pe}\ee
Note that the pressure $\tp$ appears on both sides of Eq.~(\ref{pe}). The
pressure on the right hand side is taken from the previous time step of the
overall explicit time-marching scheme. It can be neglected if the baroclinic torque is
not considered important in a given simulation. The pressure on the left hand side (incorporated
in the variable $\cal H$) is solved for directly.
The reason for the decomposition of the pressure term is so that the linear algebraic system
arising from the discretization of Eq.~(\ref{pe})
has constant coefficients ({\em i.e.} it is {\em separable}) and can be solved to machine accuracy
by a fast, direct ({\em i.e.} non-iterative) method that utilizes
Fast Fourier Transforms (FFT).
No-flux or forced-flow boundary conditions are specified by asserting that
\be \dod{{\cal H}}{n} = -F_n - \dod{u_n}{t} \label{bc} \ee
where $F_n$ is the normal component of $\bF$ at the vent or solid wall,
and $\partial u_n/\partial t$ is the prescribed rate of change
in the normal component of
velocity at a forced vent. Initially, the velocity is zero everywhere.

At open external boundaries the pressure-like term $\cal H$ is prescribed
depending on whether the flow is outgoing or incoming
\be \begin{array}{ll}
     {\cal H} = |\bu|^2/2  & \hbox{outgoing}    \\
     {\cal H} = 0          & \hbox{incoming}
     \end{array} \ee
The outgoing boundary condition assumes that the pressure perturbation
$\tp$ is zero at an outgoing boundary and that $\cal H$ is
constant along streamlines. The incoming boundary condition assumes that
$\cal H$ is zero infinitely far away. At the boundary between two meshes, the
pressure boundary condition is similar to that at an external open
boundary, except that where the flow is incoming, $\cal H$ is taken from the adjacent mesh.


\subsection{Large Eddy Simulation (LES)}
\label{LES}

The most distinguishing feature of any CFD model is its treatment of turbulence.
Chapter 1 contains a brief history of turbulence modeling as it has been applied to the fire
problem. Of the three main techniques of simulating turbulence, FDS contains only Large Eddy
Simulation (LES) and Direct Numerical Simulation (DNS). There is no Reynolds-Averaged Navier-Stokes (RANS)
capability in FDS.

LES is a technique used to model the dissipative processes (viscosity,
thermal conductivity, material diffusivity) that occur at length scales smaller than those that
are explicitly resolved on the numerical grid. This means that the parameters $\mu$, $k$ and $D$ in the equations
above cannot be used directly in most practical simulations. They must be replaced by surrogate expressions
that ``model'' their impact on the approximate form of the governing equations.
This section contains a simple explanation of how these terms are modeled
in FDS. Note that this discussion is quite different than what it typically found in the
literature, thus the reader is encouraged to consider other explanations of the technique in the references that are
listed in a review article by Pope~\cite{Pope:LES}.

Recall from Section~\ref{basicequations} that there is a small term in the energy equation
known as the dissipation rate $\epsilon$,
the rate at which kinetic energy is converted to thermal energy by viscosity.
To understand where this term originates, form an evolution equation for the kinetic energy of the fluid by
taking the dot product of the momentum equation (\ref{momentum}) with the velocity vector\footnote{In this section
it is convenient to work with the Lagrangian form of the conservation equations. }:
\be \rho \, \DoD{\bu}{t} \cdot \bu = \rho  \, \DoD{\left(|\bu|^2/2\right)}{t} = \rho \bof_b \cdot \bu -
\nabla p \cdot \bu + \nabla \cdot (\btau_{ij} \cdot \bu) - \epsilon \ee
As mentioned above $\epsilon$ is a neglible quantity in the energy equation. However, its functional form
is useful in representing the dissipation of kinetic energy from the resolved flow field.
Following the analysis of Smagorinsky~\cite{Smagorinsky:1}, the viscosity $\mu$ is modeled
\be \mu_{\hbox{\tiny LES}} = \rho \, (C_s\, \Delta)^2 \,
   \left(2 \; \overline{\bS}_{ij} \cdot \overline{\bS}_{ij} - \frac{2}{3} (\nabla \cdot \overline{\bu})^2 \right)^\ha \ee
where $C_s$ is an empirical constant and $\Delta$ is a length on the
order of the size of a grid cell.
The bar above the various quantities denotes that these are the resolved, or {\em filtered}, values, meaning
that they are computed on a numerical grid.
The other diffusive parameters,
the thermal conductivity and material diffusivity, are related to the turbulent viscosity by
\be k_{\hbox{\tiny LES}} = \frac{\mu_{\hbox{\tiny LES}} \, c_p}{\PR_t}
\quad ; \quad
 (\rho D)_{l,\hbox{\tiny LES}} =\frac{\mu_{\hbox{\tiny LES}}}{\SC_t} \ee
The turbulent Prandtl number $\PR_t$ and the turbulent Schmidt number $\SC_t$ are assumed to be
constant for a given scenario.

The model for the viscosity, $\mu_{\hbox{\tiny LES}}$, serves two roles: first, it provides a stabilizing
effect in the numerical
algorithm, damping out numerical instabilities as they arise in the flow field, especially where vorticity is
generated. Second, it has the appropriate mathematical form to describe the dissipation of kinetic energy from the flow.
Note the similar mathematical form of $\mu_{\hbox{\tiny LES}}$ and
the dissipation rate, $\epsilon$, defined in Eq.~(\ref{dissipation}).
In the parlance of the turbulence community, the dissipation
rate is related to the turbulent kinetic energy (most often denoted by $k$) by the
relation $\epsilon \approx k^{3/2}/L$, where $L$ is a length scale.

There have been numerous refinements of the original Smagorinsky
model~\cite{Deardorff:1,Germano:1,Lilly:1},
but it is difficult to assess the improvements offered by these newer
schemes for fires. There are two reasons for this. First, the structure of the
fire plume is so dominated by the large-scale resolvable eddies that
even a constant eddy viscosity gives results comparable to
those obtained using the Smagorinsky model~\cite{Baum:4}. Second, the lack
of precision in most large-scale fire test data makes it difficult to
assess the relative accuracy of each model.
The Smagorinsky model with constant $C_s$ produces satisfactory results
for most large-scale applications where boundary layers are not
well-resolved~\cite{FDS_VV_Guide_5}. In fact, experience to date using the simple form of LES described above
has shown that the best results are obtained when the Smagorinsky constant $C_s$ is set
as low as possible to maintain numerical stability. In other words, the most realistic
flow simulations are obtained when resolvable eddies are not ``damped'' by excessive
amounts of artificial viscosity.



\subsection{Direct Numerical Simulation (DNS)}
\label{DNS}

There are some flow scenarios where it is possible to use the molecular properties
$\mu$, $k$ and $D$ directly. Usually, this means that the numerical grid cells are on the
order of 1~mm or less, and the simulation is regarded as a
Direct Numerical Simulation (DNS).
For a DNS, the viscosity, thermal conductivity
and material diffusivity are approximated from kinetic theory because the temperature
dependence of each is important in combustion scenarios.
The viscosity of the species $\alpha$ is given by
\be \mu_\alpha = \frac{26.69\times 10^{-7} (W_\alpha \, T)^\ha}{\sigma_\alpha^2 \, \Omega_v}
\quad \quad \frac{\hbox{kg}}{\hbox{m s}} \ee
where $\sigma_\alpha$ is the Lennard-Jones
hard-sphere diameter ($\AA$) and $\Omega_v$ is the
collision integral, an empirical function of the
temperature $T$. The thermal conductivity of species $\alpha$ is given by
\be k_\alpha = \frac{\mu_\alpha \, c_{p,\alpha}}{\PR}  \quad \quad \frac{\hbox{W}}{\hbox{m K}}  \ee
where the Prandtl number $\PR$ is 0.7.
The viscosity and thermal conductivity of a gas mixture are given by
\be \mu_{\hbox{\tiny DNS}} = \sum_\alpha \; Y_\alpha \; \mu_\alpha  \quad ; \quad
k_{\hbox{\tiny DNS}} = \sum_\alpha \; Y_\alpha \; k_\alpha  \ee
The binary diffusion coefficient of species $\alpha$
diffusing into species $\beta$ is given by
\be D_{\alpha \beta} = \frac{2.66\times 10^{-7} \, T^{3/2} }{W_{\alpha \beta}^\ha \, \sigma_{\alpha \beta}^2 \, \Omega_D }
\quad \quad \frac{\hbox{m$^2$}}{s} \ee
where $W_{\alpha \beta}=2(1/W_\alpha+1/W_\beta)^{-1}$, $\sigma_{\alpha \beta}=(\sigma_\alpha+\sigma_\beta)/2$, and
$\Omega_D$ is the diffusion collision integral, an empirical
function of the temperature $T$~\cite{Poling:1}.
It is assumed that nitrogen is the dominant species in any combustion
scenario considered here, thus the diffusion coefficient in the
species mass conservation equations is that of the given species diffusing
into nitrogen
\be (\rho D)_{\alpha,\hbox{\tiny DNS}} = \rho \;  D_{\alpha 0} \ee
where species 0 is nitrogen.





\clearpage
\section{Combustion Model}

\label{combustionsection}

There are two types of combustion models used in FDS. The default model makes use of
the mixture fraction, a quantity representing the fuel and the products of combustion.
For the second model, individual gas species react according to
specified Arrhenius reaction parameters. This latter model is most often used in a
direct numerical simulation (DNS) where the diffusion of fuel and oxygen can be
modeled directly.
However, most often for large eddy simulations (LES), where the grid is not
fine enough to resolve the diffusion of fuel and oxygen,
the mixture fraction-based combustion model is assumed.

\subsection{Mixture Fraction Combustion Model}

Given a volume containing a mixture of gas species, a mixture fraction can be defined that
is the ratio of the mass of a subset of the species to the total mass present in the volume.
In combustion, the mixture fraction is a conserved quantity traditionally defined as the (mass) fraction of the
gas mixture that originates in
the fuel stream. Thus, at a burner surface the mixture fraction is 1 and in fresh air it is 0.
In a region where combustion has occurred this fraction will be comprised of any unburned fuel and
that portion of the combustion products that came from the fuel.
The mixture fraction is a function of space and time, commonly denoted $Z(\bx,t)$.
If it can be assumed that, upon mixing, the reaction of fuel and oxygen occurs rapidly and completely,
the combustion process is referred to as ``mixing-controlled.'' This implies that all species of
interest can be described in terms of the mixture fraction alone.
The correspondence between the
mass fraction of an individual species and the mixture fraction is called its ``state relation.''
FDS versions 2 through 4 assumed that the gas mixture could be uniquely determined by the mixture fraction alone,
an assumption that implies that fuel and oxygen react instantaneously upon mixing.

For many applications, ``mixed is burned'' is a reasonable assumption. However, for fire scenarios where
it cannot be assumed that fuel and oxygen react completely upon mixing, for example in under-ventilated
compartments. The mixture fraction itself remains a valid quantity,
but it can no longer be assumed that it completely defines the composition of the gas mixture. If fuel and oxygen are to mix
and not burn, at least two scalar variables are needed to describe the extent to which the fuel and oxygen react.
The strategy for moving beyond the ``mixed is burned'' model is as follows. Instead of solving a single
transport equation for the mixture fraction $Z$, multiple transport equations are solved for components of the mixture
fraction $Z_\alpha$. Fuel mass is still conserved, since $\sum Z_\alpha=Z$. For example, if $Z_1$ represents
the (unburned) fuel mass fraction,
$Y_\F$, and we define $Z_2 = Z-Z_1$, then $Z_2$ is the the mass fraction of burned fuel and is
the component of $Z$ that originates from the combustion products.
With this approach it is possible to account for the mixing of fuel and oxygen without burning. In the sections
to follow, various multi-step reaction mechanisms are discussed using this system of accounting. First, however, the
single-step mixture fraction model is described.



\subsubsection{A Single-Step, Instantaneous Reaction}

Consider a simple, one-step reaction of fuel and oxygen:
\be  \mathrm{C_xH_yO_zN_aM_b} +  \nu_\OTWO \, \mathrm{O_2}  \rightarrow  \nu_\COTWO \, \mathrm{CO_2} + \nu_\HTWOO \, \mathrm{H_2O} + \nu_\CO \, \mathrm{CO} +
     \nu_\So \, \mathrm{S}  + \nu_\NTWO \, \mathrm{N_2} + \nu_\M \mathrm{M}  \label{stoich} \ee
Note that the nitrogen in the fuel molecule is assumed to form $\mathrm{N_2}$ only. Addition product species can be
specified as some number of moles of an average molecular weight species $\mathrm{M}$.  These products are presumed to
not consume oxygen during their formation.  Soot is assumed to be a mixture of carbon and hydrogen with the hydrogen atomic
fraction given by $X_\Hy$. The stoichiometric coefficient, $\nu_\So$, represents the amount of fuel that is converted to soot. It is related to the
{\em soot yield}, $y_\So$, via the relation:
\be
   \nu_\So = \frac{W_\F}{W_\So} \; y_\So  \quad ; \quad  W_\So = X_\Hy W_\Hy + (1 - X_\Hy) W_\C  \label{soot_yield}
\ee
Likewise, the stoichiometric coefficient of CO, $\nu_\CO$, is related to the {\em CO yield}, $y_\CO$, via:
\be
   \nu_\CO = \frac{W_\F}{W_\CO} \; y_\CO  \label{CO_yield}
\ee
The yields of soot and CO are based on ``well-ventilated'' or ``post-flame'' measurements. The increased production of CO and soot in an under-ventilated
compartment will be addressed in the following sections.

The mixture fraction, $Z$, can be defined in terms of the mass fraction of fuel
and the carbon-carrying products of combustion:
\be
   Z = Y_\F + \frac{W_\F}{\hbox{x} W_\COTWO} \, Y_\COTWO + \frac{W_\F}{\hbox{x} W_\CO} \, Y_\CO  + \frac{W_\F}{\hbox{x} W_\So} \, Y_\So
\label{Zdef} \ee
The mixture fraction satisfies the conservation equation
\be \rho \DoD{Z}{t} = \nabla \cdot \rho D \nabla Z \label{Zeqn} \ee
obtained by taking a linear combination of the transport equations for the fuel and the carbon carrying products.
If it is assumed that combustion occurs so rapidly that the fuel and oxygen cannot
co-exist, then both simultaneously vanish at a flame surface:
\be Z(\bx,t) = Z_f \quad ; \quad
Z_f = \frac{Y_\OTWO^\infty}{s \, Y_\F^I + Y_\OTWO^\infty} \label{flamesheet} \ee
and all species can be related to $Z$ via the  ``state relations'' shown in Fig.~\ref{staterelations}.
\begin{figure}
\begin{minipage}[t]{4.1in}
\includegraphics[height=4.in]{FIGURES/State_Relations_3_Full_Labelled}
\caption{State relations for methane.}
\label{staterelations}
\end{minipage}
\end{figure}
In versions of FDS prior to 5, this one-step, instantaneous reaction of fuel and oxygen was assumed. However, starting in version 5, a more
generalized formulation has been implemented and is described next.


\subsubsection{A Single-Step Reaction, but with Local Extinction}

\label{extinction}

The physical limitation of the single-step reaction model described in the previous section is that it assumes fuel and
oxygen burn instantaneously when mixed. For large-scale, well-ventilated
fires, this is a good assumption. However, if a fire is in an
under-ventilated compartment, or if a suppression agent like water
mist or CO$_2$ is introduced, or if the shear layer between fuel and oxidizing streams
has a sufficiently large local strain rate,
fuel and oxygen may mix but may not burn.
The physical mechanisms underlying these phenomena are complex, and
even simplified models still rely on an accurate prediction
of the flame temperature and local strain rate.
Subgrid-scale modeling of gas phase suppression and
extinction is still an area of active research in the combustion
community.

Simple empirical rules can be used to predict local
extinction based on the oxygen concentration and temperature of
the gases in the vicinity of the flame sheet.
Figure~\ref{plotsupp} shows values of temperature and oxygen
concentration for which burning can and cannot take place. A derivation of the model,
based on the critical flame temperature concept, is given in Appendix~\ref{mowrer_model}.
\begin{figure}[ht]
\begin{minipage}[t]{4.1in}
\includegraphics[width=4.in]{FIGURES/Extinction_Criteria}
\caption{Oxygen-temperature phase space showing where combustion
is allowed and not allowed to take place.}
\label{plotsupp}
\end{minipage}
\end{figure}
Note that once the local state of the gases falls into the ``No Burn'' zone,
the state relations (Fig.~\ref{staterelations}) are no longer valid for
values of $Z$ below stoichiometric, since now some fuel may be mixed
with the other combustion products. Essentially, there
are now two reactions to consider -- the ``null'' reaction, where fuel and oxygen simply mix and do not burn;
and the ``complete'' reaction,
where fuel and oxygen react and form products according to Eq.~(\ref{stoich}). Note that the term ``complete'' does not imply that
no soot or CO is formed, but rather that their respective production rates are proportional to the fuel consumption rate.

With the definition of the mixture fraction, Eq.~(\ref{Zdef}), in mind, consider a partitioning of $Z$ into the following components:
\begin{eqnarray}  Z_1 &=& Y_\F  \label{Z1first}  \\*[.1in]
                  Z_2 &=& \frac{W_\F}{\hbox{x} \, W_\COTWO } \,  Y_\COTWO  + \frac{W_\F}{\hbox{x} \, W_\CO } \, Y_\CO  + \frac{W_\F}{\hbox{x} \, W_\So }  \label{Z2first} \, Y_\So
\end{eqnarray}
such that $Z=Z_1+Z_2$. Transport equations are required for both $Z_1$ and $Z_2$.
At the burner surface, $Z_1$ is assigned the mass flux of fuel,
while the mass flux for $Z_2$ is zero. In other words, no combustion products are generated at the fuel source. Where fuel and
oxygen co-exist, a reaction occurs if conditions are favorable in the sense shown by Fig.~\ref{plotsupp}.
If a reaction occurs, $Z_1$ is converted to $Z_2$ representing the
conversion of fuel to products. The heat release rate of the fire is obtained by multiplying the fuel consumption rate by the
heat of combustion.

We expand upon this by allowing for the fuel stream to be diluted with an inert gas (nitrogen) and we note that since we are using our extra
parameter to determine extinction, that the yields of CO and soot are still fixed.  Thus we can restate
Eq.~(\ref{Z1first}) and Eq.~(\ref{Z2first}) as:
\begin{eqnarray} Z_1 &=& \frac{Y_\F}{Y_\F^I} \\*[.1in]
                 Z_2 &=& \frac{W_\F}{\big[\hbox{x}-\nu_\CO-(1-X_\Hy) \nu_\So \big] \, W_\COTWO } \frac{Y_\COTWO}{Y_\F^I}
\end{eqnarray}
where $Y_\F^I$ is the fuel mass fraction at the burner surface ($Y_\NTWO^I$ would be the diluent mass fraction at the
burner surface).

Since the mixture fraction variables result from linear combinations of the species transport equations, the converse
is also true -- that species mass fractions are linear combinations of the mixture fraction variables.
The mass fractions of the species in the mixture, $Y_{\alpha}(Z_1,Z_2)$, are found via:

\parbox{2.5in}{
\begin{eqnarray*}  Y_\F     &=& Y_\F^I \; Z_1  \\*[.1in]
                  Y_\NTWO  &=& (1 - Z) \; Y_\NTWO^\infty  + Y_\NTWO^I \; Z_1 + \frac{\nu_\NTWO W_\NTWO}{W_\F} \; Y_\F^I \; Z_2  \\*[.1in]
                  Y_\OTWO  &=& (1 - Z) \; Y_\OTWO^\infty - \frac{\nu_\OTWO W_\OTWO}{W_\F} \; Y_\F^I \; Z_2 \\*[.1in]
                  Y_\COTWO &=& \frac{\nu_\COTWO W_\COTWO}{W_\F} \; Y_\F^I \; Z_2 \end{eqnarray*} }
\hfill \parbox{3.5in}{\begin{eqnarray}
                  Y_\HTWOO &=& \frac{\nu_\HTWOO W_\HTWOO}{W_\F} \; Y_\F^I \; Z_2 \\*[.1in]
                  Y_\CO    &=& \frac{\nu_\CO W_\CO}{W_\F} \; Y_\F^I \; Z_2  \\*[.1in]
                  Y_\So    &=& \frac{\nu_\So W_\So}{W_\F} \; Y_\F^I \; Z_2 \\*[.1in]
                  Y_\M     &=& \frac{\nu_\M W_\M}{W_\F} \; Y_\F^I \; Z_2
\end{eqnarray} }

\noindent
The stoichiometric coefficients are defined:

\parbox{2.5in}{
\begin{eqnarray*}  \nu_\NTWO  &=& \frac{\hbox{a}}{2}\\*[.1in]
                  \nu_\OTWO  &=& \nu_\COTWO + \frac{\nu_\CO+\nu_\HTWOO-z}{2}\\*[.1in]
                  \nu_\COTWO &=& \hbox{x} - \nu_\CO - (1-X_\Hy) \nu_\So  \\*[.1in]
                  \nu_\M     &=& \hbox{b}  \end{eqnarray*} }
\hfill \parbox{3.5in}{\begin{eqnarray}
                  \nu_\HTWOO &=& \frac{\hbox{y}}{2}- X_\Hy \nu_\So\\*[.1in]
                  \nu_\CO    &=& \frac{W_\F}{W_\CO} \; y_\CO \\*[.1in]
                  \nu_\So    &=& \frac{W_\F}{W_\So} \; y_\So
\end{eqnarray} }
It is important to note that the definitions of $Z_1$ and $Z_2$, unlike in the single parameter model,
do not imply anything regarding the rate of combustion, only that the combustion occurs in a single step.


\subsubsection{CO Production (Two-Step Reaction with Extinction)}

\label{co_production}

The previous section describes the ``complete'' reaction as the conversion of fuel to
products such that the production rate of each product species is proportional to the fuel consumption rate.
This means that for each fuel molecule, fixed amounts of CO$_2$, H$_2$O, CO, and soot are formed and these products
persist in the plume indefinitely with no further reaction. This is not an unreasonable assumption if
the purpose of the fire simulation is to assess the impact of the fire on the larger space.
However, in under-ventilated fires, soot and CO are produced at higher rates,
and exist within the fuel-rich flame envelope at higher concentrations,
than would otherwise be predicted with a single set of fixed yields that are based on post-flame measurements. To account for the
production of CO and its eventual oxidation at the flame envelope or within a hot upper layer,
an additional reaction is now needed:
\begin{eqnarray}
\mathrm{C_xH_yO_zN_aM_b} +  \nu_\OTWO' \mathrm{O_2}  &\rightarrow&  \nu_\HTWOO \mathrm{H_2O} + (\nu_\CO'+ \nu_\CO) \, \mathrm{CO} +
     \nu_\So \, \mathrm{S}  + \nu_\NTWO \, \mathrm{N_2} + \nu_\M \mathrm{M}  \\*[.1in]
\nu_\CO' \; \Big[ \mathrm{CO} + \ha \mathrm{O_2}  &\rightarrow&  \mathrm{CO_2}  \Big]
\label{3reac} \end{eqnarray}
The brackets around the second reaction are there merely to emphasize that the sum of the two reactions equal Eq.~(\ref{stoich}).
There are two stoichiometric coefficients for CO -- the first, $\nu_\CO'=\hbox{x}-(1-X_\Hy) \nu_\So-\nu_\CO$,
represents CO that is produced in the first
step of the reaction that can potentially be converted to CO$_2$ assuming the conditions are favorable. $\nu_\CO'$ is equivalent to $\nu_\COTWO$ in
Eq.~(\ref{stoich}). The second coefficient, $\nu_\CO$,
is the so-called ``well-ventilated,'' or ``post-flame,'' value that was introduced in the previous section. The proposed model of CO production
still does not contain the necessary kinetic mechanism to predict the ``post-flame'' concentration of CO without the prescription of the
measured value of the post-flame CO yield. Rather, the proposed model includes the production of large amounts of CO in the first step of a two-step
reaction, followed by a partial conversion to CO$_2$ if there is a sufficient amount of oxygen present.

To describe the composition of the gas species, the mixture fraction, $Z$, must now be decomposed into three components:
\begin{eqnarray}  Z_1 &=& \frac{Y_\F}{Y_\F^I}   \\*[.1in]
                  Z_2 &=& \frac{W_\F}{\big[\hbox{x}-(1-X_\Hy) \nu_\So \big] \, W_\CO }    \;  \frac{Y_\CO}{Y_\F^I} \\*[.1in]
                  Z_3 &=& \frac{W_\F}{\big[\hbox{x}-(1-X_\Hy) \nu_\So \big] \, W_\COTWO } \;  \frac{Y_\COTWO}{Y_\F^I}  \end{eqnarray}
Here, $\hbox{x}-(1-X_\Hy) \nu_\So$ represents the number of carbon atoms in the fuel molecule that are oxidized; that is, the carbon atoms that are not converted to soot.
The decomposition of $Z$ into three components is numerically convenient. However, the recovery of the
individual species mass fractions requires some care.  The mass fraction of any species in the mixture,
$Y_\alpha(Z_1,Z_2,Z_3)$, is still found via linear combinations of the mixture fraction variables:

\parbox{2.5in}{
\begin{eqnarray*}  Y_\F    &=& Z_1 \; Y_\F^I   \\*[.1in]
                  Y_\NTWO  &=& (1 - Z) \; Y_\NTWO^\infty  + Y_\NTWO^I \; Z_1 + \frac{\nu_\NTWO W_\NTWO}{W_\F} \; Y_\F^I \; (Z_2+Z_3)  \\*[.1in]
                  Y_\OTWO  &=& (1 - Z) \; Y_\OTWO^\infty  - \frac{W_\OTWO \; Y_\F^I}{W_\F} (\nu_\OTWO'  \; Z_2 +\nu_\OTWO  \; Z_3) \\*[.1in]
                  Y_\COTWO &=& \frac{\nu_\COTWO W_\COTWO}{W_\F} \; Y_\F^I \; Z_3 \end{eqnarray*} }
\hfill \parbox{3.5in}{\begin{eqnarray}
                  Y_\HTWOO &=& \frac{\nu_\HTWOO W_\HTWOO}{W_\F} \; Y_\F^I \; (Z_2 + Z_3) \\*[.1in]
                  Y_\CO    &=& \frac{(\nu_\CO+\nu_\CO') W_\COTWO}{W_\F} \; Y_\F^I \; Z_2 \\*[.1in]
                  Y_\So    &=& \frac{\nu_\So W_\So}{W_\F} \; Y_\F^I \; (Z_2 + Z_3) \\*[.1in]
                  Y_\M     &=& \frac{\nu_\M W_\M}{W_\F} \; Y_\F^I \; (Z_2 + Z_3)
\end{eqnarray} }
The stoichiometric coefficients are defined:

\parbox{2.5in}{
\begin{eqnarray*} \nu_\NTWO  &=& \frac{\hbox{a}}{2}\\*[.1in]
                  \nu_\OTWO' &=& \frac{\nu_\CO'+\nu_\HTWOO}{2}\\*[.1in]
                  \nu_\OTWO  &=& \nu_\COTWO + \frac{\nu_\CO+\nu_\HTWOO-z}{2}\\*[.1in]
                  \nu_\COTWO &=& \hbox{x} - \nu_\CO - (1-X_\Hy) \nu_\So \\*[.1in]
                  \nu_\M     &=& \hbox{b} \end{eqnarray*} }
\hfill \parbox{3.5in}{\begin{eqnarray}
                  \nu_\HTWOO &=& \frac{\hbox{y}}{2}- X_\Hy \nu_\So\\*[.1in]
                  \nu_\CO'   &=& x - \nu_\CO - (1-X_\Hy) \nu_\So \\*[.1in]
                  \nu_\CO    &=& \frac{W_\F}{W_\CO} \; y_\CO \\*[.1in]
                  \nu_\So    &=& \frac{W_\F}{W_\So} \; y_\So
\end{eqnarray} }
Although these formulae appear complicated, most are determined directly from the composition of the fuel molecule. The only information
expected of the modeler are the fuel composition, the soot and CO yields, and the atomic fraction of hydrogen in the soot.


\subsubsection{Heat Release Rate}

The discussion of the various multi-step reactions above is essentially book-keeping, the accounting of the gas
molecules formed in the combustion process. But what of the heat released?

When tracking species with two mixture fraction parameters, there is a single step reaction, the conversion of fuel into a
fixed, predefined set of combustion products.  Fuel and oxygen react within a grid cell and release energy according to
the rate of fuel consumption:
\be \dq''' = \dm_\F''' \Delta H_{\F} \ee
The amount of fuel in the grid cell is obtained directly from the definition of $Z_1$ whereas the amount of $O_2$ is
obtained using the state relationships discussed in the prior section.  Combustion is either allowed or
disallowed\footnote{Note that the user has control over the parameters associated with local gas phase extinction.}
using the relation shown in Fig.~\ref{plotsupp}.
If combustion is allowed to occur in a grid
cell, the single step combustion is assumed to be
infinitely fast.  Combustion will consume all of either the fuel or the oxygen in the grid cell.


In the case of the two-step, three parameter mixture fraction model, the first step converts the
fuel to CO and other combustion products, and the second step oxidizes the CO into $CO_2$.  The first step is
determined as it is for the single step reaction.  The second step, however, is assumed to be rate dependent.
Determining if the second step can take place in a grid cell is done in one of two ways.  For most LES
simulations, flame temperatures will not be resolved on the grid because the heat released is smeared out over a grid cell much larger than
the flame sheet thickness.  Since a grid cell with heat released from the first step has a flame, CO conversion is
always allowed in these grid cells.  In all other grid cells, the local temperature is used to perform a finite-rate
computation that determines the rate of CO conversion.



\subsubsection{On-Going Research}

One can envision extending the three parameter mixture fraction model to a fourth parameter for soot production:
\be Z_4 = \frac {W_F}{x W_S} Y_S \ee
This extenstion would allow for combustion models that predict soot formation along with extinction and CO production.

Additional partitionings of the mixture fraction could be added as well; however, there is a point of diminishing
returns.  The present advantage of using a mixture fraction approach is to track the movement of a number of gas
species using a smaller number of scalar variables at the expense of having to use state relationships to obtain
species mass fractions.  As more mixture fraction variables are added, the net savings in fewer species transport
equations vs. the expense of state relationships will diminish.

\clearpage



\subsection{Finite-Rate Reaction (DNS)}

In a DNS calculation, the fine grid resolution enables the direct modeling of the diffusion of chemical species (fuel,
oxygen, and combustion products).  Since the flame is being resolved in a DNS calculation, the local gas
temperatures can be used to determine the reaction kinetics.  Thus, it is possible to implement a relatively simple
set of one or more chemical reactions to model the combustion. Consider the reaction of oxygen and a hydrocarbon
fuel

\be  \mathrm{ C_xH_y + \nu_{O_2}  \, O_2 \longrightarrow
     \nu_{CO_2}  \,  CO_2 +
     \nu_{H_2 O}  \, H_2O }   \ee
If this were modeled as a single-step reaction, the reaction rate would be given by the expression
\be \frac{d[\mathrm{C_xH_y}]}{dt} = -B \, [\mathrm{C_xH_y}]^a \, [\mathrm{O_2}]^b \, e^{-E/RT}
   \label{reaction} \ee
Suggested values of $B$, $E$, $a$ and $b$ for various hydrocarbon
fuels are given
in Refs.~\cite{Puri:1,Westbrook:1}. It should be understood that the
implementation of any of these one-step reaction schemes is still very
much a research exercise because it is not universally accepted that
combustion phenomena can be represented by such a simple mechanism.
Improved predictions of the heat release rate may be possible by considering a multi-step set of reactions.
However, each additional gas species defined in the computation incurs a roughly 5~\% increase in the CPU time.





\clearpage
\section{Thermal Radiation}

Energy transport consists of convection, conduction and
radiation. Convection of heat is accomplished via the solution of the
basic conservation equations. Gains and losses of heat via conduction
and radiation are represented by the divergence of the heat flux
vector in the energy equation, $\nabla \cdot \dbq''$. This section
describes the equations associated with the radiative part, $\dbq''_r$.


The Radiative Transport Equation (RTE) for an absorbing/emitting
and scattering medium is
\be \bs \cdot \nabla I_{\la}(\bx,\bs) =
 -\Big[ \kappa(\bx,\la) + \sigma_s(\bx,\la) \Big] \;
I_\la(\bx,\bs) +B(\bx,\la) + \frac{\sigma_s(\bx,\la)}{4\pi}
\int_{4\pi}\Phi(\bs,\bs') \; I_{\la}(\bx,\bs') \; d\bs'
\label{RTEbasic} \ee
where $I_{\la}(\bx,\bs)$ is the radiation intensity at wavelength
$\la$, $\bs$ is the direction vector of the intensity,
$\kappa(\bx,\la)$ and $\sigma_s(\bx,\la)$ are the local absorption
and scattering coefficients,
respectively, and $B(\bx,\la)$ is the emission source term.
The integral on the
right hand side describes the in-scattering from other directions.
In the case of a non-scattering gas the RTE becomes
\be \bs \cdot \nabla I_{\la}(\bx,\bs) = \kappa(\bx,\la) \; \Big[ I_b(\bx)
- I_\la(\bx,\bs) \Big] \label{RTE} \ee
where $I_b(\bx)$ is the source term given by the Planck function (see below).

In practical simulations the spectral ($\la$) dependence cannot be solved
accurately. Instead, the radiation spectrum is divided into
a relatively small number of bands and a separate RTE is derived for
each band. The band specific RTE is
\be   \bs \cdot \nabla I_n(\bx,\bs) = \kappa_n(\bx) \;
        \left[ I_{b,n}(\bx) - I_n(\bx,\bs) \right],\;\; n = 1...N
\label{bandRTE} \ee
where $I_n$ is the intensity integrated over the band $n$, and $\kappa_n$
is the appropriate mean absorption coefficient inside the band. The
source term can be written as a fraction of the blackbody radiation
\be I_{b,n} = F_n(\la_{\rm min},\la_{\rm max}) \; \sigma \; T^4/\pi \ee
where $\sigma$ is the Stefan-Boltzmann constant.
The calculation of factors $F_n$ is explained in Ref.~\cite{Siegel:1}.
When the intensities corresponding to the bands are known, the total
intensity is calculated by summing over all the bands
\be I(\bx,\bs) = \sum_{n=1}^N I_n(\bx,\bs) \ee

Even with a reasonably small number of bands, solving multiple
RTEs is very time consuming. Fortunately, in most large-scale fire
scenarios soot is the most important combustion product controlling the
thermal radiation from the fire and hot smoke. As the radiation spectrum of
soot is continuous, it is possible to assume that the gas behaves as a
gray medium.  The spectral dependence is then lumped into one
absorption coefficient ($N=1$) and the source term is given by the
blackbody radiation intensity
\be I_b(\bx) = \sigma \, T(\bx)^4/\pi \ee
This is the default mode of FDS and appropriate for most problems of
fire engineering. In optically thin flames, where the amount of soot
is small compared to the amount of $\rm CO_2$ and water, the gray gas
assumption may produce significant overpredictions of the emitted
radiation. From a series of numerical experiments it has been found
that six bands ($N=6$) are usually enough to improve the accuracy in
these cases.  The limits of the bands are selected to give an accurate
representation of the most important radiation bands of $\rm CO_2$ and
water. If the absorption of the fuel is known to be important,
separate bands can be reserved for fuel, and the total number of
bands is increased to nine ($N=9$).
For simplicity, the fuel is assumed to be $\rm CH_4$.
The limits of the bands are shown in Table~\ref{banditos}.

\begin{table}[ht]
\caption{Limits of the spectral bands.}
\vspace{0.1in}
\label{banditos}
\small
\begin{tabular}{|*{10}{c|}}
\hline
\hspace{0.5in} \underline{9 Band Model} \hspace{0.5in} & 1  & 2  & 3 & 4  & 5 & 6 & 7 & 8 & 9 \\ \cline{2-10}
                                     & Soot   & CO$_2$       & CH$_4$ & Soot & CO$_2$ & H$_2$O & H$_2$O       & Soot & Soot          \\
\raisebox{1.5ex}[0pt]{Major Species} &        & H$_2$O, Soot & Soot   &      & Soot   & Soot   & CH$_4$, Soot &      &      \\ \hline
\multicolumn{1}{c}{$\nu$ (1/cm)}
             & \multicolumn{1}{@{\hspace{-.2in}10000}c@{}}{ }
             & \multicolumn{1}{@{\hspace{-.2in} 3800}c@{}}{ }
             & \multicolumn{1}{@{\hspace{-.2in} 3400}c@{}}{ }
             & \multicolumn{1}{@{\hspace{-.2in} 2800}c@{}}{ }
             & \multicolumn{1}{@{\hspace{-.2in} 2400}c@{}}{ }
             & \multicolumn{1}{@{\hspace{-.2in} 2174}c@{}}{ }
             & \multicolumn{1}{@{\hspace{-.2in} 1429}c@{}}{ }
             & \multicolumn{1}{@{\hspace{-.2in} 1160}c@{}}{ }
             & \multicolumn{1}{@{\hspace{-.2in} 1000}c@{50}}{ } \\
\multicolumn{1}{c}{$\la$ ($\mu$m)}
             & \multicolumn{1}{@{\hspace{-.2in} 1.00}c@{}}{ }
             & \multicolumn{1}{@{\hspace{-.2in} 2.63}c@{}}{ }
             & \multicolumn{1}{@{\hspace{-.2in} 2.94}c@{}}{ }
             & \multicolumn{1}{@{\hspace{-.2in} 3.57}c@{}}{ }
             & \multicolumn{1}{@{\hspace{-.2in} 4.17}c@{}}{ }
             & \multicolumn{1}{@{\hspace{-.2in} 4.70}c@{}}{ }
             & \multicolumn{1}{@{\hspace{-.2in} 7.00}c@{}}{ }
             & \multicolumn{1}{@{\hspace{-.2in} 8.62}c@{}}{ }
             & \multicolumn{1}{@{\hspace{-.2in} 10.0}c@{200}}{ } \\ \hline
\underline{6 Band Model}  & 1  & 2  & \multicolumn{2}{|c|}{3} & 4  & \multicolumn{3}{|c|}{5} & 6  \\ \cline{2-10}
          & Soot   & CO$_2$       & \multicolumn{2}{|c|}{CH$_4$      } & CO$_2$ & \multicolumn{3}{|c|}{H$_2$O, CH$_4$, Soot} & Soot  \\
\raisebox{1.5ex}[0pt]{Major Species} &        & H$_2$O, Soot & \multicolumn{2}{|c|}{Soot} & Soot   & \multicolumn{3}{|c|}{  } &       \\
               \hline
\end{tabular}
\end{table}
\normalsize


For the calculation of the gray or band-mean absorption coefficients,
$\kappa_n$, a narrow-band model, RadCal~\cite{RadCal}, has been
implemented in FDS. At the start of a simulation the absorption
coefficient(s) are tabulated as a function of mixture fraction and
temperature. During the simulation the local absorption coefficient is
found by table-lookup.

In calculations of limited spatial resolution, the source term, $I_b$,
in the RTE requires special treatment in the neighborhood of the flame
sheet because the temperatures are smeared out over a grid
cell and are thus considerably lower than
one would expect in a diffusion flame.
Because of its fourth-power dependence on the temperature,
the source term must be modeled in those grid cells cut by the flame
sheet. Elsewhere, there is greater confidence in the computed temperature,
and the source term can be computed directly
\be \kappa \; I_b = \left\{ \begin{array}{cl}
    \kappa \, \sigma \, T^4/\pi                                           & \hbox{Outside flame zone} \\
    \max(\chi_r \, \dot{q}'''/4 \pi \; , \; \kappa \, \sigma \, T^4/\pi)  & \hbox{Inside flame zone}
    \end{array} \right.  \label{radapprox} \ee
Here, $\dot{q}'''$ is the chemical heat release rate per unit volume and
$\chi_r$ is an empirical estimate
of the {\em local} fraction of that energy emitted as
thermal radiation.\footnote{The radiative fraction, $\chi_r$, is a useful quantity in fire
science. Usually, it is understood to be the fraction of the total heat release rate that takes the form of thermal radiation. For
most combustibles, $\chi_r$ is between 0.3 and 0.4. However, in Eq.~(\ref{radapprox}), $\chi_r$ is interpreted as the fraction
of energy radiated from the combustion region.
For a small fire ($D<1$~m), the local $\chi_r$ is approximately equal to its
global counterpart. However, as the fire increases in size, the global
value will typically decrease due to a net re-absorption of the thermal
radiation by the increasing smoke mantle.}
Near the flame in large scale calculations, neither $\kappa$ nor $T$ can be computed
reliably, hence the inclusion of the empirical radiation loss term which is designed to partition the fire's
heat release rate in accordance with measured values.


The boundary condition for the radiation intensity leaving
a gray diffuse wall is given as
\be I_w(\bs) = \frac{\epsilon\,\sigma\,T_w^4}{\pi} + \frac{1-\epsilon}{\pi}
 \int_{\bs'\cdot \bn_w < 0} I_w(\bs')\; |\bs'\cdot \bn_w | \; d\bs'
 \label{RTEbc} \ee
where $I_w(\bs)$ is the intensity at the wall, $\epsilon$ is the
emissivity, and $T_{w}$ is the wall surface temperature.
The radiant heat flux vector $\dbq_r''$ is defined
\be \dbq_r''(\bx) = \int_{4\pi} \; \bs' \, I(\bx,\bs') \; d\bs'   \ee
The gas phase contribution to the radiative loss term in the energy
equation is
\be -\nabla \cdot \dbq_r''(\bx)(\mbox{gas}) =
    \kappa(\bx) \, \left[ U(\bx) - 4 \pi \, I_b(\bx) \right]  \quad ; \quad
    U(\bx) = \int_{4\pi} \, I(\bx,\bs') \, d\bs'
\ee
In words, the net radiant energy gained by a grid cell is the
difference between that which is absorbed and that which is emitted.






\clearpage
\section{Solid Phase Model}

FDS assumes that solid obstructions consist of multiple layers, with each
layer composed of multiple material components that can undergo multiple thermal degradation reactions.
Each reaction forms a combination of solid residue ({\em i.e.} another material component), water vapor, and/or fuel vapor. Heat
conduction is assumed only in the direction normal to the surface. This section describes the single
mass and energy conservation equation for solid materials, plus the various coefficients, source terms, and
boundary conditions.

\subsection{The Heat Conduction Equation for a Solid}

A one-dimensional heat conduction equation for the solid phase
temperature $T_s(x,t)$ is applied in the direction $x$ pointing into
the solid (the point $x = 0$ represents the surface)\footnote{In cylindrical and spherical coordinates, the heat conduction
equation is written
\be
  \rho_s c_s \; \dod{T_s}{t} = \frac{1}{r} \, \dod{}{r}
  \left(rk_s \dod{T_s}{r} \right)+\dq_s'''
  \label{1dheatcyl} \quad ; \quad
  \rho_s c_s \; \dod{T_s}{t} = \frac{1}{r^2} \, \dod{}{r}
  \left(r^2k_s \dod{T_s}{r} \right)+\dq_s'''
  \label{1dheatcyl}
\ee
FDS offers the user these options, with the assumption that the obstruction is not actually recti-linear, but rather
cylindrical or spherical in shape. This option is useful in describing the behavior of small, complicated ``targets''
like cables or heat detection devices.}
\be
  \rho_s c_s \; \dod{T_s}{t} = \dod{}{x} k_s \dod{T_s}{x} +
    \dq_s'''
  \label{1dheat}
\ee
Section~\ref{matcoefs} describes the component-averaged material
properties, $k_s$ and $\rho_s c_s$. The source term, $\dq_s'''$,
consists of chemical reactions and radiative absorption:
\be
  \dq_s'''=\dq_{s,c}'''+\dq_{s,r}'''
\ee
Section~\ref{pyrosection} describes the term $\dq_{s,c}'''$, which
is essentially the heat production (loss) rate given by the  pyrolysis
models for different types of solid and liquid fuels.
Section~\ref{inradsection} describes the term
$\dq_{s,r}'''$, the radiative absorption and emission in depth.
Section~\ref{conflux} describes the convective heat transfer to the
solid surface. Finally, Section~\ref{matbc} describes how the
radiative and convective heat fluxes are applied as boundary
conditions for Eq.~(\ref{1dheat}).




\subsection{Component-Averaged Thermal Properties}
\label{matcoefs}

The conductivity and volumetric heat capacity of the solid are defined
\be
   k_s = \sum_{\alpha=1}^{N_m} X_\alpha \; k_{s,\alpha} \quad ; \quad
   \rho_s c_s = \sum_{\alpha=1}^{N_m} \rho_{s,\alpha} \; c_{s,\alpha}
\ee
$N_m$ is the number of material components forming the
solid. $\rho_{s,\alpha}$ is the
{\em component density}
\be
  \rho_{s,\alpha}=\rho_s \, Y_\alpha
\ee
where $\rho_s$ is the density of the composite material and $Y_\alpha$ is the mass fraction of material component $\alpha$.
The solid density is the sum of the component densities
\be
  \rho_s = \sum_{\alpha=1}^{N_m} \rho_{s,\alpha}
\ee
$X_\alpha$ is the volume fraction of component $\alpha$
\be
  X_\alpha = \frac{\rho_{s,\alpha}}{\rho_\alpha}  \left/ \sum_{\alpha'=1}^{N_m}\frac{\rho_{s,\alpha'}}{\rho_{\alpha'} }  \right.
  \label{volfrac}
\ee
where $\rho_\alpha$ is the density of material $\alpha$ in its pure form.
Multi-component solids are defined by specifying the mass fractions, $Y_\alpha$, and densities, $\rho_\alpha$,
of the individual components of the composite.



\subsection{Pyrolysis Models}
\label{pyrosection}

This section describes how solid phase reactions and the chemical
source term in the solid phase heat conduction equation,
$\dot{q}_{s,c}'''$,  are modeled. This is what is commonly referred to
as the ``pyrolysis model,'' but it actually can represent any number
of reactive processes, including evaporation, charring, and internal
heating.


\subsubsection{Specified Heat Release Rate}

Often the intent of a fire simulation is merely to predict the
transport of smoke and heat from a {\em specified} fire. In other
words, the heat release rate is a user input, not something the model
predicts. In these instances, the desired HRR is translated into a
mass flux for fuel at a given solid surface, which can be thought of
as the surface of a burner: \be \dm_f'' = \frac{ f(t) \;
\dq_{\hbox{user}}''}{\Delta H} \ee Usually, the user specifies a
desired heat release rate per unit area, $\dq_{\hbox{user}}''$, plus a
time ramp, $f(t)$, and the mass loss rate is computed accordingly.

\subsubsection{Solid Fuels}

Solids can undergo simultaneous reactions under the following assumptions:
\begin{itemize}
\setlength{\itemsep}{0.0in}
\item instantaneous release of volatiles from solid to the gas phase,
\item local thermal equilibrium between the solid and the volatiles,
\item no condensation of gaseous products, and
\item no porosity effects\footnote{Although porosity effects are not explicitly included in the model, it is possible to account for it
because the volume fractions defined by Eq.~(\ref{volfrac}) need not
sum to unity, in which case the thermal conductivity and absorption
coefficient are effectively reduced.}
\end{itemize}
Each material component may undergo several competing reactions, and
each of these reactions may produce some other solid component
(residue), gaseous fuel, and/or water vapor according to
the yield coefficients $\nu_s$, $\nu_f$ and $\nu_w$, respectively.
These coefficients should usually satisfy $\nu_s + \nu_f + \nu_w = 1$,
but smaller yields may also be used to take into account the gaseous
products that are not explicitly included in the simulation.

Consider material component $\alpha$ that undergoes $N_{r,\alpha}$
separate reactions. We will use the index $\beta$ to represent one of
these reactions:
\be \hbox{Material}_\alpha \rightarrow \nu_{s,\alpha \beta} \;
    \hbox{Residue}_{\alpha \beta} + \nu_{w,\alpha \beta} \; \hbox{H$_2$O} + \nu_{f,\alpha \beta} \; \hbox{HC} \ee
The local density of material component $\alpha$ evolves in time
according to the solid phase species conservation equation
\be
  \dod{ }{t} \left( \frac{\rho_{s,\alpha}}{\rho_{s0}} \right) =
    -\sum_{\beta=1}^{N_{r,\alpha}} r_{\alpha \beta} + S_\alpha
\ee
which says that the mass of conponent $\alpha$ is consumed by the
solid phase reactions $r_{\alpha \beta}$ and produced by other
reactions. $r_{\alpha \beta}$ is the rate of reaction $\beta$ in units
(1/s) and $\rho_{s0}$ is the initial density of the material layer.
$S_\alpha$ is the production rate of material component
$\alpha$ as a result of the reactions of the other
components. The reaction rates are functions of local mass
concentration and temperature, and calculated as a combination of
Arrhenius and power functions:
\be
r_{\alpha \beta} =
    \left( \frac{\rho_{s,\alpha}}{\rho_{s0}}\right)^{n_{s,\alpha\beta}}
    A_{\alpha \beta} \; \exp \left(-\frac{E_{\alpha\beta}}{RT_s}\right)
    \; \max \big[0,T_s-T_{thr,\alpha \beta} \big]^{n_{t,\alpha\beta}}
\ee
where $T_{thr,\alpha \beta}$ is a threshold temperature that can be
used to dictate that the reaction must not occur below a
user-specified temperature. By default, the term is deactivated
($T_{thr,\alpha\beta}=0$ K).

The production term $S_\alpha$ is the sum over all the reactions where the
solid residue is material $\alpha$
\be
S_\alpha = \sum_{\alpha'=1}^{N_m} \sum_{\beta=1}^{N_{r,\alpha'}}
           \nu_{s,\alpha' \beta} \; r_{\alpha' \beta}
       \quad \quad
           \hbox{(where Residue$_{\alpha' \beta}$ = Material$_\alpha$) }
\ee
The volumetric production rate of fuel gases and water vapor are
\be
\dot{m}_f''' = \rho_{s0}\; \sum_{\alpha=1}^{N_m} \sum_{\beta=1}^{N_{r,\alpha}}
    \nu_{f,\alpha \beta} \; r_{\alpha \beta}
\ee
\be
\dot{m}_w''' = \rho_{s0}\; \sum_{\alpha=1}^{N_m}
    \sum_{\beta=1}^{N_{r,\alpha}}
    \nu_{w,\alpha \beta} \; r_{\alpha \beta}
\ee
It is assumed that the fuel gases and water vapor are transported instantaneously to the surface, where the
mass fluxes are given by:
\be
   \dm_f'' = \int_0^L \dm_f'''(x) \, dx  \quad ; \quad
   \dm_w'' = \int_0^L \dm_w'''(x) \, dx
\ee
where $L$ is the thickness of the surface. The chemical source term of
the heat conduction equation consists of the heats of reaction and the
differences in latent heats between the original material and the
products
\be
\dot{q}_{s,c}'''(x) = \rho_{s0}\;
    \sum_{\alpha=1}^{N_m} \sum_{\beta=1}^{N_{r,\alpha}}
    r_{\alpha \beta}(x) \;
    \left[H_{r,\alpha \beta}-
    \int_{T_0}^{T(x)} \Delta c(\theta)d\theta \right]
\ee
It is possible to define a reaction where the sum of the residue, fuel
and water yields is less than one. FDS then assumes that the solid is
converted to a gas that is not tracked by the gas phase
calculation. In the computation of $\Delta c(T)$, the
specific heat of this dummy species is assumed to be equal to the
specific heat of the original material $c_{s,\alpha}$
\be
  \Delta c(T) = c_{s,\alpha}(T)
        - \nu_{s,\alpha \beta} \; c_{s,\alpha \beta}(T)
        - \nu_{f,\alpha \beta} \; c_{f}(T)
        - \nu_{w,\alpha \beta} \; c_{w}(T)
        - (1-\sum_{\gamma=s,f,w} \nu_{\gamma,\alpha\beta}) \; c_{s,\alpha}(T)
\ee
$c_{s,\alpha \beta}$ is the specific heat of the residue of the
$\beta$th reaction of material $\alpha$. The temperature integrals of the
latent heat contents are computed in advance for all the reactions of
each material and the values of $\dot{q}_{s,c}'''$ are tabulated
between $T_0$ and $T_0 \;+\;2000$~K. During the pyrolysis calculation,
the values are found from a ``look-up'' table.



\subsubsection{Liquid Fuels}

The rate at which liquid fuel evaporates when burning is a function of
the liquid temperature and the concentration of fuel vapor above the
pool surface. According to the Clausius-Clapeyron relation, the volume fraction of the
fuel vapor above the surface is a function of the liquid boiling temperature
\be X_f = \exp \left[ -\frac{ h_v W_f}{\cal R} \left( \frac{1}{T_s}-\frac{1}{T_b} \right) \right]    \ee
where $h_v$ is the heat of vaporization, $W_f$ is the
molecular weight, $T_s$ is the surface temperature, and
$T_b$ is the boiling temperature of the fuel~\cite{Prasad:1}.

For simplicity, the liquid fuel itself is treated like a thermally-thick
solid for the purpose of computing the heat conduction. There is no
computation of the convection of the liquid within the pool.



\subsection{Radiation Heat Transfer to Solids}
\label{inradsection}

If it is assumed that the thermal radiation from the surrounding gases is
absorbed within an infinitely thin layer at the surface of the solid
obstruction, then the net radiative heat flux is the sum of incoming and outgoing
components, $\dq_r'' = \dq_{r,in}'' - \dq_{r,out}''$, where:
\begin{eqnarray}
 \dq_{r,in}'' &=& \epsilon\,
 \int_{\bs'\cdot \bn_w < 0} I_w(\bs')\; |\bs'\cdot \bn_w | \; d\bO
 \label{RFluxIn1} \\
 \dq_{r,out}'' &=& \epsilon\,\sigma\,T_w^4
 \label{RFluxOut1}
\end{eqnarray}
However, many common materials do not have infinite optical
thickness. Rather, the radiation penetrates the material
to some finite depth. The radiative transport within the solid (or
liquid) can be described as a source term in Eq.~(\ref{1dheat}).
A ``two-flux'' model based on the Schuster-Schwarzschild
approximation~\cite{Siegel:1} assumes the radiative
intensity is constant inside the ``forward'' and ``backward''
hemispheres. The transport equation for the intensity in the ``forward''
direction is
\be
 \frac{1}{2}\frac{dI^+(x)}{dx}=\kappa_s\,\left(I_b-I^+(x)\right)
 \label{RInForward}
\ee
where $x$ is the distance from the material surface and $\kappa_s$ is
the absorption coefficient
\be
   \kappa_s = \sum_{\alpha=1}^{N_m} X_\alpha \; \kappa_{s,\alpha}
\ee
A corresponding formula can be given for
the ``backward'' direction. Multiplying Eq.~\ref{RInForward} by $\pi$
gives us the ``forward'' radiative heat flux into the solid
\be
 \frac{1}{2}\frac{{d\dq^+_r(x)} }{dx}=\kappa_s\,
       \left(\sigma\,T_s^4-\dq_r^+(x)\right)
 \label{RFluxForward}
\ee
The radiative source term in the heat conduction equation is a sum of the
``forward'' and ``backward'' flux gradients
\be
  \dq_{s,r}'''(x) = \frac{d\dq_r^+(x)}{dx}+\frac{d\dq_r^-(x)}{dx}
\ee
The boundary condition for Eq.~\ref{RFluxForward} at the solid (or liquid)
surface is given by
\be
 \dq_r^+(0) = \dq_{r,in}'' + (1-\epsilon)\,\dq_r^-(0)
 \label{RFluxInBC}
\ee
where $\dq_r^-(0)$ is the ``backward'' radiative heat flux at the
surface. In this formulation, the surface emissivity and the internal
absorption are assumed to be independent properties of the
material.


\subsection{Convective Heat Transfer to Solids}
\label{conflux}

The calculation of the convective heat flux depends on whether one is
performing a Direct Numerical Simulation (DNS) or a
Large Eddy Simulation (LES).
In a DNS calculation, the convective heat flux to a solid surface $\dq''_c$
is obtained directly from the gas temperature gradient at the boundary
\be \dq_c'' = - k \; \dod{T}{n} \ee
where $n$ is the spatial coordinate pointing into the solid.
In an LES calculation, the convective heat flux to the surface is
obtained from a combination of natural and forced
convection correlations
\be \dq_c'' = h \; \Delta T
    \quad \hbox{W/m}^2 \quad ; \quad h =
    \max \; \left[ \; C\, |\Delta T|^\ot \; , \;
            \frac{k}{L} \; 0.037 \; \RE^\fofi \; \PR^\ot \;
            \right]  \quad
    \hbox{W/m$^2$/K} \ee
where $\Delta T$ is the difference between the wall and the gas temperature
(taken at the center of the grid cell abutting the wall),
$C$ is the coefficient for natural convection (1.52 for a horizontal surface
and 1.31 for a vertical surface)~\cite{Holman:1},
$L$ is a characteristic length related to the size of the physical
obstruction, $k$ is the thermal conductivity of the
gas, and the Reynolds $\RE$ and Prandtl $\PR$ numbers are based on the
gas flowing
past the obstruction. Since the Reynolds number is proportional to the
characteristic length, $L$, the heat transfer coefficient is weakly
related to $L$. For this reason, $L$ is taken to be 1~m for most
calculations.


\subsection{Boundary Conditions}
\label{matbc}

The boundary condition on the front surface is
\be  -k_s \dod{T_s}{x}(0,t) =  \dq''_c + \dq''_r \ee
If the internal radiation is solved for a solid, the radiation
boundary condition $\dq''_r$ is not used.

On the back surface, two possible boundary condition types may be
prescribed by the user. (1) If the back surface is assumed to be open
either to an ambient
void or to another part of the computational domain,
the back side boundary condition is similar to that of
the front side. (2) If the back side is assumed to be perfectly insulated,
an adiabatic boundary condition is used
\be  k_s \dod{T_s}{x} =  0 \ee




\clearpage
\section{Fire Detection Devices}

FDS predicts the thermal environment resulting from a fire, but it relies on various empirical models that describe the
activation of various fire detection devices. These models are described in this section.


\subsection{Sprinkler Activation}

The temperature of the sensing element (or ``link'') of an automatic fire sprinkler is estimated from
the differential equation put forth by Heskestad and Bill~\cite{Heskestad:3},
with the addition of a term to account for the cooling of the link
by water droplets in the gas stream from previously activated
sprinklers
\be \frac{dT_l}{dt} = \frac{\sqrt{|\bu|}}{\hbox{RTI}} (T_g - T_l) -
   \frac{C}{\hbox{RTI}} (T_l - T_m) - \frac{C_2}{\hbox{RTI}} \beta |\bu|
   \label{actode} \ee
Here $T_l$ is the link temperature, $T_g$ is the gas temperature in the
neighborhood of the link, $T_m$ is the temperature of the sprinkler
mount (assumed ambient),
and $\beta$ is the volume fraction of (liquid) water in the
gas stream. The sensitivity of the detector is characterized by
the value of RTI. The amount of heat conducted away from the
link by the mount is indicated by the ``C-Factor'', $C$. The RTI and
C-Factor are determined experimentally. The constant
$C_2$ has been empirically determined by DiMarzo and
co-workers~\cite{Ruffino:1,Ruffino:2,Gavelli:1} to
be $6\times 10^6$~K/(m/s)$^\ha$, and its value is relatively constant
for different types of sprinklers.

The algorithm for heat detector activation is exactly the same as for
sprinkler activation, except there is no accounting for conductive losses or
droplet cooling. Note that neither the sprinkler nor heat detector models account for
thermal radiation.

\subsection{Heat Detectors}

As far as FDS is concerned, a heat detector is just a sprinkler with no water spray. In other words, the activation of a heat
detector is governed by Eq.~(\ref{actode}), but with just the first term on the right hand side:
\be \frac{dT_l}{dt} = \frac{\sqrt{|\bu|}}{\hbox{RTI}} (T_g - T_l)  \label{heatactode} \ee
Both the RTI and activation temperature are determined empirically.


\subsection{Smoke Detectors}

An informative discussion of the issues associated with smoke detection can be found in the
SFPE Handbook chapter ``Design of Detection Systems,'' by Schifiliti, Meacham and Custer~\cite{SFPE}.
The authors point out that the difficulty in modeling smoke detector activation stems from a number of issues:
(1) the production and transport of smoke in the early stage of a fire are not well-understood, (2) detectors often use
complex response algorithms rather than simple threshold or rate-of-change criteria, (3) detectors can be sensitive
to smoke particle number density, size distribution, refractive index, composition, {\em etc.}, and (4) most
computer models, including FDS, do not provide detailed descriptions of the smoke besides its bulk transport. This
last point is the most important. At best, in its present form, FDS can only provide to an activation algorithm the
velocity and smoke concentration of the ceiling jet flowing past the detector. Regardless of the
detailed mechanism within
the device, any activation model included within FDS can only account for the entry resistance of the smoke due to the
geometry of the detector. Issues related to the effectiveness of ionization or photoelectric detectors cannot be
addressed by FDS.

Consider the simple idealization of a ``spot-type'' smoke detector. A disk-shaped cover lined with a fine mesh screen
forms the external housing of the device, which is usually mounted to the ceiling.
Somewhere within the device is a relatively small sensing chamber where the smoke is actually detected in some way.
A simple model of this device has been proposed by Heskestad~\cite{SFPE}. He suggested that the mass fraction of smoke in the
sensing chamber of the detector $Y_c$ lags behind the mass fraction in the
external free stream $Y_e$ by a time period $\delta t = L/u$,
where $u$ is the free stream velocity and $L$ is a length characteristic of the detector geometry.
The change in the mass fraction of smoke in the sensing chamber can be found by solving the following equation:
\be \frac{d Y_c }{dt} = \frac{ Y_e(t) - Y_c(t)}{L/\bu} \label{HYoeq} \ee
The detector activates when $Y_c$ rises above a detector-specific threshold.

A more detailed model of smoke detection involving two filling times rather than one has also been proposed.
Smoke passing into the sensing chamber must first pass through the exterior housing, then it must pass through a series
of baffles before arriving at the sensing chamber. There is a time lag
associated with the passing of the smoke through the housing and also the entry of the smoke into the sensing chamber.
Let $\delta t_e$ be the characteristic filling time of the entire volume enclosed by the external housing. Let
$\delta t_c$ be the characteristic filling time of the sensing chamber.
Cleary~{\em et al.}~\cite{Cleary:IAFSS6} suggested that each characteristic filling time is a function of the
free-stream velocity $u$ outside the detector
\be
\delta t_e = \alpha_e u^{\beta_e} \quad ; \quad \delta t_c = \alpha_c u^{\beta_c}
\ee
The $\alpha$'s and $\beta$'s are empirical constants related to the specific detector geometry
The change in the mass fraction of smoke in the sensing chamber $Y_c$ can be found by solving the following equation:
\be
\frac{d Y_c}{dt} = \frac{ Y_e(t-\delta t_e) - Y_c(t)}{\delta t_c} \label{Yoeq}
\ee
where $Y_e$ is the mass fraction of smoke outside of the detector in the free-stream.
A simple interpretation of the equation is that the concentration of the smoke that enters the sensing chamber at time $t$
is that of the free-stream at time $t-\delta t_e$.

An analytical solution for Eq.~(\ref{Yoeq}) can be found, but it is more convenient to simply integrate it numerically
as is done for sprinklers and heat detectors. Then, the predicted mass fraction of smoke in the sensing chamber,
$Y_c(t)$, can be converted into an expression for the percent obscuration per unit length by computing:
\be
\left( 1 - e^{-\kappa \rho Y_c l} \right) \times 100
\ee
where $\kappa$ is the specific extinction coefficient, $\rho$ is the density of the external gases in the ceiling jet,
and $l$ is the preferred unit of length (typically 1~m or 1~ft).
For most flaming fuels, a suggested value for $\kappa$ is 8700~m$^2$/kg~$\pm$~1100~m$^2$/kg at a
wavelength of 633~nm~\cite{Mulholland:F+M}.

The SFPE Handbook has references to various works on smoke detection and suggested values for the
characteristic length $L$. FDS includes the one parameter Heskestad model as a special case of the four parameter
Cleary model. For the Cleary model, one must set
$\alpha_e$, $\beta_e$, $\alpha_c$, and $\beta_c$, whereas for the Heskestad model only $L=\alpha_c$ needs to be
prescribed. Eq.~(\ref{Yoeq})
is still used, with $\alpha_e=0$ and $\beta_e=\beta_c=-1$.
Proponents of the four-parameter model claim that the two filling times are needed to
better capture the behavior of detectors in a very
slow free-stream ($u<0.5$~m/s). Rather than declaring one model better than another,
the algorithm included in FDS allows the
user to pick these various parameters, and in so doing, pick whichever model the user feels is appropriate~\cite{CSE_GCR}.

Additionally, FDS can model the behavior of beam and aspiration smoke detectors.  For a beam detector the user
specifies the emitter and receiver positions and the total obscuration at which the detector will alarm.  FDS will then
integrate the obscuration over the path length using the predicted soot concentration in each grid cell along
the path.  For an aspiration detector the user specifies the sampling locations, the flow rate at each location, the
transport time from each sampling point to the detector, the flow rate of any bypass flow, and the total obscuration at which the
detector will alarm.  FDS will compute that soot concentration at the detector by weighting the predicted soot concentrations
at the sampling locations with their flow rates after applying the appropriate time delay.





\clearpage
\section{Liquid Sprays}

There are several uses for liquid sprays within FDS, most notably for sprinklers, but also for liquid fuel sprays. The basic physics is the
same in each case, but the liquid properties are different.
Also, simulating the effects of a sprinkler spray involves a number of
elements beyond just activation: computing the droplet trajectories and
tracking the water as it drips onto the burning surface.



\subsection{Droplet Size Distribution}

A spray consists of a sampled set of spherical droplets. The size distribution is
expressed in terms of its Cumulative Volume Fraction (CVF), a function
that relates the fraction of the liquid volume (mass) transported by
droplets less than a given diameter. Researchers at Factory Mutual have
suggested that the CVF for a sprinkler may be represented by a combination of log-normal and
Rosin-Rammler distributions~\cite{Chan:1}
\be F(d) = \left\{ \begin{array}{ll}
   \frac{1}{\sqrt{2\pi}} {\displaystyle \int_0^d} \, \frac{1}{\sigma\, d'} \,
   e^{-\frac{[\ln(d'/d_m)]^2}{2\sigma^2}} \; \mbox{d}d'       & (d \le d_m) \\
   1 - e^{-0.693 \left(\frac{d}{d_m}\right)^\gamma }  & (d_m < d)
   \end{array} \right.  \ee
where $d_m$ is the median droplet diameter ({\em i.e.} half the mass
is carried by droplets with diameters of $d_m$ or less), and $\gamma$ and
$\sigma$ are empirical constants equal to about 2.4 and 0.6, respectively.\footnote{The Rosin-Rammler and
log-normal distributions are smoothly
joined if $\sigma=2/(\sqrt{2\pi} \, (\ln\,2) \; \gamma)=1.15/\gamma$ .}
The median droplet diameter is a function of the sprinkler orifice
diameter, operating pressure, and geometry. Research at Factory Mutual
has yielded a correlation for the median droplet diameter~\cite{Yu:2}
\be \frac{d_m}{D} \propto \WE^{-\ot}  \label{dropcor} \ee
where $D$ is the orifice diameter of the sprinkler.
The Weber number, the ratio of inertial forces
to surface tension forces, is given by
\be \WE = \frac{\rho_d u_d^2 D}{\sigma_d}  \label{Weber} \ee
where $\rho_d$ is the density of liquid, $u_d$ is the discharge
velocity, and $\sigma_d$ is the liquid surface tension ($72.8 \times 10^{-3}$
N/m at 20~$^\circ$C for water). The discharge velocity can be computed from the
mass flow rate, a function of the sprinkler's
operating pressure and orifice coefficient known as the ``K-Factor.''
FM reports that the constant of proportionality in Eq.~(\ref{dropcor})
appears to be independent of flow
rate and operating pressure. Three different sprinklers were tested in
their study with orifice diameters of 16.3~mm, 13.5~mm, 12.7~mm and
the constants were approximately 4.3, 2.9, 2.3, respectively. The strike
plates of the two smaller sprinklers were notched, while that of the
largest sprinkler was not~\cite{Yu:2}.

In the numerical algorithm,
the size of the droplets are chosen to mimic the
Rosin-Rammler/log-normal distribution. A Probability Density Function (PDF)
for the droplet diameter is defined
\be f(d) = \frac{F'(d)}{d^3}  \left/ \int_0^\infty \, \frac{F'(d')}{d'^3}
     \, \mbox{d}d' \right. \quad ; \quad F' \equiv \frac{\mbox{d}F}{\mbox{d}d}   \ee
Droplet diameters are randomly selected by equating the Cumulative
Number Fraction of the droplet distribution with a uniformly
distributed random variable $U$
\be U(d) = \int_0^d \, f(d') \, \mbox{d}d'  \label{Ud}  \ee
Figure~\ref{rosin} displays typical Cumulative Volume Fraction and
Cumulative Number Fraction functions.
\begin{figure}[t]
\includegraphics[width=4.5in]{FIGURES/rosin}
\caption{Cumulative Volume Fraction and Cumulative Number
Fraction functions of the droplet size distribution from
a typical industrial-scale sprinkler. The median diameter $d_m$ is
1~mm, $\sigma=0.6$ and $\gamma=2.4$.}
\label{rosin}
\end{figure}



\subsection{Droplet Transport in the Gas Phase}

Every droplet from a given sprinkler or nozzle is not tracked. Instead,
a sampled set of the droplets is tracked.
The procedure for selecting droplet sizes is as follows:
Suppose the mass flow rate of the liquid is $\dm$.
Suppose also that the time interval for droplet insertion
into the numerical simulation is $\dt$, and the number of droplets
inserted each time interval is $N$. Choose $N$ uniformly distributed random
numbers between 0 and 1, call them $U_i$,
obtain $N$ droplet diameters $d_i$ based on the
given droplet size distribution, Eq.~(\ref{Ud}), and then compute
a weighting constant C from the mass balance
\be \dm \; \dt = C \, \sum_{i=1}^N \;  \frac{4}{3} \pi \rho_w
      \left( \frac{d_i}{2} \right)^3 \ee
The mass and heat transferred from each droplet will be multiplied by
the weighting factor $C$.

For a spray, the force term $\bof_b$ in Eq.~(\ref{momentum})
represents the momentum transferred from the droplets to the gas.
It is obtained by summing the force transferred from each droplet
in a grid cell and dividing by the cell volume
\be {\bof_b} = \ha \frac{\sum \rho C_D \pi  r_d^2 (\bu_d-\bu) |\bu_d-\bu|}
    {\dx \, \dy \, \dz} \ee
where $C_D$ is the drag coefficient, $r_d$ is the droplet radius,
$\bu_d$ is the velocity of the droplet, $\bu$ is the
velocity of the gas, $\rho$ is the density of the gas,
and $\dx \, \dy \, \dz$ is the volume of the grid cell.
The trajectory of an individual droplet is governed by the equation
\be \frac{d}{dt} (m_d \bu_d) = m_d\, \bg - \ha \rho\,C_D \, \pi  r_d^2 \,
    (\bu_d-\bu) |\bu_d-\bu| \ee
where $m_d$ is the mass of the droplet.
The drag coefficient is a function of the local Reynolds number (based on droplet diameter)
\begin{eqnarray}
 C_D &=& \left\{ \begin{array}{ll}
     24/\RE_D                                       & \RE_D < 1    \\
     24\left(1+0.15 \, \RE_D^{0.687} \right)/\RE_D  & 1 < \RE_D < 1000 \\
     0.44                                           & 1000 < \RE_D
     \end{array} \right.  \\
\RE_D &=& \frac{\rho \, |\bu_d-\bu| \, 2 r_d}{\mu(T)} \end{eqnarray}
where $\mu(T)$ is the dynamic viscosity of air at temperature $T$.



\subsection{Droplet Transport on a Surface}

When a liquid droplet hits a solid horizontal surface, it is assigned a
random horizontal direction and moves at a fixed velocity until it
reaches the edge, at which point it drops straight down at the same
fixed velocity. This ``dripping'' velocity has been measured for water to be on
the order of 0.5~m/s~\cite{Hamins:1,Hamins:IAFSS2002}. While attached to a surface, the ``droplet'' is assumed to form a thin film of liquid that transfers heat to the solid, and heat and mass to
the gas. Details are included in the next section.



\subsection{Absorption and Scattering of Thermal Radiation by Droplets}

The attenuation of thermal radiation by liquid droplets is an
important consideration, especially for water mist
systems~\cite{Ravigururajan:1}.  Liquid droplets attenuate thermal
radiation through a combination of scattering and
absorption~\cite{Tuntomo:1}.  The radiation-droplet interaction must
therefore be solved for both the accurate prediction of the radiation
field and for the droplet energy balance.

If the gas phase absorption and emission in Eq.~(\ref{RTEbasic})
are temporarily neglected for simplicity, the radiative transport
equation becomes
\be \bs \cdot \nabla I_{\la}(\bx,\bs) = -\left[\kappa_d(\bx,\la) + \sigma_d(\bx,\la)\right]
I(\bx,\bs) +\kappa_d(\bx,\la) \; I_{b,d}(\bx,\la) +
\frac{\sigma_d(\bx,\la)}{4\pi}
\int_{4\pi}\Phi(\bs,\bs') \; I_{\la}(\bx,\bs') \; d\bs'
\label{RTEspray} \ee
where $\kappa_d$ is the droplet absorption coefficient, $\sigma_d$ is the
droplet scattering coefficient and $I_{b,d}$ is the emission
term of the droplets. $\Phi(\bs,\bs')$ is a scattering phase function
that gives the scattered intensity from direction $\bs'$ to $\bs$.
The local absorption and scattering coefficients are calculated
from the local droplet number density $N(\bx)$ and mean diameter $d_m(\bx)$ as
\be
\begin{array}{l}
\kappa_d(\bx,\la) = N(\bx)\int_0^\infty f(r,d_m(\bx)) \; C_a(r,\la) \; dr \\
\sigma_d(\bx,\la) = N(\bx)\int_0^\infty f(r,d_m(\bx)) \; C_s(r,\la) \; dr
\end{array}\ee
where $r$ is the droplet radius and $C_a$ and $C_s$ are absorption and
scattering cross sections, respectively, given by Mie theory.
The droplet number density function $f(r,d_m)$ is assumed to have
the same form as the initial droplet size distribution, but a
mean diameter depending on the location $\bx$. For the numerical
implementation, the above equations are written in the form
\be
\begin{array}{l}
\kappa_d(\bx,\la) = A_d(\bx)\int_0^\infty
    \frac{f(r,d_m(\bx))\;C_a(r,\la)}
    {\pi(d_m(\bx)/2)^2} \; dr \\
\sigma_d(\bx,\la) = A_d(\bx)\int_0^\infty
    \frac{f(r,d_m(\bx)) \; C_s(r,\la)}
    {\pi(d_m(\bx)/2)^2} \; dr \\
\end{array}
\ee
where $A_d$ is the total cross sectional area per unit volume of the
droplets. $A_d$ is approximated as
\be
A_d \approx \frac{\rho_d(\bx)}{2\rho_w d_m(\bx)/3}
\ee
% WHY? We could write N(\bx)=\rho_d/((4/3)*pi*rho_w*r_m^3) -- Simo

An accurate computation of the in-scattering integral on the right
hand side of Eq.~(\ref{RTEspray}) would be extremely time
consuming. It is here approximated by dividing the total $4\pi$ solid
angle to a ``forward angle'' $\delta\Omega^l$ and ``ambient angle''
$\delta\Omega^*=4\pi - \delta\Omega^l$.  For compatibility with the
FVM solver, $\delta\Omega^l$ is set equal to the control angle given
by the angular discretization.  However, it is assumed to be symmetric
around the center of the control angle.  Within $\delta\Omega^l$ the
intensity is $I_{\la}(\bx,\bs)$ and elsewhere it is approximated as
\be
U^*(\bx,\la) = \frac{U(\bx,\la) - \delta\Omega^l \, I_{\la}(\bx,\bs)}{\delta\Omega^*}
\ee
where $U(\bx)$ is the total integrated intensity. The in-scattering
integral can now be written as
\be
\frac{\sigma_d(\bx,\la)}{4\pi}\int_{4\pi}\Phi(\bs,\bs') \; I_{\la}(\bx,\bs')
  \; d\Omega' =
\sigma_d(\bx,\la)\left[\chi_f \; I_{\la}(\bx,\bs) + (1-\chi_f)U^*(\bx)\right]
\ee
where $\chi_f = \chi_f(r,\la)$ is a fraction of the total intensity
originally within the solid angle $\delta\Omega^l$ that is scattered
into the same angle $\delta\Omega^l$.  Defining an effective
scattering coefficient section
\be
\overline{\sigma_d}(\bx,\la) =
\frac{4\pi N(\bx)}{4\pi-\delta\Omega^l}
\int_0^\infty(1-\chi_f) \; C_s(r,\la) \; dr
\ee
the spray RTE becomes
\be
\bs \cdot \nabla I_{\la}(\bx,\bs) =
-\left[\kappa_d(\bx,\la) + \overline{\sigma_d}(\bx,\la)\right] I(\bx,\bs)
+\kappa_d(\bx,\la) \; I_{b,d}(\bx,\la)
+\frac{\overline{\sigma_d}(\bx,\la)}{4\pi}U(\bx,\la)
\ee
This equation can be integrated over the spectrum to get the band
specific RTE's. The procedure is exactly the same as that used for the
gas phase RTE. After the band integrations, the spray RTE for band $n$
becomes
\be
\bs \cdot \nabla I_{n}(\bx,\bs) =
-\left[\kappa_{d,n}(\bx) + \overline{\sigma_{d,n}}(\bx)\right] I_n(\bx,\bs)
+\kappa_{d,n}(\bx) \; I_{b,d,n}(\bx)
+\frac{\overline{\sigma_d}(\bx,\la)}{4\pi}U_n(\bx)
\ee
where the source function is based on the average droplet
temperature within a cell. The droplet contribution to the radiative
loss term in the energy equation is
\be -\nabla \cdot \dbq_r''(\bx)(\mbox{droplets}) =
    \kappa_d(\bx) \, \left[ U(\bx) - 4 \pi \, I_{b,d}(\bx) \right]
\ee
For each individual droplet, the radiative heating/cooling power is
computed as
\be
\dq_r = \frac{m_d}{\rho_d(\bx)}
    \kappa_d(\bx) \, \left[ U(\bx) - 4\pi \, I_{b,d}(\bx) \right]
\ee
where $m_d$ is the mass of the droplet and $\rho_d(\bx)$ is the total
density of droplets in the cell.

The absorption and scattering cross sections and the scattering phase
function are calculated using the MieV code developed by
Wiscombe~\cite{Wiscombe}.  Both $\kappa_d$ and $\overline{\sigma_d}$
are averaged over the possible droplet radii and wavelength before the
actual simulation.  A constant ``radiation'' temperature, $T_{rad}$,
is used in the wavelength averaging.  $T_{rad}$ should be selected to
represent a typical radiating flame temperature. A value of 1173~K is
used by default.  The averaged quantities, now functions of the
droplet mean diameter only, are saved in one-dimensional arrays.
During the simulation, the local properties are calculated as a table
look-up using the local mean droplet diameter~\cite{Hale:1}.  Details
of the computation are given in Section~\ref{dropnumericalmethod}.



\subsection{Heating and Evaporation of Liquid Droplets}

Liquid ``droplets'' are represented either as discrete airborne spheres propelled through the gas, or as rectangular blocks that collectively form a thin liquid film on solid objects.
These ``droplets'' are still individually tracked as
lagrangian particles, but the heat and mass transfer coefficients are different. In the discussion to follow, the term ``droplets'' will be used to describe either form.

Over the course of a time step of the gas phase solver, the droplets in a
given grid cell evaporate as a function of
the liquid equilibrium vapor mass fraction, $Y_l$,
the local gas phase vapor mass fraction, $Y_g$, the (assumed uniform) liquid temperature, $T_l$,
and the local gas temperature, $T_g$. If the droplet is attached to a surface, $T_s$ is the solid temperature.
The mass and energy transfer between the gas and the liquid can be described by the
following set of equations~\cite{Cheremisinoff:1}
\begin{eqnarray}
\frac{dm_l}{dt}               & = & - A \; h_m \, \rho \, (Y_l - Y_g) \\ [0.1in]
m_l \; c_l \; \frac{dT_l}{dt} & = &   A \, h  \, (T_g-T_l) + A \, h_s \, (T_s-T_l) + \dq_r + \frac{dm_l}{dt} \; h_v  \label{droplet_temp}   \end{eqnarray}
Here, $m_l$ is the mass of the liquid droplet (or that fraction of the surface film associated with the formerly airborne droplet), $A$ is the surface area of the liquid droplet (or that fraction of the
film exposed to the gas and to the wall), $h_m$ is the mass transfer coefficient to be discussed below,
$\rho$ is the gas density, $c_l$ is the liquid specific heat, $h$ is the heat transfer coefficient between the liquid and the gas, $h_s$ is the heat transfer coefficient between the liquid and the
solid surface, $\dq_r$ is the rate of radiative heating of the droplet, and $h_v$ is the latent heat of vaporization of the
liquid. The vapor mass fraction of the gas, $Y_g$, is obtained from the gas phase mass conservation equations, and the liquid equilibrium vapor mass fraction
is obtained from the Clausius-Clapeyron equation
\be X_l = \exp \left[ \frac{h_v \, W_l}{\cal R}
      \left( \frac{1}{T_b}-\frac{1}{T_l} \right) \right]  \quad ; \quad
      Y_l = \frac{X_l}{X_l \, (1-W_a/W_l) + W_a/W_l}  \label{clausius_clapeyron} \ee
where $X_d$ is the equilibrium vapor {\em volume} fraction, $W_l$ is the molecular weight
of the evaporated liquid, $W_a$ is the molecular weight of air,
$\cal R$ is the universal gas constant, and $T_b$ is the boiling temperature
of the liquid.

Mass and heat transfer between liquid and gas are described with analogous empirical correlations.
The mass transfer coefficient, $h_m$, is described by the empirical relationships~\cite{Incropera:1}:
\be
   h_m = \frac{\SH \; D_{lg}}{L} \quad ; \quad \SH = \left\{ \begin{array}{ll} 2 + 0.6 \; \RE_D^\ha \;           \SC^\ot & \hbox{droplet} \\ [0.1in]
                                                                                 0.037 \;   \RE_L^{\frac{4}{5}} \; \SC^\ot & \hbox{film}     \end{array} \right.
\ee
$\SH$ is the Sherwood number, $D_{lg}$ is the binary diffusion coefficient between the liquid vapor and the surrounding gas (usually assumed air), $L$ is a length scale equal to either the droplet diameter or
1~m for a surface film, $\RE_D$ is the Reynolds number of the droplet (based on the diameter, $D$, and the relative air-droplet velocity),
$\RE_L$ is the Reynolds number based on the length scale $L$, and $\SC$ is the Schmidt number
($\nu/D_{lg}$, assumed 0.6 for all cases).

An analogous relationship exists for the heat transfer coefficient:
\be
   h  = \frac{\NU \; k}{L} \quad ; \quad \NU = \left\{ \begin{array}{ll} 2 + 0.6 \; \RE_D^\ha \; \PR^\ot           & \hbox{droplet} \\ [0.1in]
                                                                         0.037 \;   \RE_L^{\frac{4}{5}} \; \PR^\ot & \hbox{film}     \end{array} \right.
\ee
$\NU$ is the Nusselt number, $k$ is the thermal conductivity of the gas, and $\PR$ is the Prandtl number (assumed 0.7 for all cases).








\subsection{Fire Suppression by Water}

The previous two sections describe heat transfer from a droplet of
water to a hot gas, a hot solid, or both. Although there is some
uncertainty in the values of the respective heat transfer coefficients,
the fundamental physics are fairly well understood. However, when
the water droplets encounter burning surfaces,
simple heat transfer correlations become more difficult to apply.
The reason for this is that the water is not only cooling the surface
and the surrounding gas, but it is also changing the pyrolysis rate
of the fuel. If the surface of the fuel is planar, it is possible
to characterize the decrease in the pyrolysis rate as a function of
the decrease in the total heat feedback to the surface. Unfortunately,
most fuels of interest in fire applications are multi-component solids
with complex geometry at scales unresolvable by the computational grid.

To date, most of the work in this area has been
performed at Factory Mutual. An important paper on the subject is
by Yu {\em et al.}~\cite{Yu:1}. The authors consider dozens of
rack storage commodity fires of different geometries and water
application rates, and characterize the suppression rates in terms of
a few global parameters. Their analysis yields an
expression for the total heat release rate from a rack storage fire
after sprinkler activation
\be \dQ = \dQ_0 \; e^{-k (t-t_0)}  \label{fmexting} \ee
where $\dQ_0$ is the total heat release rate at the time of application
$t_0$, and $k$ is a fuel-dependent constant.
For the FMRC Standard Plastic commodity $k$ is given as
\be k = 0.716 \; \dot{m}_w'' - 0.0131 \quad  \hbox{s}^{-1} \ee
where $\dot{m}_w''$ is the flow rate of water impinging on the
box tops, divided by the area of exposed surface (top and sides). It is
expressed in units of kg/m$^2$/s. For the Class II commodity, $k$ is
given as
\be k = 0.536 \; \dot{m}_w'' - 0.0040 \quad  \hbox{s}^{-1} \ee

Unfortunately, this analysis is based on global water flow and
burning rates. Equation~(\ref{fmexting})
accounts for both the cooling of non-burning surfaces as well as the
decrease in heat release rate of burning surfaces. In the FDS model,
the cooling of unburned surfaces and the reduction in the heat
release rate are computed locally. Thus, it is awkward to apply a
global suppression rule.
However, the exponential nature of suppression by water is observed
both locally and globally, thus it is assumed that the local burning rate
of the fuel can be expressed in the form~\cite{Hamins:1,Hamins:IAFSS2002}
\be \dm_f''(t) = \dm_{f,0}''(t) \; e^{-\int k(t) \, dt}
\label{nistexting} \ee
Here $\dm_{f,0}''(t)$ is the burning rate per unit area of the fuel
when no water is applied and $k(t)$ is a linear function of the local water
mass per unit area, $m_w''$, expressed in units of kg/m$^2$,
\be k(t) = a \; m_w''(t) \quad   \hbox{s}^{-1} \ee
Note that $a$ is an empirical constant.







\clearpage
\chapter{Numerical Method}

This chapter presents a description of the numerical methods used to solve the equations of the previous chapter, with sections
in roughly the same order.

\section{Hydrodynamic Model}

Each of the conservation equations emphasizes the importance of the velocity
divergence and vorticity fields, as well as the close relationship between
the thermally expandable fluid equations~\cite{Rehm:1} and the Boussinesq
equations for which the authors have developed highly efficient solution
procedures~\cite{McGrattan:1,Baum:1}.
All spatial derivatives are approximated by second-order finite
differences and the flow variables are updated in time using an
explicit second-order predictor-corrector scheme.

\subsection{Simplified Equations}

\leftline{\underline{Conservation of Mass}}
\be \dod{\rho}{t} + \bu \cdot \nabla \rho
       =  -\rho \nabla \cdot \bu  \label{mass2} \ee
\leftline{\underline{Conservation of Species}}
\be \dod{\rho Y_\alpha}{t} + \bu \cdot \nabla \rho Y_\alpha  =
-\rho Y_\alpha \nabla \cdot \bu + \nabla \cdot \rho D_\alpha \nabla Y_\alpha + \dm_\alpha'''
  \label{species2} \ee

\leftline{\underline{Conservation of Momentum}}
\be \dod{\bu}{t} + \bu \times \bo + \nabla {\cal H} =
  \frac{1}{\rho} \left((\rho-\rho_0)\bg + {\bof_b}
  + \nabla \cdot \btau_{ij} \right) \label{momentum2} \ee

\leftline{\underline{Pressure Equation}}
\be \nabla^2 {\cal H} = -\dod{ (\nabla \cdot \bu) }{t} - \nabla \cdot \bF  \quad ; \quad
\bF = \bu \times \bo - \frac{1}{\rho} \left((\rho-\rho_m)\bg - {\bof_b} - \nabla \cdot \btau_{ij} \right)   \label{simplephi2} \ee

\leftline{\underline{Equation of State}}
\be \bp_m(z,t) = \rho T {\cal R} \sum_\alpha  Y_\alpha/W_\alpha \ee
Notice that the source terms from the energy conservation equation have
been incorporated into the divergence and ultimately are involved in
the mass conservation equation. The temperature is found from the
density and background pressure via the equation of state.

\subsubsection{Temporal Discretization}

All calculations start with ambient initial conditions.
At the beginning of each time step, the density,
$\rho^n$, species mass fractions, $Y_\alpha^n$, velocity, $\bu^n$, modified pressure, ${\cal H}^n$, and background
pressure in zone $m$, $\bp_m^n$ are known. All
other quantities can be derived from them.
Note that the superscript $(n+1)_e$ refers to an estimate of the value
of the quantities at the $(n+1)$st time step.

\begin{enumerate}
\item The thermodynamic quantities $\rho$, $Y_\alpha$, and $\bp_m$ are
estimated at the next time step with an explicit Euler step. For
example, the density is estimated by
\be \rho^{(n+1)_e} = \rho^n - \dt (\bu^n \cdot \nabla \rho^n + \rho^n \nabla \cdot \bu^n) \ee

\item The divergence $(\nabla \cdot \bu)^{(n+1)_e}$ is formed from these
estimated thermodynamic quantities. The normal velocity components at
boundaries that are needed to form the divergence are assumed known.

\item A Poisson equation for the pressure is solved with a direct solver
\be \nabla^2{\cal H}^n = - \, \frac{ (\nabla \cdot \bu)^{(n+1)_e} -
  (\nabla \cdot \bu)^n }{\dt} - \nabla \cdot \bF^n  \ee
Note that the vector $\bF$ contains the convective, diffusive and force
terms of the momentum equation. These will be described in detail below.

\item The velocity is estimated at the next time step
\be \bu^{(n+1)_e} = \bu^n - \dt \left( \bF^n + \nabla {\cal H}^n \right) \ee
Note that the divergence of the estimated velocity field is identically
equal to the estimated divergence $(\nabla \cdot \bu)^{(n+1)_e}$ that
was derived from the estimated thermodynamic quantities.
The time step is checked at this point to ensure that
\be \dt \; \hbox{max} \left( \frac{|u|}{\dx},\frac{|v|}{\dy},\frac{|w|}{\dz} \right) < 1 \quad ; \quad
    2 \; \dt \; \nu \; \left(\frac{1}{\dx^2} + \frac{1}{\dy^2} + \frac{1}{\dz^2} \right) < 1 \ee
If the time step is too large, it is reduced so that it satisfies
both constraints and the procedure returns to the beginning of the time step.
If the time step satisfies the stability criteria, the procedure continues to the corrector step.
See Section~\ref{stability} for more details on stability.

\item The thermodynamic quantities $\rho$, $Y_\alpha$, and $\bp_m$ are
``corrected'' at the next time step. For example, the density is corrected
\be \rho^{n+1} =
   \ha \left(\rho^n + \rho^{(n+1)_e} - \dt
    (\bu^{(n+1)_e} \cdot \nabla \rho^{(n+1)_e} + \rho^{(n+1)_e} \nabla \cdot \bu^{(n+1)_e}) \right)  \ee

\item The corrected divergence $(\nabla \cdot \bu)^{(n+1)}$ is computed from the
corrected thermodynamic quantities.

\item The pressure is recomputed using estimated quantities
\be \nabla^2{\cal H}^{(n+1)_e} = - \, \frac{ 2(\nabla\cdot\bu)^{n+1}
       - (\nabla\cdot\bu)^{(n+1)_e} - (\nabla\cdot\bu)^n }{\dt}
  - \nabla \cdot \bF^{(n+1)_e}  \ee

\item The velocity is corrected
\be \bu^{n+1} = \ha \left[ \bu^n + \bu^{(n+1)_e}
     - \dt \left( \bF^{(n+1)_e} + \nabla {\cal H}^{(n+1)_e} \right) \right] \ee
Note again that the divergence of the corrected velocity field is
identically equal to the corrected divergence.

\end{enumerate}

\subsubsection{Spatial Discretization}

Spatial derivatives in the governing equations are written as second-order accurate
finite differences on a rectilinear grid. The overall
domain is a rectangular box that is divided into rectangular grid cells.
Each cell is assigned indices $i$, $j$ and $k$ representing the
position of the cell in the $x$, $y$ and $z$ directions, respectively.
Scalar quantities are assigned in the center of each grid cell; thus,
$\rho_{ijk}^n$ is the density at the $n$th time step in the center
of the cell whose indices are $i$, $j$ and $k$. Vector quantities like
velocity are assigned at their appropriate cell faces. For example,
$u_{ijk}^n$ is the $x$-component of velocity at the
positive-oriented face of the $ijk$th cell; $u_{i-1,jk}^n$ is defined at the
negative-oriented face of the same cell.


\clearpage
\subsection{The Mass Transport Equations and the Divergence}

Due to the use of the low Mach number approximation, the
mass and energy equations are combined through the divergence. The
divergence of the flow field contains
many of the fire-specific source terms described above.

\subsubsection{Discretizing the Convective and Diffusive Transport Terms}

The density at the center of the $ijk$th cell
is updated in time with the following predictor-corrector scheme. In the
predictor step, the density at the $(n+1)$st time level is estimated
based on information at the $n$th level
\be  \frac{\rho_{ijk}^{(n+1)_e}-\rho_{ijk}^n}{\dt}
    + (\bu \cdot \nabla \rho)_{ijk}^n = -\rho_{ijk}^n (\nabla \cdot \bu)^n_{ijk} \ee
Following the prediction of the velocity and background pressure
at the $(n+1)$st time level, the density is corrected
\be \frac{\rho_{ijk}^{(n+1)}-\ha\left(\rho_{ijk}^n
     +\rho_{ijk}^{(n+1)_e}\right)} {\ha \dt}
    + (\bu \cdot \nabla \rho)_{ijk}^{(n+1)_e}
    = -\rho_{ijk}^{(n+1)_e} (\nabla \cdot \bu)^{(n+1)_e}_{ijk} \ee
The species conservation equations are differenced the same way, with the addition of diffusion and source terms:
\be  \frac{(\rho Y_\alpha)_{ijk}^{(n+1)_e}-(\rho Y_\alpha)_{ijk}^n}{\dt}
    + \cdots  = \cdots +
    (\nabla \cdot \rho D_\alpha \nabla Y_\alpha)_{ijk}^n  + \dm_{\alpha,ijk}''' \ee
at the predictor step, and
\be \frac{(\rho Y_\alpha)_{ijk}^{(n+1)}-\ha\left((\rho Y_\alpha)_{ijk}^n
     +(\rho Y_\alpha)_{ijk}^{(n+1)_e}\right)} {\ha \dt}
    + \cdots = \cdots
    + (\nabla \cdot \rho D_\alpha \nabla Y_\alpha)_{ijk}^{(n+1)_e}  + \dm_{\alpha,ijk}''' \ee
at the corrector step.

The convective terms are written as upwind-biased differences in the
predictor step and downwind-biased differences in the corrector step~\cite{Continillo:1}.
In the definition to follow, the symbol $\pm$ means $+$ in the predictor
step and $-$ in the corrector step. The opposite is true for $\mp$.
\begin{eqnarray}
(\bu\cdot\nabla \rho)_{ijk} &\equiv&
\frac{1\mp\epsilon_u}{2} u_{ijk}  \frac{\rho_{i+1,jk} -\rho_{ijk}}{\dx} +
\frac{1\pm\epsilon_u}{2} u_{i-1,jk}  \frac{\rho_{ijk} -\rho_{i-1,jk}}{\dx}+
\nonumber \\
& & \frac{1\mp\epsilon_v}{2} v_{ijk}  \frac{\rho_{i,j+1,k} -\rho_{ijk}}{\dy} +
\frac{1\pm\epsilon_v}{2} v_{i,j-1,k}  \frac{\rho_{ijk} -\rho_{i,j-1,k}}{\dy}+
\nonumber \\
& & \frac{1\mp\epsilon_w}{2} w_{ijk}  \frac{\rho_{ij,k+1} -\rho_{ijk}}{\dz} +
\frac{1\pm\epsilon_w}{2} w_{ij,k-1}  \frac{\rho_{ijk} -\rho_{ij,k-1}}{\dz}
\end{eqnarray}
The convective term in the species transport equation, $(\bu\cdot\nabla \rho Y_\alpha)_{ijk}$ is differenced
the exact same way.
Note that without the inclusion of the $\epsilon$'s, these are simple
central difference approximations. The $\epsilon$'s are local Courant numbers,
$\epsilon_u = u \dt/\dx$, $\epsilon_v = v \dt/\dy$, and
$\epsilon_w = w \dt/\dz$, where the velocity components are those that
immediately follow.
Their role is to bias the differencing upwind at the predictor step. Where the local Courant number is
near unity, the difference becomes nearly fully upwinded. Where the local Courant
number is much less than unity, the differencing is
more centralized.


\subsubsection{Flux Correction}

The second-order finite-differencing scheme used in FDS cannot fully resolve sharp gradients on a relatively coarse grid.
Instead, steep gradients cause local {\em over-shoots} and {\em under-shoots} of quantities like temperature, density and species mass fraction.
For mass fraction in particular, this can result in a solution where the mass
fraction exceeds its permissible limits ({\em i.e.} the numerical method transports into or out of a grid
cell more mass than is physically possible).  The overall numerical scheme is still mass conserving,
but non-physical in regions of high gradients.  This problem can be reduced, but not eliminated,
by using higher order numerical methods.  These methods are, however, more expensive.  Another
solution is to perform a flux transport correction.  This involves examining the solution and
locating regions where a non-physical solution exists and then redistributing mass to correct it.
Typically this results in some increased numerical diffusion; however, this is partially mitigated
since at any one time step, the correction is applied to a small number of cells.

The flux correction scheme is performed in both the predictor and corrector steps after updating the species mass
fractions, $Y_\alpha$.  For each species, two loops are performed over each computational mesh.  The first loop searches for
and corrects under-shoots, and the second loop searches for and corrects over-shoots.  An under-shoot occurs if the
mass fraction of a species is less than its permissible minimum, typically 0,  or if there was an outflux of that
species in the prior time step, $\nabla \, \rho Y_{\alpha,ijk} \, < \, 0$ , and the new species mass fraction
is less than all of its surrounding cells.  An over-shoot occurs if the
mass fraction of a species is greater than its permissible maximum,  or if there was an influx of that
species in the prior time step, $\nabla \, \rho Y_{\alpha,ijk} \, > \, 0$ , and the new species mass fraction
is greater than all of its surrounding cells.  In each loop a temporary array is used to store the corrected values
which are then applied globally at the end of each loop.  Using a temporary array rather than a cell by cell
immediate correction ensures that cells requiring correction are not bypassed due to the sweep direction of the loop.
Under-shoots are corrected first because in a typical mixture fraction computation it is more likely to have an under-shoot
with a value less than the absolute minimum for the species than it is for an over-shoot to exceed the absolute maximum
for the species.





\subsubsection{Discretizing the Divergence}
\label{div_discret}

The divergence (see Eq.~(\ref{phi})) in the $m$th pressure zone in both the predictor and corrector step
is discretized
\be (\nabla \cdot \bu)_{ijk} = \frac{\R}{\bW c_p \bp_m} \left( \dq_{ijk}''' + (\nabla \cdot k \nabla T)_{ijk}
    + \ldots \right) + \frac{1}{\bp_n} \left( \frac{\R}{\bW c_p} -1 \right)
      \left( \dod{\bp_m}{t} - w_{ijk} \rho_{0,k} g \right) \label{divdis} \ee
The thermal and material diffusion terms are pure central differences,
with no upwind or downwind bias, thus they are differenced the same
way in both the predictor and corrector steps. For example, the thermal
conduction term is differenced as follows:
\begin{eqnarray}
(\nabla \cdot k \nabla T)_{ijk} &=&
              \frac{1}{\dx}
         \left[k_{i+\ha,jk}\frac{T_{i+1,jk}-T_{ijk}}{\dx}
              -k_{i-\ha,jk}\frac{T_{ijk}-T_{i-1,jk}}{\dx}\right]+  \nonumber \\
            &&\frac{1}{\dy}
         \left[k_{i,j+\ha,k}\frac{T_{i,j+1,k}-T_{ijk}}{\dy}
              -k_{i,j-\ha,k}\frac{T_{ijk}-T_{i,j-1,k}}{\dy}\right]+ \nonumber \\
            &&\frac{1}{\dz}
         \left[k_{ij,k+\ha}\frac{T_{ij,k+1}-T_{ijk}}{\dz}
              -k_{ij,k-\ha}\frac{T_{ijk}-T_{ij,k-1}}{\dz}\right]
\end{eqnarray}
The temperature is extracted from the density via the equation of state
\be T_{ijk} = \frac{\bp_m}{\rho_{ijk} {\cal R}\, \sum_{l=0}^{N_s} (Y_{\alpha,ijk}/W_\alpha)}\ee
Because only species 1 through $N_s$ are explicitly computed, the summation
is rewritten
\be \bW \equiv \sum_{\alpha=0}^{N_s} \frac{Y_{\alpha,ijk}}{W_\alpha} = \frac{1}{W_0} + \sum_{\alpha=1}^{N_s}
   \left( \frac{1}{W_\alpha}-\frac{1}{W_0} \right) Y_\alpha \ee
In isothermal calculations involving multiple species, the density
can be extracted from the average molecular weight
\be \rho_{ijk} =  \frac{ p_m}{T_\infty {\cal R } \bW } =
   \frac{W_0\,p_m}{T_\infty {\cal R }} + \sum_{\alpha=1}^{N_s} \left( 1-\frac{W_0}{W_\alpha} \right)
   (\rho Y_\alpha)_{ijk} \ee



\clearpage
\subsection{The Momentum Equation}

\label{themom}

The three components of the momentum equation are
\begin{eqnarray}
\dod{\hu}{t} + F_x + \dod{\hp}{x} = 0 \quad &;& \quad
F_x = \hw \, \omy - \hv \, \omz - \frac{1}{\rho} \left( f_x
  +  \dod{\tau_{xx}}{x} + \dod{\tau_{xy}}{y} + \dod{\tau_{xz}}{z} \right) \\
\dod{\hv}{t} + F_y + \dod{\hp}{y} = 0 \quad &;& \quad
F_y = \hu \, \omz - \hw \, \omx - \frac{1}{\rho} \left( f_y
  +  \dod{\tau_{yx}}{x} + \dod{\tau_{yy}}{y} + \dod{\tau_{yz}}{z} \right) \\
\dod{\hw}{t} + F_z + \dod{\hp}{z} = 0 \quad &;& \quad
F_z = \hv \, \omx - \hu \, \omy - \frac{1}{\rho} \left( f_z
  +  \dod{\tau_{zx}}{x} + \dod{\tau_{zy}}{y} + \dod{\tau_{zz}}{z} \right)
\end{eqnarray}
The spatial discretization of the momentum equations takes the form
\begin{eqnarray}
\dod{\hu}{t} + F_{\x,ijk} + \frac{\hp_{i+1,jk} -\hp_{ijk}}{\dx} = 0  \label{umom} \\
\dod{\hv}{t} + F_{\y,ijk} + \frac{\hp_{i,j+1,k}-\hp_{ijk}}{\dy} = 0  \label{vmom} \\
\dod{\hw}{t} + F_{\z,ijk} + \frac{\hp_{ij,k+1} -\hp_{ijk}}{\dz} = 0  \label{wmom}
\end{eqnarray}
where $\hp_{ijk}$ is taken at center of cell $ijk$,
$\hu_{ijk}$ and $F_{\x,ijk}$ are taken at the side of the cell facing
in the forward $x$ direction, $\hv_{ijk}$ and $F_{\y,ijk}$ at the side
facing in the forward $y$ direction, and $\hw_{ijk}$ and $F_{\z,ijk}$
at the side facing in the forward $z$ (vertical) direction. In the
definitions to follow, the components of the vorticity $(\om_x,\om_y,\om_z)$
are located at cell edges pointing in the $x$, $y$ and $z$ directions,
respectively. The same is true for the off-diagonal terms of the viscous
stress tensor: $\tau_{zy}=\tau_{yz}$, $\tau_{xz}=\tau_{zx}$, and
$\tau_{xy}=\tau_{yx}$. The diagonal components of the stress
tensor $\tau_{xx}$, $\tau_{yy}$, and $\tau_{zz}$; the external force
components $(f_x,f_y,f_z)$; and the Courant numbers
$\epsilon_u$, $\epsilon_v$, and $\epsilon_w$ are located at their
respective cell faces.
\begin{eqnarray}
F_{\x,ijk} &=&  \left(
 \frac{1\mp\epsilon_w}{2} w_{i+\ha,jk} \; \om_{y,ijk} +
 \frac{1\pm\epsilon_w}{2} w_{i+\ha,j,k-1} \; \om_{y,ij,k-1} \right) \nonumber \\
          & & - \left(
 \frac{1\mp\epsilon_v}{2} v_{i+\ha,jk} \; \om_{z,ijk} +
 \frac{1\pm\epsilon_v}{2} v_{i+\ha,j-1,k} \; \om_{z,i,j-1,k} \right) \nonumber\\
           & & - \frac{1}{\rho_{i+\ha,jk}} \left( f_{x,ijk}
  + \frac{\tau_{xx,i+1,jk}-\tau_{xx,ijk}}{\dx}
  + \frac{\tau_{xy,ijk}-\tau_{xy,i,j-1,k}}{\dy}
  + \frac{\tau_{xz,ijk}-\tau_{xz,i,j,k-1}}{\dz}  \right)  \\
F_{\y,ijk} &=& \left(
 \frac{1\mp\epsilon_u}{2} u_{i,j+\ha,k} \; \om_{z,ijk} +
 \frac{1\pm\epsilon_u}{2} u_{i-1,j+\ha,k} \; \om_{z,i-1,jk} \right) \nonumber \\
           & & - \left(
 \frac{1\mp\epsilon_w}{2} w_{i,j+\ha,k} \; \om_{x,ijk}+
 \frac{1\pm\epsilon_w}{2} w_{i,j+\ha,k-1} \; \om_{x,ij,k-1} \right) \nonumber \\
           & & - \frac{1}{\rho_{i,j+\ha,k}} \left( f_{y,ijk}
  + \frac{\tau_{yx,ijk}-\tau_{yx,i-1,jk}}{\dx}
  + \frac{\tau_{yy,i,j+1,k}-\tau_{yy,ijk}}{\dy}
  + \frac{\tau_{yz,ijk}-\tau_{yz,i,j,k-1}}{\dz} \right) \\
F_{\z,ijk} &=&  \left(
 \frac{1\mp\epsilon_v}{2} v_{ij,k+\ha} \; \om_{x,ijk} +
 \frac{1\pm\epsilon_v}{2} v_{i,j-1,k+\ha} \; \om_{x,i,j-1,k} \right)\nonumber \\
           & & -\left(
 \frac{1\mp\epsilon_u}{2} u_{ij,k+\ha} \; \om_{y,ijk} +
 \frac{1\pm\epsilon_u}{2} u_{i-1,j,k+\ha} \; \om_{y,i-1,jk} \right) \nonumber \\
           & & - \frac{1}{\rho_{ij,k+\ha}} \left( f_{z,ijk}
  + \frac{\tau_{zx,ijk}-\tau_{zx,i-1,jk}}{\dx}
  + \frac{\tau_{zy,ijk}-\tau_{zy,i,j-1,k}}{\dy}
  + \frac{\tau_{zz,ij,k+1}-\tau_{zz,ijk}}{\dz} \right) \\
\om_{x,ijk} &=& \frac{\hw_{i,j+1,k}-\hw_{ijk}}{\dy} -
             \frac{\hv_{ij,k+1}-\hv_{ijk}}{ \dz}  \\
\om_{y,ijk} &=& \frac{\hu_{ij,k+1}-\hu_{ijk}}{\dz} -
             \frac{\hw_{i+1,jk}-\hw_{ijk}}{\dx}  \\
\om_{z,ijk} &=& \frac{\hv_{i+1,jk}- \hv_{ijk}}{\dx} -
             \frac{\hu_{i,j+1,k}-\hu_{ijk}}{ \dy} \\
\tau_{xx,ijk} &=& \mu_{ijk} \left( \ft (\nabla \cdot \bu)_{ijk} - 2 \frac{v_{ijk}-v_{i,j-1,k}}{\dy} - 2 \frac{w_{ijk}-w_{ij,k-1}}{\dz} \right)  \\
\tau_{yy,ijk} &=& \mu_{ijk} \left( \ft (\nabla \cdot \bu)_{ijk} - 2 \frac{u_{ijk}-u_{i-1,jk}}{\dx}  - 2 \frac{w_{ijk}-w_{ij,k-1}}{\dz} \right)  \\
\tau_{zz,ijk} &=& \mu_{ijk} \left( \ft (\nabla \cdot \bu)_{ijk} - 2 \frac{u_{ijk}-u_{i-1,jk}}{\dx}  - 2 \frac{v_{ijk}-v_{i,j-1,k}}{\dy} \right)  \\
\tau_{xy,ijk} &=& \tau_{yx,ijk}
        = \mu_{i+\ha,j+\ha,k} \left( \frac{u_{i,j+1,k}-u_{ijk}}{\dy}
                         + \frac{v_{i+1,jk} -v_{ijk}}{\dx} \right) \\
\tau_{xz,ijk} &=& \tau_{zx,ijk}
        =\mu_{i+\ha,j,k+\ha} \left( \frac{u_{ij,k+1}-u_{ijk}}{\dz}
                        + \frac{w_{i+1,jk}-w_{ijk}}{\dx} \right) \\
\tau_{yz,ijk} &=& \tau_{zy,ijk}
        =\mu_{i,j+\ha,k+\ha} \left( \frac{v_{ij,k+1}-v_{ijk}}{\dz}
                        + \frac{w_{i,j+1,k}-w_{ijk}}{\dy} \right) \\
\epsilon_u &=& \frac{u \, \dt}{\dx} \\
\epsilon_v &=& \frac{v \, \dt}{\dy} \\
\epsilon_w &=& \frac{w \, \dt}{\dz}
\end{eqnarray}
The variables $\epsilon_u$, $\epsilon_v$ and $\epsilon_w$ are local
Courant numbers evaluated at the same locations as the velocity component
immediately following them, and serve to bias the differencing of
the convective terms in the upwind direction.
The subscript $i+\ha$ indicates that a variable is an average of its
values at the $i$th and the $(i+1)$th cell.
By construction, the divergence defined in Eq.~(\ref{divdis})
is identically equal to the divergence defined by
\be (\nabla \cdot \bu)_{ijk} = \frac{u_{ijk}-u_{i-1,jk}}{\dx} +
                               \frac{v_{ijk}-v_{i,j-1,k}}{\dy} +
                               \frac{w_{ijk}-w_{ij,k-1}}{\dz}   \ee
The equivalence of the two definitions of the divergence is a result
of the form of the discretized equations, the time-stepping scheme, and
the direct solution of the Poisson equation for the pressure.


\subsubsection{Viscous Terms (LES)}

The major difference between an LES and a DNS calculation is the form of
the viscosity, and the thermal and material diffusivities.
For a Large Eddy Simulation, the dynamic viscosity
is defined at cell centers
\be \mu_{ijk} = \rho_{ijk} \, (C_s\, \Delta)^2 \, |S|   \ee
where $C_s$ is an empirical constant, $\Delta=(\dx\,\dy\,\dz)^\ot$, and
\be |S|^2 = 2\left(\dod{u}{x}\right)^2 + 2\left(\dod{v}{y}\right)^2+
  2\left( \dod{w}{z}\right)^2
       + \left( \dod{u}{y} + \dod{v}{x} \right)^2
       + \left( \dod{u}{z} + \dod{w}{x} \right)^2
       + \left( \dod{v}{z} + \dod{w}{y} \right)^2
       - \frac{2}{3} (\nabla \cdot \bu)^2  \ee
The quantity $|S|$ consists of second order spatial differences
averaged at cell centers. For example
\begin{eqnarray}
\dod{u}{x} &\approx& \frac{u_{ijk}-u_{i-1,jk}}{\dx_i} \\
\dod{u}{y} &\approx& \frac{1}{2} \left( \frac{u_{i,j+1,k}-u_{ijk}}{\dy_{j+\ha}} + \frac{u_{ijk}-u_{i,j-1,k}}{\dy_{j-\ha}} \right) \end{eqnarray}
The divergence is described in Section~\ref{div_discret}.

The thermal conductivity and material
diffusivity of the fluid are related to the viscosity by
\be k_{ijk} = \frac{c_{p,0} \, \mu_{ijk}}{\PR_t}  \quad ; \quad
   (\rho D)_{ijk} = \frac{\mu_{ijk}}{\SC_t}  \ee
where $\PR_t$ is the turbulent Prandtl number and $\SC_t$ is the turbulent Schmidt number, both
assumed constant. Note that the specific heat $c_{p,0}$ is that of the
dominant species of the mixture. Based on simulations of smoke plumes,
$C_s$ is 0.20, $\PR_t$ and $\SC_t$ are 0.5. There are no rigorous justifications
for these choices other than through comparison with
experimental data~\cite{Zhang:1}.

\subsubsection{Viscous Terms (DNS)}

The dynamic viscosity, thermal conductivity and diffusion coefficients
for a DNS calculation are defined at cell centers
\begin{eqnarray}
\mu_{ijk} &=& \sum_l Y_{l,ijk} \; \mu_l(T_{ijk})  \\
k_{ijk}   &=& \sum_l Y_{l,ijk} \; k_l(T_{ijk})   \\
D_{l,ijk} &=& D_{l0}(T_{ijk})  \end{eqnarray}
where the values for each individual species are approximated from
kinetic theory~\cite{Poling:1}. The term $D_{l0}$ is the binary diffusion
coefficient for species $l$ diffusing into the predominant species $0$,
usually nitrogen. It is often the case that the
numerical grid is too coarse to resolve steep gradients in flow
quantities when the temperature is near ambient.
However, as the temperature increases and
the diffusion coefficients increase in value, the situation improves.
As a consequence, there is a provision in the numerical algorithm to place
a lower bound on the viscous coefficients to avoid numerical instabilities
at temperatures close to ambient.


\subsubsection{Velocity Boundary Conditions at a Solid Surface}

At solid boundaries, the tangential components of velocity need to be set inside the solid in order to establish a velocity gradient for the
numerical velocity field. As an example, consider
a solid surface forming the lower boundary of the computational domain ($z=0$). The horizontal velocity components $u_{ij,1}$ and
$v_{ij,1}$ are computed at the vertical faces of the first layer of gas phase grid cells above the solid surface. What should the
values of $u_{ij,0}$ and $v_{ij,0}$ be set to such that the flow near the wall is impeded by the same amount as a
real turbulent flow? Assuming that the numerical grid is too coarse to resolve the boundary layer (of nominal thickness $d$), and assuming
a turbulent velocity profile
\be U(z) = \left\{ \begin{array}{ll}  U_0  (z/d)^{1/7} &   z<d  \\
                                      U_0              &   z>d \end{array} \right. \ee
a simple slip boundary condition can be worked out such that
\be u_{ij,0} = u_{ij,1} \; \left( 1- \frac{d}{\dz} \right)  \quad ; \quad  v_{ij,0} = v_{ij,1} \; \left( 1- \frac{d}{\dz} \right) \ee
where $\dz$ is the height of the grid cell. For coarse grids,
the boundary condition is very similar to a ``free-slip'' condition because $d << \dz$.

Past and present versions of FDS have employed a default boundary condition for LES calculations
where $u_{ij,0}$ and $v_{ij,0}$ are set to one-half the values of
$u_{ij,1}$ and $v_{ij,1}$, respectively. Although from the discussion above this implies that the thickness of the
boundary layer $d$ is assumed to
be half that of the grid cell $\dz$, in practice, it really implies that real walls
are covered with small obstructions that are half a cell thick or
less in size, in which case they do not appear explicitly in the simulation.
The ``half-slip'' condition is an attempt to account for this wall clutter.

In a DNS calculation, this discussion is moot. The inner-wall ``ghost'' cell velocities are
set so that the velocity at the wall surface is zero.



\subsubsection{Force Terms}

The external force term components are defined at cell faces, just like the velocity components. For example, the drag force components
from sprinkler droplets are:
\begin{eqnarray}
f_{x,ijk} &=&\frac{1}{2}\frac{\sum \rho C_D\pi r_d^2 (u_d-u_{ijk}) |\bu_d-\bu|}{\dx\, \dy\, \dz}  \\
f_{y,ijk} &=&\frac{1}{2}\frac{\sum \rho C_D\pi r_d^2 (v_d-v_{ijk}) |\bu_d-\bu|}{\dx\, \dy\, \dz}  \\
f_{z,ijk} &=&\frac{1}{2}\frac{\sum \rho C_D\pi r_d^2 (w_d-w_{ijk}) |\bu_d-\bu|}{\dx\, \dy \,\dz}
\end{eqnarray}
where $r_d$ is the radius of a droplet, $\bu=(u_d,v_d,w_d)$ the velocity of a droplet, $C_D$ the drag
coefficient, and $\dx \, \dy \,\dz$ the volume of the $ijk$th cell.
The summations represent all droplets within a grid cell centered about
the $x$, $y$ and $z$ faces of a grid cell respectively.


\subsubsection{Time Step}

\label{stability}

The time step is constrained by the convective and diffusive
transport speeds via two conditions. The first is known as the
Courant-Friedrichs-Lewy (CFL) condition:
\be \dt \; \max \left(\frac{|\hu_{ijk}|}{\dx},\frac{|\hv_{ijk}|}{\dy},\frac{|\hw_{ijk}|}{\dz} \right) < 1  \label{cfl} \ee
The estimated velocities
$\hu^{(n+1)_e}$, $\hv^{(n+1)_e}$ and $\hw^{(n+1)_e}$ are tested at each
time step to ensure that the CFL condition is satisfied. If it is not,
then the time step is set to 0.8 of its allowed maximum value
and the estimated velocities are recomputed (and checked again).
The CFL condition asserts that the solution of the equations cannot be updated with a time step
larger than that allowing a parcel of fluid to cross a grid cell. For most large-scale calculations where
convective transport dominates diffusive, the CFL condition restricts the time step.

However, in small, finely-gridded domains, a second condition often dominates:
\be 2 \, \max \left( \nu , D , \frac{k}{\rho c_p} \right)  \; \dt  \left(
              \frac{1}{\dx^2}+\frac{1}{dy^2}+\frac{1}{\dz^2} \right) < 1  \label{vn} \ee
Note that this constraint is applied to the momentum, mass and energy equations via the
relevant diffusion parameter -- viscosity, material diffusivity or thermal conductivity.
This constraint on the time step, often referred to as the Von Neumann criterion, is typical
of any explicit, second-order numerical scheme for solving a parabolic partial differential
equation. To save CPU time, the Von Neumann criterion is only invoked for DNS calculations or for LES
calculations with grid cells smaller than 5~mm.

\clearpage
\subsection{The Pressure Equation}

The momentum equation is
\be \dod{\bu}{t} + \bF + \nabla \cH = 0 \quad ; \quad \cH = \frac{|\bu|^2}{2} + \frac{\tp}{\rho}  \ee
The divergence operator applied to the three components of the
discretized momentum equation (\ref{umom}--\ref{wmom}) yields a
single elliptic partial differential equation for the modified pressure, $\hp$, known as the Poisson equation:
\begin{eqnarray}
\frac{\hp_{i+1,jk}-2\hp_{ijk}+\hp_{i-1,jk}}{\dx^2} +
\frac{\hp_{i,j+1,k}-2\hp_{ijk}+\hp_{i,j-1,k}}{\dy^2} +
\frac{\hp_{ij,k+1}-2\hp_{ijk}+\hp_{ij,k-1}}{\dz^2} \nonumber \\ =
    -\frac{F_{\x,ijk} - F_{\x,i-1,jk}}{\dx}
    -\frac{F_{\y,ijk} - F_{\y,i,j-1,k}}{\dy}
    -\frac{F_{\z,ijk} - F_{\z,ij,k-1}}{\dz} - \dod{ }{t}(\nabla \cdot \bu)_{ijk}
\end{eqnarray}
The lack of a superscript implies that all quantities are to be
evaluated at the same time level.
This elliptic partial differential equation is solved using a direct
(non-iterative) FFT-based solver~\cite{Sweet:1} that is part of a library of routines
for solving elliptic PDEs called CRAYFISHPAK\footnote{CRAYFISHPAK, a vectorized form of the
elliptic equation solver FISHPAK, was originally developed at the National Center for Atmospheric
Research (NCAR) in Boulder, Colorado.}.
To ensure that the divergence of the fluid is consistent with the definition
given in Eq.~(\ref{phi}), the time derivative of the divergence is defined
\be \dod{ }{t}(\nabla \cdot \bu)_{ijk} =
          \frac{(\nabla \cdot \bu)_{ijk}^{(n+1)_e}
              - (\nabla \cdot \bu)_{ijk}^n}{\dt} \ee
at the predictor step, and then
\be \dod{ }{t}(\nabla \cdot \bu)_{ijk} =
         \frac{2(\nabla \cdot \bu)_{ijk}^{n+1} -
         (\nabla \cdot \bu)_{ijk}^{(n+1)_e}
       - (\nabla \cdot \bu)_{ijk}^n}{\dt} \ee
at the corrector step. The discretization of the divergence was
given in Eq.~(\ref{divdis}).

Direct Poisson solvers are most efficient if the domain is a
rectangular region, although other geometries such as cylinders
and spheres can be handled almost as easily. For these solvers,
a no-flux condition
is simple to prescribe at external boundaries.
For example, at the floor, $z=0$, the Poisson solver is
supplied with the Neumann boundary condition
\be \frac{\hp_{ij,1}-\hp_{ij,0}}{\dz} = -F_{z,ij,0} \label{dbc} \ee
However, many practical problems involve more
complicated geometries. For building fires,
doors and windows within multi-room enclosures are very important features
of the simulations. These elements may be included
in the overall domain as masked grid cells,
but the no-flux condition (\ref{dbc}) cannot be directly prescribed
at the boundaries of these blocked cells.
Fortunately, it is possible to exploit
the relatively small changes in the pressure from one time
step to the next to enforce the no-flux condition.
At the start of a time step,
the components of the convection/diffusion term $\bF$ are computed
at all cell faces that do not correspond to walls.
At those cell faces that do correspond to solid walls, we prescribe
\be
F_n = -\dod{{\cal H}}{n}^* + \frac{u_n}{\dt} \label{sbc}
\ee
where $F_n$ is the normal component of $\bF$ at the wall.

The asterisk indicates the most recent value of the pressure. The pressure at this
particular time step is not known until the Poisson equation is solved.
Equation~(\ref{sbc}) asserts that following the solution of the Poisson
equation for the pressure, the normal component of velocity $u_n$ will
be driven nearly (but not exactly) to zero.
This is approximate because the true value of the velocity time
derivative depends on the solution of the pressure equation, but since
the most recent estimate of pressure is used, the approximation is very
good. Also, even though there are small errors in normal velocity at solid
surfaces, the divergence of each blocked cell
remains exactly zero for the duration of the calculation.
In other words, the total flux into a given obstruction is always identically
zero, and the error in normal velocity is usually at least
several orders of magnitude smaller than the characteristic flow velocity.
When implemented as part of a predictor-corrector updating scheme,
the no-flux condition at solid surfaces is maintained remarkably well.

At open boundaries (say $i=I$),
$\cal H$ is prescribed depending on whether the
flow is incoming or outgoing
\be
\begin{array}{ll}
     {\cal H}_{I+\ha,jk} = (u_{I,jk}^2 + v_{I,j-\ha,k}^2 + w_{I,j-\ha,k}^2)/2  & u_{I,jk} > 0    \\
     {\cal H}_{I+\ha,jk} = 0      & u_{I,jk} < 0
     \end{array} \ee
where $I$ is the index of the last gas phase cell in the $x$ direction and
$u_{I,jk}$ is the $x$ component of velocity at the boundary.
The value of $\cal H$ in the ghost cell is
\be {\cal H}_{I+1,jk} = 2 {\cal H}_{I+\ha,jk} - {\cal H}_{I,jk} \ee

\subsubsection{Multiple Mesh Considerations (On-Going Research)}

Solving the Poisson equation on muliple meshes is considerably more difficult than on a single
mesh. On a single mesh, the linear system of equations involves a simple block tri-diagonal
matrix for which efficient solvers have been developed. For multiple mesh calculations, the same
solver is used for each mesh as described above, with the following (Dirichlet) boundary conditions applied at mesh
interfaces (in this case, the interface $x=x_{\max}=x^+$):
\be
\begin{array}{ll}
     {\cal H}_{I+\ha,jk} = \cH_{I,jk} - \frac{\dx}{2} \left(\frac{u^+_{0,jk}-u_{I,jk}}{\dt}+F_{x,Ijk} \right) & u_{I,jk} > 0    \\
     {\cal H}_{I+\ha,jk} = {\cal H}_{-\ha,jk}^+      & u_{I,jk} < 0
     \end{array} \ee
Here, the term $u^+$ indicates that the value is from the other mesh. In short, if the flow is incoming, then
$\cH$ is taken from the other mesh, if the flow is outgoing, then $\cH$ is chosen so that the value of $u$
at the next time step is driven
towards the value of its counterpart ($u^+$) in the other mesh. For abutting meshes of the same cell size,
$u_{I,jk}$ and $u^+_{0,jk}$ overlap each other, and the goal is to keep their values as close as possible.

This procedure is fairly stable, and provides a reasonable exchange of momentum mesh to mesh.
However, there is no guarantee that the individual solutions
of the Poisson equation on each mesh lead to consistent flows into and out of neighboring meshes. Using
similar boundary conditions as described above for open boundaries, it is possible to ensure a continuous
solution of the pressure term, ${\cal H}$, across mesh boundaries, but it is not possible to ensure a
continuous gradient. Thus, it is not possible to ensure that the flow out of one mesh is the same as the
flow into the next.

The strategy of approximating the solution of the Poisson equation on multiple meshes is as follows. First,
decompose the pressure term ${\cal H}={\cal H}^0 + {\cal H}'$. Solve
\be \nabla^2 {\cal H}^0 = -\nabla \cdot \bF - \dod{}{t} (\nabla \cdot \bu) \ee
on each mesh, the same way as described above. At this point, each mesh has a pressure field, but when
the velocity is updated in each mesh, the normal components will not match at mesh interfaces. At the very
least, we need ensure that the volume flux, $\dot{V} = \int \bu \cdot \dS  $,
at each mesh interface is consistent mesh to mesh. What should the volume flux be? There is no way to
get it locally. Instead, re-discretize the original Poisson equation over a coarse mesh spanning the entire
domain, and apply the divergence theorem to the coarse grid cells:
\be \int_{\partial \Omega_m} \nabla \cH \cdot \dS = -\int_{\partial \Omega_m} \bF \cdot \dS
    - \int_{\Omega_m} \dod{(\nabla \cdot \bu)}{t} \; dV \ee
The subscript $m$ refers to individual cells of the coarse mesh.
Introduce a single scalar $\overline{\cH}_m$ for each coarse grid cell that satisfies:
\be \frac{\overline{\cH}_{m+}  - \overline{\cH}_{m} }{\dx} A_{x^+} -
    \frac{\overline{\cH}_{m  } - \overline{\cH}_{m-}}{\dx} A_{x^-} + \; \cdots =
    -\int_{x^+} F_x \, dy \, dz  + \int_{x^-} F_x \, dy \, dz  + \;  \cdots
    - \int_{\Omega_m} \dod{(\nabla \cdot \bu)}{t} dV   \ee
The subscripts $m^+$ and $m^-$ refer to coarse neighboring cells, and $A_{x^+}$ and $A_{x^-}$ and so on
refer to the area of the coarse cell interfaces. At a mesh interface, say an $x$ interface, there are two values of $F_x$.
Denote the average of the two values $\overline{F}_x$, and use this in the coarse Poisson solve.

Solve the coarse grid Poisson equation with whatever sparse method that is best. At the interface between two
neighboring meshes, we now have an estimate of the rate of change of the volume flux. For example, at an $x$ interface:
\be \left. \frac{d\dot{V}}{dt}  \right|_{x^+} = -\int_{x^+} \overline{F}_x \, dy \, dz
- \frac{\overline{\cH}_{m+} - \overline{\cH}_m}{\dx} A_{x^+} \ee
We want the fine grid solution of the Poisson equation on the individual abutting meshes to satisfy this
constraint at their interface. To force this, solve
\be \nabla^2 {\cal H}' = 0 \ee
with Neumann boundary conditions on each of the abutting meshes
such that at the interface the gradient of $\cH'$ is constant. For example, at the $x^+$ interface, the gradient of
$\cH'$ on the mesh to the left must satisfy:
\be \left. \frac{d\dot{V}}{dt} \right|_{x^+} =
    \int_{x^+} \left( F_{x,Ijk}  + \frac{\cH^0_{I+1,jk}-\cH^0_{Ijk}}{\dx} \right) \; dy \, dz +
   \int_{x^+} \dod{\cH'}{x} \; dy \, dz \label{constraint} \ee
A similar constraint must be satisfied by the mesh to the right.
In this way, the resulting volume fluxes at the next time step $\dot{V}=\int \bu \cdot \dS$ into and
out of each mesh match. It does not
guarantee, however, that the normal component of velocity at each cell matches its counterpart in the other
mesh.

The question now is -- what values should be assigned to $\nabla \cH'$ at each cell of each mesh interface?
The only constraint is that the
integral $\int \nabla \cH' \cdot \dS$ at each face satisfy an equation like (\ref{constraint}) above. Various strategies
have been tried, but as yet no stable, robust strategy has been found. The research on this problem is on-going.




\clearpage
\section{Combustion}

\subsection{Heat Release Rate (Mixture Fraction)}


When using the mixture fraction-based combustion model, we must extract the local heat release rate per unit
volume from the computed mixture
fraction field. The mixture fraction, $Z$, is partitioned into at least 2 components, $Z_\alpha$, such that the
sum of the components equals the mixture
fraction. Each component is tracked via a transport equation, and the conversion of mass from one
component to another represents a reaction step and an associated release of energy.


\subsubsection{Two-Parameter Mixture Fraction}

When the mixture fraction is divided into two components, $Z_1$ and $Z_2$, there is one chemical
reaction that converts $Z_1$ to $Z_2$.  Recall from Section~\ref{extinction}
that this represents single-step combustion with the possibility of local extinction.  From the mixture fraction
variables one can determine the amount of fuel and oxygen present in a grid cell.  Ideally, we could use these values
in a finite-rate computation to determine the heat release rate; however, for most computations, the
grid resolution is too coarse to resolve the flame.  Thus, flame temperatures will not be realized and a
finite-rate computation will not succeed.

Instead, if any grid cell surrounding one containing both fuel and oxygen satisfies the ``Burn'' criteria depicted in Fig.~\ref{plotsupp},
combustion is assumed to occur.  The heat release rate is computed as
\be \dq''' = \min \left[ \frac{ \max \left( \rho Y_\F , s \rho Y_\OTWO \right) }{\dt} \, \Delta H \; , \; \dq_{\max}'''  \right]
  \quad ; \quad s=\frac{W_\F}{\nu_{\OTWO} W_{\OTWO} }   \ee
where $\dq_{\max}'''$ provides an upper bound of the volumetric heat release rate.
It is an empirical parameter that is tied to the resolution of the
underlying grid:
\be \dq_{\max}''' = \frac{ \dq_{\max}''}{(\dx \dy \dz)^\ot }  \quad ; \quad  \dq_{\max}'' = 200 \; \; \hbox{kW/m}^2  \ee
The 200~kW/m$^2$ value is a user-controlled parameter.  It is an estimate of the maximum heat release rate per unit area of flame sheet.

Once the heat release rate is computed, the mixture fraction variables are updated:
\be {Z_1}^{n+1} = {Z_1}^n - \frac {\dq''' {\Delta t}}{\rho  \Delta H}  \quad ; \quad  {Z_2}^{n+1} =
{Z_2}^n + \frac {\dq''' {\Delta t}}{\rho  \Delta H} \ee
Note that the total fuel mass is conserved in this process; $Z=Z_1+Z_2$ is still a conserved quantity.


\subsubsection{Three-Parameter Mixture Fraction}

When the mixture fraction is divided into three components, $Z_1$, $Z_2$, and $Z_3$, there are two chemical
reactions that convert $Z_1$ to $Z_2$ and $Z_2$ to $Z_3$.  Recall from Section~\ref{co_production}
that this represents two-step combustion (fuel to CO and CO to CO$_2$).
The first step occurs as it does for the two-parameter mixture fraction with a modified heat of combustion that
accounts for the conversion of fuel to CO rather than CO$_2$.
The second step is performed for all grid cells that contain CO and O$_2$.   If $\dq''' \neq 0$ in a grid
cell after the first step, then additional heat is released according to
\be \dq_{\CO}''' = \min \left[ \frac{ \max \left( \rho Z_2 , s \rho Y_\OTWO \right) }{\dt} \,
  \Delta H_\CO \; , \; \dq_{\max}'''-\dq'''  \right]     \ee
If $\dq'''=0$ after the first step, then it is presumed that the cell is out of the combustion region (say in the upper layer of
smoke-filled compartment), and a finite-rate reaction computation is performed to convert CO to CO$_2$ (see the next section for
a discussion of the algorithm for computing a finite-rate reaction).  The $\dq'''_{CO}$ computed using the finite-rate
reaction is still limited by $\dq_{\max}'''$.  Once $\dq'''_{\CO}$ is computed the mixture fraction variables are updated:
\be {Z_2}^{n+1} = {Z_2}^n - \frac {\dq'''_{CO}{\Delta t}}{\rho \, \Delta H_{CO}} \quad ; \quad
{Z_3}^{n+1} = {Z_3}^n + \frac {\dq'''_{CO}{\Delta t}}{\rho \, \Delta H_{CO}} \ee




\subsection{Heat Release Rate (Finite-Rate Reaction)}

For simulations that do not employ a mixture fraction-based combustion model,
multi-step, finite-rate reactions are assumed, taking the form:
\be  \mathrm{ \nu_{C_x H_y} \, C_x H_y
   + \nu_{O_2} \, O_2 \longrightarrow
     \nu_{CO_2}\,  CO_2 +
     \nu_{H_2 O} \, H_2O }  \ee
It is assumed that the chemical reaction
time scale is much shorter than any convective or diffusive
transport time scale. Thus, it makes sense to calculate the
consequences of the reaction assuming all other processes are
frozen in a state corresponding to the beginning of the time step.
For each grid cell, at the start of a time step where $t=t^n$ and
$Y_{C_x H_y,ijk}^n/\rho_{ijk} W_F       \equiv X_\F(t^n)$ and
$Y_{\OTWO,ijk}^n    /\rho_{ijk} W_\OTWO \equiv X_\OTWO(t^n)$,
the following set of ODEs is solved numerically with a second-order Runge-Kutta scheme
\begin{eqnarray}
\frac{dX_\F}{dt}    &=& -B \; X_\F(t)^a \, X_\OTWO(t)^b \; e^{-E/RT_{ijk}} \label{finitefuel} \\
\frac{dX_\OTWO}{dt} &=& -\frac{\nu_\OTWO}{\nu_\F} \; \frac{dX_\F}{dt} \label{finiteoxidizer}
\end{eqnarray}
The temperature $T_{ijk}$ and density $\rho_{ijk}$ are fixed at
their values at time
$t^n$ and the ODEs are iterated from $t^n$ to $t^{n+1}$ in about 20 time
steps. The pre-exponential factor $B$, the activation energy
$E$, and the exponents $a$ and $b$ are input parameters which are in typically assigned the values of
$\nu_\F$ and $\nu_\OTWO$.  At the end of each sub-time step the values of $X_\F$ and $X_\OTWO$ are updated.

The average heat release rate over the entire time step is given by
\be \dq_{ijk}^{'''n} = \Delta H \; \rho_{ijk}^n \frac{Y_\F(t^n) - Y_\F(t^{n+1})}{\dt} \ee
where $\dt = t^{n+1}-t^n$.
The species mass fractions are adjusted at this point in the calculation
(before the convection and diffusion update)
\be Y_{\alpha,ijk}^n = Y_\alpha(t^n) - \frac{\nu_\alpha \, W_\alpha}{\nu_\F \, W_\F} \big(Y_\F(t^n)-Y_\F(t^{n+1})\big) \ee
If multiple chemical reactions have been specified, equations \ref{finitefuel} and \ref{finiteoxidizer}
are evaluated for each reaction during each of the 100 time steps.  The reactions are evaluated in the order
that they are entered in the input file.



\clearpage
\section{Thermal Radiation}
\label{radnumericalmethodsection}

This section describes how $\nabla \cdot \dbq_r''$ (the radiative loss term) is computed at all gas-phase cells. Plus, it describes how
the radiative and convective heat fluxes, $\dq_r''$ and $\dq_c''$, are computed at solid boundaries.

\subsection{Discretization of the Radiative Heat Transfer Equation}

The radiative transport equation~(\ref{bandRTE}) is solved using
techniques similar to those for convective transport in finite volume
methods for fluid flow~\cite{Raithby}, thus the name given to it is the
Finite Volume Method (FVM). To obtain the discretized form of the RTE,
the unit sphere is divided into a finite number of solid angles.
In each grid cell a discretized equation is derived by integrating
Eq.~(\ref{bandRTE}) over the volume of cell $ijk$ and the control angle
$\delta \Omega^l$, to obtain
\be
  \int_{\delta \Omega^l} \int_{V_{ijk}}
   \bs \cdot \nabla I(\bx',\bs') d\bx' d\bs' =
   \int_{\delta \Omega^l} \int_{V_{ijk}} \kappa(\bx') \;
    \left[ I_{b}(\bx') - I(\bx',\bs') \right] d\bx' d\bs'
\ee
The volume integral on the left hand side is replaced by a surface integral
over the cell faces using the divergence theorem. Note that the procedure outlined below is
appropriate for each band of a narrow band model, thus the subscript $n$ has been removed for clarity.

Assuming that the radiation intensity $I(\bx,\bs)$ is constant
on each of the cell faces, the surface integral can be approximated
by a sum over the cell faces.
Assuming further that $I(\bx,\bs)$ is constant within
the volume $V_{ijk}$ and over the angle $\delta \bO^l$ we obtain
\be  \sum_{m=1}^6 A_m \; I_m^l \;
      \int_{\Omega^l}(\bs' \cdot \bn_m) d\bs'
   = \kappa_{ijk} \,
     \left[ I_{b,ijk} - I_{ijk}^l \right] \; V_{ijk} \,
     \delta \Omega^l   \label{RTEdiscrete2}
\ee
where
\begin{tabbing}
$I_{ijk}^l$ \hspace{1in}  \=  radiant intensity in direction $l$ \\
$I_m^l$                   \>  radiant intensity at cell face $m$ \\
$I_{b,ijk}$               \>  radiant blackbody Intensity in cell \\
$\delta \Omega^l$         \>  solid angle corresponding to direction $l$ \\
$V_{ijk}$                 \>  volume of cell $ijk$ \\
$A_m$                     \>  area of cell face $m$ \\
$\bn_m$                   \>  unit normal vector of the cell face $m$
\end{tabbing}
Note that while the intensity is assumed constant within
the angle $\delta \bO^l$, its direction covers the angle $\delta \bO^l$
exactly.

In Cartesian coordinates\footnote{In the axisymmetric case
equation~(\ref{RTEdiscrete2}) becomes
a little bit more complicated, as the cell face normal vectors $\bn_m$
are not always constant. However, the computational efficiency can still be
retained.},
the normal vectors $\bn_m$ are the base
vectors of the coordinate system and the integrals over the solid
angle do not depend on the physical coordinate, but the direction
only. The intensities on the cell boundaries, $I_m^l$, are calculated
using a first-order upwind scheme.  If the physical space is swept in
the direction $\bs^l$, the intensity $I_{ijk}^l$ can be directly obtained
from an algebraic equation. This makes the numerical solution of the
FVM very fast.  Iterations are needed only to account for the
reflective boundaries. However, this is seldom necessary in
practice, because the time step set by the flow solver is small.

The grid used for the RTE solver is the same as for
the fluid solver.
The coordinate system used to discretize the solid angle is
shown in Figure~\ref{Angular}.
\begin{figure}[ht]
\begin{center}
\includegraphics[height=2in]{FIGURES/RadCoord}
\caption{Coordinate system of the angular discretization.}
\label{Angular}
\end{center}
\end{figure}
The discretization of the solid angle is done by dividing first
the polar angle, $\theta$, into $N_{\theta}$ bands, where
$N_{\theta}$ is an even integer.
Each $\theta$-band is then divided into
$N_{\phi}(\theta)$ parts in the azimuthal ($\phi$) direction.
$N_{\phi}(\theta)$ must be divisible by 4.
The numbers $N_{\theta}$ and $N_{\phi}(\theta)$ are chosen
to give the total number of angles $N_{\Omega}$ as close to
the value defined by the user as possible.
$N_{\Omega}$ is calculated as
\be
 N_{\Omega} = \sum_{i=1}^{N_{\theta}} N_{\phi}(\theta_i)
\ee
The distribution of the angles is based on empirical rules that try
to produce equal solid angles $\delta \bO^l = 4\pi/N_{\Omega}$. The
number of $\theta$-bands is
\be
 N_{\theta} = 1.17 \; N_{\Omega}^{1/2.26}
\ee
rounded to the nearest even integer. The number of $\phi$-angles
on each band is
\be
 N_{\phi}(\theta) = \max\left\{4,
        0.5\,N_{\Omega}\,\left[\cos(\theta^-)-\cos(\theta^+)\right]\right\}
\ee
rounded to the nearest integer that is divisible by 4.
$\theta^-$ and $\theta^+$ are
the lower and upper bounds of the $\theta$-band, respectively.
The discretization is symmetric with respect to the planes $x=0$, $y=0$, and
$z=0$. This symmetry has three important benefits:
First, it avoids the problems caused by the fact that the first-order
upwind scheme, used to calculate intensities on the cell boundaries,
is more diffusive in non-axial directions than axial.
Second, the treatment of the mirror boundaries becomes very simple, as
will be shown later. Third,
it avoids so called
``overhang'' situations, where $\bs\cdot {\bf i}$, $\bs\cdot {\bf j}$
or $\bs\cdot {\bf k}$ changes sign inside
the control angle. These ``overhangs'' would make the resulting system of
linear equations more complicated.

In the axially symmetric case these ``overhangs'' can not be avoided, and a
special treatment, developed by Murthy and Mathur~\cite{Murthy}, is
applied. In these cases $N_{\phi}(\theta_i)$ is kept constant, and
the total number of angles is $N_{\Omega} = N_{\theta} \times
N_{\phi}$. In addition, the angle of the vertical slice of the cylinder is
chosen to be same as $\delta\phi$.

The cell face intensities, $I_m^l$ appearing on the left hand side of
(\ref{RTEdiscrete2}) are calculated using a first order
upwind scheme. Consider, for example, a control angle having a
direction vector $\bs$. If the radiation is traveling in the positive
$x$-direction, {\em i.e.} $\bs\cdot {\bf i} \geq 0$, the intensity on the
upwind side, $I_{xu}^l$ is assumed to be
the intensity in the neighboring cell, $I_{i-1\,jk}^l$,
and the intensity on the downwind
side is the intensity in the cell itself $I_{ijk}^l$.

On a rectilinear grid, the normal vectors $\bn_m$ are the base vectors
of the coordinate system and the integrals over the solid angle can be
calculated analytically. Equation (\ref{RTEdiscrete2}) can be simplified
\be
  a_{ijk}^l I_{ijk}^l =
  a_{x}^l   I_{xu}^l +
  a_{y}^l   I_{yu}^l +
  a_{z}^l   I_{zu}^l +   b_{ijk}^l \label{RTEdiscrete3}
\ee
where
\begin{eqnarray}
  a_{ijk}^l & = & A_x |D_x^l| + A_y |D_y^l| +A_z |D_z^l| +
        \kappa_{ijk} \; V_{ijk} \delta\Omega^l \\
  \nonumber \\
  a_{x}^l & = & A_x |D_x^l| \\
  a_{y}^l & = & A_y |D_y^l| \\
  a_{z}^l & = & A_z |D_z^l| \\
  \nonumber \\
  b_{ijk}^l & = &
  \kappa_{ijk} \; I_{b,ijk} \; V_{ijk} \; \delta\Omega^l \\
  \nonumber \\
  \delta \Omega^l & = & \int_{\Omega^l}d\Omega
          = \int_{\dph}\int_{\dth} \sin\theta \; d\theta \; d\phi \\
  \nonumber \\
   D_x^l & = & \int_{\Omega^l}(\bs^l\cdot {\bf i})d\Omega \\
         & = & \int_{\dph}\int_{\dth} (\bs^l\cdot{\bf i})
                        \sin\theta \; d\theta d\phi \nonumber \\
         & = & \int_{\dph}\int_{\dth} \cos\phi\sin\theta\sin\theta\;
                        d\theta d\phi \nonumber \\
         & = & \frac{1}{2}\left( \sin\phi^+ - \sin\phi^- \right)
               \left[\Delta\theta - \left(\cos\theta^+\sin\theta^+
                          - \cos\theta^-\sin\theta^-\right)\right] \nonumber \\
   D_y^l & = & \int_{\Omega^l}(\bs^l\cdot {\bf j})d\Omega \\
         & = & \int_{\dph}\int_{\dth} \sin\phi\sin\theta\sin\theta\;
                        d\theta d\phi \nonumber \\
         & = & \frac{1}{2}\left( \cos\phi^- - \cos\phi^+ \right)
               \left[\Delta\theta - \left(\cos\theta^+\sin\theta^+
                          - \cos\theta^-\sin\theta^-\right)\right]\nonumber \\
   D_z^l & = & \int_{\Omega^l}(\bs^l\cdot {\bf k})d\Omega \\
         & = & \int_{\dph}\int_{\dth} \cos\theta\sin\theta\;
                        d\theta d\phi \nonumber \\
         & = & \frac{1}{2}\Delta\phi
               \left[\left(\sin\theta^+\right)^2 -
                     \left(\sin\theta^-\right)^2\right] \nonumber
\end{eqnarray}
Here $\bf i$, $\bf j$ and $\bf k$ are the base vectors of the Cartesian
coordinate system. $\theta^+$, $\theta^-$, $\phi^+$ and $\phi^+$ are
the upper and lower boundaries of the control angle in the polar and
azimuthal directions, respectively,
and $\Delta\theta = \theta^+ - \theta^-$ and
$\Delta\phi = \phi^+ - \phi^-$. The solution method of (\ref{RTEdiscrete3}) is
based on an explicit marching sequence~\cite{Kim}. The marching direction
depends on the propagation direction of the radiation
intensity. As the marching is done in the ``downwind'' direction,
the ``upwind''
intensities in all three spatial directions are known, and the
intensity $I_{ijk}^l$ can be solved directly. Iterations may be needed
only with the reflective walls and optically thick situations.
Currently, no iterations are made.

\subsection{Radiation Heat Transfer to Solid Objects}

The boundary condition on a solid wall is given as
\be I_w^l = \epsilon \; \frac{\sigma T_w^4}{\pi} + \frac{1-\epsilon}{\pi} \sum_{D_w^{l'}<0} I_w^{l'}\; |D_w^{l'} |  \ee
where $D_w^{l'}= \int_{\Omega^{l'}}(\bs\cdot \bn_w)d\Omega$.
The constraint $D_w^{l'}<0$ means that only the ``incoming'' directions
are taken into account when calculating the reflection.
The {\em net} radiative heat flux on the wall is
\be \dq_r'' = \sum_{l=1}^{N_{\Omega}} I_w^l \int_{\delta \Omega^l} (\bs' \cdot \bn_w) \, d\bs'
     = \sum_{l=1}^{N_{\Omega}} I_w^l D_n^l \label{qrdef} \ee
where the coefficients $D_n^l$ are equal to $\pm D_x^l$, $\pm D_y^l$ or
$\pm D_z^l$, and can be calculated for each wall element at the start of the
calculation.

The open boundaries are treated as black walls, where the incoming intensity is
the black body intensity of the ambient temperature. On mirror
boundaries the intensities leaving the wall
are calculated from the incoming intensities using a
predefined connection matrix:
\be  I_{w,ijk}^l = I^{l'} \ee
Computationally intensive integration over all the incoming directions
is avoided by keeping the solid angle discretization symmetric on the $x$, $y$ and $z$ planes.
The connection matrix associates one incoming direction $l'$ to each mirrored direction on each wall cell.

The local incident radiation intensity is
\be
 U_{ijk} = \sum_{l=1}^{N_{\Omega}} I_{ijk}^l \delta\Omega^l
\ee





\clearpage
\section{Solid Phase Model}

\subsection{Solid Surface and Interior Temperatures}

A one dimensional heat transfer calculation is performed at each solid
boundary cell for which the user has prescribed thermal
properties. The solid can consist of multiple layers of materials.
Each layer is partitioned into non-uniform cells, clustered near the
front and back faces.  The smallest cells are chosen based on the
criteria
\be \dx < S_s\sqrt{\frac{k_s}{\rho_s c_s}} \ee
where $S_s$ is a cell size factor defined by the user. By default,
$F_s$ is 1.0.  Interior cells increase in size by a user-defined
stretch factor when moving inwards from the surfaces. By default, the
stretch factor is 2.0. The cell boundaries are located at points
$x_i$. The temperature at the center of the $i$th cell is denoted $T_{s,i}$.
The (temperature-dependent) thermal conductivity of the solid
at the center of the $i$th cell is denoted $k_{s,i}$.
The temperatures are updated in time using an implicit
Crank-Nicolson scheme
\begin{eqnarray}
    \frac{T_{s,i}^{n+1}-T_{s,i}^n}{\dt} = \frac{1}{2 (\rho_s c_s)_i \dx_{i} }
& & \left(
    k_{s,i+\ha}\frac{T_{s,i+1}^n-T_{s,i}^n}{\dx_{s,i+\ha}} - \right.
    k_{s,i-\ha}\frac{T_{s,i}^n-T_{s,i-1}^n}{\dx_{s,i-\ha}} +  \nonumber \\
& & k_{s,i+\ha}\frac{T_{s,i+1}^{n+1}-T_{s,i}^{n+1}}{\dx_{i+\ha}} -
    \left.
    k_{s,i-\ha}\frac{T_{s,i}^{n+1}-T_{s,i-1}^{n+1}}{\dx_{i-\ha}}
     \right) \end{eqnarray}
for $1 \le i \le N$.
The width of each cell is $\dx_i$. The distance from the center of cell $i$ to the center
of cell $i+1$ is $\dx_{i+\ha}$.

The boundary condition is discretized
\be -k_{s,1} \frac{T_{s,1}^{n+1}-T_{s,0}^{n+1}}{\dx_{\ha}} = \dq''_c{}^{(n+1)} + \dq''_r{}^{(n+1)} \ee
The convective and radiative fluxes at the next time step are approximated
\begin{eqnarray}
\dq''_c{}^{(n+1)} &\approx& \dq''_c{}^n - h\left( T_{s,\ha}^{n+1} - T_{s,\ha}^n \right) \\
\dq''_r{}^{(n+1)} &\approx& \dq''_r{}^n - 4 \; \epsilon \; \sigma \; T_{s,\ha}^{n^3} \left(
  T_{s,\ha}^{n+1} - T_{s,\ha}^n \right)  \end{eqnarray}
The wall temperature is defined $T_w \equiv T_{s,\ha}=(T_{s,0}+T_{s,1})/2$.


\subsection{Convective Heat Transfer to Solid Objects}

In a DNS calculation where the boundary layer is resolved, the convective flux to the wall
is given by
\be
\dq''_c = -k \frac{T_{gas}-T_w}{\dn/2}
\ee
where $T_{gas}$ is the gas temperature in the center of the first gas-phase grid cell,
$T_w$ is the surface or ``wall'' temperature, $\dn$ is the
size of a grid cell in the normal direction to the wall, and $k$ is
the temperature-dependent thermal conductivity of the gas mixture.

In an LES calculation, where the boundary layer is not resolved, an empirical heat transfer coefficient, $h$, is used: \be \dq''_c = h \,
(T_{gas}-T_w)\quad \hbox{W/m}^2 \ee The heat transfer coefficient, $h$, is the maximum of its natural and forced empirical values:
 \be h = \max \left( C |T_{gas} -T_w|^\ot \, , \, \frac{k_{air}}{L} 0.037 \, \RE_L^{4/5} \PR^{1/3} \right)  \quad \hbox{W/m}^2\hbox{/K}  \ee
$C$ is an empirical coefficient (1.31 for vertical surface; 1.52 for horizontal), $k_{air}$ is
the thermal conductivity of air (not the ``turbulent''
value used in the LES solver), $L$ is a length scale (assumed to be 1 m), the Reynolds number
is based on the density and velocity of the gases in
the middle of the first grid cell and the length scale $L$, and the Prandtl number is assumed to be 0.7.


\subsection{Coupling the Gas and Solid Phase}

Gas phase temperatures are defined at cell centers; solid surfaces lie at the
interface of the bordering gas phase cell and a ``ghost'' cell inside the
solid. As far as the gas phase calculation is concerned,
the normal temperature gradient at the surface is expressed in terms of
the temperature difference between the ``gas'' cell and the ``ghost'' cell.
The solid surface temperature is not used directly in the gas phase calculation.
Rather, the ghost cell temperature is used to couple the gas and solid phases.
The ghost cell temperature
has no physical meaning on its own. Only the difference between ghost and
gas cell temperatures matters, for this defines the heat transfer to the
wall.

In a DNS calculation, the wall temperature is assumed to be an average of the
ghost cell temperature and the temperature of the first cell in the gas,
thus the ghost cell temperature is defined
\be T_{ghost} =  2 T_w - T_{gas}  \ee
For an LES calculation, the numerical expression for the heat lost to the boundary is
equated with the empirical convective heat transfer
\be k_{\hbox{\tiny LES}} \frac{T_{gas} - T_{ghost}}{\dn} - c_p u_n \rho \frac{T_{gas} + T_{ghost}}{2}
   = h \; (T_{gas}-T_s) - c_p u_n \rho T_s  \ee
where $\dn$ is the distance between the center of the ghost cell
and the center of the gas cell. This equation is solved for $T_{ghost}$,
so that when the conservation equations are updated, the amount of heat
lost to the wall is equivalent to the empirical expression on the right
hand side. Note that $T_{ghost}$ is purely a numerical construct. It does
not represent the temperature within the wall, but rather establishes
a temperature gradient at the wall consistent with the empirical
correlation.

At solid walls there is no transfer of mass, thus the boundary condition
for the $l$th species at a wall is simply
\be Y_{l,ghost} = Y_{l,gas} \ee
where the subscripts ``ghost'' and ``gas'' are the same as above since
the mass fraction, like temperature, is defined at cell centers.
At forced flow boundaries either the mass fraction $Y_{l,w}$ or
the mass flux $\dot{m}_l''$ of species $l$ may be prescribed.
Then the ghost cell mass fraction can be derived because, as with
temperature, the normal gradient of mass fraction is needed in the gas phase
calculation.
For cases where the mass fraction is prescribed
\be Y_{l,ghost} = 2 Y_{l,w} - Y_{l,gas}  \ee
For cases where the mass flux is prescribed,
the following equation must be solved iteratively
\be \dot{m}_l'' = u_n \frac{\rho_{ghost} Y_{l,ghost} + \rho_{gas} Y_{l,gas}}{2}
  - \rho D \frac{Y_{l,gas}-Y_{l,ghost}}{\dn}
  \mp \frac{\dt \, u_n^2}{2} \frac{\rho_{gas} Y_{l,gas}
   -\rho_{ghost} Y_{l,ghost}}{\dn} \ee
where $\dot{m}_l''$ is the mass flux of species $l$ per unit area,
$u_n$ is the normal component of velocity at the wall pointing into
the flow domain, and $\dn$ is the distance between the center of the ghost
cell and the center of the gas cell. Notice that the last term on the
right hand side is subtracted at the predictor step and added at the
corrector step, consistent with the biased upwinding introduced earlier.

Once the temperature and species mass fractions have been defined in the
ghost cell, the density in the ghost cell is computed from the equation of
state
\be  \rho_{ghost} = \frac{p_0}{ {\cal R} \, T_{ghost} \, \sum_l (Y_{l,ghost}/W_l) }  \ee









\clearpage

\section{Liquid Sprays}

Lagrangian particles are sometimes introduced into the flow field as a means of
visualization or simulation of droplets, aerosols, {\em etc.}. The
position $\bx_p$ of each particle is governed by the equations
\be \frac{d\bx_p}{dt} = \bu  \ee
The particle positions are updated according to the same
predictor-corrector
scheme that is applied to the other flow quantities. Briefly, the
position $\bx_p$ of a given particle is updated according to the two
step scheme
\begin{eqnarray}
\bx_p^{(n+1)_e} &=& \bx_p^n + \dt \, \overline{\bu}^n \\
\bx_p^{n+1} &=& \ha \left(\bx_p^n + \bx_p^{(n+1)_e} + \dt \,
\overline{\bu}^{(n+1)_e}  \right)
\end{eqnarray}
where the bar over the velocity vector indicates that the velocity of
the fluid is linearly interpolated at the particle's position.


\subsection{Interaction of Droplets and Radiation}

\label{dropnumericalmethod}

The computation of $\chi_f$ for a similar but simpler situation has
been derived in Ref.~\cite{Yang:3}. It can be shown that here
$\chi_f$ becomes
\be
\chi_f = \frac{1}{\delta\Omega^l}
\int_0^{\mu^l}\int_0^{\mu^l}\int_{\mu_{d,0}}^{\mu_{d,\pi}}
\frac{P_0(\theta_d)}{(1-\mu2)(1-\mu'^2)-(\mu_d-\mu\mu')2 }
\; d\mu_d \, d\mu \, d\mu'
\ee
where $\mu_d$ is a cosine of the scattering angle $\theta_d$ and
$P_0(\theta_d)$ is a single droplet scattering phase function
\be
P_0(\theta_d) =
\frac{\la^2\left(|S_1(\theta_d)|^2+|S_2(\theta_d)|^2\right)}{2 \, C_s(r,\la)}
\ee
$S_1(\theta_d)$ and $S_2(\theta_d)$ are the scattering amplitudes,
given by
Mie-theory. The integration limit $\mu^l$ is a cosine of the polar angle
defining the boundary of the symmetric control angle $\delta\Omega^l$
\be
\mu^l = \cos(\theta^l) = 1 - \frac{2}{N_\Omega}
\ee
The limits of the innermost integral are
\be
\mu_{d,0}   = \mu\mu' + \sqrt{1-\mu^2}\sqrt{1-\mu'^2}  \quad ; \quad
\mu_{d,\pi} = \mu\mu' - \sqrt{1-\mu^2}\sqrt{1-\mu'^2}
\ee
When $\chi_f$ is integrated over the droplet size distribution to get
an averaged value, it is multiplied by $C_s(r,\la)$. It is therefore
$|S_1|^2+|S_2|^2$, not $P_0(\theta_d)$, that is integrated. Physically,
this means that intensities are added, not
probabilities~\cite{Wiscombe}.


\subsection{Heating and Evaporation of Liquid Droplets}

The exchange of mass and energy between liquid droplets and the surrounding gases (or solid surfaces) is computed droplet by droplet. After the temperature of each droplet is computed, the
appropriate amount of vaporized liquid is added to the given mesh cell, and the cell gas temperature is reduced slightly based on the energy lost to the droplet.

Equation~(\ref{droplet_temp}) is solved semi-implicitly over the course of a gas phase time step as follows. Note that a few terms have been left out to make the algorithm more clear.
\be
   \frac{T_l^{n+1}-T_l^n}{\dt} = \frac{1}{m_l \, c_l} \left[ A \, h \, \left( T_g-\frac{T_l^{n+1}+T_l^n}{2} \right) - A \, h_m \, \rho \left( \frac{Y_l^{n+1}+Y_l^n}{2} - Y_g \right) h_v  \right]
\ee
The equilibrium vapor mass fraction, $Y_l^n$, is a function of $T_l^n$ via Eq.~(\ref{clausius_clapeyron}), and its value at the next time step is approximated via
\be
   Y_l^{n+1} \approx Y_l^n + \left( \frac{dY_l}{dT_l} \right)^n \; \Big( T_l^{n+1}-T_l^n \Big)
\ee
where the derivative of $Y_l$ with respect to temperature is obtained via the chain rule:
\be
   \frac{dY_l}{dT_l} = \frac{dY_l}{dX_l} \, \frac{dX_l}{dT_l}  = \frac{W_a/W_l}{ (X_l (1-W_a/W_l) + W_a/W_l)^2 } \;
     \frac{h_v W_l}{{\cal R} \, T_l^2} \, \exp \left[ \frac{h_v \, W_l}{\cal R} \left( \frac{1}{T_b}-\frac{1}{T_l} \right) \right]
\ee
The amount of evaporated liquid is given by
\be
   \delta m_l = \dt \, A \, h_m \, \rho  \left[  Y_l^n + \ha \left( \frac{dY_l}{dT_l} \right)^n \; \Big( T_l^{n+1}-T_l^n \Big)  \right]
\ee
The amount of heat extracted from the gas is
\be
   \delta q = \dt \, A \, h \, \left( T_g - \frac{T_l^n+T_l^{n+1}}{2} \right)
\ee














\chapter{Conclusion}

The equations and numerical algorithm described in this document form
the core of an evolving fire model. As research into specific
fire-related phenomena continues, the relevant parts of the model
can be improved. Because the model was originally designed to predict the
transport of heat and exhaust products from fires,
it can be used reliably when the fire is prescribed and the numerical grid
is sufficiently resolved to capture enough of the flow structure for the
application at hand. It is the job of the user to determine what level
of accuracy is needed.

Any user of the numerical model must be aware of the assumptions and
approximations being employed. There are two issues for any potential
user to consider before embarking on calculations.
First, for both real and simulated fires, the growth of the fire
is very sensitive
to the thermal properties (conductivity, specific heat, density, burning
rate, {\em etc.})
of the surrounding materials. Second, even if all
the material properties are known, the physical phenomena of interest
may not be simulated due to limitations in the model algorithms or
numerical grid. Except for those few
materials that have been studied to date at NIST,
the user must supply the thermal properties
of the materials, and then validate the performance of the model with
experiments to ensure that the model has the necessary physics included.
Only then can the model be expected to predict the
outcome of fire scenarios that are similar to those that have actually been
tested.


\bibliography{../Bibliography/FDS_refs,../Bibliography/FDS_general,../Bibliography/FDS_mathcomp}

\addcontentsline{toc}{chapter}{References}



\appendix

%\backmatter

\chapter{Nomenclature}
\label{nomenclature}

\begin{tabbing}
$A_s$ \hspace{1in}        \= droplet surface area \\
$A_{\alpha\beta}$          \> pre-exponential factor for solid phase Arrhenius reaction \\
$B$                       \> pre-exponential factor for gas phase Arrhenius reaction \\
$C$                       \> Sprinkler C-Factor \\
$C_D$                     \> drag coefficient \\
$C_s$                     \> Smagorinsky constant (LES)  \\
$c_s$             \> Solid material specific heat \\
$c_p$                     \> constant pressure specific heat \\
$D$                       \> diffusion coefficient   \\
$d_m$                     \> median volumetric droplet diameter \\
$E$                       \> activation energy \\
$\bof_b$                  \> external force vector (excluding gravity) \\
$g$                       \> acceleration of gravity \\
$\bg$                     \> gravity vector, normally $(0,0,-g)$ \\
$\cal H$                  \> total pressure divided by the density \\
$H_{r,\alpha\beta}$   \> heat of reaction for a solid phase reaction \\
$h$                       \> enthalpy; heat transfer coefficient   \\
$h_\alpha$                \> enthalpy of species $\alpha$   \\
$h_\alpha^0$              \> heat of formation of species $\alpha$   \\
$I$                       \> radiation intensity   \\
$I_b$                     \> radiation blackbody intensity   \\
$k$                       \> thermal conductivity; suppression decay factor \\
$\dm_f''$                 \> fuel mass flux \\
$\dm_\alpha'''$           \> mass production rate of species $\alpha$ per unit volume \\
$\dm_w''$                 \> water mass flux  \\
$m_w''$                   \> water mass per unit area \\
$\NU$                     \> Nusselt number \\
$\PR$                     \> Prandtl number \\
$p$                       \> pressure \\
$\bp_0$                   \> atmospheric pressure profile \\
$\bp_m$                   \> background pressure of $m$th pressure zone \\
$\tp$                     \> pressure perturbation \\
$\dbq''$                  \> heat flux vector \\
$\dq'''$                  \> heat release rate per unit volume \\
$\dq_r''$                 \> radiative flux to a solid surface \\
$\dq_c''$                 \> convective flux to a solid surface \\
$\dQ$                     \> total heat release rate \\
$Q^*$                     \> characteristic fire size \\
$\R$                      \> universal gas constant \\
$\RE$                     \> Reynolds number \\
$r_d$                     \> droplet radius \\
$r_{\alpha\beta}$     \> solid phase reaction rate \\
$\hbox{RTI}$              \> Response Time Index of sprinkler \\
$\bs$                     \> unit vector in direction of radiation intensity\\
$\SC$                     \> Schmidt number \\
$\SH$                     \> Sherwood number \\
$S_\alpha$        \> solid component production rate \\
$T$                       \> temperature \\
$t$                       \> time           \\
$U$                       \> integrated radiant intensity \\
$\bu=(u,v,w)$             \> velocity vector  \\
$W_\alpha$                \> molecular weight of gas species $\alpha$ \\
$\bW$                     \> molecular weight of the gas mixture \\
$\WE$                     \> Weber number \\
$\bx=(x,y,z)$             \> position vector  \\
$X_\alpha$                \> volume fraction of species $\alpha$   \\
$Y_\alpha$                \> mass fraction of species $\alpha$   \\
$Y_\OTWO^\infty$          \> mass fraction of oxygen in the ambient   \\
$Y_\F^I$                  \> mass fraction of fuel in the fuel stream   \\
$y_s$                     \> soot yield \\
$Z$                       \> mixture fraction   \\
$Z_f$                     \> stoichiometric value of the mixture fraction   \\
$\gamma$                  \> ratio of specific heats; Rosin-Rammler exponent \\
$\Delta H$                \> heat of combustion \\
$\Delta H_\OTWO$          \> energy released per unit mass oxygen consumed \\
$\delta$                  \> wall thickness \\
$\epsilon$                \> dissipation rate \\
$\kappa$                  \> absorption coefficient \\
$\mu$                     \> dynamic viscosity \\
$\nu_\alpha$              \> stoichiometric coefficient, species $\alpha$ \\
$\nu_s$           \> yield of solid residue in solid phase reaction \\
$\nu_f$           \> yield of gaseous fuel in solid phase reaction \\
$\nu_w$           \> yield of gaseous water in solid phase reaction \\
$\rho$                    \> density \\
$\btau_{ij}$              \> viscous stress tensor \\
$\chi_r$                  \> radiative loss fraction \\
$\sigma$                  \> Stefan-Boltzmann constant; constant in droplet size distribution; surface tension \\
$\sigma_d$                \> droplet scattering coefficient \\
$\sigma_s$                \> scattering coefficient \\
$\bo=(\omx,\omy,\omz)$    \> vorticity vector \\
\end{tabbing}





\chapter{Derivation of the Velocity Divergence}

\subsubsection{Randall McDermott, NIST Postdoctoral Fellow}

In this appendix we derive the divergence of the velocity field as presented in Eq.~(\ref{phi}).
Note that the constitutive relationships presented here for the mass diffusion and thermal heat fluxes are valid for direct numerical simulations (i.e., well-resolved calculations).
The minor modifications required of the transport coefficients for large-eddy simulation are presented in Section~\ref{LES}.  We start the derivation by rearranging the continuity equation.
Next, we differentiate the equation of state to reveal the relationship between transport equations for mass and energy.
We then show how the transport equations may be combined to yield the velocity divergence constraint.  In the last section we present the final result in FDS notation.

\subsubsection{Continuity Equation}
\label{continuity}

Let $\rho$ denote the fluid mass density; let $\mathbf{u} = [u,v,w]^T$ denote the fluid mass-average velocity; and let $\dot{m}_b^{\tripleprime}$
denote a bulk source of mass per unit volume (which may come from the evaporation of water droplets, for example).
The continuity equation may be rearranged to yield the following divergence constraint on the velocity
\begin{equation}
\label{eqn_divconstraint1}
\Div \mathbf{u} = \frac{1}{\rho} \left( \dot{m}_b^{\tripleprime} -  \frac{\mbox{D} \rho}{\mbox{D} t} \right)
\end{equation}
where $\mbox{D}(\,\,\,)/\mbox{D} t \equiv \partial (\,\,\,)/\partial t + \mathbf{u}\cdot\nabla(\,\,\,)$ is the material derivative.

\subsubsection{Equation of State}
\label{EOS}

We consider the transport of $n_s$ species mass fractions $Y_\alpha$ for $\alpha = \{1,\ldots,n_s\}$, $n_s-1$ of which are independent.
The molecular weight of a given species is denoted $W_\alpha$ and the molecular weight of the mixture, $\overline{W}$, is given by
\begin{equation}
\label{eqn_mixmolewt}
\overline{W} = \left( \sum_{\alpha} \frac{Y_\alpha}{W_\alpha} \right)^{-1}
\end{equation}
where as a shorthand notation, which is used throughout this document, we write $\sum_\alpha$ for $\sum_{\alpha = 1}^{n_s}$.
Let $\overline{p}_i(\mathbf{x},t)$ denote the hydrostatic pressure in the $i$th zone of the domain, which in general we take to be a function of space and time.
In practice, however, $\overline{p}_i = \overline{p}_i(t)$ for closed (i.e., sealed or pressurized) domains and $\overline{p}_i = \overline{p}_i(z)$,
where $z$ represents the coordinate aligned with the gravity vector, for large, open domains (e.g., forest fires large enough to interact with the stratified atmosphere).
The divergence constraint derived below is based on the ideal gas equation of state (EOS), which, for low-Mach flows, we write as
\begin{equation}
\label{eqn_idealgaslaw}
\overline{p}_i = \frac{\rho \mathcal{R} T}{\overline{W}}
\end{equation}
where $\mathcal{R} = 8.3145$~kJ/(kmol K) is the gas law constant.

Differentiating the EOS (\ref{eqn_idealgaslaw}) we obtain
\begin{equation}
\label{eqn_DEOS1}
\frac{\mbox{D} \overline{p}_i}{\mbox{D} t} = \rho \mathcal{R} T \frac{\mbox{D}}{\mbox{D} t}\left(\frac{1}{\overline{W}}\right) +
\frac{\rho \mathcal{R}}{\overline{W}} \frac{\mbox{D} T}{\mbox{D} t} + \frac{\mathcal{R}T}{\overline{W}} \frac{\mbox{D} \rho}{\mbox{D} t}
\end{equation}
which rearranges to
\begin{equation}
\label{eqn_DEOS2}
\frac{\mbox{D} \rho}{\mbox{D} t} = \frac{\overline{W}}{\mathcal{R}T} \frac{\mbox{D}\overline{p}_i}{\mbox{D} t} -
\rho \overline{W} \frac{\mbox{D}}{\mbox{D} t}\left(\frac{1}{\overline{W}}\right) - \frac{\rho}{T} \frac{\mbox{D} T}{\mbox{D} t}
\end{equation}


\subsubsection{Species Transport Equation}
\label{species_transport}

The species transport equation plays a role in both the second and third terms on the RHS of (\ref{eqn_DEOS2}).  Including the bulk mass source, the evolution of species mass fractions is governed by
\begin{equation}
\label{eqn_speciestransport}
\frac{\partial \left(\rho Y_\alpha\right)}{\partial t} + \Div \left(\rho Y_\alpha \mathbf{u}\right)  = - \Div \mathbf{J}_{\alpha} + \dot{m}_\alpha^{\tripleprime} + \dot{m}_{b,\alpha}^{\tripleprime}
\end{equation}
where $\mathbf{J}_{\alpha}$ is the diffusive mass flux vector for species $\alpha$ (relative to the mass-average velocity),
$\dot{m}_\alpha^{\tripleprime}$ is the chemical mass production rate of $\alpha$ per unit volume [kg-$\alpha$ produced /(sec m$^3$)],
and $\dot{m}_{b,\alpha}^{\tripleprime}$ is the bulk mass source of $\alpha$ per unit volume [kg-$\alpha$ introduced /(sec m$^3$)].  Note that
\begin{equation}
\label{eqn_bulksum}
\sum_\alpha \dot{m}_{b,\alpha}^{\tripleprime} = \dot{m}_b^{\tripleprime}
\end{equation}
and
\begin{equation}
\label{eqn_massconservation}
\sum_\alpha \dot{m}_\alpha^{\tripleprime} = 0
\end{equation}
Additionally, by construction, the $i$th component of the species diffusive fluxes sum to zero,
\begin{equation}
\label{eqn_sumdiffflux}
\sum_\alpha J_{\alpha,i} = 0
\end{equation}
Thus, as must be the case, summing (\ref{eqn_speciestransport}) over $\alpha$ yields the continuity equation.

It is convenient to work in terms of the material derivative of the mass fraction.  Care must be exercised in obtaining this expression because the continuity equation is of a non-standard form.
Expanding (\ref{eqn_speciestransport}) we obtain
\begin{eqnarray}
\label{eqn_speciesexpansion}
\rho\frac{\partial Y_\alpha}{\partial t} + Y_\alpha \frac{\partial \rho}{\partial t} + \rho \mathbf{u}\cdot \nabla Y_\alpha + Y_\alpha \Div \left(\rho \mathbf{u}\right) &=&
- \Div \mathbf{J}_{\alpha} + \dot{m}_\alpha^{\tripleprime} + \dot{m}_{b,\alpha}^{\tripleprime} \,\mbox{,} \nonumber\vspace{0.3cm}\\
\rho \frac{\mbox{D} Y_\alpha}{\mbox{D} t} + Y_\alpha \underbrace{\left[ \frac{\partial \rho}{\partial t} + \Div \left(\rho \mathbf{u}\right) \right]}_{\displaystyle \dot{m}_b^{\tripleprime}} &=& \mbox{}
\end{eqnarray}
Thus, the material derivative of the mass fraction can be written as
\begin{equation}
\label{eqn_matdermassfrac}
\frac{\mbox{D} Y_\alpha}{\mbox{D} t} = \frac{1}{\rho}\left(  \dot{m}_\alpha^{\tripleprime} + \dot{m}_{b,\alpha}^{\tripleprime} -
Y_\alpha \dot{m}_b^{\tripleprime}  - \Div \mathbf{J}_{\alpha} \right)
= \frac{1}{\rho}\left(  \dot{m}_\alpha^{\tripleprime} + \dot{m}_{b}^{\tripleprime}[Y_{b,\alpha} - Y_\alpha] - \Div \mathbf{J}_{\alpha} \right)
\end{equation}
where in the second step we use the identity $\dot{m}_{b,\alpha}^{\tripleprime} = Y_\alpha \dot{m}_{b}^{\tripleprime}$ with $Y_{b,\alpha}$ being the
mass fraction of $\alpha$ in the bulk prior to its introduction into the fluid mixture.

Utilizing (\ref{eqn_mixmolewt}) and (\ref{eqn_matdermassfrac}) we obtain
\begin{eqnarray}
\label{eqn_transportexpression}
\frac{\mbox{D}}{\mbox{D} t}\left(\frac{1}{\overline{W}}\right) &=& \frac{\mbox{D}}{\mbox{D} t}\left(\sum_\alpha \frac{Y_\alpha}{W_\alpha} \right) \,\mbox{,} \nonumber \vspace{0.3cm}\\
&=& \sum_\alpha \frac{1}{W_\alpha} \frac{\mbox{D}Y_\alpha }{\mbox{D} t} \,\mbox{,} \nonumber \vspace{0.3cm}\\
&=& \frac{1}{\rho} \sum_\alpha \frac{1}{W_\alpha} \left(  \dot{m}_\alpha^{\tripleprime} + \dot{m}_{b}^{\tripleprime}[Y_{b,\alpha} - Y_\alpha] - \Div \mathbf{J}_{\alpha} \right)  \,\mbox{,}
\end{eqnarray}
which is needed in the second term on the RHS of (\ref{eqn_DEOS2}).


\subsubsection{Enthalpy Transport Equation}
\label{enthalpy_definitions}

The specific sensible enthalpy of species $\alpha$ relative to reference temperature $T_0$ is
\begin{equation}
\label{eqn_sensible}
h_{\alpha}(T) = \int_{T_0}^{T} c_{p,\alpha}(T^\prime) \,\dif T^\prime \,\mbox{,}
\end{equation}
where the specific heat of $\alpha$ is
\begin{equation}
\label{eqn_specificheat}
c_{p,\alpha} \equiv \frac{\partial h_{\alpha}}{\partial T} \,\mbox{.}
\end{equation}
The specific sensible enthalpy of the mixture is then given by
\begin{equation}
\label{eqn_chemsensmix}
h_s(\mathbf{Y},T) = \sum_\alpha Y_\alpha h_{\alpha}(T) \,\mbox{.}
\end{equation}

Neglecting viscous heating and the effect of the fluctuating pressure on dilation work (both assumptions are valid for low-Mach flows), the transport equation for the sensible enthalpy is
\begin{equation}
\label{eqn_enthalpytransport}
\rho \frac{\mbox{D} h_s }{\mbox{D} t} = -\sum_\alpha \Delta h^0_\alpha \dot{m}_{\alpha}^{\tripleprime}
+ \frac{\mbox{D} \overline{p}_i }{\mbox{D} t} - \Div \dot{\mathbf{q}}^{\prime\prime} + \dot{m}_{b}^{\tripleprime} ( h_{s,b} - h_s )
\end{equation}
where $\Delta h^0_\alpha$ is the heat of formation of $\alpha$ at reference temperature $T_0$, $h_{s,b}$ is the specific sensible enthalpy of the bulk mass source,
and $\dot{\mathbf{q}}^{\prime\prime}$ is the heat flux vector which contains contributions from conduction, molecular diffusion, and radiation,
\begin{equation}
\label{eqn_heatflux}
\dot{\mathbf{q}}^{\prime\prime} = -k \nabla T + \sum_\alpha h_{s,\alpha} \mathbf{J}_{\alpha} + \dot{\mathbf{q}}^{\prime\prime}_{r}
\end{equation}
Here $k$ is the thermal conductivity of the mixture and $\dot{\mathbf{q}}^{\prime\prime}_{r}$ is the radiant heat flux.


\subsubsection{Relating Enthalpy, Temperature, and Species}
\label{eqn_enthalpy_temperature}

Using the chain rule of calculus, we may expand the derivative of the sensible enthalpy $h_s(\mathbf{Y},T)$ to obtain
\begin{equation}
\label{eqn_chainrule}
\frac{\mbox{D} h_s}{\mbox{D} t} = \left(\frac{\partial h_s}{\partial T}\right) \frac{\mbox{D} T }{\mbox{D} t} +
\sum_\alpha \left( \frac{\partial h_s}{\partial Y_\alpha} \right) \frac{\mbox{D} Y_\alpha }{\mbox{D} t}
\end{equation}
Note that since $h_s = \sum_\alpha Y_\alpha h_{\alpha}$ we have
\begin{equation}
\label{eqn_dhdY}
\frac{\partial h_s}{\partial Y_\alpha} = \frac{\partial}{\partial Y_\alpha} \sum_\beta (Y_\beta h_{\beta} )
= \sum_\beta h_{\beta} \,\delta_{\alpha \beta} = h_{\alpha}
\end{equation}
where $\delta_{\alpha \beta}$ is the Kronecker delta. Also,
\begin{equation}
\label{eqn_dhdT}
\frac{\partial h_s}{\partial T} = \frac{\partial}{\partial T} \sum_\alpha Y_\alpha h_{\alpha} =
\sum_\alpha Y_\alpha \left(\frac{\partial h_{\alpha}}{\partial T}\right) = \sum_\alpha Y_\alpha c_{p,\alpha} \equiv c_p
\end{equation}
defining the specific heat of the mixture.  Thus, by rearranging (\ref{eqn_chainrule}) and utilizing (\ref{eqn_dhdY}) and (\ref{eqn_dhdT}) we obtain
\begin{equation}
\label{eqn_DTDt1}
\frac{\mbox{D} T}{\mbox{D} t} = \frac{1}{c_p} \left[ \frac{\mbox{D} h_s}{\mbox{D} t} - \sum_\alpha h_{\alpha} \frac{\mbox{D} Y_\alpha}{\mbox{D} t} \right]
\end{equation}
Utilizing (\ref{eqn_matdermassfrac}) and (\ref{eqn_enthalpytransport}) in (\ref{eqn_DTDt1}) yields
\begin{eqnarray}
\frac{\mbox{D} T}{\mbox{D} t} &=& \frac{1}{\rho c_p}   \left[  -\sum_\alpha \Delta h^0_\alpha \dot{m}_{\alpha}^{\tripleprime} +
\frac{\mbox{D} \overline{p}_i }{\mbox{D} t} - \Div\dot{\mathbf{q}}^{\prime\prime} - \dq_b''' + \dot{m}_{b}^{\tripleprime} ( h_{s,b} - h_s )  \right.   \nonumber \vspace{0.3cm} \\
& &  \left.  - \sum_\alpha h_{s,\alpha} \Big\{  \dot{m}_\alpha^{\tripleprime} + \dot{m}_{b}^{\tripleprime}[Y_{b,\alpha} - Y_\alpha] - \Div \mathbf{J}_{\alpha} \Big\}  \right]
\label{eqn_DTDt2}
\end{eqnarray}
Note that $\sum_\alpha h_{s,\alpha} \dot{m}_{b}^{\tripleprime}[Y_{b,\alpha} - Y_\alpha] = \dot{m}_{b}^{\tripleprime} ( h_{s,b} - h_s )$ and hence these terms cancel, leaving
\begin{eqnarray}
\label{eqn_DTDt}
\frac{\mbox{D} T}{\mbox{D} t} &=& \frac{1}{\rho c_p} \left[ -\sum_\alpha \Delta h^0_\alpha \dot{m}_{\alpha}^{\tripleprime} + \frac{\mbox{D} \overline{p}_i }{\mbox{D} t}
- \Div\dot{\mathbf{q}}^{\prime\prime} - \dq_b'''  - \sum_\alpha h_{\alpha} \Big\{  \dot{m}_\alpha^{\tripleprime} - \Div \mathbf{J}_{\alpha} \Big\} \right]
\end{eqnarray}

\subsubsection{Assembling Terms}
\label{putting_it_all_together}

We now have all the pieces we need to construct the divergence constraint which we introduced in Eq. (\ref{eqn_divconstraint1}).  Using (\ref{eqn_DEOS2}) in (\ref{eqn_divconstraint1}) we obtain
\begin{eqnarray}
\label{eqn_divconstraint2}
\Div\mathbf{u} &=& \frac{1}{\rho} \left( \dot{m}_b^{\tripleprime} -  \left[ \frac{\overline{W}}{\mathcal{R}T} \frac{\mbox{D}\overline{p}_i}{\mbox{D} t} -
\rho \overline{W} \frac{\mbox{D}}{\mbox{D} t}\left(\frac{1}{\overline{W}}\right) - \frac{\rho}{T} \frac{\mbox{D} T}{\mbox{D} t} \right] \right)  \nonumber \vspace{0.3cm} \\
&=& \frac{1}{\rho} \,\dot{m}_b^{\tripleprime} -  \frac{1}{\overline{p}_i} \frac{\mbox{D}\overline{p}_i}{\mbox{D} t} + \overline{W} \frac{\mbox{D}}{\mbox{D} t}\left(\frac{1}{\overline{W}}\right) +
\frac{1}{T} \frac{\mbox{D} T}{\mbox{D} t}
\end{eqnarray}
where in the second step the EOS (\ref{eqn_idealgaslaw}) is used to simplify the second term on the RHS.
Using (\ref{eqn_transportexpression}) and (\ref{eqn_DTDt}) in (\ref{eqn_divconstraint2}) yields
\begin{eqnarray}
\label{eqn_divconstraint3}
\Div\mathbf{u} &=& \frac{1}{\rho} \,\dot{m}_b^{\tripleprime} -  \frac{1}{\overline{p}_i} \frac{\mbox{D}\overline{p}_i}{\mbox{D} t} \nonumber \vspace{0.3cm}\\
&+& \overline{W} \left[\frac{1}{\rho} \sum_\alpha \frac{1}{W_\alpha} \left\{  \dot{m}_\alpha^{\tripleprime}
+ \dot{m}_{b}^{\tripleprime}[Y_{b,\alpha} - Y_\alpha] - \Div \mathbf{J}_{\alpha} \right\} \right] \nonumber \vspace{0.3cm}\\
&+&  \frac{1}{T} \left[ \frac{1}{\rho c_p} \left\{-\sum_\alpha \Delta h^0_\alpha \dot{m}_{\alpha}^{\tripleprime} +
\frac{\mbox{D} \overline{p}_i }{\mbox{D} t} - \Div\dot{\mathbf{q}}^{\prime\prime} -\dq_b''' - \sum_\alpha h_{\alpha} \Big(  \dot{m}_\alpha^{\tripleprime} - \Div \mathbf{J}_{\alpha} \Big) \right\} \right]
\end{eqnarray}

Note that $\overline{W} \sum_\alpha (Y_\alpha/W_\alpha) = 1$ and also $\overline{W} \sum_\alpha (Y_{b,\alpha}/W_\alpha) = \overline{W}/\overline{W}_b$,
where $\overline{W}_b$ is the molecular weight of the bulk mixture prior to its introduction into the fluid mixture.  Equation (\ref{eqn_divconstraint3}) thus simplifies to
\begin{eqnarray}
\label{eqn_divconstraint_final}
\Div\mathbf{u}  &=& \left(\frac{1}{\rho c_p T}  -  \frac{1}{\overline{p}_i}\right) \frac{\mbox{D}\overline{p}_i}{\mbox{D} t} \nonumber \vspace{0.3cm}\\
&+& \frac{1}{\rho} \left[ \dot{m}_b^{\tripleprime} \frac{\overline{W}}{\overline{W}_b} +  \overline{W}\sum_\alpha \frac{1}{W_\alpha} \left\{  \dot{m}_\alpha^{\tripleprime}
- \Div\mathbf{J}_{\alpha} \right\} \right] \nonumber \vspace{0.3cm}\\
&+&  \frac{1}{\rho c_p T} \left[ -\sum_\alpha \Delta h^0_\alpha \dot{m}_{\alpha}^{\tripleprime} -
\Div \dot{\mathbf{q}}^{\prime\prime} -\dq_b''' - \sum_\alpha h_{\alpha} \left\{  \dot{m}_\alpha^{\tripleprime} - \Div \mathbf{J}_{\alpha} \right\} \right]
\end{eqnarray}


\subsubsection{FDS Notation}
\label{fds_notation}

The following relationships are used to rearrange (\ref{eqn_divconstraint_final}) into
the form shown in the FDS Technical Reference Guide.

We employ the binary form of Fick's law using mixture-averaged diffusivities $D_\alpha$ as a constitutive relation for the diffusive flux,
\begin{equation}
\label{eqn_fickslaw}
\mathbf{J}_{\alpha} = - \rho D_\alpha \nabla Y_\alpha
\end{equation}
Note that summation is not implied over repeated suffixes.
The heat release rate per unit volume is defined by
\begin{equation}
\label{eqn_heatrelease}
\dot{q}^\tripleprime \equiv -\sum_\alpha \Delta h^0_\alpha \dot{m}_{\alpha}^\tripleprime
\end{equation}
Taking the $z$ direction to be aligned with the gravity vector we have $\partial \overline{p}_i/\partial z = -\rho_i g$, where $g = 9.8$ m/s$^2$ and $\rho_i$ is a specified background density for the $i$th zone.
Thus, the material derivative of the background pressure may be written as
\begin{eqnarray}
\label{eqn_expandp0}
\frac{\mbox{D}\overline{p}_i}{\mbox{D}t} &=& \frac{\partial \overline{p}_i}{\partial t} - w \rho_i g \,\mbox{.}
\end{eqnarray}
Hence, utilizing (\ref{eqn_fickslaw}), (\ref{eqn_heatrelease}), and (\ref{eqn_expandp0}), and noting $1/(\rho T) = \mathcal{R}/(\overline{W} \overline{p}_i)$ from the EOS,
for the $i$th zone we may write the divergence (\ref{eqn_divconstraint_final}) as
\begin{equation}
\label{eqn_fds_divcontraint}
\Div \mathbf{u} = \mathcal{D} + \mathcal{P} \frac{\partial \overline{p}_i}{\partial t}
\end{equation}
where
\be
\label{eqn_fdsP}
\mathcal{P} = \frac{1}{\overline{p}_i}\left( \frac{\overline{p}_i}{\rho \,c_p T} - 1 \right) = \frac{1}{\overline{p}_i}\left( \frac{\mathcal{R}}{\overline{W} c_p} - 1 \right)
\ee
and
\begin{eqnarray}
\label{eqn_fdsD}
\mathcal{D} &=& \frac{\dot{m}_b^\tripleprime}{\rho} \frac{\overline{W}}{\overline{W}_b} + \frac{\mathcal{R}}{\overline{W} c_p \overline{p}_i} \left( \dot{q}^\tripleprime - \dq_b'''
- \Div \dot{\mathbf{q}}^{\prime\prime} \right) - \mathcal{P} w \rho_i g \nonumber \vspace{0.3cm} \\
&+& \frac{\overline{W}}{\rho} \sum_\alpha \Div \left( \rho D_\alpha \nabla[Y_\alpha/W_\alpha] \right) -
\frac{\mathcal{R}}{\overline{W}c_p \overline{p}_i} \sum_\alpha h_{\alpha} \Div\left( \rho D_\alpha \nabla Y_\alpha \right)
+ \frac{1}{\rho} \sum_\alpha \left( \frac{\overline{W}}{W_\alpha} - \frac{h_{\alpha}}{c_p T} \right) \dot{m}_{\alpha}^\tripleprime
\end{eqnarray}




\chapter{A Simple Model of Flame Extinction}

\subsubsection{Frederick W. Mowrer, Department of Fire Protection Engineering, University of Maryland}

\label{mowrer_model}


A diffusion flame immersed in a vitiated atmosphere will extinguish before consuming all the
available oxygen from the atmosphere.  The classic example of this behavior is a candle burning
within an inverted jar.  This same concept has been applied within FDS
to determine the conditions under which the local ambient oxygen concentration will no longer
support a diffusion flame.  In this appendix, the critical adiabatic flame temperature
concept is used to estimate the local ambient oxygen concentration at which extinction will
occur.

Consider a control volume characterized
by a bulk temperature, $T_m$, a mass, $m$, an average specific heat, $\overline{c_p}$, and an oxygen mass
fraction, $Y_\OTWO$.  Complete combustion of the oxygen within the control volume would release a
quantity of energy given by:
\be
   Q = m \, Y_\OTWO \, \left( \frac{\Delta H}{r_\OTWO}  \right)  \label{bbb}
\ee
where $\Delta H/r_\OTWO$ has a relatively constant value of
approximately 13100~kJ/kg for most fuels of interest for fire applications.\footnote{C. Huggett, ``Estimation of the Rate of Heat Release by Means of Oxygen
Consumption,'' {\em Fire and Materials}, Vol.~12, pp.~61-65, 1980.}
Under adiabatic conditions, the energy released by combustion of the available oxygen within
the control volume would raise the bulk temperature of the gases within the control volume by an
amount equal to:
\be
   Q = m \, \overline{c_p} \, (T_f - T_m)  \label{eee}
\ee
The average specific heat of the gases within the control volume can be calculated based on the
composition of the combustion products as:
\be
   \overline{c_p} = \frac{1}{(T_f-T_m)} \, \sum_\alpha \int_{T_m}^{T_f} c_{p,\alpha} (T) \, dT
\ee
To simplify the analysis, the combustion products are assumed to have an average specific heat
of 1.1~kJ/kg/K over the temperature range of interest, a value similar to that of air, the primary
component of the products.
The relationship between the oxygen mass fraction within the control volume and the adiabatic
temperature rise of the control volume is evaluated by equating Eqs.~(\ref{bbb}) and (\ref{eee}):
\be
   Y_\OTWO = \frac{ \overline{c_p} (T_f-T_m) }{\Delta H/r_\OTWO}
\ee
If the critical adiabatic flame temperature is assumed to have a constant value of approximately
1700~K for hydrocarbon diffusion flames, as suggested by Beyler,\footnote{C. Beyler, ``Flammability Limits of Premixed and Diffusion Flames,''
{\em SFPE Handbook of Fire Protection Engineering} (3rd Ed.), National Fire
Protection Association, Quincy, MA, 2003.} then the relationship
between the limiting oxygen mass fraction and the bulk temperature of a control volume is given
by:
\be
   Y_{\OTWO,lim} = \frac{ \overline{c_p} (T_{f,lim}-T_m) }{\Delta H/r_\OTWO} \approx  \frac{ 1.1 \, (1700-T_m) }{ 13100}  \label{extinction_model}
\ee
The relationship represented by Eq.~(\ref{extinction_model}) is shown in Fig.~\ref{plotsupp}.
For a control volume at a temperature of 300~K, i.e., near room temperature, the limiting oxygen
mass fraction would evaluate to $Y_{\OTWO,lim}=0.118$.  This value is consistent with the measurements
of Morehart, Zukoski and Kubota,\footnote{Morehart, J., Zukoski, E., and Kubota, T., ``Characteristics of Large Diffusion Flames
Burning in a Vitiated Atmosphere,'' {\em Third International Symposium on Fire Safety
Science}, Elsevier Science Publishers, pp.~575-583, 1991.} although it is about 10 percent lower than their measured
values and the limiting oxygen index values determined by Simmons and Wolfhard.\footnote{R.F.~Simmons and H.G.~Wolfhard, {\em Combustion and Flame}, Vol.~1, p.~155, 1957.}
This difference would tend to be conservative in terms of indicating flammability of gas mixtures at
oxygen concentrations lower than might occur in practice.
Comparisons with additional
experimental data are needed to further validate the concept.  Some improvement might result
from a more exact analysis of the average specific heat of the combustion products or from the
use of a slightly higher critical adiabatic flame temperature in the range of 1700~K to 1800~K,
as suggested by the limiting oxygen index data of Simmons and Wolfhard that is
summarized in the SFPE Handbook Chapter of Beyler.

















\end{document}
