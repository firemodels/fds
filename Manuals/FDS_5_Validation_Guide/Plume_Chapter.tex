\chapter{Fire Plumes}

Plume temperature measurements are available from the VTT Large Hall and the FM/SNL series.
For all the other series of experiments, the temperature above the fire is not reported, or the fire plume
leans because of the flow pattern within the compartment, or the fire is positioned against a wall.
Only for the VTT and the FM/SNL series are the plumes relatively free from perturbations.

\section{VTT Large Hall Test Series}

These experiments consist of liquid fuel pan fires conducted in the middle of a large fire test hall.
Plume temperatures are measured at two heights above the fire, 6 m and 12 m.
The flames extend to about 4 m above the fire pan.

Photographs from the VTT tests are available. It is difficult to precisely measure the flame height,
but the photos and videos allow one to make estimates accurate to within a pan diameter.
Similarly, flame height in FDS is assessed using the visualization program Smokeview.
There are various ways to render the fire in Smokeview.  The
most direct method is to show, via three dimensional surface plots, the volume within which the energy from the fire is being released.
The other method is to show the stoichiometric iso-surface of the mixture fraction.
FDS tracks the fuel and oxygen via a single scalar variable called the mixture fraction.
The stoichiometric iso-surface is essentially a sheet on which combustion occurs.  The average vertical extent of either the volume
in which energy is being released or the stoichiometric mixture fraction iso-surface is the FDS predicted flame height.
Shown in Figure~\ref{FDS_Flame} are snapshots from the simulation of the 1.6 m diameter heptane pan fire.
The pan has been approximated as a square because of the requirement by FDS of rectangular geometry.
Figure~\ref{Simo_Photos} contains photographs of the actual fire.
The height of the visible flame in the photographs has been estimated to be between 2.4 and 3 pan diameters (3.8 m to 4.8 m).
The height of the simulated fire fluctuates from 5 m to 6 m during the peak heat release rate phase.


\section{FM/SNL Test Series}

In Tests 4 and 5, thermocouples are positioned near the ceiling directly over the fire pan.
In Test 21, the fire is located within an empty electrical cabinet, and the closest near ceiling thermocouple
is used to assess the plume temperature.  Note that in Test 5, the FDS plume temperature curve has been smoothed
to better assess the relative difference between the peak values of the model and the measurement.


\begin{figure}[p]
\begin{tabular*}{\textwidth}{l@{\extracolsep{\fill}}r}
\includegraphics[width=2.6in]{FIGURES/VTT_01_v5_Plume_Temperature} &
\includegraphics[width=2.6in]{FIGURES/FM_SNL_04_v5_Plume_Temperature} \\
\includegraphics[width=2.6in]{FIGURES/VTT_02_v5_Plume_Temperature} &
\includegraphics[width=2.6in]{FIGURES/FM_SNL_05_v5_Plume_Temperature} \\
\includegraphics[width=2.6in]{FIGURES/VTT_03_v5_Plume_Temperature} &
\includegraphics[width=2.6in]{FIGURES/FM_SNL_21_v5_Plume_Temperature}
\end{tabular*}
\caption{Plume Temperatures from the VTT Large Hall (Left) and FM/SNL Tests (Right).}
\label{VTT_FM_SNL_Plume}
\end{figure}

\clearpage
