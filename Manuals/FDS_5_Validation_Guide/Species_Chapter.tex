\chapter{Gas Species and Smoke}

FDS uses a mixture fraction combustion model, meaning that all gas species within the compartment are
assumed to be functions of a single scalar variable.  FDS solves only one transport equation for this variable,
and reports gas concentrations at any given point at any given time by extracting
its value from a pre-computed ``look-up'' table.  For the major species, like carbon dioxide and oxygen,
the predictions are essentially an indicator of how well FDS is predicting the bulk transport of combustion products throughout the space.
For minor species, like carbon monoxide and soot, FDS version 4 does not account for changes in combustion efficiency,
relying only on fixed yields of CO and soot from the combustion process.
In reality, the generation rate of CO and soot change depending on the ventilation conditions in the compartment.


\section{NIST/WTC Test Series, Oxygen and CO$_2$}


\begin{figure}[p]
\begin{tabular*}{\textwidth}{l@{\extracolsep{\fill}}r}
\includegraphics[width=2.6in]{FIGURES/WTC_01_v5_Gas_Concentration} &
\includegraphics[width=2.6in]{FIGURES/WTC_02_v5_Gas_Concentration} \\
\includegraphics[width=2.6in]{FIGURES/WTC_03_v5_Gas_Concentration} &
\includegraphics[width=2.6in]{FIGURES/WTC_04_v5_Gas_Concentration} \\
\includegraphics[width=2.6in]{FIGURES/WTC_05_v5_Gas_Concentration} &
\includegraphics[width=2.6in]{FIGURES/WTC_06_v5_Gas_Concentration}
\end{tabular*}
\caption{Oxygen and CO$_2$ for the NIST/WTC Test Series.}
\label{NIST_WTC_Gas}
\end{figure}

\clearpage

\section{NIST/NRC Test Series, Oxygen and CO$_2$}

The following pages present comparisons of oxygen and carbon dioxide concentration predictions with measurement for the
NIST/NRC series. There were two oxygen measurements, one in the upper layer, one in the lower.  There was only one carbon
dioxide measurement in the upper layer.

\begin{figure}[p]
\begin{tabular*}{\textwidth}{l@{\extracolsep{\fill}}r}
\includegraphics[width=2.6in]{FIGURES/NIST_NRC_01_v5_Oxygen_Concentration} &
\includegraphics[width=2.6in]{FIGURES/NIST_NRC_07_v5_Oxygen_Concentration} \\
\includegraphics[width=2.6in]{FIGURES/NIST_NRC_02_v5_Oxygen_Concentration} &
\includegraphics[width=2.6in]{FIGURES/NIST_NRC_08_v5_Oxygen_Concentration} \\
\includegraphics[width=2.6in]{FIGURES/NIST_NRC_04_v5_Oxygen_Concentration} &
\includegraphics[width=2.6in]{FIGURES/NIST_NRC_10_v5_Oxygen_Concentration} \\
\includegraphics[width=2.6in]{FIGURES/NIST_NRC_13_v5_Oxygen_Concentration} &
\includegraphics[width=2.6in]{FIGURES/NIST_NRC_16_v5_Oxygen_Concentration}
\end{tabular*}
\caption{Oxygen and CO$_2$ for the NIST/NRC Series, closed door tests.}
\label{NIST_NRC_Gas_Closed}
\end{figure}

\begin{figure}[p]
\begin{tabular*}{\textwidth}{l@{\extracolsep{\fill}}r}
\includegraphics[width=2.6in]{FIGURES/NIST_NRC_17_v5_Oxygen_Concentration} &
 \\
\includegraphics[width=2.6in]{FIGURES/NIST_NRC_03_v5_Oxygen_Concentration} &
\includegraphics[width=2.6in]{FIGURES/NIST_NRC_09_v5_Oxygen_Concentration} \\
\includegraphics[width=2.6in]{FIGURES/NIST_NRC_05_v5_Oxygen_Concentration} &
\includegraphics[width=2.6in]{FIGURES/NIST_NRC_14_v5_Oxygen_Concentration} \\
\includegraphics[width=2.6in]{FIGURES/NIST_NRC_15_v5_Oxygen_Concentration} &
\includegraphics[width=2.6in]{FIGURES/NIST_NRC_18_v5_Oxygen_Concentration}
\end{tabular*}
\caption{Oxygen and CO$_2$ for the NIST/NRC Series, open door tests.}
\label{NIST_NRC_Gas_Open}
\end{figure}

\clearpage


\section{NIST/NRC Test Series, Smoke}

FDS treats smoke like all other combustion products, basically a tracer gas whose mass fraction is a function of the mixture fraction.
To model smoke movement, the user need only prescribe the smoke yield, that is, the fraction of the fuel mass that is
converted to smoke particulate.  For the simulations of the NIST/NRC tests, the smoke yield is specified as one of the test parameters.
Figure and Figure contain comparisons of measured and predicted smoke concentration at one measuring station in the upper layer.
There are two obvious trends in the figures: first, the predicted concentrations are about 50 \% higher than the measured
in the open door tests.  Second,
the predicted concentrations are roughly three times the measured concentrations in the closed door tests.
As a contrast, Figure displays the time history of CO concentration for 6 of the NIST/NRC tests.
Like smoke, the CO is specified in FDS via a fixed yield, measured along with smoke and reported in the test document.
The large differences between model and measurement seen in the smoke data do not appear in the CO data.

\begin{figure}[p]
\begin{tabular*}{\textwidth}{l@{\extracolsep{\fill}}r}
\includegraphics[width=2.6in]{FIGURES/NIST_NRC_01_v5_Smoke_Concentration} &
\includegraphics[width=2.6in]{FIGURES/NIST_NRC_07_v5_Smoke_Concentration} \\
\includegraphics[width=2.6in]{FIGURES/NIST_NRC_02_v5_Smoke_Concentration} &
\includegraphics[width=2.6in]{FIGURES/NIST_NRC_08_v5_Smoke_Concentration} \\
\includegraphics[width=2.6in]{FIGURES/NIST_NRC_04_v5_Smoke_Concentration} &
\includegraphics[width=2.6in]{FIGURES/NIST_NRC_10_v5_Smoke_Concentration} \\
\includegraphics[width=2.6in]{FIGURES/NIST_NRC_13_v5_Smoke_Concentration} &
\includegraphics[width=2.6in]{FIGURES/NIST_NRC_16_v5_Smoke_Concentration}
\end{tabular*}
\caption{Smoke concentration for the NIST/NRC Series, closed door tests.}
\label{NIST_NRC_Smoke_Closed}
\end{figure}

\begin{figure}[p]
\begin{tabular*}{\textwidth}{l@{\extracolsep{\fill}}r}
\includegraphics[width=2.6in]{FIGURES/NIST_NRC_17_v5_Smoke_Concentration} &
 \\
\includegraphics[width=2.6in]{FIGURES/NIST_NRC_03_v5_Smoke_Concentration} &
\includegraphics[width=2.6in]{FIGURES/NIST_NRC_09_v5_Smoke_Concentration} \\
\includegraphics[width=2.6in]{FIGURES/NIST_NRC_05_v5_Smoke_Concentration} &
\includegraphics[width=2.6in]{FIGURES/NIST_NRC_14_v5_Smoke_Concentration} \\
\includegraphics[width=2.6in]{FIGURES/NIST_NRC_15_v5_Smoke_Concentration} &
\includegraphics[width=2.6in]{FIGURES/NIST_NRC_18_v5_Smoke_Concentration}
\end{tabular*}
\caption{Smoke concentration for the NIST/NRC Series, open door tests.}
\label{NIST_NRC_Smoke_Open}
\end{figure}

\clearpage

\section{NIST Diffusion Flame Test Series}

The CO production model was used to simulate the methane/air slot burner flame.  The Figure shows predicted and 
measured values of CH4 + O2, CO,  CO2 at three elevations above the burner along with a contour plot of CO near the 
burner lip (black rectangle is burner).  Note that the test data shows that a small quantity of oxygen exists along 
the burner centerline which is not echoed in the predictions.  This behavior is expected. The new combustion model considers 
the first step, Fuel to CO, to be infinitely fast, assuming that the local oxygen concentration satisfies the 
flammability criteria.  This is true in the vicinity of the lip of the burner.  In reality, the cold fuel and air 
streams do not react infinitely fast there and some oxygen penetrates the flame at the base, resulting in cooler 
gases being entrained into the core of the flame.  As a result, along the centerline, the model predicts higher values of fuel and higher values of 
products than measured.  The species profiles are also slightly narrower than measured, echoing the temperature plot \
in Fig.~\ref{WP_CH4_temp}.  The peak CH4 values along the burner centerline have errors ranging from 6 \% to 13 \%, 
the peak CO2 values in the flame have errors ranging from 3 \% to 10 \%, and the peak CO values in the flame have 
errors ranging from 9 \% to 25 \%.  Given the reported 10 \% to 20 \% species measurement uncertainty, the new 
combustion model predicts well both the magnitude and shape of the mixture fraction profiles.

The contour of CO illustrates the combustion model quite well.  At the burner lip, fuel and oxygen first meet.  
Since ambient levels of oxygen exist on the air side of the slot and the temperatures are relatively low at the 
burner lip, combustion occurs and converts $Z_1$ to $Z_2$.  This consumes the oxygen present and the CO does not have an 
opportunity to fully convert to CO2.  As the CO rises in the flame, oxygen is entrained and the temperatures are high 
enough that the CO reacts with the entrained oxygen to form CO2.  Eventually, at a point high enough in the flame, 
all of the fuel is consumed and all of the CO is oxidized.

\begin{figure}[p]
\includegraphics[width=6in]{FIGURES/WP_CH4_species}
\caption{Predicted and measured mole fractions at three elevations (7 mm, 9mm and 11 mm) above a methane-air slot burner, 
along with contours of CO near the burner lip}
\label{WP_CH4_Species}
\end{figure}
