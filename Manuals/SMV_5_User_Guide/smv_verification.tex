\newcommand{\figheightD}{2.75in}
\newcommand{\figheightE}{2.5in}
\newcommand{\figheightF}{2.25in}
\newcommand{\figheightG}{1.9in}
\newcolumntype{M}[1]{>{\centering\arraybackslash}m{#1}}

\chapter{Verifying Smokeview}
This chapter presents FDS results in the form of Smokeview images used
to verify that Smokeview is working as expected.  The version of Smokeview being
tested is
{
%\scriptsize
\verbatiminput{scriptfigures/smokeview.version}
}

FDS generated data is presumed
to be correct.  FDS has its own set of verification cases to test the correctness of the data.
The purpose of the cases here is to confirm that data is drawn correctly.
In particular these cases confirm
that correct files are loaded, data is scaled and drawn correctly, geometry is drawn correctly
{\em etc.}   Three types of verification cases are presented. The first set are the most important, those cases that verify that data is drawn correctly.  The second set of
cases verify that various geometric elements are drawn correctly and the third set verifies
that the various options and underlying features are implemented and perform properly.



\section{Data Presentation}
Figure \ref{figboundtest} shows a series of boundary file images drawn at 5.0, 10.0 and 30.0
seconds.
The first column of images colors data according to temperature using temperature while the second column of images
highlights regions where ignition has occurred defined by wherever the surface temperature exceeds a specified value.
Figure \ref{figisotest} shows a series of temperature isosurface file drawn images at 5.0, 10.0 and 30.0
seconds.  The three columns of images present the isosurfaces drawn as a solid, outline and points.
Figure \ref{figparttest} shows a series of particle file images drawn at 5.0, 10.0 and 30.0
seconds.
The first column shows particles while the second and third columns shows streaks of length 0.5 s and 1.0 s.
Figure \ref{figslicetest} shows a series of slice file images drawn at 5.0, 10.0 and 30.0
seconds.
The first column draws all of the data.  The second data discards or chops data below 140~\degC.
Figure \ref{figvslicetest} shows a series of vector slice file images drawn at 5.0, 10.0 and 30.0
seconds.
Similar to the slice file images, the first column draws all of the data while the second column discards or chops data below 140~\degC.
Figure \ref{figsmoketest} shows a series of 3D smoke images drawn at 5.0, 10.0 and 30.0
seconds.  The images contain both opacity slices derived from soot densities and heat release rate  per unit volume (HRRPUV).

\begin{figure}[\figoptions]
\begin{center}
\begin{tabular}{cccl}
 \includegraphics[width=\figheightG]{scriptfigures/colorconv_slice_00000}&
 \includegraphics[width=\figheightG]{scriptfigures/colorconv_slice_00025}&
 \includegraphics[width=\figheightG]{scriptfigures/colorconv_slice_00050}\\
 \includegraphics[width=\figheightG]{scriptfigures/colorconv_slice_00075}&
 \includegraphics[width=\figheightG]{scriptfigures/colorconv_slice_00100}&
 \includegraphics[width=\figheightG]{scriptfigures/colorconv_slice_00125}\\
 \includegraphics[width=\figheightG]{scriptfigures/colorconv_slice_00150}&
 \includegraphics[width=\figheightG]{scriptfigures/colorconv_slice_00175}&
 \includegraphics[width=\figheightG]{scriptfigures/colorconv_slice_00200}\\
 \includegraphics[width=\figheightG]{scriptfigures/colorconv_slice_00225}&
 \includegraphics[width=\figheightG]{scriptfigures/colorconv_slice_00250}&
 \includegraphics[width=\figheightG]{scriptfigures/colorconv_slice_00275}\\
 \includegraphics[width=\figheightG]{scriptfigures/colorconv_slice_00300}&
 \includegraphics[width=\figheightG]{scriptfigures/colorconv_slice_00325}&
 \includegraphics[width=\figheightG]{scriptfigures/colorconv_slice_10000}\\
&&&\raisebox{0.0in}[0pt]{\includegraphics[height=8.0in]{figures/colorbar_20_100}}\\
\end{tabular}
\end{center}
 \caption[Color conversion test.]{Color conversion test.}
\label{figboundtest}%
\end{figure}


\begin{figure}[\figoptions]
\begin{center}
\begin{tabular}{rccl}
 1.0 s
 & \includegraphics[height=\figheightD]{scriptfigures/plume5c_bound_01}
 & \includegraphics[height=\figheightD]{scriptfigures/plume5c_bound_cell_01}\\
 10.0 s&
 \includegraphics[height=\figheightD]{scriptfigures/plume5c_bound_10}&
 \includegraphics[height=\figheightD]{scriptfigures/plume5c_bound_cell_10}\\
 30.0 s&
 \includegraphics[height=\figheightD]{scriptfigures/plume5c_bound_30}&
 \includegraphics[height=\figheightD]{scriptfigures/plume5c_bound_cell_30}\\
&\\
&cell averaged  data&cell centered data\\
 &&&\raisebox{1.0in}[0pt]{\includegraphics[height=7.0in]{figures/colorbar_20_620}}\\
  \end{tabular}
\end{center}
 \caption[Boundary file test.]{Boundary file test.}
\label{figboundtest}%
\end{figure}


\begin{figure}[\figoptions]
\begin{center}
\begin{tabular}{rcc}
 1.0 s&
 \includegraphics[height=\figheightD]{scriptfigures/plume5c_iso_solid_01}&
 \includegraphics[height=\figheightD]{scriptfigures/plume5c_iso_solid_normal_01}\\
 10.0 s&
 \includegraphics[height=\figheightD]{scriptfigures/plume5c_iso_solid_10}&
 \includegraphics[height=\figheightD]{scriptfigures/plume5c_iso_solid_normal_10}\\
 30.0 s&
 \includegraphics[height=\figheightD]{scriptfigures/plume5c_iso_solid_30}&
 \includegraphics[height=\figheightD]{scriptfigures/plume5c_iso_solid_normal_30}\\
 &solid&solid with normals
  \end{tabular}
\end{center}
 \caption{Isosurface file test 1.}
\label{figisotest}%
\end{figure}

\begin{figure}[\figoptions]
\begin{center}
\begin{tabular}{rcc}
 1.0 s&
 \includegraphics[height=\figheightD]{scriptfigures/plume5c_iso_outline_01}&
 \includegraphics[height=\figheightD]{scriptfigures/plume5c_iso_points_01}\\
 10.0 s&
 \includegraphics[height=\figheightD]{scriptfigures/plume5c_iso_outline_10}&
 \includegraphics[height=\figheightD]{scriptfigures/plume5c_iso_points_10}\\
 30.0 s&
 \includegraphics[height=\figheightD]{scriptfigures/plume5c_iso_outline_30}&
 \includegraphics[height=\figheightD]{scriptfigures/plume5c_iso_points_30}\\
 &outline&points
  \end{tabular}
\end{center}
 \caption{Isosurface file test 2.}
\label{figisotest}%
\end{figure}

\begin{figure}[\figoptions]
\begin{center}
\begin{tabular}{rccc}
 1.0 s&
 \includegraphics[height=\figheightD]{scriptfigures/plume5c_part_01}&
 \includegraphics[height=\figheightD]{scriptfigures/plume5c_part_streak_01}&
 \includegraphics[height=\figheightD]{scriptfigures/plume5c_part_streak2_01}\\
 10.0 s&
 \includegraphics[height=\figheightD]{scriptfigures/plume5c_part_10}&
 \includegraphics[height=\figheightD]{scriptfigures/plume5c_part_streak_10}&
 \includegraphics[height=\figheightD]{scriptfigures/plume5c_part_streak2_10}\\
  30.0 s&
 \includegraphics[height=\figheightD]{scriptfigures/plume5c_part_30}&
 \includegraphics[height=\figheightD]{scriptfigures/plume5c_part_streak_30}&
 \includegraphics[height=\figheightD]{scriptfigures/plume5c_part_streak2_30}\\
 &points&0.5 s streaks&1.0 s streaks\\
  \end{tabular}
\end{center}
 \caption{Particle file test.}
\label{figparttest}%
\end{figure}

\begin{figure}[\figoptions]
\begin{center}
\begin{tabular}{cccl}
 \includegraphics[height=\figheightD]{scriptfigures/plume5c_slice_01}&
 \includegraphics[height=\figheightD]{scriptfigures/plume5c_slice_chop_01}&
 \includegraphics[height=\figheightD]{scriptfigures/plume5c_slice_cell_01}\\

 \includegraphics[height=\figheightD]{scriptfigures/plume5c_slice_10}&
 \includegraphics[height=\figheightD]{scriptfigures/plume5c_slice_chop_10}&
 \includegraphics[height=\figheightD]{scriptfigures/plume5c_slice_cell_10}\\

 \includegraphics[height=\figheightD]{scriptfigures/plume5c_slice_30}&
 \includegraphics[height=\figheightD]{scriptfigures/plume5c_slice_chop_30}&
 \includegraphics[height=\figheightD]{scriptfigures/plume5c_slice_cell_30}\\

 &data chopped below 140~\degC&cell centered data\\
 &&&\raisebox{1.0in}[0pt]{\includegraphics[height=7.0in]{figures/colorbar_20_620}}\\
 \end{tabular}
\end{center}
 \caption{Slice file test.}
\label{figslicetest}%
\end{figure}

\begin{figure}[\figoptions]
\begin{center}
\begin{tabular}{rccl}
 1.0 s&
 \includegraphics[height=\figheightD]{scriptfigures/plume5c_vslice_01}&
 \includegraphics[height=\figheightD]{scriptfigures/plume5c_vslicechop_01}\\
 10.0 s&
 \includegraphics[height=\figheightD]{scriptfigures/plume5c_vslice_10}&
 \includegraphics[height=\figheightD]{scriptfigures/plume5c_vslicechop_10}\\
 30.0 s&
 \includegraphics[height=\figheightD]{scriptfigures/plume5c_vslice_30}&
 \includegraphics[height=\figheightD]{scriptfigures/plume5c_vslicechop_30}\\
 &&data chopped below 140~\degC\\
 &&&\raisebox{1.0in}[0pt]{\includegraphics[height=7.0in]{figures/colorbar_20_620}}\\

 \end{tabular}
\end{center}
 \caption{Vector slice file test.}
\label{figvslicetest}%
\end{figure}

\begin{figure}[\figoptions]
\begin{center}
\begin{tabular}{rc}
 1.0 s&
 \includegraphics[height=\figheightD]{scriptfigures/plume5c_smoke_01}\\
 10.0 s&
 \includegraphics[height=\figheightD]{scriptfigures/plume5c_smoke_10}\\
 30.0 s&
 \includegraphics[height=\figheightD]{scriptfigures/plume5c_smoke_30}\\

 \end{tabular}
\end{center}
 \caption{3D smoke file test.}
\label{figsmoketest}%
\end{figure}

\begin{figure}[\figoptions]
\begin{center}
 \centering
\begin{tabular}{c}
\includegraphics[height=\figheightF]{scriptfigures/smoke_sensor_l}\\
\includegraphics[height=\figheightF]{scriptfigures/smoke_sensor_c}\\
\includegraphics[height=\figheightF]{scriptfigures/smoke_sensor_r}\\

 \end{tabular}
\end{center}
\caption[Smoke sensor test.]{Smoke sensor test.
A small white (255,255,255) smokesensor appears in front of a grey (128,128,128) obstacle.
The red dot indicates where the smoke opacity is recorded.
}
\label{figsmokesensor}%
\end{figure}

\begin{figure}[\figoptions]
\begin{center}
 \centering
\begin{tabular}{c}
\includegraphics[width=6.0in]{figures/graysquares2}\\
\includegraphics[width=6.0in]{figures/graysquares3}\\
 \end{tabular}
\end{center}
 \caption[Shade of gray resolution test.]{Shade of gray resolution test.
 The number within each square represents the shade of gray used to color that square,
 0 for black and 255 for white.  Adjacent squares are drawn with nearly equal shades
 testing the ability of sensors such as the eye, computer monitor or the printed page
 to distinguish them.
 }
\label{figgraysquare}%
\end{figure}

\begin{figure}[\figoptions]
\begin{center}
 \centering
\begin{tabular}{m{1in}m{3in}m{3in}}
 &0.1 m grid&0.2 m grid\\
 all planes&\includegraphics[height=\figheightF]{scriptfigures/smoke_test_all}&
 \includegraphics[height=\figheightF]{scriptfigures/smoke_test2_all}\\
 every 2nd plane&\includegraphics[height=\figheightF]{scriptfigures/smoke_test_every2}&
 \includegraphics[height=\figheightF]{scriptfigures/smoke_test2_every2}\\
 every 3rd plane&\includegraphics[height=\figheightF]{scriptfigures/smoke_test_every3}&
  \includegraphics[height=\figheightF]{scriptfigures/smoke_test2_every3}\\
 theoretical&\multicolumn{2}{c}{\includegraphics[height=1.0in]{figures/graysquares}}\\

 \end{tabular}
\end{center}
 \caption[3D smoke file test 2.]{3D smoke file test 2.
 A quantitative test of the smoke opacity calculation in Smokeview.  This test simplifies
  the general case by assuming a uniform distribution of smoke.  Smoke grey levels are computed
  using $grey level (GL) = 255*exp(-K*S*\Delta X)$
  where $K=8700$~m2/kg is the mass extinction value, $S=79.67$~mg/m3 is the soot density
  and $\Delta x$ is the smoke path length.  Path lengths (smoke sensor locations) are chosen to obtain grey levels of 192, 128, 64, 32, 16 and 8.
 }
\label{figsmoketest2}%
\end{figure}


\begin{figure}[\figoptions]
\begin{center}
\begin{tabular}{cc}
 \includegraphics[height=\figheightD]{scriptfigures/thouse5_smoke_005}&
 \includegraphics[height=\figheightD]{scriptfigures/thouse5_smoke_gpu_005}\\
 \includegraphics[height=\figheightD]{scriptfigures/thouse5_smoke_010}&
 \includegraphics[height=\figheightD]{scriptfigures/thouse5_smoke_gpu_010}\\
 \includegraphics[height=\figheightD]{scriptfigures/thouse5_smoke_030}&
 \includegraphics[height=\figheightD]{scriptfigures/thouse5_smoke_gpu_030}\\
 CPU corrected&GPU corrected\\
 \end{tabular}
\end{center}
 \caption[3D smoke file test 3 - GPU Test]{3D smoke file test 3 - GPU Test}
\label{figsmoketest3}%
\end{figure}

\begin{figure}[\figoptions]
\begin{center}
\begin{tabular}{c}
 \includegraphics[height=\figheightE]{scriptfigures/thouse5_plot3d_step}\\
 step contours\\
 \includegraphics[height=\figheightE]{scriptfigures/thouse5_plot3d_line}\\
 line contours\\
 \includegraphics[height=\figheightE]{scriptfigures/thouse5_plot3d_cont}\\
 continuous contours
 \end{tabular}
\end{center}
 \caption{PLOT3D file test}
\label{figPLOT3Dtest}%
\end{figure}
\section{Geometry}
\begin{figure}[\figoptions]
\begin{center}
\begin{tabular}{c}
 \includegraphics[height=\figheightE]{scriptfigures/thouse5_solid}\\
 solid\\
 \includegraphics[height=\figheightE]{scriptfigures/thouse5_outline}\\
 outline\\
 \includegraphics[height=\figheightE]{scriptfigures/thouse5_hidden}\\
 hidden\\

 \end{tabular}
\end{center}
 \caption{Obstacle view test.}
\label{figobstest}%
\end{figure}

\begin{figure}[\figoptions]
\begin{center}
\begin{tabular}{c}
 \includegraphics[height=\figheightE]{scriptfigures/thouse5_solid}\\
 all vents\\
 \includegraphics[height=\figheightE]{scriptfigures/thouse5_noopen}\\
 no open vents\\
 \includegraphics[height=\figheightE]{scriptfigures/thouse5_novents}\\
 no vents\\

 \end{tabular}
\end{center}
 \caption{Vent view test.}
\label{figventest}%
\end{figure}


\section{Other}
