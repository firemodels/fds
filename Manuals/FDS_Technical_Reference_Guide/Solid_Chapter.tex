\chapter{Solid Phase} \label{SolidPhase}
\label{chapter:solid_phase}

FDS assumes that solid obstructions consist of multiple layers, with each
layer composed of multiple material components that can undergo multiple thermal degradation reactions.
Each reaction forms a combination of solid residue ({\em i.e.} another material component), water vapor, and/or fuel vapor. Heat
conduction is assumed only in the direction normal to the surface. This section describes the single
mass and energy conservation equation for solid materials, plus the
various coefficients, source terms, and boundary conditions, including the computation of the
convective heat flux $\dq_c''$ at solid boundaries.



\section{The Heat Conduction Equation for a Solid}

A one-dimensional heat conduction equation for the solid phase
temperature $T_s(x,t)$ is applied in the direction $x$ pointing into
the solid (the point $x = 0$ represents the surface)\footnote{In cylindrical and spherical coordinates, the heat conduction
equation is written
\be
  \rho_s c_s \; \dod{T_s}{t} = \frac{1}{r} \, \dod{}{r}
  \left(rk_s \dod{T_s}{r} \right)+\dq_s'''
  \quad ; \quad
  \rho_s c_s \; \dod{T_s}{t} = \frac{1}{r^2} \, \dod{}{r}
  \left(r^2k_s \dod{T_s}{r} \right)+\dq_s'''
  \label{1dheatcyl}
\ee
FDS offers the user these options, with the assumption that the
obstruction is not actually recti-linear, but rather cylindrical or
spherical in shape. This option is useful in describing the behavior
of small, complicated ``targets'' like cables or heat detection
devices.}
\be
  \rho_s c_s \; \dod{T_s}{t} = \dod{}{x} k_s \dod{T_s}{x} +
    \dq_s'''
  \label{1dheat}
\ee
Section~\ref{matcoefs} describes the component-averaged material
properties, $k_s$ and $\rho_s c_s$. The source term, $\dq_s'''$,
consists of chemical reactions and radiative absorption:
\be
  \dq_s'''=\dq_{s,c}'''+\dq_{s,r}'''
\ee
Section~\ref{pyrosection} describes the term $\dq_{s,c}'''$, which
is essentially the heat production (loss) rate given by the  pyrolysis
models for different types of solid and liquid fuels.
Section~\ref{inradsection} describes the term
$\dq_{s,r}'''$, the radiative absorption and emission in depth.
Section~\ref{conflux} describes the convective heat transfer to the
solid surface.

\subsection{Numerical Model}

A one dimensional heat transfer calculation is performed at each solid
boundary cell for which the user has prescribed thermal
properties. The solid can consist of multiple layers of materials.
Each layer is partitioned into non-uniform cells, clustered near the
front and back faces.  The smallest cells are chosen based on the
criteria
\be \dx < S_s\sqrt{\frac{k_s}{\rho_s c_s}} \ee
where $S_s$ is a cell size factor defined by the user. By default,
$S_s$ is 1.0.  Interior cells increase in size by a user-defined
stretch factor when moving inwards from the surfaces. By default, the
stretch factor is 2.0. The cell boundaries are located at points
$x_i$. The temperature at the center of the $i$th cell is denoted $T_{s,i}$.
The (temperature-dependent) thermal conductivity of the solid
at the center of the $i$th cell is denoted $k_{s,i}$.
The temperatures are updated in time using an implicit
Crank-Nicolson scheme
\begin{eqnarray}
    \frac{T_{s,i}^{n+1}-T_{s,i}^n}{\dt} = \frac{1}{2 (\rho_s c_s)_i \dx_{i} }
& & \left(
    k_{s,i+\ha}\frac{T_{s,i+1}^n-T_{s,i}^n}{\dx_{s,i+\ha}} - \right.
    k_{s,i-\ha}\frac{T_{s,i}^n-T_{s,i-1}^n}{\dx_{s,i-\ha}} +  \nonumber \\
& & k_{s,i+\ha}\frac{T_{s,i+1}^{n+1}-T_{s,i}^{n+1}}{\dx_{i+\ha}} -
    \left.
    k_{s,i-\ha}\frac{T_{s,i}^{n+1}-T_{s,i-1}^{n+1}}{\dx_{i-\ha}}
     \right)
    + \frac{\dq_s'''}{\rho_s c_s}
\end{eqnarray}
for $1 \le i \le N$. The width of each cell is $\dx_i$. The distance
from the center of cell $i$ to the center of cell $i+1$ is
$\dx_{i+\ha}$. However, the material properties $k_s$, $c_s$, $\rho_s$
and source terms $\dq_s'''$ are updated in an explicit manner, using
the temperature information from time step $n$.


The boundary condition on the front surface is
\be  -k_s \dod{T_s}{x}(0,t) =  \dq''_c + \dq''_r \ee
If the internal radiation is solved for a solid, the radiation
boundary condition $\dq''_r$ is not used.

On the back surface, two possible boundary condition types may be
specified by the user. (1) If the back surface is assumed to be open
either to an ambient
void or to another part of the computational domain,
the back side boundary condition is similar to that of
the front side. (2) If the back side is assumed to be perfectly insulated,
an adiabatic boundary condition is used
\be  k_s \dod{T_s}{x} =  0 \ee

The boundary condition is discretized
\be -k_{s,1} \frac{T_{s,1}^{n+1}-T_{s,0}^{n+1}}{\dx_{\ha}} = \dq''_c{}^{(n+1)} + \dq''_r{}^{(n+1)} \ee
The convective flux at the next time step is computed as
\be
\dq''_c{}^{(n+1)} = h\left( T_{g} -
 	0.5 \left(T_{s,\ha}^n + T_{s,\ha}^{n+1}\right) \right)
\ee
and the radiative flux at the next time step is approximated with a
linearized form
\be
\dq''_r{}^{(n+1)} \approx \dq''_r{}^n - 4 \; \epsilon \; \sigma \; T_{s,\ha}^{n^3} \left(
  T_{s,\ha}^{n+1} - T_{s,\ha}^n \right)
\ee
The wall temperature is defined $T_w \equiv T_{s,\ha}=(T_{s,0}+T_{s,1})/2$.


The size and number of solid phase cells
can change during the course of a calculation as solid material is converted to gas. The size of each cell is reduced such that the cell density remains
equal to the density of the virgin material.
If the cell size gets below a pre-defined threshold (1~$\mu$m), the cell is completely removed.
Following cell shrinking or cell removal, the solid phase mesh is re-gridded and the mass and enthalpy values are interpolated from the old mesh to the new mesh.



\subsection{Radiation Heat Transfer to Solids}
\label{inradsection}

If it is assumed that the thermal radiation from the surrounding gases is
absorbed within an infinitely thin layer at the surface of the solid
obstruction, then the net radiative heat flux is the sum of incoming and outgoing
components, $\dq_r'' = \dq_{r,in}'' - \dq_{r,out}''$, where:
\begin{eqnarray}
 \dq_{r,in}'' &=& \epsilon\,
 \int_{\bs'\cdot \bn_w < 0} I_w(\bs')\; |\bs'\cdot \bn_w | \; d\bO
 \label{RFluxIn1} \\
 \dq_{r,out}'' &=& \epsilon\,\sigma\,T_w^4
 \label{RFluxOut1}
\end{eqnarray}
However, many common materials do not have infinite optical
thickness. Rather, the radiation penetrates the material
to some finite depth. The radiative transport within the solid (or
liquid) can be described as a source term in Eq.~(\ref{1dheat}).
A ``two-flux'' model based on the Schuster-Schwarzschild
approximation~\cite{Siegel:1} assumes the radiative
intensity is constant inside the ``forward'' and ``backward''
hemispheres. The transport equation for the intensity in the ``forward''
direction is
\be
 \frac{1}{2}\frac{dI^+(x)}{dx}=\kappa_s\,\left(I_b-I^+(x)\right)
 \label{RInForward}
\ee
where $x$ is the distance from the material surface and $\kappa_s$ is
the absorption coefficient
\be
   \kappa_s = \sum_{\alpha=1}^{N_m} X_\alpha \; \kappa_{s,\alpha}
\ee
A corresponding formula can be given for
the ``backward'' direction. Multiplying Eq.~\ref{RInForward} by $\pi$
gives us the ``forward'' radiative heat flux into the solid
\be
 \frac{1}{2}\frac{{d\dq^+_r(x)} }{dx}=\kappa_s\,
       \left(\sigma\,T_s^4-\dq_r^+(x)\right)
 \label{RFluxForward}
\ee
The radiative source term in the heat conduction equation is a sum of the
``forward'' and ``backward'' flux gradients
\be
  \dq_{s,r}'''(x) = \frac{d\dq_r^+(x)}{dx}+\frac{d\dq_r^-(x)}{dx}
\ee
The boundary condition for Eq.~\ref{RFluxForward} at the solid (or liquid)
surface is given by
\be
 \dq_r^+(0) = \dq_{r,in}'' + (1-\epsilon)\,\dq_r^-(0)
 \label{RFluxInBC}
\ee
where $\dq_r^-(0)$ is the ``backward'' radiative heat flux at the
surface. In this formulation, the surface emissivity and the internal
absorption are assumed to be independent properties of the
material.


\subsection{Convective Heat Transfer to Solids}
\label{conflux}

The calculation of the convective heat flux depends on whether one is
performing a Direct Numerical Simulation (DNS) or a
Large Eddy Simulation (LES).
In a DNS calculation, the convective heat flux to a solid surface $\dq''_c$
is obtained directly from the gas temperature gradient at the boundary
\be \dq_c'' = - k \; \dod{T}{n} = -k \frac{T_w-T_g}{\dn/2}\ee
where $k$ is the thermal conductivity of the gas,
$n$ is the spatial coordinate pointing into the solid, $\dn$ is the normal
grid spacing, $T_g$ is the gas temperature in the center of the first gas phase cell, and
$T_w$ is the wall surface temperature.

In an LES calculation, the convective heat flux to the surface is
obtained from a combination of natural and forced
convection correlations
\be \dq_c'' = h (T_g - T_w)
    \quad \hbox{W/m}^2 \quad ; \quad h =
    \max \; \left[ \; C\, |T_g-T_w|^\ot \; , \;
            \frac{k}{L} \; 0.037 \; \RE^\fofi \; \PR^\ot \;
            \right]  \quad
    \hbox{W/m$^2$/K} \ee
where $C$ is the coefficient for natural convection (1.52 for a horizontal surface
and 1.31 for a vertical surface)~\cite{Holman:1},
$L$ is a characteristic length related to the size of the physical
obstruction, $k$ is the thermal conductivity of the
gas, and the Reynolds $\RE$ and Prandtl $\PR$ numbers are based on the
gas flowing
past the obstruction. Since the Reynolds number is proportional to the
characteristic length, $L$, the heat transfer coefficient is weakly
related to $L$. For this reason, $L$ is taken to be 1~m for most
calculations.



\subsection{Component-Averaged Thermal Properties}
\label{matcoefs}

The conductivity and volumetric heat capacity of the solid are defined
\be
   k_s = \sum_{\alpha=1}^{N_m} X_\alpha \; k_{s,\alpha} \quad ; \quad
   \rho_s c_s = \sum_{\alpha=1}^{N_m} \rho_{s,\alpha} \; c_{s,\alpha}
\ee
$N_m$ is the number of material components forming the
solid. $\rho_{s,\alpha}$ is the
{\em component density}
\be
  \rho_{s,\alpha}=\rho_s \, Y_\alpha
\ee
where $\rho_s$ is the density of the composite material and $Y_\alpha$ is the mass fraction of material component $\alpha$.
The solid density is the sum of the component densities
\be
  \rho_s = \sum_{\alpha=1}^{N_m} \rho_{s,\alpha}
\ee
$X_\alpha$ is the volume fraction of component $\alpha$
\be
  X_\alpha = \frac{\rho_{s,\alpha}}{\rho_\alpha}  \left/ \sum_{\alpha'=1}^{N_m}\frac{\rho_{s,\alpha'}}{\rho_{\alpha'} }  \right.
  \label{volfrac}
\ee
where $\rho_\alpha$ is the density of material $\alpha$ in its pure form.
Multi-component solids are defined by specifying the mass fractions, $Y_\alpha$, and densities, $\rho_\alpha$,
of the individual components of the composite.



\section{Pyrolysis Models}
\label{pyrosection}

This section describes how solid phase reactions and the chemical
source term in the solid phase heat conduction equation,
$\dot{q}_{s,c}'''$,  are modeled. This is what is commonly referred to
as the ``pyrolysis model,'' but it actually can represent any number
of reactive processes, including evaporation, charring, and internal
heating.


\subsection{Specified Heat Release Rate}

Often the intent of a fire simulation is merely to predict the
transport of smoke and heat from a {\em specified} fire. In other
words, the heat release rate is a user input, not something the model
predicts. In these instances, the desired HRR is translated into a
mass flux for fuel at a given solid surface, which can be thought of
as the surface of a burner: \be \dm_f'' = \frac{ f(t) \;
\dq_{\hbox{user}}''}{\Delta H} \ee Usually, the user specifies a
desired heat release rate per unit area, $\dq_{\hbox{user}}''$, plus a
time ramp, $f(t)$, and the mass loss rate is computed accordingly.

\subsection{Solid Fuels}

Solids can undergo simultaneous reactions under the following assumptions:
\begin{itemize}
\setlength{\itemsep}{0.0in}
\item instantaneous release of volatiles from solid to the gas phase,
\item local thermal equilibrium between the solid and the volatiles,
\item no condensation of gaseous products, and
\item no porosity effects\footnote{Although porosity effects are not explicitly included in the model, it is possible to account for it
because the volume fractions defined by Eq.~(\ref{volfrac}) need not
sum to unity, in which case the thermal conductivity and absorption
coefficient are effectively reduced.}
\end{itemize}
Each material component may undergo several competing reactions, and
each of these reactions may produce some other solid component
(residue) and gaseous volatiles according to the yield coefficients
$\nu_s$ and $\nu_{g,\gamma}$, respectively.  These coefficients should
usually satisfy $\nu_s + \sum_\gamma \nu_{g,\gamma} = 1$, but smaller yields may
also be used to take into account the gaseous products that are not
explicitly included in the simulation.

Consider material component $\alpha$ that undergoes $N_{r,\alpha}$ separate reactions. We will use the index $\beta$ to represent one of
these reactions. For example:
\be \hbox{Material}_\alpha \rightarrow \nu_{s,\alpha \beta} \;
    \hbox{Residue}_{\alpha \beta} + \nu_{g,\alpha \beta,w} \; \hbox{H$_2$O} + \nu_{g,\alpha \beta,f} \; \hbox{HC} \ee
In this his particular reaction, condenced phase residue, water vapor
and hydrocarbon fuel are produced.

The local density of material component $\alpha$ evolves in time
according to the solid phase species conservation equation
\be
  \dod{ }{t} \left( \frac{\rho_{s,\alpha}}{\rho_{s0}} \right) =
    -\sum_{\beta=1}^{N_{r,\alpha}} r_{\alpha \beta} + S_\alpha
\ee
which says that the mass of conponent $\alpha$ is consumed by the
solid phase reactions $r_{\alpha \beta}$ and produced by other
reactions. $r_{\alpha \beta}$ is the rate of reaction $\beta$ in units
(1/s) and $\rho_{s0}$ is the initial density of the material layer.
$S_\alpha$ is the production rate of material component
$\alpha$ as a result of the reactions of the other
components. The reaction rates are functions of local mass
concentration and temperature, and calculated as a combination of
Arrhenius and power functions:
\be
r_{\alpha \beta} =
    \left( \frac{\rho_{s,\alpha}}{\rho_{s0}}\right)^{n_{s,\alpha\beta}}
    A_{\alpha \beta} \; \exp \left(-\frac{E_{\alpha\beta}}{RT_s}\right)
    \; \max \big[0,S_{thr,\alpha,\beta}(T_s-T_{thr,\alpha \beta}) \big]^{n_{t,\alpha\beta}}
	\label{Arrhenius}
\ee
where $T_{thr,\alpha \beta}$ is a threshold temperature that can be
used to dictate that the reaction must not occur below
($S_{thr,\alpha,\beta}=+1$)  or above ($S_{thr,\alpha,\beta}=-1$) a
user-specified temperature. By default, the term is deactivated
($S_{thr,\alpha,\beta}=+1, T_{thr,\alpha\beta}=0$ K). The chapter on pyrolysis in the FDS Verification Guide describes methods
for determining the kinetic parameters $A_{\alpha \beta}$ and $E_{\alpha\beta}$ using bench-scale measurement techniques.

The production term $S_\alpha$ is the sum over all the reactions where the
solid residue is material $\alpha$
\be
S_\alpha = \sum_{\alpha'=1}^{N_m} \sum_{\beta=1}^{N_{r,\alpha'}}
           \nu_{s,\alpha' \beta} \; r_{\alpha' \beta}
       \quad \quad
           \hbox{(where Residue$_{\alpha' \beta}$ = Material$_\alpha$) }
\ee
The volumetric production rate of each gaseous volatile is
\be
\dot{m}_{\gamma}''' = \rho_{s0}\; \sum_{\alpha=1}^{N_m} \sum_{\beta=1}^{N_{r,\alpha}}
    \nu_{g,\alpha \beta,\gamma} \; r_{\alpha \beta}
\ee
It is assumed that the gases are transported instantaneously to the surface, where the
mass fluxes are given by:
\footnote{In cylindrical and spherical coordinates, the mass fluxes are
\be
   \dm_\gamma'' =\frac{1}{R} \int_0^R \dm_\gamma'''(x) \,r dr \;\; ; \;\;
   \dm_\gamma'' =\frac{1}{R^2} \int_0^R \dm_\gamma'''(x) \,r^2 dr \;\;
\ee}

\be
   \dm_\gamma'' = \int_0^L \dm_\gamma'''(x) \, dx
\ee
where $L$ is the thickness of the surface. The chemical source term of
the heat conduction equation consists of the heats of reaction
\be
\dot{q}_{s,c}'''(x) = -\rho_{s0}\;
    \sum_{\alpha=1}^{N_m} \sum_{\beta=1}^{N_{r,\alpha}}
    r_{\alpha \beta}(x) H_{r,\alpha \beta}
\ee

\subsection{Phase-Change Materials}

Simple phase change problems, such as freezing or melting, are usually modelled as a Stefan-problems,
where the boundary between two phases is described as a sharp interface at constant phase-change temperature $T_f$.
The location of the phase boundary $x_f$ is governed by the equation
\be
k_{s,1} \dod{T_{s,1}}{x}-k_{s,2}\dod{T_{s,2}}{x} = \rho_s H_{r,\alpha\beta} \dod{x_f}{t}
\ee
where 1 and 2 refer to the materials on the two sides of the boundary. In the context of the fixed-grid finite-difference method, it is more convenient
to allow a small deviation from $T_f$ and solve the amount of mass reacting during the time step $\Delta t$
from the energy required to convert the mass from phase to another
\be
\dot{m}''' \Delta t = \frac{\rho_s c_s(T_s-T_f)}{H_{r,\alpha\beta}}
\ee
This reaction can be implemented by setting $T_{thr,\alpha\beta} = T_f$ and $A_{\alpha\beta} = c_s$ and turning on a specific
{\em phase-change reaction} mode. The reaction rate given by Eq.~\ref{Arrhenius} is then divided by factor $H_{r,\alpha\beta}\Delta t$.

\subsection{Liquid Fuels}

The rate at which liquid fuel evaporates when burning is a function of
the liquid temperature and the concentration of fuel vapor above the
pool surface. According to the Clausius-Clapeyron relation, the volume fraction of the
fuel vapor above the surface is a function of the liquid boiling temperature
\be X_f = \exp \left[ -\frac{ h_v W_f}{\cal R} \left(\frac{1}{T_s}-\frac{1}{T_b} \right) \right]
\label{CC_liquid}
\ee
where $h_v$ is the heat of vaporization, $W_f$ is the
molecular weight, $T_s$ is the surface temperature, and
$T_b$ is the boiling temperature of the fuel~\cite{Prasad:1}.

In the beginning of the simulation, an initial guess is made for the
fuel vapor mass flux
\be
\dot{m}''_i= \frac{\dot{V}''_i W_f}{{\cal R}T_a /p_0}
\ee
where $\dot{V}''_i$ is the initial vapor volume flux, defined by the
user (default $\dot{V}''_i = 5\cdot10^{-4}$~m$^3$/(sm$^2$)).
During the simulation, the evaporation mass flux is updated based on
the difference between current close-to-the-surface volume fraction of
fuel vapor and the equilibrium value given by Eq.~\ref{CC_liquid}.

For simplicity, the liquid fuel itself is treated like a thermally-thick
solid for the purpose of computing the heat conduction. There is no
computation of the convection of the liquid within the pool.



