% !TEX root = FDS_Validation_Guide.tex

\chapter{Heat Release Rate}

In many fire simulations, a  so-called  ``design  fire'' is  specified
either by  a regulatory authority  or by the engineers  performing the
analysis. Because the  fire's heat release rate is  specified, the role of
the model is to predict  the transport of heat and combustion products
throughout  the room or  rooms of  interest. Detailed  descriptions of  the
contents of the building are usually not necessary because these items
are assumed to not contribute to the fire,  and even if they are, the burning rate
will be specified, not predicted.  Sometimes, it is necessary  to predict
the heat  flux from the fire  to a nearby ``target,''  and even though
the target  may heat up  to some specified ignition  temperature, the
subsequent spread  of the  fire usually goes  beyond the scope  of the
analysis because of the uncertainty  inherent in object to object fire
spread.

Validation  studies  of FDS  to  date  have  focussed more  on  design
applications   than  reconstructions.  The   reason  is   that  design
applications usually involve specified  fires and demand a minimum of
thermophysical properties  of real materials.  Transport  of smoke and
heat  is  the  primary  focus,  and measurements  can  be  limited  to
well-placed thermocouples, a few  heat flux gauges, gas samplers,
etc. Phenomena of importance in forensic reconstructions, like second
item  ignition, flame  spread, vitiation  effects and  extinction, are
more   difficult  to   model  and   more  difficult   to   study  with
well-controlled experiments. Uncertainties  in material properties and
measurements, as  well as simplifying assumptions in  the model, often
force the  comparison between model and measurement  to be qualitative
at best.  Nevertheless, current validation  efforts are moving  in the
direction of these more difficult issues.



\section{Victoria University Ethanol Pan Fire}
\label{ethanol_pan}

In this example, a steel pan (0.7~m by 0.8~m) is filled with a thin layer (about 5~L, 9~mm) of ethanol, which burns out within about 10~min. This case tests a number of features -- burning
liquids, multiple layers of solids/liquids, and, most importantly, the absorption coefficient of the liquid. The pyrolysis models in FDS prior to version 5
assumed that radiative feedback from the fire and hot gases within a compartment were absorbed at the surface. In reality, this energy is absorbed in
depth; the extent of which is characterized by the absorption coefficient, $\kappa$. This is a property of the liquid, as well as the gaseous vapors. FDS now uses
an absorption coefficient for both the gas and solid/liquid phases. 

The results of three calculations are shown below, each identical except for the value of absorption coefficient. The results of a single experiment
are also shown, courtesy of Ian Thomas, Victoria University, Australia~\cite{Thomas:JFPE}.


\begin{center}
\includegraphics[height=2.2in]{SCRIPT_FIGURES/VU_Ethanol_Pan_Fire/VU_Ethanol_Pan_Fire_HRR}
\caption{Comparison of predicted and measured HRR from a burning pool of ethanol.}
\end{center}

