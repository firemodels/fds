\chapter{Gas Species and Smoke}

FDS uses a mixture fraction combustion model, meaning that all gas species within the compartment are
assumed to be functions of a single scalar variable.  FDS solves only one transport equation for this variable,
and reports gas concentrations at any given point at any given time by extracting
its value from a pre-computed ``look-up'' table.  For the major species, like carbon dioxide and oxygen,
the predictions are essentially an indicator of how well FDS is predicting the bulk transport of combustion products throughout the space.
For minor species, like carbon monoxide and soot, FDS version 4 does not account for changes in combustion efficiency,
relying only on fixed yields of CO and soot from the combustion process.
In reality, the generation rate of CO and soot change depending on the ventilation conditions in the compartment.


\section{NIST/WTC Test Series, Oxygen and CO$_2$}


\begin{figure}[p]
\begin{tabular*}{\textwidth}{l@{\extracolsep{\fill}}r}
\includegraphics[width=2.6in]{FIGURES/WTC_01_v5_Gas_Concentration} &
\includegraphics[width=2.6in]{FIGURES/WTC_02_v5_Gas_Concentration} \\
\includegraphics[width=2.6in]{FIGURES/WTC_03_v5_Gas_Concentration} &
\includegraphics[width=2.6in]{FIGURES/WTC_04_v5_Gas_Concentration} \\
\includegraphics[width=2.6in]{FIGURES/WTC_05_v5_Gas_Concentration} &
\includegraphics[width=2.6in]{FIGURES/WTC_06_v5_Gas_Concentration}
\end{tabular*}
\caption{Oxygen and CO$_2$ for the NIST/WTC Test Series.}
\label{NIST_WTC_Gas}
\end{figure}

\clearpage

\section{NIST/NRC Test Series, Oxygen and CO$_2$}

The following pages present comparisons of oxygen and carbon dioxide concentration predictions with measurement for the
NIST/NRC series. There were two oxygen measurements, one in the upper layer, one in the lower.  There was only one carbon
dioxide measurement in the upper layer.

\begin{figure}[p]
\begin{tabular*}{\textwidth}{l@{\extracolsep{\fill}}r}
\includegraphics[width=2.6in]{FIGURES/NIST_NRC_01_v5_Oxygen_Concentration} &
\includegraphics[width=2.6in]{FIGURES/NIST_NRC_07_v5_Oxygen_Concentration} \\
\includegraphics[width=2.6in]{FIGURES/NIST_NRC_02_v5_Oxygen_Concentration} &
\includegraphics[width=2.6in]{FIGURES/NIST_NRC_08_v5_Oxygen_Concentration} \\
\includegraphics[width=2.6in]{FIGURES/NIST_NRC_04_v5_Oxygen_Concentration} &
\includegraphics[width=2.6in]{FIGURES/NIST_NRC_10_v5_Oxygen_Concentration} \\
\includegraphics[width=2.6in]{FIGURES/NIST_NRC_13_v5_Oxygen_Concentration} &
\includegraphics[width=2.6in]{FIGURES/NIST_NRC_16_v5_Oxygen_Concentration}
\end{tabular*}
\caption{Oxygen and CO$_2$ for the NIST/NRC Series, closed door tests.}
\label{NIST_NRC_Gas_Closed}
\end{figure}

\begin{figure}[p]
\begin{tabular*}{\textwidth}{l@{\extracolsep{\fill}}r}
\includegraphics[width=2.6in]{FIGURES/NIST_NRC_17_v5_Oxygen_Concentration} &
 \\
\includegraphics[width=2.6in]{FIGURES/NIST_NRC_03_v5_Oxygen_Concentration} &
\includegraphics[width=2.6in]{FIGURES/NIST_NRC_09_v5_Oxygen_Concentration} \\
\includegraphics[width=2.6in]{FIGURES/NIST_NRC_05_v5_Oxygen_Concentration} &
\includegraphics[width=2.6in]{FIGURES/NIST_NRC_14_v5_Oxygen_Concentration} \\
\includegraphics[width=2.6in]{FIGURES/NIST_NRC_15_v5_Oxygen_Concentration} &
\includegraphics[width=2.6in]{FIGURES/NIST_NRC_18_v5_Oxygen_Concentration}
\end{tabular*}
\caption{Oxygen and CO$_2$ for the NIST/NRC Series, open door tests.}
\label{NIST_NRC_Gas_Open}
\end{figure}

\clearpage


\section{NIST/NRC Test Series, Smoke}

FDS treats smoke like all other combustion products, basically a tracer gas whose mass fraction is a function of the mixture fraction.
To model smoke movement, the user need only prescribe the smoke yield, that is, the fraction of the fuel mass that is
converted to smoke particulate.  For the simulations of the NIST/NRC tests, the smoke yield is specified as one of the test parameters.
Figure and Figure contain comparisons of measured and predicted smoke concentration at one measuring station in the upper layer.
There are two obvious trends in the figures: first, the predicted concentrations are about 50 \% higher than the measured
in the open door tests.  Second,
the predicted concentrations are roughly three times the measured concentrations in the closed door tests.
As a contrast, Figure displays the time history of CO concentration for 6 of the NIST/NRC tests.
Like smoke, the CO is specified in FDS via a fixed yield, measured along with smoke and reported in the test document.
The large differences between model and measurement seen in the smoke data do not appear in the CO data.

\begin{figure}[p]
\begin{tabular*}{\textwidth}{l@{\extracolsep{\fill}}r}
\includegraphics[width=2.6in]{FIGURES/NIST_NRC_01_v5_Smoke_Concentration} &
\includegraphics[width=2.6in]{FIGURES/NIST_NRC_07_v5_Smoke_Concentration} \\
\includegraphics[width=2.6in]{FIGURES/NIST_NRC_02_v5_Smoke_Concentration} &
\includegraphics[width=2.6in]{FIGURES/NIST_NRC_08_v5_Smoke_Concentration} \\
\includegraphics[width=2.6in]{FIGURES/NIST_NRC_04_v5_Smoke_Concentration} &
\includegraphics[width=2.6in]{FIGURES/NIST_NRC_10_v5_Smoke_Concentration} \\
\includegraphics[width=2.6in]{FIGURES/NIST_NRC_13_v5_Smoke_Concentration} &
\includegraphics[width=2.6in]{FIGURES/NIST_NRC_16_v5_Smoke_Concentration}
\end{tabular*}
\caption{Smoke concentration for the NIST/NRC Series, closed door tests.}
\label{NIST_NRC_Smoke_Closed}
\end{figure}

\begin{figure}[p]
\begin{tabular*}{\textwidth}{l@{\extracolsep{\fill}}r}
\includegraphics[width=2.6in]{FIGURES/NIST_NRC_17_v5_Smoke_Concentration} &
 \\
\includegraphics[width=2.6in]{FIGURES/NIST_NRC_03_v5_Smoke_Concentration} &
\includegraphics[width=2.6in]{FIGURES/NIST_NRC_09_v5_Smoke_Concentration} \\
\includegraphics[width=2.6in]{FIGURES/NIST_NRC_05_v5_Smoke_Concentration} &
\includegraphics[width=2.6in]{FIGURES/NIST_NRC_14_v5_Smoke_Concentration} \\
\includegraphics[width=2.6in]{FIGURES/NIST_NRC_15_v5_Smoke_Concentration} &
\includegraphics[width=2.6in]{FIGURES/NIST_NRC_18_v5_Smoke_Concentration}
\end{tabular*}
\caption{Smoke concentration for the NIST/NRC Series, open door tests.}
\label{NIST_NRC_Smoke_Open}
\end{figure}

\clearpage
