\documentclass[11pt]{book}
\usepackage{mathptm,times}
\usepackage[pdftex]{graphicx}
\usepackage{hyperref}

%\usepackage{eso-pic}
%\usepackage{graphicx}
%\usepackage{color}
%\usepackage{type1cm}

%\makeatletter
%   \AddToShipoutPicture{%
%     \setlength{\@tempdimb}{.5\paperwidth}%
%    \setlength{\@tempdimc}{.5\paperheight}%
%   \setlength{\unitlength}{1pt}%
%  \put(\strip@pt\@tempdimb,\strip@pt\@tempdimc){%
%     \makebox(0,0){\rotatebox{45}{\textcolor[gray]{0.75}{\fontsize{8cm}{8cm}\selectfont{DRAFT}}}}}}
%\makeatother

\setlength{\textwidth}{6.5in}
\setlength{\textheight}{9.0in}
\setlength{\topmargin}{0.in}
\setlength{\headheight}{0.in}
\setlength{\headsep}{0.in}
\setlength{\parindent}{0.25in}
\setlength{\oddsidemargin}{0.0in}
\setlength{\evensidemargin}{0.0in}

\begin{document}

\bibliographystyle{unsrt}

\newcommand{\dod}[2]{\frac{\partial #1}{\partial #2}}
\newcommand{\DoD}[2]{\frac{D #1}{D #2}}
\newcommand{\dsods}[2]{\frac{\partial^2 #1}{\partial #2^2}}
\newcommand{\dx}{\delta x}
\newcommand{\dy}{\delta y}
\newcommand{\dz}{\delta z}
\newcommand{\x}{x}
\newcommand{\y}{y}
\newcommand{\z}{z}
\newcommand{\dt}{\delta t}
\newcommand{\dn}{\delta n}
\newcommand{\cH}{{\cal H}}
\newcommand{\hu}{u}
\newcommand{\hv}{v}
\newcommand{\hw}{w}
\newcommand{\la}{\lambda}
%\newcommand{\bO}{\mbox{\boldmath $\Omega$}}
\newcommand{\bO}{{\Omega}}
\newcommand{\bo}{{\bf \omega}}
%\newcommand{\btau}{\mbox{\boldmath $\tau$}}
\newcommand{\btau}{{\bf \tau}}
\newcommand{\bdelta}{{\bf \delta}}
\newcommand{\sumym}{\sum (Y_i/W_i)}
\newcommand{\oW}{\overline{W}}
\newcommand{\om}{\omega}
\newcommand{\omx}{\omega_x}
\newcommand{\omy}{\omega_y}
\newcommand{\omz}{\omega_z}
\newcommand{\erf}{\hbox{erf}}
\newcommand{\bF}{{\bf F}}
\newcommand{\bof}{{\bf f}}
\newcommand{\bq}{{\bf q}}
\newcommand{\br}{{\bf r}}
\newcommand{\bu}{{\bf u}}
\newcommand{\bx}{{\bf x}}
\newcommand{\bk}{{\bf k}}
\newcommand{\bv}{{\bf v}}
\newcommand{\bg}{{\bf g}}
\newcommand{\bn}{{\bf n}}
\newcommand{\bS}{{\bf S}}
\newcommand{\dS}{d{\bf S}}
\newcommand{\bs}{{\bf s}}
\newcommand{\bI}{{\bf I}}
\newcommand{\hp}{{\cal H}}
\newcommand{\trho}{\tilde{\rho}}
\newcommand{\dph}{{\delta\phi}}
\newcommand{\dth}{{\delta\theta}}
\newcommand{\tp}{\tilde{p}}
\newcommand{\dQ}{\dot{Q}}
\newcommand{\dq}{\dot{q}}
\newcommand{\dm}{\dot{m}}
\newcommand{\ha}{\frac{1}{2}}
\newcommand{\ft}{\frac{4}{3}}
\newcommand{\ot}{\frac{1}{3}}
\newcommand{\fofi}{\frac{4}{5}}
\newcommand{\of}{\frac{1}{4}}
\newcommand{\twth}{\frac{2}{3}}
\newcommand{\R}{{\cal R}}
\newcommand{\be}{\begin{equation}}
\newcommand{\ee}{\end{equation}}
\newcommand{\RE}{\hbox{Re}}
\newcommand{\LE}{\hbox{Le}}
\newcommand{\PR}{\hbox{Pr}}
\newcommand{\PE}{\hbox{Pe}}
\newcommand{\NU}{\hbox{Nu}}
\newcommand{\SC}{\hbox{Sc}}
\newcommand{\SH}{\hbox{Sh}}
\newcommand{\WE}{\hbox{We}}
\newcommand{\COTWO}{{\tiny \hbox{CO}_2}}
\newcommand{\OTWO}{{\tiny \hbox{O}_2}}
\newcommand{\CO}{{\tiny \hbox{CO}}}
\newcommand{\F}{{\tiny \hbox{F}}}

\pagestyle{empty}


\section{CAROLFIRE -- CAble Response to Live FIRE}


% Enter Nowlen's report into the bibliography

These experiments include the fire-induced cable failure test results
from the CAROLFIRE Project. The 14 intermediate-scale tests involved
exposure to cables in trays generally in bundles of 6 to 12 cables each. The
cables were arranged in various routing configurations, placed at various
locations within a relatively open test structure, and were exposed to
open fires created by a propene (propylene) gas diffusion burner. The 
tests included exposure of cables both in the fire plume and under
hot gas layer exposure conditions. A broad range of representative
cable products were tested including both thermoset and thermoplastic 
insulated cablesthat are typical of the cable types and configurations
currently used in U.S. nuclear power plants. The fire typically spread,
at a minimum, to those cables located directly above the fire source.

The dimensions of the CAROLFIRE test structure were
2.4 m by 3.7 m by 3.0 m. The test structure consisted of
a steel framework of which only the upper 40\% was enclosed.
That is, the framework had an overall height of about 3 m
but remained open on all sides up to a height of about 1.8 meters.
Each of the four sides from a height of 1.8 m up and
the top of the structure were enclosed with a nonmetallic
material. This test structure acted to focus the fire's heat output
initially to this confined volume creating the desired hot gas layer
exposure conditions. As the fire progressed the hot gas layer depth
increased and ultimately smoke and hot gasses spilled out naturally 
from under the sides of the enclosed area.

%\begin{figure}[ht!]
%\begin{center}
%\includegraphics[width=*]{FIGURES/CAROLFIRE_IT}\\
%\caption{Schematic of the CAROLFIRE Intermediate Scale test Structure}\label{CAROLFIRE_SCHEMATIC}
%\end{center}
%\end{figure}


% General schematic from page 12 from Nowlen's report



\clearpage


\section{CAROLFIRE}

This section will go into the chapter entitled ``Target Response'' or whatever. There will be several sections
describing results from other experiments with measurements of various device temperatures, activation times,
etc. This section should start with the HRRs, then list hood temperatures, Tray A, B, etc. Do it Tray by Tray, not
test by test.




\begin{figure}[p]
\begin{tabular*}{\textwidth}{l@{\extracolsep{\fill}}r}
%\includegraphics[width=2.6in]{FIGURES/CAROLFIRE_IT_1_Heat_Release_Rate} &
%\includegraphics[width=2.6in]{FIGURES/CAROLFIRE_IT_2_Heat_Release_Rate} \\
%\includegraphics[width=2.6in]{FIGURES/CAROLFIRE_IT_3_Heat_Release_Rate} &
%\includegraphics[width=2.6in]{FIGURES/CAROLFIRE_IT_4_Heat_Release_Rate} \\
%\includegraphics[width=2.6in]{FIGURES/CAROLFIRE_IT_5_Heat_Release_Rate} &
%\includegraphics[width=2.6in]{FIGURES/CAROLFIRE_IT_6_Heat_Release_Rate} \\
%\includegraphics[width=2.6in]{FIGURES/CAROLFIRE_IT_7_Heat_Release_Rate} &
%\includegraphics[width=2.6in]{FIGURES/CAROLFIRE_IT_8_Heat_Release_Rate}
\end{tabular*}
\caption{Heat Release Rate for the CAROLFIRE Intermediate Scale Tests.}
\label{CAROLFIRE_HRR_1-8}
\end{figure}


\begin{figure}[p]
\begin{tabular*}{\textwidth}{l@{\extracolsep{\fill}}r}
%\includegraphics[width=2.6in]{FIGURES/CAROLFIRE_IT_9_Heat_Release_Rate} &
%\includegraphics[width=2.6in]{FIGURES/CAROLFIRE_IT_10_Heat_Release_Rate} \\
%\includegraphics[width=2.6in]{FIGURES/CAROLFIRE_IT_11_Heat_Release_Rate} &
%\includegraphics[width=2.6in]{FIGURES/CAROLFIRE_IT_12_Heat_Release_Rate} \\
%\includegraphics[width=2.6in]{FIGURES/CAROLFIRE_IT_13_Heat_Release_Rate} &
%\includegraphics[width=2.6in]{FIGURES/CAROLFIRE_IT_14_Heat_Release_Rate}
\end{tabular*}
\caption{Heat Release Rate for the CAROLFIRE Intermediate Scale Tests.}
\label{CAROLFIRE_HRR_9-14}
\end{figure}


\clearpage

\bibliography{../Bibliography/FDS_refs,../Bibliography/FDSVVBiBnew,../Bibliography/FDS_mathcomp}



\end{document}
