\chapter{Heating, Ventilating, and Air Conditioning (HVAC)}

Gas velocity is often measured at compartment inlets and outlets as part of a global assessment of mass and
energy conservation.  This chapter contains measurements of gas velocity and related quantities.

\section{ASHRAE Sample Problem 7}

ASRHAE sample problem 7 ~\cite{ASHRAE} is a simple HVAC system that represents a metal working exhaust system for a machine shop where particulates from three pieces of equipment are removed by a dust collector.  The network, shown in Figure~\ref{ASHRAE7_schematic} consists of three inlets, two tees, the dust collector, and a fan.  The lengths, diameters, and friction losses for all the components were defined as well as the fan flow rate.  Pressure drops and flow rates in the remaining components can be computed by hand.  
Steckler {\em et al.}~\cite{Steckler:NBSIR_82-2520} mapped the doorway/window flows in 55 

\begin{figure}[p]
\begin{center}
\begin{tabular}{l}
\includegraphics[width=5.0in]{FIGURES/HVAC/ashrae_7.pdf}
\end{tabular}
\end{center}
\caption {HVAC network for ASHRAE sample problem 7}
\label{ASHRAE7_schematic}
\end{figure}

\newpage





\begin{figure}[p]
\begin{tabular*}{\textwidth}{l@{\extracolsep{\fill}}r}
\includegraphics[height=2.2in]{FIGURES/Steckler_Compartment/Steckler_010_Vel} &
\includegraphics[height=2.2in]{FIGURES/Steckler_Compartment/Steckler_011_Vel} \\
\includegraphics[height=2.2in]{FIGURES/Steckler_Compartment/Steckler_012_Vel} &
\includegraphics[height=2.2in]{FIGURES/Steckler_Compartment/Steckler_612_Vel} \\
\includegraphics[height=2.2in]{FIGURES/Steckler_Compartment/Steckler_013_Vel} &
\includegraphics[height=2.2in]{FIGURES/Steckler_Compartment/Steckler_014_Vel} \\
\includegraphics[height=2.2in]{FIGURES/Steckler_Compartment/Steckler_018_Vel} &
\includegraphics[height=2.2in]{FIGURES/Steckler_Compartment/Steckler_710_Vel}
\end{tabular*}
\label{Steckler_Vel_1}
\end{figure}

\begin{figure}[p]
\begin{tabular*}{\textwidth}{l@{\extracolsep{\fill}}r}
\includegraphics[height=2.2in]{FIGURES/Steckler_Compartment/Steckler_810_Vel} &
\includegraphics[height=2.2in]{FIGURES/Steckler_Compartment/Steckler_016_Vel} \\
\includegraphics[height=2.2in]{FIGURES/Steckler_Compartment/Steckler_017_Vel} &
\includegraphics[height=2.2in]{FIGURES/Steckler_Compartment/Steckler_022_Vel} \\
\includegraphics[height=2.2in]{FIGURES/Steckler_Compartment/Steckler_023_Vel} &
\includegraphics[height=2.2in]{FIGURES/Steckler_Compartment/Steckler_030_Vel} \\
\includegraphics[height=2.2in]{FIGURES/Steckler_Compartment/Steckler_041_Vel} &
\includegraphics[height=2.2in]{FIGURES/Steckler_Compartment/Steckler_019_Vel}
\end{tabular*}
\label{Steckler_Vel_2}
\end{figure}

\begin{figure}[p]
\begin{tabular*}{\textwidth}{l@{\extracolsep{\fill}}r}
\includegraphics[height=2.2in]{FIGURES/Steckler_Compartment/Steckler_020_Vel} &
\includegraphics[height=2.2in]{FIGURES/Steckler_Compartment/Steckler_021_Vel} \\
\includegraphics[height=2.2in]{FIGURES/Steckler_Compartment/Steckler_114_Vel} &
\includegraphics[height=2.2in]{FIGURES/Steckler_Compartment/Steckler_144_Vel} \\
\includegraphics[height=2.2in]{FIGURES/Steckler_Compartment/Steckler_212_Vel} &
\includegraphics[height=2.2in]{FIGURES/Steckler_Compartment/Steckler_242_Vel} \\
\includegraphics[height=2.2in]{FIGURES/Steckler_Compartment/Steckler_410_Vel} &
\includegraphics[height=2.2in]{FIGURES/Steckler_Compartment/Steckler_210_Vel}
\end{tabular*}
\label{Steckler_Vel_3}
\end{figure}

\begin{figure}[p]
\begin{tabular*}{\textwidth}{l@{\extracolsep{\fill}}r}
\includegraphics[height=2.2in]{FIGURES/Steckler_Compartment/Steckler_310_Vel} &
\includegraphics[height=2.2in]{FIGURES/Steckler_Compartment/Steckler_240_Vel} \\
\includegraphics[height=2.2in]{FIGURES/Steckler_Compartment/Steckler_116_Vel} &
\includegraphics[height=2.2in]{FIGURES/Steckler_Compartment/Steckler_122_Vel} \\
\includegraphics[height=2.2in]{FIGURES/Steckler_Compartment/Steckler_224_Vel} &
\includegraphics[height=2.2in]{FIGURES/Steckler_Compartment/Steckler_324_Vel} \\
\includegraphics[height=2.2in]{FIGURES/Steckler_Compartment/Steckler_220_Vel} &
\includegraphics[height=2.2in]{FIGURES/Steckler_Compartment/Steckler_221_Vel}
\end{tabular*}
\label{Steckler_Vel_4}
\end{figure}

\begin{figure}[p]
\begin{tabular*}{\textwidth}{l@{\extracolsep{\fill}}r}
\includegraphics[height=2.2in]{FIGURES/Steckler_Compartment/Steckler_514_Vel} &
\includegraphics[height=2.2in]{FIGURES/Steckler_Compartment/Steckler_544_Vel} \\
\includegraphics[height=2.2in]{FIGURES/Steckler_Compartment/Steckler_512_Vel} &
\includegraphics[height=2.2in]{FIGURES/Steckler_Compartment/Steckler_542_Vel} \\
\includegraphics[height=2.2in]{FIGURES/Steckler_Compartment/Steckler_610_Vel} &
\includegraphics[height=2.2in]{FIGURES/Steckler_Compartment/Steckler_510_Vel} \\
\includegraphics[height=2.2in]{FIGURES/Steckler_Compartment/Steckler_540_Vel} &
\includegraphics[height=2.2in]{FIGURES/Steckler_Compartment/Steckler_517_Vel}
\end{tabular*}
\label{Steckler_Vel_5}
\end{figure}

\begin{figure}[p]
\begin{tabular*}{\textwidth}{l@{\extracolsep{\fill}}r}
\includegraphics[height=2.2in]{FIGURES/Steckler_Compartment/Steckler_622_Vel} &
\includegraphics[height=2.2in]{FIGURES/Steckler_Compartment/Steckler_522_Vel} \\
\includegraphics[height=2.2in]{FIGURES/Steckler_Compartment/Steckler_524_Vel} &
\includegraphics[height=2.2in]{FIGURES/Steckler_Compartment/Steckler_541_Vel} \\
\includegraphics[height=2.2in]{FIGURES/Steckler_Compartment/Steckler_520_Vel} &
\includegraphics[height=2.2in]{FIGURES/Steckler_Compartment/Steckler_521_Vel} \\
\includegraphics[height=2.2in]{FIGURES/Steckler_Compartment/Steckler_513_Vel} &
\includegraphics[height=2.2in]{FIGURES/Steckler_Compartment/Steckler_160_Vel}
\end{tabular*}
\label{Steckler_Vel_6}
\end{figure}

\begin{figure}[p]
\begin{tabular*}{\textwidth}{l@{\extracolsep{\fill}}r}
\includegraphics[height=2.2in]{FIGURES/Steckler_Compartment/Steckler_163_Vel} &
\includegraphics[height=2.2in]{FIGURES/Steckler_Compartment/Steckler_164_Vel} \\
\includegraphics[height=2.2in]{FIGURES/Steckler_Compartment/Steckler_165_Vel} &
\includegraphics[height=2.2in]{FIGURES/Steckler_Compartment/Steckler_162_Vel} \\
\includegraphics[height=2.2in]{FIGURES/Steckler_Compartment/Steckler_167_Vel} &
\includegraphics[height=2.2in]{FIGURES/Steckler_Compartment/Steckler_161_Vel} \\
\includegraphics[height=2.2in]{FIGURES/Steckler_Compartment/Steckler_166_Vel} &

\end{tabular*}
\label{Steckler_Vel_7}
\end{figure}




\clearpage

\section{Bryant Doorway Experiments}

On the following page there are seven plots comparing the predicted and measured centerline velocity\footnote{Note that the quantity
that is being compared is the total velocity multiplied by the sign of its normal component.} profiles
in a doorway of a standard ISO~9705 compartment. The measurements shown are based on PIV (Particle Image Velocimetry).
Note that the measurements do not extend to the top of the
doorway 1.96~m above the compartment floor because the heat from the fire prevented adequate laser resolution of
the particles. Velocity measurements were also made using bi-directional probes~\cite{Bryant:FSJ2009}, but these
measurements were shown to be up to 20~\% greater in magnitude than the comparable PIV measurement.



\begin{figure}[p]
\begin{tabular*}{\textwidth}{l@{\extracolsep{\fill}}r}
\includegraphics[height=2.2in]{FIGURES/Bryant_Doorway/Bryant_Doorway_034_kW} &
\includegraphics[height=2.2in]{FIGURES/Bryant_Doorway/Bryant_Doorway_065_kW} \\
\includegraphics[height=2.2in]{FIGURES/Bryant_Doorway/Bryant_Doorway_096_kW} &
\includegraphics[height=2.2in]{FIGURES/Bryant_Doorway/Bryant_Doorway_128_kW} \\
\includegraphics[height=2.2in]{FIGURES/Bryant_Doorway/Bryant_Doorway_160_kW} &
\includegraphics[height=2.2in]{FIGURES/Bryant_Doorway/Bryant_Doorway_320_kW} \\
\includegraphics[height=2.2in]{FIGURES/Bryant_Doorway/Bryant_Doorway_511_kW} &
\end{tabular*}
\label{Bryant_Doorway}
\end{figure}


%\clearpage


%\section{NIST/WTC Test Series}

%\begin{figure}[p]
%\begin{tabular*}{\textwidth}{l@{\extracolsep{\fill}}r}
%\includegraphics[height=2.2in]{FIGURES/WTC/WTC_01_v5_Inlet_Velocity} &
%\includegraphics[height=2.2in]{FIGURES/WTC/WTC_01_v5_Outlet_Velocity} \\
%\includegraphics[height=2.2in]{FIGURES/WTC/WTC_02_v5_Inlet_Velocity} &
%\includegraphics[height=2.2in]{FIGURES/WTC/WTC_02_v5_Outlet_Velocity} \\
%\includegraphics[height=2.2in]{FIGURES/WTC/WTC_03_v5_Inlet_Velocity} &
%\includegraphics[height=2.2in]{FIGURES/WTC/WTC_03_v5_Outlet_Velocity}
%\end{tabular*}
%\label{NIST_WTC_Velocity_1}
%\end{figure}


%\begin{figure}[p]
%\begin{tabular*}{\textwidth}{l@{\extracolsep{\fill}}r}
%\includegraphics[height=2.2in]{FIGURES/WTC/WTC_04_v5_Inlet_Velocity} &
%\includegraphics[height=2.2in]{FIGURES/WTC/WTC_04_v5_Outlet_Velocity} \\
%\includegraphics[height=2.2in]{FIGURES/WTC/WTC_05_v5_Inlet_Velocity} &
%\includegraphics[height=2.2in]{FIGURES/WTC/WTC_05_v5_Outlet_Velocity} \\
%\includegraphics[height=2.2in]{FIGURES/WTC/WTC_06_v5_Inlet_Velocity} &
%\includegraphics[height=2.2in]{FIGURES/WTC/WTC_06_v5_Outlet_Velocity}
%\end{tabular*}
%\label{NIST_WTC_Velocity_2}
%\end{figure}



%\begin{figure}[ht]
%\begin{tabular*}{\textwidth}{l@{\extracolsep{\fill}}r}
%\includegraphics[width=3.0in]{FIGURES/ScatterPlots/Velocity} &
%\end{tabular*}
%\caption{Summary of Velocity Results.}
%\end{figure}


\clearpage

\section{Restivo Experiment}

The results of a simulation of Restivo's room ventilation experiment are presented below.
To capture the forced inlet flow, the volume near the supply slot needs a fairly
fine grid to capture the mixing of air at the shear layer. For the results shown here, the height of the inlet
was spanned with 6 grid cells, roughly 3~cm in the vertical dimension, 6~cm in the other two. Finer grids
were used in the Musser study~\cite{Musser:1}, but with no appreciable change in results. The component
of velocity in the lengthwise direction was measured in four arrays: two vertical arrays located 3~m and 6~m  from the inlet along the
centerline of the room, and two horizontal arrays located 8.4~cm above the floor and below the ceiling, respectively.
These measurements were taken using hot-wire anemometers. While data on the specific
instrumentation used are not readily available, hot-wire systems tend to have limitations at low velocities,
with typical thresholds of approximately 0.1~m/s.

\begin{figure}[h!]
\begin{tabular*}{\textwidth}{l@{\extracolsep{\fill}}r}
\includegraphics[height=2.2in]{FIGURES/Restivo_Experiment/Restivo_3m_Velocity} &
\includegraphics[height=2.2in]{FIGURES/Restivo_Experiment/Restivo_6m_Velocity} \\
\includegraphics[height=2.2in]{FIGURES/Restivo_Experiment/Restivo_Ceiling_Velocity} &
\includegraphics[height=2.2in]{FIGURES/Restivo_Experiment/Restivo_Floor_Velocity}
\end{tabular*}
\label{Restivo_Velocity}
\end{figure}

\clearpage

\section{ATF Corridor}

Comparisons of bi-directional velocity measurements with FDS predictions for the ATF Corridor experiments are presented on the following
pages. Velocity measurements were made at four locations, two on the first level (Trees H and I) and two on the second level (Trees J and K).
Shown are the upper-most and lower-most probe for each vertical array. Typically there were four probes per tree, with the number 1 indicating the
upper-most probe.

\begin{figure}[p]
\begin{tabular*}{\textwidth}{l@{\extracolsep{\fill}}r}
\includegraphics[height=2.2in]{FIGURES/ATF_Corridors/ATF_Corridors_Velocity_H_050_kW} &
\includegraphics[height=2.2in]{FIGURES/ATF_Corridors/ATF_Corridors_Velocity_H_100_kW} \\
 &
\includegraphics[height=2.2in]{FIGURES/ATF_Corridors/ATF_Corridors_Velocity_H_250_kW} \\
\includegraphics[height=2.2in]{FIGURES/ATF_Corridors/ATF_Corridors_Velocity_H_500_kW} &
\includegraphics[height=2.2in]{FIGURES/ATF_Corridors/ATF_Corridors_Velocity_H_Pulsed_HRR}
\end{tabular*}
\label{ATF_Velocity_H}
\end{figure}

\begin{figure}[p]
\begin{tabular*}{\textwidth}{l@{\extracolsep{\fill}}r}
\includegraphics[height=2.2in]{FIGURES/ATF_Corridors/ATF_Corridors_Velocity_I_050_kW} &
\includegraphics[height=2.2in]{FIGURES/ATF_Corridors/ATF_Corridors_Velocity_I_100_kW} \\
\includegraphics[height=2.2in]{FIGURES/ATF_Corridors/ATF_Corridors_Velocity_I_240_kW} &
\includegraphics[height=2.2in]{FIGURES/ATF_Corridors/ATF_Corridors_Velocity_I_250_kW} \\
\includegraphics[height=2.2in]{FIGURES/ATF_Corridors/ATF_Corridors_Velocity_I_500_kW} &
\includegraphics[height=2.2in]{FIGURES/ATF_Corridors/ATF_Corridors_Velocity_I_Pulsed_HRR}
\end{tabular*}
\label{ATF_Velocity_I}
\end{figure}

\begin{figure}[p]
\begin{tabular*}{\textwidth}{l@{\extracolsep{\fill}}r}
\includegraphics[height=2.2in]{FIGURES/ATF_Corridors/ATF_Corridors_Velocity_J_050_kW} &
\includegraphics[height=2.2in]{FIGURES/ATF_Corridors/ATF_Corridors_Velocity_J_100_kW} \\
\includegraphics[height=2.2in]{FIGURES/ATF_Corridors/ATF_Corridors_Velocity_J_240_kW} &
\includegraphics[height=2.2in]{FIGURES/ATF_Corridors/ATF_Corridors_Velocity_J_250_kW} \\
\includegraphics[height=2.2in]{FIGURES/ATF_Corridors/ATF_Corridors_Velocity_J_500_kW} &
\includegraphics[height=2.2in]{FIGURES/ATF_Corridors/ATF_Corridors_Velocity_J_Pulsed_HRR}
\end{tabular*}
\label{ATF_Velocity_J}
\end{figure}

\begin{figure}[p]
\begin{tabular*}{\textwidth}{l@{\extracolsep{\fill}}r}
\includegraphics[height=2.2in]{FIGURES/ATF_Corridors/ATF_Corridors_Velocity_K_050_kW} &
\includegraphics[height=2.2in]{FIGURES/ATF_Corridors/ATF_Corridors_Velocity_K_100_kW} \\
\includegraphics[height=2.2in]{FIGURES/ATF_Corridors/ATF_Corridors_Velocity_K_240_kW} &
\includegraphics[height=2.2in]{FIGURES/ATF_Corridors/ATF_Corridors_Velocity_K_250_kW} \\
\includegraphics[height=2.2in]{FIGURES/ATF_Corridors/ATF_Corridors_Velocity_K_500_kW} &
\includegraphics[height=2.2in]{FIGURES/ATF_Corridors/ATF_Corridors_Velocity_K_Pulsed_HRR}
\end{tabular*}
\label{ATF_Velocity_K}
\end{figure}

