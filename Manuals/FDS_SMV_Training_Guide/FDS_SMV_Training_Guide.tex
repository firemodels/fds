\documentclass[11pt]{book}
\usepackage{times,mathptm}
\usepackage[pdftex]{graphicx}
\usepackage[pdftex,
        colorlinks=true,
        urlcolor=linkblue,     % \href{...}{...} external (URL)
        citecolor=linkred,     % citation number colors
        linkcolor=linknavy,    % \ref{...} and \pageref{...}
        pdftitle={Fire Fighter Trainer Scenario Guide},
        pdfauthor={Glenn Forney, Kevin McGrattan, Daniel Madrykowski and Stephen Kerber},
        pdfsubject={User Guide},
        pdfkeywords={FDS, Fire Model, NIST, BFRL},
        pdfproducer={pdflatex},
        pagebackref,
        pdfpagemode=UseNone,
        bookmarksopen=true,
        plainpages=false]{hyperref}
\usepackage{color}
\definecolor{linknavy}{rgb}{0,0,0.50196}
\definecolor{linkred}{rgb}{1,0,0}
\definecolor{linkblue}{rgb}{0,0,1}
\usepackage{caption}
\usepackage{graphpap}
\usepackage{rotating}
\usepackage{epsfig,psfrag}
%\usepackage{wrapfig}
%\usepackage{picins}
\usepackage{geometry}
\usepackage{tabularx}
\usepackage{longtable}
\usepackage{lscape}
\usepackage{amssymb}
\usepackage{makeidx} % Create index at end of document
\usepackage[nottoc,notlof,notlot]{tocbibind} % Put the bibliography and index in the ToC
\usepackage{float}
\usepackage{lastpage} % Automatic last page number reference.
\usepackage[T1]{fontenc}
\usepackage{upquote}
\usepackage{array,eqnarray}
\newcommand{\nopart}{\expandafter\def\csname Parent-1\endcsname{}} % To fix table of contents in pdf.
\newcommand{\ct}{\tt\small}

% The Following commented code makes the ``Draft'' watermark on each page.
%\usepackage{eso-pic}
%\usepackage{type1cm}
%\makeatletter
%   \AddToShipoutPicture{
%     \setlength{\@tempdimb}{.5\paperwidth}
%     \setlength{\@tempdimc}{.5\paperheight}
%     \setlength{\unitlength}{1pt}
%     \put(\strip@pt\@tempdimb,\strip@pt\@tempdimc){
%     \makebox(0,0){\rotatebox{45}{\textcolor[gray]{0.75}{\fontsize{8cm}\selectfont{RC6}}}}}
% }
%\makeatother

\setlength{\textwidth}{6.5in}
\setlength{\textheight}{9.0in}
\setlength{\topmargin}{0.in}
\setlength{\headheight}{0.in}
\setlength{\headsep}{0.in}
\setlength{\parindent}{0.25in}
\setlength{\oddsidemargin}{0.0in}
\setlength{\evensidemargin}{0.0in}

\newcommand{\be}{\begin{equation}}
\newcommand{\ee}{\end{equation}}
\newcommand{\note}{{\bf Note:}}

\begin{document}
\bibliographystyle{unsrt}

\pagestyle{empty}
\pagenumbering{alph}

\begin{minipage}[t][9in][s]{6.25in}

\huge
\flushright{NIST Special Publication XXXX}

\vspace{1in}

\Huge \flushright{Fire Fighter Trainer \\ Scenario Guide }

\vspace{.5in}

\normalsize

\large
\flushright{
Glenn Forney \\
Kevin McGrattan \\
Daniel Madrzykowski \\
Stephen Kerber \\
 }


\vfill

\flushright{\includegraphics[width=2.in]{FIGURES/nistident_flright_vec}}


\end{minipage}

\newpage
\hspace{5in}
\newpage

\begin{minipage}[t][9in][s]{6.25in}

\huge
\flushright{NIST Special Publication XXXX}

\vspace{.75in}

\Huge
\flushright{Fire Fighter Trainer \\ Scenario Guide }

\vspace{.25in}

\normalsize
\flushright{
Glenn Forney \\
Kevin McGrattan \\
Daniel Madrzykowski \\
Stephen Kerber \\
{\em NIST Building and Fire Research Laboratory} \\
{\em Gaithersburg, Maryland, USA}}

\vspace{.25in}

\flushright{\today \\
$SVN Repository$~$Revision: 1815 $}

\vfill

\flushright{\includegraphics[width=1in]{FIGURES/doc.pdf} }

\small
\flushright{U.S. Department of Commerce \\
{\em Carlos M. Gutierrez, Secretary} \\
\hspace{1in} \\
National Institute of Standards and Technology \\
{\em James M. Turner, Acting Director} }


\end{minipage}

\newpage

\begin{minipage}[t][9in][s]{6.25in}

\flushright{Certain commercial entities, equipment, or materials may be identified in this \\
document in order to describe an experimental procedure or concept adequately. Such \\
identification is not intended to imply recommendation or endorsement by the \\
National Institute of Standards and Technology, nor is it intended to imply that the \\
entities, materials, or equipment are necessarily the best available for the purpose.
}

\vspace{3in}

\large
\flushright{\bf National Institute of Standards and Technology Special Publication XXXX \\
Natl.~Inst.~Stand.~Technol.~Spec.~Publ.~XXXX, \pageref{LastPage} pages (October 2008) \\
CODEN: NSPUE2 }

\vfill

\flushright{U.S. GOVERNMENT PRINTING OFFICE \\
WASHINGTON: 2007 \\
\rule{3.5in}{0.01in} \\
For sale by the Superintendent of Documents, U.S. Government Printing Office \\
Internet: bookstore.gpo.gov -- Phone: (202) 512-1800 -- Fax: (202) 512-2250 \\
Mail: Stop SSOP, Washington, DC 20402-0001 }
\end{minipage}

\clearpage

\frontmatter

\pagestyle{plain}
\pagenumbering{roman}


\chapter{Preface}

This Guide contains a description of experiments and corresponding numerical simulations performed in
facilities used  to train fire fighters.  It also contains descriptions of demonstration
numerical simulations designed to illustrate fundamental fire phenomena encountered while
fighting fire, phenomena such as how smoke layers behave in the presense or absence of
door openings.  These simulations are performed by the Fire Dynamics Simulator (FDS)
and visualized  by the visualization program, Smokeview.

Note that this Guide does not provide the background theory for FDS nor a detailed description of Smokeview. The FDS
Technical Reference Guide~\cite{FDS_Tech_Guide_5} contains details about the governing
equations and numerical methods. The FDS User's Guide~\cite{FDS_Users_Guide_5} describes how to run FDS. Smokeview is described in the ``User's Guide for
Smokeview Version~5''~\cite{Smokeview_Users_Guide_5}.


\chapter{Disclaimer}

The US Department of Commerce makes no warranty, expressed or implied, to
users of the Fire Dynamics Simulator (FDS) or Smokeview, and accepts no responsibility for their
use. Users of FDS and Smokeview assume sole responsibility under Federal law for
determining the appropriateness of its use in any particular application;
for any conclusions drawn from the results of its use; and for any actions
taken or not taken as a result of analyses performed using these tools.

Users are warned that FDS and Smokeview are intended for use only by those competent in
the fields of fluid dynamics, thermodynamics, combustion, and heat transfer,
and is intended only to supplement the
informed judgment of the qualified user. The software package is a
computer model that may or may not have predictive capability when applied
to a specific set of factual circumstances. Lack of accurate predictions by
the model could lead to erroneous conclusions with regard to fire safety.
All results should be evaluated by an informed user.

Throughout this document, the mention of computer hardware or
commercial software does not constitute endorsement by NIST, nor does
it indicate that the products are necessarily those best suited for the
intended purpose.


\chapter{About the Authors}

\begin{description}
\item[Glenn Forney] is a computer scientist in the Building and Fire
Research Laboratory (BFRL) of NIST.
\item[Kevin McGrattan] is a mathematician in BFRL. He received a bachelors of science degree
from the School of Engineering and Applied Science of Columbia
University in 1987 and a doctorate at the Courant Institute of New
York University in 1991. He joined the NIST staff in 1992 and has
since worked on the development of fire models, most notably the Fire
Dynamics Simulator.
\item[Daniel Madrzykowski] is a fire protection engineer
in BFRL of NIST.
\item[Stephen Kerber] is a fire protection engineer
in BFRL of NIST.
\end{description}



\chapter{Acknowledgments}


\tableofcontents
\listoffigures
\listoftables

\mainmatter


\chapter{Introduction}
The most effective method for learning how to fight fire is to fight fire.  This however is expensive and
particularly dangerous for the trainee.   In particular, some fire situations that must be trained for are too
large and dangerous to re-create in a training setting involving real fires.  Environmental concerns are
limiting the amount and kind of live fire training available in many areas of the country.  Methods are then
needed to allow fire fighters to gain valuable experience using virtual reality techniques already applied in
other fields so that they may learn without the possibility of harming themselves or others.  It is a challenge
to simulate fire fighter training scenarios that are accurate and to present them in a form that are realistic
and practical due to the large amounts of computational resources required.   Presently, fire fighter trainers
either concentrate on incident command issues or depend on an {\em expert}\ to alter the fire.
The trainer developed
here will be Physics based so that the fire and smoke visualized will be closer to what one would expect to find
when fighting real fires.

Presently, fire fighters train in the classroom and in live fire training using real fires in real
facilities.  Visual based training presently relies on experts to run the fire and are not physics based.
This project will continue to develop a computer based fire fighting training tool to improve training
opportunities while lowering the cost and risk of death and injury.   Two methods are being used to create a
training tool.   The first and simpler method  is to use FDS and Smokeview to create animations of fire
scenarios.  These animations will be viewable in a standard DVD player.  The DVD menus will be used to walk a
fire fighter through a series of decisions.  The second method will be more interactive.  Using Smokeview,
the trainee will {\em walk}\ through an FDS generated fire scene observing and making decisions.  Several  scenarios
will be simulated with FDS.  These scenarios involve cases where experimental data is available; the Maryland
Fire Research Institute's (MFRI) training tower and the Montgomery County's flashover trainer.  In subsequent
years, more generic scenarios will be modeled; a ranch house (one level) and a townhouse
(multi-level).

NIST is uniquely positioned to apply research to this problem that it developed for modeling fire and smoke
movement.  The fire science software tools, NIST FDS and Smokeview will be used as a basis for developing a
virtual reality based fire fighting training tool.  As a fire simulation unfolds a series of natural break point
will be encountered where the trainee will be asked questions such as: Should the window be opened?  Should the
door be closed? Should the hose stream be opened?  The simulation will continue based on the answers provided.
These questions will be designed to teach the trainee about tactics relevant to fire fighting such as the
effect ventilation on fire spread.    Since FDS cannot yet perform calculations in real time, simulations
will be pre-computed for all possible question outcomes.  Smokeview will be enhanced to be able to {\em jump}
from one scenario to another according to the trainee's responses.


\section{Numerical Fire Fighting Training Tools}
Two software tools are used to generate the Physics based fire fighter training scenarios
described in this document.  The first tool is the Fire Dynamics Simulator (FDS).
It performs numerical simulations and is a computational
fluid dynamics (CFD) model of fire-driven fluid flow. FDS solves a form of the
Navier-Stokes equations appropriate for low-speed, thermally-driven flow
with an emphasis on smoke and heat transport from fires.
These approximations allow the computations to be performed both accurately and efficiently.

The second tool is called Smokeview.  Smokeview {\em translates}\ data simulated by FDS into
various types of moving or animated images.  These images range in complexity from simple tracer particles,
to 2D shaded contours, 3D level surfaces or realistic looking smoke and fire.  It cannot be emphasized enough
that these images are not cartoons. they are drawn based on numerical results generated by FDS.

\section{Overview}
This publication documents a series of FDS test cases used as a basis for fire fighter training.
These test cases fall into two logical groups, those that are based on a full scale experiments and those that are meant to be demonstrations of fire phenomena encountered by fire fighters.

The FDS scenarios that are experimentally based are documented by 1) briefly describing the experiment with references to a more complete description, 2) briefly describing the numerical modeling that was performed giving the relevant input files and 3) giving a comparison of some of the experimental data collected with the corresponding numerical results.

\part{Demonstration Cases}
\chapter{Two Room}
\section{Description of Test Cases}
\section{Results}

\part{Experimental Cases}
\chapter{Flashover Trainer}
\section{Description of Experiment}
\note\ Need the following:
\begin{enumerate}
\item drawings containing measurements of facility,
\item photos,
\item test report describing
measurements taken, where they were taken
\item a cross-walk between measurements taken and where these measurements are recored (ie
in what column of a spread sheet file).
\end{enumerate}
\section{Description of FDS setup}
\section{Comparisons}
\chapter{Training Tower}
\section{Description of Experiment}
\section{Description of FDS setup}
\section{Comparisons}
\chapter{Ranch House}
\section{Description of Experiment (\note\ to be performed about 9/2008)}
\section{Description of FDS setup}
\section{Comparisons}

\bibliography{../Bibliography/FDS_refs,../Bibliography/FDS_general,../Bibliography/FDS_mathcomp}


\end{document}


