\chapter{Gas Species and Smoke}

For most applications, FDS uses a single step, mixing-controlled combustion model. The products of combustion are ``lumped'' together and tracked as a single gas mixture. These products include CO$_2$, H$_2$O, CO, and soot. However, in some cases, the combustion is incomplete due to a lack of oxygen. In others, a multiple-step reaction scheme is used to predict the production of CO. 

\section{Major Combustion Products, O$_2$ and CO$_2$}

For any hydrocarbon fuel, the major combustion products are oxygen and carbon dioxide. Accurate predictions of these gases requires knowledge of the chemical composition of the fuel and an accurate transport algorithm for the combustion products.


\clearpage

\subsection{WTC Experiments}

The following pages present comparisons of oxygen and carbon dioxide concentration predictions and measurements for the
WTC experiments. There was only one measurement of each made near the ceiling of the compartment roughly 2~m from the fire. 


\begin{figure}[h!]
\begin{tabular*}{\textwidth}{l@{\extracolsep{\fill}}r}
\includegraphics[height=2.2in]{FIGURES/WTC/WTC_01_Oxygen} &
\includegraphics[height=2.2in]{FIGURES/WTC/WTC_01_CO2} \\
\includegraphics[height=2.2in]{FIGURES/WTC/WTC_02_Oxygen} &
\includegraphics[height=2.2in]{FIGURES/WTC/WTC_02_CO2} \\
\includegraphics[height=2.2in]{FIGURES/WTC/WTC_03_Oxygen} &
\includegraphics[height=2.2in]{FIGURES/WTC/WTC_03_CO2}
\end{tabular*}
\label{NIST_WTC_Oxygen_CO2_1}
\end{figure}

\begin{figure}[p]
\begin{tabular*}{\textwidth}{l@{\extracolsep{\fill}}r}
\includegraphics[height=2.2in]{FIGURES/WTC/WTC_04_Oxygen} &
\includegraphics[height=2.2in]{FIGURES/WTC/WTC_04_CO2} \\
\includegraphics[height=2.2in]{FIGURES/WTC/WTC_05_Oxygen} &
\includegraphics[height=2.2in]{FIGURES/WTC/WTC_05_CO2} \\
\includegraphics[height=2.2in]{FIGURES/WTC/WTC_06_Oxygen} &
\includegraphics[height=2.2in]{FIGURES/WTC/WTC_06_CO2}
\end{tabular*}
\label{NIST_WTC_Oxygen_CO2_2}
\end{figure}


\clearpage

\subsection{NIST/NRC Experiments}

The following pages present comparisons of oxygen and carbon dioxide concentration predictions and measurements for the
NIST/NRC series. There were two oxygen measurements, one in the upper layer, one in the lower.  There was only one carbon
dioxide measurement in the upper layer.

\begin{figure}[h!]
\begin{tabular*}{\textwidth}{l@{\extracolsep{\fill}}r}
\includegraphics[height=2.2in]{FIGURES/NIST_NRC/NIST_NRC_17_Oxygen} &
\includegraphics[height=2.2in]{FIGURES/NIST_NRC/NIST_NRC_17_CO2} \\
\includegraphics[height=2.2in]{FIGURES/NIST_NRC/NIST_NRC_03_Oxygen} &
\includegraphics[height=2.2in]{FIGURES/NIST_NRC/NIST_NRC_03_CO2} \\
\includegraphics[height=2.2in]{FIGURES/NIST_NRC/NIST_NRC_09_Oxygen} &
\includegraphics[height=2.2in]{FIGURES/NIST_NRC/NIST_NRC_09_CO2} 
\end{tabular*}
\label{NIST_NRC_Gas_Open_1}
\end{figure}

\begin{figure}[p]
\begin{tabular*}{\textwidth}{l@{\extracolsep{\fill}}r}
\includegraphics[height=2.2in]{FIGURES/NIST_NRC/NIST_NRC_05_Oxygen} &
\includegraphics[height=2.2in]{FIGURES/NIST_NRC/NIST_NRC_05_CO2} \\
\includegraphics[height=2.2in]{FIGURES/NIST_NRC/NIST_NRC_14_Oxygen} &
\includegraphics[height=2.2in]{FIGURES/NIST_NRC/NIST_NRC_14_CO2} \\
\includegraphics[height=2.2in]{FIGURES/NIST_NRC/NIST_NRC_15_Oxygen} &
\includegraphics[height=2.2in]{FIGURES/NIST_NRC/NIST_NRC_15_CO2} \\
\includegraphics[height=2.2in]{FIGURES/NIST_NRC/NIST_NRC_18_Oxygen} &
\includegraphics[height=2.2in]{FIGURES/NIST_NRC/NIST_NRC_18_CO2}
\end{tabular*}
\label{NIST_NRC_Gas_Open_2}
\end{figure}

\begin{figure}[p]
\begin{tabular*}{\textwidth}{l@{\extracolsep{\fill}}r}
\includegraphics[height=2.2in]{FIGURES/NIST_NRC/NIST_NRC_01_Oxygen} &
\includegraphics[height=2.2in]{FIGURES/NIST_NRC/NIST_NRC_01_CO2} \\
\includegraphics[height=2.2in]{FIGURES/NIST_NRC/NIST_NRC_07_Oxygen} &
\includegraphics[height=2.2in]{FIGURES/NIST_NRC/NIST_NRC_07_CO2} \\
\includegraphics[height=2.2in]{FIGURES/NIST_NRC/NIST_NRC_02_Oxygen} &
\includegraphics[height=2.2in]{FIGURES/NIST_NRC/NIST_NRC_02_CO2} \\
\includegraphics[height=2.2in]{FIGURES/NIST_NRC/NIST_NRC_08_Oxygen} &
\includegraphics[height=2.2in]{FIGURES/NIST_NRC/NIST_NRC_08_CO2} 
\end{tabular*}
\label{NIST_NRC_Gas_Closed_1}
\end{figure}

\begin{figure}[p]
\begin{tabular*}{\textwidth}{l@{\extracolsep{\fill}}r}
\includegraphics[height=2.2in]{FIGURES/NIST_NRC/NIST_NRC_04_Oxygen} &
\includegraphics[height=2.2in]{FIGURES/NIST_NRC/NIST_NRC_04_CO2} \\
\includegraphics[height=2.2in]{FIGURES/NIST_NRC/NIST_NRC_10_Oxygen} &
\includegraphics[height=2.2in]{FIGURES/NIST_NRC/NIST_NRC_10_CO2} \\
\includegraphics[height=2.2in]{FIGURES/NIST_NRC/NIST_NRC_13_Oxygen} &
\includegraphics[height=2.2in]{FIGURES/NIST_NRC/NIST_NRC_13_CO2} \\
\includegraphics[height=2.2in]{FIGURES/NIST_NRC/NIST_NRC_16_Oxygen} &
\includegraphics[height=2.2in]{FIGURES/NIST_NRC/NIST_NRC_16_CO2}
\end{tabular*}
\label{NIST_NRC_Gas_Closed_2}
\end{figure}

\clearpage

\subsection{Summary of Major Combustion Products Predictions}


\begin{figure}[h!]
\begin{center}
\begin{tabular}{c}
\includegraphics[width=3.5in]{FIGURES/ScatterPlots/Carbon_Dioxide_Concentration} \\
\includegraphics[width=3.5in]{FIGURES/ScatterPlots/Oxygen_Concentration}\\
\end{tabular}
\end{center}
\caption[Summary of major gas species predictions]
{Summary of major gas species predictions.}
\end{figure}

\clearpage


\section{Smoke}

FDS treats smoke particulate in a similar way to other gaseous combustion products, basically a tracer gas whose production rate is a fixed fraction of the fuel consumption rate. However, there is an option in the model to allow smoke to deposit on solid surfaces, thus reducing its concentration in the product stream. By default, to model smoke movement it is only necessary to specify the soot yield, that is, the fraction of the fuel mass that is
converted to soot. However, additional parameters can be added to account for soot deposition. 

\subsection{NIST/NRC Experiments}

For the simulations of the NIST/NRC tests, the smoke yield is specified as one of the test parameters.
The figures on the following pages contain comparisons of measured and predicted smoke concentration at one measuring station in the upper layer.
There are two obvious trends in the figures: first, the predicted concentrations are about 50~\% higher than the measured
in the open door tests.  Second,
the predicted concentrations are roughly three times the measured concentrations in the closed door tests.
As a contrast, Figure displays the time history of CO concentration for 6 of the NIST/NRC tests.
Like smoke, the CO is specified in FDS via a fixed yield, measured along with smoke and reported in the test document.
The large differences between model and measurement seen in the smoke data do not appear in the CO data.

\begin{figure}[p]
\begin{tabular*}{\textwidth}{l@{\extracolsep{\fill}}r}
\includegraphics[height=2.2in]{FIGURES/NIST_NRC/NIST_NRC_01_Smoke} &
\includegraphics[height=2.2in]{FIGURES/NIST_NRC/NIST_NRC_07_Smoke} \\
\includegraphics[height=2.2in]{FIGURES/NIST_NRC/NIST_NRC_02_Smoke} &
\includegraphics[height=2.2in]{FIGURES/NIST_NRC/NIST_NRC_08_Smoke} \\
\includegraphics[height=2.2in]{FIGURES/NIST_NRC/NIST_NRC_04_Smoke} &
\includegraphics[height=2.2in]{FIGURES/NIST_NRC/NIST_NRC_10_Smoke} \\
\includegraphics[height=2.2in]{FIGURES/NIST_NRC/NIST_NRC_13_Smoke} &
\includegraphics[height=2.2in]{FIGURES/NIST_NRC/NIST_NRC_16_Smoke}
\end{tabular*}
\end{figure}

\begin{figure}[p]
\begin{tabular*}{\textwidth}{l@{\extracolsep{\fill}}r}
\includegraphics[height=2.2in]{FIGURES/NIST_NRC/NIST_NRC_17_Smoke} &
 \\
\includegraphics[height=2.2in]{FIGURES/NIST_NRC/NIST_NRC_03_Smoke} &
\includegraphics[height=2.2in]{FIGURES/NIST_NRC/NIST_NRC_09_Smoke} \\
\includegraphics[height=2.2in]{FIGURES/NIST_NRC/NIST_NRC_05_Smoke} &
\includegraphics[height=2.2in]{FIGURES/NIST_NRC/NIST_NRC_14_Smoke} \\
\includegraphics[height=2.2in]{FIGURES/NIST_NRC/NIST_NRC_15_Smoke} &
\includegraphics[height=2.2in]{FIGURES/NIST_NRC/NIST_NRC_18_Smoke}
\end{tabular*}
\end{figure}

\clearpage

\subsection{Summary of Smoke Concentration Predictions}

\begin{figure}[h!]
\begin{center}
\begin{tabular}{c}
\includegraphics[width=4.0in]{FIGURES/ScatterPlots/Smoke_Concentration} \\
\vspace{0.25in} \\
\end{tabular}
\end{center}
\caption[Summary of smoke concentration predictions]
{Summary of smoke concentration predictions.}
\end{figure}


\clearpage

\subsection{Sippola Aerosol Deposition Experiments}

A total of 16 aerosol deposition tests were conducted in a straight duct for 5 different
particle sizes and 3 different air velocities. In these cases, the aerosol species is tracked
explicitly and the aerosol deposition routines are enabled (refer to the Aerosol Deposition section
in the FDS User Guide~\cite{FDS_Users_Guide} and FDS Technical Reference Guide~\cite{FDS_Math_Guide}
for more details). A summary of the 16 tests is shown in Table~\ref{Sippola_Aerosol_Deposition_Summary}.

\begin{table}[h!]
\caption{Summary of Sippola aerosol deposition experiments selected for model validation.}
\begin{center}
\begin{tabular}{|c|c|c|c|c|c|}
\hline
Test No.  &  Air Speed (m/s)  &  Particle Diameter ($\mu$m)  \\ \hline \hline
1         &  2.2              &  1.0                         \\ \hline
2         &  2.2              &  2.8                         \\ \hline
3         &  2.1              &  5.2                         \\ \hline
4         &  2.2              &  9.1                         \\ \hline
5         &  2.2              &  16                          \\ \hline
6         &  5.3              &  1.0                         \\ \hline
7         &  5.2              &  1.0                         \\ \hline
8         &  5.2              &  3.1                         \\ \hline
9         &  5.4              &  5.2                         \\ \hline
10        &  5.3              &  9.8                         \\ \hline
11        &  5.3              &  16                          \\ \hline
12        &  9.0              &  1.0                         \\ \hline
13        &  9.0              &  3.1                         \\ \hline
14        &  8.8              &  5.4                         \\ \hline
15        &  9.2              &  8.7                         \\ \hline
16        &  9.1              &  15                          \\ \hline
\end{tabular}
\end{center}
\label{Sippola_Aerosol_Deposition_Summary}
\end{table}

The particle deposition velocity, $V_d$, was determined by

\be V_d = \frac{J_1 + J_2 + J_3 + J_4}{4 C_{avg}} \ee

where $J_1$ through $J_4$ are the deposition fluxes for duct panels 1 through 4 (kg/m$^2$-s)
given by

\be J_i = \frac{m_{d}}{A_d t} \ee

where $m_d$ is the mass of particles on the duct panel (kg), $A_d$ is the area of the duct panel (m$^2$),
and $t$ is the duration over which soot deposits on the plate (s). $C_{avg}$ is the average aerosol
concentration in the duct test section (kg/m$^3$) and is given by

\be C_{avg} = \frac{C_{up} + C_{down}}{2} \ee

Figure~\ref{Sippola_Aerosol_Deposition_Velocity} compares the measured versus predicted deposition velocities.

\begin{figure}[p]
\begin{center}
\begin{tabular}{c}
\includegraphics[height=2.2in]{FIGURES/Sippola_Aerosol_Deposition/Sippola_Aerosol_Ceiling_Deposition} \\
\includegraphics[height=2.2in]{FIGURES/Sippola_Aerosol_Deposition/Sippola_Aerosol_Wall_Deposition} \\
\includegraphics[height=2.2in]{FIGURES/Sippola_Aerosol_Deposition/Sippola_Aerosol_Floor_Deposition}
\end{tabular}
\end{center}
\caption[Summary of Sippola aerosol deposition cases]{Summary of Sippola aerosol deposition cases.}
\label{Sippola_Aerosol_Deposition_Velocity}
\end{figure}



\clearpage

\section{Products of Incomplete Combustion}

Predicting the concentration of products of incomplete combustion is challenging because it requires information about the chemical composition of the fuel and the multiple reactions that convert fuel to products. FDS contains a fairly general framework by which users can specify the reaction mechanism, and the examples in the following subsections highlight some of the more commonly used schemes.

\clearpage

\subsection{Smyth Slot Burner Experiment}

The two-step, CO production model in FDS was used to simulate a methane/air slot burner diffusion flame. Figures~\ref{Smyth_Slot_Burner_7} through \ref{Smyth_Slot_Burner_11} show predicted and measured temperatures at three elevations above the burner. Figures~\ref{Smyth_Slot_Burner_7} through \ref{Smyth_Slot_Burner_11} show predicted and measured concentrations of CH$_4$, O$_2$, CO,  and CO$_2$ at these same elevations. The reported uncertainty in the species concentration measurements ranges from 10~\% to 20~\%.

\begin{figure}[h!]
\begin{tabular*}{\textwidth}{l@{\extracolsep{\fill}}r}
\includegraphics[height=2.2in]{FIGURES/Smyth_Slot_Burner/Smyth_Slot_Burner_7mm_Temperature} &
\includegraphics[height=2.2in]{FIGURES/Smyth_Slot_Burner/Smyth_Slot_Burner_7mm_Fuel} \\
\includegraphics[height=2.2in]{FIGURES/Smyth_Slot_Burner/Smyth_Slot_Burner_7mm_Carbon_Dioxide} &
\includegraphics[height=2.2in]{FIGURES/Smyth_Slot_Burner/Smyth_Slot_Burner_7mm_Oxygen} \\
\includegraphics[height=2.2in]{FIGURES/Smyth_Slot_Burner/Smyth_Slot_Burner_7mm_Carbon_Monoxide} &
\end{tabular*}
\caption[Temperature and gas species predictions 7~mm above burner, Smyth experiment]
{Predicted and measured temperature and gas species 7~mm above a methane-air slot burner.}
\label{Smyth_Slot_Burner_7}
\end{figure}

\begin{figure}[p]
\begin{tabular*}{\textwidth}{l@{\extracolsep{\fill}}r}
\includegraphics[height=2.2in]{FIGURES/Smyth_Slot_Burner/Smyth_Slot_Burner_9mm_Temperature} &
\includegraphics[height=2.2in]{FIGURES/Smyth_Slot_Burner/Smyth_Slot_Burner_9mm_Fuel} \\
\includegraphics[height=2.2in]{FIGURES/Smyth_Slot_Burner/Smyth_Slot_Burner_9mm_Carbon_Dioxide} &
\includegraphics[height=2.2in]{FIGURES/Smyth_Slot_Burner/Smyth_Slot_Burner_9mm_Oxygen} \\
\includegraphics[height=2.2in]{FIGURES/Smyth_Slot_Burner/Smyth_Slot_Burner_9mm_Carbon_Monoxide} &
\end{tabular*}
\caption[Temperature and gas species predictions 9~mm above burner, Smyth experiment]
{Predicted and measured temperature and gas species 9~mm above a methane-air slot burner.}
\label{Smyth_Slot_Burner_9}
\end{figure}

\begin{figure}[p]
\begin{tabular*}{\textwidth}{l@{\extracolsep{\fill}}r}
\includegraphics[height=2.2in]{FIGURES/Smyth_Slot_Burner/Smyth_Slot_Burner_11mm_Temperature} &
\includegraphics[height=2.2in]{FIGURES/Smyth_Slot_Burner/Smyth_Slot_Burner_11mm_Fuel} \\
\includegraphics[height=2.2in]{FIGURES/Smyth_Slot_Burner/Smyth_Slot_Burner_11mm_Carbon_Dioxide} &
\includegraphics[height=2.2in]{FIGURES/Smyth_Slot_Burner/Smyth_Slot_Burner_11mm_Oxygen} \\
\includegraphics[height=2.2in]{FIGURES/Smyth_Slot_Burner/Smyth_Slot_Burner_11mm_Carbon_Monoxide} &
\end{tabular*}
\caption[Temperature and gas species predictions 11~mm above burner, Smyth experiment]
{Predicted and measured temperature and gas species 11~mm above a methane-air slot burner.}
\label{Smyth_Slot_Burner_11}
\end{figure}



\clearpage

\subsection{Beyler Hood Experiments}

Fig.~\ref{Beyler_Species} shows species predictions made by the two-step model compared with measured data for a range of fire sizes and burner positions.  The model outputs are the time-averaged species concentration at the hood exhaust vent whereas the experiment is the time-averaged species concentration downstream in the exhaust duct.

\begin{figure}[h!]
\begin{tabular*}{\textwidth}{l@{\extracolsep{\fill}}r}
\includegraphics[height=2.5in]{FIGURES/Beyler_Hood/Beyler_Hood_O2} &
\includegraphics[height=2.5in]{FIGURES/Beyler_Hood/Beyler_Hood_CO2} \\
\includegraphics[height=2.5in]{FIGURES/Beyler_Hood/Beyler_Hood_H2O} &
\includegraphics[height=2.5in]{FIGURES/Beyler_Hood/Beyler_Hood_CO} \\
\includegraphics[height=2.5in]{FIGURES/Beyler_Hood/Beyler_Hood_Soot} &
\includegraphics[height=2.5in]{FIGURES/Beyler_Hood/Beyler_Hood_UHC} \\
\end{tabular*}
\caption[Summary of gas species predictions, Beyler hood experiments]
{Comparison of measured and predicted species concentrations in the Beyler hood experiments}
\label{Beyler_Species}
\end{figure}

\clearpage

\subsection{NIST Reduced Scale Enclosure (RSE) Test Series, 1994}

The RSE natural gas experiments were selected to assess the CO production capability rather than soot production.
Nine fire sizes were simulated: 50~kW, 75~kW, 100~kW, 150~kW, 200~kW, 300~kW, 400~kW,
500~kW, and 600~kW.  The experiments were modeled using properties of the natural gas supplied to the test facility.
The model geometry included the compartment interior along with a 0.6~m deep region outside the door.
Figure~\ref{NIST_RSE_1994} shows the measured and predicted CO, CO$_2$ and O$_2$ concentrations.  The measured values are
from the test series performed by Bryner, Johnsson, and Pitts~\cite{Bryner:1}.

\begin{figure}[h!]
\begin{tabular*}{\textwidth}{l@{\extracolsep{\fill}}r}
\includegraphics[height=2.2in]{FIGURES/NIST_RSE_1994/NIST_RSE_1994_CO_Front} &
\includegraphics[height=2.2in]{FIGURES/NIST_RSE_1994/NIST_RSE_1994_CO_Rear} \\
\includegraphics[height=2.2in]{FIGURES/NIST_RSE_1994/NIST_RSE_1994_CO2_Front} &
\includegraphics[height=2.2in]{FIGURES/NIST_RSE_1994/NIST_RSE_1994_CO2_Rear} \\
\includegraphics[height=2.2in]{FIGURES/NIST_RSE_1994/NIST_RSE_1994_O2_Front} &
\includegraphics[height=2.2in]{FIGURES/NIST_RSE_1994/NIST_RSE_1994_O2_Rear}
\end{tabular*}
\caption[Summary of NIST Reduced Scale Enclosure experiments]{Summary of NIST Reduced Scale Enclosure experiments.}
\label{NIST_RSE_1994}
\end{figure}

