
\chapter{Burning Rate}

This chapter looks at validation exercises where the aim is to {\em predict} the burning rate of the fuel. Most of the simulations included in the previous chapters involved
a {\em specified} burning or heat release rate.


\section{FAA Polymers}

A non-charring polymer is considered one of the easier solids to model because it typically involves only a single, first order reaction that converts solid plastic to fuel vapor. 
No residue is formed and the plastic is completely pyrolyzed. Table~\ref{FAA_Properties}
lists nine parameters for each polymer studied. These values have been input directly into FDS, and the predicted burning rates are compared with measured values from the NIST
Gasification Apparatus, a device that pyrolyzes the solid in a nitrogen environment to prevent combustion of fuel gases. The results are shown in Fig.~\ref{FAA_Polymers}. The exposing
heat flux was 52~kW/m$^2$. A 1~cm layer of insulation was placed under the sample. Its properties are given in Ref.~\cite{Stoliarov:CF2009}.

\begin{table}[h!]
\caption[FAA Polymer Properties]{Input parameters for FAA Polymers non-charring examples. Courtesy S.~Stoliarov, M.~McKinnon and J.~Li, University of Maryland.}
\begin{tabular}{|l|c|c|c|c|c|l|l|}
\hline
Property                    & Units         & HDPE                  & HIPS                  & PMMA                  & Unc. (\%)  & Method                &  Ref.                    \\ \hline \hline
Density                     & kg/m$^3$      & 860                   & 950                   & 1100                  & 5     & Direct                &  \cite{Stoliarov:CF2009}  \\ \hline
Conductivity                & W/m/K         & 0.29                  & 0.22                  & 0.20                  & 15    & Thermoflixer          &  \cite{Stoliarov:CF2009}  \\ \hline
Specific Heat               & kJ/kg/K       & 3.5                   & 2.0                   & 2.2                   & 15    & DSC                   &  \cite{Stoliarov:PDS2008}  \\ \hline
Emissivity                  &               & 0.92                  & 0.86                  & 0.85                  & 20    & Sphere                &  \cite{Hallman:PES1974}  \\ \hline
Absorption Coef.            & m$^{-1}$      & 1300                  & 2700                  & 2700                  & 50    & FTIR                  &  \cite{Tsilingiris:ECM2003}  \\ \hline
Pre-Exp.~Factor             & s$^{-1}$      & $4.8 \times 10^{22}$  & $1.2 \times 10^{16}$  & $8.5 \times 10^{12}$  & 50    & TGA                   &  \cite{Stoliarov:CF2009}  \\ \hline
Activation Energy           & kJ/kmol       & $3.49 \times 10^{5}$  & $2.47 \times 10^{5}$  & $1.88 \times 10^{5}$  & 3     & TGA                   &  \cite{Stoliarov:CF2009}  \\ \hline
Heat of Reaction            & kJ/kg         & 920                   & 1000                  & 870                   & 15    & DSC                   &  \cite{Stoliarov:PDS2008}  \\ \hline
\end{tabular}
\label{FAA_Properties}
\end{table}

\begin{tabbing}
Direct  \hspace{0.5in}     \= Direct measurement of mass and volume \\
Thermoflixer               \> Transient line source method \\
DSC                        \> Differential Scanning Calorimetry \\
Sphere                     \> Integrating sphere \\
FTIR                       \> Fourier Transform Infrared Spectroscopy \\
TGA                        \> Thermogravimetric Analysis
\end{tabbing}

\newpage

\begin{figure}[p]
\begin{center}
\begin{tabular}{c}
\includegraphics[height=2.2in]{FIGURES/FAA_Polymers/FAA_Polymers_HDPE} \\
\includegraphics[height=2.2in]{FIGURES/FAA_Polymers/FAA_Polymers_HIPS} \\
\includegraphics[height=2.2in]{FIGURES/FAA_Polymers/FAA_Polymers_PMMA}
\end{tabular}
\end{center}
\caption[Results of FAA Polymers comparison]{Comparison of predicted and measured mass loss rates for three non-charring polymers exposed to a heat flux of 52~kW/m$^2$ in a
nitrogen environment.}
\label{FAA_Polymers}
\end{figure}

