\chapter{Nomenclature}
\label{nomenclature}

\begin{tabbing}
$A_s$ \hspace{1in}        \= droplet surface area \\
$A_{\alpha\beta}$          \> pre-exponential factor for solid phase Arrhenius reaction \\
$B$                       \> pre-exponential factor for gas phase Arrhenius reaction \\
$C$                       \> Sprinkler C-Factor \\
$C_D$                     \> drag coefficient \\
$C_s$                     \> Smagorinsky constant (LES)  \\
$c_s$             \> Solid material specific heat \\
$c_p$                     \> constant pressure specific heat \\
$D$                       \> diffusion coefficient   \\
$d_m$                     \> median volumetric droplet diameter \\
$E$                       \> activation energy \\
$\bof_b$                  \> external force vector (excluding gravity) \\
$g$                       \> acceleration of gravity \\
$\bg$                     \> gravity vector, normally $(0,0,-g)$ \\
$\cal H$                  \> total pressure divided by the density \\
$H_{r,\alpha\beta}$   \> heat of reaction for a solid phase reaction \\
$h$                       \> enthalpy; heat transfer coefficient   \\
$h_\alpha$                \> enthalpy of species $\alpha$   \\
$h_\alpha^0$              \> heat of formation of species $\alpha$   \\
$I$                       \> radiation intensity   \\
$I_b$                     \> radiation blackbody intensity   \\
$k$                       \> thermal conductivity; suppression decay factor \\
$\dm_{b,\alpha}'''$       \> mass production rate of species $\alpha$ by evaporating droplets/particles \\
$\dm_f''$                 \> fuel mass flux \\
$\dm_\alpha'''$           \> mass production rate of species $\alpha$ per unit volume \\
$\dm_w''$                 \> water mass flux  \\
$m_w''$                   \> water mass per unit area \\
$\NU$                     \> Nusselt number \\
$\PR$                     \> Prandtl number \\
$p$                       \> pressure \\
$\bp_0$                   \> atmospheric pressure profile \\
$\bp_m$                   \> background pressure of $m$th pressure zone \\
$\tp$                     \> pressure perturbation \\
$\dbq''$                  \> heat flux vector \\
$\dq'''$                  \> heat release rate per unit volume \\
$\dq_r''$                 \> radiative flux to a solid surface \\
$\dq_c''$                 \> convective flux to a solid surface \\
$\dQ$                     \> total heat release rate \\
$Q^*$                     \> characteristic fire size \\
$\R$                      \> universal gas constant \\
$\RE$                     \> Reynolds number \\
$r_d$                     \> droplet radius \\
$r_{\alpha\beta}$     \> solid phase reaction rate \\
$\hbox{RTI}$              \> Response Time Index of sprinkler \\
$\bs$                     \> unit vector in direction of radiation intensity\\
$\SC$                     \> Schmidt number \\
$\SH$                     \> Sherwood number \\
$S_\alpha$        \> solid component production rate \\
$T$                       \> temperature \\
$t$                       \> time           \\
$U$                       \> integrated radiant intensity \\
$\bu=(u,v,w)$             \> velocity vector  \\
$W_\alpha$                \> molecular weight of gas species $\alpha$ \\
$\bW$                     \> molecular weight of the gas mixture \\
$\WE$                     \> Weber number \\
$\bx=(x,y,z)$             \> position vector  \\
$X_\alpha$                \> volume fraction of species $\alpha$   \\
$Y_\alpha$                \> mass fraction of species $\alpha$   \\
$Y_\OTWO^\infty$          \> mass fraction of oxygen in the ambient   \\
$Y_\F^I$                  \> mass fraction of fuel in the fuel stream   \\
$y_s$                     \> soot yield \\
$Z$                       \> mixture fraction   \\
$Z_f$                     \> stoichiometric value of the mixture fraction   \\
$\gamma$                  \> ratio of specific heats; Rosin-Rammler exponent \\
$\Delta H$                \> heat of combustion \\
$\Delta H_\OTWO$          \> energy released per unit mass oxygen consumed \\
$\delta$                  \> wall thickness \\
$\epsilon$                \> dissipation rate \\
$\kappa$                  \> absorption coefficient \\
$\mu$                     \> dynamic viscosity \\
$\nu_\alpha$              \> stoichiometric coefficient, species $\alpha$ \\
$\nu_s$           \> yield of solid residue in solid phase reaction \\
$\nu_g,\gamma$    \> yield of gaseous species $\gamma$ in solid phase reaction \\
$\rho$                    \> density \\
$\btau_{ij}$              \> viscous stress tensor \\
$\chi_r$                  \> radiative loss fraction \\
$\sigma$                  \> Stefan-Boltzmann constant; constant in droplet size distribution; surface tension \\
$\sigma_d$                \> droplet scattering coefficient \\
$\sigma_s$                \> scattering coefficient \\
$\bo=(\omx,\omy,\omz)$    \> vorticity vector \\
\end{tabbing}





\chapter{Derivation of the Velocity Divergence Constraint}
\label{app_divergence}

In this appendix we derive the divergence of the velocity field as presented in Eq.~(\ref{eqn_divfromeos}). Note that the constitutive relationships presented here for the mass diffusion and thermal heat fluxes are valid for direct numerical simulations (i.e., well-resolved calculations). The minor modifications required of the transport coefficients for large-eddy simulation are presented in Section~\ref{LES}.  We start the derivation by rearranging the continuity equation. Next, we differentiate the equation of state to reveal the relationship between the transport equations for mass and energy. We then show how the transport equations may be combined to yield the velocity divergence constraint.  In the last section we present the final result in FDS notation.

\subsubsection{Continuity Equation}
\label{continuity}

Let $\rho$ denote the fluid mass density; let $\mathbf{u} = [u,v,w]^T$ denote the fluid mass-average velocity; and let $\dot{m}_b^{\tripleprime}$
denote a bulk source of mass per unit volume (which may come from the evaporation of water droplets, for example).
The continuity equation may be rearranged to yield the following divergence constraint on the velocity
\begin{equation}
\label{eqn_divconstraint1}
\Div \mathbf{u} = \frac{1}{\rho} \left( \dot{m}_b^{\tripleprime} -  \frac{\mbox{D} \rho}{\mbox{D} t} \right)
\end{equation}
where $\mbox{D}(\,\,\,)/\mbox{D} t \equiv \partial (\,\,\,)/\partial t + \mathbf{u}\cdot\nabla(\,\,\,)$ is the material derivative.

\subsubsection{Equation of State}
\label{EOS}

We consider the transport of $N_s$ primitive species mass fractions $Y_\alpha$ for $\alpha = \{1,\ldots,N_s\}$, $N_s-1$ of which are independent.  Later we will discuss code modifications for lumped species (Sec.~\ref{sec_lumped_species}). The molecular weight of a given species is denoted $W_\alpha$ and the molecular weight of the mixture, $\overline{W}$, is given by
\begin{equation}
\label{eqn_mixmolewt}
\overline{W} = \left( \sum_{\alpha} \frac{Y_\alpha}{W_\alpha} \right)^{-1}
\end{equation}
where as a shorthand notation, which is used throughout this document, we write $\sum_\alpha$ for $\sum_{\alpha = 1}^{N_s}$. Let $\overline{p}_i(\mathbf{x},t)$ denote the hydrostatic pressure in the $i$th zone of the domain, which in general we take to be a function of space and time. In practice, however, $\overline{p}_i = \overline{p}_i(t)$ for closed (i.e., sealed or pressurized) domains and $\overline{p}_i = \overline{p}_i(z)$, where $z$ represents the coordinate aligned with the gravity vector, for large, open domains (e.g., forest fires large enough to interact with the stratified atmosphere). The divergence constraint derived below is based on the ideal gas equation of state (EOS), which, for low-Mach flows, we write as
\begin{equation}
\label{eqn_idealgaslaw}
\overline{p}_i = \frac{\rho \mathcal{R} T}{\overline{W}}
\end{equation}
where $\mathcal{R} = 8.3145$~kJ/(kmol K) is the gas law constant.

Differentiating the EOS (\ref{eqn_idealgaslaw}) we obtain
\begin{equation}
\label{eqn_DEOS1}
\frac{\mbox{D} \overline{p}_i}{\mbox{D} t} = \rho \mathcal{R} T \frac{\mbox{D}}{\mbox{D} t}\left(\frac{1}{\overline{W}}\right) +
\frac{\rho \mathcal{R}}{\overline{W}} \frac{\mbox{D} T}{\mbox{D} t} + \frac{\mathcal{R}T}{\overline{W}} \frac{\mbox{D} \rho}{\mbox{D} t}
\end{equation}
which rearranges to
\begin{equation}
\label{eqn_DEOS2}
\frac{\mbox{D} \rho}{\mbox{D} t} = \frac{\overline{W}}{\mathcal{R}T} \frac{\mbox{D}\overline{p}_i}{\mbox{D} t} -
\rho \overline{W} \frac{\mbox{D}}{\mbox{D} t}\left(\frac{1}{\overline{W}}\right) - \frac{\rho}{T} \frac{\mbox{D} T}{\mbox{D} t}
\end{equation}


\subsubsection{Species Transport Equation}
\label{species_transport}

The species transport equation plays a role in both the second and third terms on the RHS of (\ref{eqn_DEOS2}).  Including the bulk mass source, the evolution of species mass fractions is governed by
\begin{equation}
\label{eqn_speciestransport}
\frac{\partial \left(\rho Y_\alpha\right)}{\partial t} + \Div \left(\rho Y_\alpha \mathbf{u}\right)  = - \Div \mathbf{J}_{\alpha} + \dot{m}_\alpha^{\tripleprime} + \dot{m}_{b,\alpha}^{\tripleprime}
\end{equation}
where $\mathbf{J}_{\alpha}$ is the diffusive mass flux vector for species $\alpha$ (relative to the mass-average velocity),
$\dot{m}_\alpha^{\tripleprime}$ is the chemical mass production rate of $\alpha$ per unit volume [kg-$\alpha$ produced /(m$^3$ s)],
and $\dot{m}_{b,\alpha}^{\tripleprime}$ is the bulk mass source of $\alpha$ per unit volume [kg-$\alpha$ introduced /(m$^3$ s)].  Note that
\begin{equation}
\label{eqn_bulksum}
\sum_\alpha \dot{m}_{b,\alpha}^{\tripleprime} = \dot{m}_b^{\tripleprime}
\end{equation}
and
\begin{equation}
\label{eqn_massconservation}
\sum_\alpha \dot{m}_\alpha^{\tripleprime} = 0
\end{equation}
Additionally, by construction, the $i$th component of the species diffusive fluxes sum to zero,
\begin{equation}
\label{eqn_sumdiffflux}
\sum_\alpha J_{\alpha,i} = 0
\end{equation}
Thus, as must be the case, summing (\ref{eqn_speciestransport}) over $\alpha$ yields the continuity equation.

It is convenient to work in terms of the material derivative of the mass fraction.  Care must be exercised in obtaining this expression because the continuity equation contains a bulk source term.
Expanding (\ref{eqn_speciestransport}) we obtain
\begin{eqnarray}
\label{eqn_speciesexpansion}
\rho\frac{\partial Y_\alpha}{\partial t} + Y_\alpha \frac{\partial \rho}{\partial t} + \rho \mathbf{u}\cdot \nabla Y_\alpha + Y_\alpha \Div \left(\rho \mathbf{u}\right) &=&
- \Div \mathbf{J}_{\alpha} + \dot{m}_\alpha^{\tripleprime} + \dot{m}_{b,\alpha}^{\tripleprime} \nonumber\\ [0.3cm]
\rho \frac{\mbox{D} Y_\alpha}{\mbox{D} t} + Y_\alpha \underbrace{\left[ \frac{\partial \rho}{\partial t} + \Div \left(\rho \mathbf{u}\right) \right]}_{\displaystyle \dot{m}_b^{\tripleprime}} &=& \mbox{}
\end{eqnarray}
Thus, the material derivative of the mass fraction can be written as
\begin{equation}
\label{eqn_matdermassfrac}
\frac{\mbox{D} Y_\alpha}{\mbox{D} t} = \frac{1}{\rho}\left(  \dot{m}_\alpha^{\tripleprime} + \dot{m}_{b,\alpha}^{\tripleprime} -
Y_\alpha \dot{m}_b^{\tripleprime}  - \Div \mathbf{J}_{\alpha} \right)
= \frac{1}{\rho}\left(  \dot{m}_\alpha^{\tripleprime} + \dot{m}_{b}^{\tripleprime}[Y_{b,\alpha} - Y_\alpha] - \Div \mathbf{J}_{\alpha} \right)
\end{equation}
where in the second step we use the identity $\dot{m}_{b,\alpha}^{\tripleprime} = Y_{b,\alpha}\,\dot{m}_{b}^{\tripleprime}$ with $Y_{b,\alpha}$ being the mass fraction of $\alpha$ in the bulk prior to its introduction into the fluid mixture.

Utilizing (\ref{eqn_mixmolewt}) and (\ref{eqn_matdermassfrac}) we obtain
\begin{eqnarray}
\label{eqn_transportexpression}
\frac{\mbox{D}}{\mbox{D} t}\left(\frac{1}{\overline{W}}\right) &=& \frac{\mbox{D}}{\mbox{D} t}\left(\sum_\alpha \frac{Y_\alpha}{W_\alpha} \right) \nonumber\\ [0.3cm]
&=& \sum_\alpha \frac{1}{W_\alpha} \frac{\mbox{D}Y_\alpha }{\mbox{D} t} \nonumber\\ [0.3cm]
&=& \frac{1}{\rho} \sum_\alpha \frac{1}{W_\alpha} \left(  \dot{m}_\alpha^{\tripleprime} + \dot{m}_{b}^{\tripleprime}[Y_{b,\alpha} - Y_\alpha] - \Div \mathbf{J}_{\alpha} \right)
\end{eqnarray}
which is needed in the second term on the RHS of (\ref{eqn_DEOS2}).


\subsubsection{Sensible Enthalpy Transport Equation}
\label{enthalpy_definitions}

The specific sensible enthalpy of species $\alpha$ is
\begin{equation}
\label{eqn_sensible}
h_{s,\alpha}(T) = h_{0,\alpha} + \int_{T_0}^{T} c_{p,\alpha}(T^\prime) \,\dif T^\prime \,\mbox{,}
\end{equation}
where $h_{0,\alpha}$ is the reference enthalpy at $T_0$ and the specific heat is
\begin{equation}
\label{eqn_specificheat}
c_{p,\alpha} \equiv \frac{\partial h_{s,\alpha}}{\partial T} \,\mbox{.}
\end{equation}
The specific sensible enthalpy of the mixture is then given by
\begin{equation}
\label{eqn_chemsensmix}
h_s(\mathbf{Y},T) = \sum_\alpha Y_\alpha h_{s,\alpha}(T) \,\mbox{.}
\end{equation}

Neglecting viscous heating and the effect of the fluctuating pressure on dilation work (both assumptions are valid for low-Mach flows), the transport equation for the sensible enthalpy is
\begin{equation}
\label{eqn_enthalpytransport}
\rho \frac{\mbox{D} h_s }{\mbox{D} t} = -\sum_\alpha \Delta h^\circ_\alpha \dot{m}_{\alpha}^{\tripleprime} + \frac{\mbox{D} \overline{p}_i }{\mbox{D} t} - \Div \dot{\mathbf{q}}^{\prime\prime} - \dq_b''' + \dot{m}_{b}^{\tripleprime} \left[ ( h_{s,b} - h_s ) + \mbox{$\frac{1}{2}$}\left|\mathbf{u}_b-\mathbf{u}\right|^2 \right]
\end{equation}
where $\Delta h^\circ_\alpha$ is the standard heat of formation of $\alpha$, $h_{s,b}$ is the specific sensible enthalpy of the bulk mass source \emph{prior} to evaporation (does not include the heat of vaporization), $\dot{q}_b^\tripleprime$ is a volumetric heat sink due to convective heat transfer to the bulk phase, and $\dot{\mathbf{q}}^{\prime\prime}$ is the heat flux vector which contains contributions from conduction, molecular diffusion of sensible enthalpy, and radiation,
\begin{equation}
\label{eqn_heatflux}
\dot{\mathbf{q}}^{\prime\prime} = -k \nabla T + \sum_\alpha h_{s,\alpha} \mathbf{J}_{\alpha} + \dot{\mathbf{q}}^{\prime\prime}_{r}
\end{equation}
Here $k$ is the thermal conductivity of the mixture and $\dot{\mathbf{q}}^{\prime\prime}_{r}$ is the radiant heat flux.  The last term in (\ref{eqn_enthalpytransport}) accounts for the kinetic energy associated with the instantaneous mixing of the bulk and gas-phase momentum.


\subsubsection{Relating Enthalpy, Temperature, and Species}
\label{eqn_enthalpy_temperature}

Using the chain rule of calculus, we may expand the derivative of the sensible enthalpy $h_s(\mathbf{Y},T)$ to obtain
\begin{equation}
\label{eqn_chainrule}
\frac{\mbox{D} h_s}{\mbox{D} t} = \left(\frac{\partial h_s}{\partial T}\right) \frac{\mbox{D} T }{\mbox{D} t} +
\sum_\alpha \left( \frac{\partial h_s}{\partial Y_\alpha} \right) \frac{\mbox{D} Y_\alpha }{\mbox{D} t}
\end{equation}
Note that since $h_s = \sum_\alpha Y_\alpha h_{s,\alpha}$ we have
\begin{equation}
\label{eqn_dhdY}
\frac{\partial h_s}{\partial Y_\alpha} = \frac{\partial}{\partial Y_\alpha} \sum_\beta (Y_\beta h_{s,\beta} )
= \sum_\beta h_{s,\beta} \,\delta_{\alpha \beta} = h_{s,\alpha}
\end{equation}
where $\delta_{\alpha \beta}$ is the Kronecker delta. Also,
\begin{equation}
\label{eqn_dhdT}
\frac{\partial h_s}{\partial T} = \frac{\partial}{\partial T} \sum_\alpha Y_\alpha h_{s,\alpha} =
\sum_\alpha Y_\alpha \left(\frac{\partial h_{s,\alpha}}{\partial T}\right) = \sum_\alpha Y_\alpha c_{p,\alpha} \equiv c_p
\end{equation}
defining the specific heat of the mixture.  Thus, by rearranging (\ref{eqn_chainrule}) and utilizing (\ref{eqn_dhdY}) and (\ref{eqn_dhdT}) we obtain
\begin{equation}
\label{eqn_DTDt1}
\frac{\mbox{D} T}{\mbox{D} t} = \frac{1}{c_p} \left[ \frac{\mbox{D} h_s}{\mbox{D} t} - \sum_\alpha h_{s,\alpha} \frac{\mbox{D} Y_\alpha}{\mbox{D} t} \right]
\end{equation}
Utilizing (\ref{eqn_matdermassfrac}) and (\ref{eqn_enthalpytransport}) in (\ref{eqn_DTDt1}) yields
\begin{eqnarray}
\frac{\mbox{D} T}{\mbox{D} t} &=& \frac{1}{\rho c_p}   \left[  -\sum_\alpha \Delta h^\circ_\alpha \dot{m}_{\alpha}^{\tripleprime} + \frac{\mbox{D} \overline{p}_i }{\mbox{D} t} - \Div\dot{\mathbf{q}}^{\prime\prime} - \dot{q}_b^\tripleprime + \dot{m}_{b}^{\tripleprime} \left[ ( h_{s,b} - h_s )  + \mbox{$\frac{1}{2}$} \left|\mathbf{u}_b-\mathbf{u}\right|^2 \right] \right.   \nonumber\\ [0.3cm]
& &  \quad\quad\,\, \left. - \sum_\alpha h_{s,\alpha} \left( \dot{m}_\alpha^{\tripleprime} + \dot{m}_{b}^{\tripleprime}[Y_{b,\alpha} - Y_\alpha] - \Div \mathbf{J}_{\alpha} \right)  \right]
\label{eqn_DTDt2}
\end{eqnarray}
The following combination of terms may be rearranged as follows:
\begin{eqnarray}
\label{eqn_latent_heat}
\dot{m}_{b}^{\tripleprime} ( h_{s,b} - h_s ) - \sum_\alpha h_{s,\alpha} \dot{m}_{b}^{\tripleprime}[Y_{b,\alpha} - Y_\alpha] &=&  \dot{m}_{b}^{\tripleprime} \left[ ( h_{s,b} - h_s ) - \sum_\alpha h_{s,\alpha} (Y_{b,\alpha} - Y_\alpha) \right] \nonumber\\ [.3cm]
&=&  \dot{m}_{b}^{\tripleprime} \left[ ( h_{s,b} - h_s ) - (\sum_\alpha h_{s,\alpha} Y_{b,\alpha}) + h_s \right] \nonumber\\  [.3cm]
&=&  \dot{m}_{b}^{\tripleprime} \left[ \sum_\alpha Y_{b,\alpha} (h_{s,b,\alpha}-h_{s,\alpha}) \right] \nonumber\\  [.3cm]
&=&  -\dot{m}_{b}^{\tripleprime} \left[ \sum_\alpha Y_{b,\alpha} \int_{T_b}^{T} c_{p,\alpha}(T^\prime) \,\mbox{d}T^\prime \right] \,\mbox{.}
\end{eqnarray}
\vspace{0.3cm}

\noindent In going from the third to the fourth step in (\ref{eqn_latent_heat}), note that $h_{s,b,\alpha}$ is the enthalpy of the bulk \emph{vapor} and already contains the heat of vaporization.  The minus sign on the RHS in the last step says that the heat needed to raise the temperature of the bulk mass from $T_b$ to $T$ must be subtracted from the local gas mixture.  Note that $T_b$ is \emph{not} the droplet temperature. [It is not clear how we handle this in the code.]

The temperature equation may now be written as
\begin{eqnarray}
\label{eqn_DTDt}
\frac{\mbox{D} T}{\mbox{D} t} &=& \frac{1}{\rho c_p} \left[\frac{\mbox{D} \overline{p}_i }{\mbox{D} t}
- \Div\dot{\mathbf{q}}^{\prime\prime} - \sum_\alpha \Delta h^\circ_\alpha \dot{m}_\alpha^{\tripleprime} - \sum_\alpha h_{s,\alpha}\left(  \dot{m}_\alpha^{\tripleprime} - \Div \mathbf{J}_{\alpha} \right) \right. \nonumber\\ [0.3cm] 
& & \left. - \dot{q}_b^\tripleprime -\dot{m}_{b}^{\tripleprime} \left\{\sum_\alpha Y_{b,\alpha} \int_{T_b}^{T} c_{p,\alpha}(T^\prime) \,\mbox{d}T^\prime + \mbox{$\frac{1}{2}$}\left|\mathbf{u}_b-\mathbf{u}\right|^2 \right\} \right] \,\mbox{.}
\end{eqnarray}

\subsubsection{Assembling Terms}
\label{putting_it_all_together}

We now have all the pieces we need to construct the divergence constraint which we introduced in Eq. (\ref{eqn_divconstraint1}).  Using (\ref{eqn_DEOS2}) in (\ref{eqn_divconstraint1}) we obtain
\begin{eqnarray}
\label{eqn_divconstraint2}
\Div\mathbf{u} &=& \frac{1}{\rho} \left( \dot{m}_b^{\tripleprime} -  \left[ \frac{\overline{W}}{\mathcal{R}T} \frac{\mbox{D}\overline{p}_i}{\mbox{D} t} -
\rho \overline{W} \frac{\mbox{D}}{\mbox{D} t}\left(\frac{1}{\overline{W}}\right) - \frac{\rho}{T} \frac{\mbox{D} T}{\mbox{D} t} \right] \right)  \nonumber\\ [0.3cm]
&=& \frac{1}{\rho} \,\dot{m}_b^\tripleprime -  \frac{1}{\overline{p}_i} \frac{\mbox{D}\overline{p}_i}{\mbox{D} t} + \overline{W} \frac{\mbox{D}}{\mbox{D} t}\left(\frac{1}{\overline{W}}\right) +
\frac{1}{T} \frac{\mbox{D} T}{\mbox{D} t}
\end{eqnarray}
where in the second step the EOS (\ref{eqn_idealgaslaw}) is used to simplify the second term on the RHS.
Using (\ref{eqn_transportexpression}) and (\ref{eqn_DTDt}) in (\ref{eqn_divconstraint2}) yields
\begin{eqnarray}
\label{eqn_divconstraint3}
\Div\mathbf{u} &=& \frac{1}{\rho} \,\dot{m}_b^{\tripleprime} \nonumber\\ [0.3cm]
&-& \frac{1}{\overline{p}_i} \frac{\mbox{D}\overline{p}_i}{\mbox{D} t} \nonumber\\ [0.3cm]
&+& \overline{W} \left[\frac{1}{\rho} \sum_\alpha \frac{1}{W_\alpha} \left(  \dot{m}_\alpha^{\tripleprime} + \dot{m}_{b}^{\tripleprime}[Y_{b,\alpha} - Y_\alpha] - \Div \mathbf{J}_{\alpha} \right) \right] \nonumber\\ [0.3cm]
&+& \frac{1}{T} \left[ \frac{1}{\rho c_p} \left( \frac{\mbox{D} \overline{p}_i }{\mbox{D} t} - \Div\dot{\mathbf{q}}^{\prime\prime}  - \sum_\alpha \Delta h^\circ_\alpha \dot{m}_\alpha^{\tripleprime} - \sum_\alpha h_{s,\alpha}\left(  \dot{m}_\alpha^{\tripleprime} - \Div \mathbf{J}_{\alpha} \right) \right. \right. \nonumber\\ [0.3cm]
& & \left. \left. \hspace{0.5cm} - \dot{q}_b^\tripleprime -\dot{m}_{b}^{\tripleprime} \left\{\sum_\alpha Y_{b,\alpha} \int_{T_b}^{T} c_{p,\alpha}(T^\prime) \,\mbox{d}T^\prime  + \mbox{$\frac{1}{2}$}\left|\mathbf{u}_b-\mathbf{u}\right|^2 \right\} \right) \right] \,\mbox{.} 
\end{eqnarray}
\vspace{0.3cm}

\noindent Note that $\overline{W} \sum_\alpha (Y_\alpha/W_\alpha) = 1$ and also $\overline{W} \sum_\alpha (Y_{b,\alpha}/W_\alpha) = \overline{W}/\overline{W}_b$,
where $\overline{W}_b$ is the molecular weight of the bulk mixture prior to its introduction into the fluid mixture.  Equation (\ref{eqn_divconstraint3}) thus simplifies to
\begin{eqnarray}
\label{eqn_divconstraint_final}
\Div\mathbf{u}  &=& \left(\frac{1}{\rho c_p T}  -  \frac{1}{\overline{p}_i}\right) \frac{\mbox{D}\overline{p}_i}{\mbox{D} t} + \frac{1}{\rho} \left[ \dot{m}_b^{\tripleprime} \frac{\overline{W}}{\overline{W}_b} +  \sum_\alpha \left( \frac{\overline{W}}{W_\alpha} - \frac{h_{s,\alpha}}{c_p T} \right)  \left(  \dot{m}_\alpha^{\tripleprime}
- \Div\mathbf{J}_{\alpha} \right) \right] \nonumber \\ [0.3cm]
&+&  \frac{1}{\rho c_p T} \left[ - \sum_\alpha \Delta h^\circ_\alpha \dot{m}_\alpha^{\tripleprime} - \Div \dot{\mathbf{q}}^{\prime\prime} - \dot{q}_b^\tripleprime -\dot{m}_{b}^{\tripleprime} \left\{\sum_\alpha Y_{b,\alpha} \int_{T_b}^{T} c_{p,\alpha}(T^\prime) \,\mbox{d}T^\prime + \mbox{$\frac{1}{2}$}\left|\mathbf{u}_b-\mathbf{u}\right|^2 \right\} \right] \,\mbox{.} \nonumber\\ [0.3cm]
\end{eqnarray}


\subsubsection{FDS Notation}
\label{fds_notation}

The following relationships are used to rearrange (\ref{eqn_divconstraint_final}) into the form shown in the FDS Technical Reference Guide.

As discussed in Sec.~\ref{sec_lumped_species}, FDS stores arrays and solves transport equations for $N_z$ independent lumped species.  The background species (usually air) is given a computational index of 0.  Because of this convention, we find it computationally convenient to make the following rearrangements to the summation terms:
\begin{eqnarray}
\label{eqn_hdiff_1}
\sum_\alpha h_{s,\alpha} \mathbf{J}_\alpha &=& h_{s,0} \mathbf{J}_0  + \sum_{n=1}^{N_z} h_{s,n} \mathbf{J}_n  \nonumber\\ [0.3cm]
&=& h_{s,0} \left(-\sum_{n=1}^{N_z} \mathbf{J}_n \right) + \sum_{n=1}^{N_z} h_{s,n} \mathbf{J}_n  \nonumber\\ [0.3cm]
&=& \sum_{n=1}^{N_z} (h_{s,n}-h_{s,0}) \mathbf{J}_n
\end{eqnarray}
\begin{eqnarray}
\label{eqn_hdiff_2}
\sum_\alpha \left( \frac{\overline{W}}{W_\alpha} - \frac{h_{s,\alpha}}{c_p T} \right)  \left(\dot{m}_\alpha^{\tripleprime} - \Div\mathbf{J}_{\alpha} \right)
&=& \left( \frac{\overline{W}}{W_0} - \frac{h_{s,0}}{c_p T} \right)  \left(\dot{m}_0^\tripleprime - \Div\mathbf{J}_0 \right) + \sum_{n=1}^{N_z} \left( \frac{\overline{W}}{W_n} - \frac{h_{s,n}}{c_p T} \right)  \left(\dot{m}_n^\tripleprime - \Div\mathbf{J}_n \right) \nonumber\\ [0.3cm]
&=& \left( \frac{\overline{W}}{W_0} - \frac{h_{s,0}}{c_p T} \right)  \left((-\sum_{n=1}^{N_z}\dot{m}_n^\tripleprime) - \Div(-\sum_{n=1}^{N_z}\mathbf{J}_n) \right) + \sum_{n=1}^{N_z} \left( \frac{\overline{W}}{W_n} - \frac{h_{s,n}}{c_p T} \right)  \left(\dot{m}_n^\tripleprime - \Div\mathbf{J}_n \right) \nonumber\\ [0.3cm]
&=& \sum_{n=1}^{N_z} \left[ \overline{W}\left(\frac{1}{W_n} - \frac{1}{W_0}\right) - \frac{(h_{s,n}-h_{s,0})}{c_p T} \right]  \left(\dot{m}_n^\tripleprime - \Div\mathbf{J}_n \right)
\end{eqnarray}
\vspace{0.3cm}

\noindent We employ the binary form of Fick's law using mixture-averaged diffusivities $D_n$ as a constitutive relation for the diffusive flux (summation over repeated suffixes \emph{not} implied),
\begin{equation}
\label{eqn_fickslaw}
\mathbf{J}_n = - \rho D_n \nabla Y_n \,\mbox{.}
\end{equation}
The heat release rate per unit volume is defined by
\begin{equation}
\label{eqn_heatrelease}
\dot{q}^\tripleprime \equiv -\sum_\alpha \dot{m}_{\alpha}^\tripleprime \Delta h^\circ_\alpha \,\mbox{.}
\end{equation}
Taking the $z$ direction to be aligned with the gravity vector we have $\partial \overline{p}_i/\partial z = \rho_i g_z$,
where $g_z = -9.8$ m/s$^2$ and $\rho_i(z)$ is a specified background density for the $i$th zone.
Thus, the material derivative of the background pressure may be written as
\begin{eqnarray}
\label{eqn_expandp0}
\frac{\mbox{D}\overline{p}_i}{\mbox{D}t} &=& \frac{\partial \overline{p}_i}{\partial t} + w \rho_i g_z \,\mbox{.}
\end{eqnarray}
Utilizing (\ref{eqn_fickslaw}), (\ref{eqn_heatrelease}), and (\ref{eqn_expandp0}), for the $i$th zone we may write the divergence (\ref{eqn_divconstraint_final}) as
\begin{equation}
\label{eqn_fds_divcontraint}
\Div \mathbf{u} = \mathcal{D} - \mathcal{P} \frac{\partial \overline{p}_i}{\partial t}
\end{equation}
where
\begin{equation}
\label{eqn_fdsP}
\mathcal{P} =  \frac{1}{\overline{p}_i}-\frac{1}{\rho \,c_p T}
\end{equation}
and
\begin{eqnarray}
\label{eqn_fdsD}
\mathcal{D} &=& \frac{1}{\rho c_p T} \left[ \Div \sum_{n=1}^{N_z} (h_{s,n}-h_{s,0}) \rho D_n \nabla Y_n +  \Div(k \nabla T)  + \dot{q}^\tripleprime - \Div\dot{\mathbf{q}}^{\prime\prime}_{r} \right] \nonumber\\ [.3cm]
&+& \frac{1}{\rho}\sum_{n=1}^{N_z} \left[\overline{W}\left(\frac{1}{W_n} - \frac{1}{W_0}\right) - \frac{(h_{s,n}-h_{s,0})}{c_p T} \right] \Div \rho D_n \nabla Y_n \nonumber\\ [.3cm]
&+& \underbrace{\frac{1}{\rho}\sum_{n=1}^{N_z} \left[\overline{W}\left(\frac{1}{W_n} - \frac{1}{W_0}\right) - \frac{(h_{s,n}-h_{s,0})}{c_p T} \right] \dot{m}_n^\tripleprime}_{\mbox{\ct D\_REACTION}} \nonumber\\ [.3cm]
&+& \underbrace{\frac{\dot{m}_b^\tripleprime}{\rho}\frac{\overline{W}}{\overline{W}_b} + \frac{1}{\rho c_p T} \left[ - \dot{q}_b^\tripleprime -\dot{m}_{b}^{\tripleprime} \left\{\sum_\alpha Y_{b,\alpha} \left( h_{s,\alpha}(T)-h_{s,\alpha}(T_b)\right) + \mbox{$\frac{1}{2}$}\left|\mathbf{u}_b-\mathbf{u}\right|^2 \right\}  \right]}_{\mbox{\ct D\_LAGRANGIAN}} \nonumber\\ [.3cm]
&-& \mathcal{P} w \rho_i g_z
\end{eqnarray}
\vspace{0.3cm}

Equations (\ref{eqn_fdsP}) and (\ref{eqn_fdsD}) correspond to (\ref{eqn_fdsP1}) and (\ref{eqn_fdsD1}) in the FDS Tech Guide.  Though, note that at present the bulk kinetic energy term is not included in the code.

\paragraph{Remark} The definition of the heat flux vector $\dot{\mathbf{q}}^{\prime\prime}$ (\ref{eqn_heatflux}) can be a source of confusion. When transporting the sensible enthalpy $h_{s,\alpha}$ only, as we are doing here, the heat flux vector does not account for the molecular transport of the chemical enthalpy (the enthalpy of formation).  If we were working in terms of the [chemical + sensible] enthalpy $h_\alpha = \Delta h_\alpha^\circ + h_{s,\alpha}$ we would not have a ``heat of reaction'' $\dot{q}^{\tripleprime}$, the heat flux vector would account for the transport of both the chemical and the sensible enthalpy, and the sensible enthalpy in (\ref{eqn_fdsD}) would be replaced by the [chemical + sensible] enthalpy, $h_{\alpha}$.




\chapter{A Simple Model of Flame Extinction}

\subsubsection{Frederick W. Mowrer, Department of Fire Protection Engineering, University of Maryland}

\label{mowrer_model}


A diffusion flame immersed in a vitiated atmosphere will extinguish before consuming all the
available oxygen from the atmosphere.  The classic example of this behavior is a candle burning
within an inverted jar.  This same concept has been applied within FDS
to determine the conditions under which the local ambient oxygen concentration will no longer
support a diffusion flame.  In this appendix, the critical adiabatic flame temperature
concept is used to estimate the local ambient oxygen concentration at which extinction will
occur.

Consider a control volume characterized
by a bulk temperature, $T_m$, a mass, $m$, an average specific heat, $\overline{c_p}$, and an oxygen mass
fraction, $Y_\OTWO$.  Complete combustion of the oxygen within the control volume would release a
quantity of energy given by:
\be
   Q = m \, Y_\OTWO \, \left( \frac{\Delta H}{r_\OTWO}  \right)  \label{bbb}
\ee
where $\Delta H/r_\OTWO$ has a relatively constant value of
approximately 13100~kJ/kg for most fuels of interest for fire applications.\footnote{C. Huggett, ``Estimation of the Rate of Heat Release by Means of Oxygen
Consumption,'' {\em Fire and Materials}, Vol.~12, pp.~61-65, 1980.}
Under adiabatic conditions, the energy released by combustion of the available oxygen within
the control volume would raise the bulk temperature of the gases within the control volume by an
amount equal to:
\be
   Q = m \, \overline{c_p} \, (T_f - T_m)  \label{eee}
\ee
The average specific heat of the gases within the control volume can be calculated based on the
composition of the combustion products as:
\be
   \overline{c_p} = \frac{1}{(T_f-T_m)} \, \sum_\alpha \int_{T_m}^{T_f} c_{p,\alpha} (T) \, dT
\ee
To simplify the analysis, the combustion products are assumed to have an average specific heat
of 1.2~kJ/kg/K over the temperature range of interest, a value similar to that of nitrogen, the primary
component of the products.
The relationship between the oxygen mass fraction within the control volume and the adiabatic
temperature rise of the control volume is evaluated by equating Eqs.~(\ref{bbb}) and (\ref{eee}):
\be
   Y_\OTWO = \frac{ \overline{c_p} (T_f-T_m) }{\Delta H/r_\OTWO}
\ee
If the critical adiabatic flame temperature is assumed to have a constant value of approximately
1700~K for hydrocarbon diffusion flames, as suggested by Beyler,\footnote{C. Beyler, ``Flammability Limits of Premixed and Diffusion Flames,''
{\em SFPE Handbook of Fire Protection Engineering} (3rd Ed.), National Fire
Protection Association, Quincy, MA, 2003.} then the relationship
between the limiting oxygen mass fraction and the bulk temperature of a control volume is given
by:
\be
   Y_{\OTWO,lim} = \frac{ \overline{c_p} (T_{f,lim}-T_m) }{\Delta H/r_\OTWO} \approx  \frac{ 1.2 \, (1700-T_m) }{ 13100}  \label{extinction_model}
\ee
For a control volume at a temperature of 300~K, i.e., near room temperature, the limiting oxygen
mass fraction would evaluate to $Y_{\OTWO,lim}=0.128$.  This value is consistent with the measurements
of Morehart, Zukoski and Kubota,\footnote{Morehart, J., Zukoski, E., and Kubota, T., ``Characteristics of Large Diffusion Flames
Burning in a Vitiated Atmosphere,'' {\em Third International Symposium on Fire Safety
Science}, Elsevier Science Publishers, pp.~575-583, 1991.} who measured the oxygen concentration at extinction of flames by dilution of air
with combustion products. They found that flames self-extinguished at oxygen concentrations of 12.4~\% to 14.3~\%. Note that their results
are expressed as volume, not mass, fractions. Beyler's chapter in the SFPE Handbook references other researchers who measured oxygen
concentrations at extinction ranging from 12~\% to 15~\%. The default value in FDS is 15~\%.



\chapter{Derivation of the Werner Wengle Wall Model}
%\subsubsection{R. McDermott, BFRL}
\label{app_WWderivation}

To obtain (\ref{eqn_tauwturb}) we take the first off-wall value of the streamwise velocity to be
\begin{equation}
\label{eqn_meanu}
\tilde{u} = \frac{1}{\Delta z} \int_0^{\Delta z} u(z) \,\mbox{d}z \,\mbox{,}
\end{equation}
and then substitute the WW profile for $u(z)$ and integrate.

Let $z_m$ denote the dimensional distance from wall where $z^+ = 11.81$.  Equation (\ref{eqn_meanu}) becomes
\begin{eqnarray}
\label{eqn_uint}
\tilde{u} &=& \frac{1}{\Delta z} \left[ \int_0^{z_m} u(z) \,\mbox{d}z + \int_{z_m}^{\Delta z} u(z) \,\mbox{d}z \right] \,\mbox{,} \nonumber\vspace{0.2cm}\\
&=& \frac{1}{\Delta z} \left[ \int_0^{z_m} u^+ u^* \,\mbox{d}z + \int_{z_m}^{\Delta z} u^+ u^* \,\mbox{d}z \right] \,\mbox{,} \nonumber\vspace{0.2cm}\\
&=& \frac{1}{\Delta z} \left[ \int_0^{z_m} z^+ u^* \,\mbox{d}z + \int_{z_m}^{\Delta z} A(z^+)^B u^* \,\mbox{d}z \right] \,\mbox{,} \nonumber\vspace{0.2cm}\\
&=& \frac{1}{\Delta z} \left[ \int_0^{z_m} \frac{z}{\ell} u^* \,\mbox{d}z + \int_{z_m}^{\Delta z} A\left(\frac{z}{\ell}\right)^B u^* \,\mbox{d}z \right] \,\mbox{,} \nonumber\vspace{0.2cm}\\
&=& \frac{1}{\Delta z} \left[ \int_0^{z_m} \frac{z\bar{\rho}u^*}{\bar{\mu}} u^* \,\mbox{d}z + \int_{z_m}^{\Delta z} A\left(\frac{z\bar{\rho}u^*}{\bar{\mu}}\right)^B u^* \,\mbox{d}z \right] \,\mbox{,} \nonumber\vspace{0.2cm}\\
&=& \frac{1}{\Delta z} \left[ \underbrace{\int_0^{z_m} \frac{\tau_w}{\bar{\mu}} z\,\mbox{d}z}_{I} + \underbrace{\int_{z_m}^{\Delta z} A\left(\frac{\bar{\rho}}{\bar{\mu}}\right)^B \left(\frac{\tau_w}{\bar{\rho}}\right)^{\frac{1+B}{2}} z^B\,\mbox{d}z}_{II} \right] \,\mbox{.}
\end{eqnarray}

We will integrate $I$ and $II$ separately.  First, however, we must find a way to eliminate the unknown $z_m$.  To do this we equate (\ref{eqn_wwlam}) and (\ref{eqn_wwturb}) at the point where the viscous and power law regions intersect, i.e. $z^+ = 11.81 \equiv z_m^+ = z_m \bar{\rho}u^*/\bar{\mu}$.
\begin{eqnarray}
\label{eqn_derivezm}
u^+(z_m^+) = A(z_m^+)^B &=& z_m^+ \nonumber\\
A &=& (z_m^+)^{1-B} \nonumber\\
A^{\frac{1}{1-B}} &=& z_m^+ = \frac{z_m\bar{\rho}u^*}{\bar{\mu}} \nonumber\\
z_m &=& \frac{\bar{\mu}A^{\frac{1}{1-B}}}{\bar{\rho}u^*} \nonumber\\
z_m &=& \frac{(\bar{\mu}/\bar{\rho})A^{\frac{1}{1-B}}}{\sqrt{\tau_w/\bar{\rho}}} \,\mbox{.}
\end{eqnarray}
We now have $z_m$ in terms of $\tau_w$ and otherwise known values.

Integrating section $I$ of (\ref{eqn_uint}) we find
\begin{eqnarray}
\label{eqn_intI}
\int_0^{z_m} \frac{\tau_w}{\bar{\mu}} z\,\mbox{d}z &=& \frac{\tau_w}{2\bar{\mu}} \left[ z^2 \right]_0^{z_m} \nonumber\\
&=& \frac{\tau_w}{2\bar{\mu}} z_m^2 \nonumber\\
&=& \frac{\tau_w}{2\bar{\mu}} \frac{(\bar{\mu}/\bar{\rho})^2A^{\frac{2}{1-B}}}{\tau_w/\bar{\rho}} \nonumber\\
&=& \frac{\bar{\mu} A^{\frac{2}{1-B}}}{2\bar{\rho}} \,\mbox{.}
\end{eqnarray}

Integrating section $II$ yields
\begin{eqnarray}
\label{eqn_intII}
\int_{z_m}^{\Delta z} A\left(\frac{\bar{\rho}}{\bar{\mu}}\right)^B \left(\frac{\tau_w}{\bar{\rho}}\right)^{\frac{1+B}{2}} z^B\,\mbox{d}z &=& \left\{A\left(\frac{\bar{\rho}}{\bar{\mu}}\right)^B \left(\frac{\tau_w}{\bar{\rho}}\right)^{\frac{1+B}{2}}\right\} \frac{1}{1+B} \left[z^{1+B}\right]_{z_m}^{\Delta z} \nonumber\\
&=& \left\{ \quad \right\} \frac{1}{1+B} \left[{\Delta z}^{1+B} - {z_m}^{1+B}\right] \nonumber\\
&=& \left\{ \quad \right\} \frac{1}{1+B} \left[{\Delta z}^{1+B} - \left(\frac{(\bar{\mu}/\bar{\rho})A^{\frac{1}{1-B}}}{\sqrt{\tau_w/\bar{\rho}}}\right)^{1+B}\right] \nonumber\\
&=& \left\{ \frac{A}{1+B} \left(\frac{\bar{\rho}}{\bar{\mu}}\right)^B \left(\frac{\tau_w}{\bar{\rho}}\right)^{\frac{1+B}{2}} \right\} \left[{\Delta z}^{1+B} - \frac{(\bar{\mu}/\bar{\rho})^{1+B} A^{\frac{1+B}{1-B}}}{\left(\frac{\tau_w}{\bar{\rho}}\right)^{\frac{1+B}{2}}}\right] \nonumber\\
&=& \frac{A}{1+B} \left(\frac{\bar{\rho}}{\bar{\mu}}\right)^B \left(\frac{\tau_w}{\bar{\rho}}\right)^{\frac{1+B}{2}} {\Delta z}^{1+B} - \frac{(\bar{\mu}/\bar{\rho})}{1+B} A^{\frac{2}{1-B}} \,\mbox{.}
\end{eqnarray}

Plugging (\ref{eqn_intI}) and (\ref{eqn_intII}) back into (\ref{eqn_uint}) gives
\begin{eqnarray}
\label{eqn_combineint}
\tilde{u} &=& \frac{1}{\Delta z} \left[ \frac{\bar{\mu} A^{\frac{2}{1-B}}}{2\bar{\rho}} + \frac{A}{1+B} \left(\frac{\bar{\rho}}{\bar{\mu}}\right)^B \left(\frac{\tau_w}{\bar{\rho}}\right)^{\frac{1+B}{2}} {\Delta z}^{1+B} - \frac{(\bar{\mu}/\bar{\rho})}{1+B} A^{\frac{2}{1-B}} \right] \nonumber\\
&=& \frac{1}{2} \left(\frac{\bar{\mu}}{\bar{\rho}\Delta z}\right) A^{\frac{2}{1-B}} - \frac{1}{1+B} \left(\frac{\bar{\mu}}{\bar{\rho}\Delta z}\right) A^{\frac{2}{1-B}} + \frac{A}{1+B} \left(\frac{\bar{\rho}\Delta z}{\bar{\mu}}\right)^B \left(\frac{\tau_w}{\bar{\rho}}\right)^{\frac{1+B}{2}} \,\mbox{.}
\end{eqnarray}
Rearranging for $\tau_w$ we find
\begin{eqnarray}
\label{eqn_rearrangefortauw}
\left(\frac{\tau_w}{\bar{\rho}}\right)^{\frac{1+B}{2}} &=& \frac{1+B}{A}\left(\frac{\bar{\mu}}{\bar{\rho}\Delta z}\right)^B \left[ \left( \frac{1}{1+B} - \frac{1}{2}\right)\left(\frac{\bar{\mu}}{\bar{\rho}\Delta z}\right)A^{\frac{2}{1-B}} + \tilde{U} \right] \nonumber\\
&=& \frac{1-B}{2} A^{\frac{1+B}{1-B}} \left(\frac{\bar{\mu}}{\bar{\rho}\Delta z}\right)^{1+B}  + \frac{1+B}{A} \left(\frac{\bar{\mu}}{\bar{\rho}\Delta z }\right)^B \tilde{U} \nonumber\\
\tau_w &=& \bar{\rho} \left[ \frac{1-B}{2} A^{\frac{1+B}{1-B}} \left(\frac{\bar{\mu}}{\bar{\rho}\Delta z}\right)^{1+B}  + \frac{1+B}{A} \left(\frac{\bar{\mu}}{\bar{\rho}\Delta z }\right)^B \tilde{u} \right]^{\frac{2}{1+B}} \,\mbox{,}
\end{eqnarray}
which corresponds to Eq. (9.46) in \cite{Sagaut:2001}.



\chapter{Scalar Boundedness Correction}
\label{app_boundedness}

Second-order central differencing of the advection term in the scalar transport equation leads to dispersion errors (spurious wiggles) and these errors, if left untreated, can lead to scalar fields which are physically not realizable, e.g., negative densities.  To prevent this, FDS employs a boundedness correction to the scalar fields after the explicit transport step.  The correction, which we describe below, acts locally and effectively adds the minimum amount of diffusion necessary to prevent boundedness violations.  It is stressed that this correction does not make the scalar transport scheme total variation diminishing (TVD); it only serves to correct for boundedness. Similar schemes are employed by others (see e.g. \cite{Herrmann:2005}).

By default, FDS employs a TVD transport scheme (Superbee \cite{Roe:1986} for LES and CHARM \cite{Zhou:1995} for DNS). These TVD schemes are applied during the transport step and each can be shown to be TVD in 1D under certain CFL constraints.  However, except for Godunov's scheme ({\ct FLUX\_LIMITER=1}), the TVD proofs do not extend to 3D \cite{Toro}.  Still, these schemes do a much better job than pure central differencing at mitigating dispersion error.  Note that even though TVD schemes are applied, by default FDS still runs through the boundedness check in case any small violations are not prevented by the flux limiter.

\subsubsection{A simple case}

For simplicity we start by considering a minimum boundedness violation for density in 1D.  That is, somewhere we have $\rho < \rho_{min}$.  Let $\rho_i^*$ denote the resulting density from the explicit transport step for cell $i$ with volume $V_i$.  Our goal is to find a correction $\delta \rho_i$ which:
\begin{enumerate}[{(}a{)}]
\item satisfies boundedness, $\rho_i = \rho_i^* + \delta \rho_i \ge \rho_{min}$ for all $i$
\item conserves mass, $\sum_i \delta \rho_i V_i = 0$
\item minimizes data variation, $\sum_i |\delta \rho_i|$ is minimized (i.e., we change the field as little as possible)
\end{enumerate}

As mentioned, the basic idea is to apply a linear smoothing operator $\cal L$ to the density field in regions where boundedness violations have occurred. So, the correction may be viewed as an explicit diffusion step applied to the uncorrected field with diffusion coefficient $c$:
\begin{equation}
\rho = \rho^* + c {\cal L} \rho^*
\end{equation}
To make matters simple, let us envision for the moment that the density in cell $i$ is negative but that the densities in cells $i-1$ and $i+1$ are both safely in bounds (this actually is what happens most of the time with dispersion error).  We therefore want a correction that takes mass away from $i-1$ and $i+1$ and moves it to $i$ to make up the deficit.  We know that for cell $i$ the minimum change in mass and therefore the minimum correction that will satisfy boundedness is $\delta \rho_i = \rho_{min} - \rho_i^*$.  The operator $\cal L$ takes the form of the standard discrete Laplacian.  The correction for cell $i$ is simply
\begin{eqnarray}
\label{eqn_rhocor}
\rho_i &=& \rho_i^* + \delta \rho_i \nonumber\\
&=& \rho_i^* + \rho_{min} - \rho_i^* \nonumber\\
&=&  \rho_i^* + c_i (\rho_{i-1}^* - 2 \rho_i^* + \rho_{i+1}^*)
\end{eqnarray}
Comparing the second and third lines, we find that the diffusion coefficient is given by
\begin{equation}
\label{eqn_diffcoef}
c_i = \frac{\rho_{min} - \rho_i^*}{\rho_{i-1}^* - 2 \rho_i^* + \rho_{i+1}^*}
\end{equation}
Based on the third line of (\ref{eqn_rhocor}), the correction for cell $i$ may be thought of as the sum to two mass fluxes from its neighboring cells.  The change in mass of cell $i$ is $\delta m_i = \delta \rho_i V_i$ and is balanced by changes in mass for cells $i-1$ and $i+1$:
\begin{eqnarray}
\delta m_{i-1} &=& - c_i (\rho_{i-1}^* - \rho_i^*) V_i \nonumber\\
\delta m_{i+1} &=& - c_i (\rho_{i+1}^* - \rho_i^*) V_i \nonumber
\end{eqnarray}
In this case the sum of the mass corrections is zero, as desired:
\begin{eqnarray}
\sum_{j=i-1}^{i+1} \delta m_j &=& \delta \rho_{i-1} V_{i-1} + \delta \rho_i V_i + \delta \rho_{i+1}V_{i+1} \nonumber\\
&=& - c_i (\rho_{i-1}^* - \rho_i^*) V_i + c_i (\rho_{i-1}^* - 2 \rho_i^* + \rho_{i+1}^*) V_i - c_i (\rho_{i+1}^* - \rho_i^*) V_i \nonumber\\
&=& 0 \nonumber
\end{eqnarray}

\subsubsection{Realistic cases}

The discussion above was to provide a simple case for understanding the basic idea behind the correction method.  In a realistic case we must account for multi-dimensional aspects of the problem and for the possibility that neighboring cells may both be out of bounds.  Consider a grid cell whose
density is outside the specified range. Denote this cell with a ``$c$'' for center. Its volume is $V_c$ and density is $\rho_c^*$, obtained from the transport scheme.  Let the subscript ``$n$'' denote any of the six neighboring cells (in other words, only include cells which share a face with cell $c$).  We want to correct any boundedness violations for the  cell $c$ by shifting mass to or from its neighboring cells $n$:
\begin{equation}
\label{eqn_rhocor2}
\rho_c = \rho_c^* + \delta \rho_c \quad ; \quad \rho_n = \rho_n^* + \delta \rho_n
\end{equation}
We first define the total amount of mass we wish to shift:
\be m_c = | \rho^*_c - \rho_{\rm cut} | \, V_c  \ee
where $\rho_{\rm cut}$ is the appropriate upper or lower bound of the density.
The amount of mass each neighboring cell can accommodate without falling outside the range is:
\be m_n = \Big| \min \Big[ \rho_{\max} , \max[\rho_{\min},\rho_n^*] \Big] - \rho_{\rm cut} \Big| \, V_n \ee
The correction terms that guarantee mass conservation ($V_c \, \delta \rho_c = - \sum V_n \, \delta \rho_n$) are:
\be
\label{eqn_rhomn}
\delta \rho_c = \pm \min \left[ m_c , \sum m_n \right] / V_c  \quad ; \quad
\delta \rho_n = \mp \min \left[ \frac{m_c}{\sum m_n} , 1 \right] m_n/V_n
\ee

Next, to correct species mass fractions that are out of bounds, we follow the exact same procedure.
\begin{equation}
\label{eqn_rhocor2}
Z_c = Z_c^* + \delta Z_c \quad ; \quad Z_n = Z_n^* + \delta Z_n
\end{equation}
We define the amount of species mass we wish to shift:
\be m_c = | Z^*_c - Z_{\rm cut} | \, \rho_c \, V_c  \ee
where $Z_{\rm cut}$ is either 0 or 1.
The amount of species mass each neighboring cell can accommodate without falling outside the range is:
\be m_n = \Big| \min \Big[ 1 , \max[0,Z_n^*] \Big] - Z_{\rm cut} \Big| \, \rho_n \, V_n \ee
The correction terms that guarantee mass conservation ($V_c \, \rho_c \, \delta Z_c = - \sum V_n \, \rho_n \, \delta Z_n$) are:
\be
\label{eqn_Zmn}
\delta Z_c = \pm \min \left[ m_c , \sum m_n \right] / (\rho_c \, V_c)  \quad ; \quad
\delta Z_n = \mp \min \left[ \frac{m_c}{\sum m_n} , 1 \right] m_n/(\rho_n \, V_n)
\ee


%\subsubsection{An alternate view by R. McDermott}
%
%The discussion above was to provide a simple case for understanding the basic idea behind the correction method.  In a realistic case we must account for multi-dimensional aspects of the problem and for the possibility that neighboring cells may both be out of bounds.  Here again we examine the case of a minimum density boundedness violation. Consider the cell $i$ in a 3D flow with volume $V_i$ and density $\rho_i^*$ obtained from the transport scheme.  Let ${\sf N}$ denote the set of cells containing $i$ and its neighbors, excluding diagonal neighbors (in other words, only include cells which share a face with $i$). We want to correct any boundedness violations for the $i$th cell via
%\begin{equation}
%\label{eqn_rhocor2}
%\rho_i = \rho_i^* + \delta \rho_i
%\end{equation}
%
%To determine $\delta \rho_i$ we must consider the mass exchange between neighboring cells. Let $\delta \rho_{ji}$ denote the density change for cell $j$ in ${\sf N}$ due to a boundedness violation in $i$.  For example, if $\rho_i^* < \rho_{min}$ then the density in $i$ will increase by drawing mass from its neighbors (if the mass is available).  The mass exchange matrix is given by
%\begin{equation}
%\label{eqn_rhoij}
%\delta \rho_{ji} = \left\{ \begin{array}{ll}  \displaystyle \max(0,\rho_{min} - \rho_i^*) & \mbox{if} \quad j=i  \\ \displaystyle -c_i( \max[\rho_{min},\rho_j^*] - \max[\rho_{min},\rho_i^*]) & \mbox{if} \quad j\ne i \end{array} \right.
%\end{equation}
%The smoothing parameter in (\ref{eqn_rhoij}) is obtained from
%\begin{equation}
%\label{eqn_reali}
%c_i = \frac{\max(0,\rho_{min} - \rho_i^*)}{ \sum_{j, j\ne i} ( \max[\rho_{min},\rho_j^*] - \max[\rho_{min},\rho_i^*] )}
%\end{equation}
%
%Mass conservation is obeyed because the mass increase in cell $i$ is balanced by a mass decrease by its neighbors.  In other words, the columns of $\delta \rho_{ji}$ sum to zero.  Note, however, that the mass exchange matrix is not symmetric.  The row sum gives the final mass correction for the $i$th cell:
%\begin{equation}
%\label{eqn_sumdrho}
%\delta \rho_i = \sum_j \delta \rho_{ij}
%\end{equation}



\chapter{The Dynamic Smagorinsky Model}
%\subsubsection{R. McDermott}
\label{app_dynsmag}

The ``subgrid-scale'' (SGS) stress, which accounts for momentum transport by unresolved eddies, emerges from decomposition of the advection term when deriving the LES equations.  It is defined as
\begin{equation}
\label{eqn_tau_sgs}
\tau_{ij}^{sgs} \equiv \bar{\rho}(\widetilde{u_i u_j} - \tilde{u}_i \tilde{u}_j) \,\mbox{.}
\end{equation}
The deviatoric (trace free) part of the SGS stress is modeled by gradient diffusion in analogy with the viscous stress,
\begin{equation}
\label{eqn_tau_sgs_deviatoric}
\tau_{ij}^{sgs} - \frac{1}{3}\tau_{kk}^{sgs} \equiv \tau_{ij}^{sgs,D} = -2 \mu_t \left(\tilde{S}_{ij} - \frac{1}{3}\tilde{S}_{kk}\delta_{ij}\right) =  -2 \mu_t \left(\tilde{S}_{ij} - \frac{1}{3} (\nabla\!\cdot\tilde{\mathbf{u}}) \delta_{ij}\right) \,\mbox{.}
\end{equation}
In FDS, the turbulent viscosity is obtained from the Smagorinsky model,
\begin{equation}
\label{eqn_mu_turb}
\mu_t = \bar{\rho}(C_s \Delta)^2 |\tilde{S}| \,\mbox{,}
\end{equation}
where $C_s$ is the model constant and $\Delta$ is the filter width taken as the geometric average of the local mesh spacing; for example, in 3D, $\Delta = (\delta x \,\delta y \,\delta z)^{1/3}$.  Note that the quantity $(C_s \Delta)$ is the local ``mixing length'' and that $|\tilde{S}|$ provides the time scale for turbulent diffusion.

In preparation for the dynamic procedure, we rewrite the model for the deviatoric SGS stress as
\begin{equation}
\label{eqn_tau_sgs_deviatoric2}
\tau_{ij}^{sgs,D} = -2 (C_s \Delta)^2 \beta_{ij} \,\mbox{,}
\end{equation}
defining
\begin{equation}
\label{eqn_beta}
\beta_{ij} = \bar{\rho} |\tilde{S}| \left(\tilde{S}_{ij} - \frac{1}{3} \tilde{S}_{kk} \delta_{ij} \right)  \,\mbox{.}
\end{equation}


\subsubsection{The Dynamic Procedure}

We will now discuss the dynamic procedure for determining $C_s$, the Smagorinsky constant.  The procedure itself is a series of explicit filtering operations leading to a simple algebraic relationship for $C_s(\mathbf{x},t)$ (see Eq.~(\ref{eqn_lengthscale}) below).  The FDS implementation basically follows the works of Germano et al. \cite{Germano:1991}, Moin et al. \cite{Moin:1991}, and Pino Martin et al. \cite{PinoMartin:2000}.

To derive the procedure, first, we apply a ``test'' filter of width $\hat{\Delta}>\Delta$ to the LES equations to obtain
\begin{equation}
\label{eqn_testfiltns}
\frac{\partial \widehat{\overline{\rho u_i}}}{\partial t} + \frac{\partial \widehat{\overline{\rho u_i u_j}}}{\partial x_j} = -\frac{\partial \widehat{\overline{\sigma}}_{ij}}{\partial x_j} \,\mbox{,}
\end{equation}
where $\sigma_{ij}$ is the total stress tensor. The $\,\,\breve{}\,\,$ is adopted from Pino Martin et al.~\cite{PinoMartin:2000} for the Favre test filter, $\widehat{\overline{\rho}} \breve{ \widetilde{u} } \equiv \widehat{ \overline{ \rho u }}$, allowing us to rewrite Eq.~(\ref{eqn_testfiltns}) as
\begin{eqnarray}
\label{eqn_testfavrefiltns}
\frac{\partial \widehat{\overline{\rho}} \breve{\widetilde{u}}_i}{\partial t} + \frac{\partial \widehat{\overline{\rho}} \breve{\widetilde{u_i u_j}}}{\partial x_j} &=& -\frac{\partial \widehat{\overline{\sigma}}_{ij}}{\partial x_j} \,\mbox{,} \nonumber\\
\frac{\partial \widehat{\overline{\rho}} \breve{\widetilde{u}}_i}{\partial t} + \frac{\partial \widehat{\overline{\rho}} \breve{\widetilde{u}}_i \breve{\widetilde{u}}_j}{\partial x_j} &=& -\frac{\partial \widehat{\overline{\sigma}}_{ij}}{\partial x_j} - \frac{\partial T_{ij}}{\partial x_j}\,\mbox{,}
\end{eqnarray}
where the ``subtest'' stress is defined as
\begin{equation}
\label{eqn_subteststress}
T_{ij} \equiv \widehat{\overline{\rho}} \left( \breve{\widetilde{u_i u_j}} - \breve{\widetilde{u}}_i \breve{\widetilde{u}}_j \right) \,\mbox{.}
\end{equation}

The deviatoric part of the subtest stress is modeled as,
\begin{equation}
\label{eqn_devtest}
T_{ij} - \frac{1}{3}T_{kk}\delta_{ij} \equiv T_{ij}^D = -2 \left( C_s \widehat{\Delta} \right)^2 \widehat{\overline{\rho}} |\breve{\widetilde{S}}|\left(\breve{\widetilde{S}}_{ij} - \frac{1}{3} \breve{\widetilde{S}}_{kk}\delta_{ij}\right)  \,\mbox{.}
\end{equation}
By applying the Germano identity \cite{Germano:1991}, we obtain the Leonard stress,
\begin{eqnarray}
L_{ij} = T_{ij} - \widehat{\tau_{ij}^{sgs}} &=& \widehat{\overline{\rho}} \left( \breve{\widetilde{u_i u_j}} - \breve{\widetilde{u}}_i \breve{\widetilde{u}}_j \right) - \widehat{ \overline{\rho} \left( \widetilde{u_i u_j} - \widetilde{u}_i \widetilde{u}_j \right)} \,\mbox{,} \nonumber\\
&=&  \widehat{\overline{\rho}} \left( \breve{\widetilde{u_i u_j}} - \breve{\widetilde{u}}_i \breve{\widetilde{u}}_j \right) - \widehat{\overline{\rho}} \left( \breve{\widetilde{u_i u_j}} - \breve{\widetilde{u}_i \widetilde{u}_j} \right) \,\mbox{,} \nonumber \\
\label{eqn_germano} &=&  \widehat{\overline{\rho}} \left( \breve{\widetilde{u}_i \widetilde{u}_j} - \breve{\widetilde{u}}_i \breve{\widetilde{u}}_j \right) \,\mbox{.}
\end{eqnarray}
Using the Favre definitions, Eq.~(\ref{eqn_germano}) may be rearranged to the form typically seen in the literature,
\begin{eqnarray}
L_{ij} &=&  \widehat{\overline{\rho} \frac{\overline{\rho u_i}}{\overline{\rho}} \frac{\overline{\rho u_j}}{\overline{\rho}}} - \widehat{\overline{\rho}} \frac{ \widehat{\overline{\rho u_i}}}{\widehat{\overline{\rho}}} \frac{\widehat{\overline{\rho u_j}}}{\widehat{\overline{\rho}}} \,\mbox{,} \nonumber \\
&=& \widehat{\frac{\overline{\rho u_i}\, \overline{\rho u_j}}{\overline{\rho}}} - \frac{ \widehat{\overline{\rho u_i}} \,\widehat{\overline{\rho u_j}}}{\widehat{\overline{\rho}}} \,\mbox{,} \nonumber \\
\label{eqn_leonard} &=& \widehat{\overline{\rho} \widetilde{u}_i \widetilde{u}}_j - \frac{ \widehat{\overline{\rho} \widetilde{u}}_i \widehat{\overline{\rho} \widetilde{u}}_j }{ \widehat{\overline{\rho}} } \,\mbox{.}
\end{eqnarray}
\LaTeX\,has a hard time covering the entire term with the ``wide'' version of the hat, but please note that the entire first term of Equation \ref{eqn_leonard} is test filtered. The Leonard term is computable from resolved LES values.

If we now look at the \emph{model} for the Germano identity (the deviatoric part) we have,
\begin{equation}
\label{eqn_germanomodel}
L_{ij}^D = T_{ij}^D - \widehat{\tau_{ij}^{sgs,D}} \approx - 2\left(C_s \widehat{\Delta}\right)^2 \widehat{\overline{\rho}} |\breve{\widetilde{S}}| \left( \breve{\widetilde{S}}_{ij} - \frac{1}{3} \breve{\widetilde{S}}_{kk} \delta_{ij} \right) +  2 \left(C_s \Delta \right)^2 \widehat{\beta}_{ij} \,\mbox{.}
\end{equation}
Note that the entire last term should be test filtered, since $C_s$ is not necessarily uniform.  However, without pulling the length scale out of the filter operation it is difficult to compute a value for $C_s$.

We now rearrange (\ref{eqn_germanomodel}) to facilitate coding,
\begin{equation}
\label{eqn_codemodel}
L_{ij}^D = \left(C_s \Delta\right)^2 M_{ij}^D \,\mbox{,}
\end{equation}
where,
\begin{equation}
\label{eqn_Mij}
M_{ij}^D = 2\left(\widehat{\beta}_{ij} - \alpha \widehat{\overline{\rho}} |\breve{\widetilde{S}}| \left( \breve{\widetilde{S}}_{ij} - \frac{1}{3}\breve{\widetilde{S}}_{kk} \delta_{ij} \right) \right) \,\mbox{,}
\end{equation}
and $\alpha = (\widehat{\Delta}/\Delta)^2$.  For a test filter two times the grid width it appears we should have $\alpha = 4$.  However, as discussed by Lund \cite{Lund:1997}, the method of discrete quadrature significantly affects the results. If using the trapezoid rule, as we do in FDS, then $\alpha = 6$.

We can now compute $L_{ij}$ and $M_{ij}^D$ from known LES quantities.  If we right multiply Eq.~(\ref{eqn_codemodel}) by $M_{ij}^D$, we obtain our desired result:
\begin{equation}
\label{eqn_lengthscale}
\left(C_s \Delta\right)^2 = \frac{ L_{ij}^D M_{ij}^D }{ M_{ij}^D M_{ij}^D } \,\mbox{.}
\end{equation}

\subsubsection{Notes on implementation}

\begin{enumerate}
\item
It is unnecessary to compute the deviatoric part of the Leonard term.  This is because, fortunately, $L_{ij} M_{ij}^D = L_{ij}^D M_{ij}^D$.  Here's the proof (thanks to Stas Borodai of Reaction Engineering International):
\begin{eqnarray}
L_{ij} M_{ij}^D &=& L_{ij} \left( M_{ij} - \frac{1}{3}M_{kk}\delta_{ij} \right) \,\mbox{,} \nonumber \\
&=& L_{ij} M_{ij} - \frac{1}{3} L_{ij}\delta_{ij} M_{kk} \,\mbox{,} \nonumber \\
\label{eqn_LMD} &=& L_{ij} M_{ij} - \frac{1}{3} L_{qq} M_{kk} \,\mbox{.}
\end{eqnarray}
\begin{eqnarray}
L_{ij}^D M_{ij}^D &=& \left( L_{ij} - \frac{1}{3}L_{qq}\delta_{ij} \right) \left( M_{ij} - \frac{1}{3}M_{kk}\delta_{ij} \right) \,\mbox{,} \nonumber \\
&=& L_{ij} M_{ij} - \frac{1}{3} L_{ij}\delta_{ij} M_{kk} - \frac{1}{3} M_{ij}\delta_{ij} L_{qq} + \frac{1}{9} \delta_{ij} \delta_{ij} L_{qq} M_{kk} \,\mbox{,} \nonumber \\
&=& L_{ij} M_{ij} - \frac{1}{3} M_{kk} L_{qq} - \frac{1}{3} L_{qq} M_{kk} + \frac{1}{3} M_{kk} L_{qq} \,\mbox{,} \nonumber \\
\label{eqn_LDMD} &=& L_{ij} M_{ij} - \frac{1}{3} L_{qq} M_{kk} \,\mbox{.}
\end{eqnarray}
Equations (\ref{eqn_LMD}) and (\ref{eqn_LDMD}) are equal and so there is no need to go to the trouble of subtracting the isotropic part out of $L_{ij}$.
\item
The length scale should be averaged over some homogeneous region to maintain stability,
\begin{equation}
\label{eqn_finallengthscale}
\left(C_s \Delta\right)^2 = \frac{ \langle L_{ij} M_{ij}^D \rangle}{ \langle M_{ij}^D M_{ij}^D \rangle } \,\mbox{.}
\end{equation}
In FDS, the brackets denote a spatial average over the test filter width.  If the denominator is zero, the constant is set to zero.
\item
It is common practice to ``clip'' the eddy viscosity.  In theory, a negative value of the eddy viscosity produces backscatter of energy from unresolved to resolved motions.  If this sounds dangerous from a stability perspective, it is.  The simple solution is to set $C_s = 0$, if $\langle L_{ij}M_{ij}^D \rangle < 0$.
\end{enumerate}

\chapter{Fluid-Particle Momentum Transfer}
\subsubsection{Ben Trettel, NIST SURF student}
The trajectories of Lagrangian particles in FDS could be calculated with foward-Euler integration. However, forward-Euler integration extracts momentum from the cell each particle started in. This can cause large changes in the flow field unless the time step is extremely small. An extremely small time step would be necessary for stability. This time step would greatly slow down the calculation. Consequently, a stable, single-step approximate solution is developed and is implemented in FDS.
\subsubsection{Relative velocities}
Let $m_p$ denote the particle mass, $\mathbf{u}_p$ the particle velocity, $A_p$ the particle cross-sectional area, $C_{d,p}$ the particle drag coefficient, $\rho_a$ the fluid mass density, $\mathbf{U}_p$ the fluid velocity around the particle, $V_g$ is the volume occupied by the fluid in a cell, and $M \equiv \rho_a V_g$ the fluid mass of a cell, $n_p$ is the number of particles in a cell, $M_p \equiv M/n_p$ is the average fluid mass per particle in a cell, and $\mathbf{g}$ is the gravitational acceleration vector.


The equations of motion of the particles and fluid are formulated as follows from Newton's second law,
\begin{align}
    \label{particle_eom}
    m_p \frac{\text{d} \mathbf{u}_p}{\text{d} t} &= - \frac{1}{2} \rho_a C_{d,p} A_p (\mathbf{u}_p - \mathbf{U}_p) |\mathbf{u}_p - \mathbf{U}_p| + m_p \mathbf{g} \\
    \label{fluid_eom}
    M_p \frac{\text{d} \mathbf{U}_p}{\text{d} t} &= \frac{1}{2} \rho_a C_{d,p} A_p (\mathbf{u}_p - \mathbf{U}_p) |\mathbf{u}_p - \mathbf{U}_p| \,.
\end{align}
Note that the fluid Eq. (\ref{fluid_eom}) does not include a gravity term. This gravity term is included in the Navier-Stokes equations; including it here would be redundant. Also note that lift is not included here.


If we define $\mathbf{u}_r \equiv \mathbf{u}_p - \mathbf{U}_p$ as the relative velocity between the fluid and the particle, we can find a single equation for the relative velocity by dividing both equations by their respective masses (i.e. $m_p$ and $M_p$) and then subtracting the second from the first. This result is
\begin{align}
    \frac{\text{d} \mathbf{u}_r}{\text{d} t} = -\frac{1}{2} \rho_a C_{d,p} A_p \left(\frac{1}{m_p} + \frac{1}{M_p} \right) \mathbf{u}_r |\mathbf{u}_r| + \mathbf{g} \,.
\end{align}
The equation above can be written in short as
\begin{align}
    \frac{\text{d} \mathbf{u}_r}{\text{d} t} = -K \mathbf{u}_r |\mathbf{u}_r| + \mathbf{g} \quad ; \quad K_p \equiv \frac{1}{2} \rho_a C_{d_p} A_p \left(\frac{1}{m_p} + \frac{1}{M_p} \right) \,.
\end{align}
Note that the $p$ subscripts have been dropped in $\mathbf{u}_r$ terms for convenience. The $p$ subscript also will be dropped from some other variables later as convenient.


This is the drag equation, which has no solution in terms of elementary functions. Our solution approach first finds a solution neglecting gravity and then adds in a series for the gravity terms. $\mathbf{u}_r \equiv \mathbf{u}_d + \mathbf{u}_g$ is the decomposition of $\mathbf{u}_r$. $\mathbf{u}_d$ and $\mathbf{u}_d$ both have the same initial condition, $\mathbf{u}_{r,0} \equiv \mathbf{u}_p(0) - \mathbf{U}_p(0) = \mathbf{u}_{p,0} - \mathbf{U}_{p,0}$. $\mathbf{u}_d$ satisfies the drag equation without gravity, specifically
\begin{align}
    \frac{\text{d} \mathbf{u}_d}{\text{d} t} = -K_p \mathbf{u}_d |\mathbf{u}_d| \,.
\end{align}
The solution subject to these initial conditions is
\begin{align}
    \label{ud_exact}
    \mathbf{u}_d = \frac{\mathbf{u}_{p,0} - \mathbf{U}_{p,0}}{1 + \beta_p t} \quad ; \quad \beta_p \equiv K_p |\mathbf{u}_{r,0}| \,.
\end{align}
A series solution for $\mathbf{u}_g$ can be found via Taylor series. Note that $\mathbf{u}_g$ can be written $\mathbf{u_g} = \mathbf{u}_r - \mathbf{u}_d$ so the differential equation for $\mathbf{u}_g$ is
\begin{align}
    \frac{\text{d} \mathbf{u}_g}{\text{d} t} = -K_p (\mathbf{u}_r |\mathbf{u}_r| - \mathbf{u}_d |\mathbf{u}_d|) + \mathbf{g} \quad ; \quad \mathbf{u}_g(0) = \mathbf{u}_r(0) - \mathbf{u}_d(0) = 0 \,.
\end{align}
The Taylor series for $\mathbf{u}_g$ about $t = 0$ is
\begin{align}
    \mathbf{u}_g(t) = \mathbf{u}_g(0) + t \frac{\text{d} \mathbf{u}_g}{\text{d} t}(0) + \frac{t^2}{2} \frac{\text{d}^2 \mathbf{u}_g}{\text{d} t^2}(0) + \frac{t^3}{6} \frac{\text{d}^3 \mathbf{u}_g}{\text{d} t^3}(0) + \cdots \,.
\end{align}
The task now is to find the derivatives of $\mathbf{u}_g$ at $t = 0$. The first derivative, $\frac{\text{d} \mathbf{u}_g}{\text{d} t}(0)$, can be seen to be $\mathbf{g}$ by inspection, as we would expect from the solution without drag. The second derivative is more complicated, and we find that
\begin{align*}
    \frac{\text{d}^2 \mathbf{u}_g}{\text{d}^2 t} = -K_p \frac{\text{d}}{\text{d} t}(\mathbf{u}_r |\mathbf{u}_r| - \mathbf{u}_d |\mathbf{u}_d|) + \frac{\text{d} \mathbf{g}}{\text{d} t} = -K_p \left(|\mathbf{u}_r| \frac{\text{d} \mathbf{u}_r}{\text{d} t} + \mathbf{u}_r \frac{\text{d} |\mathbf{u}_r|}{\text{d} t} - |\mathbf{u}_d| \frac{\text{d} \mathbf{u}_d}{\text{d} t} - \mathbf{u}_d \frac{\text{d} |\mathbf{u}_d|}{\text{d} t}\right) \,,
\end{align*}
\begin{align}
    \frac{\text{d}^2 \mathbf{u}_g}{\text{d}^2 t}(0) = -K_p \left(|\mathbf{u}_r(0)| \frac{\text{d} \mathbf{u}_r}{\text{d} t}(0) + \mathbf{u}_r(0) \frac{\text{d} |\mathbf{u}_r|}{\text{d} t}(0) - |\mathbf{u}_d(0)| \frac{\text{d} \mathbf{u}_d}{\text{d} t}(0) - \mathbf{u}_d(0) \frac{\text{d} |\mathbf{u}_d|}{\text{d} t}(0)\right) \,.
\end{align}
The values of $\frac{\text{d} \mathbf{u}_r}{\text{d} t}(0)$ and $\frac{\text{d} \mathbf{u}_d}{\text{d} t}(0)$ are
\begin{align*}
    \frac{\text{d} \mathbf{u}_r}{\text{d} t}(0) &= -K_p \mathbf{u}_{r,0} |\mathbf{u}_{r,0}| + \mathbf{g} \,, \\
    \frac{\text{d} \mathbf{u}_d}{\text{d} t}(0) &= -K_p \mathbf{u}_{r,0} |\mathbf{u}_{r,0}| \,.
\end{align*}
% = -K_p \left(\frac{\text{d} \mathbf{u} |\mathbf{u}|}{\text{d} t} - \frac{\text{d} \mathbf{u}_d |\mathbf{u}_d|}{\text{d} t} \right)
The derivative of the L$_2$ norm must now be found. It can be shown that for an arbitrary vector $\mathbf{a}$,
\begin{align*}
    \frac{\text{d} |\mathbf{a}|}{\text{d} t} = \left(\frac{\mathbf{a}}{|\mathbf{a}|}\right) \cdot \frac{\text{d} \mathbf{a}}{\text{d} t} \,.
\end{align*}
So, the derivatives of the vector norms can be written as
\begin{align}
    \frac{\text{d} |\mathbf{u}_r|}{\text{d} t} &= \left(\frac{\mathbf{u}_r}{|\mathbf{u}_r|}\right) \cdot \frac{\text{d} \mathbf{u}_r}{\text{d} t} = \left(\frac{\mathbf{u}_{r,0}}{|\mathbf{u}_{r,0}|}\right) \cdot (-K_p \mathbf{u}_{r,0} |\mathbf{u}_{r,0}| + \mathbf{g}) \,,  \\
    \frac{\text{d} |\mathbf{u}_d|}{\text{d} t} &= \left(\frac{\mathbf{u}_d}{|\mathbf{u}_d|}\right) \cdot \frac{\text{d} \mathbf{u}_d}{\text{d} t} = \left(\frac{\mathbf{u}_{r,0}}{|\mathbf{u}_{r,0}|}\right) \cdot (-K_p \mathbf{u}_{r,0} |\mathbf{u}_{r,0}|) \,.
\end{align}
Once all of this is written out and expanded, the second derivative of $\mathbf{u}_g$ at $t = 0$ is
\begin{align}
    \frac{\text{d}^2 \mathbf{u}_g}{\text{d}^2 t}(0) = -K_p \left[\mathbf{u}_{r,0} \left(\frac{\mathbf{u}_{r,0} \cdot \mathbf{g}}{|\mathbf{u}_{r,0}|}\right) + \mathbf{g} |\mathbf{u}_{r,0}|\right] = - \beta_p \left[\mathbf{u}_{r,0} \left(\frac{\mathbf{u}_{r,0} \cdot \mathbf{g}}{|\mathbf{u}_{r,0}|^2}\right) + \mathbf{g}\right] \,.
\end{align}
Note that this has a term parallel to the initial relative velocity and a term parallel to the gravitational acceleration vector.


Assembling all these terms, $\mathbf{u}_g$ can be found to be
\begin{align}
    \label{ug_second_order}
    \mathbf{u}_g = \mathbf{g} t - \frac{\beta_p t^2}{2} \left[\mathbf{u}_{r,0} \left(\frac{\mathbf{u}_{r,0} \cdot \mathbf{g}}{|\mathbf{u}_{r,0}|^2}\right) + \mathbf{g}\right] + \mathcal{O}(t^3) \,.
\end{align}
Assembling Eqs. (\ref{ud_exact}) and (\ref{ug_second_order}), the solution for $\mathbf{u}_r$ is
\begin{align}
    \label{ur_short}
    \mathbf{u}_r &= \frac{\mathbf{u}_{r,0}}{1 + \beta_p t} + \mathbf{g} t - \frac{\beta_p t^2}{2} \left[\mathbf{u}_{r,0} \left(\frac{\mathbf{u}_{r,0} \cdot \mathbf{g}}{|\mathbf{u}_{r,0}|^2}\right) + \mathbf{g}\right] + \mathcal{O}(t^3) \quad ; \quad \beta_p \equiv K_p |\mathbf{u}_{r,0}| \,. %\\
    %\label{ur_expanded}
    %\mathbf{u}_p - \mathbf{U}_p &= \frac{\mathbf{u}_{p,0} - \mathbf{U}_{p,0}}{1 + \beta_p t} + \mathbf{g} t - \frac{\beta_p t^2}{2} \left[(\mathbf{u}_{p,0} - \mathbf{U}_{p,0}) \left(\frac{(\mathbf{u}_{p,0} - \mathbf{U}_{p,0}) \cdot \mathbf{g}}{|\mathbf{u}_{p,0} - \mathbf{U}_{p,0}|^2}\right) + \mathbf{g}\right] + \mathcal{O}(t^3) \quad ; \quad \beta_p \equiv K_p |\mathbf{u}_{p,0} - \mathbf{U}_{p,0}|
\end{align}
\subsubsection{Particle velocities and positions}
The results of the previous section are not directly useful unless $\mathbf{U}_p(t)$ is known. $\mathbf{U}_p(t)$ can be found via the conservation of momentum and this leads to a solution for the particle velocities and positions.


The fluid and particles can gain or lose momentum due to gravity. Drag forces exchange momentum between the fluid and particle, which does not change the total momentum of the system. As gravity is the only force that can change momentum, the time derivative of the fluid-particle system momentum is $m_p \mathbf{g}$ by Newton's second law, so we can write
\begin{align}
    m_p \mathbf{u}_p + M_p \mathbf{U}_p &= m_p \mathbf{u}_{p,0} + M_p \mathbf{U}_{p,0} + m_p \mathbf{g} t \,, \\
    \label{part_mom_cons}
    \mathbf{u}_p + \alpha_p \mathbf{U}_p &= \mathbf{u}_{p,0} + \alpha_p \mathbf{U}_{p,0} + \mathbf{g} t \quad ; \quad \alpha_p = \frac{M_p}{m_p} \,.
\end{align}
%Eq. (\ref{part_mom_cons}) can be combined with Eq. (\ref{ur_short}) to get solution for $\mathbf{u}_p$.
Solving for $\mathbf{U}_p$ leads to
\begin{align*}
    \mathbf{U}_p + \frac{\mathbf{u}_{r,0}}{1 + \beta_p t} + \mathbf{u}_g + \alpha_p \mathbf{U}_p = \mathbf{u}_{p,0} + \alpha_p \mathbf{U}_{p,0} + \mathbf{g} t \,,
\end{align*}
\begin{align}
    \label{mom_U_p}
    \mathbf{U}_p = \frac{\mathbf{u}_{p,0} + \alpha_p \mathbf{U}_{p,0} + \mathbf{g} t - \mathbf{u}_g}{1 + \alpha_p} - \frac{\mathbf{u}_{r,0}}{(1 + \beta_p t)(1 + \alpha_p)} \,.
\end{align}
Eq. (\ref{mom_U_p}) can be substituted into Eq. (\ref{ur_short}) to get the solution for $\mathbf{u}_p$.
\begin{align*}
    \mathbf{u}_p &= \mathbf{U}_p + \frac{\mathbf{u}_{p,0} - \mathbf{U}_{p,0}}{1 + \beta_p t} + \mathbf{u}_g \\
    &= \frac{\mathbf{u}_{p,0} + \alpha_p \mathbf{U}_{p,0} + \mathbf{g} t - \mathbf{u}_g}{1 + \alpha_p} - \frac{\mathbf{u}_{p,0} - \mathbf{U}_{p,0}}{(1 + \beta_p t)(1 + \alpha_p)} + \frac{\mathbf{u}_{p,0} - \mathbf{U}_{p,0}}{1 + \beta_p t} + \mathbf{u}_g \\
    &= \frac{\mathbf{u}_{p,0}}{1 + \beta_p t} + \frac{(\mathbf{u}_{p,0} + \alpha_p \mathbf{U}_{p,0})\beta_p t}{(1 + \beta_p t)(1 + \alpha_p)} + \frac{\mathbf{g} t + \alpha_p \mathbf{u}_g}{1 + \alpha_p}
\end{align*}
\begin{align}
    \label{mom_u_p_full}
    \mathbf{u}_p = \frac{\mathbf{u}_{p,0}}{1 + \beta_p t} + \frac{(\mathbf{u}_{p,0} + \alpha_p \mathbf{U}_{p,0})\beta_p t}{(1 + \beta_p t)(1 + \alpha_p)} + \mathbf{g} t - \frac{\alpha_p \beta_p t^2}{2 (1 + \alpha_p)} \left[\mathbf{u}_{r,0} \left(\frac{\mathbf{u}_{r,0} \cdot \mathbf{g}}{|\mathbf{u}_{r,0}|^2}\right) + \mathbf{g}\right] + \mathcal{O}(t^3)
\end{align}
Integrating Eq. (\ref{mom_u_p_full}) leads to an equation for the particle positions,
\begin{align}
    \label{mom_x_p_full}
    \mathbf{x}_p = \mathbf{x}_{p,0} + \left(\frac{\mathbf{u}_{p,0} + \alpha_p \mathbf{U}_{p,0}}{1 + \alpha_p}\right) t - \frac{\alpha_p (\mathbf{u}_{p,0} - \mathbf{U}_{p,0})}{\beta_p (1 + \alpha_p)} \text{ln}(\beta_p t + 1) + \frac{\mathbf{g} t^2}{2} - \frac{\alpha_p \beta_p t^3}{6 (1 + \alpha_p)} \left[\mathbf{u}_{r,0} \left(\frac{\mathbf{u}_{r,0} \cdot \mathbf{g}}{|\mathbf{u}_{r,0}|^2}\right) + \mathbf{g}\right] + \mathcal{O}(t^4) \,.
\end{align}
\subsubsection{Implementation in FDS}
The particle positions are computed in {\ct part.f90} in {\ct DROPLET\_LOOP}. The solutions are used to advance $\Delta t$ forward in time much like a normal finite-difference scheme. The exact solution is used for the case without drag.
\begin{align}
    \mathbf{u}_p^{n+1} &= \frac{\mathbf{u}_p^n}{1 + \beta_p \Delta t} + \frac{(\mathbf{u}_p^n + \alpha_p \mathbf{U}_p^n)\beta_p \Delta t}{(1 + \beta_p \Delta t)(1 + \alpha_p)} + \mathbf{g} \Delta t - \frac{\alpha_p \beta_p (\Delta t)^2}{2 (1 + \alpha_p)} \left[\mathbf{u}_r^n \left(\frac{\mathbf{u}_r^n \cdot \mathbf{g}}{|\mathbf{u}_r^n|^2}\right) + \mathbf{g}\right] \\[1cm]
    \mathbf{x}_p^{n+1} &= \mathbf{x}_p^n + \left(\frac{\mathbf{u}_p^n + \alpha_p \mathbf{U}_p^n}{1 + \alpha_p}\right) \Delta t + \frac{\alpha_p (\mathbf{u}_p^n - \mathbf{U}_p^n)}{\beta_p (1 + \alpha_p)} \text{ln}(\beta_p \Delta t + 1) + \frac{\mathbf{g} (\Delta t)^2}{2} - \frac{\alpha_p \beta_p (\Delta t)^3}{6 (1 + \alpha_p)} \left[\mathbf{u}_p^n \left(\frac{\mathbf{u}_p^n \cdot \mathbf{g}}{|\mathbf{u}_p^n|^2}\right) + \mathbf{g}\right]
\end{align}
The theoretical accuracy of the equation for $\mathbf{x}^{n+1}$ appears to be $\mathcal{O}(\Delta t^3)$ at first glance, but actual computation reveals it is $\mathcal{O}(\Delta t^2)$. This is due to the error in the velocity reducing the overall order of accuracy. The $\Delta t^3$ term in the equation for $\mathbf{x}^{n+1}$ can be dropped without a loss in accuracy for this reason. More details about this are available in the verification guide's Lagrangian particles chapter.

\chapter{Finding Absorption Coefficients of Liquid Fuels}
%\subsubsection{Topi Sikanen, entered by Michael Van Order}
\label{app_abscoeff}

Pool fires depend on radiation feedback from the flames to the fuel surface. The feedback comes through convection, conduction and radiation. For large pool fires radiation heat transfer dominates. Radiation transport in pool fires has received a lot of attention. Studies have been made to determine the spectra of emitted radiation \cite{Suo-Anttila:PCT2009} as well as to characterize the radiation absorption by the gasses within the flame \cite{Wakatsuki:CST2008}. The transport and absorption of thermal radiation within the liquid fuel has received considerably less attention.

In order to model the response of the fuel to the incident radiation it is necessary to model the absorption of this radiation. For fuels such as wood, most of the incident radiation is absorbed at a thin region near the surface. For semitransparent materials such as plastics or liquid fuels thermal radiation may penetrate a finite depth in to the fuel.  The in-depth radiation absorption by semitransparent fuels has been to best of our knowledge been studied only in the case of PMMA ~\cite{Stoliarov:CF2009}, polymer films ~\cite{Tsilingiris:ECM2003} and in liquid pool fires ~\cite{Suo-Anttila:PCT2009}.  In the latter study, the effect of radiation absorption on the burning rates was not investigated.  Most of the research considering the in-depth radiation absorption in liquids concerns boil-over of liquid pool fires on water \cite{Broeckmann:JLPPI1995}. The effect of in-depth radiation absorption on evaporation of fuel droplets has also received some attention \cite{Sazhin:IJHMT2004b}.

Most liquids are highly selective absorbers, absorbing intensively in some wavelength regions while being transparent in other regions.  This leads to models of radiation transport that are both computationally intensive and for which experimental data is scarce. In this paper we attempt to characterize the absorption of radiation by liquid fuels using effective absorption coefficients similar to those used by \cite{Madhav:IJMP1995} and \cite{Manohar:JHT1995} in modeling effect of radiation absorption on PMMA pyrolysis.

\subsubsection{Determining Absorption Coefficients}

Absorption coefficient measures attenuation of radiation in matter.  Intensity of radiation after it has travelled for path-length $S$ (m) can be calculated from Beer's law

\begin{equation}
\label{eqn_beerslaw}
\ I_{S,\lambda} = I_{0,\lambda} exp\left( -\kappa_\lambda S \right) \,\mbox{,}
\end{equation}
where $\kappa_{\lambda}$ (m$^{-1}$) is the absorption coefficient for wavelength $\lambda$ (m).  Absorption coefficient can be determined from experiment by measuring the transmittance $\tau_{S}=\frac{I_{S,\lambda}}{I_{0,\lambda}}$ for a path length $S$:

\begin{equation}
\label{eqn_klambda}
\ \kappa_\lambda = -\frac{1}{S} \log [\tau_S] \mbox{.}
\end{equation}

Absorption coefficient data for liquid fuels is rather scarce and data has to be collected from various sources of sometimes-questionable accuracy. The size and scaling of graphics depicting absorption coefficients, pose further difficulties in collecting data on absorption coefficients.

Where data on absorption coefficients of liquids exists in the open literature, such as Coblentz Society data found on the NIST Chemistry WebBook \cite{Coblentz:1}, it usually  only contains data for wavelengths from approximately 2.5 $\mu$m upwards. A large part of the total energy in emission spectrum of flames may easily be contained in wavelengths shorter than 2.5 $\mu$m thus rendering these data useless for our purposes.

However absorption spectra that begin from 1 $\mu$m exists for a few liquids of which toluene (\cite{Bertie:JMS2005}, \cite{Bertie:AS1994b}, \cite{Bertie:AS1994a}), methanol \cite{Bertie:AS1993a}, benzene \cite{Bertie:AS1993b} and water \cite{Bertie:AS1996} are of interest for our current purposes. Furthermore, \cite{Suo-Anttila:PCT2009} included in their work on pool fires spectrally resolved transmission spectra of ethanol, heptane, JP-8 and an ethanol-toluene blend and calculations. Complex refractive index spectra for few diesel fuels were reported in \cite{Sazhin:IJHMT2004b}. Different diesel fuels have slightly different absorption spectra, due to differing additives. However the data reported in \cite{Sazhin:IJHMT2004b} can perhaps be used to obtain a order of magnitude estimate for the absorption coefficient of diesel fuel.

\begin{figure}[ht]
    \label{fig_absspec}
    \centering
    \includegraphics{FIGURES/Absorption_spectra.pdf}
    \caption{Absorption spectra of liquids considered in this study.}
\end{figure}

\subsubsection{Effective Absorption Coefficients}

Often we are not interested in resolving the spectra of the transmitted radiation, but rather we are interested in modelling the total transmitted radiation.  In these cases it is convenient to write the radiation transport equations in terms of mean absorption coefficients. This is done in order to avoid the time-consuming integrations over all wavelengths. For this end a number of mean absorption coefficients have been introduced, such as the Rosselland-mean absorption coefficient and the Planck-mean absorption coefficient. These correspond to the optically thin approximation and the Rosselland diffusion approximation of radiation transport.

The absorption coefficients of liquids are highly wavelength dependent being transparent in some areas. In this case, the Planck-mean absorption coefficients lead to mean absorption coefficients that are too large by several orders of magnitude. Tsiliringis et al. \cite{Tsilingiris:ECM2003} proposed the use of absorption coefficient defined as, "in conjuction with Beer's law model of radiation transport."

It is preferable to determine an effective absorption coefficient that attempts to replicate the absorption of radiation over a certain path length $S$. Here the path length is not a physical measure of the system but some characteristic length over which the majority of the radiation is assumed absorbed.

Here we find the intensity for finite number of wavelength bands from Equation (\ref{eqn_beerslaw}) . Then the total transmitted intensity is found by integrating over all wavelengths

\begin{equation}
\label{eqn_totintens}
     \ \bar{I}_S = \int_{0}^{\infty} I_{\lambda,S} d\lambda \mbox{.}
\end{equation}

The effective mean absorption coefficient is then found by minimizing the norm of error between the predicted radiation intensity at path lengths $S-_i$ and $\bar{I}_{S-_i}$, where $i=1,2,...,N$:

\begin{equation}
\label{eqn_effcoeff}
     \ \bar{\kappa} = \mbox{arg } \underbrace{\mbox{min}}_{\kappa} \sum_{i=1}^{N} \left(\hat{I}_{S_i}-\bar{I}_{S_i}\right)^{2}  \mbox{.}
\end{equation}

Here $\hat{I}_S$ is the intensity predicted by our model.  The absorption coefficients so calculated, depend on both the path-length $S$ and the spectrum of the incoming radiation. The absorption coefficient $\bar{\kappa}$ itself should be seen as a curve fit parameter.

The least squares fit is chosen here since we want to reproduce the distribution of the heat source in a one dimensional heat conduction equation for the liquid. Alternatively we might choose reproduce exactly the transmitted fraction of radiation on at some path length $S$. We could select the path length to be the thickness of the fuel sample. In this case, the fuel would absorb the correct amount of radiation energy. However , the distribution of the heat source is likely to be wrong.

It can be easily seen that increasing the path-length leads to lower effective absorption coefficients.  The attenuation of radiation in the fuels shows a phase of exponential decay followed by approximately constant transmitted fraction of radiation. If the liquid layer is thick enough, that we can assume that all radiation is absorbed within the layer, we can concentrate on replicating the profile of radiation attenuation. If, however, it is expected that the radiation passing through the liquid layer play an important role, we should select the absorption coefficient so that the correct amount of radiation is transmitted through the layer. Such situation might be for example a thin layer of fuel burning on top of water. Here the radiation passing through the fuel layer may heat up the underlying water and lead to boil-over earlier than only conduction would predict.

\subsubsection{Absorption Coefficients of Selected Fuels}

Absorption coefficients are determined from Eqn. \ref{eqn_effcoeff}. The data used in determining the coefficients is listed in Table \ref{tbl_wavnum}. When available, spectrally resolved absorption coefficients are used for determining the total transmitted fraction of radiation according to Eqns. \ref{eqn_beerslaw} and \ref{eqn_totintens}.  The transmission data in Suo-Anttila et al. 2009 \cite{Suo-Anttila:PCT2009} was used for JP-8, ethanol and toluene. The spectrally resolved absorption coefficients from Keefe et al. were used to calculate the total transmission for toluene, benzene, water and methanol. Complex refractive index spectra of diesel reported by Sazhin et al. 2004a \cite{Sazhin:JHT2004a} was used to calculate the transmission.

Figure \ref{fig_trans2} and Figure \ref{fig_trans3} show the transmitted fraction of radiation predicted by the FDS radiation model with the absorption coefficients listed in Table \ref{tbl_abscoeff}, compared with the total transmission calculated from Eqn. \ref{eqn_beerslaw}.

The two-flux model of radiation transport predicts the attenuation of radiation well for toluene and methanol, both of which have rather low absorption coefficients. In the other extreme, the attenuation of radiation by water and ethanol is poorly replicated. Here only the exponential attenuation near the surface is reasonably replicated.

\begin{table}[ht]
\caption{Characterization of data sources used for determining absorption coefficients and effective absorption coefficients for selected liquids. Blackbody temperature 1450 K and path length 3 mm. Data from Suo-Anttila et al. 2009 \cite{Suo-Anttila:PCT2009}.}
\centering
\begin{tabular}{l r r}
\hline\hline
Liquid & Wavenumber Range (cm$^{-1}$) & Effective Absorption Coefficient,$\kappa$ \\ [0.5ex]
\hline
JP-8                                 &  -          & 301.4   \\
Ethanol-Toluene blend                &  -          & 680.1   \\
Ethanol                              &  -          & 1534.3  \\
Toluene \cite{Suo-Anttila:PCT2009}   &  -          & 187.5   \\
Toluene \cite{Bertie:AS1994a}        &  436-6500   & 160.8   \\
Methanol                             &  2-8000     & 52      \\
Water                                &  xxxx-xxxx  & 1578    \\
Benzene                              &  xxxx-xxxx  & 123     \\
\hline
\end{tabular}
\label{tbl_abscoeff}
\end{table}

\begin{figure}[ht]
    \centering
    \includegraphics[width=5.0in]{FIGURES/Suo_anttila_effective_FDS1.pdf}
    \caption{Predicted (full lines) vs. experimental (symbols) transmission.  Experimental transmission data from [Suo-Anttila et al. 2009] \cite{Suo-Anttila:PCT2009}.}
    \label{fig_trans2}
\end{figure}

\begin{figure}[ht]
    \centering
    \includegraphics[width=5.0in]{FIGURES/Suo_anttila_effective_FDS2.pdf}
    \caption{Transmission from two-flux model (full-lines) vs. transmission calculated from spectral absorption coefficients. Blackbody temperature 1450 K was assumed.}
    \label{fig_trans3}
\end{figure}

\begin{figure}[ht]
    \centering
    \includegraphics[width=3.0in]{FIGURES/KAPPA_VS_TEMP_WD.pdf}
    \includegraphics[width=3.0in]{FIGURES/KAPPA_VS_TEMP_TMB.pdf}

    \caption{Sensitivity of effective absorption coefficient to incoming radiation temperature.}
    \label{fig_kapvtemp}
\end{figure}

\begin{figure}[ht]
    \centering
    \includegraphics[width=3.0in]{FIGURES/KAPPA_VS_PATH_LENGTH_WD.pdf}
    \includegraphics[width=3.0in]{FIGURES/KAPPA_VS_PATH_LENGTH_TMB.pdf}

    \caption{Sensitivity of effective absorption coefficient to path length used in fitting.}
    \label{fig_kapvPlength}
\end{figure}
