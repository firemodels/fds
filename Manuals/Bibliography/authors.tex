\chapter{Authors}

The Fire Dynamics Simulator and Smokeview are the products of an international collaborative effort led by
the National Institute of Standards and Technology (NIST) and VTT Technical Research Centre of Finland. Its developers and
contributors are listed below.

\vspace{0.5in}

\begin{flushleft}

Principal Developers (in alphabetical order) \\ [0.2in]

Jason Floyd, Hughes Associates, Inc., Baltimore, Maryland, USA \\
Glenn Forney, NIST \\
Simo Hostikka, VTT \\
Timo Korhonen, VTT  \\
Randall McDermott, NIST \\
Kevin McGrattan, NIST \\ [0.5in]

Contributers \\ [0.2in]

Elizabeth Blanchard, Centre Scientifique et Technique du B\^{a}timent (CSTB), Paris, France \\
Susan Killian, hhpberlin, Germany \\
Charles Luo, Global Engineering and Materials, Inc., Princeton, New Jersey, USA \\
Anna Matala, VTT \\
William Mell, U.S. Forest Service, Seattle, Washington, USA \\
Christian Rogsch, Neustadt/Wstr., Germany \\
Topi Sikanen, VTT \\
Ben Trettel, University of Maryland, USA \\
Craig Weinschenk, NIST

\end{flushleft}


\chapter{About the Authors}

\begin{description}

\item[Elizabeth Blanchard] is a fire protection engineer at the French building agency CSTB. She holds a master of science degree in mathematical modeling and a doctorate in mechanics and thermal engineering. She is mainly involved at CSTB in the research program concerning water spray.

\item[Jason Floyd] is a Senior Engineer at Hughes Associates, Inc., in Baltimore, Maryland. He received a bachelors of science and Ph.D. in the Nuclear Engineering Program of the University of Maryland. After graduating, he won a National Research Council Post-Doctoral Fellowship at the Building and Fire Research Laboratory of NIST, where he developed the combustion algorithm within FDS. He is a principal developer of the combustion model and control logic within FDS.

\item[Glenn Forney] is a computer scientist in the Engineering Laboratory of NIST. He received a bachelors of science degree in mathematics from Salisbury State College in 1978 and a master of science and a doctorate in mathematics at Clemson University in 1980 and 1984.  He joined the NIST staff in 1986 (then the National Bureau of Standards) and has since worked on developing tools that provide a better understanding of fire phenomena, most notably Smokeview, a software tool for visualizing Fire Dynamics Simulation data.

\item[Simo Hostikka] is a Senior Research Scientist at VTT Technical Research Centre of Finland. He received a master of science (technology) degree in 1997 and a doctorate in 2008 from the Department of Engineering Physics and Mathematics of the Helsinki University of Technology.  He is the principal developer of the radiation and solid phase sub-models within FDS.

\item[Susan Kilian] is a mathematician with numerics and scientific computing expertise. She received her diploma from the University of Heidelberg and received her doctorate from the Technical University of Dortmund in 2002. Since 2007 she has been a research scientist for hhpberlin, a fire safety engineering firm located in Berlin, Germany. Her research interests include high performance computing and the development of efficient parallel solvers for the pressure Poisson equation. 

\item[Charles Luo] is a Senior Research Scientist at Global Engineering and Materials, Inc., in Princeton, New Jersey. He received a B.S.~in theoretical and applied mechanics from the University of Science and Technology of China in 2002, and a doctorate in mechanical engineering from the State University of New York at Buffalo in 2010. His research interests include fire-structure interaction, immersed boundary methods, and fire response of composite and aluminum structures.

\item[Anna Matala] is a Research Scientist at VTT Technical Research Centre of Finland and a PhD candidate at Aalto University School of Science. She received her M.Sc.~degree in Systems and Operations Research from Helsinki University of Technology in 2008. Her research concentrates on pyrolysis modelling and parameter estimation in fire simulations.

\item[Randall McDermott] joined the research staff of the Building and Fire Research Lab in 2008. He received a B.S.~from the University of Tulsa in Chemical Engineering in 1994 and a doctorate at the University of Utah in 2005. His research interests include subgrid-scale models and numerical methods for large-eddy simulation, adaptive mesh refinement, immersed boundary methods, and Lagrangian particle methods.

\item[Kevin McGrattan] is a mathematician in the Engineering Laboratory of NIST. He received a bachelors of science degree from the School of Engineering and Applied Science of Columbia University in 1987 and a doctorate at the Courant Institute of New York University in 1991. He joined the NIST staff in 1992 and has since worked on the development of fire models, most notably the Fire Dynamics Simulator.

\item[William (Ruddy) Mell] is an applied mathematician currently at the U.S. Forest Service in Seattle, Washington. He holds a B.S. degree from the University of Minnesota (1981) and doctorate from the University of Washington (1994). His research interests include the development of large eddy simulation methods and sub-models applicable to the physics of large fires in buildings, vegetation, and the wildland-urban interface.

\item[Christian Rogsch] received a Diploma degree (like M.Sc.) in Safety Engineering from the University of Wuppertal, Germany. He works on shared-memory parallelization (OpenMP) of the Fire Dynamics Simulator.

\item[Topi Sikanen] is a Research Scientist at VTT Technical Research Centre of Finland and a graduate student at Aalto University School of Science. He received his M.Sc.~degree in Systems and Operations Research from Helsinki University of Technology in 2008. He works on the Lagrangian particle and liquid evaporation models. 

\item[Ben Trettel] is a graduate student at the University of Maryland. He received a B.S.~degree from the University of Maryland in Mechanical Engineering in 2011. He develops models for the transport of Lagrangian particles for the Fire Dynamics Simulator.

\item[Craig Weinschenk] joined the Fire Research Division as a National Research Council Postdoctoral Research Associate in 2011. He received a B.S.~from Rowan University in Mechanical Engineering in 2006, an M.S.~from the University of Texas-Austin in Mechanical Engineering in 2007, and a doctorate from the University of Texas-Austin in Mechanical Engineering in 2011. His research interests include numerical combustion, quadrature method of moments, and human factors research of fire-fighting tactics.

\end{description}


