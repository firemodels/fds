\chapter{Heat Flux}

\section{Point Source Radiation Heat Flux}

The point source model assumes that radiant energy is released at a point located at the center of the fire. The radiant heat flux at any distance $x$ from the source fire is inversely related to the horizontal separation distance $r$, by the following equation

\be
\dot q_r'' = \frac{x}{r} \left( \frac{\chi_r \dot Q}{4 \pi R^2} \right)
\label{eq:point_source}
\ee

\be
r = \sqrt{x^2 + \left(z - \frac{L_f}{3} \right)^2}
\ee

\be
D = \sqrt{\frac{4 A}{\pi}}
\label{eq:point_source_D}
\ee

\be
Q^* = \frac{\dot Q}{\rho_\infty c_p T_\infty \sqrt{g} D^{5/2}}
\ee

\be
L_f = D (3.7 Q^{*^{2/5}} - 1.02)
\ee

In Eq.~\ref{eq:point_source}, $\dot q_r''$ is the radiant heat flux~(kW/m$^2$), $\chi_r$ is the radiative fraction~(-), $\dot Q$ is the HRR of the fire~(kW), and $R$ is the radial distance from the center of the flame to the edge of the target~(m).

In Eq.~\ref{eq:point_source_D}, $D$ is the diameter of the fire source~(m).

\clearpage

A summary scatter plot of point source radiation heat flux predictions is given in Fig.~\ref{point_source_heat_flux_summary}. 

\begin{figure}[ht]
\begin{center}
\begin{tabular}{l}
\includegraphics[width=4.0in]{FIGURES/Scatterplots/Point_Source_Radiation_Heat_Flux}
\end{tabular}
\end{center}
\caption[Summary of point source radiation heat flux predictions.]
{Summary of point source radiation heat flux predictions.}
\label{point_source_heat_flux_summary}
\end{figure}


\clearpage


\section{Immersed HGL Radiation Heat Flux}

The immersed HGL radiation heat flux can be calculated by

\be
\dot q_r'' = \sigma (T_g^4 - T_\infty^4)
\ee

\noindent where $\sigma$ is the Stefan-Boltzmann constant ($W/m^2/K^4$), $T_g$ is the hot gas layer temperature from the appropriate method described in Chapter~\ref{HGL:Chapter}, and $T_\infty$ is the ambient temperature (K).

\clearpage

A summary scatter plot of immersed HGL radiation heat flux predictions is given in Fig.~\ref{immersed_HGL_heat_flux_summary}.

\begin{figure}[ht]
\begin{center}
\begin{tabular}{l}
\includegraphics[width=4.0in]{FIGURES/Scatterplots/Immersed_HGL_Radiation_Heat_Flux}
\end{tabular}
\end{center}
\caption[Summary of immersed HGL radiation heat flux predictions.]
{Summary of immersed HGL radiation heat flux predictions.}
\label{immersed_HGL_heat_flux_summary}
\end{figure}


\clearpage


\section{Immersed Plume Radiation Heat Flux}

The immersed plume radiation heat flux can be calculated by

\be
\dot q_r'' = \sigma (T_p^4)
\ee

\noindent where $\sigma$ is the Stefan-Boltzmann constant ($W/m^2/K^4$), and $T_p$ is the plume temperature (K) at a given height using the McCaffrey centerline plume temperature correlation described in Eq.~\ref{eq:McCaffrey}.

\clearpage

A summary scatter plot of immersed plume radiation heat flux predictions is given in Fig.~\ref{immersed_plume_heat_flux_summary}.

\begin{figure}[ht]
\begin{center}
\begin{tabular}{l}
\includegraphics[width=4.0in]{FIGURES/Scatterplots/Immersed_Plume_Radiation_Heat_Flux}
\end{tabular}
\end{center}
\caption[Summary of immersed plume radiation heat flux predictions.]
{Summary of immersed plume radiation heat flux predictions.}
\label{immersed_plume_heat_flux_summary}
\end{figure}



