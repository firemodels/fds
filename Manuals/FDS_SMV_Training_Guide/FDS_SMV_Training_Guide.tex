\documentclass[11pt]{book}
\usepackage{times,mathptm,moreverb}
\usepackage[pdftex]{graphicx}
\usepackage[pdftex,
        colorlinks=true,
        urlcolor=linkblue,     % \href{...}{...} external (URL)
        citecolor=linkred,     % citation number colors
        linkcolor=linknavy,    % \ref{...} and \pageref{...}
        pdftitle={Fire Fighter Trainer Scenario Guide},
        pdfauthor={Glenn Forney, Kevin McGrattan, Daniel Madrykowski and Stephen Kerber},
        pdfsubject={User Guide},
        pdfkeywords={FDS, Fire Model, NIST, BFRL},
        pdfproducer={pdflatex},
        pagebackref,
        pdfpagemode=UseNone,
        bookmarksopen=true,
        plainpages=false]{hyperref}
\usepackage{color}
\definecolor{linknavy}{rgb}{0,0,0.50196}
\definecolor{linkred}{rgb}{1,0,0}
\definecolor{linkblue}{rgb}{0,0,1}
\usepackage{caption}
\usepackage{graphpap}
\usepackage{rotating}
\usepackage{epsfig,psfrag}
%\usepackage{wrapfig}
%\usepackage{picins}
\usepackage{geometry}
\usepackage{tabularx}
\usepackage{longtable}
\usepackage{lscape}
\usepackage{amssymb}
\usepackage{makeidx} % Create index at end of document
\usepackage[nottoc,notlof,notlot]{tocbibind} % Put the bibliography and index in the ToC
\usepackage{float}
\usepackage{lastpage} % Automatic last page number reference.
\usepackage[T1]{fontenc}
\usepackage{upquote}
\usepackage{array}
\newcommand{\nopart}{\expandafter\def\csname Parent-1\endcsname{}} % To fix table of contents in pdf.
\newcommand{\ct}{\tt\small}

% The Following commented code makes the ``Draft'' watermark on each page.
%\usepackage{eso-pic}
%\usepackage{type1cm}
%\makeatletter
%   \AddToShipoutPicture{
%     \setlength{\@tempdimb}{.5\paperwidth}
%     \setlength{\@tempdimc}{.5\paperheight}
%     \setlength{\unitlength}{1pt}
%     \put(\strip@pt\@tempdimb,\strip@pt\@tempdimc){
%     \makebox(0,0){\rotatebox{45}{\textcolor[gray]{0.75}{\fontsize{8cm}\selectfont{RC6}}}}}
% }
%\makeatother

\newcommand{\figheightA}{1.5in}
\newcommand{\fdsinput}[1]{
{
\scriptsize
\verbatiminput{../../Training/#1}
}
}
\setlength{\textwidth}{6.5in}
\setlength{\textheight}{9.0in}
\setlength{\topmargin}{0.in}
\setlength{\headheight}{0.in}
\setlength{\headsep}{0.in}
\setlength{\parindent}{0.25in}
\setlength{\oddsidemargin}{0.0in}
\setlength{\evensidemargin}{0.0in}

\newcommand{\be}{\begin{equation}}
\newcommand{\ee}{\end{equation}}
\newcommand{\note}{{\bf Note:}}

\begin{document}
\bibliographystyle{unsrt}

\pagestyle{empty}
\pagenumbering{alph}

\begin{minipage}[t][9in][s]{6.25in}

\huge
\flushright{NIST Special Publication XXXX}

\vspace{1in}

\Huge \flushright{Fire Fighter Trainer \\ Scenario Guide }

\vspace{.5in}

\normalsize

\large
\flushright{
Glenn Forney \\
Kevin McGrattan \\
Daniel Madrzykowski \\
Stephen Kerber \\
 }


\vfill

\flushright{\includegraphics[width=2.in]{FIGURES/nistident_flright_vec}}


\end{minipage}

\newpage
\hspace{5in}
\newpage

\begin{minipage}[t][9in][s]{6.25in}

\huge
\flushright{NIST Special Publication XXXX}

\vspace{.75in}

\Huge
\flushright{Fire Fighter Trainer \\ Scenario Guide }

\vspace{.25in}

\normalsize
\flushright{
Glenn Forney \\
Kevin McGrattan \\
Daniel Madrzykowski \\
Stephen Kerber \\
{\em NIST Building and Fire Research Laboratory} \\
{\em Gaithersburg, Maryland, USA}}

\vspace{.25in}

\flushright{\today \\
$SVN Repository$~$Revision$}

\vfill

\flushright{\includegraphics[width=1in]{FIGURES/doc.pdf} }

\small
\flushright{U.S. Department of Commerce \\
{\em Carlos M. Gutierrez, Secretary} \\
\hspace{1in} \\
National Institute of Standards and Technology \\
{\em Patrick Gallagher, Acting Director} }

\end{minipage}

\newpage

\begin{minipage}[t][9in][s]{6.25in}

\flushright{Certain commercial entities, equipment, or materials may be identified in this \\
document in order to describe an experimental procedure or concept adequately. Such \\
identification is not intended to imply recommendation or endorsement by the \\
National Institute of Standards and Technology, nor is it intended to imply that the \\
entities, materials, or equipment are necessarily the best available for the purpose.
}

\vspace{3in}

\large
\flushright{\bf National Institute of Standards and Technology Special Publication XXXX \\
Natl.~Inst.~Stand.~Technol.~Spec.~Publ.~XXXX, \pageref{LastPage} pages (October 2008) \\
CODEN: NSPUE2 }

\vfill

\flushright{U.S. GOVERNMENT PRINTING OFFICE \\
WASHINGTON: 2007 \\
\rule{3.5in}{0.01in} \\
For sale by the Superintendent of Documents, U.S. Government Printing Office \\
Internet: bookstore.gpo.gov -- Phone: (202) 512-1800 -- Fax: (202) 512-2250 \\
Mail: Stop SSOP, Washington, DC 20402-0001 }
\end{minipage}

\clearpage

\frontmatter

\pagestyle{plain}
\pagenumbering{roman}

% ---------------------------- Preface -----------------------------------

\chapter{Preface}

This Guide contains a description of experiments and corresponding numerical simulations performed in
facilities used  to train fire fighters.  It also contains descriptions of demonstration
numerical simulations designed to illustrate fundamental fire phenomena encountered while
fighting fire, phenomena such as how smoke layers behave in the presense or absence of
door openings.  These simulations are performed by the Fire Dynamics Simulator (FDS)
and visualized  by the visualization program, Smokeview.

Note that this Guide does not provide the background theory for FDS nor a detailed description of Smokeview. The FDS
Technical Reference Guide~\cite{FDS_Tech_Guide_5} contains details about the governing
equations and numerical methods. The FDS User's Guide~\cite{FDS_Users_Guide_5} describes how to run FDS. Smokeview is described in the ``User's Guide for
Smokeview Version~5''~\cite{Smokeview_Users_Guide_5}.

% ---------------------------- Disclaimer -----------------------------------

\chapter{Disclaimer}

The US Department of Commerce makes no warranty, expressed or implied, to
users of the Fire Dynamics Simulator (FDS) or Smokeview, and accepts no responsibility for their
use. Users of FDS and Smokeview assume sole responsibility under Federal law for
determining the appropriateness of its use in any particular application;
for any conclusions drawn from the results of its use; and for any actions
taken or not taken as a result of analyses performed using these tools.

Users are warned that FDS and Smokeview are intended for use only by those competent in
the fields of fluid dynamics, thermodynamics, combustion, and heat transfer,
and is intended only to supplement the
informed judgment of the qualified user. The software package is a
computer model that may or may not have predictive capability when applied
to a specific set of factual circumstances. Lack of accurate predictions by
the model could lead to erroneous conclusions with regard to fire safety.
All results should be evaluated by an informed user.

Throughout this document, the mention of computer hardware or
commercial software does not constitute endorsement by NIST, nor does
it indicate that the products are necessarily those best suited for the
intended purpose.

% ---------------------------- Acknowledgments -----------------------------------

\chapter{Acknowledgments}

\tableofcontents
\listoffigures
%\listoftables

\mainmatter

% ---------------------------- Introduction -----------------------------------

\chapter{Introduction}
The most effective method for learning how to fight fire is to fight fire.  This however is expensive and
particularly dangerous for the trainee.   In particular, some fire situations that must be trained for are too
large and dangerous to re-create in a training setting involving real fires.  Environmental concerns are
limiting the amount and kind of live fire training available in many areas of the country.  Methods are then
needed to allow fire fighters to gain valuable experience using virtual reality techniques already applied in
other fields so that they may learn without the possibility of harming themselves or others.  It is a challenge
to simulate fire fighter training scenarios that are accurate and to present them in a form that are realistic
and practical due to the large amounts of computational resources required.   Presently, fire fighter trainers
either concentrate on incident command issues or depend on an {\em expert}\ to alter the fire.
The trainer developed
here will be Physics based so that the fire and smoke visualized will be closer to what one would expect to find
when fighting real fires.

Presently, fire fighters train in the classroom and in live fire training using real fires in real
facilities.  Visual based training presently relies on experts to run the fire and are not physics based.
This project will continue to develop a computer based fire fighting training tool to improve training
opportunities while lowering the cost and risk of death and injury.   Two methods are being used to create a
training tool.   The first and simpler method  is to use FDS and Smokeview to create animations of fire
scenarios.  These animations will be viewable in a standard DVD player.  The DVD menus will be used to walk a
fire fighter through a series of decisions.  The second method will be more interactive.  Using Smokeview,
the trainee will {\em walk}\ through an FDS generated fire scene observing and making decisions.  Several  scenarios
will be simulated with FDS.  These scenarios involve cases where experimental data is available; the Maryland
Fire Research Institute's (MFRI) training tower and the Montgomery County's flashover trainer.  In subsequent
years, more generic scenarios will be modeled; a ranch house (one level) and a townhouse
(multi-level).

NIST is uniquely positioned to apply research to this problem that it developed for modeling fire and smoke
movement.  The fire science software tools, NIST FDS and Smokeview will be used as a basis for developing a
virtual reality based fire fighting training tool.  As a fire simulation unfolds a series of natural break point
will be encountered where the trainee will be asked questions such as: Should the window be opened?  Should the
door be closed? Should the hose stream be opened?  The simulation will continue based on the answers provided.
These questions will be designed to teach the trainee about tactics relevant to fire fighting such as the
effect ventilation on fire spread.    Since FDS cannot yet perform calculations in real time, simulations
will be pre-computed for all possible question outcomes.  Smokeview will be enhanced to be able to {\em jump}
from one scenario to another according to the trainee's responses.

\section{Numerical Fire Fighting Training Tools}
Two software tools are used to generate the Physics based fire fighter training scenarios
described in this document.  The first tool is the Fire Dynamics Simulator (FDS).
It is used to perform numerical simulations using established Physics based algorithms.
It is a computational
fluid dynamics or CFD model of fire-driven fluid flow. FDS solves a form of the
Navier-Stokes equations appropriate for low-speed, thermally-driven flow
with an emphasis on smoke and heat transport from fires.
These approximations allow the computations to be performed both accurately and efficiently.

The second tool is called Smokeview.  It is used to visualize or interpret the data
simulated by FDS.  Smokeview {\em translates}\ data simulated by FDS into
various types of moving or animated images.  These images range in complexity from simple tracer particles,
to 2D shaded contours, 3D level surfaces and finally realistic looking smoke and fire.  These visualizations use
algorithms known as Beer's law that associate computed soot density with optical opacity.  They are also Physics based.  It cannot be emphasized enough
that displayed images are not cartoons, they are drawn based on numerical results generated by the fire model FDS.

\section{Overview}
This publication documents a series of FDS test cases used as a basis for fire fighter training.
These test cases fall into two logical groups, those that are based on a full scale experiments and those that are meant to be demonstrations of fire phenomena encountered by fire fighters.

The FDS scenarios that are experimentally based are documented by 1) briefly describing the experiment with references to a more complete description, 2) briefly describing the numerical modeling that was performed giving the relevant input files and 3) giving a comparison of some of the experimental data collected with the corresponding numerical results.

% +++++++++++++++++++++++++++++ Demonstration Cases +++++++++++++++++++++++++++

%\part{Demonstration Cases}
%The training scenarios presented in this part are designed to demonstrate fire flow phenomena often %encountered while fighting fire.


% ---------------------------- Two Room -----------------------------------
%\chapter{Two Room}
\section{Training Objectives}
The training scenarios in this chapter consist of a series of two room cases.  Each case uses different ventilation configurations
involving open or closed doors/windows and a vented/un-vented roof.  The training
objective then is to illustrate how these different ventilation conditions effect smoke and fire behavior.

\section{Design Data}
\subsection{Geometry}
The space, illustrated in Figure \ref{fig2roomplan}, consists of two rooms connected by an open door.  The first room contains a couch
with dimensions X x Y x Z and the second
room contains a a twin sized bed with dimension a x b.  The rooms
each have dimension X x Y x Z.  The interior and exterior doors both have
dimensions X x Y.

\begin{figure}[\figoptions]
\begin{center}
\includegraphics[trim=5.25in 0.75in 5.25in 4.25in clip, width=2.25in]{scriptfigures/ranch_00_side}\\
a) side view\\
\includegraphics[trim=0.75in 0.75in 1.75in 4.25in, clip, width=4.5in]{scriptfigures/ranch_00_front}\\
b) front view\\
\includegraphics[trim=0.75in 3.25in 1.75in 3.75in, clip, width=4.5in]{scriptfigures/ranch_00_top}\\
c) top view
\end{center}
\caption {Schematic of two room test case}
\label{fig2roomplan}%
\end{figure}

\begin{figure}[\figoptions]
\begin{center}
\includegraphics[trim=3.25in 0.25in 1.0in 4.5in clip, width=5.0in]{scriptfigures/ranch_00_3D}\\
\end{center}
\caption {3D view of a two room test case}
\label{fig2roomplan}%
\end{figure}

\subsection{Fire}
A couch is assumed to have caught fire due to a piloted ignition source of XX kW/m2 over
a A~$m^2$ area.


\subsection{Materials}
The walls and ceiling are standard gypsum board.  The floor is
covered with carpet.  Wall materials are assumed to be xxxx with thermal properties taken from yyy.  Materials for the couch
are adjusted so that the heat release curve a free-burn couch matches experimental data given by wwww.

\subsection{Ventilation}
Four cases are modeled each with different ventilation conditions.  Two configurations are varied.  The first configuration is whether the exterior doors and windows are open or closed.  The second configuration is whether the roof is vented or not.  The ventilation for these cases are summarized in Table \ref{tab:demoranch}.

\begin{table}[ht]
\caption{Summary of two room demonstration cases}
\vspace{0.1in}
\label{tab:demoranch}
\begin{center}
\begin{tabular}{|r||c|c|}
  \hline
   & vented roof & non-vented roof \\ \hline
  Open door and windows & Case 1 & Case 2 \\
  Leaky door and windows & Case 3 & Case 4 \\
  \hline
  \end{tabular}
\end{center}
\end{table}

\section{Physical Assumptions}
\subsection{Fire}
The fire is assumed to spread given an initial constant ignition in a corner
(back right) of the couch.

\subsection{Materials}
The thermal properties (thermal conductivity, density and specific heat) for gypsum board are used the wall, floor and ceiling materials.  These values are obtained from XXXX.  The thermal properties and combustion properties of the couch and twin mattresses were chosen to match experimental data obtained from ZZZZ.


\section{Model Assumptions and Analysis}
The two room scenarios are modeled using the Fire Dynamics Simulator and
visualized using Smokeview.

\subsection{Geometry}
The computational domain includes both the space required to model the two room
structure and a space surrounding the structure to account for air flow entering and leaving two room structure.  The domain is modeled using grid cells 0.1 m on a side.


\subsection{Fire}
The fire is modeled by imposing a XX constant fire on the back right corner of the
couch.  Fire growth occurs as a result of radiative and convective flux from the fire heating up and igniting spaces adjacent to the fire.

\subsection{Ventilation}
All ventilation is natural.  Open boundaries are used for boundary conditions for the side and top bounding surfaces.  Ventilation is varid from case to case by altering the inflow into the two room structure by opening or closing doors/windows and/or
venting or not venting the ceiling and roof space over the room with the couch.


\section{Summary of Results}


\begin{figure}[\figoptions]
\begin{center}
\begin{tabular}{cc}
 \includegraphics[height=2.5in]{datafigures/ranch_01_hrr}&
 \includegraphics[height=2.5in]{datafigures/ranch_02_hrr}\\
a) leaks&b) open doors/windows and vented ceiling\\
\\
\\
 \includegraphics[height=2.5in]{datafigures/ranch_03_hrr}&
 \includegraphics[height=2.5in]{datafigures/ranch_04_hrr}\\
c) open door/windows&d) vented ceiling\\
\end{tabular}
\end{center}
\caption{
FDS heat release rate results for four 2 room test cases.
  }
\label{fig2roomhrr}%
\end{figure}

\begin{figure}[\figoptions]
\begin{center}
\begin{tabular}{cc}
 \includegraphics[height=2.5in]{datafigures/ranch_01_thcp_a}&
 \includegraphics[height=2.5in]{datafigures/ranch_02_thcp_a}\\
a) leaks&b) open doors/windows and vented ceiling\\
\\
\\
 \includegraphics[height=2.5in]{datafigures/ranch_03_thcp_a}&
 \includegraphics[height=2.5in]{datafigures/ranch_04_thcp_a}\\
c) open door/windows&d) vented ceiling\\
\end{tabular}
\end{center}
\caption{FDS temperature results at location THCP A in the fire room
(see Figure \ref{fig2roomplan}) for four 2 room test cases.
  }
\label{fig2roomthcpa}%
\end{figure}

\begin{figure}[\figoptions]
\begin{center}
\begin{tabular}{cc}
 \includegraphics[height=2.5in]{datafigures/ranch_01_thcp_b}&
 \includegraphics[height=2.5in]{datafigures/ranch_02_thcp_b}\\
a) leaks&b) open doors/windows and vented ceiling\\
\\
\\
 \includegraphics[height=2.5in]{datafigures/ranch_03_thcp_b}&
 \includegraphics[height=2.5in]{datafigures/ranch_04_thcp_b}\\
c) open door/windows&d) vented ceiling\\
\end{tabular}
\end{center}
\caption{FDS temperature results at location THCP B in the target room
(see Figure \ref{fig2roomplan}) for four 2 room test cases.}
\label{fig2roomthcpa}%
\end{figure}


\begin{figure}[\figoptions]
\begin{center}
\begin{tabular}{ccc}
 \includegraphics[width=2.25in]{scriptfigures/ranch_01_temp_slice_030}&
 \includegraphics[width=2.25in]{scriptfigures/ranch_01_temp_slice_060}&
 \includegraphics[width=2.25in]{scriptfigures/ranch_01_temp_slice_090}
\\
 \includegraphics[width=2.25in]{scriptfigures/ranch_01_temp_slice_120}&
 \includegraphics[width=2.25in]{scriptfigures/ranch_01_temp_slice_150}&
 \includegraphics[width=2.25in]{scriptfigures/ranch_01_temp_slice_180}
\\
 \includegraphics[width=2.25in]{scriptfigures/ranch_01_temp_slice_210}&
 \includegraphics[width=2.25in]{scriptfigures/ranch_01_temp_slice_240}&
 \includegraphics[width=2.25in]{scriptfigures/ranch_01_temp_slice_270}
\\
 \includegraphics[width=2.25in]{scriptfigures/ranch_01_temp_slice_300}&
 \includegraphics[width=2.25in]{scriptfigures/ranch_01_temp_slice_330}&
 \includegraphics[width=2.25in]{scriptfigures/ranch_01_temp_slice_360}
\\
 \includegraphics[width=2.25in]{scriptfigures/ranch_01_temp_slice_390}&
 \includegraphics[width=2.25in]{scriptfigures/ranch_01_temp_slice_420}&
 \includegraphics[width=2.25in]{scriptfigures/ranch_01_temp_slice_450}
 %\includegraphics[width=1.5in]{scriptfigures/ranch_01_temp_slice_480}
\\
\end{tabular}
\end{center}
\caption{2 Room test case with only leaks to the outside showing shaded temperature contours from 30 to 480 s every 30 s.
  }
\label{fig2roomsmoke}%
\end{figure}


\begin{figure}[\figoptions]
\begin{center}
\begin{tabular}{ccc}
 \includegraphics[width=2.25in]{scriptfigures/ranch_02_temp_slice_030}&
 \includegraphics[width=2.25in]{scriptfigures/ranch_02_temp_slice_060}&
 \includegraphics[width=2.25in]{scriptfigures/ranch_02_temp_slice_090}
\\
 \includegraphics[width=2.25in]{scriptfigures/ranch_02_temp_slice_120}&
 \includegraphics[width=2.25in]{scriptfigures/ranch_02_temp_slice_150}&
 \includegraphics[width=2.25in]{scriptfigures/ranch_02_temp_slice_180}
\\
 \includegraphics[width=2.25in]{scriptfigures/ranch_02_temp_slice_210}&
 \includegraphics[width=2.25in]{scriptfigures/ranch_02_temp_slice_240}&
 \includegraphics[width=2.25in]{scriptfigures/ranch_02_temp_slice_270}
\\
 \includegraphics[width=2.25in]{scriptfigures/ranch_02_temp_slice_300}&
 \includegraphics[width=2.25in]{scriptfigures/ranch_02_temp_slice_330}&
 \includegraphics[width=2.25in]{scriptfigures/ranch_02_temp_slice_360}
\\
 \includegraphics[width=2.25in]{scriptfigures/ranch_02_temp_slice_390}&
 \includegraphics[width=2.25in]{scriptfigures/ranch_02_temp_slice_420}&
 \includegraphics[width=2.25in]{scriptfigures/ranch_02_temp_slice_450}
 %\includegraphics[width=1.5in]{scriptfigures/ranch_02_temp_slice_480}
\\
\end{tabular}
\end{center}
\caption{2 Room test case with open doors, windows and a vented roof showing shaded temperature contours from 30 to 480 s every 30 s.
  }
\label{fig2roomsmoke}%
\end{figure}


\begin{figure}[\figoptions]
\begin{center}
\begin{tabular}{ccc}
 \includegraphics[width=2.25in]{scriptfigures/ranch_03_temp_slice_030}&
 \includegraphics[width=2.25in]{scriptfigures/ranch_03_temp_slice_060}&
 \includegraphics[width=2.25in]{scriptfigures/ranch_03_temp_slice_090}
\\
 \includegraphics[width=2.25in]{scriptfigures/ranch_03_temp_slice_120}&
 \includegraphics[width=2.25in]{scriptfigures/ranch_03_temp_slice_150}&
 \includegraphics[width=2.25in]{scriptfigures/ranch_03_temp_slice_180}
\\
 \includegraphics[width=2.25in]{scriptfigures/ranch_03_temp_slice_210}&
 \includegraphics[width=2.25in]{scriptfigures/ranch_03_temp_slice_240}&
 \includegraphics[width=2.25in]{scriptfigures/ranch_03_temp_slice_270}
\\
 \includegraphics[width=2.25in]{scriptfigures/ranch_03_temp_slice_300}&
 \includegraphics[width=2.25in]{scriptfigures/ranch_03_temp_slice_330}&
 \includegraphics[width=2.25in]{scriptfigures/ranch_03_temp_slice_360}
\\
 \includegraphics[width=2.25in]{scriptfigures/ranch_03_temp_slice_390}&
 \includegraphics[width=2.25in]{scriptfigures/ranch_03_temp_slice_420}&
 \includegraphics[width=2.25in]{scriptfigures/ranch_03_temp_slice_450}
 %\includegraphics[width=1.5in]{scriptfigures/ranch_03_temp_slice_480}
\\
\end{tabular}
\end{center}
\caption{2 Room test case with open doors, windows and a vented roof showing shaded temperature contours from 30 to 480 s every 30 s.
  }
\label{fig2roomsmoke}%
\end{figure}

\begin{figure}[\figoptions]
\begin{center}
\begin{tabular}{ccc}
 \includegraphics[width=2.25in]{scriptfigures/ranch_04_temp_slice_030}&
 \includegraphics[width=2.25in]{scriptfigures/ranch_04_temp_slice_060}&
 \includegraphics[width=2.25in]{scriptfigures/ranch_04_temp_slice_090}
\\
 \includegraphics[width=2.25in]{scriptfigures/ranch_04_temp_slice_120}&
 \includegraphics[width=2.25in]{scriptfigures/ranch_04_temp_slice_150}&
 \includegraphics[width=2.25in]{scriptfigures/ranch_04_temp_slice_180}
\\
 \includegraphics[width=2.25in]{scriptfigures/ranch_04_temp_slice_210}&
 \includegraphics[width=2.25in]{scriptfigures/ranch_04_temp_slice_240}&
 \includegraphics[width=2.25in]{scriptfigures/ranch_04_temp_slice_270}
\\
 \includegraphics[width=2.25in]{scriptfigures/ranch_04_temp_slice_300}&
 \includegraphics[width=2.25in]{scriptfigures/ranch_04_temp_slice_330}&
 \includegraphics[width=2.25in]{scriptfigures/ranch_04_temp_slice_360}
\\
 \includegraphics[width=2.25in]{scriptfigures/ranch_04_temp_slice_390}&
 \includegraphics[width=2.25in]{scriptfigures/ranch_04_temp_slice_420}&
 \includegraphics[width=2.25in]{scriptfigures/ranch_04_temp_slice_450}
 %\includegraphics[width=1.5in]{scriptfigures/ranch_04_temp_slice_480}
\\
\end{tabular}
\end{center}
\caption{2 Room test case with a vented roof showing shaded temperature contours from 30 to 480 s every 30 s.
  }
\label{fig2roomsmoke}%
\end{figure}

\begin{figure}[\figoptions]
\begin{center}
\begin{tabular}{ccc}
 \includegraphics[width=2.25in]{scriptfigures/ranch_01_3dsmoke_fire_030}&
 \includegraphics[width=2.25in]{scriptfigures/ranch_01_3dsmoke_fire_060}&
 \includegraphics[width=2.25in]{scriptfigures/ranch_01_3dsmoke_fire_090}
\\
 \includegraphics[width=2.25in]{scriptfigures/ranch_01_3dsmoke_fire_120}&
 \includegraphics[width=2.25in]{scriptfigures/ranch_01_3dsmoke_fire_150}&
 \includegraphics[width=2.25in]{scriptfigures/ranch_01_3dsmoke_fire_180}
\\
 \includegraphics[width=2.25in]{scriptfigures/ranch_01_3dsmoke_fire_210}&
 \includegraphics[width=2.25in]{scriptfigures/ranch_01_3dsmoke_fire_240}&
 \includegraphics[width=2.25in]{scriptfigures/ranch_01_3dsmoke_fire_270}
\\
 \includegraphics[width=2.25in]{scriptfigures/ranch_01_3dsmoke_fire_300}&
 \includegraphics[width=2.25in]{scriptfigures/ranch_01_3dsmoke_fire_330}&
 \includegraphics[width=2.25in]{scriptfigures/ranch_01_3dsmoke_fire_360}
\\
 \includegraphics[width=2.25in]{scriptfigures/ranch_01_3dsmoke_fire_390}&
 \includegraphics[width=2.25in]{scriptfigures/ranch_01_3dsmoke_fire_420}&
 \includegraphics[width=2.25in]{scriptfigures/ranch_01_3dsmoke_fire_450}
 %\includegraphics[width=1.5in]{scriptfigures/ranch_01_3dsmoke_fire_480}
\\
\end{tabular}
\end{center}
\caption{2 Room test case with only leaks to the outside showing realistic smoke from 30 to 480 s every 30 s.
  }
\label{fig2roomsmoke}%
\end{figure}


\begin{figure}[\figoptions]
\begin{center}
\begin{tabular}{ccc}
 \includegraphics[width=2.25in]{scriptfigures/ranch_02_3dsmoke_fire_030}&
 \includegraphics[width=2.25in]{scriptfigures/ranch_02_3dsmoke_fire_060}&
 \includegraphics[width=2.25in]{scriptfigures/ranch_02_3dsmoke_fire_090}
\\
 \includegraphics[width=2.25in]{scriptfigures/ranch_02_3dsmoke_fire_120}&
 \includegraphics[width=2.25in]{scriptfigures/ranch_02_3dsmoke_fire_150}&
 \includegraphics[width=2.25in]{scriptfigures/ranch_02_3dsmoke_fire_180}
\\
 \includegraphics[width=2.25in]{scriptfigures/ranch_02_3dsmoke_fire_210}&
 \includegraphics[width=2.25in]{scriptfigures/ranch_02_3dsmoke_fire_240}&
 \includegraphics[width=2.25in]{scriptfigures/ranch_02_3dsmoke_fire_270}
\\
 \includegraphics[width=2.25in]{scriptfigures/ranch_02_3dsmoke_fire_300}&
 \includegraphics[width=2.25in]{scriptfigures/ranch_02_3dsmoke_fire_330}&
 \includegraphics[width=2.25in]{scriptfigures/ranch_02_3dsmoke_fire_360}
\\
 \includegraphics[width=2.25in]{scriptfigures/ranch_02_3dsmoke_fire_390}&
 \includegraphics[width=2.25in]{scriptfigures/ranch_02_3dsmoke_fire_420}&
 \includegraphics[width=2.25in]{scriptfigures/ranch_02_3dsmoke_fire_450}
 %\includegraphics[width=1.5in]{scriptfigures/ranch_02_3dsmoke_fire_480}
\\
\end{tabular}
\end{center}
\caption{2 Room test case with open doors, windows and a vented roof showing realistic smoke from 30 to 480 s every 30 s.
  }
\label{fig2roomsmoke}%
\end{figure}

\begin{figure}[\figoptions]
\begin{center}
\begin{tabular}{ccc}
 \includegraphics[width=2.25in]{scriptfigures/ranch_03_3dsmoke_fire_030}&
 \includegraphics[width=2.25in]{scriptfigures/ranch_03_3dsmoke_fire_060}&
 \includegraphics[width=2.25in]{scriptfigures/ranch_03_3dsmoke_fire_090}
\\
 \includegraphics[width=2.25in]{scriptfigures/ranch_03_3dsmoke_fire_120}&
 \includegraphics[width=2.25in]{scriptfigures/ranch_03_3dsmoke_fire_150}&
 \includegraphics[width=2.25in]{scriptfigures/ranch_03_3dsmoke_fire_180}
\\
 \includegraphics[width=2.25in]{scriptfigures/ranch_03_3dsmoke_fire_210}&
 \includegraphics[width=2.25in]{scriptfigures/ranch_03_3dsmoke_fire_240}&
 \includegraphics[width=2.25in]{scriptfigures/ranch_03_3dsmoke_fire_270}
\\
 \includegraphics[width=2.25in]{scriptfigures/ranch_03_3dsmoke_fire_300}&
 \includegraphics[width=2.25in]{scriptfigures/ranch_03_3dsmoke_fire_330}&
 \includegraphics[width=2.25in]{scriptfigures/ranch_03_3dsmoke_fire_360}
\\
 \includegraphics[width=2.25in]{scriptfigures/ranch_03_3dsmoke_fire_390}&
 \includegraphics[width=2.25in]{scriptfigures/ranch_03_3dsmoke_fire_420}&
 \includegraphics[width=2.25in]{scriptfigures/ranch_03_3dsmoke_fire_450}
 %\includegraphics[width=1.5in]{scriptfigures/ranch_03_3dsmoke_fire_480}
\\
\end{tabular}
\end{center}
\caption{2 Room test case with open doors, windows and a vented roof showing realistic smoke from 30 to 480 s every 30 s.
  }
\label{fig2roomsmoke}%
\end{figure}


\begin{figure}[\figoptions]
\begin{center}
\begin{tabular}{ccc}
 \includegraphics[width=2.25in]{scriptfigures/ranch_04_3dsmoke_fire_030}&
 \includegraphics[width=2.25in]{scriptfigures/ranch_04_3dsmoke_fire_060}&
 \includegraphics[width=2.25in]{scriptfigures/ranch_04_3dsmoke_fire_090}
\\
 \includegraphics[width=2.25in]{scriptfigures/ranch_04_3dsmoke_fire_120}&
 \includegraphics[width=2.25in]{scriptfigures/ranch_04_3dsmoke_fire_150}&
 \includegraphics[width=2.25in]{scriptfigures/ranch_04_3dsmoke_fire_180}
\\
 \includegraphics[width=2.25in]{scriptfigures/ranch_04_3dsmoke_fire_210}&
 \includegraphics[width=2.25in]{scriptfigures/ranch_04_3dsmoke_fire_240}&
 \includegraphics[width=2.25in]{scriptfigures/ranch_04_3dsmoke_fire_270}
\\
 \includegraphics[width=2.25in]{scriptfigures/ranch_04_3dsmoke_fire_300}&
 \includegraphics[width=2.25in]{scriptfigures/ranch_04_3dsmoke_fire_330}&
 \includegraphics[width=2.25in]{scriptfigures/ranch_04_3dsmoke_fire_360}
\\
 \includegraphics[width=2.25in]{scriptfigures/ranch_04_3dsmoke_fire_390}&
 \includegraphics[width=2.25in]{scriptfigures/ranch_04_3dsmoke_fire_420}&
 \includegraphics[width=2.25in]{scriptfigures/ranch_04_3dsmoke_fire_450}
 %\includegraphics[width=1.5in]{scriptfigures/ranch_04_3dsmoke_fire_480}
\\
\end{tabular}
\end{center}
\caption{2 Room test case with a vented roof showing realistic smoke
from 30 to 480 s every 30 s.
  }
\label{fig2roomsmoke}%
\end{figure}


\begin{figure}[\figoptions]
\begin{center}
\begin{tabular}{cc}
 \includegraphics[height=2.5in]{scriptfigures/ranch_01_3dsmoke_fire_300}&
 \includegraphics[height=2.5in]{scriptfigures/ranch_02_3dsmoke_fire_300}\\
a) leaks&b) open doors/windows and vented ceiling\\
\\
\includegraphics[height=2.5in]{scriptfigures/ranch_03_3dsmoke_fire_300}&
\includegraphics[height=2.5in]{scriptfigures/ranch_04_3dsmoke_fire_300}\\
c) open door/windows&d) vented ceiling\\
\end{tabular}
\end{center}
\caption{2 Room test cases showing realistic smoke and fire at 300.0 seconds to illustrate the effect of openings on smoke flow.
  }
\label{fig2roomsmoke}%
\end{figure}

\begin{figure}[\figoptions]
\begin{center}
\begin{tabular}{ccc}
 \includegraphics[height=2.25in]{scriptfigures/ranch_01_temp_slice_300}&
 \includegraphics[height=2.25in]{scriptfigures/ranch_02_temp_slice_300}\\
a) leaks&b) open doors/windows and vented ceiling\\
\includegraphics[height=2.25in]{scriptfigures/ranch_03_temp_slice_300}&
\includegraphics[height=2.25in]{scriptfigures/ranch_04_temp_slice_300}\\
c) open door/windows&d) vented ceiling\\
&&\raisebox{0.0ex}[0pt]{\includegraphics[height=5.0in]{figures/colorbar_20_620}}\\
\end{tabular}
\end{center}
\caption{2 Room test cases showing 2D temperature contours at 300.0 seconds to illustrate the effect of openings on smoke flow.
  }
\label{fig2roomslice}%
\end{figure}

\section{Conclusions}


% +++++++++++++++++++++++++++++ Experimental Cases +++++++++++++++++++++++++++

%\part{Experimental Based Cases}

% ----------------- Flashover Trainer -----------------------------------

\chapter{Flashover Trainer}

\section{Training Objectives}

\section{Design Data}
\subsection{Geometry}
\begin{figure}[\figoptions]
\begin{center}
\includegraphics[trim=1.0in 3.0inin 0.5in 4.5in, clip, width=6.0in]{scriptfigures/MCFRS_Flashover_00_top}\\
a) top view\\
\includegraphics[trim=1.0 2.25in 0.5in 4.25in, clip, width=6.0in]{scriptfigures/MCFRS_Flashover_00_front}\\
b) front view\\
\end{center}
\caption {Schematic of flashover trainer layout}
\label{figflashoverplan}%
\end{figure}

\begin{figure}[\figoptions]
\begin{center}
\includegraphics[trim=1.5 1.0in 1.0in 3.75in, clip, width=6.0in]{scriptfigures/MCFRS_Flashover_00_3D}\\
\end{center}
\caption {Schematic of flashover trainer layout}
\label{figflashoverplan}%
\end{figure}

\note\ Need the following (remove from check list when we get it):
\begin{enumerate}
\item drawings containing measurements of facility,
\item photos showing outside of facility before the test and inside at various times during the test,
\item test report describing
measurements taken, where they were taken
\item a cross-walk between measurements taken and where these measurements are recorded (ie
in what column of a spread sheet file).
\end{enumerate}

\subsection{Fire}

\subsection{Materials}

\subsection{Ventilation}

\section{Physical Assumptions}
\subsection{Fire}

\subsection{Materials}

\section{Model Assumptions and Analysis}
\subsection{Geometry}
\begin{figure}[\figoptions]
\begin{center}
\includegraphics[height=4.0in]{FIGURES/mcfrs_flashover}
\end{center}
\caption {Flashover trainer at MCFRS}
\label{figflashoversmoke}%
\end{figure}

\subsection{Fire}

\subsection{Ventilation}

\section{Summary of Results}

\begin{figure}[\figoptions]
\begin{center}
\begin{tabular}{cc}
 \includegraphics[height=2.0in]{scriptfigures/flashover_slice_temp_020}&
 \includegraphics[height=2.0in]{scriptfigures/flashover_slice_oxy_020}
 \\
 \includegraphics[height=2.0in]{scriptfigures/flashover_slice_temp_040}&
 \includegraphics[height=2.25in]{scriptfigures/flashover_slice_oxy_040}
 \\
 \includegraphics[height=2.0in]{scriptfigures/flashover_slice_temp_060}&
 \includegraphics[height=2.0in]{scriptfigures/flashover_slice_oxy_060}
 \\
 \includegraphics[height=2.0in]{scriptfigures/flashover_slice_temp_080}&
 \includegraphics[height=2.0in]{scriptfigures/flashover_slice_oxy_080}
 \\
\end{tabular}
\end{center}
\caption {Smokeview visualizations of FDS Flashover trainer simulations comparing temperature and oxygen
results from 20 s to 80 s every 20 s.}
\label{figflashoversmoke}%
\end{figure}

\begin{figure}[\figoptions]
\begin{center}
\begin{tabular}{cc}
 \includegraphics[height=2.25in]{scriptfigures/flashover_3dsmoke_fire_020}&
 \includegraphics[height=2.25in]{scriptfigures/flashover_3dsmoke_ifire_020}
 \\
 \includegraphics[height=2.25in]{scriptfigures/flashover_3dsmoke_fire_040}&
 \includegraphics[height=2.25in]{scriptfigures/flashover_3dsmoke_ifire_040}
 \\
 \includegraphics[height=2.25in]{scriptfigures/flashover_3dsmoke_fire_060}&
 \includegraphics[height=2.25in]{scriptfigures/flashover_3dsmoke_ifire_060}
 \\
 \includegraphics[height=2.25in]{scriptfigures/flashover_3dsmoke_fire_080}&
 \includegraphics[height=2.25in]{scriptfigures/flashover_3dsmoke_ifire_080}
 \\
\end{tabular}
\end{center}
\caption {FDS simulation results of the flashover trainer visualized
using realistic smoke and fire. Front and side views are compared from
20 s to 80 s every 20 s.}
\label{figflashoversmoke}%
\end{figure}

\begin{figure}[\figoptions]
\begin{center}
\begin{tabular}{c}
 \includegraphics[width=4.5in]{datafigures/MCFRS_Flashover_00leak_hrr}\\
 \includegraphics[width=4.5in]{datafigures/MCFRS_Flashover_00open_hrr}\\
\end{tabular}
\end{center}
\caption {FDS heat release rate results for the Flashover trainer case}
\label{figflashoverhrr}%
\end{figure}

\begin{figure}[\figoptions]
\begin{center}
\begin{tabular}{cc}
\includegraphics[width=3.0in]{datafigures/MCFRS_Flashover_00leakLL}&
\includegraphics[width=3.0in]{datafigures/MCFRS_Flashover_00leakMM}\\
\includegraphics[width=3.0in]{datafigures/MCFRS_Flashover_00leakRR}&
\includegraphics[width=3.0in]{datafigures/MCFRS_Flashover_00openLL}\\
\includegraphics[width=3.0in]{datafigures/MCFRS_Flashover_00openMM}&
\includegraphics[width=3.0in]{datafigures/MCFRS_Flashover_00openRR}\\
\end{tabular}
\end{center}
\caption {FDS temperatures as a function of height
at 3 locations within the MCFRS flashover trainer at 60 s, 120 s and 180 s.  See Figure \ref{figflashoverplan}a for location details.}
\label{figflashovertempa}%
\end{figure}

\begin{figure}[\figoptions]
\begin{center}
\begin{tabular}{cc}
\includegraphics[width=3.0in]{datafigures/MCFRS_Flashover_00leak_temp_0060}&
\includegraphics[width=3.0in]{datafigures/MCFRS_Flashover_00open_temp_0060}\\
\includegraphics[width=3.0in]{datafigures/MCFRS_Flashover_00leak_temp_0120}&
\includegraphics[width=3.0in]{datafigures/MCFRS_Flashover_00open_temp_0120}\\
\includegraphics[width=3.0in]{datafigures/MCFRS_Flashover_00leak_temp_0180}&
\includegraphics[width=3.0in]{datafigures/MCFRS_Flashover_00open_temp_0180}\\
\end{tabular}
\end{center}
\caption {FDS temperatures as a function of height 
at 3 locations within the MCFRS flashover trainer at 60 s, 120 s and 180 s.  See Figure \ref{figflashoverplan}a for location details.}
\label{figflashovertempa}%
\end{figure}

\begin{figure}[\figoptions]
\begin{center}
\begin{tabular}{rcc}
THCB A& \includegraphics[width=3.0in]{FIGURES/placeholder}&
\includegraphics[width=3.0in]{datafigures/MCFRS_Flashover_00_temp_a}\\
THCB B& \includegraphics[width=3.0in]{FIGURES/placeholder}&
\includegraphics[width=3.0in]{datafigures/MCFRS_Flashover_00_temp_b}\\
THCB A& \includegraphics[width=3.0in]{FIGURES/placeholder}&
\includegraphics[width=3.0in]{datafigures/MCFRS_Flashover_00_temp_c}\\
&experiment&FDS calculation\\
\end{tabular}
\end{center}
\caption {Comparison of experimental and FDS temperatures results
at 3 locations within the MCFRS flashover trainer.  See Figure \ref{figflashoverplan}a for location details.}
\label{figflashovertempa}%
\end{figure}

\section{Conclusions}


% --------------- Training Tower -----------------------------------

\chapter{Training Tower}

\section{Training Objectives}

\section{Design Data}
\subsection{Geometry}

%\begin{figure}[\figoptions]
%\begin{center}
%\includegraphics[width=6.5in]{FIGURES/MFRI_tower_setup}
%\end{center}
%\caption {Schematic of the training tower layout}
%\label{figtowerplan}%
%\end{figure}

\begin{figure}[\figoptions]
\begin{center}
\includegraphics[width=5.0in]{scriptfigures/MFRItower_00_top}\\
a) top view\\
\includegraphics[width=5.0in]{scriptfigures/MFRItower_00_3D}\\
a) 3D perspective view\\
\end{center}
\caption {Schematic of training tower layout}
\label{figflashoverplan}%
\end{figure}
\subsection{Fire}

\subsection{Materials}

\subsection{Ventilation}

\section{Physical Assumptions}
\subsection{Fire}

\subsection{Materials}

\section{Model Assumptions and Analysis}
\subsection{Geometry}

\subsection{Fire}

\subsection{Ventilation}

\section{Summary of Results}

\begin{figure}[\figoptions]
\begin{center}
\begin{tabular}{c}
 \includegraphics[width=3.0in]{datafigures/MFRI_Training_Tower_min_hrr}\\
 minimum HRR\\
 \\
 \includegraphics[width=3.0in]{datafigures/MFRI_Training_Tower_avg_hrr}\\
 average HRR\\
 \\
 \includegraphics[width=3.0in]{datafigures/MFRI_Training_Tower_max_hrr}\\
 maximum HRR\\
\end{tabular}
\end{center}
\caption {Experimental and FDS heat release results for the
Training Tower case}
\label{figtrainingtowerhrr}%
\end{figure}

\begin{figure}[\figoptions]
\begin{center}
\begin{tabular}{cc}
\includegraphics[width=3.0in]{datafigures/MFRI_Training_Tower_01_temp_all_layer_060}&
\includegraphics[width=3.0in]{datafigures/MFRI_Training_Tower_02_temp_all_layer_060}\\
Test 1 at 60 s&Test 2 at 60 s\\

\includegraphics[width=3.0in]{datafigures/MFRI_Training_Tower_01_temp_all_layer_120}&
\includegraphics[width=3.0in]{datafigures/MFRI_Training_Tower_02_temp_all_layer_120}\\
Test 1 at 120 s&Test 2 at 120 s\\

\includegraphics[width=3.0in]{datafigures/MFRI_Training_Tower_01_temp_all_layer_180}&
\includegraphics[width=3.0in]{datafigures/MFRI_Training_Tower_02_temp_all_layer_180}\\
Test 1 at 180 s&Test 2 at 180 s\\
\end{tabular}
\end{center}
\caption[Comparison of temperature as a function of elevation at 60 s, 120 s and 180 s for tests 1 and 2.]
{
Comparison of experimental and FDS generated temperatures for tests 1 and 2 at 60 s, 120 s and 180 s.
Temperatures were compared at 8 sensor locations spaced vertically  (0.3 m apart) located from floor to ceiling in the fire room.
The heat release rates used to generate the numerical results was obtained from Figure ??.
}
\label{figtrainingtowerhrr}%
\end{figure}

\begin{figure}[\figoptions]
\begin{center}
\begin{tabular}{cc}
\includegraphics[width=3.0in]{datafigures/MFRI_Training_Tower_03_temp_all_layer_060}&
\includegraphics[width=3.0in]{datafigures/MFRI_Training_Tower_04_temp_all_layer_060}\\
Test 3 at 60 s&Test 4 at 60 s\\

\includegraphics[width=3.0in]{datafigures/MFRI_Training_Tower_03_temp_all_layer_120}&
\includegraphics[width=3.0in]{datafigures/MFRI_Training_Tower_04_temp_all_layer_120}\\
Test 3 at 120 s&Test 4 at 120 s\\

\includegraphics[width=3.0in]{datafigures/MFRI_Training_Tower_03_temp_all_layer_180}&
\includegraphics[width=3.0in]{datafigures/MFRI_Training_Tower_04_temp_all_layer_180}\\
Test 3 at 180 s&Test 4 at 180 s\\
\end{tabular}
\end{center}
\caption[Comparison of temperature as a function of elevation at 60 s, 120 s and 180 s for tests 3 and 4.]
{
Comparison of experimental and FDS generated temperatures for tests 3 and 4 at 60 s, 120 s and 180 s.
Temperatures were compared at 8 sensor locations spaced vertically  (0.3 m apart) located from floor to ceiling in the fire room.
The heat release rates used to generate the numerical results was obtained from Figure ??.
}
\label{figtrainingtowerhrr}%
\end{figure}

\begin{figure}[\figoptions]
\begin{center}
\begin{tabular}{cc}
\includegraphics[width=3.0in]{datafigures/MFRI_Training_Tower_05_temp_all_layer_060}&
\includegraphics[width=3.0in]{datafigures/MFRI_Training_Tower_06_temp_all_layer_060}\\
Test 5 at 60 s&Test 6 at 60 s\\

\includegraphics[width=3.0in]{datafigures/MFRI_Training_Tower_05_temp_all_layer_120}&
\includegraphics[width=3.0in]{datafigures/MFRI_Training_Tower_06_temp_all_layer_120}\\
Test 5 at 120 s&Test 6 at 120 s\\

\includegraphics[width=3.0in]{datafigures/MFRI_Training_Tower_05_temp_all_layer_180}&
\includegraphics[width=3.0in]{datafigures/MFRI_Training_Tower_06_temp_all_layer_180}\\
Test 5 at 180 s&Test 6 at 180 s\\
\end{tabular}
\end{center}
\caption[Comparison of temperature as a function of elevation at 60 s, 120 s and 180 s for tests 5 and 6.]
{
Comparison of experimental and FDS generated temperatures for tests 5 and 6 at 60 s, 120 s and 180 s.
Temperatures were compared at 8 sensor locations spaced vertically  (0.3 m apart) located from floor to ceiling in the fire room.
The heat release rates used to generate the numerical results was obtained from Figure ??.
}
\label{figtrainingtowerhrr}%
\end{figure}

\begin{figure}[\figoptions]
\begin{center}
\begin{tabular}{c}
\includegraphics[width=3.0in]{datafigures/MFRI_Training_Tower_07_temp_all_layer_060}\\
Test 7 at 60 s\\

\includegraphics[width=3.0in]{datafigures/MFRI_Training_Tower_07_temp_all_layer_120}\\
Test 7 at 120 s\\

\includegraphics[width=3.0in]{datafigures/MFRI_Training_Tower_07_temp_all_layer_180}\\
Test 7 at 180 s\\
\end{tabular}
\end{center}
\caption[Comparison of temperature as a function of elevation at 60 s, 120 s and 180 s for test 7.]
{
Comparison of experimental and FDS generated temperatures for test 7 at 60 s, 120 s and 180 s.
Temperatures were compared at 8 sensor locations spaced vertically  (0.3 m apart) located from floor to ceiling in the fire room.
The heat release rates used to generate the numerical results was obtained from Figure ??.
}
\label{figtrainingtowerhrr}%
\end{figure}




\begin{figure}[\figoptions]
\begin{center}
\begin{tabular}{c}
\includegraphics[width=3.0in]{datafigures/MFRI_Training_Tower_01_temp_min}\\
FDS run with minimum HRR\\

\includegraphics[width=3.0in]{datafigures/MFRI_Training_Tower_01_temp_avg}\\
FDS run with average HRR\\

\includegraphics[width=3.0in]{datafigures/MFRI_Training_Tower_01_temp_max}\\
FDS run with maximum HRR\\
\end{tabular}
\end{center}
\caption[Comparison of temperature as a function of time for test 1 using 3 simulated HRR rates.] {
Comparison of experimental (test 1) and FDS generated temperatures for 3  simulated fire sizes as a function of time.
Temperatures were compared at 8 sensor locations spaced vertically  (0.3 m apart) located from floor to ceiling in the fire room.
The minimum, average and maximum heat release rate used to generate the numerical results was obtained from Figure ??.
}
\label{figtrainingtowerhrr}%
\end{figure}




\begin{figure}[\figoptions]
\begin{center}
\begin{tabular}{cc}
\includegraphics[width=3.0in]{datafigures/MFRI_Training_Tower_02_temp_avg}&
\includegraphics[width=3.0in]{datafigures/MFRI_Training_Tower_03_temp_avg}\\
Test 2&Test 3\\

\includegraphics[width=3.0in]{datafigures/MFRI_Training_Tower_04_temp_avg}&
\includegraphics[width=3.0in]{datafigures/MFRI_Training_Tower_05_temp_avg}\\
Test 4&Test 5\\

\includegraphics[width=3.0in]{datafigures/MFRI_Training_Tower_06_temp_avg}&
\includegraphics[width=3.0in]{datafigures/MFRI_Training_Tower_07_temp_avg}\\
Test 6&Test 7\\
\end{tabular}
\end{center}
\caption[Comparison of temperature as a function of time for tests 2 through 7.] {
Comparison of experimental and FDS generated temperatures for tests 2 through 7.
Temperatures were compared at 8 sensor locations spaced vertically  (0.3 m apart) located from floor to ceiling in the fire room.
The average heat release rate used to generate the numerical results was obtained from Figure ??.
}
\label{figtrainingtowerhrr}%
\end{figure}

\begin{figure}[\figoptions]
\begin{center}
\begin{tabular}{cc}
 \includegraphics[height=2.25in]{scriptfigures/MFRItower_000}&
 \includegraphics[height=2.25in]{scriptfigures/MFRItower_015}\\
a) 0 s&b) 15 s\\
 \includegraphics[height=2.25in]{scriptfigures/MFRItower_030}&
 \includegraphics[height=2.25in]{scriptfigures/MFRItower_060}\\
a) 30 s&b) 60 s\\
 \includegraphics[height=2.25in]{scriptfigures/MFRItower_120}&
 \includegraphics[height=2.25in]{scriptfigures/MFRItower_240}\\
a) 120 s&b) 240 s\\
\end{tabular}
\end{center}
\caption {Realistic visualizations of an FDS simulation of an experimental burn performed in the third floor of the MFRI training tower.}
\label{figtowersmoke}%
\end{figure}

\begin{figure}[\figoptions]
\begin{center}
\begin{tabular}{ccc}
 \includegraphics[height=2.25in]{scriptfigures/MFRItower_slice_temp_000}&
 \includegraphics[height=2.25in]{scriptfigures/MFRItower_slice_temp_015}\\
a) 0 s&b) 15 s\\
 \includegraphics[height=2.25in]{scriptfigures/MFRItower_slice_temp_030}&
 \includegraphics[height=2.25in]{scriptfigures/MFRItower_slice_temp_060}\\
a) 30 s&b) 60 s\\
 \includegraphics[height=2.25in]{scriptfigures/MFRItower_slice_temp_120}&
 \includegraphics[height=2.25in]{scriptfigures/MFRItower_slice_temp_240}\\
&&\raisebox{0.5in}[0pt]{\includegraphics[height=7.0in]{figures/colorbar_20_620}}\\
a) 120 s&b) 240 s\\
\end{tabular}
\end{center}
\caption {Shaded temperature slice visualizations of an FDS simulation of an experimental burn performed in the third floor of the MFRI training tower.}
\label{figtowersmoke}%
\end{figure}

\section{Conclusions}

% ---------------------------- Ranch House -----------------------------------

\chapter{Two Level Ranch House}

\section{Training Objectives}

\section{Design Data}
\subsection{Geometry}

\begin{figure}[\figoptions]
\begin{center}
\includegraphics[width=5.5in]{scriptfigures/MCFRS_Tower_00_front}\\
a) front view\\
\includegraphics[width=5.5in]{scriptfigures/MCFRS_Tower_00_3D}\\
b) 3D view\\
\end{center}
\caption{Schematic of the MCFRS ranch house layout}
\label{figflashoverplan}%
\end{figure}

\begin{figure}[\figoptions]
\begin{center}
\includegraphics[width=5.5in]{scriptfigures/MCFRS_Tower_00_2nd_top}\\
a) top view 2nd floor\\
\includegraphics[width=5.5in]{scriptfigures/MCFRS_Tower_00_1st_top}\\
b) top view 1st floor\\
\end{center}
\caption{Schematic of the MCFRS ranch house layout}
\label{figflashoverplan}%
\end{figure}

\subsection{Fire}

\subsection{Materials}

\subsection{Ventilation}

\section{Physical Assumptions}
\subsection{Fire}

\subsection{Materials}

\section{Model Assumptions and Analysis}
\subsection{Geometry}

\subsection{Fire}

\subsection{Ventilation}

\section{Summary of Results}

\begin{figure}[\figoptions]
\begin{center}
\begin{tabular}{cc}
 \includegraphics[height=2.5in]{datafigures/MCFRS_Tower_leak_4pal_hrr}&
 \includegraphics[height=2.5in]{datafigures/MCFRS_Tower_leak_6pal_hrr}\\
 leaks to outside, 4 pallet fire& leaks to outside, 6 pallet fire\\
 \\
 \includegraphics[height=2.5in]{datafigures/MCFRS_Tower_open_4pal_hrr}&
 \includegraphics[height=2.5in]{datafigures/MCFRS_Tower_open_6pal_hrr}\\
 open door and windows, 4 pallet fire& open door and window, 6 pallet fire\\
\end{tabular}
\end{center}
\caption {FDS heat release results for the MCFRS two level tower case}
\label{figMCFRStower_hrr}%
\end{figure}

\begin{figure}[\figoptions]
\begin{center}
\begin{tabular}{cc}
 \includegraphics[height=2.5in]{datafigures/MCFRS_Tower_leak_4pal_temp}&
 \includegraphics[height=2.5in]{datafigures/MCFRS_Tower_leak_6pal_temp}\\
 leaks to outside, 4 pallet fire& leaks to outside, 6 pallet fire\\
 \\
 \includegraphics[height=2.5in]{datafigures/MCFRS_Tower_open_4pal_temp}&
 \includegraphics[height=2.5in]{datafigures/MCFRS_Tower_open_6pal_temp}\\
 open door and windows, 4 pallet fire& open door and window, 6 pallet fire\\
\end{tabular}
\end{center}
\caption {FDS temperature results for the MCFRS two level tower case at
the 5 ft level at 4 locations within the structure}
\label{figMCFRStower_hrr}%
\end{figure}

\begin{figure}[\figoptions]
\begin{center}
\begin{tabular}{cc}
 \includegraphics[height=2.5in]{datafigures/MCFRS_Tower_leak_4pal_thcp}&
 \includegraphics[height=2.5in]{datafigures/MCFRS_Tower_leak_6pal_thcp}\\
 leaks to outside, 4 pallet fire& leaks to outside, 6 pallet fire\\
 \\
 \includegraphics[height=2.5in]{datafigures/MCFRS_Tower_open_4pal_thcp}&
 \includegraphics[height=2.5in]{datafigures/MCFRS_Tower_open_6pal_thcp}\\
 open door and windows, 4 pallet fire& open door and window, 6 pallet fire\\
\end{tabular}
\end{center}
\caption {FDS thermocouple temperature (THCP) results for the MCFRS two level tower case at
the 5 ft level at 4 locations within the structure}
\label{figMCFRStower_hrr}%
\end{figure}

\begin{figure}[\figoptions]
\begin{center}
\begin{tabular}{cc}
 \includegraphics[height=2.5in]{datafigures/MCFRS_Tower_leak_4pal_co}&
 \includegraphics[height=2.5in]{datafigures/MCFRS_Tower_leak_6pal_co}\\
 leaks to outside, 4 pallet fire& leaks to outside, 6 pallet fire\\
 \\
 \includegraphics[height=2.5in]{datafigures/MCFRS_Tower_open_4pal_co}&
 \includegraphics[height=2.5in]{datafigures/MCFRS_Tower_open_6pal_co}\\
 open door and windows, 4 pallet fire& open door and window, 6 pallet fire\\
\end{tabular}
\end{center}
\caption {FDS CO results for the MCFRS two level tower case at
the 5 ft level at 4 locations within the structure}
\label{figMCFRStower_hrr}%
\end{figure}

\begin{figure}[\figoptions]
\begin{center}
\begin{tabular}{cc}
 \includegraphics[height=2.5in]{datafigures/MCFRS_Tower_leak_4pal_co2}&
 \includegraphics[height=2.5in]{datafigures/MCFRS_Tower_leak_6pal_co2}\\
 leaks to outside, 4 pallet fire& leaks to outside, 6 pallet fire\\
 \\
 \includegraphics[height=2.5in]{datafigures/MCFRS_Tower_open_4pal_co2}&
 \includegraphics[height=2.5in]{datafigures/MCFRS_Tower_open_6pal_co2}\\
 open door and windows, 4 pallet fire& open door and window, 6 pallet fire\\
\end{tabular}
\end{center}
\caption {FDS CO2 results for the MCFRS two level tower case at
the 5 ft level at 4 locations within the structure}
\label{figMCFRStower_hrr}%
\end{figure}

\begin{figure}[\figoptions]
\begin{center}
\begin{tabular}{ccl}
 \includegraphics[height=2.25in]{scriptfigures/ranch_slice_temp_030}&
 \includegraphics[height=2.25in]{scriptfigures/ranch_slice_temp_060}\\
a) 30 s&b) 60 s\\
 \includegraphics[height=2.25in]{scriptfigures/ranch_slice_temp_120}&
 \includegraphics[height=2.25in]{scriptfigures/ranch_slice_temp_240}\\
&&\raisebox{0.5in}[0pt]{\includegraphics[height=5.0in]{figures/colorbar_20_620}}\\
a) 120 s&b) 240 s\\
\end{tabular}
\end{center}
\caption {2D slice visualization of the MCFRS ranch house case.}
\label{figranchsmoke}%
\end{figure}

\begin{figure}[\figoptions]
\begin{center}
\begin{tabular}{cc}
 \includegraphics[height=2.5in]{scriptfigures/ranch_smoke_030}&
 \includegraphics[height=2.5in]{scriptfigures/ranch_smoke_060}\\
a) 30 s&b) 60 s\\
 \includegraphics[height=2.5in]{scriptfigures/ranch_smoke_120}&
 \includegraphics[height=2.5in]{scriptfigures/ranch_smoke_240}\\
a) 120 s&b) 240 s\\
\end{tabular}
\end{center}
\caption {Smokeview visualization of the MCFRS ranch house case.}
\label{figranchsmoke}%
\end{figure}

\section{Conclusions}

\bibliography{../Bibliography/FDS_refs,../Bibliography/FDS_general,../Bibliography/FDS_mathcomp}

\appendix

\chapter{FDS Input Files}

\section{Two Room Ranch - Demonstration Case}
\fdsinput{Demonstrations/2Room_Ranch/ranch_00.fds}

\section{Flashover Trainer - MCFRS Case}
\fdsinput{MCFRS/MCFRS_Flashover/MCFRS_Flashover_00.fds}

\section{Two level ranch House - MCFRS Case}
\subsection{Open Doors and Windows}
\fdsinput{MCFRS/MCFRS_Tower/MCFRS_Tower_00.fds}

\subsection{Leaks}
\fdsinput{MCFRS/MCFRS_Tower/MCFRS_Tower_01.fds}

\section{Training Tower - MFRI Case}
\fdsinput{MFRI/MFRI_Training_Tower/MFRI_Training_Tower_01_avg.fds}


\end{document}


