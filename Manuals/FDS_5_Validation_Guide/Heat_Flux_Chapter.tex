\chapter{Heat Flux}




\section{NIST/NRC Test Series, Cables}

Cables in various types (power and control), and configurations (horizontal, vertical, in trays or free-hanging), were installed in
the test compartment.
For each of the four cable targets considered, measurements of the local gas temperature, surface temperature, radiative heat flux,
and total heat flux are available.  The following pages display comparisons of these quantities for
Control Cable B, Horizontal Cable Tray D, Power Cable F and Vertical Cable Tray G.
FDS does not have a detailed solid phase model that can account for the heat transfer within the bundled,
cylindrical, non-homogenous cables.  For the bundled cables within horizontal and vertical trays (Targets D and G),
FDS assumes them to be rectangular slabs of thickness comparable to the diameter of the individual cables.
For the free-hanging cables B and F, FDS assumes them to be cylinders of uniform composition into which it
computes the radial heat transfer as a function of the heat flux to a designated location.
The superposition of gas temperature, heat flux and surface temperature in the figures on the following pages
provides information about how cables heat up in fires.  Favorable or unfavorable predictions of cable surface
temperatures can often be explained in terms of comparable errors in the prediction of the thermal environment in the vicinity of the cable.




\section{NIST/WTC Test Series, Heat Fluxes}








\section{NIST/NRC Test Series, Compartment Walls, Floor and Ceiling}

Thirty-six heat flux gauges were positioned at various locations on all four walls of the compartment,
plus the ceiling and floor.  Comparisons between measured and predicted heat fluxes and surface temperatures are shown
on the following pages for a selected number of locations.
Over half of the measurement points are in roughly the same relative location to the fire and hence
the measurements and predictions are similar.  For this reason, data for the east and north walls are shown
because the data from the south and west walls are comparable.  Data from the south wall is used in cases where
the corresponding instrument on the north wall failed, or in cases where the fire is positioned close to the south wall.
For each test, eight locations are used for comparison, two on the long (mainly north) wall,
two on the short (east) wall, two on the floor, and two on the ceiling.  Of the two locations for each panel,
one is considered in the far-field, relatively remote from the fire; one is in the near-field,
relatively close to the fire.  How close or far varies from test to test, depending on the availability of working flux gauges.
The two short wall locations are equally remote from the fire; thus, one location is in the lower layer, one in the upper.
Table lists the locations for each test.
The heat flux gauges used on the compartment walls measured the net, not total, heat flux.
FDS predicts the net heat flux, but this prediction cannot be compared directly with the measured net heat
flux because the predicted and measured wall temperatures can differ, and this affects the net heat flux.
In a sense, the net heat flux and surface temperature are coupled, and it is difficult to assess the accuracy of the models
if the two quantities cannot be decoupled.  For the purpose of comparing prediction and measurement,
the following correction has been applied to both the measured and predicted net heat fluxes:
\be  \dq_{\hbox{\tiny total}}'' = \dq_{\hbox{\tiny net}}'' + \sigma (T_s^4-T_\infty^4) + h (T_s - T_\infty) \ee
where $T_s$ is the temperature of the surface.  A constant convective heat transfer coefficient is assumed
(5 W/m$^2$/K) and an emissivity of 1.
After applying the correction, it is easier to compare total heat fluxes that are independent of the surface temperature.


\begin{figure}[p]
\begin{center}
\begin{tabular}{c}
\includegraphics[width=4.0in]{FIGURES/ScatterPlots/Near_Field_Heat_Flux} \\
\vspace{0.25in} \\
\includegraphics[width=4.0in]{FIGURES/ScatterPlots/Far_Field_Heat_Flux}
\vspace{0.25in}
\end{tabular}
\caption{Summary of Heat Flux Results.}
\end{center}
\end{figure}





\clearpage

\section{Hamins Methane Burner Radiation Measurements}

Predicted and measured radial and vertical heat flux profiles from six experiments conducted by Anthony Hamins at NIST are shown on the following pages. The
relevant information about the fires is included in Table~\ref{Hamins_Table}. These are challenging simulations because the neither the gray gas assumption
nor the radiative fraction is assumed. Rather, the model is calculating the temperature and species concentrations necessary to predict the radiant energy from the fire.

\begin{table}[ht]
\label{Hamins_Table}
\begin{center}
\caption{Summary of Hamins methane burner experiments. }
\vspace{\baselineskip}
\begin{tabular}{|c|c|c|c|c|c|}
\hline
Case     & Test     & D    & $\dot{Q}$   &  $\dot{Q}''$   & $Q^*$   \\
         & Number   & (m)  & (kW)        &  (kW/m$^2$)    &         \\ \hline \hline
A        & 1        & 0.10 & 0.42        &  53.8          & 0.12    \\ \hline
B        & 5        & 0.10 & 1.88        &  240           & 0.53    \\ \hline
C        & 23       & 0.38 & 33.5        &  295           & 0.34    \\ \hline
D        & 21       & 0.38 & 175         &  1550          & 1.8     \\ \hline
E        & 7        & 1.0  & 49.0        &  62.4          & 0.044   \\ \hline
F        & 19       & 1.0  & 162         &  206           & 0.14    \\ \hline
\end{tabular}
\end{center}
\end{table}

\begin{figure}[p]
\begin{tabular*}{\textwidth}{l@{\extracolsep{\fill}}r}
\includegraphics[height=2.2in]{FIGURES/Hamins_CH4/Hamins_CH4_01_Radial_Heat_Flux} &
\includegraphics[height=2.2in]{FIGURES/Hamins_CH4/Hamins_CH4_05_Radial_Heat_Flux} \\
\includegraphics[height=2.2in]{FIGURES/Hamins_CH4/Hamins_CH4_23_Radial_Heat_Flux} &
\includegraphics[height=2.2in]{FIGURES/Hamins_CH4/Hamins_CH4_21_Radial_Heat_Flux} \\
\includegraphics[height=2.2in]{FIGURES/Hamins_CH4/Hamins_CH4_07_Radial_Heat_Flux} &
\includegraphics[height=2.2in]{FIGURES/Hamins_CH4/Hamins_CH4_19_Radial_Heat_Flux}
\end{tabular*}
\label{Hamins_CH4_Radial}
\caption{Comparison of predicted and measured heat fluxes to the ``floor'' as a function of radial distance from a methane burner.} 
\end{figure}

\begin{figure}[p]
\begin{tabular*}{\textwidth}{l@{\extracolsep{\fill}}r}
\includegraphics[height=2.2in]{FIGURES/Hamins_CH4/Hamins_CH4_01_Vertical_Heat_Flux} &
\includegraphics[height=2.2in]{FIGURES/Hamins_CH4/Hamins_CH4_05_Vertical_Heat_Flux} \\
\includegraphics[height=2.2in]{FIGURES/Hamins_CH4/Hamins_CH4_23_Vertical_Heat_Flux} &
\includegraphics[height=2.2in]{FIGURES/Hamins_CH4/Hamins_CH4_21_Vertical_Heat_Flux} \\
\includegraphics[height=2.2in]{FIGURES/Hamins_CH4/Hamins_CH4_07_Vertical_Heat_Flux} &
\includegraphics[height=2.2in]{FIGURES/Hamins_CH4/Hamins_CH4_19_Vertical_Heat_Flux}
\end{tabular*}
\label{Hamins_CH4_Vertical}
\caption{Comparison of predicted and measured heat fluxes from a methane burner to a ``wall'' as a function of the height from the burner surface.} 
\end{figure}





\clearpage
