\chapter{Gas Species and Smoke}

For most applications, FDS uses a single step chemical reaction whose products are tracked via
a two-parameter mixture fraction model.  The mixture fraction is a conserved
scalar quantity that represents the mass fraction of one or more components of the gas at
a given point in the flow field.  By default, two components of the mixture fraction are explicitly
computed. The first is the mass fraction of unburned fuel and
the second is the mass fraction of burned fuel (i.e., the mass of the combustion products
that originated as fuel). When the default model is used, O$_2$, CO$_2$ and smoke concentrations are obtained
from the explicitly computed mixture fraction variables. Their yields are specified by the user and do not
change.

FDS has an optional two-step chemical reaction with a three parameter
mixture fraction decomposition with the first step being oxidation of fuel
to carbon monoxide and the second step the oxidation of carbon monoxide to carbon dioxide.
The three mixture fraction components for the two step reaction
are unburned fuel, mass of fuel that has completed the first reaction step, and the mass
of fuel that has completed the second reaction step.  The mass fractions of all of the major
reactants and products can be derived from the mixture fraction parameters by means of
``state relations.'' Examples of this more detailed model can be found later in this chapter.




\section{WTC and NIST/NRC Test Series, Oxygen and CO$_2$}

The following pages present comparisons of oxygen and carbon dioxide concentration predictions and measurements for the
WTC and NIST/NRC series. In the WTC tests, there was only one measurement of each made near the ceiling of the compartment roughly 2~m from the
seat of the fire. In the NIST/NRC tests, there were two oxygen measurements, one in the upper layer, one in the lower.  There was only one carbon
dioxide measurement in the upper layer.

\newpage

\begin{figure}[p]
\begin{tabular*}{\textwidth}{l@{\extracolsep{\fill}}r}
\includegraphics[height=2.2in]{FIGURES/WTC/WTC_01_Oxygen} &
\includegraphics[height=2.2in]{FIGURES/WTC/WTC_02_Oxygen} \\
\includegraphics[height=2.2in]{FIGURES/WTC/WTC_03_Oxygen} &
\includegraphics[height=2.2in]{FIGURES/WTC/WTC_04_Oxygen} \\
\includegraphics[height=2.2in]{FIGURES/WTC/WTC_05_Oxygen} &
\includegraphics[height=2.2in]{FIGURES/WTC/WTC_06_Oxygen}
\end{tabular*}
\label{NIST_WTC_Oxygen}
\end{figure}


\begin{figure}[p]
\begin{tabular*}{\textwidth}{l@{\extracolsep{\fill}}r}
\includegraphics[height=2.2in]{FIGURES/NIST_NRC/NIST_NRC_01_Oxygen} &
\includegraphics[height=2.2in]{FIGURES/NIST_NRC/NIST_NRC_07_Oxygen} \\
\includegraphics[height=2.2in]{FIGURES/NIST_NRC/NIST_NRC_02_Oxygen} &
\includegraphics[height=2.2in]{FIGURES/NIST_NRC/NIST_NRC_08_Oxygen} \\
\includegraphics[height=2.2in]{FIGURES/NIST_NRC/NIST_NRC_04_Oxygen} &
\includegraphics[height=2.2in]{FIGURES/NIST_NRC/NIST_NRC_10_Oxygen} \\
\includegraphics[height=2.2in]{FIGURES/NIST_NRC/NIST_NRC_13_Oxygen} &
\includegraphics[height=2.2in]{FIGURES/NIST_NRC/NIST_NRC_16_Oxygen}
\end{tabular*}
\label{NIST_NRC_Gas_Closed}
\end{figure}

\begin{figure}[p]
\begin{tabular*}{\textwidth}{l@{\extracolsep{\fill}}r}
\includegraphics[height=2.2in]{FIGURES/NIST_NRC/NIST_NRC_17_Oxygen} &
 \\
\includegraphics[height=2.2in]{FIGURES/NIST_NRC/NIST_NRC_03_Oxygen} &
\includegraphics[height=2.2in]{FIGURES/NIST_NRC/NIST_NRC_09_Oxygen} \\
\includegraphics[height=2.2in]{FIGURES/NIST_NRC/NIST_NRC_05_Oxygen} &
\includegraphics[height=2.2in]{FIGURES/NIST_NRC/NIST_NRC_14_Oxygen} \\
\includegraphics[height=2.2in]{FIGURES/NIST_NRC/NIST_NRC_15_Oxygen} &
\includegraphics[height=2.2in]{FIGURES/NIST_NRC/NIST_NRC_18_Oxygen}
\end{tabular*}
\label{NIST_NRC_Gas_Open}
\end{figure}


\begin{figure}[p]
\begin{tabular*}{\textwidth}{l@{\extracolsep{\fill}}r}
\includegraphics[height=2.2in]{FIGURES/WTC/WTC_01_CO2} &
\includegraphics[height=2.2in]{FIGURES/WTC/WTC_02_CO2} \\
\includegraphics[height=2.2in]{FIGURES/WTC/WTC_03_CO2} &
\includegraphics[height=2.2in]{FIGURES/WTC/WTC_04_CO2} \\
\includegraphics[height=2.2in]{FIGURES/WTC/WTC_05_CO2} &
\includegraphics[height=2.2in]{FIGURES/WTC/WTC_06_CO2}
\end{tabular*}
\label{NIST_WTC_CO2}
\end{figure}

\begin{figure}[p]
\begin{tabular*}{\textwidth}{l@{\extracolsep{\fill}}r}
\includegraphics[height=2.2in]{FIGURES/NIST_NRC/NIST_NRC_01_CO2} &
\includegraphics[height=2.2in]{FIGURES/NIST_NRC/NIST_NRC_07_CO2} \\
\includegraphics[height=2.2in]{FIGURES/NIST_NRC/NIST_NRC_02_CO2} &
\includegraphics[height=2.2in]{FIGURES/NIST_NRC/NIST_NRC_08_CO2} \\
\includegraphics[height=2.2in]{FIGURES/NIST_NRC/NIST_NRC_04_CO2} &
\includegraphics[height=2.2in]{FIGURES/NIST_NRC/NIST_NRC_10_CO2} \\
\includegraphics[height=2.2in]{FIGURES/NIST_NRC/NIST_NRC_13_CO2} &
\includegraphics[height=2.2in]{FIGURES/NIST_NRC/NIST_NRC_16_CO2}
\end{tabular*}
\end{figure}

\begin{figure}[p]
\begin{tabular*}{\textwidth}{l@{\extracolsep{\fill}}r}
\includegraphics[height=2.2in]{FIGURES/NIST_NRC/NIST_NRC_17_CO2} &
 \\
\includegraphics[height=2.2in]{FIGURES/NIST_NRC/NIST_NRC_03_CO2} &
\includegraphics[height=2.2in]{FIGURES/NIST_NRC/NIST_NRC_09_CO2} \\
\includegraphics[height=2.2in]{FIGURES/NIST_NRC/NIST_NRC_05_CO2} &
\includegraphics[height=2.2in]{FIGURES/NIST_NRC/NIST_NRC_14_CO2} \\
\includegraphics[height=2.2in]{FIGURES/NIST_NRC/NIST_NRC_15_CO2} &
\includegraphics[height=2.2in]{FIGURES/NIST_NRC/NIST_NRC_18_CO2}
\end{tabular*}
\end{figure}


\begin{figure}[p]
\begin{center}
\begin{tabular}{c}
\includegraphics[width=3.5in]{FIGURES/ScatterPlots/Carbon_Dioxide_Concentration} \\
\includegraphics[width=3.5in]{FIGURES/ScatterPlots/Oxygen_Concentration}\\
\end{tabular}
\end{center}
\caption[Summary of major gas species predictions]
{Summary of major gas species predictions.}
\end{figure}

\clearpage


\section{NIST/NRC Test Series, Smoke}

FDS treats smoke like all other combustion products, basically a tracer gas whose mass fraction is a function of the mixture fraction.
To model smoke movement, the user need only prescribe the smoke yield, that is, the fraction of the fuel mass that is
converted to smoke particulate.  For the simulations of the NIST/NRC tests, the smoke yield is specified as one of the test parameters.
The figures on the following pages contain comparisons of measured and predicted smoke concentration at one measuring station in the upper layer.
There are two obvious trends in the figures: first, the predicted concentrations are about 50~\% higher than the measured
in the open door tests.  Second,
the predicted concentrations are roughly three times the measured concentrations in the closed door tests.
As a contrast, Figure displays the time history of CO concentration for 6 of the NIST/NRC tests.
Like smoke, the CO is specified in FDS via a fixed yield, measured along with smoke and reported in the test document.
The large differences between model and measurement seen in the smoke data do not appear in the CO data.

\newpage

\begin{figure}[p]
\begin{tabular*}{\textwidth}{l@{\extracolsep{\fill}}r}
\includegraphics[height=2.2in]{FIGURES/NIST_NRC/NIST_NRC_01_Smoke} &
\includegraphics[height=2.2in]{FIGURES/NIST_NRC/NIST_NRC_07_Smoke} \\
\includegraphics[height=2.2in]{FIGURES/NIST_NRC/NIST_NRC_02_Smoke} &
\includegraphics[height=2.2in]{FIGURES/NIST_NRC/NIST_NRC_08_Smoke} \\
\includegraphics[height=2.2in]{FIGURES/NIST_NRC/NIST_NRC_04_Smoke} &
\includegraphics[height=2.2in]{FIGURES/NIST_NRC/NIST_NRC_10_Smoke} \\
\includegraphics[height=2.2in]{FIGURES/NIST_NRC/NIST_NRC_13_Smoke} &
\includegraphics[height=2.2in]{FIGURES/NIST_NRC/NIST_NRC_16_Smoke}
\end{tabular*}
\end{figure}

\begin{figure}[p]
\begin{tabular*}{\textwidth}{l@{\extracolsep{\fill}}r}
\includegraphics[height=2.2in]{FIGURES/NIST_NRC/NIST_NRC_17_Smoke} &
 \\
\includegraphics[height=2.2in]{FIGURES/NIST_NRC/NIST_NRC_03_Smoke} &
\includegraphics[height=2.2in]{FIGURES/NIST_NRC/NIST_NRC_09_Smoke} \\
\includegraphics[height=2.2in]{FIGURES/NIST_NRC/NIST_NRC_05_Smoke} &
\includegraphics[height=2.2in]{FIGURES/NIST_NRC/NIST_NRC_14_Smoke} \\
\includegraphics[height=2.2in]{FIGURES/NIST_NRC/NIST_NRC_15_Smoke} &
\includegraphics[height=2.2in]{FIGURES/NIST_NRC/NIST_NRC_18_Smoke}
\end{tabular*}
\end{figure}



\begin{figure}[p]
\begin{center}
\begin{tabular}{c}
\includegraphics[width=4.0in]{FIGURES/ScatterPlots/Smoke_Concentration} \\
\vspace{0.25in} \\
\end{tabular}
\end{center}
\caption[Summary of smoke concentration predictions]
{Summary of smoke concentration predictions.}
\end{figure}



\clearpage

\section{Smyth Slot Burner Experiment}

The two-step, CO production model in FDS was used to simulate a methane/air slot burner diffusion flame.
Figures~\ref{Smyth_Slot_Burner_7} through \ref{Smyth_Slot_Burner_11}
show predicted and measured temperatures at three elevations above the burner.  The model predicts a flame that is slightly narrower
and cooler than measured.  The model
also predicts higher centerline temperatures.  These results are not surprising.  The two-step combustion model considers
the first step, F~$\longrightarrow$~CO, to be infinitely fast, assuming that the local oxygen concentration satisfies a
flammability criterion.  This is true in the vicinity of the lip of the burner.  In reality, the cold fuel and air
streams do not react infinitely fast at this location and some oxygen penetrates the flame at the base, resulting in cooler
gases being entrained into the core of the flame with a resulting drop in the centerline temperature.

Figures~\ref{Smyth_Slot_Burner_7} through \ref{Smyth_Slot_Burner_11} also show predicted and
measured values of CH$_4$, O$_2$, CO,  and CO$_2$ at three elevations above the burner along.
Note that the test data shows a small quantity of oxygen along
the burner centerline which is not captured in the simulation.  Along the centerline, the model predicts higher values of fuel and higher values of
products than measured.  The species profiles are also slightly narrower than measured, consistent with the temperature prediction.
The reported uncertainty in the species concentration measurements ranges from 10~\% to 20~\%.

\begin{figure}[p]
\begin{tabular*}{\textwidth}{l@{\extracolsep{\fill}}r}
\includegraphics[height=2.2in]{FIGURES/Smyth_Slot_Burner/Smyth_Slot_Burner_7mm_Temperature} &
\includegraphics[height=2.2in]{FIGURES/Smyth_Slot_Burner/Smyth_Slot_Burner_7mm_Fuel} \\
\includegraphics[height=2.2in]{FIGURES/Smyth_Slot_Burner/Smyth_Slot_Burner_7mm_Carbon_Dioxide} &
\includegraphics[height=2.2in]{FIGURES/Smyth_Slot_Burner/Smyth_Slot_Burner_7mm_Oxygen} \\
\includegraphics[height=2.2in]{FIGURES/Smyth_Slot_Burner/Smyth_Slot_Burner_7mm_Carbon_Monoxide} &
\end{tabular*}
\caption[Temperature and gas species predictions 7~mm above burner, Smyth experiment]
{Predicted and measured temperature and gas species 7~mm above a methane-air slot burner.}
\label{Smyth_Slot_Burner_7}
\end{figure}

\begin{figure}[p]
\begin{tabular*}{\textwidth}{l@{\extracolsep{\fill}}r}
\includegraphics[height=2.2in]{FIGURES/Smyth_Slot_Burner/Smyth_Slot_Burner_9mm_Temperature} &
\includegraphics[height=2.2in]{FIGURES/Smyth_Slot_Burner/Smyth_Slot_Burner_9mm_Fuel} \\
\includegraphics[height=2.2in]{FIGURES/Smyth_Slot_Burner/Smyth_Slot_Burner_9mm_Carbon_Dioxide} &
\includegraphics[height=2.2in]{FIGURES/Smyth_Slot_Burner/Smyth_Slot_Burner_9mm_Oxygen} \\
\includegraphics[height=2.2in]{FIGURES/Smyth_Slot_Burner/Smyth_Slot_Burner_9mm_Carbon_Monoxide} &
\end{tabular*}
\caption[Temperature and gas species predictions 9~mm above burner, Smyth experiment]
{Predicted and measured temperature and gas species 9~mm above a methane-air slot burner.}
\label{Smyth_Slot_Burner_9}
\end{figure}

\begin{figure}[p]
\begin{tabular*}{\textwidth}{l@{\extracolsep{\fill}}r}
\includegraphics[height=2.2in]{FIGURES/Smyth_Slot_Burner/Smyth_Slot_Burner_11mm_Temperature} &
\includegraphics[height=2.2in]{FIGURES/Smyth_Slot_Burner/Smyth_Slot_Burner_11mm_Fuel} \\
\includegraphics[height=2.2in]{FIGURES/Smyth_Slot_Burner/Smyth_Slot_Burner_11mm_Carbon_Dioxide} &
\includegraphics[height=2.2in]{FIGURES/Smyth_Slot_Burner/Smyth_Slot_Burner_11mm_Oxygen} \\
\includegraphics[height=2.2in]{FIGURES/Smyth_Slot_Burner/Smyth_Slot_Burner_11mm_Carbon_Monoxide} &
\end{tabular*}
\caption[Temperature and gas species predictions 11~mm above burner, Smyth experiment]
{Predicted and measured temperature and gas species 11~mm above a methane-air slot burner.}
\label{Smyth_Slot_Burner_11}
\end{figure}



\clearpage

\section{Beyler Hood Experiments}

Fig.~\ref{Beyler_Species} shows species predictions made by the two-step model compared with measured data for a
range of fire sizes and burner positions.  The
model outputs are the time-averaged species concentration at the hood exhaust vent whereas the experiment is the time
averaged species concentration downstream in the exhaust duct.

\begin{figure}[p]
\begin{tabular*}{\textwidth}{l@{\extracolsep{\fill}}r}
\includegraphics[height=3.2in]{FIGURES/Beyler_Hood/Beyler_Hood_O2} &
\includegraphics[height=3.2in]{FIGURES/Beyler_Hood/Beyler_Hood_CO2} \\
\includegraphics[height=3.2in]{FIGURES/Beyler_Hood/Beyler_Hood_H2O} &
\includegraphics[height=3.2in]{FIGURES/Beyler_Hood/Beyler_Hood_CO} \\
\includegraphics[height=3.2in]{FIGURES/Beyler_Hood/Beyler_Hood_Soot} &
\includegraphics[height=3.2in]{FIGURES/Beyler_Hood/Beyler_Hood_UHC} \\
\end{tabular*}
\caption[Summary of gas species predictions, Beyler hood experiments]
{Comparison of measured and predicted species concentrations in the Beyler hood experiments}
\label{Beyler_Species}
\end{figure}

\clearpage

\section{NIST Reduced Scale Enclosure (RSE) Test Series, 1994}

The RSE natural gas experiments were selected to assess the CO production capability rather than soot production.
Nine fire sizes were simulated: 50~kW, 75~kW, 100~kW, 150~kW, 200~kW, 300~kW, 400~kW,
500~kW, and 600~kW.  The experiments were modeled using properties of the natural gas supplied to the test facility.
The model geometry included the compartment interior along with a 0.6~m deep region outside the door.
Figure~\ref{NIST_RSE_1994} shows the measured and predicted CO, CO$_2$ and O$_2$ concentrations.  The measured values are
from the test series performed by Bryner, Johnsson, and Pitts~\cite{Bryner:1}.

\newpage

\begin{figure}[p]
\begin{tabular*}{\textwidth}{l@{\extracolsep{\fill}}r}
\includegraphics[height=2.2in]{FIGURES/NIST_RSE_1994/NIST_RSE_1994_CO_Front} &
\includegraphics[height=2.2in]{FIGURES/NIST_RSE_1994/NIST_RSE_1994_CO_Rear} \\
\includegraphics[height=2.2in]{FIGURES/NIST_RSE_1994/NIST_RSE_1994_CO2_Front} &
\includegraphics[height=2.2in]{FIGURES/NIST_RSE_1994/NIST_RSE_1994_CO2_Rear} \\
\includegraphics[height=2.2in]{FIGURES/NIST_RSE_1994/NIST_RSE_1994_O2_Front} &
\includegraphics[height=2.2in]{FIGURES/NIST_RSE_1994/NIST_RSE_1994_O2_Rear}
\end{tabular*}
\caption[Summary of NIST Reduced Scale Enclosure experiments]{Summary of NIST Reduced Scale Enclosure experiments.}
\label{NIST_RSE_1994}
\end{figure}

\clearpage

\section{Sippola Aerosol Deposition Cases}

A total of 16 aerosol deposition tests were conducted in a straight duct for 5 different
particle sizes and 3 different air velocities. In these cases, the aerosol species is tracked
explicitly and the aerosol deposition routines are enabled (refer to the Aerosol Deposition section
in the FDS User Guide~\cite{FDS_Users_Guide} and FDS Technical Reference Guide~\cite{FDS_Math_Guide}
for more details). A summary of the 16 tests is shown in Table~\ref{Sippola_Aerosol_Deposition_Summary}.

\begin{table}[h!]
\caption{Summary of Sippola aerosol deposition experiments selected for model validation.}
\begin{center}
\begin{tabular}{|c|c|c|c|c|c|}
\hline
Test No.  &  Air Speed (m/s)  &  Particle Diameter ($\mu$m)  \\ \hline \hline
1         &  2.2              &  1.0                         \\ \hline
2         &  2.2              &  2.8                         \\ \hline
3         &  2.1              &  5.2                         \\ \hline
4         &  2.2              &  9.1                         \\ \hline
5         &  2.2              &  16                          \\ \hline
6         &  5.3              &  1.0                         \\ \hline
7         &  5.2              &  1.0                         \\ \hline
8         &  5.2              &  3.1                         \\ \hline
9         &  5.4              &  5.2                         \\ \hline
10        &  5.3              &  9.8                         \\ \hline
11        &  5.3              &  16                          \\ \hline
12        &  9.0              &  1.0                         \\ \hline
13        &  9.0              &  3.1                         \\ \hline
14        &  8.8              &  5.4                         \\ \hline
15        &  9.2              &  8.7                         \\ \hline
16        &  9.1              &  15                          \\ \hline
\end{tabular}
\end{center}
\label{Sippola_Aerosol_Deposition_Summary}
\end{table}

The particle deposition velocity, $V_d$, was determined by

\be V_d = \frac{J_1 + J_2 + J_3 + J_4}{4 C_{avg}} \ee

where $J_1$ through $J_4$ are the deposition fluxes for duct panels 1 through 4 (kg/m$^2$-s)
given by

\be J_i = \frac{m_{d}}{A_d t} \ee

where $m_d$ is the mass of particles on the duct panel (kg), $A_d$ is the area of the duct panel (m$^2$),
and $t$ is the duration over which soot deposits on the plate (s). $C_{avg}$ is the average aerosol
concentration in the duct test section (kg/m$^3$) and is given by

\be C_{avg} = \frac{C_{up} + C_{down}}{2} \ee

Figure~\ref{Sippola_Aerosol_Deposition_Velocity} compares the measured versus predicted deposition velocities.

\begin{figure}[p]
\begin{center}
\begin{tabular}{c}
\includegraphics[height=2.2in]{FIGURES/Sippola_Aerosol_Deposition/Sippola_Aerosol_Ceiling_Deposition} \\
\includegraphics[height=2.2in]{FIGURES/Sippola_Aerosol_Deposition/Sippola_Aerosol_Wall_Deposition} \\
\includegraphics[height=2.2in]{FIGURES/Sippola_Aerosol_Deposition/Sippola_Aerosol_Floor_Deposition}
\end{tabular}
\end{center}
\caption[Summary of Sippola aerosol deposition cases]{Summary of Sippola aerosol deposition cases.}
\label{Sippola_Aerosol_Deposition_Velocity}
\end{figure}

