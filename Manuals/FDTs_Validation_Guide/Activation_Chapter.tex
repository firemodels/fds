
\chapter{Smoke Detector and Sprinkler Activation Time}

This chapter looks at validation exercises where the aim is to predict the activation of a smoke detector or sprinkler.

\section{NIST Dunes 2000}
\label{NIST_Dunes_2000_Results}

Description of NIST Dunes 2000 results.


\clearpage


\section{UL/NFPRF Sprinkler, Vent, and Draft Curtain Experiments}
\label{UL_NFPRF:Results}

The ceiling jet is an important fire phenomenon because of the presence of automatic fire protection devices at the ceiling, like
sprinklers and smoke/heat vents. The results of the UL/NFPRF experiments provide useful data to assess the accuracy of FDS in predicting
the velocity and temperature near the ceiling, and consequently the resulting activation of sprinklers.
The UL/NFPRF test results (Series I) are summarized in Table~\ref{ULmatrix}.

\begin{table}[h!]
\begin{center}
\begin{tabular}{|c||c|c|c|c|c|c|}
\hline
\multicolumn{7}{|c|}{\bf Heptane Spray Burner Test Series I}  \\ \hline \hline
Test & Burner & Vent                    & First         & Total      & Draft    & Heat Release Rate \\
No.  & Pos.   & Operation               & Actuation (s) & Actuations & Curtains & MW @ s \\
\hline \hline
I-1   & B  & Closed                     & 65            & 11        & Yes  & 4.4 @ 50  \\ \hline
I-2   & B  & Manual (0:40)              & 66            & 12        & Yes  & 4.4 @ 50  \\ \hline
I-3   & B  & Manual (1:30)              & 64            & 12        & Yes  & 4.4 @ 50  \\ \hline
I-4   & C  & Closed                     & 60            & 10        & Yes  & 4.4 @ 50  \\ \hline
I-5   & C  & Manual (0:40)              & 72            & 9         & Yes  & 4.4 @ 50  \\ \hline
I-6   & C  & Manual (1:30)              & 62            & 8         & Yes  & 4.4 @ 50  \\ \hline
I-7   & C  & 74$^\circ$C link (DNO)     & 70            & 10        & Yes  & 4.4 @ 50  \\ \hline
I-8   & B  & 74$^\circ$C link (9:26)    & 60            & 11        & Yes  & 4.4 @ 50  \\ \hline
I-9   & D  & 74$^\circ$C link (DNO)     & 70            & 12        & Yes  & 4.4 @ 50  \\ \hline
I-10  & D  & Manual (0:40)              & 72            & 13        & Yes  & 4.4 @ 50  \\ \hline
I-11  & D  & 74$^\circ$C link (4:48)    & N/A           & N/A       & Yes  & 4.4 @ 50  \\ \hline
I-12  & A  & Closed                     & 68            & 14        & Yes  & 4.4 @ 50  \\ \hline
I-13  & A  & 74$^\circ$C link (1:04)    & 69            & 5         & Yes  & 6.0 @ 60  \\ \hline
I-14  & A  & Manual (0:40)              & 74            & 7         & Yes  & 5.8 @ 60  \\ \hline
I-15  & A  & Manual (1:30)              & 64            & 5         & Yes  & 5.8 @ 60  \\ \hline
I-16  & A  & 74$^\circ$C link (1:46)    & 106           & 4         & Yes  & 5.0 @ 110 \\ \hline \hline
I-17  & B  & 100$^\circ$C link (DNO)    & 58            & 4         & No   & 4.6 @ 50 \\ \hline
I-18  & C  & 100$^\circ$C link (DNO)    & 58            & 4         & No   & 3.7 @ 50 \\ \hline
I-19  & A  & 100$^\circ$C link (10:00)  & 56            & 10        & No   & 4.6 @ 50 \\ \hline
I-20  & A  & 74$^\circ$C link (1:20)    & 54            & 4         & No   & 4.2 @ 50 \\ \hline
I-21  & C  & 74$^\circ$C link (7:00)    & 58            & 10        & No   & 4.6 @ 50 \\ \hline
I-22  & D  & 100$^\circ$C link (DNO)    & 60            & 6         & No   & 4.6 @ 50 \\ \hline
\end{tabular}
\end{center}
\caption[Results of the UL/NFPRF Experiments, Series~I.]
{Results of the UL/NFPRF Series~I Experiments. Note that DNO means
``Did Not Open''. Also note, the fires grew at a rate proportional
to the square of the time until a certain flow rate of fuel was achieved
at which time the flow rate was held steady. Thus, the ``Heat Release Rate''
was the size of the fire at the time when the fuel supply was leveled off.}
\label{ULmatrix}
\end{table}

%The figures on the following pages display the number of sprinklers actuated as a function of time.
%The results are then summarized in Fig.~\ref{UL_NFPRF}. Note that there are no experimental uncertainty bounds on the plot because it is difficult to estimate the
%combined uncertainty related to the various parameters that are input into the model. In Fig.~\ref{UL_NFPRF_Repeatability}, the results of three replicate experiments
%demonstrate that the total number of actuated sprinklers in each experiment is repeatable, even though individual actuation times may vary. Based on these
%three replicates, there is very little, if any, uncertainty in the total number of actuated sprinklers for each test. However, the test report~\cite{Sheppard:1} does not
%include uncertainty estimates for the heat release rate, thermal properties of the ceiling, sprinkler RTI, conductivity factor, actuation temperature,
%median droplet diameter, and various other parameters that have been input into the model. Consequently, it is not possible to estimate the uncertainty in the
%total number of actuated sprinklers due to the uncertainty in the reported parameters. The only sensitivity analysis conducted for this set of experiments was
%to change the median volumetric droplet size from 1000~$\mu$m to 750~$\mu$m, which led to a reduction of approximately 50~\% in the number of predicted sprinkler actuations.

\clearpage

The UL/NFPRF test results (Series II) are summarized in Table~\ref{ULburnermatrixII}.

\begin{table}[ht!]
\begin{center}
\begin{tabular}{|c||c|c|c|c|c|c|c|}
\hline
\multicolumn{8}{|c|}{\bf Heptane Spray Burner Test Series II (10 MW Fires)}\\ \hline \hline
Test & Burner   & Vent      & Sprinklers & First      & Last      & \multicolumn{2}{|c|}{Avg.~Peak Temp.} \\ \cline{7-8}
No.  & Position & Operation & Opened     & Activation & Activation & $^\circ$C & $^\circ$F   \\
\hline \hline
II-1  & D  & 74$^\circ$C link (DNO)  & 27 & 1:15 & 6:13 & 129.4 &264.9 \\ \hline
II-2  & D  & All Open at Start       & 28 & 1:05 & 5:53 & 128.8 &263.8 \\ \hline
II-3  & A  & 74$^\circ$C link (1:15) & 12 & 1:08 & 4:00 & 101.8 &215.2 \\ \hline
II-4  & B  & 74$^\circ$C link (1:48) & 16 & 1:03 & 5:54 & 108.8 &227.8 \\ \hline
II-5  & D  & 74$^\circ$C link (DNO)  & 28 & 1:10 & 7:07 & 130.0 &266.0 \\ \hline
II-6  & D  & All Open at Start       & 27 & 1:10 & 5:21 & 127.5 &261.5 \\ \hline
II-7  & A  & Closed                  & 18 & 1:09 & 4:11 & 117.2 &243.0 \\ \hline
II-8  & B  & 74$^\circ$C link (1:12) & 13 & 1:10 & 3:34 & 107.7 &225.9 \\ \hline
II-9  & E  & 74$^\circ$C link (DNO)  & 23 & 1:07 & 3:28 & 115.8 &240.4 \\ \hline
II-10 & F  & 74$^\circ$C link (3:20) & 19+& 1:14 & 3:01 & 108.4 &227.1 \\ \hline
II-11 & C  & 74$^\circ$C link (DNO)  & 23 & 1:02 & 3:56 & 123.4 &254.1 \\ \hline
II-12 & C  & All Open at Start       & 23 & 0:58 & 4:55 & 119.0 &246.2 \\ \hline
\end{tabular}
\end{center}
\caption[Results of the UL/NFPRF Experiments, Series~II.]
{Results of the heptane spray burner Series II. Note that all fires
were ramped up to 10~MW in 75~s following a $t$-squared curve. Also, the
plus sign appended to a value in the ``Sprinklers Opened''
column indicates that the area of sprinkler activation spread to the
edge of the adjustable height ceiling, thus more activations might have
occurred had the ceiling extended further.}
\label{ULburnermatrixII}
\end{table}


\clearpage


\section{Vettori Flat Ceiling Experiments}
\label{Vettori_Flat_Results}

The behavior of ceiling jets is an important component in the design of automatic fire suppression systems, such as sprinklers. The Vettori cases examine the effects of fire location, fire growth rate, and ceiling obstructions on the temperature of ceiling jets and activation time of sprinklers. The results that follow provide useful data as to the predictive capacity of FDS with regards to ceiling jet behavior.

The following figures display the temperature of the ceiling jet at the location of the first two activating sprinklers for each of the 18 scenarios compared
with the appropriate experimental data. The
experiments consisted of either Smooth or Obstructed ceilings; Slow, Medium or Fast fires; and a burner in the Open, at the Wall, or in the Corner.
The experiments performed included three replicates of each of the smooth ceiling configurations and two replicates of each of the obstructed ceiling configurations.
Also included is a plot comparing predicted sprinkler activation times with the experimental results.

%\section{USCG/HAI Water Mist Suppression Tests}
%
%The following pages contain comparisons of the predicted heat release rates for fires that are suppressed with a water mist system. In all cases, the flow rate of liquid
%fuel is specified in the model, but the decrease in HRR due to the extinguishing system is predicted by the model. Table~\ref{USCG_HAI_Times} reports the observed extinguishment
%times. Figure~\ref{USCG_Scatter} compares the measured versus predicted extinguishment times. For the simulations, the extinguishment time is taken to be when the HRR drops to
%half of its specified value.
%
%\begin{table}[h!]
%\caption[USCG/HAI water mist suppression extinguishment times.]{Recorded extinguishment times for the USCG/HAI water mist suppression tests in a small shipboard machinery space. ``No''
%means that the fire was not extinguished within 600 s of nozzle activation.}
%\begin{center}
%\begin{tabular}{|l|c|c|c|c|c|c|}
%\hline
%\multicolumn{2}{|l|}{System}                            & Navy  & Grinnell  & Fogtec    & Chemetron & Fike   \\ \hline  \hline
%\multicolumn{2}{|l|}{Number of Nozzles}                 & 6     & 6         & 6         & 15        & 6      \\ \hline
%\multicolumn{2}{|l|}{Operating Pressure (bar)}          & 70    & 13        & 100       & 12        & 21     \\ \hline
%\multicolumn{2}{|l|}{Flow Rate (L/min)}                 & 68    & 75        & 22        & 70        & 48     \\ \hline
%\multicolumn{2}{|l|}{Assumed Median Drop Size ($\mu$m)} & 175   & 225       & 100       &           & 200    \\ \hline
%\multicolumn{2}{|l|}{Assumed Initial Velocity (m/s)}    & 75    & 32        & 90        &           & 41     \\ \hline
%\multicolumn{2}{|l|}{Assumed Spray Angle (deg.)}        & 120   & 90        & 120       &           & 90     \\ \hline \hline
%Fire Scenario       & Ventilation                       & \multicolumn{5}{c|}{Extinguishment Time (s)}      \\ \hline \hline
%1.0 MW Spray        & Closed                            & 15    & 26        & 21        & 27        & 21     \\ \hline
%1.0 MW Spray        & Natural                           & 15    & 40        & 32        & 43        & 35     \\ \hline
%1.0 MW Spray        & Forced                            & 17    & 55        & 76        & 357       & 133    \\ \hline
%0.5 MW Spray        & Closed                            & 34    & 70        & 39        & 53        & 56     \\ \hline
%0.5 MW Spray        & Natural                           & 41    & 117       & 67        & 158       & 140    \\ \hline
%0.5 MW Spray        & Forced                            & 124   & No        & No        & No        & No     \\ \hline
%0.25 MW Spray       & Closed                            & 157   & 360       & 169       & 314       & 277    \\ \hline
%0.25 MW Spray       & Natural                           & 206   & No        & 290       & 525       & 566    \\ \hline
%0.25 MW Spray       & Forced                            & No    & No        & No        & No        & No     \\ \hline
%\end{tabular}
%\end{center}
%\label{USCG_HAI_Times}
%\end{table}
%
%\begin{figure}[h!]
%\begin{center}
%\includegraphics[height=4in]{FIGURES/ScatterPlots/USCG_HAI_Extinction}
%\caption[Comparison of measured and predicted extinguishment times for the USCG/HAI water mist suppression tests.]{Comparison of measured and predicted extinguishment times for the USCG/HAI water mist suppression tests.}
%\label{USCG_Scatter}
%\end{center}
%\end{figure}
%
%
%
%
%\begin{figure}[p]
%\begin{tabular*}{\textwidth}{l@{\extracolsep{\fill}}r}
%\includegraphics[height=2.2in]{FIGURES/USCG_HAI/USCG_HAI_HRR_1000_Closed_Grinnell} &
%\includegraphics[height=2.2in]{FIGURES/USCG_HAI/USCG_HAI_HRR_1000_Closed_Navy} \\
%\includegraphics[height=2.2in]{FIGURES/USCG_HAI/USCG_HAI_HRR_1000_Closed_Fogtec} &
%\includegraphics[height=2.2in]{FIGURES/USCG_HAI/USCG_HAI_HRR_1000_Closed_Fike}
%\end{tabular*}
%\label{USCG_HAI_1}
%\end{figure}
%
%\begin{figure}[p]
%\begin{tabular*}{\textwidth}{l@{\extracolsep{\fill}}r}
%\includegraphics[height=2.2in]{FIGURES/USCG_HAI/USCG_HAI_HRR_1000_Natural_Grinnell} &
%\includegraphics[height=2.2in]{FIGURES/USCG_HAI/USCG_HAI_HRR_1000_Natural_Navy} \\
%\includegraphics[height=2.2in]{FIGURES/USCG_HAI/USCG_HAI_HRR_1000_Natural_Fogtec} &
%\includegraphics[height=2.2in]{FIGURES/USCG_HAI/USCG_HAI_HRR_1000_Natural_Fike}
%\end{tabular*}
%\label{USCG_HAI_2}
%\end{figure}
%
%\begin{figure}[p]
%\begin{tabular*}{\textwidth}{l@{\extracolsep{\fill}}r}
%\includegraphics[height=2.2in]{FIGURES/USCG_HAI/USCG_HAI_HRR_1000_Forced_Grinnell} &
%\includegraphics[height=2.2in]{FIGURES/USCG_HAI/USCG_HAI_HRR_1000_Forced_Navy} \\
%\includegraphics[height=2.2in]{FIGURES/USCG_HAI/USCG_HAI_HRR_1000_Forced_Fogtec} &
%\includegraphics[height=2.2in]{FIGURES/USCG_HAI/USCG_HAI_HRR_1000_Forced_Fike}
%\end{tabular*}
%\label{USCG_HAI_3}
%\end{figure}
%
%\begin{figure}[p]
%\begin{tabular*}{\textwidth}{l@{\extracolsep{\fill}}r}
%\includegraphics[height=2.2in]{FIGURES/USCG_HAI/USCG_HAI_HRR_500_Closed_Grinnell} &
%\includegraphics[height=2.2in]{FIGURES/USCG_HAI/USCG_HAI_HRR_500_Closed_Navy} \\
%\includegraphics[height=2.2in]{FIGURES/USCG_HAI/USCG_HAI_HRR_500_Closed_Fogtec} &
%\includegraphics[height=2.2in]{FIGURES/USCG_HAI/USCG_HAI_HRR_500_Closed_Fike}
%\end{tabular*}
%\label{USCG_HAI_4}
%\end{figure}
%
%\begin{figure}[p]
%\begin{tabular*}{\textwidth}{l@{\extracolsep{\fill}}r}
%\includegraphics[height=2.2in]{FIGURES/USCG_HAI/USCG_HAI_HRR_500_Natural_Grinnell} &
%\includegraphics[height=2.2in]{FIGURES/USCG_HAI/USCG_HAI_HRR_500_Natural_Navy} \\
%\includegraphics[height=2.2in]{FIGURES/USCG_HAI/USCG_HAI_HRR_500_Natural_Fogtec} &
%\includegraphics[height=2.2in]{FIGURES/USCG_HAI/USCG_HAI_HRR_500_Natural_Fike}
%\end{tabular*}
%\label{USCG_HAI_5}
%\end{figure}
%
%\begin{figure}[p]
%\begin{tabular*}{\textwidth}{l@{\extracolsep{\fill}}r}
%\includegraphics[height=2.2in]{FIGURES/USCG_HAI/USCG_HAI_HRR_500_Forced_Grinnell} &
%\includegraphics[height=2.2in]{FIGURES/USCG_HAI/USCG_HAI_HRR_500_Forced_Navy} \\
%\includegraphics[height=2.2in]{FIGURES/USCG_HAI/USCG_HAI_HRR_500_Forced_Fogtec} &
%\includegraphics[height=2.2in]{FIGURES/USCG_HAI/USCG_HAI_HRR_500_Forced_Fike}
%\end{tabular*}
%\label{USCG_HAI_6}
%\end{figure}
%
%\begin{figure}[p]
%\begin{tabular*}{\textwidth}{l@{\extracolsep{\fill}}r}
%\includegraphics[height=2.2in]{FIGURES/USCG_HAI/USCG_HAI_HRR_250_Closed_Grinnell} &
%\includegraphics[height=2.2in]{FIGURES/USCG_HAI/USCG_HAI_HRR_250_Closed_Navy} \\
%\includegraphics[height=2.2in]{FIGURES/USCG_HAI/USCG_HAI_HRR_250_Closed_Fogtec} &
%\includegraphics[height=2.2in]{FIGURES/USCG_HAI/USCG_HAI_HRR_250_Closed_Fike}
%\end{tabular*}
%\label{USCG_HAI_7}
%\end{figure}
%
%\begin{figure}[p]
%\begin{tabular*}{\textwidth}{l@{\extracolsep{\fill}}r}
%\includegraphics[height=2.2in]{FIGURES/USCG_HAI/USCG_HAI_HRR_250_Natural_Grinnell} &
%\includegraphics[height=2.2in]{FIGURES/USCG_HAI/USCG_HAI_HRR_250_Natural_Navy} \\
%\includegraphics[height=2.2in]{FIGURES/USCG_HAI/USCG_HAI_HRR_250_Natural_Fogtec} &
%\includegraphics[height=2.2in]{FIGURES/USCG_HAI/USCG_HAI_HRR_250_Natural_Fike}
%\end{tabular*}
%\label{USCG_HAI_8}
%\end{figure}
%
%
%\begin{figure}[p]
%\begin{tabular*}{\textwidth}{l@{\extracolsep{\fill}}r}
%\includegraphics[height=2.2in]{FIGURES/USCG_HAI/USCG_HAI_HRR_250_Forced_Grinnell} &
%\includegraphics[height=2.2in]{FIGURES/USCG_HAI/USCG_HAI_HRR_250_Forced_Navy} \\
%\includegraphics[height=2.2in]{FIGURES/USCG_HAI/USCG_HAI_HRR_250_Forced_Fogtec} &
%\includegraphics[height=2.2in]{FIGURES/USCG_HAI/USCG_HAI_HRR_250_Forced_Fike}
%\end{tabular*}
%\label{USCG_HAI_9}
%\end{figure}


\clearpage


\section{Summary of Smoke Detector Activation Time (Milke)}

A summary smoke detector activation time predictions is given in Fig.~\ref{smoke_detector_activation_milke_summary}. 

\begin{figure}[ht]
\begin{center}
\begin{tabular}{l}
\includegraphics[width=4.0in]{FIGURES/Scatterplots/Smoke_Detector_Activation_Time_Milke}
\end{tabular}
\end{center}
\caption[Summary of smoke detector activation time predictions (Milke).]
{Summary of smoke detector activation time predictions (Milke).}
\label{smoke_detector_activation_milke_summary}
\end{figure}


\clearpage


\section{Summary of Smoke Detector Activation Time (Mowrer)}

A summary smoke detector activation time predictions is given in Fig.~\ref{smoke_detector_activation_mowrer_summary}. 

\begin{figure}[ht]
\begin{center}
\begin{tabular}{l}
\includegraphics[width=4.0in]{FIGURES/Scatterplots/Smoke_Detector_Activation_Time_Mowrer}
\end{tabular}
\end{center}
\caption[Summary of smoke detector activation time predictions (Mowrer).]
{Summary of smoke detector activation time predictions (Mowrer).}
\label{smoke_detector_activation_mowrer_summary}
\end{figure}


\clearpage


\section{Summary of Sprinkler Activation Time}

A summary sprinkler activation time predictions is given in Fig.~\ref{sprinkler_activation_summary}. 

\begin{figure}[ht]
\begin{center}
\begin{tabular}{l}
\includegraphics[width=4.0in]{FIGURES/Scatterplots/Sprinkler_Activation_Time}
\end{tabular}
\end{center}
\caption[Summary of sprinkler activation time predictions.]
{Summary of sprinkler activation time predictions.}
\label{sprinkler_activation_summary}
\end{figure}


