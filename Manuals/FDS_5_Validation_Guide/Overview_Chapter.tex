
\chapter{Overview}

Because FDS is used both  for research and for practical applications,
there  are  some  routines  within   the  model  that  have  not  been
comprehensively validated. Even  those parts of the model  for which a
substantial  amount of  validation  work has  been  performed will  be
pushed beyond these limits by  the need to perform increasingly comlex
analyses.  Consequently,  it is  the  responsibility  of  the user  to
demonstrate the applicability of the model for scenarios that have not
yet been validated.

FDS  is  suited  for  a  wide range  of  thermally-driven  fluid  flow
scenarios,  including fire, both  in the  open ({\em  e.g.} unconfined
fire plumes) as  well as within the built  environment. To date, about
half of  the applications of  FDS have been  for design, and  half for
forensic reconstruction.

Design  applications  typically  involve  an existing  building  or  a
building  under  design. A  so-called  ``design  fire'' is  prescribed
either by  a regulatory authority  or by the engineers  performing the
analysis. Because the  fire's heat release rate is  known, the role of
the model is to predict  the transport of heat and combustion products
throughout  the room or  rooms of  interest. Ventilation  equipment is
often included  in the simulation, like fans,  blowers, exhaust hoods,
HVAC ducts,  smoke management systems,  {\em etc.} Sprinkler  and heat
and smoke detector activation are also of interest.  The effect of the
sprinkler spray on the fire is usually less of interest since the fire
is  prescribed rather  than  predicted. Detailed  descriptions of  the
contents of the building are usually not necessary because these items
are not assumed to be burning,  and even if they are, the burning rate
will be  fixed, not predicted.  Sometimes, it is necessary  to predict
the heat  flux from the fire  to a nearby ``target,''  and even though
the target  may heat up  to some prescribed ignition  temperature, the
subsequent spread  of the  fire usually goes  beyond the scope  of the
analysis because of the uncertainty  inherent in object to object fire
spread.

Forensic reconstructions require the  model to simulate an actual fire
based on  information that is collected  after the event,  such as eye
witness accounts, unburned materials,  burn signatures, {\em etc.} The
purpose  of  the simulation  is  to  connect  a sequence  of  discrete
observations  with  a continuous  description  of  the fire  dynamics.
Usually,  reconstructions  involve  more gas/solid  phase  interaction
because  virtually  all  objects  in  a  given  room  are  potentially
ignitable, especially when flashover  occurs. Thus, there is much more
emphasis on  such phenomena as  heat transfer to  surfaces, pyrolysis,
flame  spread, and suppression.  In general,  forensic reconstructions
are more challenging simulations  to perform because they require more
detailed  information  about the  room  contents,  and  there is  much
greater uncertainty in the total heat release rate as the fire spreads
from object to object.

Validation  studies  of FDS  to  date  have  focussed more  on  design
applications   than  reconstructions.  The   reason  is   that  design
applications usually involve prescribed  fires and demand a minimum of
thermophysical properties  of real materials.  Transport  of smoke and
heat  is  the  primary  focus,  and measurements  can  be  limited  to
well-placed thermocouples, a few  heat flux gauges, gas samplers, {\em
etc.} Phenomena of importance in forensic reconstructions, like second
item  ignition, flame  spread, vitiation  effects and  extinction, are
more   difficult  to   model  and   more  difficult   to   study  with
well-controlled experiments. Uncertainties  in material properties and
measurements, as  well as simplifying assumptions in  the model, often
force the  comparison between model and measurement  to be qualitative
at best.  Nevertheless, current validation  efforts are moving  in the
direction of these more difficult issues.




\subsection{Model Accuracy}

The degree of  accuracy for each output variable  required by the user
is  highly  dependent on  the  technical  issues  associated with  the
analysis.  The user  must ask: How accurate does  the analysis have to
be  to  answer  the  technical  question posed?  Thus,  a  generalized
definition of the  accuracy required for each quantity  with no regard
as  to the specifics  of a  particular analysis  is not  practical and
would be limited in its usefulness.

Returning   to    the   earlier   definitions    of   ``design''   and
``reconstruction,''  design applications  typically are  more accurate
because the heat release rate is prescribed rather than predicted, and
the    initial    and    boundary    conditions   are    far    better
characterized. Mathematically, a design calculation is an example of a
``well-posed''  problem  in  which   the  solution  of  the  governing
equations is  advanced in  time starting from  a known set  of initial
conditions and constrained by a known set of boundary conditions.  The
accuracy of the results is a function of the fidelity of the numerical
solution, which is  mainly dependent on the size  of the computational
grid. The FDS Validation Guide~\cite{FDS_Validation_Guide_5} describes
efforts to date involving well-characterized geometries and prescribed
fires. These studies show that  FDS predictions vary from being within
experimental   uncertainty  to  being   about  20~\%   different  than
measurements of temperature, heat flux, gas concentration, {\em etc}.

A reconstruction is an example of an ``ill-posed'' problem because the
outcome  is known  whereas  the initial  and  boundary conditions  are
not. There is  no single, unique solution to the  problem, that is, it
is possible to simulate numerous fires that produce the given outcome.
There is no right or wrong answer, but rather a small set of plausible
fire scenarios that are  consistent with the collected evidence. These
simulations are then used to demonstrate to fire service personnel why
the fire behaved as it did  based on the current understanding of fire
physics  incorporated in  the model.  Most  often, the  result of  the
analysis is only  qualitative. If there is any  quantification at all,
it could be in the time to reach critical events, like a roof collapse
or room flashover.
