\chapter{Thermal Radiation}

Energy transport consists of convection, conduction and
radiation. Convection of heat is accomplished via the solution of the
basic conservation equations. Gains and losses of heat via conduction
and radiation are represented by the divergence of the heat flux
vector in the energy equation, $\nabla\!\cdot \dbq''$. This section
describes the equations associated with the radiative part, $\dbq''_r$.

\section{Radiation Transport Equation}

The Radiative Transport Equation (RTE) for an absorbing, emitting, and scattering medium is
\begin{eqnarray}
\bs \cdot \nabla I_{\la}(\bx,\bs) &=&
\underbrace{ - \kappa(\bx,\la)   \; I_\la(\bx,\bs) }_{\textrm{Energy loss by absorption}} -
\underbrace{\sigma_s(\bx,\la) \; I_\la(\bx,\bs) }_{\textrm{Energy loss by diffusion}} +  \nonumber  \\ [0.25in]
& & \underbrace{   B(\bx,\la) }_{\textrm{Emission source term}} + \quad
\underbrace{   \frac{\sigma_s(\bx,\la)}{4\pi}
\int_{4\pi}\Phi(\bs',\bs) \; I_{\la}(\bx,\bs') \; d\bs'
 }_{\textrm{In-scattering term}}
\label{RTEbasic}
\end{eqnarray}
where $I_{\la}(\bx,\bs)$ is the radiation intensity at wavelength,
$\la$, $\bs$ is the direction vector of the intensity, and
$\kappa(\bx,\la)$ and $\sigma_s(\bx,\la)$ are the local absorption
and scattering coefficients,
respectively. $B(\bx,\la)$ is the emission source term.
The integral on the
right hand side describes the in-scattering from other directions. The in-scattering and scattering terms are detailed in section~\ref{droplet-radiation}.

In practical simulations, the spectral ($\la$) dependence of the RTE cannot be resolved
accurately. Instead, the radiation spectrum is divided into
a relatively small number of bands and a separate RTE is derived for
each band. For instance, the band specific RTE for a non-scattering gas is
\be   \bs \cdot \nabla I_n(\bx,\bs) = \kappa_n(\bx) \;
        \left[ I_{b,n}(\bx) - I_n(\bx,\bs) \right],\;\; n = 1...N
\label{bandRTE} \ee
where $I_n$ is the intensity integrated over the band $n$, and $\kappa_n$
is the appropriate mean absorption coefficient for the band.
The limits of the bands are selected to give an accurate
representation of the most important radiation bands of $\rm CO_2$ and
water vapor. If the absorption of the fuel is known to be important,
separate bands can be reserved for fuel, and the total number of
bands, $N$, is increased from six to nine.
For simplicity, the fuel is assumed to be $\rm CH_4$.
The limits of the bands are shown in Table~\ref{banditos}.

\begin{table}[ht]
\caption{Limits of the spectral bands.}
\vspace{0.1in}
\label{banditos}
\small
\begin{tabular}{|*{10}{c|}}
\hline
\hspace{0.5in} \underline{9 Band Model} \hspace{0.5in} & 1  & 2  & 3 & 4  & 5 & 6 & 7 & 8 & 9 \\ \cline{2-10}
                                     & Soot   & CO$_2$       & CH$_4$ & Soot & CO$_2$ & H$_2$O & H$_2$O       & Soot & Soot          \\
\raisebox{1.5ex}[0pt]{Major Species} &        & H$_2$O, Soot & Soot   &      & Soot   & Soot   & CH$_4$, Soot &      &      \\ \hline
\multicolumn{1}{c}{$\nu$ (1/cm)}
             & \multicolumn{1}{@{\hspace{-.2in}10000}c@{}}{ }
             & \multicolumn{1}{@{\hspace{-.2in} 3800}c@{}}{ }
             & \multicolumn{1}{@{\hspace{-.2in} 3400}c@{}}{ }
             & \multicolumn{1}{@{\hspace{-.2in} 2800}c@{}}{ }
             & \multicolumn{1}{@{\hspace{-.2in} 2400}c@{}}{ }
             & \multicolumn{1}{@{\hspace{-.2in} 2174}c@{}}{ }
             & \multicolumn{1}{@{\hspace{-.2in} 1429}c@{}}{ }
             & \multicolumn{1}{@{\hspace{-.2in} 1160}c@{}}{ }
             & \multicolumn{1}{@{\hspace{-.2in} 1000}c@{50}}{ } \\
\multicolumn{1}{c}{$\la$ ($\mu$m)}
             & \multicolumn{1}{@{\hspace{-.2in} 1.00}c@{}}{ }
             & \multicolumn{1}{@{\hspace{-.2in} 2.63}c@{}}{ }
             & \multicolumn{1}{@{\hspace{-.2in} 2.94}c@{}}{ }
             & \multicolumn{1}{@{\hspace{-.2in} 3.57}c@{}}{ }
             & \multicolumn{1}{@{\hspace{-.2in} 4.17}c@{}}{ }
             & \multicolumn{1}{@{\hspace{-.2in} 4.70}c@{}}{ }
             & \multicolumn{1}{@{\hspace{-.2in} 7.00}c@{}}{ }
             & \multicolumn{1}{@{\hspace{-.2in} 8.62}c@{}}{ }
             & \multicolumn{1}{@{\hspace{-.2in} 10.0}c@{200}}{ } \\ \hline
\underline{6 Band Model}  & 1  & 2  & \multicolumn{2}{|c|}{3} & 4  & \multicolumn{3}{|c|}{5} & 6  \\ \cline{2-10}
          & Soot   & CO$_2$       & \multicolumn{2}{|c|}{CH$_4$      } & CO$_2$ & \multicolumn{3}{|c|}{H$_2$O, CH$_4$, Soot} & Soot  \\
\raisebox{1.5ex}[0pt]{Major Species} &        & H$_2$O, Soot & \multicolumn{2}{|c|}{Soot} & Soot   & \multicolumn{3}{|c|}{  } &       \\
               \hline
\end{tabular}
\end{table}
\normalsize


\subsection{Emission Source Term}

For each spectral band, $n$, the emission term is a function of the refractive index, $n_n$, absorption coefficient, $\kappa_n$, and a source term, $I_{b,n}$:
\be
B(\bx,\la) =
n_{n} \kappa_n(\bx) I_{b,n}(\bx) =
\kappa_n(\bx) I_{b,n}(\bx)
\ee
The source term can be written as a fraction of the black body radiation
\be I_{b,n} = F_n(\la_{\rm min},\la_{\rm max}) \; \sigma \; T^4/\pi \ee
where $\sigma$ is the Stefan-Boltzmann constant.
The calculation of factors $F_n$ is explained in Ref.~\cite{Siegel:1}.
When the intensities corresponding to the bands are known, the total
intensity is calculated by summing over all the bands
\be
   I(\bx,\bs) = \sum_{n=1}^N I_n(\bx,\bs)
\ee
Even with a reasonably small number of bands, solving multiple
RTEs is very time consuming. Fortunately, in most large-scale fire
scenarios soot is the most important combustion product controlling the
thermal radiation from the fire and hot smoke. As the radiation spectrum of
soot is continuous, it is possible to assume that the gas behaves as a
gray medium.  The spectral dependence is then lumped into one
absorption coefficient ($N=1$) and the source term is given by the
black body radiation intensity
\be I_b(\bx) = \frac{\sigma \, T(\bx)^4}{\pi} \label{emission_source_term} \ee
This is the default mode of FDS and appropriate for most problems of
fire engineering. In optically thin flames, where the amount of soot
is small compared to the amount of $\rm CO_2$ and water vapor, the gray gas
assumption may produce significant over-predictions of the emitted
radiation. From a series of numerical experiments it has been found
that six bands ($N=6$) are usually enough to improve the accuracy in
these cases.


\subsection{Absorption Coefficient}

For the calculation of the gray or band-mean absorption coefficients,
$\kappa_n$, a narrow-band model, RadCal~\cite{RadCal}, has been
implemented in FDS. At the start of a simulation, the absorption
coefficient(s) are tabulated as a function of mixture fraction and
temperature. During the simulation the local absorption coefficient is
found by table-lookup.


\subsection{Radiation Contribution to Energy Equation}

The radiant heat flux vector $\dbq_r''$ is defined
\be \dbq_r''(\bx) = \int_{4\pi} \; \bs' \, I(\bx,\bs') \; d\bs'   \ee
The gas phase contribution to the radiative loss term in the energy equation is
\be -\nabla\!\cdot \dbq_r''(\bx)(\mbox{gas}) =
    \kappa(\bx) \, \left[ U(\bx) - 4 \pi \, I_b(\bx) \right]  \quad ; \quad
    U(\bx) = \int_{4\pi} \, I(\bx,\bs') \, d\bs'  \label{net_emission}
\ee
In words, the net radiant energy gained by a grid cell is the
difference between that which is absorbed and that which is emitted.

\subsection{Correction of the Emission Source Term}

In calculations of limited spatial resolution, the source term, $I_b$,
defined in Eq.~(\ref{emission_source_term}) requires special treatment in the flaming region. Typical FDS calculations
use grid cells that are tens of centimeters in size, and consequently the computed temperatures constitute a bulk average and are considerably lower than
one would expect in a diffusion flame.
Because of its fourth-power dependence on the temperature,
the source term must be modeled in those grid cells that are expected to contain flames. Elsewhere, there is greater confidence in the computed temperature,
and the source term can be computed directly. 
It is assumed that this ``flaming region'' is where the radiative fraction of the local heat release rate per unit volume, $\chi_r \dq'''$, exceeds
that which is calculated using the emission source term given in Eq.~(\ref{emission_source_term}), $4 \kappa \sigma T^4$. In this region, we
replace the emission source term with the following:   
\be I_{b,f}(\bx) = \frac{C \, \sigma \, T(\bx)^4}{\pi} + \frac{U(\bx)}{4\pi}  \quad ; \quad  
    C = \frac{\int \chi_r \, \dq''' \, dV}{\int 4 \kappa \sigma T^4 \, dV} \label{corrected_emission_source_term} 
\ee
When the source term defined in Eq.~(\ref{corrected_emission_source_term}) is
substituted into Eq.~(\ref{net_emission}), we assert that the net radiative emission from the flaming region is simply the
specified radiative fraction of the rate of heat released within this region.

The radiative
fraction, $\chi_r$, is a useful quantity in fire science. It is the nominal fraction of the heat release rate is emitted as thermal radiation. For most combustibles, $\chi_r$
is between 0.3 and 0.4. However, in Eq.~(\ref{corrected_emission_source_term}), $\chi_r$ is
interpreted as the fraction of energy radiated from the flaming
region.  For a small fire ($D<1$~m), the local $\chi_r$ is
approximately equal to its global counterpart. However, as the fire
increases in size, the global value will typically decrease due to a
net re-absorption of the thermal radiation by the increasing smoke
mantle.  

\newpage

\section{Numerical Method}
\label{radnumericalmethodsection}

This section describes how $\nabla\!\cdot \dbq_r''$ (the radiative loss
term) is computed at all gas-phase cells, and how the
the radiative heat flux $\dq_r''$ is computed at solid boundaries.

\subsection{Angular Discretization}
\label{radiation-discre}


To obtain the discretized form of the
RTE, the unit sphere is divided into a finite number of solid angles.
The coordinate system used to discretize the solid angle is
shown in Figure~\ref{Angular}.
\begin{figure}[ht]
\begin{center}
\includegraphics[height=2in]{FIGURES/RadCoord}
\caption{Coordinate system of the angular discretization.}
\label{Angular}
\end{center}
\end{figure}
The discretization of the solid angle is done by dividing first
the polar angle, $\theta$, into $N_{\theta}$ bands, where
$N_{\theta}$ is an even integer.
Each $\theta$-band is then divided into
$N_{\phi}(\theta)$ parts in the azimuthal ($\phi$) direction.
$N_{\phi}(\theta)$ must be divisible by 4.
The numbers $N_{\theta}$ and $N_{\phi}(\theta)$ are chosen
to give the total number of angles $N_{\Omega}$ as close to
the value defined by the user as possible.
$N_{\Omega}$ is calculated as
\be
 N_{\Omega} = \sum_{i=1}^{N_{\theta}} N_{\phi}(\theta_i)
\ee
The distribution of the angles is based on empirical rules that try
to produce equal solid angles $\delta \bO^l = 4\pi/N_{\Omega}$. The
number of $\theta$-bands is
\be
 N_{\theta} = 1.17 \; N_{\Omega}^{1/2.26}
\ee
rounded to the nearest even integer. The number of $\phi$-angles
on each band is
\be
 N_{\phi}(\theta) = \max\left\{4,
        0.5\,N_{\Omega}\,\left[\cos(\theta^-)-\cos(\theta^+)\right]\right\}
\ee
rounded to the nearest integer that is divisible by 4.
$\theta^-$ and $\theta^+$ are
the lower and upper bounds of the $\theta$-band, respectively.
The discretization is symmetric with respect to the planes $x=0$, $y=0$, and
$z=0$. This symmetry has three important benefits:
First, it avoids the problems caused by the fact that the first-order
upwind scheme, used to calculate intensities on the cell boundaries,
is more diffusive in non-axial directions than axial.
Second, the treatment of the mirror boundaries becomes very simple, as
will be shown later. Third,
it avoids so called
``overhang'' situations, where $\bs\cdot {\bf i}$, $\bs\cdot {\bf j}$
or $\bs\cdot {\bf k}$ changes sign inside
the control angle. These ``overhangs'' would make the resulting system of
linear equations more complicated.

In the axially symmetric case these ``overhangs'' cannot be avoided, and a
special treatment, developed by Murthy and Mathur~\cite{Murthy}, is
applied. In these cases $N_{\phi}(\theta_i)$ is kept constant, and
the total number of angles is $N_{\Omega} = N_{\theta} \times
N_{\phi}$. In addition, the angle of the vertical slice of the cylinder is
chosen to be same as $\delta\phi$.



\subsection{Spatial Discretization}

The grid used for the RTE solver is the same as for the fluid solver.
The radiative transport equation~(\ref{bandRTE}) is solved using
techniques similar to those for convective transport in finite volume
methods for fluid flow~\cite{Raithby}; thus, the name given to it is
the Finite Volume Method (FVM). More details of the model
implementation can be found from~\cite{Hostikka:2008}.
To obtain the discretized form of the
RTE, the computational domain, the unit sphere and the spectrum are divided as explained in section~\ref{radiation-discre}.
In each grid cell a discretized equation is derived by integrating
Eq.~(\ref{bandRTE}) over the volume of cell $ijk$ and the control
angle $\delta \Omega^l$, to obtain
\be
  \int_{\delta \Omega^l} \int_{V_{ijk}}
   \bs' \cdot \nabla I(\bx',\bs') d\bx' d\bs' =
   \int_{\delta \Omega^l} \int_{V_{ijk}} \kappa(\bx') \;
    \left[ I_{b}(\bx') - I(\bx',\bs') \right] d\bx' d\bs'
\ee
The volume integral on the left hand side is replaced by a surface
integral over the cell faces using the divergence theorem. Note that
the procedure outlined below is appropriate for each band of a wide
band model, thus the subscript $n$ has been removed for clarity.

Assuming that the radiation intensity $I(\bx,\bs)$ is constant on each
of the cell faces, the surface integral can be approximated by a sum
over the cell faces.  Assuming further that $I(\bx,\bs)$ is constant
within the volume $V_{ijk}$ and over the angle $\delta \bO^l$ we
obtain
\be  \sum_{m=1}^6 A_m \; I_m^l \;
      \int_{\Omega^l}(\bs' \cdot \bn_m) d\bs'
   = \kappa_{ijk} \,
     \left[ I_{b,ijk} - I_{ijk}^l \right] \; V_{ijk} \,
     \delta \Omega^l   \label{RTEdiscrete2}
\ee
where
\begin{tabbing}
$I_{ijk}^l$ \hspace{1in}  \=  radiant intensity in direction $l$ \\
$I_m^l$                   \>  radiant intensity at cell face $m$ \\
$I_{b,ijk}$               \>  radiant black body Intensity in cell \\
$\delta \Omega^l$         \>  solid angle corresponding to direction $l$ \\
$V_{ijk}$                 \>  volume of cell $ijk$ \\
$A_m$                     \>  area of cell face $m$ \\
$\bn_m$                   \>  unit normal vector of the cell face $m$
\end{tabbing}
Note that while the intensity is assumed constant within
the angle $\delta \bO^l$, its direction covers the angle $\delta \bO^l$
exactly.The local incident radiation intensity is
\be
 U_{ijk} = \sum_{l=1}^{N_{\Omega}} I_{ijk}^l \delta\Omega^l
\ee

In Cartesian coordinates\footnote{In the axisymmetric case
equation~(\ref{RTEdiscrete2}) becomes
a little bit more complicated, as the cell face normal vectors $\bn_m$
are not always constant. However, the computational efficiency can still be
retained.},
the normal vectors $\bn_m$ are the base
vectors of the coordinate system and the integrals over the solid
angle do not depend on the physical coordinate, but the direction
only. The intensities on the cell boundaries, $I_m^l$, are calculated
using a first-order upwind scheme.  If the physical space is swept in
the direction $\bs^l$, the intensity $I_{ijk}^l$ can be directly obtained
from an algebraic equation. This makes the numerical solution of the
FVM very fast.  Iterations are needed only to account for the
reflective boundaries. However, this is seldom necessary in
practice, because the time step set by the flow solver is small.


The cell face intensities, $I_m^l$ appearing on the left hand side of
(\ref{RTEdiscrete2}) are calculated using a first order
upwind scheme. Consider, for example, a control angle having a
direction vector $\bs$. If the radiation is traveling in the positive
$x$-direction, {\em i.e.} $\bs\cdot {\bf i} \geq 0$, the intensity on the
upwind side, $I_{xu}^l$ is assumed to be
the intensity in the neighboring cell, $I_{i-1\,jk}^l$,
and the intensity on the downwind
side is the intensity in the cell itself $I_{ijk}^l$.

On a rectilinear grid, the normal vectors $\bn_m$ are the base vectors
of the coordinate system and the integrals over the solid angle can be
calculated analytically. Equation (\ref{RTEdiscrete2}) can be simplified
\be
  a_{ijk}^l I_{ijk}^l =
  a_{x}^l   I_{xu}^l +
  a_{y}^l   I_{yu}^l +
  a_{z}^l   I_{zu}^l +   b_{ijk}^l \label{RTEdiscrete3}
\ee
where
\begin{eqnarray}
  a_{ijk}^l & = & A_x |D_x^l| + A_y |D_y^l| +A_z |D_z^l| +
        \kappa_{ijk} \; V_{ijk} \delta\Omega^l \\
  \nonumber \\
  a_{x}^l & = & A_x |D_x^l| \\
  a_{y}^l & = & A_y |D_y^l| \\
  a_{z}^l & = & A_z |D_z^l| \\
  \nonumber \\
  b_{ijk}^l & = &
  \kappa_{ijk} \; I_{b,ijk} \; V_{ijk} \; \delta\Omega^l \\
  \nonumber \\
  \delta \Omega^l & = & \int_{\Omega^l}d\Omega
          = \int_{\dph}\int_{\dth} \sin\theta \; d\theta \; d\phi \\
  \nonumber \\
   D_x^l & = & \int_{\Omega^l}(\bs^l\cdot {\bf i})d\Omega \\
         & = & \int_{\dph}\int_{\dth} (\bs^l\cdot{\bf i})
                        \sin\theta \; d\theta d\phi \nonumber \\
         & = & \int_{\dph}\int_{\dth} \cos\phi\sin\theta\sin\theta\;
                        d\theta d\phi \nonumber \\
         & = & \frac{1}{2}\left( \sin\phi^+ - \sin\phi^- \right)
               \left[\Delta\theta - \left(\cos\theta^+\sin\theta^+
                          - \cos\theta^-\sin\theta^-\right)\right] \nonumber \\
   D_y^l & = & \int_{\Omega^l}(\bs^l\cdot {\bf j})d\Omega \\
         & = & \int_{\dph}\int_{\dth} \sin\phi\sin\theta\sin\theta\;
                        d\theta d\phi \nonumber \\
         & = & \frac{1}{2}\left( \cos\phi^- - \cos\phi^+ \right)
               \left[\Delta\theta - \left(\cos\theta^+\sin\theta^+
                          - \cos\theta^-\sin\theta^-\right)\right]\nonumber \\
   D_z^l & = & \int_{\Omega^l}(\bs^l\cdot {\bf k})d\Omega \\
         & = & \int_{\dph}\int_{\dth} \cos\theta\sin\theta\;
                        d\theta d\phi \nonumber \\
         & = & \frac{1}{2}\Delta\phi
               \left[\left(\sin\theta^+\right)^2 -
                     \left(\sin\theta^-\right)^2\right] \nonumber
\end{eqnarray}
Here $\bf i$, $\bf j$ and $\bf k$ are the base vectors of the
Cartesian coordinate system. $\theta^+$, $\theta^-$, $\phi^+$ and
$\phi^-$ are the upper and lower boundaries of the control angle in
the polar and azimuthal directions, respectively, and $\Delta\theta =
\theta^+ - \theta^-$ and $\Delta\phi = \phi^+ - \phi^-$. In the cells
next to a wall, the areas $A_m$ of the faces, that are perpendicular to
the wall, are multiplied by 0.5.

The solution method of (\ref{RTEdiscrete3}) is based on an explicit
marching sequence~\cite{Kim}. The marching direction depends on the
propagation direction of the radiation intensity. As the marching is
done in the ``downwind'' direction, the ``upwind'' intensities in all
three spatial directions are known, and the intensity $I_{ijk}^l$ can
be solved directly. Iterations may be needed only with the reflective
walls and optically thick situations.  Currently, no iterations are
made.

\subsection{Boundary Conditions}

The boundary condition for the radiation intensity leaving
a gray diffuse wall is given as
\be I_w(\bs) = \frac{\epsilon\,\sigma\,T_w^4}{\pi} + \frac{1-\epsilon}{\pi}
 \int_{\bs'\cdot \bn_w < 0} I_w(\bs')\; |\bs'\cdot \bn_w | \; d\bs'
 \label{RTEbc} \ee
where $I_w(\bs)$ is the intensity at the wall, $\epsilon$ is the
emissivity, and $T_{w}$ is the wall surface temperature.

In discretized form, the boundary condition on a solid wall is given as
\be I_w^l = \epsilon \; \frac{\sigma T_w^4}{\pi} + \frac{1-\epsilon}{\pi} \sum_{D_w^{l'}<0} I_w^{l'}\; |D_w^{l'} |  \ee
where $D_w^{l'}= \int_{\Omega^{l'}}(\bs\cdot \bn_w)d\Omega$.
The constraint $D_w^{l'}<0$ means that only the ``incoming'' directions
are taken into account when calculating the reflection.
The {\em net} radiative heat flux on the wall is
\be \dq_r'' = \sum_{l=1}^{N_{\Omega}} I_w^l \int_{\delta \Omega^l} (\bs' \cdot \bn_w) \, d\bs'
     = \sum_{l=1}^{N_{\Omega}} I_w^l D_n^l \label{qrdef} \ee
where the coefficients $D_n^l$ are equal to $\pm D_x^l$, $\pm D_y^l$ or
$\pm D_z^l$, and can be calculated for each wall element at the start of the
calculation.

The open boundaries are treated as black walls, where the incoming intensity is
the black body intensity of the ambient temperature. On mirror
boundaries the intensities leaving the wall
are calculated from the incoming intensities using a
predefined connection matrix:
\be  I_{w,ijk}^l = I^{l'} \ee
Computationally intensive integration over all the incoming directions
is avoided by keeping the solid angle discretization symmetric on the $x$, $y$ and $z$ planes.
The connection matrix associates one incoming direction $l'$ to each mirrored direction on each wall cell.


\newpage

\section{Absorption and Scattering of Thermal Radiation by Droplets/Particles}
\label{droplet-radiation}

The attenuation of thermal radiation by liquid droplets and particles is an important consideration, especially for water mist
systems~\cite{Ravigururajan:1}.  Droplets and particles attenuate thermal radiation through a combination of scattering and
absorption~\cite{Tuntomo:1}.  The radiation-spray interaction must therefore be solved for both the accurate prediction of the radiation
field and for the particle energy balance.

If the gas phase absorption and emission in Eq.~(\ref{RTEbasic}) are temporarily neglected for simplicity, the radiative transport
equation becomes
\be \bs \cdot \nabla I_{\la}(\bx,\bs) = -\left[\kappa_d(\bx,\la) + \sigma_d(\bx,\la)\right]
I_{\la}(\bx,\bs) +\kappa_d(\bx,\la) \; I_{b,d}(\bx,\la) +
\frac{\sigma_d(\bx,\la)}{4\pi}
\int_{4\pi}\Phi(\bs',\bs) \; I_{\la}(\bx,\bs') \; d\bs'
\label{RTEspray} \ee
where $\kappa_d$ is the particle absorption coefficient, $\sigma_d$ is the
particle scattering coefficient and $I_{b,d}$ is the emission
term of the particles. $\Phi(\bs',\bs)$ is a scattering phase function
that gives the scattered intensity fraction from direction $\bs'$ to $\bs$.\\

\subsection{Absorption and scattering coefficients}

The local absorption and scattering coefficients within a spray/particle cloud can be calculated
from the local droplet number density and size distribution
\begin{align}
\kappa_d(\bx,\la) &= \int_0^\infty n(r(\bx)) \; Q_a(r,\la) \; \pi r^2 \; dr \\ %N(\bx)\int_0^\infty f(r,d_m(\bx)) \; C_a(r,\la) \; dr \\
\sigma_d(\bx,\la) &= \int_0^\infty n(r(\bx)) \; Q_s(r,\la) \; \pi r^2 \; dr%N(\bx)\int_0^\infty f(r,d_m(\bx)) \; C_s(r,\la) \; dr
\end{align}
where $r$ is the particle radius and $Q_a$ and $Q_s$ are absorption and
scattering efficiencies, respectively.  For the computation of the spray/particle cloud radiative properties,
spherical particles are assumed and the radiative properties of the individual particles
are computed using Mie theory.

Based on \cite{Collin} and \cite{Maruyama}, the real particle size distribution inside a grid cell is modeled as a mono-disperse suspension
whose diameter corresponds to the Sauter mean ($d_{32}$) diameter of the poly-disperse spray.
This assumption leads to a simplified expression of the radiative coefficients
\begin{align}
\kappa_d(\bx,\la) &= A_d(\bx) \; Q_a(r_{32},\la) \\
\sigma_d(\bx,\la) &= A_d(\bx) \; Q_s(r_{32},\la)
\end{align}
These expressions are functions of the total cross sectional area per unit volume of the droplets $A_d$. $A_d$ is computed simply by summing the cross sectional areas of all the droplets within a cell and divided by the cell volume. For practical reasons, a relaxation factor of 0.5 is
used to smooth slightly the temporal variation of $A_d$.

\subsection{In-scattering term}

An accurate computation of the in-scattering integral on the right
hand side of Eq.~(\ref{RTEspray}) would be extremely time
consuming. It is here approximated by dividing the total $4\pi$ solid
angle to a ``forward angle'' $\delta\Omega^l$ and ``ambient angle''
$\delta\Omega^*=4\pi - \delta\Omega^l$.  For compatibility with the
FVM solver, $\delta\Omega^l$ is set equal to the control angle given
by the angular discretization.  However, it is assumed to be symmetric
around the center of the control angle.  Within $\delta\Omega^l$ the
intensity is $I_{\la}(\bx,\bs)$ and elsewhere it is approximated as
\be
U^*(\bx,\la) = \frac{U(\bx,\la) - \delta\Omega^l \, I_{\la}(\bx,\bs)}{\delta\Omega^*}
\ee
where $U(\bx)$ is the total intensity integrated over the unit sphere. The in-scattering
integral can now be written as
\begin{align}
\frac{\sigma_d(\bx,\la)}{4\pi}\int_{4\pi}\Phi(\bs,\bs') \; I_{\la}(\bx,\bs')
  \; d\Omega'
  &=
\sigma_d(\bx,\la)~ \left( \chi_f I_{\la}(\bx,\bs) ~+~\frac{1}{\delta\Omega^{*}}~\left(1-\chi_f\right)~
\int\limits_{\delta\Omega^{*}} I_{\la}(\bx,\bs') \; \textrm{d} \Omega' \right)\\
  &=
\sigma_d(\bx,\la)~ \left( \chi_f \; I_{\la}(\bx,\bs) ~+~
(1-\chi_f)U^*(\bx,\la) \right)
\end{align}

where $\chi_f = \chi_f(r,\la)$ is a fraction of the total intensity
originally within the solid angle $\delta\Omega^l$ that is scattered
into the same angle $\delta\Omega^l$.

\subsection{Solution Procedure}

Defining an effective scattering coefficient section
\be
\overline{\sigma_d}(\bx,\la) =
\frac{4\pi N(\bx)}{4\pi-\delta\Omega^l}
\int_0^\infty(1-\chi_f) \; C_s(r,\la) \; dr
\ee
the spray RTE becomes
\be
\bs \cdot \nabla I_{\la}(\bx,\bs) =
-\left[\kappa_d(\bx,\la) + \overline{\sigma_d}(\bx,\la)\right]
I_{\la}(\bx,\bs)
+\kappa_d(\bx,\la) \; I_{b,d}(\bx,\la)
+\frac{\overline{\sigma_d}(\bx,\la)}{4\pi}U(\bx,\la)
\ee
This equation can be integrated over the spectrum to get the band
specific RTE's. The procedure is exactly the same as that used for the
gas phase RTE. After the band integrations, the spray RTE for band $n$
becomes
\be
\bs \cdot \nabla I_{n}(\bx,\bs) =
-\left[\kappa_{d,n}(\bx) + \overline{\sigma_{d,n}}(\bx)\right] I_n(\bx,\bs)
+\kappa_{d,n}(\bx) \; I_{b,d,n}(\bx)
+\frac{\overline{\sigma_d}(\bx,\la)}{4\pi}U_n(\bx)
\ee
where the source function is based on the average particle temperature within a cell.

Before the actual simulation, both $\kappa_d$ and $\overline{\sigma_d}$ are averaged over the
infrared wavelengths (1-200 $\mu$m).  A constant ``radiation'' temperature, $T_{rad}$, is used
in the wavelength weighting.  $T_{rad}$ should be selected to
represent a typical radiating flame temperature. A value of 1173~K is
used by default.  The averaged quantities, now functions of the
droplet mean diameter only, are stored in one-dimensional arrays.
During the simulation, the local properties are found by table
look-up using the local Sauter mean droplet diameter.

The computation of $\chi_f$ for a similar but simpler situation has been derived in Ref.~\cite{Yang:3}. It can be shown that here
$\chi_f$ becomes
\be
\chi_f = \frac{1}{\delta\Omega^l}
\int_0^{\mu^l}\int_0^{\mu^l}\int_{\mu_{d,0}}^{\mu_{d,\pi}}
\frac{P_0(\theta_d)}
{\left[(1-\mu^2)(1-\mu'^2)-(\mu_d-\mu\mu')^2\right]^{1/2}}
\; d\mu_d \, d\mu \, d\mu'
\ee
where $\mu_d$ is a cosine of the scattering angle $\theta_d$ and
$P_0(\theta_d)$ is a single droplet scattering phase function
\be
P_0(\theta_d) =
\frac{\la^2\left(|S_1(\theta_d)|^2+|S_2(\theta_d)|^2\right)}{2 \, C_s(r,\la)}
\ee
$S_1(\theta_d)$ and $S_2(\theta_d)$ are the scattering amplitudes, given by
Mie-theory. The integration limit $\mu^l$ is a cosine of the polar angle
defining the boundary of the symmetric control angle $\delta\Omega^l$
\be
\mu^l = \cos(\theta^l) = 1 - \frac{2}{N_\Omega}
\ee
The limits of the innermost integral are
\be
\mu_{d,0}   = \mu\mu' + \sqrt{1-\mu^2}\sqrt{1-\mu'^2}  \quad ; \quad
\mu_{d,\pi} = \mu\mu' - \sqrt{1-\mu^2}\sqrt{1-\mu'^2}
\ee
When $\chi_f$ is integrated over the particle size distribution to get
an averaged value, it is multiplied by $C_s(r,\la)$. It is therefore
$|S_1|^2+|S_2|^2$, not $P_0(\theta_d)$, that is integrated. Physically,
this means that intensities are added, not
probabilities~\cite{Wiscombe}.\\


The absorption and scattering efficiencies and the scattering phase
function are calculated using the MieV code developed by Wiscombe~\cite{Wiscombe}.
The spectral properties of the particles can be specified in terms of wavelength dependent complex refractive index.
Pre-compiled data are included for water and a generic hydrocarbon fuel based on diesel and heptane properties.
In case of water, the values of the imaginary part of the complex refractive index (related to absorption coefficient) are taken from
Ref.~\cite{Hale:1}, and value 1.33 is used for the real part.
For fuel, the droplet spectral properties are taken from~\cite{Dombrovsky:1}, who measured the refractive index
of a diesel fuel and compared the values with those of heptane. The diesel properties are used for the real part of the refractive index. For the complex
part, the heptane properties are used to avoid the uncertainty associated with different types of diesel oils. Usually, the radiative properties
of the particle cloud are much more sensitive to the particle size and concentration than to values of the refractive index.




\subsection{Heat absorbed by droplets}

The droplet contribution to the radiative
loss term is
\be -\nabla\!\cdot \dbq_r''(\bx)(\mbox{droplets}) =
    \kappa_d(\bx) \, \left[ U(\bx) - 4 \pi \, I_{b,d}(\bx) \right]
\ee
For each individual droplet, the radiative heating/cooling power is
computed as
\be
\dq_r = \frac{m_d}{\rho_d(\bx)}
    \kappa_d(\bx) \, \left[ U(\bx) - 4\pi \, I_{b,d}(\bx) \right]
\ee
where $m_d$ is the mass of the droplet and $\rho_d(\bx)$ is the total
density of droplets in the cell.\\





