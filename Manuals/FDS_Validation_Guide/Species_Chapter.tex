% !TEX root = FDS_Validation_Guide.tex

\chapter{Gas Species and Smoke}

For most applications, FDS uses a single step, mixing-controlled combustion model. The products of combustion are ``lumped'' together and tracked as a single gas mixture. These products include CO$_2$, H$_2$O, CO, and soot. However, in some cases, the combustion is incomplete due to a lack of oxygen. In others, a multiple-step reaction scheme is used to predict the production of CO.

\section{Major Combustion Products, O$_2$ and CO$_2$}

For any hydrocarbon fuel, the major combustion products are oxygen and carbon dioxide. Accurate predictions of these gases requires knowledge of the chemical composition of the fuel and an accurate transport algorithm for the combustion products.

\clearpage

\subsection{FAA Cargo Compartments}

Carbon dioxide and carbon monoxide were measured near the ceiling in the forward, middle, and aft sections of the compartment. Note that all but the middle compartment concentrations were measured in Tests~2~and~3.


\begin{figure}[h]
\begin{tabular*}{\textwidth}{l@{\extracolsep{\fill}}r}
\includegraphics[height=2.15in]{SCRIPT_FIGURES/FAA_Cargo_Compartments/FAA_Cargo_Compartments_Test_1_CO2} &
\includegraphics[height=2.15in]{SCRIPT_FIGURES/FAA_Cargo_Compartments/FAA_Cargo_Compartments_Test_1_CO} \\
\includegraphics[height=2.15in]{SCRIPT_FIGURES/FAA_Cargo_Compartments/FAA_Cargo_Compartments_Test_2_CO2} &
\includegraphics[height=2.15in]{SCRIPT_FIGURES/FAA_Cargo_Compartments/FAA_Cargo_Compartments_Test_2_CO} \\
\includegraphics[height=2.15in]{SCRIPT_FIGURES/FAA_Cargo_Compartments/FAA_Cargo_Compartments_Test_3_CO2} &
\includegraphics[height=2.15in]{SCRIPT_FIGURES/FAA_Cargo_Compartments/FAA_Cargo_Compartments_Test_3_CO}
\end{tabular*}
\caption{FAA Cargo Compartment experiments, CO$_2$ and O$_2$ concentration.}
\label{FAA_Cargo_CO2_CO}
\end{figure}

\clearpage

\subsection{NIST/NRC Experiments}

The following pages present comparisons of oxygen and carbon dioxide concentration predictions and measurements for the
NIST/NRC series. There were two oxygen measurements, one in the upper layer, one in the lower.  There was only one carbon
dioxide measurement in the upper layer.

\begin{figure}[h]
\begin{tabular*}{\textwidth}{l@{\extracolsep{\fill}}r}
\includegraphics[height=2.15in]{SCRIPT_FIGURES/NIST_NRC/NIST_NRC_17_Oxygen} &
\includegraphics[height=2.15in]{SCRIPT_FIGURES/NIST_NRC/NIST_NRC_17_CO2} \\
\includegraphics[height=2.15in]{SCRIPT_FIGURES/NIST_NRC/NIST_NRC_03_Oxygen} &
\includegraphics[height=2.15in]{SCRIPT_FIGURES/NIST_NRC/NIST_NRC_03_CO2} \\
\includegraphics[height=2.15in]{SCRIPT_FIGURES/NIST_NRC/NIST_NRC_09_Oxygen} &
\includegraphics[height=2.15in]{SCRIPT_FIGURES/NIST_NRC/NIST_NRC_09_CO2}
\end{tabular*}
\caption{NIST/NRC experiments, CO$_2$ and O$_2$ concentration, Tests 3, 9, 17.}
\label{NIST_NRC_Gas_Open_1}
\end{figure}

\newpage

\begin{figure}[p]
\begin{tabular*}{\textwidth}{l@{\extracolsep{\fill}}r}
\includegraphics[height=2.15in]{SCRIPT_FIGURES/NIST_NRC/NIST_NRC_05_Oxygen} &
\includegraphics[height=2.15in]{SCRIPT_FIGURES/NIST_NRC/NIST_NRC_05_CO2} \\
\includegraphics[height=2.15in]{SCRIPT_FIGURES/NIST_NRC/NIST_NRC_14_Oxygen} &
\includegraphics[height=2.15in]{SCRIPT_FIGURES/NIST_NRC/NIST_NRC_14_CO2} \\
\includegraphics[height=2.15in]{SCRIPT_FIGURES/NIST_NRC/NIST_NRC_15_Oxygen} &
\includegraphics[height=2.15in]{SCRIPT_FIGURES/NIST_NRC/NIST_NRC_15_CO2} \\
\includegraphics[height=2.15in]{SCRIPT_FIGURES/NIST_NRC/NIST_NRC_18_Oxygen} &
\includegraphics[height=2.15in]{SCRIPT_FIGURES/NIST_NRC/NIST_NRC_18_CO2}
\end{tabular*}
\caption{NIST/NRC experiments, CO$_2$ and O$_2$ concentration, Tests 5, 14, 15, 18.}
\label{NIST_NRC_Gas_Open_2}
\end{figure}

\begin{figure}[p]
\begin{tabular*}{\textwidth}{l@{\extracolsep{\fill}}r}
\includegraphics[height=2.15in]{SCRIPT_FIGURES/NIST_NRC/NIST_NRC_01_Oxygen} &
\includegraphics[height=2.15in]{SCRIPT_FIGURES/NIST_NRC/NIST_NRC_01_CO2} \\
\includegraphics[height=2.15in]{SCRIPT_FIGURES/NIST_NRC/NIST_NRC_07_Oxygen} &
\includegraphics[height=2.15in]{SCRIPT_FIGURES/NIST_NRC/NIST_NRC_07_CO2} \\
\includegraphics[height=2.15in]{SCRIPT_FIGURES/NIST_NRC/NIST_NRC_02_Oxygen} &
\includegraphics[height=2.15in]{SCRIPT_FIGURES/NIST_NRC/NIST_NRC_02_CO2} \\
\includegraphics[height=2.15in]{SCRIPT_FIGURES/NIST_NRC/NIST_NRC_08_Oxygen} &
\includegraphics[height=2.15in]{SCRIPT_FIGURES/NIST_NRC/NIST_NRC_08_CO2}
\end{tabular*}
\caption{NIST/NRC experiments, CO$_2$ and O$_2$ concentration, Tests 1, 2, 7, 8.}
\label{NIST_NRC_Gas_Closed_1}
\end{figure}

\begin{figure}[p]
\begin{tabular*}{\textwidth}{l@{\extracolsep{\fill}}r}
\includegraphics[height=2.15in]{SCRIPT_FIGURES/NIST_NRC/NIST_NRC_04_Oxygen} &
\includegraphics[height=2.15in]{SCRIPT_FIGURES/NIST_NRC/NIST_NRC_04_CO2} \\
\includegraphics[height=2.15in]{SCRIPT_FIGURES/NIST_NRC/NIST_NRC_10_Oxygen} &
\includegraphics[height=2.15in]{SCRIPT_FIGURES/NIST_NRC/NIST_NRC_10_CO2} \\
\includegraphics[height=2.15in]{SCRIPT_FIGURES/NIST_NRC/NIST_NRC_13_Oxygen} &
\includegraphics[height=2.15in]{SCRIPT_FIGURES/NIST_NRC/NIST_NRC_13_CO2} \\
\includegraphics[height=2.15in]{SCRIPT_FIGURES/NIST_NRC/NIST_NRC_16_Oxygen} &
\includegraphics[height=2.15in]{SCRIPT_FIGURES/NIST_NRC/NIST_NRC_16_CO2}
\end{tabular*}
\caption{NIST/NRC experiments, CO$_2$ and O$_2$ concentration, Tests 4, 10, 13, 16.}
\label{NIST_NRC_Gas_Closed_2}
\end{figure}


\clearpage

\subsection{NRCC Smoke Tower}

In the NRCC Smoke Tower experiments, there were oxygen and carbon dioxide analysers in the stair shaft on the second floor just outside the door of the fire compartment.

\begin{figure}[!ht]
\begin{tabular*}{\textwidth}{l@{\extracolsep{\fill}}r}
\includegraphics[height=2.15in]{SCRIPT_FIGURES/NRCC_Smoke_Tower/BK-R_Fire_Lobby_O2} &
\includegraphics[height=2.15in]{SCRIPT_FIGURES/NRCC_Smoke_Tower/BK-R_Fire_Lobby_CO2} \\
\includegraphics[height=2.15in]{SCRIPT_FIGURES/NRCC_Smoke_Tower/CMP-R_Fire_Lobby_O2} &
\includegraphics[height=2.15in]{SCRIPT_FIGURES/NRCC_Smoke_Tower/CMP-R_Fire_Lobby_CO2}
\end{tabular*}
\caption{NRCC Smoke Tower, CO$_2$ and O$_2$ concentration, Tests BK-R and COMP-R.}
\label{NRCC_Smoke_Tower_O2_CO2_1}
\end{figure}

\begin{figure}[!ht]
\begin{tabular*}{\textwidth}{l@{\extracolsep{\fill}}r}
\includegraphics[height=2.15in]{SCRIPT_FIGURES/NRCC_Smoke_Tower/CLC-I-R_Fire_Lobby_O2} &
\includegraphics[height=2.15in]{SCRIPT_FIGURES/NRCC_Smoke_Tower/CLC-I-R_Fire_Lobby_CO2} \\
\includegraphics[height=2.15in]{SCRIPT_FIGURES/NRCC_Smoke_Tower/CLC-II-R_Fire_Lobby_O2} &
\includegraphics[height=2.15in]{SCRIPT_FIGURES/NRCC_Smoke_Tower/CLC-II-R_Fire_Lobby_CO2}
\end{tabular*}
\caption{NRCC Smoke Tower, CO$_2$ and O$_2$ concentration, Tests CLC-I-R and CLC-II-R.}
\label{NRCC_Smoke_Tower_O2_CO2_2}
\end{figure}


\clearpage

\subsection{PRISME DOOR Experiments}

Each compartment in the PRISME DOOR experiments contained an oxygen and carbon dioxide measurement in the upper (haut) and lower (bas) layers.

\begin{figure}[!ht]
\begin{tabular*}{\textwidth}{l@{\extracolsep{\fill}}r}
\includegraphics[height=2.15in]{SCRIPT_FIGURES/PRISME/PRS_D1_Room_1_O2} &
\includegraphics[height=2.15in]{SCRIPT_FIGURES/PRISME/PRS_D1_Room_1_CO2} \\
\includegraphics[height=2.15in]{SCRIPT_FIGURES/PRISME/PRS_D2_Room_1_O2} &
\includegraphics[height=2.15in]{SCRIPT_FIGURES/PRISME/PRS_D2_Room_1_CO2} \\
\includegraphics[height=2.15in]{SCRIPT_FIGURES/PRISME/PRS_D3_Room_1_O2} &
\includegraphics[height=2.15in]{SCRIPT_FIGURES/PRISME/PRS_D3_Room_1_CO2}
\end{tabular*}
\caption{PRISME DOOR experiments, CO$_2$ and O$_2$ concentration, Room 1, Tests 1-3.}
\label{PRISME_Gas_1}
\end{figure}

\begin{figure}[p]
\begin{tabular*}{\textwidth}{l@{\extracolsep{\fill}}r}
\includegraphics[height=2.15in]{SCRIPT_FIGURES/PRISME/PRS_D4_Room_1_O2} &
\includegraphics[height=2.15in]{SCRIPT_FIGURES/PRISME/PRS_D4_Room_1_CO2} \\
\includegraphics[height=2.15in]{SCRIPT_FIGURES/PRISME/PRS_D5_Room_1_O2} &
\includegraphics[height=2.15in]{SCRIPT_FIGURES/PRISME/PRS_D5_Room_1_CO2} \\
\includegraphics[height=2.15in]{SCRIPT_FIGURES/PRISME/PRS_D6_Room_1_O2} &
\includegraphics[height=2.15in]{SCRIPT_FIGURES/PRISME/PRS_D6_Room_1_CO2}
\end{tabular*}
\caption{PRISME DOOR experiments, CO$_2$ and O$_2$ concentration, Room 1, Tests 4-6.}
\label{PRISME_Gas_2}
\end{figure}

\begin{figure}[p]
\begin{tabular*}{\textwidth}{l@{\extracolsep{\fill}}r}
\includegraphics[height=2.15in]{SCRIPT_FIGURES/PRISME/PRS_D1_Room_2_O2} &
\includegraphics[height=2.15in]{SCRIPT_FIGURES/PRISME/PRS_D1_Room_2_CO2} \\
\includegraphics[height=2.15in]{SCRIPT_FIGURES/PRISME/PRS_D2_Room_2_O2} &
\includegraphics[height=2.15in]{SCRIPT_FIGURES/PRISME/PRS_D2_Room_2_CO2} \\
\includegraphics[height=2.15in]{SCRIPT_FIGURES/PRISME/PRS_D3_Room_2_O2} &
\includegraphics[height=2.15in]{SCRIPT_FIGURES/PRISME/PRS_D3_Room_2_CO2}
\end{tabular*}
\caption{PRISME DOOR experiments, CO$_2$ and O$_2$ concentration, Room 2, Tests 1-3.}
\label{PRISME_Gas_3}
\end{figure}

\begin{figure}[p]
\begin{tabular*}{\textwidth}{l@{\extracolsep{\fill}}r}
\includegraphics[height=2.15in]{SCRIPT_FIGURES/PRISME/PRS_D4_Room_2_O2} &
\includegraphics[height=2.15in]{SCRIPT_FIGURES/PRISME/PRS_D4_Room_2_CO2} \\
\includegraphics[height=2.15in]{SCRIPT_FIGURES/PRISME/PRS_D5_Room_2_O2} &
\includegraphics[height=2.15in]{SCRIPT_FIGURES/PRISME/PRS_D5_Room_2_CO2} \\
\includegraphics[height=2.15in]{SCRIPT_FIGURES/PRISME/PRS_D6_Room_2_O2} &
\includegraphics[height=2.15in]{SCRIPT_FIGURES/PRISME/PRS_D6_Room_2_CO2}
\end{tabular*}
\caption{PRISME DOOR experiments, CO$_2$ and O$_2$ concentration, Room 2, Tests 4-6.}
\label{PRISME_Gas_4}
\end{figure}

\clearpage

\subsection{PRISME SOURCE Experiments}

The compartment in the PRISME SOURCE experiments contained an oxygen and carbon dioxide measurement in the upper (haut) and lower (bas) layers.

\begin{figure}[!ht]
\begin{tabular*}{\textwidth}{l@{\extracolsep{\fill}}r}
\includegraphics[height=2.15in]{SCRIPT_FIGURES/PRISME/PRS_SI_D1_Room_2_O2} &
\includegraphics[height=2.15in]{SCRIPT_FIGURES/PRISME/PRS_SI_D1_Room_2_CO2} \\
\includegraphics[height=2.15in]{SCRIPT_FIGURES/PRISME/PRS_SI_D2_Room_2_O2} &
\includegraphics[height=2.15in]{SCRIPT_FIGURES/PRISME/PRS_SI_D2_Room_2_CO2} \\
\includegraphics[height=2.15in]{SCRIPT_FIGURES/PRISME/PRS_SI_D3_Room_2_O2} &
\includegraphics[height=2.15in]{SCRIPT_FIGURES/PRISME/PRS_SI_D3_Room_2_CO2} \\
\includegraphics[height=2.15in]{SCRIPT_FIGURES/PRISME/PRS_SI_D4_Room_2_O2} &
\includegraphics[height=2.15in]{SCRIPT_FIGURES/PRISME/PRS_SI_D4_Room_2_CO2}
\end{tabular*}
\caption{PRISME SOURCE experiments, CO$_2$ and O$_2$ concentration, Room 2, Tests 1-4.}
\label{PRISME_SOURCE_Gas_1}
\end{figure}

\begin{figure}[p]
\begin{tabular*}{\textwidth}{l@{\extracolsep{\fill}}r}
\includegraphics[height=2.15in]{SCRIPT_FIGURES/PRISME/PRS_SI_D5_Room_2_O2} &
\includegraphics[height=2.15in]{SCRIPT_FIGURES/PRISME/PRS_SI_D5_Room_2_CO2} \\
\includegraphics[height=2.15in]{SCRIPT_FIGURES/PRISME/PRS_SI_D5a_Room_2_O2} &
\includegraphics[height=2.15in]{SCRIPT_FIGURES/PRISME/PRS_SI_D5a_Room_2_CO2} \\
\includegraphics[height=2.15in]{SCRIPT_FIGURES/PRISME/PRS_SI_D6_Room_2_O2} &
\includegraphics[height=2.15in]{SCRIPT_FIGURES/PRISME/PRS_SI_D6_Room_2_CO2} \\
\includegraphics[height=2.15in]{SCRIPT_FIGURES/PRISME/PRS_SI_D6a_Room_2_O2} &
\includegraphics[height=2.15in]{SCRIPT_FIGURES/PRISME/PRS_SI_D6a_Room_2_CO2}
\end{tabular*}
\caption{PRISME SOURCE experiments, CO$_2$ and O$_2$ concentration, Room 2, Tests 5-6.}
\label{PRISME_SOURCE_Gas_2}
\end{figure}

\clearpage


\subsection{WTC Experiments}

The following pages present comparisons of oxygen and carbon dioxide concentration predictions and measurements for the
WTC experiments. There was only one measurement of each made near the ceiling of the compartment roughly 2~m from the fire.


\begin{figure}[h]
\begin{tabular*}{\textwidth}{l@{\extracolsep{\fill}}r}
\includegraphics[height=2.15in]{SCRIPT_FIGURES/WTC/WTC_01_Oxygen} &
\includegraphics[height=2.15in]{SCRIPT_FIGURES/WTC/WTC_01_CO2} \\
\includegraphics[height=2.15in]{SCRIPT_FIGURES/WTC/WTC_02_Oxygen} &
\includegraphics[height=2.15in]{SCRIPT_FIGURES/WTC/WTC_02_CO2} \\
\includegraphics[height=2.15in]{SCRIPT_FIGURES/WTC/WTC_03_Oxygen} &
\includegraphics[height=2.15in]{SCRIPT_FIGURES/WTC/WTC_03_CO2}
\end{tabular*}
\caption{WTC experiments, CO$_2$ and O$_2$ concentration, Tests 1-3.}
\label{NIST_WTC_Oxygen_CO2_1}
\end{figure}

\newpage

\begin{figure}[p]
\begin{tabular*}{\textwidth}{l@{\extracolsep{\fill}}r}
\includegraphics[height=2.15in]{SCRIPT_FIGURES/WTC/WTC_04_Oxygen} &
\includegraphics[height=2.15in]{SCRIPT_FIGURES/WTC/WTC_04_CO2} \\
\includegraphics[height=2.15in]{SCRIPT_FIGURES/WTC/WTC_05_Oxygen} &
\includegraphics[height=2.15in]{SCRIPT_FIGURES/WTC/WTC_05_CO2} \\
\includegraphics[height=2.15in]{SCRIPT_FIGURES/WTC/WTC_06_Oxygen} &
\includegraphics[height=2.15in]{SCRIPT_FIGURES/WTC/WTC_06_CO2}
\end{tabular*}
\caption{WTC experiments, CO$_2$ and O$_2$ concentration, Tests 4-6.}
\label{NIST_WTC_Oxygen_CO2_2}
\end{figure}

\clearpage


\subsection{UMD Line Burner}
\label{UMD_Line_Burner_species}

Oxygen concentration measurements were made across the coflow section of the burner.  Fig.~\ref{UMD_Line_Burner_methane_O2_p18_O2} shows mean volume fraction O$_2$ profiles for two heights, $z$, above the burner surface for the experiment with nitrogen dilution of the coflowing air to 18 vol.~\% O$_2$ with methane as fuel.  Notice that the O$_2$ level at the outer edge of the burner is the ambient value of 21 vol.~\%.  Further experimental details may be found in White et al.~\cite{White:2015}.  

FDS simulations are peformed at three grid resolutions corresponding to $W/\delta x = 4, 8, 16$, where $W = 5$ cm is the width of the line burner (see Fig.~\ref{fig:umd_line_burner_plan_view}).  Note that White et al.~\cite{White:2015} report measured global radiant fraction of $\chi_{\si{r}} = 0.18$ for 18 vol.~\% O$_2$ level with methane fuel.  We use this value in the simulations.  Also, because these simulations are well-resolved and we are interested in convergence of the numerical solution, we employ the CHARM flux limiter for scalar transport.  Second-order interpolated boundaries are used at mesh interfaces.  The domain decompositions, in order of increasing grid resolution, use 8, 16, and 128 meshes, respectively.

\begin{figure}[h]
\begin{tabular*}{\textwidth}{l@{\extracolsep{\fill}}r}
\includegraphics[height=2.15in]{SCRIPT_FIGURES/UMD_Line_Burner/methane_O2_p18_O2_z_p125} &
\includegraphics[height=2.15in]{SCRIPT_FIGURES/UMD_Line_Burner/methane_O2_p18_O2_z_p250}
\end{tabular*}
\caption[UMD\_Line\_Burner oxygen concentration profiles]
{Measured and computed mean oxygen concentration profiles at 18 vol \% O$_2$.}
\label{UMD_Line_Burner_methane_O2_p18_O2}
\end{figure}

\clearpage


\subsection{Summary of Major Combustion Products Predictions}
\label{Carbon Dioxide Concentration}
\label{Oxygen Concentration}


\begin{figure}[!h]
\centering
\begin{tabular}{c}
\includegraphics[width=3.5in]{SCRIPT_FIGURES/ScatterPlots/FDS_Carbon_Dioxide_Concentration} \\
\includegraphics[width=3.5in]{SCRIPT_FIGURES/ScatterPlots/FDS_Oxygen_Concentration}
\end{tabular}
\caption[Summary of major gas species predictions]
{Summary of major gas species predictions.}
\end{figure}

\clearpage


\section{Smoke and Aerosols}

\subsection{FM/FPRF Datacenter Experiments}
\label{FM Smoke Concentration}

Results of the low speed (78 ACH) and high speed (265 ACH) tests for propylene and cables is shown in the figure below. The error is a propagation of the non-linear error in the FM laser aspiration device combined with the fan flow error. Each test had three measurement locations (subfloor, ceiling, and ceiling plenum); however, not all locations for all tests had a measurement above background noise in the laser signal.

\begin{figure}[!h]
\begin{center}
\begin{tabular}{c}
\includegraphics[width=4.0in]{SCRIPT_FIGURES/FM_FPRF_Datacenter/FM_Datacenter_Soot.pdf}
\end{tabular}
\end{center}
\caption[Summary of smoke concentration predictions for the FM/FPRF Datacenter Tests]{Summary of smoke concentration predictions for the FM/FPRF Datacenter Tests.}
\end{figure}

\clearpage

\subsection{NIST/NRC Experiments}
\label{Smoke Concentration}

For the simulations of the NIST/NRC tests, the smoke yield is specified as one of the test parameters.
The figures on the following pages contain comparisons of measured and predicted smoke concentration at one measuring station in the upper layer.

\newpage

\begin{figure}[p]
\begin{tabular*}{\textwidth}{l@{\extracolsep{\fill}}r}
\includegraphics[height=2.15in]{SCRIPT_FIGURES/NIST_NRC/NIST_NRC_01_Smoke} &
\includegraphics[height=2.15in]{SCRIPT_FIGURES/NIST_NRC/NIST_NRC_07_Smoke} \\
\includegraphics[height=2.15in]{SCRIPT_FIGURES/NIST_NRC/NIST_NRC_02_Smoke} &
\includegraphics[height=2.15in]{SCRIPT_FIGURES/NIST_NRC/NIST_NRC_08_Smoke} \\
\includegraphics[height=2.15in]{SCRIPT_FIGURES/NIST_NRC/NIST_NRC_04_Smoke} &
\includegraphics[height=2.15in]{SCRIPT_FIGURES/NIST_NRC/NIST_NRC_10_Smoke} \\
\includegraphics[height=2.15in]{SCRIPT_FIGURES/NIST_NRC/NIST_NRC_13_Smoke} &
\includegraphics[height=2.15in]{SCRIPT_FIGURES/NIST_NRC/NIST_NRC_16_Smoke}
\end{tabular*}
\caption{NIST/NRC experiments, smoke concentration, Tests 1, 2, 4, 7, 8, 10, 13, 16.}
\end{figure}

\begin{figure}[p]
\begin{tabular*}{\textwidth}{l@{\extracolsep{\fill}}r}
\includegraphics[height=2.15in]{SCRIPT_FIGURES/NIST_NRC/NIST_NRC_17_Smoke} &
 \\
\includegraphics[height=2.15in]{SCRIPT_FIGURES/NIST_NRC/NIST_NRC_03_Smoke} &
\includegraphics[height=2.15in]{SCRIPT_FIGURES/NIST_NRC/NIST_NRC_09_Smoke} \\
\includegraphics[height=2.15in]{SCRIPT_FIGURES/NIST_NRC/NIST_NRC_05_Smoke} &
\includegraphics[height=2.15in]{SCRIPT_FIGURES/NIST_NRC/NIST_NRC_14_Smoke} \\
\includegraphics[height=2.15in]{SCRIPT_FIGURES/NIST_NRC/NIST_NRC_15_Smoke} &
\includegraphics[height=2.15in]{SCRIPT_FIGURES/NIST_NRC/NIST_NRC_18_Smoke}
\end{tabular*}
\caption{NIST/NRC experiments, smoke concentration, Tests 3, 5, 9, 14, 15, 17, 18.}
\end{figure}


\begin{figure}[p]
\begin{center}
\begin{tabular}{c}
\includegraphics[width=4.0in]{SCRIPT_FIGURES/ScatterPlots/FDS_Smoke_Concentration}
\end{tabular}
\end{center}
\caption[Summary of smoke concentration predictions]{Summary of smoke concentration predictions.}
\end{figure}

\clearpage

\subsection{FAA Cargo Compartments}
\label{Smoke Obscuration}

Beam obscuration measurements were made at different locations within the compartment (see Fig.~\ref{FAA_Cargo_probe_locations}). The data is presented below in terms of percent transmission per meter, $100(I/I_0)^{1/L}$, where $I$ is the light intensity and $L$ is the beam pathlength in units of meters.

\begin{figure}[h]
\begin{tabular*}{\textwidth}{l@{\extracolsep{\fill}}r}
\includegraphics[height=2.15in]{SCRIPT_FIGURES/FAA_Cargo_Compartments/FAA_Cargo_Compartments_Test_1_Ceiling_Transmission} &
\includegraphics[height=2.15in]{SCRIPT_FIGURES/FAA_Cargo_Compartments/FAA_Cargo_Compartments_Test_1_Cargo_Transmission} \\
\includegraphics[height=2.15in]{SCRIPT_FIGURES/FAA_Cargo_Compartments/FAA_Cargo_Compartments_Test_2_Ceiling_Transmission} &
\includegraphics[height=2.15in]{SCRIPT_FIGURES/FAA_Cargo_Compartments/FAA_Cargo_Compartments_Test_2_Cargo_Transmission} \\
\includegraphics[height=2.15in]{SCRIPT_FIGURES/FAA_Cargo_Compartments/FAA_Cargo_Compartments_Test_3_Ceiling_Transmission} &
\includegraphics[height=2.15in]{SCRIPT_FIGURES/FAA_Cargo_Compartments/FAA_Cargo_Compartments_Test_3_Cargo_Transmission}
\end{tabular*}
\caption{FAA Cargo Compartments experiments, smoke obscuration.}
\end{figure}

\newpage

\begin{figure}[p]
\begin{center}
\begin{tabular}{c}
\includegraphics[width=4.0in]{SCRIPT_FIGURES/ScatterPlots/FDS_Smoke_Obscuration}
\end{tabular}
\end{center}
\caption[Summary of smoke obscuration predictions]{Summary of smoke obscuration predictions.}
\end{figure}


\clearpage

\subsection{Sippola Aerosol Deposition Experiments}
\label{Aerosol Deposition Velocity}

FDS treats smoke particulate and aerosols in a similar way to other gaseous combustion products, basically a tracer gas whose production rate is a fixed fraction of the fuel consumption rate. However, there is an option in the model to allow smoke or aerosols to deposit on solid surfaces, thus reducing its concentration in the product stream. A total of 16 aerosol deposition experiments were conducted in a straight steel duct with smooth walls for 5 different particle diameters (1~$\mu$m, 3~$\mu$m, 5~$\mu$m, 9~$\mu$m, and 16~$\mu$m) and 3 different air velocities (2.2~m/s, 5.3~m/s, and 9.0~m/s). In the simulations, the aerosol is tracked explicitly, and the aerosol deposition routines are enabled (refer to the Aerosol Deposition section in the FDS User Guide~\cite{FDS_Users_Guide} and FDS Technical Reference Guide~\cite{FDS_Math_Guide} for more details). A summary of the 16~experiments is shown in Table~\ref{Sippola_Aerosol_Deposition_Summary}.

\begin{table}[h]
\caption{Summary of Sippola aerosol deposition experiments selected for model validation.}
\begin{center}
\begin{tabular}{|c|c|c|c|}
\hline
Test      &  Air Speed        &  Particle Diameter          &  Particle Density             \\
No.       &  (m/s)            &  ($\mu$m)                   &  (kg/m$^3$)                   \\ \hline \hline
1         &  2.2              &  1.0                        &  1350                         \\ \hline
2         &  2.2              &  2.8                        &  1170                         \\ \hline
3         &  2.1              &  5.2                        &  1210                         \\ \hline
4         &  2.2              &  9.1                        &  1030                         \\ \hline
5         &  2.2              &  16                         &  950                          \\ \hline
6         &  5.3              &  1.0                        &  1350                         \\ \hline
7         &  5.2              &  1.0                        &  1350                         \\ \hline
8         &  5.2              &  3.1                        &  1170                         \\ \hline
9         &  5.4              &  5.2                        &  1210                         \\ \hline
10        &  5.3              &  9.8                        &  1030                         \\ \hline
11        &  5.3              &  16                         &  950                          \\ \hline
12        &  9.0              &  1.0                        &  1350                         \\ \hline
13        &  9.0              &  3.1                        &  1170                         \\ \hline
14        &  8.8              &  5.4                        &  1210                         \\ \hline
15        &  9.2              &  8.7                        &  1030                         \\ \hline
16        &  9.1              &  15                         &  950                          \\ \hline
\end{tabular}
\end{center}
\label{Sippola_Aerosol_Deposition_Summary}
\end{table}

\noindent The particle deposition velocity, $u_{\rm dep}$, is calculated by
\be
   u_{\rm dep} = \frac{J_1 + J_2 + J_3 + J_4}{4 \; C_{\rm avg}}
\ee
where $J_1$ through $J_4$ are the deposition fluxes (\si{kg/(m^2.s)}) for duct panels 1 through 4 given by
\be
   J = \frac{m_{\rm d}}{A_{\rm d} \; \Delta t}
\ee
where $m_{\rm d}$ is the mass of particles on the duct panel (kg), $A_{\rm d}$ is the area of the duct panel (m$^2$),
and $\Delta t$ is the duration over which the aerosol deposits onto the panel (s). $C_{\rm avg}$ is the
average aerosol concentration in the duct test section (kg/m$^3$) and is given by
\be
   C_{\rm avg} = \frac{C_{\rm upstream} + C_{\rm downstream}}{2}
\ee
Figure~\ref{Sippola_Aerosol_Deposition_Velocity} compares the measured and predicted aerosol deposition velocities,
and Figure~\ref{Summary_Aerosol_Deposition_Velocity} shows a summary of the results.

\begin{figure}[!ht]
\begin{center}
\begin{tabular}{c}
\includegraphics[height=2.15in]{SCRIPT_FIGURES/Sippola_Aerosol_Deposition/Sippola_Aerosol_Ceiling_Deposition} \\
\includegraphics[height=2.15in]{SCRIPT_FIGURES/Sippola_Aerosol_Deposition/Sippola_Aerosol_Wall_Deposition} \\
\includegraphics[height=2.15in]{SCRIPT_FIGURES/Sippola_Aerosol_Deposition/Sippola_Aerosol_Floor_Deposition}
\end{tabular}
\end{center}
\caption[Predicted and measured aerosol deposition velocities, Sippola experiments]
{Predicted and measured aerosol deposition velocities, Sippola experiments.}
\label{Sippola_Aerosol_Deposition_Velocity}
\end{figure}

\begin{figure}[!ht]
\begin{center}
\begin{tabular}{c}
\includegraphics[width=4.0in]{SCRIPT_FIGURES/ScatterPlots/FDS_Aerosol_Deposition_Velocity} \\
\vspace{0.25in} \\
\end{tabular}
\end{center}
\caption[Summary of aerosol deposition velocity predictions]
{Summary of aerosol deposition velocity predictions.}
\label{Summary_Aerosol_Deposition_Velocity}
\end{figure}


\clearpage

\section{Products of Incomplete Combustion}

Predicting the concentration of products of incomplete combustion is challenging because it requires information about the chemical composition of the fuel and the multiple reactions that convert fuel to products. FDS contains a fairly general framework by which users can specify the reaction mechanism, and the examples in the following subsections highlight some of the more commonly used schemes.


\subsection{Smyth Slot Burner Experiment}

A two-step CO production model (a modified version of the mechanism by Andersen et al.~\cite{AndersenJ:1}) is used to simulate a methane/air slot burner diffusion flame.  A 2D DNS calculation is run at two different grid resolutions: $\delta x = 0.250 \;\si{mm}$ and $\delta x = 0.125 \;\si{mm}$.  In our modified mechanism the hydrocarbon/oxygen reaction to CO is assumed to be infinitely fast (mixed is burnt) to avoid complications of modeling ignition.  The reversible CO to CO$_2$ reaction is modeled with Arrhenius kinetics.  As discussed by Westbrook and Dryer~\cite{Westbrook:1}, the kinetics constants for the reduced CO mechanisms may be model dependent.  Here, the Arrhenius constant for the forward CO to CO$_2$ reaction is tuned to match the Smyth experimental data.  These same model parameters are then also used in the NIST Reduced Scale Enclosure LES without modification (see Section \ref{sec:NIST_RSE_1994}).  Figures~\ref{Smyth_Slot_Burner_fuel_ox} through \ref{Smyth_Slot_Burner_co_co2} show predicted and measured concentrations of CH$_4$, O$_2$, CO,  and CO$_2$ at three elevations above the burner. Figure~\ref{Smyth_Slot_Burner_temp} shows predicted and measured temperatures at these same elevations.  The reported uncertainty in the species concentration measurements ranges from 10~\% to 20~\%.

\begin{figure}[p]
\begin{tabular*}{\textwidth}{l@{\extracolsep{\fill}}r}
\includegraphics[height=2.15in]{SCRIPT_FIGURES/Smyth_Slot_Burner/Smyth_Slot_Burner_11mm_Fuel} &
\includegraphics[height=2.15in]{SCRIPT_FIGURES/Smyth_Slot_Burner/Smyth_Slot_Burner_11mm_Oxygen} \\
\includegraphics[height=2.15in]{SCRIPT_FIGURES/Smyth_Slot_Burner/Smyth_Slot_Burner_9mm_Fuel} &
\includegraphics[height=2.15in]{SCRIPT_FIGURES/Smyth_Slot_Burner/Smyth_Slot_Burner_9mm_Oxygen} \\
\includegraphics[height=2.15in]{SCRIPT_FIGURES/Smyth_Slot_Burner/Smyth_Slot_Burner_7mm_Fuel} &
\includegraphics[height=2.15in]{SCRIPT_FIGURES/Smyth_Slot_Burner/Smyth_Slot_Burner_7mm_Oxygen}
\end{tabular*}
\caption[Species predictions at 7~mm, 9~mm, and 11~mm above burner, Smyth experiment]
{Predicted and measured fuel and oxygen species at 7~mm, 9~mm, and 11~mm above a methane-air slot burner.}
\label{Smyth_Slot_Burner_fuel_ox}
\end{figure}

\begin{figure}[p]
\begin{tabular*}{\textwidth}{l@{\extracolsep{\fill}}r}
\includegraphics[height=2.15in]{SCRIPT_FIGURES/Smyth_Slot_Burner/Smyth_Slot_Burner_11mm_Carbon_Dioxide} &
\includegraphics[height=2.15in]{SCRIPT_FIGURES/Smyth_Slot_Burner/Smyth_Slot_Burner_11mm_Carbon_Monoxide} \\
\includegraphics[height=2.15in]{SCRIPT_FIGURES/Smyth_Slot_Burner/Smyth_Slot_Burner_9mm_Carbon_Dioxide} &
\includegraphics[height=2.15in]{SCRIPT_FIGURES/Smyth_Slot_Burner/Smyth_Slot_Burner_9mm_Carbon_Monoxide} \\
\includegraphics[height=2.15in]{SCRIPT_FIGURES/Smyth_Slot_Burner/Smyth_Slot_Burner_7mm_Carbon_Dioxide} &
\includegraphics[height=2.15in]{SCRIPT_FIGURES/Smyth_Slot_Burner/Smyth_Slot_Burner_7mm_Carbon_Monoxide}
\end{tabular*}
\caption[Species predictions at 7~mm, 9~mm, and 11~mm above burner, Smyth experiment]
{Predicted and measured carbon dioxide and carbon monoxide species at 7~mm, 9~mm, and 11~mm above a methane-air slot burner.}
\label{Smyth_Slot_Burner_co_co2}
\end{figure}

\begin{figure}[p]
\centering
\begin{tabular}{c}
\includegraphics[height=2.15in]{SCRIPT_FIGURES/Smyth_Slot_Burner/Smyth_Slot_Burner_11mm_Temperature} \\
\includegraphics[height=2.15in]{SCRIPT_FIGURES/Smyth_Slot_Burner/Smyth_Slot_Burner_9mm_Temperature} \\
\includegraphics[height=2.15in]{SCRIPT_FIGURES/Smyth_Slot_Burner/Smyth_Slot_Burner_7mm_Temperature}
\end{tabular}
\caption[Temperature predictions at 7~mm, 9~mm, and 11~mm above burner, Smyth experiment]
{Predicted and measured temperature at 7~mm, 9~mm, and 11~mm above a methane-air slot burner.}
\label{Smyth_Slot_Burner_temp}
\end{figure}

\clearpage

\subsection{Beyler Hood Experiments}

Fig.~\ref{Beyler_Species} shows species predictions made by the two-step model compared with measured data for a range of fire sizes and burner positions.  The model outputs are the time-averaged species concentration at the hood exhaust vent whereas the experiment is the time-averaged species concentration downstream in the exhaust duct.

\begin{figure}[h]
\begin{tabular*}{\textwidth}{c@{\extracolsep{\fill}}c}
\includegraphics[height=2.8in]{SCRIPT_FIGURES/Beyler_Hood/Beyler_Hood_O2} &
\includegraphics[height=2.8in]{SCRIPT_FIGURES/Beyler_Hood/Beyler_Hood_CO2} \\
\includegraphics[height=2.8in]{SCRIPT_FIGURES/Beyler_Hood/Beyler_Hood_H2O} &
\includegraphics[height=2.8in]{SCRIPT_FIGURES/Beyler_Hood/Beyler_Hood_CO} \\
\includegraphics[height=2.8in]{SCRIPT_FIGURES/Beyler_Hood/Beyler_Hood_Soot} &
\includegraphics[height=2.8in]{SCRIPT_FIGURES/Beyler_Hood/Beyler_Hood_UHC}
\end{tabular*}
\caption[Summary of gas species predictions, Beyler hood experiments]
{Comparison of measured and predicted species concentrations in the Beyler hood experiments}
\label{Beyler_Species}
\end{figure}

\clearpage

\subsection{NIST Reduced Scale Enclosure (RSE) Experiments, 1994}
\label{sec:NIST_RSE_1994}

The RSE natural gas experiments were selected to assess the CO production capability rather than soot production. Nine fire sizes were simulated: 50~kW, 75~kW, 100~kW, 150~kW, 200~kW, 300~kW, 400~kW, 500~kW, and 600~kW. The experiments were modeled using properties of the natural gas supplied to the test facility. The model geometry included the compartment interior along with a 0.6~m deep region outside the door. Figures~\ref{NIST_RSE_1994_spec1} and \ref{NIST_RSE_1994_spec1} show the measured and predicted CO, CO$_2$ and O$_2$, and H$_2$O concentrations. Figure~\ref{NIST_RSE_1994_temp} shows the measured and predicted thermocouple temperatures in the front and rear of the compartment. The measured values are from the test series performed by Bryner, Johnsson, and Pitts~\cite{Bryner:1}.

\begin{figure}[!h]
\begin{tabular*}{\textwidth}{l@{\extracolsep{\fill}}r}
\includegraphics[height=2.15in]{SCRIPT_FIGURES/NIST_RSE_1994/NIST_RSE_1994_CO_Front} &
\includegraphics[height=2.15in]{SCRIPT_FIGURES/NIST_RSE_1994/NIST_RSE_1994_CO_Rear} \\
\includegraphics[height=2.15in]{SCRIPT_FIGURES/NIST_RSE_1994/NIST_RSE_1994_CO2_Front} &
\includegraphics[height=2.15in]{SCRIPT_FIGURES/NIST_RSE_1994/NIST_RSE_1994_CO2_Rear}
\end{tabular*}
\caption[Summary of species concentrations in NIST RSE experiments]{Summary of species concentrations in NIST Reduced Scale Enclosure experiments.}
\label{NIST_RSE_1994_spec1}
\end{figure}

\newpage

\begin{figure}[!h]
\begin{tabular*}{\textwidth}{l@{\extracolsep{\fill}}r}
\includegraphics[height=2.15in]{SCRIPT_FIGURES/NIST_RSE_1994/NIST_RSE_1994_O2_Front} &
\includegraphics[height=2.15in]{SCRIPT_FIGURES/NIST_RSE_1994/NIST_RSE_1994_O2_Rear} \\
\includegraphics[height=2.15in]{SCRIPT_FIGURES/NIST_RSE_1994/NIST_RSE_1994_H2O_Front} &
\includegraphics[height=2.15in]{SCRIPT_FIGURES/NIST_RSE_1994/NIST_RSE_1994_H2O_Rear}
\end{tabular*}
\caption[Summary of species concentrations in NIST RSE experiments]{Summary of species concentrations in NIST Reduced Scale Enclosure experiments.}
\label{NIST_RSE_1994_spec2}
\end{figure}

\begin{figure}[!h]
\begin{tabular*}{\textwidth}{l@{\extracolsep{\fill}}r}
\includegraphics[height=2.15in]{SCRIPT_FIGURES/NIST_RSE_1994/NIST_RSE_1994_TC_Front} &
\includegraphics[height=2.15in]{SCRIPT_FIGURES/NIST_RSE_1994/NIST_RSE_1994_TC_Rear}
\end{tabular*}
\caption[Summary of thermocouple values in NIST RSE experiments]{Summary of thermocouple values in NIST Reduced Scale Enclosure experiments.}
\label{NIST_RSE_1994_temp}
\end{figure}

\clearpage

\subsection{NIST Full-Scale Enclosure (FSE) Experiments, 2008}

The FSE experiments~\cite{Lock:1} were selected to assess CO production capability in under-ventilated compartment fires over a range of fuels types and fire sizes in a full-scale ISO 9705 enclosure. The 30 experiments (of which 27 were selected) featured 7 different fuels: heptane, natural gas, nylon, propylene, isopropanol, styrene, and toluene. Peak heat release rates ranged from approximately 100~kW to 2.5~MW. Compartment ventilation was varied from a full 80~cm door opening to 40~cm, 20~cm, and 10~cm openings. The fuel sources in the experiments were either pool fires or spray burners, both of which were modeled. The fuel flow rates were specified in the experiments even though the fuel did not always burn.

In the simulations, there are fuels which are not pre-defined in FDS (e.g., styrene, nylon). In these cases, the fuel's enthalpy of formation is specified instead of the heat of combustion. Typically, the heat of combustion is reported for complete combustion. For the reaction mechanism used to model these experiments, however, the oxidation of fuel produces only carbon monoxide (i.e., incomplete combustion). Thus, typical literature values are not appropriate. An alternative to specifying the heat of combustion is to use the enthalpy of formation of the fuel. If that value cannot be found from literature or if a fuel mixture has been created, FDS can be used to determine an approximate value. Consider the following {\ct REAC} line:
\begin{lstlisting}
&REAC ID='NG', FUEL='NATURAL_GAS',FORMULA='C1.06084H4.076451N0.015529O0.014848' /
\end{lstlisting}
FDS will compute a balanced reaction assuming complete combustion and the heat of combustion will be determined based on oxygen consumption (see the FDS User's Guide discussion of the parameter {\ct EPUMO2}). A benefit of this process is that the enthalpy of formation of the fuel is automatically computed and is written to the {\ct CHID.out} file. Enthalpy of formation is a fundamental fuel property and not a reaction property like heat of combustion. As a result, this value can be used to define the fuel for the reaction mechanism with incomplete combustion.

\newpage

\begin{figure}[p]
\begin{tabular*}{\textwidth}{l@{\extracolsep{\fill}}r}
\includegraphics[height=2.15in]{SCRIPT_FIGURES/NIST_FSE_2008/ISONG3_Carbon_Dioxide} &
\includegraphics[height=2.15in]{SCRIPT_FIGURES/NIST_FSE_2008/ISONG3_Carbon_Monoxide} \\
\includegraphics[height=2.15in]{SCRIPT_FIGURES/NIST_FSE_2008/ISONG3_Oxygen} &
\includegraphics[height=2.15in]{SCRIPT_FIGURES/NIST_FSE_2008/ISONG3_Unburned_Hydrocarbons} \\
\includegraphics[height=2.15in]{SCRIPT_FIGURES/NIST_FSE_2008/ISONG3_Temperature} &
\includegraphics[height=2.15in]{SCRIPT_FIGURES/NIST_FSE_2008/ISONG3_HRR}
\end{tabular*}
\caption[Summary of ISONG3, NIST FSE 2008]{Summary of ISONG3, NIST FSE 2008.}
\label{NIST_FSE_1994_ISONG3}
\end{figure}

\begin{figure}[p]
\begin{tabular*}{\textwidth}{l@{\extracolsep{\fill}}r}
\includegraphics[height=2.15in]{SCRIPT_FIGURES/NIST_FSE_2008/ISOHept4_Carbon_Dioxide} &
\includegraphics[height=2.15in]{SCRIPT_FIGURES/NIST_FSE_2008/ISOHept4_Carbon_Monoxide} \\
\includegraphics[height=2.15in]{SCRIPT_FIGURES/NIST_FSE_2008/ISOHept4_Oxygen} &
\includegraphics[height=2.15in]{SCRIPT_FIGURES/NIST_FSE_2008/ISOHept4_Unburned_Hydrocarbons} \\
\includegraphics[height=2.15in]{SCRIPT_FIGURES/NIST_FSE_2008/ISOHept4_Temperature} &
\includegraphics[height=2.15in]{SCRIPT_FIGURES/NIST_FSE_2008/ISOHept4_HRR}
\end{tabular*}
\caption[Summary of ISOHept4, NIST FSE 2008]{Summary of ISOHept4, NIST FSE 2008.}
\label{NIST_FSE_1994_ISOHept4}
\end{figure}

\begin{figure}[p]
\begin{tabular*}{\textwidth}{l@{\extracolsep{\fill}}r}
\includegraphics[height=2.15in]{SCRIPT_FIGURES/NIST_FSE_2008/ISOHept5_Carbon_Dioxide} &
\includegraphics[height=2.15in]{SCRIPT_FIGURES/NIST_FSE_2008/ISOHept5_Carbon_Monoxide} \\
\includegraphics[height=2.15in]{SCRIPT_FIGURES/NIST_FSE_2008/ISOHept5_Oxygen} &
\includegraphics[height=2.15in]{SCRIPT_FIGURES/NIST_FSE_2008/ISOHept5_Unburned_Hydrocarbons} \\
\includegraphics[height=2.15in]{SCRIPT_FIGURES/NIST_FSE_2008/ISOHept5_Temperature} &
\includegraphics[height=2.15in]{SCRIPT_FIGURES/NIST_FSE_2008/ISOHept5_HRR}
\end{tabular*}
\caption[Summary of ISOHept5, NIST FSE 2008]{Summary of ISOHept5, NIST FSE 2008.}
\label{NIST_FSE_1994_ISOHept5}
\end{figure}

\begin{figure}[p]
\begin{tabular*}{\textwidth}{l@{\extracolsep{\fill}}r}
\includegraphics[height=2.15in]{SCRIPT_FIGURES/NIST_FSE_2008/ISOHept8_Carbon_Dioxide} &
\includegraphics[height=2.15in]{SCRIPT_FIGURES/NIST_FSE_2008/ISOHept8_Carbon_Monoxide} \\
\includegraphics[height=2.15in]{SCRIPT_FIGURES/NIST_FSE_2008/ISOHept8_Oxygen} &
\includegraphics[height=2.15in]{SCRIPT_FIGURES/NIST_FSE_2008/ISOHept8_Unburned_Hydrocarbons} \\
\includegraphics[height=2.15in]{SCRIPT_FIGURES/NIST_FSE_2008/ISOHept8_Temperature} &
\includegraphics[height=2.15in]{SCRIPT_FIGURES/NIST_FSE_2008/ISOHept8_HRR}
\end{tabular*}
\caption[Summary of ISOHept8, NIST FSE 2008]{Summary of ISOHept8, NIST FSE 2008.}
\label{NIST_FSE_1994_ISOHept8}
\end{figure}

\begin{figure}[p]
\begin{tabular*}{\textwidth}{l@{\extracolsep{\fill}}r}
\includegraphics[height=2.15in]{SCRIPT_FIGURES/NIST_FSE_2008/ISOHept9_Carbon_Dioxide} &
\includegraphics[height=2.15in]{SCRIPT_FIGURES/NIST_FSE_2008/ISOHept9_Carbon_Monoxide} \\
\includegraphics[height=2.15in]{SCRIPT_FIGURES/NIST_FSE_2008/ISOHept9_Oxygen} &
\includegraphics[height=2.15in]{SCRIPT_FIGURES/NIST_FSE_2008/ISOHept9_Unburned_Hydrocarbons} \\
\includegraphics[height=2.15in]{SCRIPT_FIGURES/NIST_FSE_2008/ISOHept9_Temperature} &
\includegraphics[height=2.15in]{SCRIPT_FIGURES/NIST_FSE_2008/ISOHept9_HRR}
\end{tabular*}
\caption[Summary of ISOHept9, NIST FSE 2008]{Summary of ISOHept9, NIST FSE 2008.}
\label{NIST_FSE_1994_ISOHept9}
\end{figure}

\begin{figure}[p]
\begin{tabular*}{\textwidth}{l@{\extracolsep{\fill}}r}
\includegraphics[height=2.15in]{SCRIPT_FIGURES/NIST_FSE_2008/ISONylon10_Carbon_Dioxide} &
\includegraphics[height=2.15in]{SCRIPT_FIGURES/NIST_FSE_2008/ISONylon10_Carbon_Monoxide} \\
\includegraphics[height=2.15in]{SCRIPT_FIGURES/NIST_FSE_2008/ISONylon10_Oxygen} &
\includegraphics[height=2.15in]{SCRIPT_FIGURES/NIST_FSE_2008/ISONylon10_Unburned_Hydrocarbons} \\
\includegraphics[height=2.15in]{SCRIPT_FIGURES/NIST_FSE_2008/ISONylon10_Temperature} &
\includegraphics[height=2.15in]{SCRIPT_FIGURES/NIST_FSE_2008/ISONylon10_HRR}
\end{tabular*}
\caption[Summary of ISONylon10, NIST FSE 2008]{Summary of ISONylon10, NIST FSE 2008.}
\label{NIST_FSE_1994_ISONylon10}
\end{figure}

\begin{figure}[p]
\begin{tabular*}{\textwidth}{l@{\extracolsep{\fill}}r}
\includegraphics[height=2.15in]{SCRIPT_FIGURES/NIST_FSE_2008/ISOPP11_Carbon_Dioxide} &
\includegraphics[height=2.15in]{SCRIPT_FIGURES/NIST_FSE_2008/ISOPP11_Carbon_Monoxide} \\
\includegraphics[height=2.15in]{SCRIPT_FIGURES/NIST_FSE_2008/ISOPP11_Oxygen} &
\includegraphics[height=2.15in]{SCRIPT_FIGURES/NIST_FSE_2008/ISOPP11_Unburned_Hydrocarbons} \\
\includegraphics[height=2.15in]{SCRIPT_FIGURES/NIST_FSE_2008/ISOPP11_Temperature} &
\includegraphics[height=2.15in]{SCRIPT_FIGURES/NIST_FSE_2008/ISOPP11_HRR}
\end{tabular*}
\caption[Summary of ISOPP11, NIST FSE 2008]{Summary of ISOPP11, NIST FSE 2008.}
\label{NIST_FSE_1994_ISOPropylene11}
\end{figure}

\begin{figure}[p]
\begin{tabular*}{\textwidth}{l@{\extracolsep{\fill}}r}
\includegraphics[height=2.15in]{SCRIPT_FIGURES/NIST_FSE_2008/ISOHeptD12_Carbon_Dioxide} &
\includegraphics[height=2.15in]{SCRIPT_FIGURES/NIST_FSE_2008/ISOHeptD12_Carbon_Monoxide} \\
\includegraphics[height=2.15in]{SCRIPT_FIGURES/NIST_FSE_2008/ISOHeptD12_Oxygen} &
\includegraphics[height=2.15in]{SCRIPT_FIGURES/NIST_FSE_2008/ISOHeptD12_Unburned_Hydrocarbons} \\
\includegraphics[height=2.15in]{SCRIPT_FIGURES/NIST_FSE_2008/ISOHeptD12_Temperature} &
\includegraphics[height=2.15in]{SCRIPT_FIGURES/NIST_FSE_2008/ISOHeptD12_HRR}
\end{tabular*}
\caption[Summary of ISOHeptD12, NIST FSE 2008]{Summary of ISOHeptD12, NIST FSE 2008.}
\label{NIST_FSE_1994_ISOHeptD12}
\end{figure}

\begin{figure}[p]
\begin{tabular*}{\textwidth}{l@{\extracolsep{\fill}}r}
\includegraphics[height=2.15in]{SCRIPT_FIGURES/NIST_FSE_2008/ISOHeptD13_Carbon_Dioxide} &
\includegraphics[height=2.15in]{SCRIPT_FIGURES/NIST_FSE_2008/ISOHeptD13_Carbon_Monoxide} \\
\includegraphics[height=2.15in]{SCRIPT_FIGURES/NIST_FSE_2008/ISOHeptD13_Oxygen} &
\includegraphics[height=2.15in]{SCRIPT_FIGURES/NIST_FSE_2008/ISOHeptD13_Unburned_Hydrocarbons} \\
\includegraphics[height=2.15in]{SCRIPT_FIGURES/NIST_FSE_2008/ISOHeptD13_Temperature} &
\includegraphics[height=2.15in]{SCRIPT_FIGURES/NIST_FSE_2008/ISOHeptD13_HRR}
\end{tabular*}
\caption[Summary of ISOHeptD13, NIST FSE 2008]{Summary of ISOHeptD13, NIST FSE 2008.}
\label{NIST_FSE_1994_ISOHeptD13}
\end{figure}

\begin{figure}[p]
\begin{tabular*}{\textwidth}{l@{\extracolsep{\fill}}r}
\includegraphics[height=2.15in]{SCRIPT_FIGURES/NIST_FSE_2008/ISOPropD14_Carbon_Dioxide} &
\includegraphics[height=2.15in]{SCRIPT_FIGURES/NIST_FSE_2008/ISOPropD14_Carbon_Monoxide} \\
\includegraphics[height=2.15in]{SCRIPT_FIGURES/NIST_FSE_2008/ISOPropD14_Oxygen} &
\includegraphics[height=2.15in]{SCRIPT_FIGURES/NIST_FSE_2008/ISOPropD14_Unburned_Hydrocarbons} \\
\includegraphics[height=2.15in]{SCRIPT_FIGURES/NIST_FSE_2008/ISOPropD14_Temperature} &
\includegraphics[height=2.15in]{SCRIPT_FIGURES/NIST_FSE_2008/ISOPropD14_HRR}
\end{tabular*}
\caption[Summary of ISOPropD14, NIST FSE 2008]{Summary of ISOPropD14, NIST FSE 2008.}
\label{NIST_FSE_1994_ISOPropD14}
\end{figure}

\begin{figure}[p]
\begin{tabular*}{\textwidth}{l@{\extracolsep{\fill}}r}
\includegraphics[height=2.15in]{SCRIPT_FIGURES/NIST_FSE_2008/ISOProp15_Carbon_Dioxide} &
\includegraphics[height=2.15in]{SCRIPT_FIGURES/NIST_FSE_2008/ISOProp15_Carbon_Monoxide} \\
\includegraphics[height=2.15in]{SCRIPT_FIGURES/NIST_FSE_2008/ISOProp15_Oxygen} &
\includegraphics[height=2.15in]{SCRIPT_FIGURES/NIST_FSE_2008/ISOProp15_Unburned_Hydrocarbons} \\
\includegraphics[height=2.15in]{SCRIPT_FIGURES/NIST_FSE_2008/ISOProp15_Temperature} &
\includegraphics[height=2.15in]{SCRIPT_FIGURES/NIST_FSE_2008/ISOProp15_HRR}
\end{tabular*}
\caption[Summary of ISOProp15, NIST FSE 2008]{Summary of ISOProp15, NIST FSE 2008.}
\label{NIST_FSE_1994_ISOProp15}
\end{figure}

\begin{figure}[p]
\begin{tabular*}{\textwidth}{l@{\extracolsep{\fill}}r}
\includegraphics[height=2.15in]{SCRIPT_FIGURES/NIST_FSE_2008/ISOStyrene16_Carbon_Dioxide} &
\includegraphics[height=2.15in]{SCRIPT_FIGURES/NIST_FSE_2008/ISOStyrene16_Carbon_Monoxide} \\
\includegraphics[height=2.15in]{SCRIPT_FIGURES/NIST_FSE_2008/ISOStyrene16_Oxygen} &
\includegraphics[height=2.15in]{SCRIPT_FIGURES/NIST_FSE_2008/ISOStyrene16_Unburned_Hydrocarbons} \\
\includegraphics[height=2.15in]{SCRIPT_FIGURES/NIST_FSE_2008/ISOStyrene16_Temperature} &
\includegraphics[height=2.15in]{SCRIPT_FIGURES/NIST_FSE_2008/ISOStyrene16_HRR}
\end{tabular*}
\caption[Summary of ISOStyrene16, NIST FSE 2008]{Summary of ISOStyrene16, NIST FSE 2008.}
\label{NIST_FSE_1994_ISOStyrene16}
\end{figure}

\begin{figure}[p]
\begin{tabular*}{\textwidth}{l@{\extracolsep{\fill}}r}
\includegraphics[height=2.15in]{SCRIPT_FIGURES/NIST_FSE_2008/ISOStyrene17_Carbon_Dioxide} &
\includegraphics[height=2.15in]{SCRIPT_FIGURES/NIST_FSE_2008/ISOStyrene17_Carbon_Monoxide} \\
\includegraphics[height=2.15in]{SCRIPT_FIGURES/NIST_FSE_2008/ISOStyrene17_Oxygen} &
\includegraphics[height=2.15in]{SCRIPT_FIGURES/NIST_FSE_2008/ISOStyrene17_Unburned_Hydrocarbons} \\
\includegraphics[height=2.15in]{SCRIPT_FIGURES/NIST_FSE_2008/ISOStyrene17_Temperature} &
\includegraphics[height=2.15in]{SCRIPT_FIGURES/NIST_FSE_2008/ISOStyrene17_HRR}
\end{tabular*}
\caption[Summary of ISOStyrene17, NIST FSE 2008]{Summary of ISOStyrene17, NIST FSE 2008.}
\label{NIST_FSE_1994_ISOStyrene17}
\end{figure}


\begin{figure}[p]
\begin{tabular*}{\textwidth}{l@{\extracolsep{\fill}}r}
\includegraphics[height=2.15in]{SCRIPT_FIGURES/NIST_FSE_2008/ISOPP18_Carbon_Dioxide} &
\includegraphics[height=2.15in]{SCRIPT_FIGURES/NIST_FSE_2008/ISOPP18_Carbon_Monoxide} \\
\includegraphics[height=2.15in]{SCRIPT_FIGURES/NIST_FSE_2008/ISOPP18_Oxygen} &
\includegraphics[height=2.15in]{SCRIPT_FIGURES/NIST_FSE_2008/ISOPP18_Unburned_Hydrocarbons} \\
\includegraphics[height=2.15in]{SCRIPT_FIGURES/NIST_FSE_2008/ISOPP18_Temperature} &
\includegraphics[height=2.15in]{SCRIPT_FIGURES/NIST_FSE_2008/ISOPP18_HRR}
\end{tabular*}
\caption[Summary of ISOPP18, NIST FSE 2008]{Summary of ISOPP18, NIST FSE 2008.}
\label{NIST_FSE_1994_ISOPP18}
\end{figure}

\begin{figure}[p]
\begin{tabular*}{\textwidth}{l@{\extracolsep{\fill}}r}
\includegraphics[height=2.15in]{SCRIPT_FIGURES/NIST_FSE_2008/ISOHept19_Carbon_Dioxide} &
\includegraphics[height=2.15in]{SCRIPT_FIGURES/NIST_FSE_2008/ISOHept19_Carbon_Monoxide} \\
\includegraphics[height=2.15in]{SCRIPT_FIGURES/NIST_FSE_2008/ISOHept19_Oxygen} &
\includegraphics[height=2.15in]{SCRIPT_FIGURES/NIST_FSE_2008/ISOHept19_Unburned_Hydrocarbons} \\
\includegraphics[height=2.15in]{SCRIPT_FIGURES/NIST_FSE_2008/ISOHept19_Temperature} &
\includegraphics[height=2.15in]{SCRIPT_FIGURES/NIST_FSE_2008/ISOHept19_HRR}
\end{tabular*}
\caption[Summary of ISOHept19, NIST FSE 2008]{Summary of ISOHept19, NIST FSE 2008.}
\label{NIST_FSE_1994_ISOHept19}
\end{figure}

\begin{figure}[p]
\begin{tabular*}{\textwidth}{l@{\extracolsep{\fill}}r}
\includegraphics[height=2.15in]{SCRIPT_FIGURES/NIST_FSE_2008/ISOToluene20_Carbon_Dioxide} &
\includegraphics[height=2.15in]{SCRIPT_FIGURES/NIST_FSE_2008/ISOToluene20_Carbon_Monoxide} \\
\includegraphics[height=2.15in]{SCRIPT_FIGURES/NIST_FSE_2008/ISOToluene20_Oxygen} &
\includegraphics[height=2.15in]{SCRIPT_FIGURES/NIST_FSE_2008/ISOToluene20_Unburned_Hydrocarbons} \\
\includegraphics[height=2.15in]{SCRIPT_FIGURES/NIST_FSE_2008/ISOToluene20_Temperature} &
\includegraphics[height=2.15in]{SCRIPT_FIGURES/NIST_FSE_2008/ISOToluene20_HRR}
\end{tabular*}
\caption[Summary of ISOToluene20, NIST FSE 2008]{Summary of ISOToluene20, NIST FSE 2008.}
\label{NIST_FSE_1994_ISOToluene20}
\end{figure}

\begin{figure}[p]
\begin{tabular*}{\textwidth}{l@{\extracolsep{\fill}}r}
\includegraphics[height=2.15in]{SCRIPT_FIGURES/NIST_FSE_2008/ISOStyrene21_Carbon_Dioxide} &
\includegraphics[height=2.15in]{SCRIPT_FIGURES/NIST_FSE_2008/ISOStyrene21_Carbon_Monoxide} \\
\includegraphics[height=2.15in]{SCRIPT_FIGURES/NIST_FSE_2008/ISOStyrene21_Oxygen} &
\includegraphics[height=2.15in]{SCRIPT_FIGURES/NIST_FSE_2008/ISOStyrene21_Unburned_Hydrocarbons} \\
\includegraphics[height=2.15in]{SCRIPT_FIGURES/NIST_FSE_2008/ISOStyrene21_Temperature} &
\includegraphics[height=2.15in]{SCRIPT_FIGURES/NIST_FSE_2008/ISOStyrene21_HRR}
\end{tabular*}
\caption[Summary of ISOStyrene21, NIST FSE 2008]{Summary of ISOStyrene21, NIST FSE 2008.}
\label{NIST_FSE_1994_ISOStyrene21}
\end{figure}

\begin{figure}[p]
\begin{tabular*}{\textwidth}{l@{\extracolsep{\fill}}r}
\includegraphics[height=2.15in]{SCRIPT_FIGURES/NIST_FSE_2008/ISOHept22_Carbon_Dioxide} &
\includegraphics[height=2.15in]{SCRIPT_FIGURES/NIST_FSE_2008/ISOHept22_Carbon_Monoxide} \\
\includegraphics[height=2.15in]{SCRIPT_FIGURES/NIST_FSE_2008/ISOHept22_Oxygen} &
\includegraphics[height=2.15in]{SCRIPT_FIGURES/NIST_FSE_2008/ISOHept22_Unburned_Hydrocarbons} \\
\includegraphics[height=2.15in]{SCRIPT_FIGURES/NIST_FSE_2008/ISOHept22_Temperature} &
\includegraphics[height=2.15in]{SCRIPT_FIGURES/NIST_FSE_2008/ISOHept22_HRR}
\end{tabular*}
\caption[Summary of ISOHept22, NIST FSE 2008]{Summary of ISOHept22, NIST FSE 2008.}
\label{NIST_FSE_1994_ISOHept22}
\end{figure}

\begin{figure}[p]
\begin{tabular*}{\textwidth}{l@{\extracolsep{\fill}}r}
\includegraphics[height=2.15in]{SCRIPT_FIGURES/NIST_FSE_2008/ISOHept23_Carbon_Dioxide} &
\includegraphics[height=2.15in]{SCRIPT_FIGURES/NIST_FSE_2008/ISOHept23_Carbon_Monoxide} \\
\includegraphics[height=2.15in]{SCRIPT_FIGURES/NIST_FSE_2008/ISOHept23_Oxygen} &
\includegraphics[height=2.15in]{SCRIPT_FIGURES/NIST_FSE_2008/ISOHept23_Unburned_Hydrocarbons} \\
\includegraphics[height=2.15in]{SCRIPT_FIGURES/NIST_FSE_2008/ISOHept23_Temperature} &
\includegraphics[height=2.15in]{SCRIPT_FIGURES/NIST_FSE_2008/ISOHept23_HRR}
\end{tabular*}
\caption[Summary of ISOHept23, NIST FSE 2008]{Summary of ISOHept23, NIST FSE 2008.}
\label{NIST_FSE_1994_ISOHept23}
\end{figure}

\begin{figure}[p]
\begin{tabular*}{\textwidth}{l@{\extracolsep{\fill}}r}
\includegraphics[height=2.15in]{SCRIPT_FIGURES/NIST_FSE_2008/ISOHept24_Carbon_Dioxide} &
\includegraphics[height=2.15in]{SCRIPT_FIGURES/NIST_FSE_2008/ISOHept24_Carbon_Monoxide} \\
\includegraphics[height=2.15in]{SCRIPT_FIGURES/NIST_FSE_2008/ISOHept24_Oxygen} &
\includegraphics[height=2.15in]{SCRIPT_FIGURES/NIST_FSE_2008/ISOHept24_Unburned_Hydrocarbons} \\
\includegraphics[height=2.15in]{SCRIPT_FIGURES/NIST_FSE_2008/ISOHept24_Temperature} &
\includegraphics[height=2.15in]{SCRIPT_FIGURES/NIST_FSE_2008/ISOHept24_HRR}
\end{tabular*}
\caption[Summary of ISOHept24, NIST FSE 2008]{Summary of ISOHept24, NIST FSE 2008.}
\label{NIST_FSE_1994_ISOHept24}
\end{figure}

\begin{figure}[p]
\begin{tabular*}{\textwidth}{l@{\extracolsep{\fill}}r}
\includegraphics[height=2.15in]{SCRIPT_FIGURES/NIST_FSE_2008/ISOHept25_Carbon_Dioxide} &
\includegraphics[height=2.15in]{SCRIPT_FIGURES/NIST_FSE_2008/ISOHept25_Carbon_Monoxide} \\
\includegraphics[height=2.15in]{SCRIPT_FIGURES/NIST_FSE_2008/ISOHept25_Oxygen} &
\includegraphics[height=2.15in]{SCRIPT_FIGURES/NIST_FSE_2008/ISOHept25_Unburned_Hydrocarbons} \\
\includegraphics[height=2.15in]{SCRIPT_FIGURES/NIST_FSE_2008/ISOHept25_Temperature} &
\includegraphics[height=2.15in]{SCRIPT_FIGURES/NIST_FSE_2008/ISOHept25_HRR}
\end{tabular*}
\caption[Summary of ISOHept25, NIST FSE 2008]{Summary of ISOHept25, NIST FSE 2008.}
\label{NIST_FSE_1994_ISOHept25}
\end{figure}

\begin{figure}[p]
\begin{tabular*}{\textwidth}{l@{\extracolsep{\fill}}r}
\includegraphics[height=2.15in]{SCRIPT_FIGURES/NIST_FSE_2008/ISOHept26_Carbon_Dioxide} &
\includegraphics[height=2.15in]{SCRIPT_FIGURES/NIST_FSE_2008/ISOHept26_Carbon_Monoxide} \\
\includegraphics[height=2.15in]{SCRIPT_FIGURES/NIST_FSE_2008/ISOHept26_Oxygen} &
\includegraphics[height=2.15in]{SCRIPT_FIGURES/NIST_FSE_2008/ISOHept26_Unburned_Hydrocarbons} \\
\includegraphics[height=2.15in]{SCRIPT_FIGURES/NIST_FSE_2008/ISOHept26_Temperature} &
\includegraphics[height=2.15in]{SCRIPT_FIGURES/NIST_FSE_2008/ISOHept26_HRR}
\end{tabular*}
\caption[Summary of ISOHept26, NIST FSE 2008]{Summary of ISOHept26, NIST FSE 2008.}
\label{NIST_FSE_1994_ISOHept26}
\end{figure}

\begin{figure}[p]
\begin{tabular*}{\textwidth}{l@{\extracolsep{\fill}}r}
\includegraphics[height=2.15in]{SCRIPT_FIGURES/NIST_FSE_2008/ISOHept27_Carbon_Dioxide} &
\includegraphics[height=2.15in]{SCRIPT_FIGURES/NIST_FSE_2008/ISOHept27_Carbon_Monoxide} \\
\includegraphics[height=2.15in]{SCRIPT_FIGURES/NIST_FSE_2008/ISOHept27_Oxygen} &
\includegraphics[height=2.15in]{SCRIPT_FIGURES/NIST_FSE_2008/ISOHept27_Unburned_Hydrocarbons} \\
\includegraphics[height=2.15in]{SCRIPT_FIGURES/NIST_FSE_2008/ISOHept27_Temperature} &
\includegraphics[height=2.15in]{SCRIPT_FIGURES/NIST_FSE_2008/ISOHept27_HRR}
\end{tabular*}
\caption[Summary of ISOHept27, NIST FSE 2008]{Summary of ISOHept27, NIST FSE 2008.}
\label{NIST_FSE_1994_ISOHept27}
\end{figure}

\begin{figure}[p]
\begin{tabular*}{\textwidth}{l@{\extracolsep{\fill}}r}
\includegraphics[height=2.15in]{SCRIPT_FIGURES/NIST_FSE_2008/ISOHept28_Carbon_Dioxide} &
\includegraphics[height=2.15in]{SCRIPT_FIGURES/NIST_FSE_2008/ISOHept28_Carbon_Monoxide} \\
\includegraphics[height=2.15in]{SCRIPT_FIGURES/NIST_FSE_2008/ISOHept28_Oxygen} &
\includegraphics[height=2.15in]{SCRIPT_FIGURES/NIST_FSE_2008/ISOHept28_Unburned_Hydrocarbons} \\
\includegraphics[height=2.15in]{SCRIPT_FIGURES/NIST_FSE_2008/ISOHept28_Temperature} &
\includegraphics[height=2.15in]{SCRIPT_FIGURES/NIST_FSE_2008/ISOHept28_HRR}
\end{tabular*}
\caption[Summary of ISOHept28, NIST FSE 2008]{Summary of ISOHept28, NIST FSE 2008.}
\label{NIST_FSE_1994_ISOHept28}
\end{figure}

\begin{figure}[p]
\begin{tabular*}{\textwidth}{l@{\extracolsep{\fill}}r}
\includegraphics[height=2.15in]{SCRIPT_FIGURES/NIST_FSE_2008/ISOToluene29_Carbon_Dioxide} &
\includegraphics[height=2.15in]{SCRIPT_FIGURES/NIST_FSE_2008/ISOToluene29_Carbon_Monoxide} \\
\includegraphics[height=2.15in]{SCRIPT_FIGURES/NIST_FSE_2008/ISOToluene29_Oxygen} &
\includegraphics[height=2.15in]{SCRIPT_FIGURES/NIST_FSE_2008/ISOToluene29_Unburned_Hydrocarbons} \\
\includegraphics[height=2.15in]{SCRIPT_FIGURES/NIST_FSE_2008/ISOToluene29_Temperature} &
\includegraphics[height=2.15in]{SCRIPT_FIGURES/NIST_FSE_2008/ISOToluene29_HRR}
\end{tabular*}
\caption[Summary of ISOToluene29, NIST FSE 2008]{Summary of ISOToluene29, NIST FSE 2008.}
\label{NIST_FSE_1994_ISOToluene29}
\end{figure}

\begin{figure}[p]
\begin{tabular*}{\textwidth}{l@{\extracolsep{\fill}}r}
\includegraphics[height=2.15in]{SCRIPT_FIGURES/NIST_FSE_2008/ISOPropanol30_Carbon_Dioxide} &
\includegraphics[height=2.15in]{SCRIPT_FIGURES/NIST_FSE_2008/ISOPropanol30_Carbon_Monoxide} \\
\includegraphics[height=2.15in]{SCRIPT_FIGURES/NIST_FSE_2008/ISOPropanol30_Oxygen} &
\includegraphics[height=2.15in]{SCRIPT_FIGURES/NIST_FSE_2008/ISOPropanol30_Unburned_Hydrocarbons} \\
\includegraphics[height=2.15in]{SCRIPT_FIGURES/NIST_FSE_2008/ISOPropanol30_Temperature} &
\includegraphics[height=2.15in]{SCRIPT_FIGURES/NIST_FSE_2008/ISOPropanol30_HRR}
\end{tabular*}
\caption[Summary of ISOPropanol30, NIST FSE 2008]{Summary of ISOPropanol30, NIST FSE 2008.}
\label{NIST_FSE_1994_ISOPropanol30}
\end{figure}

\begin{figure}[p]
\begin{tabular*}{\textwidth}{l@{\extracolsep{\fill}}r}
\includegraphics[height=2.15in]{SCRIPT_FIGURES/NIST_FSE_2008/ISONG32_Carbon_Dioxide} &
\includegraphics[height=2.15in]{SCRIPT_FIGURES/NIST_FSE_2008/ISONG32_Carbon_Monoxide} \\
\includegraphics[height=2.15in]{SCRIPT_FIGURES/NIST_FSE_2008/ISONG32_Oxygen} &
\includegraphics[height=2.15in]{SCRIPT_FIGURES/NIST_FSE_2008/ISONG32_Unburned_Hydrocarbons} \\
\includegraphics[height=2.15in]{SCRIPT_FIGURES/NIST_FSE_2008/ISONG32_Temperature} &
\includegraphics[height=2.15in]{SCRIPT_FIGURES/NIST_FSE_2008/ISONG32_HRR}
\end{tabular*}
\caption[Summary of ISONG32, NIST FSE 2008]{Summary of ISONG32, NIST FSE 2008.}
\label{NIST_FSE_1994_ISONG32}
\end{figure}


\clearpage

%\subsection{PRISME DOOR Experiments}
%
%Each compartment in the PRISME DOOR experiments contained carbon monoxide measurements in the upper (haut) and lower (bas) layers.
%
%\begin{figure}[!ht]
%\begin{tabular*}{\textwidth}{l@{\extracolsep{\fill}}r}
%\includegraphics[height=2.15in]{SCRIPT_FIGURES/PRISME/PRS_D1_Room_1_CO} &
%\includegraphics[height=2.15in]{SCRIPT_FIGURES/PRISME/PRS_D2_Room_1_CO} \\
%\includegraphics[height=2.15in]{SCRIPT_FIGURES/PRISME/PRS_D3_Room_1_CO} &
%\includegraphics[height=2.15in]{SCRIPT_FIGURES/PRISME/PRS_D4_Room_1_CO} \\
%\includegraphics[height=2.15in]{SCRIPT_FIGURES/PRISME/PRS_D5_Room_1_CO} &
%\includegraphics[height=2.15in]{SCRIPT_FIGURES/PRISME/PRS_D6_Room_1_CO}
%\end{tabular*}
%\label{PRISME_CO_1}
%\end{figure}
%
%\begin{figure}[p]
%\begin{tabular*}{\textwidth}{l@{\extracolsep{\fill}}r}
%\includegraphics[height=2.15in]{SCRIPT_FIGURES/PRISME/PRS_D1_Room_2_CO} &
%\includegraphics[height=2.15in]{SCRIPT_FIGURES/PRISME/PRS_D2_Room_2_CO} \\
%\includegraphics[height=2.15in]{SCRIPT_FIGURES/PRISME/PRS_D3_Room_2_CO} &
%\includegraphics[height=2.15in]{SCRIPT_FIGURES/PRISME/PRS_D4_Room_2_CO} \\
%\includegraphics[height=2.15in]{SCRIPT_FIGURES/PRISME/PRS_D5_Room_2_CO} &
%\includegraphics[height=2.15in]{SCRIPT_FIGURES/PRISME/PRS_D6_Room_2_CO}
%\end{tabular*}
%\label{PRISME_CO_2}
%\end{figure}
%
%\clearpage

\subsection{Summary, Products of Incomplete Combustion}
\label{Carbon Monoxide Concentration}

\begin{figure}[h]
\begin{center}
\begin{tabular}{c}
\includegraphics[width=4.0in]{SCRIPT_FIGURES/ScatterPlots/FDS_Carbon_Monoxide_Concentration}
\end{tabular}
\end{center}
\caption[Summary of carbon monoxide predictions]{Summary of carbon monoxide predictions.}
\label{Summary_CO_Concentration}
\end{figure}

\clearpage

\section{Helium Release in a Reduced Scale Garage Geometry}
\label{Species Concentration}

FDS simulations were performed to predict the helium release and dispersion in a reduced scale garage geometry. The figures on the following pages show the comparison between the FDS predictions and the measured values for the eighteen experiments. Table~\ref{NIST_He_2009_Parameters} lists the experimental parameters, including the release duration, release location (21~cm off the floor at the center of the compartment, 21~cm off the floor and 5~cm from the center of the rear wall, and 2.5~cm below the ceiling at the center of the compartment), and the leak area (single small vent, 2.4~cm by 2.4~cm, at the center of the front wall, single large vent, 3.05~cm by 3.05~cm, at the center of the front wall, and a pair of vents, 2.15~cm by 2.15~cm, centered on the front wall, 2.5~cm from the floor and ceiling, respectively). The seven sensors were located 37.5~cm from the front and side walls, at heights of 9~cm, 19~cm, 28~cm, 37~cm, 47~cm, 56~cm, and 65~cm off the floor. In the figures on the following pages, the highest concentrations correspond to the highest measurement locations.

\begin{table}[h]
\centering
\caption[Test parameters of the NIST\_He\_2009 experiments]{Test parameters of the NIST\_He\_2009 experiments.}
\begin{tabular}{|c|c|c|c|}
\hline
Test         &  Release          &  Release          &  Leak                         \\
Label        &  Duration (h)     &  Location         &  Configuration                \\ \hline \hline
3600-LC-SSV  &  1                &  Lower Center     &  Single Small Vent            \\ \hline
3600-LC-SLV  &  1                &  Lower Center     &  Single Large Vent            \\ \hline
3600-LC-ULV  &  1                &  Lower Center     &  Dual Vents                   \\ \hline
3600-LR-SSV  &  1                &  Lower Rear       &  Single Small Vent            \\ \hline
3600-LR-SLV  &  1                &  Lower Rear       &  Single Large Vent            \\ \hline
3600-LR-ULV  &  1                &  Lower Rear       &  Dual Vents                   \\ \hline
3600-UC-SSV  &  1                &  Upper Center     &  Single Small Vent            \\ \hline
3600-UC-SLV  &  1                &  Upper Center     &  Single Large Vent            \\ \hline
3600-UC-ULV  &  1                &  Upper Center     &  Dual Vents                   \\ \hline
14400-LC-SSV &  4                &  Lower Center     &  Single Small Vent            \\ \hline
14400-LC-SLV &  4                &  Lower Center     &  Single Large Vent            \\ \hline
14400-LC-ULV &  4                &  Lower Center     &  Dual Vents                   \\ \hline
14400-LR-SSV &  4                &  Lower Rear       &  Single Small Vent            \\ \hline
14400-LR-SLV &  4                &  Lower Rear       &  Single Large Vent            \\ \hline
14400-LR-ULV &  4                &  Lower Rear       &  Dual Vents                   \\ \hline
14400-UC-SSV &  4                &  Upper Center     &  Single Small Vent            \\ \hline
14400-UC-SLV &  4                &  Upper Center     &  Single Large Vent            \\ \hline
14400-UC-ULV &  4                &  Upper Center     &  Dual Vents                   \\ \hline
\end{tabular}
\label{NIST_He_2009_Parameters}
\end{table}

\newpage

\begin{figure}[p]
\begin{tabular*}{\textwidth}{l@{\extracolsep{\fill}}r}
\includegraphics[height=2.15in]{SCRIPT_FIGURES/NIST_He_2009/NIST_He_3600_LC_SSV} &
\includegraphics[height=2.15in]{SCRIPT_FIGURES/NIST_He_2009/NIST_He_3600_LC_SLV} \\
\includegraphics[height=2.15in]{SCRIPT_FIGURES/NIST_He_2009/NIST_He_3600_LC_ULV} &
\includegraphics[height=2.15in]{SCRIPT_FIGURES/NIST_He_2009/NIST_He_3600_LR_SSV} \\
\includegraphics[height=2.15in]{SCRIPT_FIGURES/NIST_He_2009/NIST_He_3600_LR_SLV} &
\includegraphics[height=2.15in]{SCRIPT_FIGURES/NIST_He_2009/NIST_He_3600_LR_ULV}
\end{tabular*}
\caption[Results of the NIST\_He\_2009 experiments]{Comparison of measured (solid lines) and predicted (dashed lines) helium concentrations in the NIST\_He\_2009 experiments.}
\label{NIST_Hydrogen_Species_1}
\end{figure}

\begin{figure}[p]
\begin{tabular*}{\textwidth}{l@{\extracolsep{\fill}}r}
\includegraphics[height=2.15in]{SCRIPT_FIGURES/NIST_He_2009/NIST_He_3600_UC_SSV} &
\includegraphics[height=2.15in]{SCRIPT_FIGURES/NIST_He_2009/NIST_He_3600_UC_SLV} \\
\includegraphics[height=2.15in]{SCRIPT_FIGURES/NIST_He_2009/NIST_He_3600_UC_ULV} &
\includegraphics[height=2.15in]{SCRIPT_FIGURES/NIST_He_2009/NIST_He_14400_LC_SSV} \\
\includegraphics[height=2.15in]{SCRIPT_FIGURES/NIST_He_2009/NIST_He_14400_LC_SLV} &
\includegraphics[height=2.15in]{SCRIPT_FIGURES/NIST_He_2009/NIST_He_14400_LC_ULV}
\end{tabular*}
\caption[Results of the NIST\_He\_2009 experiments]{Comparison of measured (solid lines) and predicted (dashed lines) helium concentrations in the NIST\_He\_2009 experiments.}
\label{NIST_Hydrogen_Species_2}
\end{figure}

\begin{figure}[p]
\begin{tabular*}{\textwidth}{l@{\extracolsep{\fill}}r}
\includegraphics[height=2.15in]{SCRIPT_FIGURES/NIST_He_2009/NIST_He_14400_LR_SSV} &
\includegraphics[height=2.15in]{SCRIPT_FIGURES/NIST_He_2009/NIST_He_14400_LR_SLV} \\
\includegraphics[height=2.15in]{SCRIPT_FIGURES/NIST_He_2009/NIST_He_14400_LR_ULV} &
\includegraphics[height=2.15in]{SCRIPT_FIGURES/NIST_He_2009/NIST_He_14400_UC_SSV} \\
\includegraphics[height=2.15in]{SCRIPT_FIGURES/NIST_He_2009/NIST_He_14400_UC_SLV} &
\includegraphics[height=2.15in]{SCRIPT_FIGURES/NIST_He_2009/NIST_He_14400_UC_ULV}
\end{tabular*}
\caption[Results of the NIST\_He\_2009 experiments]{Comparison of measured (solid lines) and predicted (dashed lines) helium concentrations in the NIST\_He\_2009 experiments.}
\label{NIST_Hydrogen_Species_3}
\end{figure}

\begin{figure}[p]
\begin{center}
\begin{tabular}{c}
\includegraphics[width=4.0in]{SCRIPT_FIGURES/ScatterPlots/FDS_Species_Concentration}
\end{tabular}
\end{center}
\caption[Summary of species concentration predictions]{Summary of species concentration predictions.}
\label{Summary_Species_Concentration}
\end{figure}


