% !TEX root = Correlation_Guide.tex

\chapter{Target Temperature}
\label{Target_Temperature_Chapter}

\section{Cable Failure Time}

\subsection*{Description}

The thermally-induced electrical failure (THIEF) of a cable can be predicted via a simple
one-dimensional heat transfer calculation, under the assumption that the cable can be
treated as a homogeneous cylinder~\cite{CAROLFIRE}. The governing equation for the cable temperature,
$T(r,t)$, is given by
\be
\rho c \left( \frac{\partial T}{\partial t} \right) = \frac{1}{r} \frac{\partial}{\partial r} k r \left( \frac{\partial T}{\partial r} \right)
\label{eq:cable_temp}
\ee
with the boundary condition
\be
\dot q'' = k \left( \frac{\partial T}{\partial r} \right) (R,t)
\ee
A finite difference approximation to Eq.~\ref{eq:cable_temp} is given by
\be
\rho c \left[ \frac{T_i^{n+1} - T_i^n}{\delta t} \right] = \frac{2 k}{(r_{i+1} + r_i)} \frac{1}{2 \delta r} \left[ r_i \frac{T_{i+1}^n - T_i^n}{\delta r} - r_{i-1} \frac{T_{i}^n - T_{i-1}^n}{\delta r} + r_i \frac{T_{i+1}^{n+1} - T_i^{n+1}}{\delta r} - r_{i-1} \frac{T_{i}^{n+1} - T_{i-1}^{n+1}}{\delta r} \right]
\ee
where the time step $\delta t$ is given by
\be
\delta t = \frac{c \rho \delta r^2}{2 k}
\ee


\clearpage


\subsection*{Verification}

Test: CAROLFIRE Penlight Test 7

\begin{table}[!ht]
\caption[Verification input parameters, cable failure time]
{Verification input parameters, cable failure time.}
\begin{center}
\begin{tabular}{|l|l|}
\hline
                             &                                \\
\rb{Input Parameter}         &  \rb{Value}                    \\ \hline \hline
Time Ramp                    &  0, 80, 820, 1240, 1800, 1900  \\ \hline
Temperature Ramp             &  24, 460, 460, 460, 460, 0     \\ \hline
Cable Diameter (mm)          &  16.3                          \\ \hline
Mass per Unit Length (kg/m)  &  0.529                         \\ \hline
Jacket Thickness (mm)        &  1.5                           \\ \hline
Conduit Diameter (mm)        &  50                            \\ \hline
Conduit Thickness (mm)       &  4.9                           \\ \hline
$T_\infty$ ($^\circ$C)       &  24                            \\ \hline
\end{tabular}
\end{center}
\end{table}

\noindent Expected result: At 50.0~s, the exposing temperature is 296.3~$^\circ$C, the cable temperature is 24.1~$^\circ$C, and the conduit temperature is 32.4~$^\circ$C; at 80~s, the exposing temperature is 460.0~$^\circ$C, the cable temperature is 24.6~$^\circ$C, and the conduit temperature is 52.3~$^\circ$C. At 1473~s, the cable reaches a failure temperature of 400~$^\circ$C.


\clearpage


\subsection*{Validation}

\begin{figure}[!ht]
\begin{center}
\begin{tabular}{l}
\includegraphics[width=4.0in]{SCRIPT_FIGURES/Scatterplots/Cable_Failure_Time}
\end{tabular}
\end{center}
\caption[Summary of cable failure time predictions]
{Summary of cable failure time predictions.}
\label{Surface_Temperature_THIEF_Summary}
\end{figure}

\clearpage


\section{Unprotected Steel Temperature}

\subsection*{Description}

The temperature rise, $\Delta T\sb{s}$, of an unprotected steel member exposed to fire can be predicted using~\cite{SFPE:Milke2}
\be
\Delta T\sb{s} = \frac{F}{V} \frac{1}{\rho\sb{s} c\sb{s}} \left[ h\sb{c} (T\sb{f} - T\sb{s}) + \sigma \epsilon (T\sb{f}^4 - T\sb{s}^4) \right] \Delta t
\label{eq:unprotected_steel}
\ee
where $F/V$ is the ratio of heated surface area to volume~(\si{m^{-1}}), $\rho\sb{s}$ is the density of steel~(\si{kg/m^3}), $c\sb{s}$ is the specific heat of steel~(\si{J/(kg.K)}), $h\sb{c}$ is the convective heat transfer coefficient~(\si{W/(m^2.K)}), $T\sb{f}$ is the exposing fire temperature~(\si{K}), $T\sb{s}$ is the steel temperature~(\si{K}), $\sigma$ is the Stefan-Boltzmann constant (\si{W/(m^2.K^4)}), $\epsilon$ is the flame emissivity~(-), and $\Delta t$ is the time step~(\si{s}). Note that the HGL temperature, plume temperature, or other exposing temperature can be used as the fire temperature, $T\sb{f}$.

\subsection*{Verification}

Test: SP AST Column, 1.1 m Diesel Fire

\begin{table}[!ht]
\caption[Verification input parameters, unprotected steel temperature]
{Verification input parameters, unprotected steel temperature.}
\begin{center}
\begin{tabular}{|l|l|}
\hline
                               &              \\
\rb{Input Parameter}           &  \rb{Value}  \\ \hline \hline
$F/V$ (1/m)                    &  205         \\ \hline
$\rho\sb{s}$ (kg/m$^3$)        &  7850        \\ \hline
$c\sb{s}$ (\si{kJ/(kg.K)})     &  0.6         \\ \hline
$\epsilon$ (-)                 &  0.7         \\ \hline
$h\sb{c}$ (\si{W/(m^2.K)})     &  25          \\ \hline \hline
Correlation for $T\sb{f}$ (-)  &  McCaffrey   \\ \hline \hline
$\dot Q$ (kW)                  &  1434        \\ \hline
Height (m)                     &  1           \\ \hline
\end{tabular}
\end{center}
\end{table}

\noindent Expected result: The fire temperature (plume temperature at a height of 1~m from McCaffrey) is 872.51~$^\circ$C. At 15~s, the steel temperature is 74.0~$^\circ$C; at 30~s, the steel temperature is 130.7~$^\circ$C; at 45~s, the steel temperature is 186.1~$^\circ$C.


\clearpage


\section{Protected Steel Temperature}
\label{info:protected_steel_temperature}

\subsection*{Description}

The temperature rise, $\Delta T\sb{s}$, of a protected steel member exposed to fire can be predicted, but we must first determine if the thermal capacity of the insulation layer should be accounted for or if it can be neglected.
\be
\Delta T\sb{s} = \left\{ \begin{array}{cl}
   k\sb{i} \left( \frac{T\sb{f} - T\sb{s}}{c\sb{s} h \frac{W}{D}} \right) \Delta t        &  c\sb{s} \frac{W}{D} > 2 c\sb{i} \rho\sb{i} h \\[0.1in]
   \frac{k\sb{i}}{h} \left( \frac{T\sb{f} - T\sb{s}}{c\sb{s} \frac{W}{D} + \frac{1}{2} c\sb{i} \rho\sb{i} h} \right) \Delta t  &  c\sb{s} \frac{W}{D} < 2 c\sb{i} \rho\sb{i} h
   \end{array} \right.
\label{eq:protected_steel}
\ee
where $k\sb{i}$ is the thermal conductivity of the insulation material~(\si{W/(m.K)}), $T\sb{f}$ is the exposing fire temperature~(\si{K}), $T\sb{s}$ is the steel temperature~(\si{K}), $c\sb{s}$ is the specific heat of steel~(\si{J/(kg.K)}), $c\sb{i}$ is the specific heat of the insulation material~(\si{J/(kg.K)}), $h$ is the thickness of the insulation~(\si{m}), $W/D$ is the ratio of the weight of steel section per unit length to the heated perimeter~(\si{kg/m^2}), $\rho\sb{i}$ is the density of the insulating material~(\si{kg/m^3}), and $\Delta t$ is the time step~(\si{s}). Note that the HGL temperature, plume temperature, or other exposing temperature can be used as the fire temperature, $T\sb{f}$.


\clearpage


\subsection*{Verification}

Test: WTC Test 4, Bar Structural Element

\begin{table}[!ht]
\caption[Verification input parameters, protected steel temperature]
{Verification input parameters, protected steel temperature.}
\begin{center}
\begin{tabular}{|l|l|}
\hline
                               &              \\
\rb{Input Parameter}           &  \rb{Value}  \\ \hline \hline
$c\sb{s}$ (\si{kJ/(kg.K)})     &  0.450       \\ \hline
$W/D$ (kg/m$^2$)               &  51.1        \\ \hline
$k\sb{i}$ (\si{W/(m.K)})       &  0.10        \\ \hline
$\rho\sb{i}$ (kg/m$^3$)        &  208         \\ \hline
$c\sb{i}$ (\si{kJ/(kg.K)})     &  2.0         \\ \hline
$h\sb{i}$ (m)                  &  0.0191      \\ \hline \hline
Correlation for $T\sb{f}$ (-)  &  MQH         \\ \hline \hline
$\dot Q$ (kW)                  &  3200        \\ \hline
$L$ (m)                        &  7.04        \\ \hline
$W$ (m)                        &  3.60        \\ \hline
$H$ (m)                        &  3.82        \\ \hline
$H\sb{o}$ (m)                  &  2.82        \\ \hline
$W\sb{o}$ (m)                  &  2.4         \\ \hline
$k$ (\si{kW/(m.K)})            &  0.00012     \\ \hline
$\rho$ (kg/m$^3$)              &  737         \\ \hline
$c$ (\si{kJ/(kg.K)})           &  0.9         \\ \hline
$\delta$ (m)                   &  0.0254      \\ \hline
$T_\infty$ ($^\circ$C)         &  20          \\ \hline
\end{tabular}
\end{center}
\end{table}

\noindent Expected result: At 15~s, the HGL temperature (from MQH) is 336.7~$^\circ$C and the steel temperature is 20.94~$^\circ$C; At 30~s, the HGL temperature (from MQH) is 375.4~$^\circ$C and the steel temperature is 22.1~$^\circ$C; at 45~s, the HGL temperature (from MQH) is 400.3~$^\circ$C and the steel temperature is 23.36~$^\circ$C.


\clearpage


\subsection*{Validation}

\begin{figure}[!ht]
\begin{center}
\begin{tabular}{l}
\includegraphics[width=4.0in]{SCRIPT_FIGURES/Scatterplots/Target_Temperature}
\end{tabular}
\end{center}
\caption[Summary of target temperature predictions]
{Summary of target temperature predictions.}
\label{Surface_Temperature_Steel_Summary}
\end{figure}

