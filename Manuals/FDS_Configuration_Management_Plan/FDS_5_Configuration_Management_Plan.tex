\documentclass[11pt]{book}
\usepackage{mathptm,times}
\usepackage[pdftex]{graphicx}
%\usepackage{hyperref}
%\usepackage{array,eqnarray}
%\usepackage{eso-pic}
%\usepackage{graphicx}
%\usepackage{color}
%\usepackage{type1cm}

%\makeatletter
%   \AddToShipoutPicture{%
%     \setlength{\@tempdimb}{.5\paperwidth}%
%    \setlength{\@tempdimc}{.5\paperheight}%
%   \setlength{\unitlength}{1pt}%
%  \put(\strip@pt\@tempdimb,\strip@pt\@tempdimc){%
%     \makebox(0,0){\rotatebox{45}{\textcolor[gray]{0.75}{\fontsize{8cm}{8cm}\selectfont{DRAFT}}}}}}
%\makeatother

\usepackage[pdftex,
        colorlinks=true,
        urlcolor=linkblue,     % \href{...}{...} external (URL)
        citecolor=linkred,     % citation number colors
        linkcolor=linknavy,    % \ref{...} and \pageref{...}
        pdftitle={Fire Dynamics Simulator (Version 5) Configuration Management Plan},
        pdfauthor={Bryan Klein, Jason Floyd},
        pdfsubject={Software Configuration Management Guide},
        pdfkeywords={FDS, Fire Model, NIST, BFRL, SCM},
        pdfproducer={pdflatex},
        pagebackref,
        pdfpagemode=UseNone,
        bookmarksopen=true,
        plainpages=false]{hyperref}
\usepackage{color}
\definecolor{linknavy}{rgb}{0,0,0.50196}
\definecolor{linkred}{rgb}{1,0,0}
\definecolor{linkblue}{rgb}{0,0,1}
\usepackage{caption}
\usepackage{graphpap}
\usepackage{rotating}
\usepackage{epsfig,psfrag}
\usepackage{geometry}
\usepackage{tabularx}
\usepackage{longtable}
\usepackage{lscape}
\usepackage{amssymb}
\usepackage{makeidx} % Create index at end of document
\usepackage[nottoc,notlof,notlot]{tocbibind} % Put the bibliography and index in the ToC
\usepackage{float}
\usepackage{lastpage} % Automatic last page number reference.
\usepackage[T1]{fontenc}
\usepackage{upquote}
\usepackage{array,eqnarray}
\newcommand{\nopart}{\expandafter\def\csname Parent-1\endcsname{}} % To fix table of contents in pdf.


\setlength{\textwidth}{6.5in}
\setlength{\textheight}{9.0in}
\setlength{\topmargin}{0.in}
\setlength{\headheight}{0.in}
\setlength{\headsep}{0.in}
\setlength{\parindent}{0.25in}
\setlength{\oddsidemargin}{0.0in}
\setlength{\evensidemargin}{0.0in}

\begin{document}

\bibliographystyle{unsrt}

\newcommand{\dod}[2]{\frac{\partial #1}{\partial #2}}
\newcommand{\DoD}[2]{\frac{D #1}{D #2}}
\newcommand{\dsods}[2]{\frac{\partial^2 #1}{\partial #2^2}}
\newcommand{\dx}{\delta x}
\newcommand{\dy}{\delta y}
\newcommand{\dz}{\delta z}
\newcommand{\x}{x}
\newcommand{\y}{y}
\newcommand{\z}{z}
\newcommand{\dt}{\delta t}
\newcommand{\dn}{\delta n}
\newcommand{\cH}{{\cal H}}
\newcommand{\hu}{u}
\newcommand{\hv}{v}
\newcommand{\hw}{w}
\newcommand{\la}{\lambda}
%\newcommand{\bO}{\mbox{\boldmath $\Omega$}}
\newcommand{\bO}{{\Omega}}
\newcommand{\bo}{{\bf \omega}}
%\newcommand{\btau}{\mbox{\boldmath $\tau$}}
\newcommand{\btau}{{\bf \tau}}
\newcommand{\bdelta}{{\bf \delta}}
\newcommand{\sumyw}{\sum (Y_\alpha/W_\alpha)}
\newcommand{\oW}{\overline{W}}
\newcommand{\om}{\omega}
\newcommand{\omx}{\omega_x}
\newcommand{\omy}{\omega_y}
\newcommand{\omz}{\omega_z}
\newcommand{\erf}{\hbox{erf}}
\newcommand{\bF}{{\bf F}}
\newcommand{\bG}{{\bf G}}
\newcommand{\bof}{{\bf f}}
\newcommand{\bq}{{\bf q}}
\newcommand{\br}{{\bf r}}
\newcommand{\bu}{{\bf u}}
\newcommand{\bx}{{\bf x}}
\newcommand{\bk}{{\bf k}}
\newcommand{\bv}{{\bf v}}
\newcommand{\bg}{{\bf g}}
\newcommand{\bn}{{\bf n}}
\newcommand{\bS}{{\bf S}}
\newcommand{\bW}{\overline{W}}
\newcommand{\dS}{d{\bf S}}
\newcommand{\bs}{{\bf s}}
\newcommand{\bI}{{\bf I}}
\newcommand{\hp}{{\cal H}}
\newcommand{\trho}{\tilde{\rho}}
\newcommand{\dph}{{\delta\phi}}
\newcommand{\dth}{{\delta\theta}}
\newcommand{\tp}{\tilde{p}}
\newcommand{\bp}{\overline{p}}
\newcommand{\dQ}{\dot{Q}}
\newcommand{\dq}{\dot{q}}
\newcommand{\dbq}{\dot{\bf q}}
\newcommand{\dm}{\dot{m}}
\newcommand{\ha}{\frac{1}{2}}
\newcommand{\ft}{\frac{4}{3}}
\newcommand{\ot}{\frac{1}{3}}
\newcommand{\fofi}{\frac{4}{5}}
\newcommand{\of}{\frac{1}{4}}
\newcommand{\twth}{\frac{2}{3}}
\newcommand{\R}{{\cal R}}
\newcommand{\be}{\begin{equation}}
\newcommand{\ee}{\end{equation}}
\newcommand{\RE}{\hbox{Re}}
\newcommand{\LE}{\hbox{Le}}
\newcommand{\PR}{\hbox{Pr}}
\newcommand{\PE}{\hbox{Pe}}
\newcommand{\NU}{\hbox{Nu}}
\newcommand{\SC}{\hbox{Sc}}
\newcommand{\SH}{\hbox{Sh}}
\newcommand{\WE}{\hbox{We}}
\newcommand{\COTWO}{{\tiny \hbox{CO}_2}}
\newcommand{\HTWOO}{{\tiny \hbox{H}_2\hbox{O}}}
\newcommand{\OTWO}{{\tiny \hbox{O}_2}}
\newcommand{\NTWO}{{\tiny \hbox{N}_2}}
\newcommand{\CO}{{\tiny \hbox{CO}}}
\newcommand{\F}{{\tiny \hbox{F}}}
\newcommand{\C}{{\tiny \hbox{C}}}
\newcommand{\Hy}{{\tiny \hbox{H}}}
\newcommand{\So}{{\tiny \hbox{S}}}
\newcommand{\M}{{\tiny \hbox{M}}}
\newcommand{\xx}{{\tiny \hbox{x}}}
\newcommand{\yy}{{\tiny \hbox{y}}}
\newcommand{\zz}{{\tiny \hbox{z}}}
\newcommand{\ct}{\tt\small}
\newcommand{\dif}{\mathrm{d}}
\newcommand{\Div}{\nabla\cdot}
\newcommand{\mhalf}{\mbox{$\frac{1}{2}$}}
\newcommand{\tripleprime}{{\prime\prime\prime}}

\pagestyle{empty}

\begin{minipage}[t][9in][s]{6.5in}

\huge \flushright{NIST Special Publication 1018-5}

\vspace{1in}

\Huge \flushright{Fire Dynamics Simulator (Version 5) \\ Technical Reference Guide \\ \Large Volume 4: Software Configuration Management Plan}

\vspace{.5in}

\large
\flushright{
Bryan Klein}

\vspace{0.5in}

\flushright{In cooperation with: \\
VTT Technical Research Centre of Finland  }

\vfill

\flushright{\includegraphics[width=2.in]{FIGURES/nistident_flright_vec}}

\end{minipage}

\newpage

\hspace{5in}

\newpage

\begin{minipage}[t][9in][s]{6.5in}

\huge \flushright{NIST Special Publication 1018-5}

\vspace{.75in}

\Huge \flushright{Fire Dynamics Simulator (Version 5) \\ Technical Reference Guide \\ \Large Volume 4: Configuration Management Plan}

\vspace{.25in}

\normalsize
\flushright{
Bryan Klein \\
{\em Fire Research Division} \\
{\em Building and Fire Research Laboratory}}


\vspace{.25in}

\flushright{\today \\
FDS Version 5.3 \\
$SVN Repository$~$Revision$}

\vfill

\flushright{\includegraphics[width=1in]{FIGURES/doc} }

\small
\flushright{U.S. Department of Commerce \\
{\em Carlos M. Gutierrez, Secretary} \\
\hspace{1in} \\
National Institute of Standards and Technology \\
{\em Patrick Gallagher, Acting Director} }

\end{minipage}

\newpage

\begin{minipage}[t][9in][s]{6.5in}

\flushright{Certain commercial entities, equipment, or materials may be identified in this \\
document in order to describe an experimental procedure or concept adequately. Such \\
identification is not intended to imply recommendation or endorsement by the \\
National Institute of Standards and Technology, nor is it intended to imply that the \\
entities, materials, or equipment are necessarily the best available for the purpose.
}

\vspace{3in}

\large
\flushright{\bf National Institute of Standards and Technology Special Publication 1018-5 \\
Natl.~Inst.~Stand.~Technol.~Spec.~Publ.~1018-5, \pageref{LastPage} pages (October 2007) \\
CODEN: NSPUE2 }

\vfill

\flushright{U.S. GOVERNMENT PRINTING OFFICE \\
WASHINGTON: 2007 \\
\rule{3.5in}{0.01in} \\
For sale by the Superintendent of Documents, U.S. Government Printing Office \\
Internet: bookstore.gpo.gov -- Phone: (202) 512-1800 -- Fax: (202) 512-2250 \\
Mail: Stop SSOP, Washington, DC 20402-0001 }

\end{minipage}



\newpage

\frontmatter

\pagestyle{plain}

\chapter{Disclaimer}

The US Department of Commerce makes no warranty, expressed or implied,
to users of the Fire Dynamics Simulator (FDS), and accepts no responsibility for its use.
Users of FDS assume sole responsibility under Federal law for determining
the appropriateness of its use in any particular application;
for any conclusions drawn from the results of its use; and for any
actions taken or not taken as a result of analysis performed using these tools.

Users are warned that FDS is intended for use only by those competent
in the fields of fluid dynamics, thermodynamics, heat transfer, combustion, and fire science,
and is intended only to supplement the informed judgment of the qualified user.
The software package is a computer model that may or may not have predictive
capability when applied to a specific set of factual circumstances.
Lack of accurate predictions by the model could lead to erroneous
conclusions with regard to fire safety. All results should be evaluated by an informed user.

Throughout this document, the mention of computer hardware or commercial
software does not constitute endorsement by NIST, nor does it indicate that
the products are necessarily those best suited for the intended purpose.


\chapter{About the Author}

\begin{description}
\item[Bryan Klein], the current Configuration Manager, is an Information Technology Specialist in the
Building and Fire Research Laboratory of NIST. His current focus is on FDS development and
user support, along with experimental model validation work.

\end{description}

\tableofcontents

\mainmatter

\chapter{Introduction}


\section{Purpose}

The purpose of this document is to describe the policies and procedures for developing and maintaining the Fire Dynamics Simulator (FDS). Such a document is 
commonly referred to as a {\em Configuration Management Plan}. This document will be updated as the necessity arises and will establish and provide 
the basis for a uniform and concise standard of practice for the FDS development process. It is based in part on IEEE Standard 828-2005.


\subsection{Overview of the Fire Dynamics Simulator (FDS) Software Project}

FDS is a Computational Fluid Dynamics (CFD) model of fire-driven fluid flow.
The model solves numerically a form of the Navier-Stokes equations appropriate
for low-speed, thermally-driven flow with an emphasis on smoke and heat transport
from fires. The partial derivatives of the conservation equations of mass, momentum and energy are approximated
as finite differences, and the solution is updated in time on a three-dimensional, rectilinear grid.
Thermal radiation is computed using a finite volume technique on the same grid as the flow solver.
Lagrangian particles are used to simulate smoke movement, sprinkler discharge, and fuel sprays.

\subsection{Applicability}

This document applies only to the program FDS, its companion visualization program called Smokeview,  the related NIST publications and Internet site, and
the various utility programs that support the two main programs. This document does not apply to third-party software developed by others not affiliated with
NIST or its software development collaborators.


\subsection{Assumptions}

The policies and procedures described in this document involve the use of a variety of open source and commercial software and Internet utilities.  
The continued availability of these resources is an important consideration in the day to day upkeep of FDS and Smokeview.  
If these resources become unavailable for any reason, the procedures will be updated accordingly.




\chapter{Roles and Responsibilities}

This chapter describes the roles and responsibilities of the FDS and Smokeview developers.

\section{Development Team Members}

Primary support for the development and maintenance of the Fire Dynamics Simulator (FDS) and Smokeview is provided by the Building and Fire
Research Laboratory (BFRL) of the National Institute of Standards and Technology (NIST), an agency of the United States Department of Commerce. 
The software developers themselves and experimental support personnel
work mainly within the Fire Research Division of BFRL. In addition, external collaborators include staff members of the research laboratory VTT of Finland, guest researchers at NIST, 
and grantees of the NIST extramural fire grants program.

Members of the FDS and Smokeview team share in the development and maintenance responsibilities, which include: 
\begin{enumerate}
\item Developing new algorithms and improving program functionality
\item Answering user questions
\item Responding to bug reports
\item Issuing periodic updates to the officially released versions of the programs
\item Maintaining the Technical Reference and Users Guides
\item Maintaining a suite of sample cases to demonstrate model use
\item Maintaining a suite of verification and validation cases to test model accuracy and reliability
\item Developing and maintaining a ``Road Map'' of future development
\end{enumerate}
Most decisions concerning these various tasks are made by consensus of the team members. In the event of a disagreement over technical issues, the decision is
made by Kevin McGrattan, leader of the Fire Modeling Group in the Fire Research Division of BFRL. 
Non-technical issues are addressed by Anthony Hamins, Chief of the Fire Research Division in BFRL, or by
Shyam Sunder, Director of BFRL, depending on the nature of the issue. All policies and procedures described in this document are
subject to review by NIST in consultation with VTT, Finland. 






\chapter{Software Maintenance}

Software maintenance consists of the following components:
control of the project files and documents, procedures for making changes, and procedures for testing new versions.


\section{Document Identification and Control}

Document identification and control consists of placing all project files in a central location and maintaining a record of
changes to those files. The central location is known as the {\em Repository}. 


\subsection{Project Repository}

All project documents are maintained using the online utility GoogleCode, a free service offered by Google
to support software development for open source applications.  GoogleCode uses
Subversion (SVN) for revision control.  Under this system a centralized repository containing all project files resides
on a GoogleCode server.  Subversion uses a single integer that identifies the version of the
entire repository rather than of a specific file (i.e. anytime a change is made to the repository all files are
incremented in version number).  A record of version number when a specific file was last changed is maintained.

As an open source program, any individual can obtain a copy of the repository or retrieve specific versions
repository.  Only Team Members described in the previous chapter can commit changes to the repository.

The current location of the FDS repository is \href{http://fds-smv.googlecode.com/svn/trunk/}
{{\ct http://fds-smv.googlecode.com/svn/trunk/}}. The repository contains the following files:
\begin{enumerate}
\item Compiled FDS and Smokeview executables
\item FDS and Smokeview source code files
\item FDS and Smokeview documentation
\item Input files for software testing, verification testing, and validation testing
\item Experimental data files used for validation testing
\item Scripts and post-processing utilities used for software testing
\item Web pages and wikis
\end{enumerate}

\subsection{Version Identification}

At the start of an FDS simulation FDS writes header information to the Smokeview output file, FDS output file,
and the FDS log file.  This header information contains the version of FDS used to perform that simulation.
While each release is tagged with a specific version number (e.g. 5.0.1), there may be many commits of source
code, documentation, or other files to the SVN repository before a new version is released with an incremented
version number.  Thus, if a developer or a user who performs their own compilation between baseline releases
discovers an error, the version number written to the output files may not be sufficient to identify the
specific set of source code files used.  Rather one would need to know the SVN revision number of the most
recently committed source file.

This is accomplished by using source file tagging.  In each source file a series of character parameter
strings are defined.  For example the strings in main.f90 are:

\footnotesize
\begin{verbatim}
CHARACTER(255), PARAMETER :: mainid='$Id$'
CHARACTER(255), PARAMETER :: mainrev='$Revision$'
CHARACTER(255), PARAMETER :: maindate='$Date$'
\end{verbatim}
\normalsize

The string contains the name of the source file, the version of the file (this reflects only the source
file version and not the overall release version number), the date and time of the file version, and
the person who checked the file in to the archive.  Upon compiling, these strings will be stored in the
executable file.  The user is then capable of searching the executable file, for example, for strings
beginning with {\ct \$Id:}.  This will result in a list of all source files compiled and their version.
Within the SVN archive any specific version of a source file can be extracted and differences between
versions can be determined.

Within the source code a series of subroutines were created, one per source file, which parses the {\ct \$Id:} in
each file and extracts from it the SVN revision number.  Each of these subroutines is called at the start
of an FDS run and the largest (and hence most recent) revision number is determined.  This number is written
along with the FDS version number to the output header information.  A user can now identify specifically
the source code used for a particular compilation of FDS when reporting an error.


\section{Software Changes}

\subsection{Creating a Change Request}

Change requests are submitted using the FDS Issue Tracker.  The Issue Tracker is an online service that is part of
GoogleCode.  The current location of the Issue Tracker is \href{http://code.google.com/p/fds-smv/issues/list}
{{\ct http://code.google.com/p/fds-smv/issues/list}}.  A change request is initiated by opening a new issue.
The issue report contains the baseline identification (version number, compile date, and SVN revision number),
operating system, and a description of the defect or enhancement request.  Input files or other supporting
documentation can be attached.

\begin{figure}[ht!]
\includegraphics[width=\textwidth]{FIGURES/defectreport.jpg}
\caption{\bf Screenshot of issue tracker reporting form}
\label{fig:issueform}
\end{figure}

If the issue is opened by a user, it will be given a status of 'New' until it is reviewed by a developer.  If the
issue is opened by a developer, the developer can immediate assign a status and an owner.

\subsection{Processing a Change Request}

A change request may be evaluated by any member of the development team.

If the request duplicates an existing request, then the issue status is changed to {\ct Duplicate} and the
requestor is sent the issue number of the existing request.

If the request is for an enhancement to FDS, then then the issue type is changed to {\ct Type-Enhancement} and the
component type is declared. The request is evaluated for suitability.  If it is a valid enhancement
request, then the status is changed to {\ct Accepted}.  If the request is not to
be addressed under the next baseline, then the status will be changed to {\ct OnHold}.  If the request is denied, then
the status is changed to {\ct WontFix}.  An accepted request will be assigned to a developer.

If the request is identifying a defect in FDS, then the issue type is changed to {\ct Type-Defect} and the
component type is declared.  The defect is evaluated for validity.  If insufficient information has
been provided in the request, then the status is changed to {\ct MoreInfo} and a description of the additional
information required is sent to the requestor.  If the defect is due to user error or is the intended function of FDS,
then the status is changed to {\ct Invalid}.  If the defect is valid, then the request reviewer will change the
status to {\ct Accepted} and assign the request to a developer.

Once a change request has be addressed by a developer and changes submitted to the repositoty, the request status
is changed to {\ct Fixed}.  Once either the requestor or another developer has verified that the changes address
the original request, then the status is changed to {\ct Verified}.


\subsection{Committing Changes}

Once a developer has addressed a change request, the modified files are committed to the SVN repository.  A description
of the changes will be added to the SVN change log.  This description first identifies the primary component being
changed (for example: FDS Source or FDS Documentation).  This component identification will be followed by a brief
summary of the changes made.  The issue identifier will be included as part of that brief description.

\subsection{Change Verification}

Once a change has been committed and the issue tracker updated to reflect that the issue has been {\ct Fixed},
the changes will be verified.  Verification can be done by either the requester of the change or by another
developer.  Once the changes have been verified to solve the problem reported in the issue, the issue status will
be changed to {\ct Verified}.  At this point the issue is closed.



\section{Issuing New Releases}

The decision to change versions of the software is made by consensus of the development team, usually after it is determined that enough
changes have been made to warrant a new release. There is no formal process to determine when a new release is to be issued. However, once the
decision is made, the new version is given a number, it is tested, and then posted to the official download site.

\subsection{Version Number}

New versions of FDS and Smokeview are identified using a specific numbering convention, for example, FDS 5.2.5.
The version number consists of three integers where the first number
is the {\em major} release, the second is the {\em minor} release, and the third is the {\em maintenance}
release.  Major releases occur every few years, and as the name implies dramatically changed functionality of the
model. Minor releases occur every few months, and may cause minor changes in functionality.
Release notes can help you decide whether the changes should effect the type of applications that you typically do.
Maintenance releases are either just bug fixes or the addition of minor enhancements (such as a new output quantity),
and should not affect code functionality.

\subsection{Testing a New Release}

Each proposed release will undergo software testing.  Three suites of test cases exists: a functional test suite,
a verification test suite, and a validation test suite.  Testing will depend upon the type of baseline release:
maintenance, minor, or major.

Each maintenance release will be tested with the functional test suite.  It will be verified that the proposed release version
successfully executes all of the functional test cases.

Each minor release will be tested with the functional test suite and the verification test suite.  It will be verified
that the proposed baseline successfully executes all of the functional and verification test cases.  The
verification manual will be updated with the results.

Each major release will be tested with all three test suites.    It will be verified
that the proposed baseline successfully executes all of the functional and verification test cases.  The verification
manual will be updated with the results.  The results of the validation cases will be evaluated to ensure that the
predictive performance of the proposed baseline as either remained equivalent or improved.  The validation manual will
be updated.

\subsection{Announcing a New Version}

Following successful completion of the required baseline testing, a baseline can be released.  Prior to release, the
version identification information within the FDS source code will be updated to reflect the new baseline.  FDS
documentation will be updated to reflect the new baseline.  The baseline will be compiled and new executable files
or installation packages will be placed on the FDS download site.  Prior baselines will be deprecated.  The
current FDS download site is \href{http://code.google.com/p/fds-smv/downloads/list}
{{\ct http://code.google.com/p/fds-smv/downloads/list}}.





\backmatter
\nopart

\bibliography{../Bibliography/FDS_refs,../Bibliography/FDS_general,../Bibliography/FDS_mathcomp}

\printindex

\end{document}
