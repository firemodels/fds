% !TEX root = FDTs_Validation_Guide.tex

\chapter{Surface Temperature}
\label{Surface_Temperature_Chapter}

\clearpage

\section{Time to Threshold Cable Temperature}

The thermally-induced electrical failure (THIEF) of a cable can be predicted via a simple
one-dimensional heat transfer calculation, under the assumption that the cable can be
treated as a homogeneous cylinder. The governing equation for the cable temperature,
$T(r,t)$, is given by
\be
\rho c \left( \frac{\partial T}{\partial t} \right) = \frac{1}{r} \frac{\partial}{\partial r} k r \left( \frac{\partial T}{\partial r} \right)
\label{eq:cable_temp}
\ee
with the boundary condition
\be
\dot q'' = k \left( \frac{\partial T}{\partial r} \right) (R,t)
\ee
A finite difference approximation to Eq.~\ref{eq:cable_temp} is given by
\be
\rho c \left[ \frac{T_i^{n+1} - T_i^n}{\delta t} \right] = \frac{2 k}{(r_{i+1} + r_i)} \frac{1}{2 \delta r} \left[ r_i \frac{T_{i+1}^n - T_i^n}{\delta r} - r_{i-1} \frac{T_{i}^n - T_{i-1}^n}{\delta r} + r_i \frac{T_{i+1}^{n+1} - T_i^{n+1}}{\delta r} - r_{i-1} \frac{T_{i}^{n+1} - T_{i-1}^{n+1}}{\delta r} \right]
\ee
where the time step $\delta t$ is given by
\be
\delta t = \frac{c \rho \delta r^2}{2 k}
\ee

\begin{figure}[!ht]
\begin{center}
\begin{tabular}{l}
\includegraphics[width=4.0in]{SCRIPT_FIGURES/Scatterplots/Cable_Temperature}
\end{tabular}
\end{center}
\caption[Summary of time to threshold cable temperature predictions]
{Summary of time to threshold cable temperature predictions.}
\label{Surface_Temperature_THIEF_Summary}
\end{figure}

\clearpage


\section{Unprotected Steel Temperature}

The temperature rise, $\Delta T_s$, of an unprotected steel member exposed to fire can be predicted using
\be
\Delta T_s = \frac{F}{V} \frac{1}{\rho_s c_s} \left[ h_c (T_f - T_s) + \sigma \epsilon (T_f^4 - T_s^4) \right] \Delta t
\label{eq:unprotected_steel}
\ee
where $F/V$ is the ratio of heated surface area to volume~(\si{m^{-1}}), $\rho_s$ is the density of steel~(\si{kg/m^3}), $c_s$ is the specific heat of steel~(\si{J/(kg.K)}), $h_c$ is the convective heat transfer coefficient~(\si{W/(m^2.K)}), $T_f$ is the exposing fire temperature~(\si{K}), $T_s$ is the steel temperature~(\si{K}), $\sigma$ is the Stefan-Boltzmann constant (\si{W/(m^2.K^4)}), $\epsilon$ is the flame emissivity~(-), and $\Delta t$ is the time step~(\si{s}). Note that the exposing fire temperature, $T_f$, can be replaced by the hot gas layer temperature, plume temperature, or other exposing temperature.


\section{Protected Steel Temperature}
\label{info:protected_steel_temperature}

The thermal capacity of the insulating material can be neglected if the following inequality is true
\be
c_s \frac{W}{D} > 2 c_i \rho_i h
\ee
where $c_s$ is the specific heat of steel~(\si{J/(kg.K)}), $W/D$ is the ratio of the weight of steel section per unit length to the heated perimeter~(\si{kg/m^2}), $c_i$ is the specific heat of the insulation material~(\si{J/(kg.K)}), $\rho_i$ is the density of the insulating material~(\si{kg/m^3}), and $h$ is the thickness of the insulation~(\si{m}).

If the thermal capacity of the insulation layer can be neglected, then the temperature rise, $\Delta T_s$, of a structural steel element exposed to fire can be predicted using
\be
\Delta T_s = k_i \left( \frac{T_f - T_s}{c_s h \frac{W}{D} + \frac{1}{2} c_i \rho_i h^2} \right) \Delta t
\label{eq:protected_steel_neglect_c}
\ee
where $k_i$ is the thermal conductivity of the insulation material~(\si{W/(m.K)}), $T_f$ is the exposing fire temperature~(\si{K}), $T_s$ is the steel temperature~(\si{K}), and $\Delta t$ is the time step~(\si{s}). Note that the exposing fire temperature, $T_f$, can be replaced by the hot gas layer temperature, plume temperature, or other exposing temperature.

If the thermal capacity of the insulation layer must be accounted for, then the temperature rise, $\Delta T_s$, of a structural steel element exposed to fire can be predicted using
\be
\Delta T_s = \frac{k_i}{h} \left( \frac{T_f - T_s}{c_s \frac{W}{D} + \frac{1}{2} c_i \rho_i h} \right) \Delta t
\label{eq:protected_steel_account_c}
\ee

\begin{figure}[!ht]
\begin{center}
\begin{tabular}{l}
\includegraphics[width=4.0in]{SCRIPT_FIGURES/Scatterplots/Steel_Temperature}
\end{tabular}
\end{center}
\caption[Summary of steel temperature predictions]
{Summary of unprotected and protected steel temperature predictions.}
\label{Surface_Temperature_Steel_Summary}
\end{figure}

