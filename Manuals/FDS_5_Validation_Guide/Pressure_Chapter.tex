\chapter{Compartment Pressure}

The pressure within the compartment was measured at a single point, near the floor.
In the simulations of the closed door tests, the compartment is assumed to leak via a small uniform flow spread
over the walls and ceiling.  The flow rate is calculated based on the assumption that the leakage rate is proportional
to the measured leakage area times the square root of compartment over-pressure.


\section{NIST/NRC Test Series}

Comparisons between measured and predicted pressures for the NIST/NRC Test Series are shown
in Figs.~\ref{NIST_NRC_Pressure_Closed} and \ref{NIST_NRC_Pressure_Open}.
For those tests in which the door to the compartment is
open, the over-pressures are only a few Pascals, whereas when the door is closed, the over-pressures are several hundred Pascals.
Note that in the closed door tests, there is often a dramatic drop in the predicted compartment pressure.
This is the result of the assumption in FDS that the heat release rate is decreased to zero in one second at the time
in the experiment when the fuel flow was stopped for safety reasons.  In reality, the fire did not extinguish
immediately because there was an excess of fuel in the pan following the flow stoppage.
For the purpose of model comparison, the peak over-pressures are differenced in the closed door tests,
and the peak (albeit small) under-pressures are compared in the open door tests.

\begin{figure}[p]
\begin{tabular*}{\textwidth}{l@{\extracolsep{\fill}}r}
\includegraphics[height=2.2in]{FIGURES/NIST_NRC/NIST_NRC_01_v5_Pressure} &
\includegraphics[height=2.2in]{FIGURES/NIST_NRC/NIST_NRC_07_v5_Pressure} \\
\includegraphics[height=2.2in]{FIGURES/NIST_NRC/NIST_NRC_02_v5_Pressure} &
\includegraphics[height=2.2in]{FIGURES/NIST_NRC/NIST_NRC_08_v5_Pressure} \\
\includegraphics[height=2.2in]{FIGURES/NIST_NRC/NIST_NRC_04_v5_Pressure} &
\includegraphics[height=2.2in]{FIGURES/NIST_NRC/NIST_NRC_10_v5_Pressure} \\
\includegraphics[height=2.2in]{FIGURES/NIST_NRC/NIST_NRC_13_v5_Pressure} &
\includegraphics[height=2.2in]{FIGURES/NIST_NRC/NIST_NRC_16_v5_Pressure}
\end{tabular*}
\label{NIST_NRC_Pressure_Closed}
\end{figure}

\begin{figure}[p]
\begin{tabular*}{\textwidth}{l@{\extracolsep{\fill}}r}
\includegraphics[height=2.2in]{FIGURES/NIST_NRC/NIST_NRC_17_v5_Pressure} &
   \\
\includegraphics[height=2.2in]{FIGURES/NIST_NRC/NIST_NRC_03_v5_Pressure} &
\includegraphics[height=2.2in]{FIGURES/NIST_NRC/NIST_NRC_09_v5_Pressure} \\
\includegraphics[height=2.2in]{FIGURES/NIST_NRC/NIST_NRC_05_v5_Pressure} &
\includegraphics[height=2.2in]{FIGURES/NIST_NRC/NIST_NRC_14_v5_Pressure} \\
\includegraphics[height=2.2in]{FIGURES/NIST_NRC/NIST_NRC_15_v5_Pressure} &
\includegraphics[height=2.2in]{FIGURES/NIST_NRC/NIST_NRC_18_v5_Pressure}
\end{tabular*}
\label{NIST_NRC_Pressure_Open}
\end{figure}

\begin{figure}[p]
\begin{center}
\begin{tabular}{c}
\includegraphics[width=4.0in]{FIGURES/ScatterPlots/Compartment_Pressure} \\
\vspace{0.25in} \\
\includegraphics[width=4.0in]{FIGURES/ScatterPlots/Open_Compartment_Pressure} \\
\vspace{0.25in}
\end{tabular}
\caption{Summary of Pressure Results.}
\end{center}
\end{figure}


