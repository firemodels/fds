
\chapter{Burning Rate}

This chapter contains a series of validation exercises where the aim is to {\em predict} the burning rate of the fuel. Most of the simulations included in the previous chapters involved
a {\em specified} burning or heat release rate. Here, the objective is to apply measured thermophysical properties of the material and predict its burning rate, either with a
specified heat flux or as a free burn.


\section{FAA Polymers}

This section presents predictions of the mass loss and/or burning rates of different types of polymers. Two types of experiments are considered. For the non-charring polymers, the
mass loss rate of non-burning samples was measured in a nitrogen atmosphere. For the charring polymers, the heat release rate of burning samples was measured in a standard
Cone Calorimeter~\cite{conecal}. When just the mass loss rate is predicted, FDS is run in ``solid phase only'' mode; that is, a 1-D heat conduction calculation is performed in a
single grid cell. The result is the predicted mass loss rate as a function of time. To simulate a cone calorimeter experiment, FDS simulates the burning of a 10~cm by 10~cm sample with
a specified heat flux to represent the effect of the cone heater. The cone itself is not included in the simulation. As the sample burns, FDS predicts the additional radiative and
convective heating of the sample as a result of the fire.

\newpage

\subsection{Non-Charring Polymers in the Gasification Apparatus}

A non-charring polymer is considered one of the easier solids to model because it typically involves only a single, first order reaction that converts solid plastic to fuel vapor.
No residue is formed and the plastic is completely pyrolyzed. Table~\ref{FAA_Properties}
lists nine parameters for each polymer studied. These values have been input directly into FDS, and the predicted mass loss rates are compared with measured values from the NIST
Gasification Apparatus, a device that pyrolyzes the solid in a nitrogen environment to prevent combustion of fuel gases. The results are shown in Fig.~\ref{FAA_Polymers}. The exposing
heat flux was 52~kW/m$^2$. A 1~cm layer of insulation was placed under the sample. Its properties are given in Ref.~\cite{Stoliarov:CF2009}.


\begin{table}[h!]
\caption[FAA non-charring polymer properties.]{Input parameters for FAA Polymers non-charring samples. Courtesy S.~Stoliarov, M.~McKinnon and J.~Li, University of Maryland.}
\begin{tabular}{|l|c|c|c|c|c|c|l|l|}
\hline
Property                    & Units         & HDPE                  & HIPS                  & PMMA       &PBT           & Unc. (\%)  & Method                &  Ref.                    \\ \hline \hline
Density                     & kg/m$^3$      & 860                   & 950                   & 1100      &1300             & 5     & Direct                &  \cite{Stoliarov:CF2009}  \\ \hline
Conductivity                & W/m/K         & 0.29                  & 0.22                  & 0.20       &.0.29           & 15    & Thermoflixer          &  \cite{Stoliarov:CF2009}  \\ \hline
Specific Heat               & kJ/kg/K       & 3.5                   & 2.0                   & 2.2    &2.2                & 15    & DSC                   &  \cite{Stoliarov:PDS2008}  \\ \hline
Emissivity                  &               & 0.92                  & 0.86                  & 0.85       &0.88           & 20    & Sphere                &  \cite{Hallman:PES1974}  \\ \hline
Absorption Coef.            & m$^{-1}$      & 1300                  & 2700                  & 2700    & 2560               & 50    & FTIR                  &  \cite{Tsilingiris:ECM2003}  \\ \hline
Pre-Exp.~Factor             & s$^{-1}$      & $4.8 \times 10^{22}$  & $1.2 \times 10^{16}$  & $8.5 \times 10^{12}$   &$2.49 \times 10^{14}$ & 50    & TGA                   &  \cite{Stoliarov:CF2009}  \\ \hline
Activation Energy           & kJ/kmol       & $3.49 \times 10^{5}$  & $2.47 \times 10^{5}$  & $1.88 \times 10^{5}$ &$2.12 \times 10^{5}$ & 3     & TGA                   &  \cite{Stoliarov:CF2009}  \\ \hline
Heat of Reaction            & kJ/kg         & 920                   & 1000                  & 870          &507        & 15    & DSC                   &  \cite{Stoliarov:PDS2008}  \\ \hline
\end{tabular}
\label{FAA_Properties}
\end{table}

\begin{tabbing}
Direct  \hspace{0.5in}     \= Direct measurement of mass and volume \\
Thermoflixer               \> Transient line source method \\
DSC                        \> Differential Scanning Calorimetry \\
Sphere                     \> Integrating sphere \\
FTIR                       \> Fourier Transform Infrared Spectroscopy \\
TGA                        \> Thermogravimetric Analysis
\end{tabbing}


\begin{figure}[h!]
\begin{tabular*}{\textwidth}{l@{\extracolsep{\fill}}r}
\includegraphics[height=2.2in]{FIGURES/FAA_Polymers/FAA_Polymers_HDPE} &
\includegraphics[height=2.2in]{FIGURES/FAA_Polymers/FAA_Polymers_HIPS} \\
\includegraphics[height=2.2in]{FIGURES/FAA_Polymers/FAA_Polymers_PMMA}&
\includegraphics[height=2.2in]{FIGURES/FAA_Polymers/FAA_Polymers_PBT} \\
\end{tabular*}
\caption[Results of FAA Polymers, non-charring, comparison]{Comparison of predicted and measured mass loss rates for three non-charring polymers exposed to a heat flux of 52~kW/m$^2$ in a
nitrogen environment.}
\label{FAA_Polymers}
\end{figure}

\clearpage

\subsection{Charring Polymers in the Cone Calorimeter}

The burning rate of a charring polymer is more difficult to predict than a non-charring one, in general, because there are more parameters that need to be measured.
Table~\ref{FAA_Properties_Charring}
lists 15 parameters for each polymer studied. These values have been input directly into FDS, and the predicted heat release rates are compared with measured values from the Cone
Calorimeter. The results for three different thicknesses are shown in Fig.~\ref{FAA_Polymers_Charring}.
A 1~cm layer of Kaowool insulation was placed under the sample. Its properties are given in Ref.~\cite{Stoliarov:CF2010}.


\begin{table}[h!]
\caption[FAA charring polymer properties.]{Input parameters for FAA Polymers charring samples. Courtesy S.~Stoliarov, University of Maryland.}
\begin{tabular}{|l|c|c|c|c|c|c|l|l|}
\hline
Property                    & Units         & Polycarbonate                     & PVC                 &PBTGF  & Method                &  Ref.                    \\ \hline \hline
Polymer Density             & kg/m$^3$      & 1180 $\pm$ 60                     &             &1520$\pm$ 80          & Direct                &  \cite{Stoliarov:CF2010}  \\ \hline
Polymer Conductivity        & W/m/K         & 0.22 $\pm$ 0.03                   &          &0.36 $\pm$ 0.06             & Literature            &  \cite{Stoliarov:CF2010}  \\ \hline
Polymer Specific Heat       & kJ/kg/K       & 1.9 $\pm$ 0.3                     &           &1.68 $\pm$  0.26          & DSC                   &  \cite{Stoliarov:PDS2008}  \\ \hline
Polymer Emissivity          &               & 0.90 $\pm$ 0.05                   &                     &0.87 $\pm$ 0.05  & Sphere                &  \cite{Hallman:PES1974}  \\ \hline
Polymer Absorption Coef.    & m$^{-1}$      & 1770 $\pm$ 590                    &       &2860 $\pm$ 150                & FTIR                  &  \cite{Tsilingiris:ECM2003}  \\ \hline
Char Density                & kg/m$^3$      & 248                               &        &482               & Cone Calorimeter      &  \cite{Stoliarov:CF2010}  \\ \hline
Char Conductivity           & W/m/K         & 0.37                              &        &0.07 $\pm$ 0.02               & Cone Calorimeter      &  \cite{Stoliarov:CF2010}  \\ \hline
Char Specific Heat          & kJ/kg/K       & 1.72 $\pm$ 0.17                   &          &0.85             & Pulsed Current        &  \cite{Stoliarov:CF2010,Matsumoto:1996}  \\ \hline
Char Emissivity             &               & 0.85 $\pm$ 0.05                   &                 &0.85      & Pulsed Current        &  \cite{Stoliarov:CF2010,Matsumoto:1996}  \\ \hline
Char Absorption Coef.       & m$^{-1}$      & Opaque                            &           & 31460           & Assumption            &  \cite{Stoliarov:CF2010}  \\ \hline
Pre-Exp.~Factor             & s$^{-1}$      & $(1.9 \pm 1.1) \times 10^{18}$    &              &$(2.49 \pm 0.63) \times 10^{14}$         & TGA                   &  \cite{Stoliarov:CF2010}  \\ \hline
Activation Energy           & kJ/kmol       & $(2.95 \pm 0.06) \times 10^{5}$   &            &$(2.12 \pm 0.53) \times 10^{5}$           & TGA                   &  \cite{Stoliarov:CF2010}  \\ \hline
Heat of Reaction            & kJ/kg         & 830 $\pm$ 140                     &                &355       & DSC                   &  \cite{Stoliarov:PDS2008}  \\ \hline
Heat of Combustion          & kJ/kg         & 25600 $\pm$ 130                   &         &19500              & MCC                   &  \cite{Stoliarov:CF2010}  \\ \hline
Combustion Efficiency       &               & 0.84 $\pm$ 0.03                   &              &1         & Cone Calorimeter      &  \cite{Stoliarov:CF2010}  \\ \hline
\end{tabular}
\label{FAA_Properties_Charring}
\end{table}

\begin{figure}[h!]
\begin{tabular*}{\textwidth}{l@{\extracolsep{\fill}}r}
\includegraphics[height=2.2in]{FIGURES/FAA_Polymers/FAA_Polymers_PC_6_75} &
\includegraphics[height=2.2in]{FIGURES/FAA_Polymers/FAA_Polymers_PVC_6_75}
\end{tabular*}
\caption[Results of FAA Polymers, charring, comparison]{Comparison of predicted and measured heat release rates for two charring polymers.}
\label{FAA_Polymers_Charring}
\end{figure}

\begin{figure}[p]
\begin{tabular*}{\textwidth}{l@{\extracolsep{\fill}}r}
\includegraphics[height=2.2in]{FIGURES/FAA_Polymers/FAA_Polymers_PC_6_92} &
\includegraphics[height=2.2in]{FIGURES/FAA_Polymers/FAA_Polymers_PVC_6_92}  \\
\includegraphics[height=2.2in]{FIGURES/FAA_Polymers/FAA_Polymers_PC_6_50} &
\includegraphics[height=2.2in]{FIGURES/FAA_Polymers/FAA_Polymers_PVC_6_50} \\
\includegraphics[height=2.2in]{FIGURES/FAA_Polymers/FAA_Polymers_PC_3_75} &
\includegraphics[height=2.2in]{FIGURES/FAA_Polymers/FAA_Polymers_PVC_3_75} \\
\includegraphics[height=2.2in]{FIGURES/FAA_Polymers/FAA_Polymers_PC_9_75} &
\includegraphics[height=2.2in]{FIGURES/FAA_Polymers/FAA_Polymers_PVC_9_75} \\
\end{tabular*}
\end{figure}

\begin{figure}[p]
\begin{tabular*}{\textwidth}{l@{\extracolsep{\fill}}r}
\includegraphics[height=2.2in]{FIGURES/FAA_Polymers/FAA_Polymers_PBTGF} 
\end{tabular*}
\end{figure}
