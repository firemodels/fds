


\chapter{Ceiling Jets and Device Activation}

FDS is a computational fluid dynamics (CFD) model and has no explicit ceiling jet model.
Rather, temperatures throughout the fire compartment are computed directly from the governing conservation equations.
Nevertheless, temperature measurements near the ceiling can be used to evaluate the model's ability to predict the flow of
hot gases across a relatively flat ceiling. Measurements for this category are available from the NIST/NRC and the FM/SNL series.

\section{NIST/NRC Test Series}

The thermocouple nearest the ceiling in Tree 7, located towards the back of the compartment,
has been chosen as a surrogate for the ceiling jet temperature.
Curiously, the difference between measured and predicted temperatures is noticeably greater for the open door tests.
Certainly, the open door changes the flow pattern of the exhaust gases.
However, the predicted HGL heights for the open door tests, shown in the previous section,
do not show a noticeable difference from their closed door counterparts.
The predicted HGL temperatures are only slightly less than those measured in the open door tests,
due in large part to the contribution of Tree 7 in the layer reduction calculation.

\begin{figure}[p]
\begin{tabular*}{\textwidth}{l@{\extracolsep{\fill}}r}
\includegraphics[width=2.6in]{FIGURES/NIST_NRC/NIST_NRC_01_v5_Ceiling_Jet} &
\includegraphics[width=2.6in]{FIGURES/NIST_NRC/NIST_NRC_07_v5_Ceiling_Jet} \\
\includegraphics[width=2.6in]{FIGURES/NIST_NRC/NIST_NRC_02_v5_Ceiling_Jet} &
\includegraphics[width=2.6in]{FIGURES/NIST_NRC/NIST_NRC_08_v5_Ceiling_Jet} \\
\includegraphics[width=2.6in]{FIGURES/NIST_NRC/NIST_NRC_04_v5_Ceiling_Jet} &
\includegraphics[width=2.6in]{FIGURES/NIST_NRC/NIST_NRC_10_v5_Ceiling_Jet} \\
\includegraphics[width=2.6in]{FIGURES/NIST_NRC/NIST_NRC_13_v5_Ceiling_Jet} &
\includegraphics[width=2.6in]{FIGURES/NIST_NRC/NIST_NRC_16_v5_Ceiling_Jet}
\end{tabular*}
\caption{Ceiling Jet Temperature for the NIST/NRC Series, closed door tests.}
\label{NIST_NRC_Jet_Closed}
\end{figure}

\begin{figure}[p]
\begin{tabular*}{\textwidth}{l@{\extracolsep{\fill}}r}
\includegraphics[width=2.6in]{FIGURES/NIST_NRC/NIST_NRC_17_v5_Ceiling_Jet} &
 \\
\includegraphics[width=2.6in]{FIGURES/NIST_NRC/NIST_NRC_03_v5_Ceiling_Jet} &
\includegraphics[width=2.6in]{FIGURES/NIST_NRC/NIST_NRC_09_v5_Ceiling_Jet} \\
\includegraphics[width=2.6in]{FIGURES/NIST_NRC/NIST_NRC_05_v5_Ceiling_Jet} &
\includegraphics[width=2.6in]{FIGURES/NIST_NRC/NIST_NRC_14_v5_Ceiling_Jet} \\
\includegraphics[width=2.6in]{FIGURES/NIST_NRC/NIST_NRC_15_v5_Ceiling_Jet} &
\includegraphics[width=2.6in]{FIGURES/NIST_NRC/NIST_NRC_18_v5_Ceiling_Jet}
\end{tabular*}
\caption{Ceiling Jet Temperature for the NIST/NRC Series, open door tests.}
\label{NIST_NRC_Jet_Open}
\end{figure}

\clearpage


\section{FM/SNL Test Series}

The near-ceiling thermocouples in Sectors 1 and 3 have been chosen as surrogates for the ceiling jet temperature.
The results are shown below.  The only noticeable discrepancy is in Test 21, and it is the same pattern that
was observed in the HGL temperature comparison for this test.





\section{UL/NFPRF Sprinkler, Vent, and Draft Curtain Experiments}
\label{UL_NFPRF:Results}

The ceiling jet is an important fire phenomenon because of the presence of automatic fire protection devices at the ceiling, like
sprinklers and smoke/heat vents. The results of the UL/NFPRF experiments provide useful data to assess the accuracy of FDS in predicting
the velocity and temperature near the ceiling, and consequently the resulting activation of sprinklers.
The UL/NFPRF test results (Series I) are summarized in Table~\ref{ULmatrix}, along with the predictions of FDS.

\begin{table}[h]
\begin{center}
\begin{tabular}{|c||c|c|c|c|c|c|c|c|}
\hline
\multicolumn{9}{|c|}{\bf Heptane Spray Burner Test Series I}  \\ \hline \hline
Test & Burner & Vent                    & \multicolumn{2}{|c|}{First Act. (s) } & \multicolumn{2}{|c|}{Total Acts.}  & Draft    & Heat Release Rate \\ \cline{4-7}
No.  & Pos.   & Operation               & Exp. & FDS                            & Exp.  & FDS                        & Curtains & MW @ s \\
\hline \hline
I-1   & B  & Closed                     & 65   & 53                             & 11   & 12     & Yes  & 4.4 @ 50  \\ \hline
I-2   & B  & Manual (0:40)              & 66   & 52                             & 12   & 8      & Yes  & 4.4 @ 50  \\ \hline
I-3   & B  & Manual (1:30)              & 64   & 53                             & 12   & 9      & Yes  & 4.4 @ 50  \\ \hline
I-4   & C  & Closed                     & 60   & 52                             & 10   & 11     & Yes  & 4.4 @ 50  \\ \hline
I-5   & C  & Manual (0:40)              & 72   & 52                             & 9    & 8      & Yes  & 4.4 @ 50  \\ \hline
I-6   & C  & Manual (1:30)              & 62   & 52                             & 8    & 8      & Yes  & 4.4 @ 50  \\ \hline
I-7   & C  & 74$^\circ$C link (DNO)     & 70   & 52                             & 10   & 11     & Yes  & 4.4 @ 50  \\ \hline
I-8   & B  & 74$^\circ$C link (9:26)    & 60   & 53                             & 11   & 12     & Yes  & 4.4 @ 50  \\ \hline
I-9   & D  & 74$^\circ$C link (DNO)     & 70   & 55                             & 12   & 15     & Yes  & 4.4 @ 50  \\ \hline
I-10  & D  & Manual (0:40)              & 72   & 54                             & 13   & 15     & Yes  & 4.4 @ 50  \\ \hline
I-11  & D  & 74$^\circ$C link (4:48)    & N/A  & N/A                            & N/A  & N/A    & Yes  & 4.4 @ 50  \\ \hline
I-12  & A  & Closed                     & 68   & 62                             & 14   & 14     & Yes  & 4.4 @ 50  \\ \hline
I-13  & A  & 74$^\circ$C link (1:04)    & 69   & 62                             & 5    & 13     & Yes  & 6.0 @ 60  \\ \hline
I-14  & A  & Manual (0:40)              & 74   & 136                            & 7    & 10     & Yes  & 5.8 @ 60  \\ \hline
I-15  & A  & Manual (1:30)              & 64   & 60                             & 5    & 9      & Yes  & 5.8 @ 60  \\ \hline
I-16  & A  & 74$^\circ$C link (1:46)    & 106  & 97                             & 4    & 7      & Yes  & 5.0 @ 110 \\ \hline
\hline
I-17  & B  & 100$^\circ$C link (DNO)    & 58   & 54                             & 4    & 4      & No   & 4.6 @ 50 \\ \hline
I-18  & C  & 100$^\circ$C link (DNO)    & 58   & 57                             & 4    & 4      & No   & 3.7 @ 50 \\ \hline
I-19  & A  & 100$^\circ$C link (10:00)  & 56   & 61                             & 10   & 5      & No   & 4.6 @ 50 \\ \hline
I-20  & A  & 74$^\circ$C link (1:20)    & 54   & 64                             & 4    & 4      & No   & 4.2 @ 50 \\ \hline
I-21  & C  & 74$^\circ$C link (7:00)    & 58   & 52                             & 10   & 4      & No   & 4.6 @ 50 \\ \hline
I-22  & D  & 100$^\circ$C link (DNO)    & 60   & 54                             & 6    & 9      & No   & 4.6 @ 50 \\ \hline
\end{tabular}
\end{center}
\caption[Results of the UL/NFPRF Experiments.]
{\bf Results of the UL/NFPRF Experiments. Note that DNO means
``Did Not Open''. Also note, the fires grew at a rate proportional
to the square of the time until a certain flow rate of fuel was achieved
at which time the flow rate was held steady. Thus, the ``Heat Release Rate''
was the size of the fire at the time when the fuel supply was leveled off.}
\label{ULmatrix}
\end{table}


\begin{figure}[ht]
\includegraphics[width=\textwidth]{FIGURES/UL_NFPRF_Scatter_Plot}
\caption{Measured vs. Predicted sprinkler activation times for the UL/NFPRF Test Series.}
\label{UL_NFPRF_Scatter_Plot}
\end{figure}
