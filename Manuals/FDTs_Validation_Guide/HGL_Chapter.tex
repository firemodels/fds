% !TEX root = FDTs_Validation_Guide.tex

\chapter{HGL Temperature and Depth}
\label{HGL_Chapter}

\clearpage

\section{Natural Ventilation}

For a compartment with natural ventilation, the correlation of McCaffrey, Quintiere, and Harkleroad (MQH)~\cite{SFPE:Walton} predicts that the HGL temperature rise, $\Delta T_g$, is given by
\be
\Delta T_g = 6.85 \left( \frac{\dot Q^2}{A_o \sqrt{H_o} h_k A_T} \right)^{1/3} \quad ^\circ{\rm C}
\label{eq:MQH}
\ee
where $\dot Q$ is the heat release rate (HRR) of the fire~(\si{kW}), $A_o$ is the area of the ventilation opening~(\si{m^2}), $H_o$ is the height of the ventilation opening~(\si{m}), and $A_T$ is the total area of the compartment enclosing surfaces~(\si{m^2}), excluding areas of vent openings. The heat transfer coefficient, $h_k$ (\si{kW/(m^2.K})), is given by
\be
h_k = \left\{ \begin{array}{cl}
   \sqrt{k \rho c/t}  & t \le t_p \\[0.1in]
   k/\delta           & t > t_p 
   \end{array} \right.
\label{eq:MQH_hk_lt}
\ee
where $k$ is the thermal conductivity of the interior lining~(\si{kW/(m. K)}), $\rho$ is its density~(\si{kg/m^3}), $c$ is its specific heat~(\si{kJ/(kg.K)}), and $\delta$ is its thickness~(\si{m}). The thermal penetration time, $t_p$ (\si{\second}), is given by
\be
t_p = \left( \frac{\rho c}{k} \right) \left( \frac{\delta}{2} \right)^2
\label{eq:MQH_tp}
\ee

\begin{figure}[!ht]
\begin{center}
\begin{tabular}{l}
\includegraphics[width=4.0in]{SCRIPT_FIGURES/Scatterplots/HGL_Temperature_MQH}
\end{tabular}
\end{center}
\caption[Summary of HGL temperature predictions for natural ventilation tests]
{Summary of HGL temperature predictions for natural ventilation tests using the MQH method.}
\label{HGL_Summary_Natural_Ventilation}
\end{figure}


\clearpage


\section{Forced Ventilation}

\subsection{FPA Method}

For a compartment with forced ventilation, the correlation of Foote, Pagni, and Alvares (FPA)~\cite{SFPE:Walton} predicts that the HGL temperature rise, $\Delta T_g$, is given by
\be
\Delta T_g = \left[ 0.63 \left( \frac{\dot Q}{\dot m_g c_p T_\infty} \right)^{0.72} \left( \frac{h_k A_T}{\dot m_g c_p} \right)^{-0.36} \right] T_\infty \quad ^\circ{\rm C}
\label{eq:FPA}
\ee
where $\dot Q$ is the HRR of the fire~(\si{kW}), $\dot m_g$ is the compartment ventilation mass flow rate~(\si{kg/s}), $c_p$ is the specific heat of air (\si{kJ/(kg.K)}), $T_\infty$ is the ambient air temperature~(\si{\celsius}), $h_k$ is the heat transfer coefficient~(\si{kW/(m^2.K)}), and $A_T$ is the total area of the compartment enclosing surfaces~(\si{m^2}), excluding areas of vent openings. The heat transfer coefficient, $h_k$ (\si{kW/(m^2.K)}), is given by
\be
h_k = \left\{ \begin{array}{cl}
   \sqrt{k \rho c/t}  & t \le t_p \\[0.1in]
   k/\delta           & t > t_p 
   \end{array} \right.
\label{eq:FPA_hk_lt}
\ee
where $k$ is the thermal conductivity of the interior lining~(\si{kW/(m.K)}), $\rho$ is its density~(\si{kg/m^3}), $c$ is its specific heat~(\si{kJ/(kg.K)}), and $\delta$ is its thickness~(\si{m}). The thermal penetration time, $t_p$ (\si{\second}), is given by
\be
t_p = \left( \frac{\rho c}{k} \right) \left( \frac{\delta}{2} \right)^2
\label{eq:FPA_tp}
\ee

\begin{figure}[!ht]
\begin{center}
\includegraphics[width=4.0in]{SCRIPT_FIGURES/Scatterplots/HGL_Temperature_FPA}
\end{center}
\caption[Summary of HGL temperature predictions for forced ventilation tests (FPA)]
{Summary of HGL temperature predictions for forced ventilation tests using the FPA method.}
\label{HGL_Summary_Forced_Ventilation_FPA}
\end{figure}

\clearpage


\subsection{Deal and Beyler Method}

For a compartment with forced ventilation, the correlation of Deal and Beyler (DB)~\cite{SFPE:Walton} predicts that the HGL temperature rise, $\Delta T_g$, is given by
\be
\Delta T_g = \left( \frac{\dot Q}{\dot m_g c_p + h_k A_T} \right) \quad ^\circ{\rm C}
\label{eq:DB}
\ee
where $\dot Q$ is the HRR of the fire~(\si{kW}), $\dot m_g$ is the compartment ventilation mass flow rate~(\si{kg/s}), $c_p$ is the specific heat of air (\si{kJ/(kg.K)}), $T_\infty$ is the ambient air temperature~(\si{\celsius}), $h_k$ is the heat transfer coefficient~(\si{kW/(m^2.K)}), and $A_T$ is the total area of compartment enclosing surfaces~(\si{m^2}), excluding areas of vent openings. The heat transfer coefficient, $h_k$ (\si{kW/(m^2.K)}), is given by 
\be
h_k = 0.4\ \mathrm{max} \left( \sqrt{\frac{k \rho c}{t}} , \frac{k}{\delta} \right)
\label{eq:DB_hk}
\ee
where $k$ is the thermal conductivity of the interior lining~(\si{kW/(m.K)}), $\rho$ is the density of the interior lining~(\si{kg/m^3}), $c$ is the specific heat of the interior lining~(\si{kJ/(kg.K)}), $t$ is the exposure time~(\si{\second}), and $\delta$ is the thickness of the interior lining~(\si{m}). This model is only valid for times up to 2000 seconds.

\begin{figure}[!ht]
\begin{center}
\includegraphics[width=4.0in]{SCRIPT_FIGURES/Scatterplots/HGL_Temperature_DB}
\end{center}
\caption[Summary of HGL temperature predictions for forced ventilation tests (DB)]
{Summary of HGL temperature predictions for forced ventilation tests using the DB method.}
\label{HGL_Summary_Forced_Ventilation_DB}
\end{figure}


\clearpage


\section{No Ventilation}

For a compartment with no ventilation (closed doors) and constant HRR, the correlation of Belyer~\cite{SFPE:Walton} predicts that the HGL temperature rise, $\Delta T_g$, is given by
\be
\Delta T_g = \frac{2 K_2}{K_1^2} (K_1 \sqrt{t} - 1 + e^{-K_1 \sqrt{t}}) \quad ^\circ{\rm C}
\label{eq:Beyler}
\ee
where $t$ is the exposure time~(\si{\second}). $K_1$ is given by
\be
K_1 = \frac{2(0.4\sqrt{k \rho c}) A_T}{m c_p}
\label{eq:Beyler_K1}
\ee
where $k$ is the thermal conductivity of the interior lining~(\si{kW/(m.K)}), $\rho$ is the density of the interior lining~(\si{kg/m^3}), $c$ is the specific heat of the interior lining~(\si{kJ/(kg.K)}), $A_T$ the total area of compartment enclosing surfaces~(\si{m^2}), $m$ is the mass of gas in the compartment~(\si{kg}), and $c_p$ is the specific heat of air~(\si{kJ/(kg.K)}). $K_2$ is given by
\be
K_2 = \frac{\dot Q}{m c_p}
\label{eq:Beyler_K2}
\ee
where $\dot Q$ is the HRR of the fire~(\si{kW}).

\begin{figure}[!ht]
\begin{center}
\begin{tabular}{l}
\includegraphics[width=4.0in]{SCRIPT_FIGURES/Scatterplots/HGL_Temperature_Beyler}
\end{tabular}
\end{center}
\caption[Summary of HGL temperature predictions for no ventilation tests]
{Summary of HGL temperature predictions for no ventilation tests using the Beyler method.}
\label{HGL_Summary_No_Ventilation}
\end{figure}

\section{HGL Depth}

\subsection{ASET}

For a compartment with no ventilation (closed doors) and constant HRR, the available safe egress time (ASET)~\cite{Walton:1}
correlation predicts that the HGL height, $z$~(\si{m}), is given by~\cite{SFPE:Milke}
\be
A_s \frac{dz}{dt} = \frac{dV_{ul}}{dt} = \dot V_{ul}
\label{eq:ASET_1}
\ee
where $A_s$ is the area of the boundary surfaces~(\si{m^2}), and $V_{ul}$ is the volume of the HGL~(\si{m^3}).
The change in volume of the upper layer, $\dot V_{ul}$~(\si{m^3/s}), is given by
\be
\dot V_{ul} = \dot V_{exp} + \dot V_{ent}
\label{eq:ASET_2}
\ee
The volumetric expansion rate, $\dot V_{exp}$~(\si{m^3/s}), is given by~\cite{SFPE:Mowrer}
\be
\dot V_{exp} = \frac{\dot Q_{net}}{\rho_g c_p T_g} \approx \frac{(1 - \chi_l) \dot Q_f}{353}
\label{eq:ASET_3}
\ee
where $\dot Q_{net}$ and $\dot Q_f$ are the net and actual HRRs~(\si{kW}), respectively, $\rho_g$, $c_p$ and $T_g$ are the density~(\si{kg/m^3}), specific heat~(\si{kJ/(kg.K)}), and temperature~(\si{K}) of air in the HGL, respectively, and $\chi_l$ is the heat loss fraction to the enclosure boundaries~(-).
The volumetric entrainment rate, $\dot V_{ent}$~(\si{m^3/s}), is given by~\cite{Zukoski:1981}
\be
\dot V_{ent} = k_v \dot Q^{1/3} z^{5/3} = \frac{0.21}{K_f} \left( \frac{g}{\rho_\infty T_\infty} \right)^{1/3} (K_f \dot Q)^{1/3} (z - z_f)^{5/3}
\label{eq:ASET_4}
\ee
where $k_v$ is the volumetric entrainment coefficient, $g$ is the acceleration due to gravity~(\si{m/s^2}), $\rho_\infty$ and $T_\infty$ are the density~(\si{kg/m^3}) and temperature~(\si{K}) of ambient air, respectively, $K_f$ is the location factor~(-), and $z_f$ is the fuel height~(\si{m}). The location factor has a value of 1, 2, or 4, which corresponds to a fire away from walls or corners, a fire adjacent to a wall, or a fire located in a corner, respectively.

The HGL height, $z$, in Eq.~\ref{eq:ASET_1} can be calculated iteratively using
\be
z|_{t+1} = z|_t - \frac{\dot V_{ul}}{L W} \Delta t
\label{eq:ASET_5}
\ee
where $L$ and $W$ are the length and width of the compartment~(\si{m}), respectively, and $\Delta t$ is the time step size~(\si{s}).

\begin{figure}[!ht]
\begin{center}
\begin{tabular}{l}
\includegraphics[width=4.0in]{SCRIPT_FIGURES/Scatterplots/HGL_Depth_ASET}
\end{tabular}
\end{center}
\caption[Summary of HGL depth predictions using ASET]
{Summary of HGL depth predictions using ASET.}
\label{HGL_Depth_ASET}
\end{figure}


\clearpage


\subsection{Yamana and Tanaka}

For a compartment with no ventilation (closed doors) and constant HRR, the correlation of Yamana and Tanaka~\cite{Tanaka:1} predicts that the HGL height, $z$, is given by
\be
z = \left( \frac{2 k \dot Q^{1/3} t}{3 A_c} + \frac{1}{h_c^{2/3}} \right)^{-3/2}
\label{eq:Yamana_Tanaka}
\ee
where $\dot Q$ is the HRR~(\si{kW}), $t$ is the time after ignition~(\si{s}), $A_c$ is the compartment floor area~(\si{m^2}), and $h_c$ is the compartment height~(\si{m}). The constant $k$ is given by
\be
k = \frac{0.076}{(353/T_g)}
\ee
where $T_g$ is the HGL temperature~(\si{K}).

\begin{figure}[!ht]
\begin{center}
\begin{tabular}{l}
\includegraphics[width=4.0in]{SCRIPT_FIGURES/Scatterplots/HGL_Depth_Yamana_Tanaka}
\end{tabular}
\end{center}
\caption[Summary of HGL depth predictions using Yamana and Tanaka]
{Summary of HGL depth predictions using Yamana and Tanaka method.}
\label{HGL_Depth_YT}
\end{figure}

