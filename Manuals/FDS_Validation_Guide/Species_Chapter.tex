% !TEX root = FDS_Validation_Guide.tex

\chapter{Gas Species and Smoke}

For most applications, FDS uses a single step, mixing-controlled combustion model. The products of combustion are ``lumped'' together and tracked as a single gas mixture. These products include CO$_2$, H$_2$O, CO, and soot. However, in some cases, the combustion is incomplete due to a lack of oxygen. In others, a multiple-step reaction scheme is used to predict the production of CO.

\section{Major Combustion Products, O$_2$ and CO$_2$}

For any hydrocarbon fuel, the major combustion products are oxygen and carbon dioxide. Accurate predictions of these gases requires knowledge of the chemical composition of the fuel and an accurate transport algorithm for the combustion products.

\clearpage

\subsection{FAA Cargo Compartments}

Carbon dioxide and carbon monoxide were measured near the ceiling in the forward, middle, and aft sections of the compartment. Note that all but the middle compartment concentrations were measured in Tests~2~and~3.


\begin{figure}[h!]
\begin{tabular*}{\textwidth}{l@{\extracolsep{\fill}}r}
\includegraphics[height=2.2in]{SCRIPT_FIGURES/FAA_Cargo_Compartments/FAA_Cargo_Compartments_Test_1_CO2} &
\includegraphics[height=2.2in]{SCRIPT_FIGURES/FAA_Cargo_Compartments/FAA_Cargo_Compartments_Test_1_CO} \\
\includegraphics[height=2.2in]{SCRIPT_FIGURES/FAA_Cargo_Compartments/FAA_Cargo_Compartments_Test_2_CO2} &
\includegraphics[height=2.2in]{SCRIPT_FIGURES/FAA_Cargo_Compartments/FAA_Cargo_Compartments_Test_2_CO} \\
\includegraphics[height=2.2in]{SCRIPT_FIGURES/FAA_Cargo_Compartments/FAA_Cargo_Compartments_Test_3_CO2} &
\includegraphics[height=2.2in]{SCRIPT_FIGURES/FAA_Cargo_Compartments/FAA_Cargo_Compartments_Test_3_CO}
\end{tabular*}
\label{FAA_Cargo_CO2_CO}
\end{figure}

\clearpage

\subsection{NIST/NRC Experiments}

The following pages present comparisons of oxygen and carbon dioxide concentration predictions and measurements for the
NIST/NRC series. There were two oxygen measurements, one in the upper layer, one in the lower.  There was only one carbon
dioxide measurement in the upper layer.

\begin{figure}[h!]
\begin{tabular*}{\textwidth}{l@{\extracolsep{\fill}}r}
\includegraphics[height=2.2in]{SCRIPT_FIGURES/NIST_NRC/NIST_NRC_17_Oxygen} &
\includegraphics[height=2.2in]{SCRIPT_FIGURES/NIST_NRC/NIST_NRC_17_CO2} \\
\includegraphics[height=2.2in]{SCRIPT_FIGURES/NIST_NRC/NIST_NRC_03_Oxygen} &
\includegraphics[height=2.2in]{SCRIPT_FIGURES/NIST_NRC/NIST_NRC_03_CO2} \\
\includegraphics[height=2.2in]{SCRIPT_FIGURES/NIST_NRC/NIST_NRC_09_Oxygen} &
\includegraphics[height=2.2in]{SCRIPT_FIGURES/NIST_NRC/NIST_NRC_09_CO2}
\end{tabular*}
\label{NIST_NRC_Gas_Open_1}
\end{figure}

\newpage

\begin{figure}[p]
\begin{tabular*}{\textwidth}{l@{\extracolsep{\fill}}r}
\includegraphics[height=2.2in]{SCRIPT_FIGURES/NIST_NRC/NIST_NRC_05_Oxygen} &
\includegraphics[height=2.2in]{SCRIPT_FIGURES/NIST_NRC/NIST_NRC_05_CO2} \\
\includegraphics[height=2.2in]{SCRIPT_FIGURES/NIST_NRC/NIST_NRC_14_Oxygen} &
\includegraphics[height=2.2in]{SCRIPT_FIGURES/NIST_NRC/NIST_NRC_14_CO2} \\
\includegraphics[height=2.2in]{SCRIPT_FIGURES/NIST_NRC/NIST_NRC_15_Oxygen} &
\includegraphics[height=2.2in]{SCRIPT_FIGURES/NIST_NRC/NIST_NRC_15_CO2} \\
\includegraphics[height=2.2in]{SCRIPT_FIGURES/NIST_NRC/NIST_NRC_18_Oxygen} &
\includegraphics[height=2.2in]{SCRIPT_FIGURES/NIST_NRC/NIST_NRC_18_CO2}
\end{tabular*}
\label{NIST_NRC_Gas_Open_2}
\end{figure}

\begin{figure}[p]
\begin{tabular*}{\textwidth}{l@{\extracolsep{\fill}}r}
\includegraphics[height=2.2in]{SCRIPT_FIGURES/NIST_NRC/NIST_NRC_01_Oxygen} &
\includegraphics[height=2.2in]{SCRIPT_FIGURES/NIST_NRC/NIST_NRC_01_CO2} \\
\includegraphics[height=2.2in]{SCRIPT_FIGURES/NIST_NRC/NIST_NRC_07_Oxygen} &
\includegraphics[height=2.2in]{SCRIPT_FIGURES/NIST_NRC/NIST_NRC_07_CO2} \\
\includegraphics[height=2.2in]{SCRIPT_FIGURES/NIST_NRC/NIST_NRC_02_Oxygen} &
\includegraphics[height=2.2in]{SCRIPT_FIGURES/NIST_NRC/NIST_NRC_02_CO2} \\
\includegraphics[height=2.2in]{SCRIPT_FIGURES/NIST_NRC/NIST_NRC_08_Oxygen} &
\includegraphics[height=2.2in]{SCRIPT_FIGURES/NIST_NRC/NIST_NRC_08_CO2}
\end{tabular*}
\label{NIST_NRC_Gas_Closed_1}
\end{figure}

\begin{figure}[p]
\begin{tabular*}{\textwidth}{l@{\extracolsep{\fill}}r}
\includegraphics[height=2.2in]{SCRIPT_FIGURES/NIST_NRC/NIST_NRC_04_Oxygen} &
\includegraphics[height=2.2in]{SCRIPT_FIGURES/NIST_NRC/NIST_NRC_04_CO2} \\
\includegraphics[height=2.2in]{SCRIPT_FIGURES/NIST_NRC/NIST_NRC_10_Oxygen} &
\includegraphics[height=2.2in]{SCRIPT_FIGURES/NIST_NRC/NIST_NRC_10_CO2} \\
\includegraphics[height=2.2in]{SCRIPT_FIGURES/NIST_NRC/NIST_NRC_13_Oxygen} &
\includegraphics[height=2.2in]{SCRIPT_FIGURES/NIST_NRC/NIST_NRC_13_CO2} \\
\includegraphics[height=2.2in]{SCRIPT_FIGURES/NIST_NRC/NIST_NRC_16_Oxygen} &
\includegraphics[height=2.2in]{SCRIPT_FIGURES/NIST_NRC/NIST_NRC_16_CO2}
\end{tabular*}
\label{NIST_NRC_Gas_Closed_2}
\end{figure}

\clearpage

\subsection{WTC Experiments}

The following pages present comparisons of oxygen and carbon dioxide concentration predictions and measurements for the
WTC experiments. There was only one measurement of each made near the ceiling of the compartment roughly 2~m from the fire.


\begin{figure}[h!]
\begin{tabular*}{\textwidth}{l@{\extracolsep{\fill}}r}
\includegraphics[height=2.2in]{SCRIPT_FIGURES/WTC/WTC_01_Oxygen} &
\includegraphics[height=2.2in]{SCRIPT_FIGURES/WTC/WTC_01_CO2} \\
\includegraphics[height=2.2in]{SCRIPT_FIGURES/WTC/WTC_02_Oxygen} &
\includegraphics[height=2.2in]{SCRIPT_FIGURES/WTC/WTC_02_CO2} \\
\includegraphics[height=2.2in]{SCRIPT_FIGURES/WTC/WTC_03_Oxygen} &
\includegraphics[height=2.2in]{SCRIPT_FIGURES/WTC/WTC_03_CO2}
\end{tabular*}
\label{NIST_WTC_Oxygen_CO2_1}
\end{figure}

\newpage

\begin{figure}[p]
\begin{tabular*}{\textwidth}{l@{\extracolsep{\fill}}r}
\includegraphics[height=2.2in]{SCRIPT_FIGURES/WTC/WTC_04_Oxygen} &
\includegraphics[height=2.2in]{SCRIPT_FIGURES/WTC/WTC_04_CO2} \\
\includegraphics[height=2.2in]{SCRIPT_FIGURES/WTC/WTC_05_Oxygen} &
\includegraphics[height=2.2in]{SCRIPT_FIGURES/WTC/WTC_05_CO2} \\
\includegraphics[height=2.2in]{SCRIPT_FIGURES/WTC/WTC_06_Oxygen} &
\includegraphics[height=2.2in]{SCRIPT_FIGURES/WTC/WTC_06_CO2}
\end{tabular*}
\label{NIST_WTC_Oxygen_CO2_2}
\end{figure}

\clearpage


\subsection{Summary of Major Combustion Products Predictions}


\begin{figure}[h!]
\begin{center}
\begin{tabular}{c}
\includegraphics[width=3.5in]{SCRIPT_FIGURES/ScatterPlots/FDS_Carbon_Dioxide_Concentration} \\
\includegraphics[width=3.5in]{SCRIPT_FIGURES/ScatterPlots/FDS_Oxygen_Concentration}\\
\end{tabular}
\end{center}
\caption[Summary of major gas species predictions]
{Summary of major gas species predictions.}
\end{figure}

\clearpage


\section{Smoke and Aerosols}



\subsection{NIST/NRC Experiments}

For the simulations of the NIST/NRC tests, the smoke yield is specified as one of the test parameters.
The figures on the following pages contain comparisons of measured and predicted smoke concentration at one measuring station in the upper layer.

\begin{figure}[p]
\begin{tabular*}{\textwidth}{l@{\extracolsep{\fill}}r}
\includegraphics[height=2.2in]{SCRIPT_FIGURES/NIST_NRC/NIST_NRC_01_Smoke} &
\includegraphics[height=2.2in]{SCRIPT_FIGURES/NIST_NRC/NIST_NRC_07_Smoke} \\
\includegraphics[height=2.2in]{SCRIPT_FIGURES/NIST_NRC/NIST_NRC_02_Smoke} &
\includegraphics[height=2.2in]{SCRIPT_FIGURES/NIST_NRC/NIST_NRC_08_Smoke} \\
\includegraphics[height=2.2in]{SCRIPT_FIGURES/NIST_NRC/NIST_NRC_04_Smoke} &
\includegraphics[height=2.2in]{SCRIPT_FIGURES/NIST_NRC/NIST_NRC_10_Smoke} \\
\includegraphics[height=2.2in]{SCRIPT_FIGURES/NIST_NRC/NIST_NRC_13_Smoke} &
\includegraphics[height=2.2in]{SCRIPT_FIGURES/NIST_NRC/NIST_NRC_16_Smoke}
\end{tabular*}
\end{figure}

\begin{figure}[p]
\begin{tabular*}{\textwidth}{l@{\extracolsep{\fill}}r}
\includegraphics[height=2.2in]{SCRIPT_FIGURES/NIST_NRC/NIST_NRC_17_Smoke} &
 \\
\includegraphics[height=2.2in]{SCRIPT_FIGURES/NIST_NRC/NIST_NRC_03_Smoke} &
\includegraphics[height=2.2in]{SCRIPT_FIGURES/NIST_NRC/NIST_NRC_09_Smoke} \\
\includegraphics[height=2.2in]{SCRIPT_FIGURES/NIST_NRC/NIST_NRC_05_Smoke} &
\includegraphics[height=2.2in]{SCRIPT_FIGURES/NIST_NRC/NIST_NRC_14_Smoke} \\
\includegraphics[height=2.2in]{SCRIPT_FIGURES/NIST_NRC/NIST_NRC_15_Smoke} &
\includegraphics[height=2.2in]{SCRIPT_FIGURES/NIST_NRC/NIST_NRC_18_Smoke}
\end{tabular*}
\end{figure}


\begin{figure}[p]
\begin{center}
\begin{tabular}{c}
\includegraphics[width=4.0in]{SCRIPT_FIGURES/ScatterPlots/FDS_Smoke_Concentration}
\end{tabular}
\end{center}
\caption[Summary of smoke concentration predictions]{Summary of smoke concentration predictions.}
\end{figure}

\clearpage

\subsection{FAA Cargo Compartments}

Beam obscuration measurements were made at different locations within the compartment (see Fig.~\ref{FAA_Cargo_probe_locations}). The data is presented below in terms of percent transmission per meter, $100(I/I_0)^{1/L}$, where $I$ is the light intensity and $L$ is the beam pathlength in units of meters.

\begin{figure}[h!]
\begin{tabular*}{\textwidth}{l@{\extracolsep{\fill}}r}
\includegraphics[height=2.2in]{SCRIPT_FIGURES/FAA_Cargo_Compartments/FAA_Cargo_Compartments_Test_1_Ceiling_Transmission} &
\includegraphics[height=2.2in]{SCRIPT_FIGURES/FAA_Cargo_Compartments/FAA_Cargo_Compartments_Test_1_Cargo_Transmission} \\
\includegraphics[height=2.2in]{SCRIPT_FIGURES/FAA_Cargo_Compartments/FAA_Cargo_Compartments_Test_2_Ceiling_Transmission} &
\includegraphics[height=2.2in]{SCRIPT_FIGURES/FAA_Cargo_Compartments/FAA_Cargo_Compartments_Test_2_Cargo_Transmission} \\
\includegraphics[height=2.2in]{SCRIPT_FIGURES/FAA_Cargo_Compartments/FAA_Cargo_Compartments_Test_3_Ceiling_Transmission} &
\includegraphics[height=2.2in]{SCRIPT_FIGURES/FAA_Cargo_Compartments/FAA_Cargo_Compartments_Test_3_Cargo_Transmission}
\end{tabular*}
\end{figure}

\newpage

\begin{figure}[p]
\begin{center}
\begin{tabular}{c}
\includegraphics[width=4.0in]{SCRIPT_FIGURES/ScatterPlots/FDS_Smoke_Obscuration}
\end{tabular}
\end{center}
\caption[Summary of smoke obscuration predictions]{Summary of smoke obscuration predictions.}
\end{figure}


\clearpage

\subsection{Sippola Aerosol Deposition Experiments}

FDS treats smoke particulate and aerosols in a similar way to other gaseous combustion products, basically a tracer gas whose production rate is a fixed fraction of the fuel consumption rate. However, there is an option in the model to allow smoke or aerosols to deposit on solid surfaces, thus reducing its concentration in the product stream. A total of 16 aerosol deposition experiments were conducted in a straight steel duct with smooth walls for 5 different particle diameters (1~$\mu$m, 3~$\mu$m, 5~$\mu$m, 9~$\mu$m, and 16~$\mu$m) and 3 different air velocities (2.2~m/s, 5.3~m/s, and 9.0~m/s). In the simulations, the aerosol is tracked explicitly, and the aerosol deposition routines are enabled (refer to the Aerosol Deposition section in the FDS User Guide~\cite{FDS_Users_Guide} and FDS Technical Reference Guide~\cite{FDS_Math_Guide} for more details). A summary of the 16~experiments is shown in Table~\ref{Sippola_Aerosol_Deposition_Summary}.

\begin{table}[h!]
\caption{Summary of Sippola aerosol deposition experiments selected for model validation.}
\begin{center}
\begin{tabular}{|c|c|c|c|}
\hline
Test      &  Air Speed        &  Particle Diameter          &  Particle Density             \\
No.       &  (m/s)            &  ($\mu$m)                   &  (kg/m$^3$)                   \\ \hline \hline
1         &  2.2              &  1.0                        &  1350                         \\ \hline
2         &  2.2              &  2.8                        &  1170                         \\ \hline
3         &  2.1              &  5.2                        &  1210                         \\ \hline
4         &  2.2              &  9.1                        &  1030                         \\ \hline
5         &  2.2              &  16                         &  950                          \\ \hline
6         &  5.3              &  1.0                        &  1350                         \\ \hline
7         &  5.2              &  1.0                        &  1350                         \\ \hline
8         &  5.2              &  3.1                        &  1170                         \\ \hline
9         &  5.4              &  5.2                        &  1210                         \\ \hline
10        &  5.3              &  9.8                        &  1030                         \\ \hline
11        &  5.3              &  16                         &  950                          \\ \hline
12        &  9.0              &  1.0                        &  1350                         \\ \hline
13        &  9.0              &  3.1                        &  1170                         \\ \hline
14        &  8.8              &  5.4                        &  1210                         \\ \hline
15        &  9.2              &  8.7                        &  1030                         \\ \hline
16        &  9.1              &  15                         &  950                          \\ \hline
\end{tabular}
\end{center}
\label{Sippola_Aerosol_Deposition_Summary}
\end{table}

\noindent The particle deposition velocity, $u_{\rm dep}$, is calculated by
\be
   u_{\rm dep} = \frac{J_1 + J_2 + J_3 + J_4}{4 \; C_{\rm avg}}
\ee
where $J_1$ through $J_4$ are the deposition fluxes (\si{kg/(m^2.s)}) for duct panels 1 through 4 given by
\be
   J = \frac{m_{\rm d}}{A_{\rm d} \; \Delta t}
\ee
where $m_{\rm d}$ is the mass of particles on the duct panel (kg), $A_{\rm d}$ is the area of the duct panel (m$^2$),
and $\Delta t$ is the duration over which the aerosol deposits onto the panel (s). $C_{\rm avg}$ is the
average aerosol concentration in the duct test section (kg/m$^3$) and is given by
\be
   C_{\rm avg} = \frac{C_{\rm upstream} + C_{\rm downstream}}{2}
\ee
Figure~\ref{Sippola_Aerosol_Deposition_Velocity} compares the measured and predicted aerosol deposition velocities,
and Figure~\ref{Summary_Aerosol_Deposition_Velocity} shows a summary of the results.

\begin{figure}[ht]
\begin{center}
\begin{tabular}{c}
\includegraphics[height=2.2in]{SCRIPT_FIGURES/Sippola_Aerosol_Deposition/Sippola_Aerosol_Ceiling_Deposition} \\
\includegraphics[height=2.2in]{SCRIPT_FIGURES/Sippola_Aerosol_Deposition/Sippola_Aerosol_Wall_Deposition} \\
\includegraphics[height=2.2in]{SCRIPT_FIGURES/Sippola_Aerosol_Deposition/Sippola_Aerosol_Floor_Deposition}
\end{tabular}
\end{center}
\caption[Predicted and measured aerosol deposition velocities, Sippola experiments]
{Predicted and measured aerosol deposition velocities, Sippola experiments.}
\label{Sippola_Aerosol_Deposition_Velocity}
\end{figure}

\begin{figure}[ht]
\begin{center}
\begin{tabular}{c}
\includegraphics[width=4.0in]{SCRIPT_FIGURES/ScatterPlots/FDS_Aerosol_Deposition_Velocity} \\
\vspace{0.25in} \\
\end{tabular}
\end{center}
\caption[Summary of aerosol deposition velocity predictions]
{Summary of aerosol deposition velocity predictions.}
\label{Summary_Aerosol_Deposition_Velocity}
\end{figure}


\clearpage

\section{Products of Incomplete Combustion}

Predicting the concentration of products of incomplete combustion is challenging because it requires information about the chemical composition of the fuel and the multiple reactions that convert fuel to products. FDS contains a fairly general framework by which users can specify the reaction mechanism, and the examples in the following subsections highlight some of the more commonly used schemes.

\clearpage

\subsection{Smyth Slot Burner Experiment}

The two-step, CO production model in FDS was used to simulate a methane/air slot burner diffusion flame. Figures~\ref{Smyth_Slot_Burner_7} through \ref{Smyth_Slot_Burner_11} show predicted and measured temperatures at three elevations above the burner. Figures~\ref{Smyth_Slot_Burner_7} through \ref{Smyth_Slot_Burner_11} show predicted and measured concentrations of CH$_4$, O$_2$, CO,  and CO$_2$ at these same elevations. The reported uncertainty in the species concentration measurements ranges from 10~\% to 20~\%.

\begin{figure}[h!]
\begin{tabular*}{\textwidth}{l@{\extracolsep{\fill}}r}
\includegraphics[height=2.2in]{SCRIPT_FIGURES/Smyth_Slot_Burner/Smyth_Slot_Burner_7mm_Fuel} &
\includegraphics[height=2.2in]{SCRIPT_FIGURES/Smyth_Slot_Burner/Smyth_Slot_Burner_7mm_Oxygen} \\
\includegraphics[height=2.2in]{SCRIPT_FIGURES/Smyth_Slot_Burner/Smyth_Slot_Burner_9mm_Fuel} &
\includegraphics[height=2.2in]{SCRIPT_FIGURES/Smyth_Slot_Burner/Smyth_Slot_Burner_9mm_Oxygen} \\
\includegraphics[height=2.2in]{SCRIPT_FIGURES/Smyth_Slot_Burner/Smyth_Slot_Burner_11mm_Fuel} &
\includegraphics[height=2.2in]{SCRIPT_FIGURES/Smyth_Slot_Burner/Smyth_Slot_Burner_11mm_Oxygen} 
\end{tabular*}
\caption[Fuel and oxygen species predictions at 7~mm, 9~mm, and 11~mm above burner, Smyth experiment]
{Predicted and measured fuel and oxygen species at 7~mm, 9~mm, and 11~mm above a methane-air slot burner.}
\label{Smyth_Slot_Burner_fuel_ox}
\end{figure}

\begin{figure}[p]
\begin{tabular*}{\textwidth}{l@{\extracolsep{\fill}}r}
\includegraphics[height=2.2in]{SCRIPT_FIGURES/Smyth_Slot_Burner/Smyth_Slot_Burner_7mm_Carbon_Dioxide} &
\includegraphics[height=2.2in]{SCRIPT_FIGURES/Smyth_Slot_Burner/Smyth_Slot_Burner_7mm_Carbon_Monoxide} \\
\includegraphics[height=2.2in]{SCRIPT_FIGURES/Smyth_Slot_Burner/Smyth_Slot_Burner_9mm_Carbon_Dioxide} &
\includegraphics[height=2.2in]{SCRIPT_FIGURES/Smyth_Slot_Burner/Smyth_Slot_Burner_9mm_Carbon_Monoxide} \\
\includegraphics[height=2.2in]{SCRIPT_FIGURES/Smyth_Slot_Burner/Smyth_Slot_Burner_11mm_Carbon_Dioxide} &
\includegraphics[height=2.2in]{SCRIPT_FIGURES/Smyth_Slot_Burner/Smyth_Slot_Burner_11mm_Carbon_Monoxide} 
\end{tabular*}
\caption[Carbon monoxide and carbons dioxide species predictions at 7~mm, 9~mm, and 11~mm above burner, Smyth experiment]
{Predicted and measured carbon dioxide and carbon monoxide species at 7~mm, 9~mm, and 11~mm above a methane-air slot burner.}
\label{Smyth_Slot_Burner_co_co2}
\end{figure}

\begin{figure}[p]
\begin{tabular*}{\textwidth}{l@{\extracolsep{\fill}}r}
\includegraphics[height=2.2in]{SCRIPT_FIGURES/Smyth_Slot_Burner/Smyth_Slot_Burner_7mm_Temperature} \\
\includegraphics[height=2.2in]{SCRIPT_FIGURES/Smyth_Slot_Burner/Smyth_Slot_Burner_9mm_Temperature} \\
\includegraphics[height=2.2in]{SCRIPT_FIGURES/Smyth_Slot_Burner/Smyth_Slot_Burner_11mm_Temperature}
\end{tabular*}
\caption[Temperature predictions at 7~mm, 9~mm, and 11~mm above burner, Smyth experiment]
{Predicted and measured temperature at 7~mm, 9~mm, and 11~mm above a methane-air slot burner.}
\label{Smyth_Slot_Burner_temp}
\end{figure}

\clearpage

\subsection{Beyler Hood Experiments}

Fig.~\ref{Beyler_Species} shows species predictions made by the two-step model compared with measured data for a range of fire sizes and burner positions.  The model outputs are the time-averaged species concentration at the hood exhaust vent whereas the experiment is the time-averaged species concentration downstream in the exhaust duct.

\begin{figure}[h!]
\begin{tabular*}{\textwidth}{l@{\extracolsep{\fill}}r}
\includegraphics[height=2.2in]{SCRIPT_FIGURES/Beyler_Hood/Beyler_Hood_O2} &
\includegraphics[height=2.2in]{SCRIPT_FIGURES/Beyler_Hood/Beyler_Hood_CO2} \\
\includegraphics[height=2.2in]{SCRIPT_FIGURES/Beyler_Hood/Beyler_Hood_H2O} &
\includegraphics[height=2.2in]{SCRIPT_FIGURES/Beyler_Hood/Beyler_Hood_CO} \\
\includegraphics[height=2.2in]{SCRIPT_FIGURES/Beyler_Hood/Beyler_Hood_Soot} &
\includegraphics[height=2.2in]{SCRIPT_FIGURES/Beyler_Hood/Beyler_Hood_UHC}
\end{tabular*}
\caption[Summary of gas species predictions, Beyler hood experiments]
{Comparison of measured and predicted species concentrations in the Beyler hood experiments}
\label{Beyler_Species}
\end{figure}

\clearpage

\subsection{NIST Reduced Scale Enclosure (RSE) Test Series, 1994}

The RSE natural gas experiments were selected to assess the CO production capability rather than soot production. Nine fire sizes were simulated: 50~kW, 75~kW, 100~kW, 150~kW, 200~kW, 300~kW, 400~kW, 500~kW, and 600~kW. The experiments were modeled using properties of the natural gas supplied to the test facility. The model geometry included the compartment interior along with a 0.6~m deep region outside the door. Figures~\ref{NIST_RSE_1994_spec1} and \ref{NIST_RSE_1994_spec1} show the measured and predicted CO, CO$_2$ and O$_2$, and H$_2$O concentrations. Figure~\ref{NIST_RSE_1994_temp} shows the measured and predicted thermocouple temperatures in the front and rear of the compartment. The measured values are from the test series performed by Bryner, Johnsson, and Pitts~\cite{Bryner:1}.

\begin{figure}[h!]
\begin{tabular*}{\textwidth}{l@{\extracolsep{\fill}}r}
\includegraphics[height=2.2in]{SCRIPT_FIGURES/NIST_RSE_1994/NIST_RSE_1994_CO_Front} &
\includegraphics[height=2.2in]{SCRIPT_FIGURES/NIST_RSE_1994/NIST_RSE_1994_CO_Rear} \\
\includegraphics[height=2.2in]{SCRIPT_FIGURES/NIST_RSE_1994/NIST_RSE_1994_CO2_Front} &
\includegraphics[height=2.2in]{SCRIPT_FIGURES/NIST_RSE_1994/NIST_RSE_1994_CO2_Rear}
\end{tabular*}
\caption[Summary of species concentrations in NIST Reduced Scale Enclosure experiments]{Summary of species concentrations in NIST Reduced Scale Enclosure experiments.}
\label{NIST_RSE_1994_spec1}
\end{figure}

\begin{figure}[h!]
\begin{tabular*}{\textwidth}{l@{\extracolsep{\fill}}r}
\includegraphics[height=2.2in]{SCRIPT_FIGURES/NIST_RSE_1994/NIST_RSE_1994_O2_Front} &
\includegraphics[height=2.2in]{SCRIPT_FIGURES/NIST_RSE_1994/NIST_RSE_1994_O2_Rear} \\
\includegraphics[height=2.2in]{SCRIPT_FIGURES/NIST_RSE_1994/NIST_RSE_1994_H2O_Front} &
\includegraphics[height=2.2in]{SCRIPT_FIGURES/NIST_RSE_1994/NIST_RSE_1994_H2O_Rear}
\end{tabular*}
\caption[Summary of species concentrations in NIST Reduced Scale Enclosure experiments]{Summary of species concentrations in NIST Reduced Scale Enclosure experiments.}
\label{NIST_RSE_1994_spec2}
\end{figure}

\begin{figure}[h!]
\begin{tabular*}{\textwidth}{l@{\extracolsep{\fill}}r}
\includegraphics[height=2.2in]{SCRIPT_FIGURES/NIST_RSE_1994/NIST_RSE_1994_TC_Front} &
\includegraphics[height=2.2in]{SCRIPT_FIGURES/NIST_RSE_1994/NIST_RSE_1994_TC_Rear} \\
\end{tabular*}
\caption[Summary of thermocouple values in NIST Reduced Scale Enclosure experiments]{Summary of thermocouple values in NIST Reduced Scale Enclosure experiments.}
\label{NIST_RSE_1994_temp}
\end{figure}


\clearpage

\section{Helium Release in a Reduced Scale Garage Geometry}

FDS simulations were performed to predict the helium release and dispersion in a reduced scale garage geometry.
The figures below show the comparison between FDS predictions and experimental data for eighteen tests with different release locations,
leak configurations, and release durations as noted in the experimental description.  

\begin{figure}[h!]
\begin{tabular*}{\textwidth}{l@{\extracolsep{\fill}}r}
\includegraphics[height=2.2in]{SCRIPT_FIGURES/NIST_He_2009/NIST_He_3600_LC_SLV} &
\includegraphics[height=2.2in]{SCRIPT_FIGURES/NIST_He_2009/NIST_He_3600_LC_SSV} \\
\includegraphics[height=2.2in]{SCRIPT_FIGURES/NIST_He_2009/NIST_He_3600_LC_ULV} &
\includegraphics[height=2.2in]{SCRIPT_FIGURES/NIST_He_2009/NIST_He_3600_LR_SLV} \\
\includegraphics[height=2.2in]{SCRIPT_FIGURES/NIST_He_2009/NIST_He_3600_LR_SSV} &
\includegraphics[height=2.2in]{SCRIPT_FIGURES/NIST_He_2009/NIST_He_3600_LR_ULV}
\end{tabular*}
\caption[Species concentrations in NIST helium release experiments]{Species concentrations in NIST helium release experiments.}
\label{NIST_Hydrogen_Species_1}
\end{figure}

\begin{figure}[h!]
\begin{tabular*}{\textwidth}{l@{\extracolsep{\fill}}r}
\includegraphics[height=2.2in]{SCRIPT_FIGURES/NIST_He_2009/NIST_He_3600_UC_SLV} &
\includegraphics[height=2.2in]{SCRIPT_FIGURES/NIST_He_2009/NIST_He_3600_UC_SSV} \\
\includegraphics[height=2.2in]{SCRIPT_FIGURES/NIST_He_2009/NIST_He_3600_UC_ULV} &
\includegraphics[height=2.2in]{SCRIPT_FIGURES/NIST_He_2009/NIST_He_14400_LC_SLV} \\
\includegraphics[height=2.2in]{SCRIPT_FIGURES/NIST_He_2009/NIST_He_14400_LC_SSV} &
\includegraphics[height=2.2in]{SCRIPT_FIGURES/NIST_He_2009/NIST_He_14400_LC_ULV}
\end{tabular*}
\caption[Species concentrations in NIST helium release experiments]{Species concentrations in NIST helium release experiments.}
\label{NIST_Hydrogen_Species_2}
\end{figure}

\begin{figure}[h!]
\begin{tabular*}{\textwidth}{l@{\extracolsep{\fill}}r}
\includegraphics[height=2.2in]{SCRIPT_FIGURES/NIST_He_2009/NIST_He_14400_LR_SLV} &
\includegraphics[height=2.2in]{SCRIPT_FIGURES/NIST_He_2009/NIST_He_14400_LR_SSV} \\
\includegraphics[height=2.2in]{SCRIPT_FIGURES/NIST_He_2009/NIST_He_14400_LR_ULV} &
\includegraphics[height=2.2in]{SCRIPT_FIGURES/NIST_He_2009/NIST_He_14400_UC_SLV} \\
\includegraphics[height=2.2in]{SCRIPT_FIGURES/NIST_He_2009/NIST_He_14400_UC_SSV} &
\includegraphics[height=2.2in]{SCRIPT_FIGURES/NIST_He_2009/NIST_He_14400_UC_ULV} 
\end{tabular*}
\caption[Species concentrations in NIST helium release experiments]{Species concentrations in NIST helium release experiments.}
\label{NIST_Hydrogen_Species_3}
\end{figure}

\clearpage

