\chapter{Combustion (Chemically Reacting Flows)}

\label{combustionsection}

FDS can model multiple, simultaneous, reversible chemical reactions.  In FDS only the species that have mixed have the potential to react. Once mixed, FDS uses one of two approaches to compute the reaction rate for a chemical reaction.  The first approach is an infinite fast chemical reaction rate. That is the reactants (e.g. fuel and air) react infinitely fast at the rate that the underlying gas transport mixes them together.  For the second model, individual gas species react according to specified Arrhenius reaction parameters. The Arrhenius rate can still be limited by the amount of mixing within a computational cell. This latter model is most often used in a direct numerical simulation (DNS) where the diffusion of fuel and oxygen can be modeled directly. However, most often for large eddy simulations (LES), where the grid is not fine enough to resolve the diffusion of fuel and oxygen,
the mixing-controlled combustion model is assumed. A single FDS simulation can contain a mix of reaction rate types.

\section{Eddy Dissipation Concept}
In the eddy dissipation concept model (EDC) each computational cell can be thought of as a batch reactor. The species contained within this reactor exist in one of two states: unmixed or mixed. The proportion of the cell in the mixed state is governed by $\zeta$, the unmixed fraction. A cell can range from completely unmixed ($\zeta=1$) to completely mixed ($\zeta=0$). In a combustion sense, $\zeta=1$  represents a diffusion flame and $\zeta=0$ represents a premixed flame. $\zeta(t;\tau_{mix})$, a function of time (t) and parameterized by the mixing time ($\tau_{mix}$), is governed by the following ordinary differential equation:

\begin{equation}\label{eq:zeta}
\frac{\mbox{d}\zeta}{\mbox{d}t}=-\frac{\zeta}{\tau_{mix}}
\end{equation}     

Only those species in the mixed state can react, therefore the proportion of the cell (reactor) composition in the mixed state must be known as function of time. 

$\tau_{mix}$ discussion

The following equations define the cell composition:
\begin{eqnarray}\label{eq:mixunmix}
\rho V = U(t) + M(t) \\
U(t) = \zeta(t)\,\rho V \\
M(t) = (1-\zeta(t))\,\rho V
\end{eqnarray} 
where $\rho V$ is the total mass in the reactor, $U(t)$ is unmixed portion of the mass, and $M(t)$ is the mixed portion of the mass. $\rho V$ is not a function of time as the total mass in a cell remains constant during a single FDS time interval. From the portion of mass that is mixed, $M(t)$, the local mixed composition ($\phi_{\alpha}(t)$) is found by:

\begin{equation}\label{eq:phi}
\phi_{\alpha}(t)=\frac{m_{\alpha}(t)}{M(t)}
\end{equation}
where $m_\alpha(t)$ is the mixed mass of each of the $\alpha$ species in the simulation. The mixed mass of each species as a function of time is governed by:

\begin{equation}\label{eq:mixmass}
\frac{\mbox{d}m_{\alpha}}{\mbox{d}t}=\frac{M(t) \, \nu_{\alpha} \, W_{\alpha}}{W_F}\displaystyle \sum_{i}^{\text{N\,Reactions}}r_{i}+Y_{\alpha,0}\frac{\mbox{d}M}{\mbox{d}t} 
\end{equation}
where $Y_{\alpha,0}$ is the mass fraction of each $\alpha$ species in the unmixed portion of the cell. The mass fractions of each species in the unmixed portion remain constant over an integration time step therefore the proportion of species which mix remains constant. $r_{i}$, the reaction rate, is discussed in detail in Section \ref{Reaction_Rate_Model}. The complete composition of the computational cell is found by a combination of the unmixed and mixed portions:

\begin{equation}\label{eq:final_comp}
Y_{\alpha}(t)=\zeta(t)Y_{\alpha,0}+(1-\zeta(t))\phi_{\alpha}(t)
\end{equation}
Finally the chemical mass production rate, $\dot{m}^{\prime\prime\prime}_{\alpha}$, is found:
\begin{equation}\label{mass_prod_rate}
\dot{m}^{\prime\prime\prime}_{\alpha}=\rho \,\frac{\mbox{d}Y_{\alpha}}{\mbox{d}t}
\end{equation}

\section{Reaction Rate} 

\label{Reaction_Rate_Model}

The rate expression (on a fuel basis) for both infinitely fast chemistry and Arrhenius rate chemistry follow the same general form:
\begin{equation}\label{eq:Arrheniusrateeqn}
\frac{\mbox{d}[X_\F]}{\mbox{d}t} = -A \; \prod \left([X_{\alpha}]^{a_{\alpha}} \right) T^n \; e^{-E/RT} 
\end{equation}
Note that the rate expression can be dependent on temperature and that the rate can be dependent upon species that do not directly participate in the reaction. Since FDS solves for mass fractions of lumped species rather than mole fractions of individual species, the only "species" that can be consumed or created are the lumped species.  However, any of the primitive species components of those lumped species could participate as a collision species.  Therefore, rather than $\frac{\mbox{d}[X_\F]}{\mbox{d}t}$,  FDS must solve for $\frac{\mbox{d}\phi_\F}{\mbox{d}t}$.  Additionally since it is quicker to determine $\phi_i$ rather than $X_i$, the relation $[X_i]=\frac{\phi_i \rho}{W_i}$ is used.  Combined, these result in a modified rate equation:

\begin{equation}\label{eq:Arrheniusratemode}
\frac{\mbox{d}\phi_\F}{\mbox{d}t} = -A_{mod} \rho^{\sum (a_{\alpha}) -1} \; \prod \left(Y_{\alpha}^{a_{\alpha}} \right) T^n \; e^{-E/RT} \;\;\; A_{mod} = A \prod \left(W_{\alpha}^{-a_{\alpha}} \right)  
\end{equation}

\subsection{Infinite Fast Reaction}
For fast chemistry reactions, $E$ and $n$ in \ref{eq:Arrheniusratemode} are set to $0$ which removes the temperature dependence from the rate expression. The species exponents, $a_{\alpha}$, are also set to zero to remove concentration dependence. Finally, the pre-exponential, $A$, is set to $1 \times 10^{16}$. In this formulation, $\frac{\mbox{d}\phi_\F}{\mbox{d}t}$ becomes sufficiently large such that all of the fuel in a computational cell is consumed in a single time step, effectively making the rate infinite. However, the rate is limited by the minimum reactant (fuel or oxidizer) in the mixed composition of the cell which makes this formulation a mixing-controlled reaction.   

\subsection{Arrhenius Reaction}
There are conditions under which the local temperature and species concentrations support the computation of the reaction rate.  One set of conditions is where the reacting regions of the flow can be considered a well-stirred reactor.  The other is during a DNS calculation.  In the first case rather than a reacting surface, there is a reacting volume.  In the second case, the fine grid resolution enables the direct modeling of the diffusion of chemical species (fuel,
oxygen, and combustion products).  This allows the flame to be resolved in a DNS calculation.  Under conditions where the local gas
temperatures can be used to determine the reaction kinetics, Arrhenius rates can be used.


\section{Reaction with Local Extinction}

\label{extinction}

The physical limitation of the mixing-controlled reaction model described in the previous section is that it assumes that fuel and oxygen burn instantaneously when mixed. For large-scale, well-ventilated
fires, this is a good assumption. However, if a fire is in an
under-ventilated compartment, or if a suppression agent like water
mist or CO$_2$ is introduced, or if the shear layer between fuel and oxidizing streams
has a sufficiently large local strain rate,
fuel and oxygen may mix but may not burn.
The physical mechanisms underlying these phenomena are complex, and
even simplified models still rely on an accurate prediction
of the flame temperature and local strain rate.
Subgrid-scale modeling of gas phase suppression and
extinction is still an area of active research in the combustion
community.

Simple empirical rules can be used to predict local
extinction based on the species and temperature present in the flame sheet.  The FDS extinction model consists of two parts. The first part, checks to see if the local temperature is above an auto-ignition temperature for the fuel.  If the temperature is too low, then combustion will not occur.  Note that this temperature is by default set to absolute zero so that typical users do not need to specify an ignition source.  The second part uses the concept of a limiting flame temperature.  If the local combustion cannot raise the local temperature above the limiting flame temperature, then combustion will not occur.  This is done in the following steps:

\begin{enumerate}
\item Search over all reactant species to find the limiting species and express that species in terms of the reaction's fuel:

\be \Delta Z_\F = \min_{i \; = \; reactant} \left(\frac{Z_i W_\F}{W_i \nu_i} \right) \ee
\item Remove that amount of fuel from the local gas, the resulting gas is the "air" for the reaction.
\item Search over all the non-fuel reactant species in the "air" to determine the limiting reactant.  This defines how much air is required to burn the fuel:

\be \Delta Z_{Air} = \min_{i \; = \; non-fuel \; reactant} \left(\frac{\Delta Z_\F W_i \nu_i}{Z_i W_\F} \right) \ee

\item Compute the enthalpy for the fuel and the air at both the current temperature and the limiting flame temperature.
\item Combustion is allowed if 

\be \Delta Z_{Air} h_{air}(T)+\Delta Z_F \left( h_\F(T)+\Delta H_\F \right) > \Delta Z_{Air} h_{air}(T_{LFT})+ \Delta Z_\F h_\F(T_{LFT}) \ee

\end{enumerate}

\section{Simplified Chemistry}

\label{simplechemistry}

\subsection{Single-step Reaction}

Most ordinary combustibles can be represented by the simple single-step reaction:

\be  \mathrm{C_xH_yO_zN_a} +  \nu_\OTWO \, \mathrm{O_2}  \rightarrow  \nu_\COTWO \, \mathrm{CO_2} + \nu_\HTWOO \, \mathrm{H_2O} + \nu_\CO \, \mathrm{CO} +
     \nu_\So \, \mathrm{S}  + \nu_\NTWO \, \mathrm{N_2}  \label{stoich}
\ee
Note that the nitrogen in the fuel molecule is assumed to form $\mathrm{N_2}$ only. Soot is assumed to be a mixture of carbon and hydrogen with the hydrogen atomic
fraction given by $X_\Hy$. The stoichiometric coefficient, $\nu_\So$, represents the amount of fuel that is converted to soot. It is related to the
{\em soot yield}, $y_\So$, via the relation:
\be
   \nu_\So = \frac{W_\F}{W_\So} \; y_\So  \quad ; \quad  W_\So = X_\Hy W_\Hy + (1 - X_\Hy) W_\C  \label{soot_yield}
\ee
Likewise, the stoichiometric coefficient of CO, $\nu_\CO$, is related to the CO {\em yield}, $y_\CO$, via:
\be
   \nu_\CO = \frac{W_\F}{W_\CO} \; y_\CO  \label{CO_yield}
\ee
The yields of soot and CO are based on ``well-ventilated'' or ``post-flame'' measurements. The increased production of CO and soot in an under-ventilated
compartment will be addressed in the following sections.

This reaction can also be represented as:

\be  \mathrm{Fuel} +  \nu_\mathrm{Air} \, \mathrm{Air}  \rightarrow  \mathrm{Products}\label{stoichair}
\ee

Air is a lumped species consisting of a mixture of nitrogen, oxygen, water vapor, and carbon dioxide (carbon dioxide is included to attenuate radiation over large distances and water vapor is included for attenuation and to support simulations with sprinklers).  Products is a lumped species consisting of all the products listed in \ref{stoich} plus the nitrogen, water vapor, and carbon dioxide from the Air that reacted with the fuel.  If we allow for the presence of a diluent gas in the fuel stream, then the Fuel becomes a lumped species consisting of fuel and diluent.  We name these species $Z_0$ for Air, $Z_1$ for Fuel, and $Z_2$ for Products.  When using simple chemistry, $Z_1$ and $Z_2$ are tracked explicitly and $Z_0$ is tracked implicitly as the background species.  The mass fractions of the component gases in these species are given as:

\noindent
$Z_0$: Air
\begin{eqnarray}
Y_{\NTWO}(Z_0) & = & Y_\NTWO^\infty\\*[.1in]
Y_{\OTWO}(Z_0) & = & Y_\OTWO^\infty\\*[.1in]
Y_{\COTWO}(Z_0) & = & Y_\COTWO^\infty\\*[.1in]
Y_{\HTWOO}(Z_0) & = & Y_\HTWOO^\infty
\end{eqnarray}
$Z_1$: Fuel
\begin{eqnarray}
Y_{\F}(Z_1) & = & Y_\F
\end{eqnarray}
$Z_2$: Products
\begin{eqnarray}
Y_{\NTWO}(Z_2) & = & \frac{\nu_{Air} W_{Air}  Y_\NTWO^\infty +\nu_{\NTWO} W_{\NTWO}}{W_F + \nu_{Air} W_{Air}}\\*[.1in]
Y_{\COTWO}(Z_2) & = & \frac{\nu_{Air} W_{Air}  Y_\COTWO^\infty +\nu_{\COTWO} W_{\COTWO}}{W_F + \nu_{Air} W_{Air}}\\*[.1in]
Y_{\HTWOO}(Z_2) & = & \frac{\nu_{Air} W_{Air}  Y_\HTWOO^\infty +\nu_{\HTWOO} W_{\HTWOO}}{W_F + \nu_{Air} W_{Air}}\\*[.1in]
Y_{\CO}(Z_2) & = & \frac{\nu_{\CO} W_{\CO}}{W_F + \nu_{Air} W_{Air}}\\*[.1in]
Y_{\So}(Z_2) & = & \frac{\nu_{\So} W_{\So}}{W_F + \nu_{Air} W_{Air}}
\end{eqnarray}
Species yields of combinations of $Z_0$, $Z_1$, and $Z_2$ are given as:
\be
Y_\alpha(Z_0,Z_1,Z_2)=Y_\alpha(Z_0) \, (1 - Z_1 - Z_2) + Y_\alpha(Z_1) \, Z_1 + Y_\alpha(Z_2) \, Z_2
\ee
The stoichiometric coefficients in the $Z_2$ species yields are:

\parbox{2.5in}{
\begin{eqnarray*}  \nu_\NTWO  &=& \frac{\hbox{a}}{2}\\*[.1in]
                  \nu_\OTWO  &=& \nu_\COTWO + \frac{\nu_\CO+\nu_\HTWOO-\hbox{z}}{2}\\*[.1in]
                  \nu_\COTWO &=& \hbox{x} - \nu_\CO - (1-X_\Hy) \nu_\So  \\*[.1in]
\end{eqnarray*} }
\hfill \parbox{3.5in}{\begin{eqnarray}
                  \nu_\HTWOO &=& \frac{\hbox{y}}{2}- X_\Hy \nu_\So\\*[.1in]
                  \nu_\CO    &=& \frac{W_\F}{W_\CO} \; y_\CO \\*[.1in]
                  \nu_\So    &=& \frac{W_\F}{W_\So} \; y_\So
\end{eqnarray} }
Remember that x is the number of carbon atoms, y is the number of hydrogen atoms, z is the number of oxygen atoms, and a is the number of nitrogen atoms in the fuel molecule.
It is important to note that the definitions of $Z_0$, $Z_1$, and $Z_2$ do not imply anything regarding the rate of combustion, only that the combustion occurs in a single step.

\subsection{Two-Step Reaction}

The previous section describes the ``complete'' reaction as the conversion of fuel to
products such that the production rate of each product species is proportional to the fuel consumption rate.
This means that for each fuel molecule, fixed amounts of CO$_2$, H$_2$O, CO, and soot are formed and these products
persist in the plume indefinitely with no further reaction. This is not an unreasonable assumption if
the purpose of the fire simulation is to assess the impact of the fire on the larger space.
However, in under-ventilated fires, soot and CO are produced at higher rates,
and exist within the fuel-rich flame envelope at higher concentrations,
than would otherwise be predicted with a single set of fixed yields that are based on post-flame measurements. To account for the
production of CO and its eventual oxidation at the flame envelope or within a hot upper layer,
an additional reaction is now needed:
\begin{eqnarray}
\mathrm{C_xH_yO_zN_a} +  \nu_\OTWO' \mathrm{O_2}  &\rightarrow&  \nu_\HTWOO \mathrm{H_2O} + (\nu_\CO'+ \nu_\CO) \, \mathrm{CO} +
     \nu_\So \, \mathrm{S}  + \nu_\NTWO \, \mathrm{N_2}   \\*[.1in]
\nu_\CO' \; \Big[ \mathrm{CO} + \ha \mathrm{O_2}  &\rightarrow&  \mathrm{CO_2}  \Big]
\label{3reac} \end{eqnarray}
The brackets around the second reaction are there merely to emphasize that the sum of the two reactions equal Eq.~(\ref{stoich}).
There are two stoichiometric coefficients for CO -- the first, $\nu_\CO'=\hbox{x}-(1-X_\Hy) \nu_\So-\nu_\CO$,
represents CO that is produced in the first
step of the reaction that can potentially be converted to CO$_2$ assuming the conditions are favorable. $\nu_\CO'$ is equivalent to $\nu_\COTWO$ in
Eq.~(\ref{stoich}). The second coefficient, $\nu_\CO$,
is the so-called ``well-ventilated,'' or ``post-flame,'' value that was introduced in the previous section. The proposed model of CO production
still does not contain the necessary kinetic mechanism to predict the ``post-flame'' concentration of CO without the prescription of the
measured value of the post-flame CO yield. Rather, the proposed model includes the production of large amounts of CO in the first step of a two-step
reaction, followed by a partial conversion to CO$_2$ if there is a sufficient amount of oxygen present.

To describe the composition of the gas species, an additional lumped species is required bringing the total to four lumped species.  We can rewrite the above two step reaction as:

\begin{eqnarray}
Fuel + \nu_{Air,1} Air &\rightarrow&  Incomplete \; Products\\*[.1in]
Incomplete \; Products + \nu_{Air,2} Air &\rightarrow&  Complete \; Products
\label{4lumped}
\end{eqnarray}

It can be seen from this that the four lumped species are: $Z_0$ for Air, $Z_1$ for Fuel, $Z_2$ for the products of incomplete combustion, and $Z_3$ for the products of complete combustion.  The species yields are:

$Z_0$: Air

\begin{eqnarray}
Y_{\NTWO}(Z_0) & = & Y_\NTWO^\infty\\*[.1in]
Y_{\OTWO}(Z_0) & = & Y_\OTWO^\infty\\*[.1in]
Y_{\COTWO}(Z_0) & = & Y_\COTWO^\infty\\*[.1in]
Y_{\HTWOO}(Z_0) & = & Y_\HTWOO^\infty
\end{eqnarray}

$Z_1$: Fuel

\begin{eqnarray}
Y_{\F}(Z_1) & = & Y_\F
\end{eqnarray}

$Z_2$: Products of Incomplete Combustion

\begin{eqnarray}
Y_{\NTWO}(Z_2) & = & \frac{\nu_{Air,1} W_{Air} Y_\NTWO^\infty +\nu_{\NTWO} W_{\NTWO}}{W_F + \nu_{Air} W_{Air}}\\*[.1in]
Y_{\COTWO}(Z_2) & = & \frac{\nu_{Air} W_{Air} Y_\COTWO^\infty}{W_F + \nu_{Air} W_{Air}} \\*[.1in]
Y_{\HTWOO}(Z_2) & = & \frac{\nu_{Air} W_{Air}  Y_\HTWOO^\infty +\nu_{\HTWOO} W_{\HTWOO}}{W_F + \nu_{Air} W_{Air}}\\*[.1in]
Y_{\CO}(Z_2) & = & \frac{\nu_{\CO'} W_{\CO}}{W_F + \nu_{Air} W_{Air}}\\*[.1in]
Y_{\So}(Z_2) & = & \frac{\nu_{\So} W_{\So}}{W_F + \nu_{Air} W_{Air}}
\end{eqnarray}

$Z_3$: Products of Complete Combustion

\begin{eqnarray}
Y_{\NTWO}(Z_2) & = & \frac{\nu_{Air,2} W_{Air}  Y_\NTWO^\infty +\nu_{\NTWO} W_{\NTWO}}{W_F + \nu_{Air} W_{Air}}\\*[.1in]
Y_{\COTWO}(Z_2) & = & \frac{\nu_{Air} W_{Air}  Y_\COTWO^\infty +\nu_{\COTWO} W_{\COTWO}}{W_F + \nu_{Air} W_{Air}}\\*[.1in]
Y_{\HTWOO}(Z_2) & = & \frac{\nu_{Air} W_{Air}  Y_\HTWOO^\infty +\nu_{\HTWOO} W_{\HTWOO}}{W_F + \nu_{Air} W_{Air}}\\*[.1in]
Y_{\CO}(Z_2) & = & \frac{\nu_\CO W_{\CO}}{W_F + \nu_{Air} W_{Air}}\\*[.1in]
Y_{\So}(Z_2) & = & \frac{\nu_{\So} W_{\So}}{W_F + \nu_{Air} W_{Air}}
\end{eqnarray}


The stoichiometric coefficients are defined:

\parbox{2.5in}{
\begin{eqnarray*} \nu_\NTWO  &=& \frac{\hbox{a}}{2}\\*[.1in]
                  \nu_\OTWO' &=& \frac{\nu_\CO'+\nu_\HTWOO-z}{2}\\*[.1in]
                  \nu_\OTWO  &=& \nu_\COTWO + \frac{\nu_\CO+\nu_\HTWOO-z}{2}\\*[.1in]
                  \nu_\COTWO &=& \hbox{x} - (1-X_\Hy) \nu_\So \\*[.1in]
                  \nu_\M     &=& \hbox{b} \end{eqnarray*} }
\hfill \parbox{3.5in}{\begin{eqnarray}
                  \nu_\HTWOO &=& \frac{\hbox{y}}{2}- X_\Hy \nu_\So\\*[.1in]
                  \nu_\CO'   &=& x - \nu_\CO - (1-X_\Hy) \nu_\So \\*[.1in]
                  \nu_\CO    &=& \frac{W_\F}{W_\CO} \; y_\CO \\*[.1in]
                  \nu_\So    &=& \frac{W_\F}{W_\So} \; y_\So
\end{eqnarray} }
Although these formulae appear complicated, most are determined directly from the composition of the fuel molecule. The only information
expected of the modeler are the fuel composition, the soot and CO yields, and the atomic fraction of hydrogen in the soot.

\newpage


\section{Time Integration of Chemical Reactions}

\subsection{Single Step, Mixing-Controlled Reaction Rates}
For single-step chemical reactions with mixing-controlled reaction rates, an explicit Euler scheme is used to solve the system of ODEs that determine the chemical mass production rate, $\dot{m}^{\prime\prime\prime}_{\alpha}$.   


\subsection{Other Chemistry Models}

For anything other than single step chemistry with mixing-controlled reaction rates a more complex ODE solver is employed.  A fourth-order explicit integrator with error control is used.  Since chemical timescales are often fast with respect to the hydrodynamic timescales, the solver uses subtimesteps during time integration of the chemical reactions to maintain the specified error tolerance.  Time integration is halted once no reactants are present of if the local heat release rate exceeds maximum allowable values.

More information on the numerical methods of the integrators can be found in Appendix~\ref{chemistry_integration}.


\section{Heat Release Rate}

The discussion of the various reaction mechanisms above is essentially book-keeping, the accounting of the gas
molecules formed in the combustion process. But what of the heat released?

Each chemical reaction in FDS must be defined with a fuel and a heat of combustion.  "Fuel" in this case merely denotes which species the heat of combustion is based upon.  During time integration of the reactions, there is a time integration of the heat release rate.  While the mixing controlled reaction rate is an easily computed and robust subgrid-scale model of and the Arrhenius rate is certainly a robust model when applicable, there is still a need, in certain situations, to put an upper bound on the local heat release rate per unit volume. The reason for
this is that FDS is applied over length scales ranging from millimeters to tens of meters, and the resolution of the numerical grid
is sometimes too coarse to expect the simple mixing time model to work effectively.
A scaling analysis of pool fires by Orloff and De Ris~\cite{Orloff:19th_Symposium} suggests that the spatial average of the
heat release rate of a fire is approximately 1200~kW/m$^3$. FDS uses by default a value of 2500~kW/m$^3$ as an upper bound\footnote{Note that
for DNS, FDS imposes a less restrictive upper bound on the local heat release rate per unit volume. It is
\be \dq_{\max}''' = 200/\dx + 2500 \quad \hbox{kW/m}^3 \ee
The value of 200~kW/m$^2$ is an upper bound on the heat release rate per unit area of flame sheet.}
on the local value of the heat release rate per unit volume. Typically, this bound only affects fires whose value of $Q^*$
is less than one\footnote{The non-dimensional quantity, $Q^*$, is a measure of the fire's heat release rate divided by the
area of its base. It is expressed as $Q^*=\dQ/(\rho_\infty c_p T_\infty \sqrt{g D} D^2)$.}.

