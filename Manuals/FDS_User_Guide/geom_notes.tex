\documentclass[12pt]{article}
% $Date$
% $Revision$
% $Author$

%%%%%%%%%%%%%%%%%%%%%%%%%%%%%%%%%%%%%%%%%%%%%%%%%%%%%%%%%%%%%%%%%%%%%%%%%%%%%%%%%%%%%%%%%%%%%%%%%%%
%                                                                                                 %
% The mathematical style of these documents follows                                               %
%                                                                                                 %
% A. Thompson and B.N. Taylor. The NIST Guide for the Use of the International System of Units.   %
%    NIST Special Publication 881, 2008.                                                          %
%                                                                                                 %
% http://www.nist.gov/pml/pubs/sp811/index.cfm                                                    %
%                                                                                                 %
%%%%%%%%%%%%%%%%%%%%%%%%%%%%%%%%%%%%%%%%%%%%%%%%%%%%%%%%%%%%%%%%%%%%%%%%%%%%%%%%%%%%%%%%%%%%%%%%%%%

% Packages which force the use of better TeX coding
% Mostly from http://tex.stackexchange.com/q/19264
%%\RequirePackage[l2tabu, orthodox]{nag}
%%\usepackage{fixltx2e}
%\usepackage{isomath} % Disabled for the moment because it changes the syntax for bold and roman Greek math symbols
%%\usepackage[all,warning]{onlyamsmath}
%\usepackage{strict} % Commented out for now because it is uncommon. A copy of style.sty is in Manuals/LaTeX_Style_Files/.

\usepackage{times,mathptmx}
\usepackage[pdftex]{graphicx} % use \usepackage[pdftex,demo]{graphicx} to suppress images
\usepackage{tabularx}
\usepackage{multirow}
%\usepackage{pdfsync}
\usepackage{tikz}
\usepackage{bm}
\usepackage{pgfplots}
%\pgfplotsset{compat=1.7}
\usepackage{tocloft}
\usepackage{color}
\definecolor{linknavy}{rgb}{0,0,0.50196}
\definecolor{linkred}{rgb}{1,0,0}
\definecolor{linkblue}{rgb}{0,0,1}
\usepackage{amsmath}
\usepackage{cancel}
\usepackage{float}
\usepackage{caption}
\usepackage{pict2e}
\usepackage{graphpap}
\usepackage{rotating}
\usepackage{geometry}
\usepackage{relsize}
\usepackage{longtable}
\usepackage{xltabular}
\usepackage{lscape}
\usepackage{booktabs}
\usepackage{colortbl}
\definecolor{lavender}{rgb}{0.9, 0.9, 0.98}
\usepackage{amssymb}
\usepackage{threeparttable}
\usepackage{makeidx} % Create index at end of document
\usepackage[nottoc,notlof,notlot]{tocbibind} % Put the bibliography and index in the ToC
\usepackage{lastpage} % Automatic last page number reference.
\usepackage[T1]{fontenc}
\usepackage{enumerate}
\usepackage{upquote}
\usepackage{moreverb}
\usepackage{morefloats}
\usepackage[section]{placeins}
\usepackage{scrextend}
\usepackage{needspace}
\usepackage[backend=biber, style=numeric, sorting=none, backref=true]{biblatex}

\newcommand{\nopart}{\expandafter\def\csname Parent-1\endcsname{}} % To fix table of contents in pdf.
\newcommand{\ct}[1]{\lstinline{#1}}
\newcommand{\tct}[1]{\lstinline[basicstyle=\scriptsize\ttfamily]!#1!}

\usepackage{siunitx}

\usepackage{listings}
\usepackage{textcomp}
\lstset{
    tabsize=4,
    rulecolor=,
    language=Fortran,
        basicstyle=\small\ttfamily,
        upquote=true,
        aboveskip={\baselineskip},
        belowskip={\baselineskip},
        columns=fixed,
        extendedchars=true,
        breaklines=true,
        breakatwhitespace=true,
        frame=none,
        showtabs=false,
        showspaces=false,
        showstringspaces=false,
        identifierstyle=\ttfamily,
        keywordstyle=\color[rgb]{0,0,0},
        commentstyle=\color[rgb]{0,0,0},
        stringstyle=\color[rgb]{0,0,0},
        literate={\_}{}{0\discretionary{\_}{}{\_}}
                 {/}{}{0\discretionary{/}{}{/}}%
}

\usepackage{xr-hyper}
\usepackage[pdftex,
        colorlinks=true,
        urlcolor=linkblue,     % \href{...}{...} external (URL)
        citecolor=linkred,     % citation number colors
        linkcolor=linknavy,    % \ref{...} and \pageref{...}
        pdfproducer={pdflatex},
        pdfpagemode=UseNone,
        bookmarksopen=true,
        plainpages=false,
        verbose]{hyperref}

% The Following commented code makes the ``Draft'' watermark on each page.
%\usepackage{eso-pic}
%\usepackage{type1cm}
%\makeatletter
%   \AddToShipoutPicture{
%     \setlength{\@tempdimb}{.5\paperwidth}
%     \setlength{\@tempdimc}{.5\paperheight}
%     \setlength{\unitlength}{1pt}
%     \put(\strip@pt\@tempdimb,\strip@pt\@tempdimc){
%     \makebox(0,0){\rotatebox{45}{\textcolor[gray]{0.75}{\fontsize{8cm}\selectfont{RC6}}}}}
% }
%\makeatother

\captionsetup[figure]{font=small}

\setlength{\textwidth}{6.5in}
\setlength{\textheight}{9.0in}
\setlength{\topmargin}{0.in}
\setlength{\headheight}{0.in}
\setlength{\headsep}{0.in}
\setlength{\parindent}{0.25in}
\setlength{\oddsidemargin}{0.0in}
\setlength{\evensidemargin}{0.0in}
\setlength{\leftmargini}{\parindent}        % Controls the indenting of the "bullets" in a list
\cftsetindents{section}{.25in}{0.40in}      % Distance from left margin to section number; Width of section number and space before section title
\cftsetindents{subsection}{0.65in}{0.60in}  % Distance from left margin to subsection number; Width of subsection number and space before subsection title
\setlength{\cftfignumwidth}{0.45in}         % Width of figure number and space before figure caption in the list of figures
\setlength{\cfttabnumwidth}{0.45in}         % Width of table number and space before table caption in the list of tables

\makeatletter
\setlength{\@fptop}{0pt}                    % Figures on separate pages pushed to the top
\setlength{\@fpbot}{0pt plus 1fil}
\makeatother

\newcommand{\authortitlesigs}
{
\begin{flushright}
Kevin McGrattan \\
Simo Hostikka \\
Jason Floyd \\
Randall McDermott \\
Marcos Vanella \\
Eric Mueller \\
Chandan Paul
\end{flushright}
}

\newcommand{\logosigs}{
\begin{minipage}[b]{6.25in}
\parbox[b]{.5\textwidth}{\flushleft{\includegraphics[height=1.5in]{../Bibliography/FDS_Logo_lock}}}
\hfill
\parbox[b]{.5\textwidth}{\flushright{\includegraphics[height=1in]{../Bibliography/nistident_flright_vec}}}
\end{minipage}
}

\newcommand{\authorsigs}
{
\begin{flushright}
Kevin McGrattan \\
Randall McDermott \\
Marcos Vanella \\
Eric Mueller \\
{\em Fire Research Division, Engineering Laboratory, Gaithersburg, Maryland} \\[.1in]
Simo Hostikka \\
{\em Aalto University, Espoo, Finland} \\[.1in]
Jason Floyd \\
{\em Fire Safety Research Institute, UL Research Institutes, Columbia, Maryland} \\[.1in]
Chandan Paul \\
{\em The George Washington University, Washington, D.C.}
\end{flushright}
}

\newcommand{\titlesigs}
{
\small
\begin{flushright}
U.S. Department of Commerce \\
{\em Howard Lutnick, Secretary} \\
\hspace{1in} \\
National Institute of Standards and Technology \\
{\em Craig Burkhardt, Acting NIST Director and Acting Under Secretary of Commerce for Standards and Technology}
\end{flushright}
}


\newcommand{\disclaimer}[1]
{
\begin{minipage}[t]{6.25in}
\fontsize{10}{12}\selectfont
\begin{flushright}
Certain commercial entities, equipment, or materials may be identified in this \\
document in order to describe an experimental procedure or concept adequately. \\
Such identification is not intended to imply recommendation or endorsement by the \\
National Institute of Standards and Technology, nor is it intended to imply that the \\
entities, materials, or equipment are necessarily the best available for the purpose.
\end{flushright}
\vspace{3in}
\large
\flushright{\bf National Institute of Standards and Technology Special Publication #1 \\
Natl.~Inst.~Stand.~Technol.~Spec.~Publ.~#1, \pageref{LastPage} pages (October 2013) \\
CODEN: NSPUE2}
\vfill
\hspace{1in}
\end{minipage}
}



\newcommand{\gforneybio}
{
\item[Glenn Forney] is a computer scientist at the Engineering Laboratory of NIST.  He received a
bachelor of science degree in mathematics from Salisbury State College and a master of
science and a doctorate in mathematics from Clemson University.  He joined NIST
in 1986 (then the National Bureau of Standards) and has since worked on developing tools that
provide a better understanding of fire phenomena, most notably Smokeview, a software tool for visualizing
Fire Dynamics Simulator data.
}

\newcommand{\smvoverview}
{
This guide is part of a three volume set of companion documents describing how to use Smokeview
in Volume I, the Smokeview User's Guide~\cite{Smokeview_Users_Guide}, describing technical details of how the visualizations are performed in Volume II, the Smokeview Technical Reference Guide~\cite{Smokeview_Tech_Guide}, and presents example cases
verifying the various visualization capabilities of Smokeview in Volume III, the Smokeview Verification Guide~\cite{Smokeview_Verification_Guide}.  Details on the use and technical background of the Fire Dynamics Simulator is contained in the FDS User's~\cite{FDS_Users_Guide} and Technical reference guide~\cite{FDS_Math_Guide}
respectively.
}

% commands to use for "official" cover and title pages
% see smokeview verification guide to see how they are used

\newcommand{\headerA}[1]{
\begin{flushright}
\fontsize{20}{24}\selectfont
\bf{NIST Special Publication #1}
\end{flushright}
}


\newcommand{\headerB}[1]{
\begin{flushright}
\fontsize{28}{33.6}\selectfont
\bf{#1}
\end{flushright}
}

\newcommand{\headerC}[1]{
\vspace{.15in}
\begin{flushright}
\fontsize{12}{14}\selectfont
#1
\end{flushright}
}

\newcommand{\headerD}[1]{
\begin{flushright}
\fontsize{12}{14}\selectfont
http://dx.doi.org/10.6028/NIST.SP.#1
\end{flushright}
}



\newcommand{\dod}[2]{\frac{\partial #1}{\partial #2}}
\newcommand{\DoD}[2]{\frac{\mathrm{D} #1}{\mathrm{D} #2}}
\newcommand{\dsods}[2]{\frac{\partial^2 #1}{\partial #2^2}}
\renewcommand{\d}{\,\mathrm{d}}
\newcommand{\dx}{\delta x}
\newcommand{\dy}{\delta y}
\newcommand{\dz}{\delta z}
\newcommand{\degF}{$^\circ$F}
\newcommand{\degC}{$^\circ$C}
\newcommand{\x}{x}
\newcommand{\y}{y}
\newcommand{\z}{z}
\newcommand{\dt}{\delta t}
\newcommand{\dn}{\delta n}
\newcommand{\cH}{H}
\newcommand{\hu}{u}
\newcommand{\hv}{v}
\newcommand{\hw}{w}
\newcommand{\la}{\lambda}
\newcommand{\bO}{{\Omega}}
\newcommand{\bo}{{\mathbf{\omega}}}
\newcommand{\btau}{\mathbf{\tau}}
\newcommand{\bdelta}{{\mathbf{\delta}}}
\newcommand{\sumyw}{\sum (Y_\alpha/W_\alpha)}
\newcommand{\oW}{\overline{W}}
\newcommand{\om}{\ensuremath{\omega}}
\newcommand{\omx}{\omega_x}
\newcommand{\omy}{\omega_y}
\newcommand{\omz}{\omega_z}
\newcommand{\erf}{\hbox{erf}}
\newcommand{\erfc}{\hbox{erfc}}
\newcommand{\bF}{{\mathbf{F}}}
\newcommand{\bG}{{\mathbf{G}}}
\newcommand{\bof}{{\mathbf{f}}}
\newcommand{\bq}{{\mathbf{q}}}
\newcommand{\br}{{\mathbf{r}}}
\newcommand{\bu}{{\mathbf{u}}}
\newcommand{\bx}{{\mathbf{x}}}
\newcommand{\bk}{{\mathbf{k}}}
\newcommand{\bv}{{\mathbf{v}}}
\newcommand{\bg}{{\mathbf{g}}}
\newcommand{\bn}{{\mathbf{n}}}
\newcommand{\bS}{{\mathbf{S}}}
\newcommand{\bW}{\overline{W}}
\newcommand{\dS}{d{\mathbf{S}}}
\newcommand{\bs}{{\mathbf{s}}}
\newcommand{\bI}{{\mathbf{I}}}
\newcommand{\hp}{H}
\newcommand{\trho}{\tilde{\rho}}
\newcommand{\dph}{{\delta\phi}}
\newcommand{\dth}{{\delta\theta}}
\newcommand{\tp}{\tilde{p}}
\newcommand{\bp}{\overline{p}}
\newcommand{\dQ}{\dot{Q}}
\newcommand{\dq}{\dot{q}}
\newcommand{\dbq}{\dot{\mathbf{q}}}
\newcommand{\dm}{\dot{m}}
\newcommand{\ha}{\frac{1}{2}}
\newcommand{\ft}{\frac{4}{3}}
\newcommand{\ot}{\frac{1}{3}}
\newcommand{\fofi}{\frac{4}{5}}
\newcommand{\of}{\frac{1}{4}}
\newcommand{\twth}{\frac{2}{3}}
\newcommand{\R}{R}
\newcommand{\be}{\begin{equation}}
\newcommand{\ee}{\end{equation}}
\newcommand{\RE}{\hbox{Re}}
\newcommand{\LE}{\hbox{Le}}
\newcommand{\PR}{\hbox{Pr}}
\newcommand{\PE}{\hbox{Pe}}
\newcommand{\NU}{\hbox{Nu}}
\newcommand{\SC}{\hbox{Sc}}
\newcommand{\SH}{\hbox{Sh}}
\newcommand{\WE}{\hbox{We}}
\newcommand{\OI}{\text{\tiny \hbox{OI}}}
\newcommand{\COTWO}{\text{\tiny \hbox{CO}$_2$}}
\newcommand{\HTWOO}{\text{\tiny \hbox{H}$_2$\hbox{O}}}
\newcommand{\OTWO}{\text{\tiny \hbox{O}$_2$}}
\newcommand{\NTWO}{\text{\tiny \hbox{N}$_2$}}
\newcommand{\CO}{\text{\tiny \hbox{CO}}}
\newcommand{\HCN}{\text{\tiny \hbox{HCN}}}
\newcommand{\F}{\text{\tiny \hbox{F}}}
\newcommand{\C}{\text{\tiny \hbox{C}}}
\newcommand{\Hy}{\text{\tiny \hbox{H}}}
\newcommand{\So}{\text{\tiny \hbox{S}}}
\newcommand{\M}{\text{\tiny \hbox{M}}}
\newcommand{\xx}{\text{\tiny \hbox{x}}}
\newcommand{\yy}{\text{\tiny \hbox{y}}}
\newcommand{\zz}{\text{\tiny \hbox{z}}}
\newcommand{\smvlines}{120~000}

\newcommand{\calH}{\mathcal{H}}
\newcommand{\calR}{\mathcal{R}}

\newcommand{\dif}{\mathrm{d}}
\newcommand{\Div}{\nabla\cdot}
\newcommand{\D}{\mbox{D}}
\newcommand{\mhalf}{\mbox{$\frac{1}{2}$}}
\newcommand{\thalf}{\mbox{\tiny $\frac{1}{2}$}}
\newcommand{\tripleprime}{{\prime\prime\prime}}
\newcommand{\ppp}{{\prime\prime\prime}}
\newcommand{\pp}{{\prime\prime}}

\newcommand{\superscript}[1]{\ensuremath{^{\textrm{\tiny #1}}}}
\newcommand{\subscript}[1]{\ensuremath{_{\textrm{\tiny #1}}}}

\newcommand{\rb}[1]{\raisebox{1.5ex}[0pt]{#1}}

\newcommand{\Ra}{$\Rightarrow$}
\newcommand{\hhref}[1]{\href{#1}{{\tt #1}}}
\newcommand{\fdsinput}[1]{{\scriptsize\verbatiminput{../../Verification/Visualization/#1}}}

\definecolor{AQUAMARINE}{rgb}{0.49804,1.00000,0.83137}
\definecolor{ANTIQUE WHITE}{rgb}{0.98039,0.92157,0.84314}
\definecolor{BEIGE}{rgb}{0.96078,0.96078,0.86275}
\definecolor{BLACK}{rgb}{0.00000,0.00000,0.00000}
\definecolor{BLUE}{rgb}{0.00000,0.00000,1.00000}
\definecolor{BLUE VIOLET}{rgb}{0.54118,0.16863,0.88627}
\definecolor{BRICK}{rgb}{0.61176,0.40000,0.12157}
\definecolor{BROWN}{rgb}{0.64706,0.16471,0.16471}
\definecolor{BURNT SIENNA}{rgb}{0.54118,0.21176,0.05882}
\definecolor{BURNT UMBER}{rgb}{0.54118,0.20000,0.14118}
\definecolor{CADET BLUE}{rgb}{0.37255,0.61961,0.62745}
\definecolor{CHOCOLATE}{rgb}{0.82353,0.41176,0.11765}
\definecolor{COBALT}{rgb}{0.23922,0.34902,0.67059}
\definecolor{CORAL}{rgb}{1.00000,0.49804,0.31373}
\definecolor{CYAN}{rgb}{0.00000,1.00000,1.00000}
\definecolor{DIM GRAY }{rgb}{0.41176,0.41176,0.41176}
\definecolor{EMERALD GREEN}{rgb}{0.00000,0.78824,0.34118}
\definecolor{FIREBRICK}{rgb}{0.69804,0.13333,0.13333}
\definecolor{FLESH}{rgb}{1.00000,0.49020,0.25098}
\definecolor{FOREST GREEN}{rgb}{0.13333,0.54510,0.13333}
\definecolor{GOLD }{rgb}{1.00000,0.84314,0.00000}
\definecolor{GOLDENROD}{rgb}{0.85490,0.64706,0.12549}
\definecolor{GRAY}{rgb}{0.50196,0.50196,0.50196}
\definecolor{GREEN}{rgb}{0.00000,1.00000,0.00000}
\definecolor{GREEN YELLOW}{rgb}{0.67843,1.00000,0.18431}
\definecolor{HONEYDEW}{rgb}{0.94118,1.00000,0.94118}
\definecolor{HOT PINK}{rgb}{1.00000,0.41176,0.70588}
\definecolor{INDIAN RED}{rgb}{0.80392,0.36078,0.36078}
\definecolor{INDIGO}{rgb}{0.29412,0.00000,0.50980}
\definecolor{IVORY}{rgb}{1.00000,1.00000,0.94118}
\definecolor{IVORY BLACK}{rgb}{0.16078,0.14118,0.12941}
\definecolor{KELLY GREEN}{rgb}{0.00000,0.50196,0.00000}
\definecolor{KHAKI}{rgb}{0.94118,0.90196,0.54902}
\definecolor{LAVENDER}{rgb}{0.90196,0.90196,0.98039}
\definecolor{LIME GREEN}{rgb}{0.19608,0.80392,0.19608}
\definecolor{MAGENTA}{rgb}{1.00000,0.00000,1.00000}
\definecolor{MAROON}{rgb}{0.50196,0.00000,0.00000}
\definecolor{MELON}{rgb}{0.89020,0.65882,0.41176}
\definecolor{MIDNIGHT BLUE}{rgb}{0.09804,0.09804,0.43922}
\definecolor{MINT}{rgb}{0.74118,0.98824,0.78824}
\definecolor{NAVY}{rgb}{0.00000,0.00000,0.50196}
\definecolor{OLIVE}{rgb}{0.50196,0.50196,0.00000}
\definecolor{OLIVE DRAB}{rgb}{0.41961,0.55686,0.13725}
\definecolor{ORANGE}{rgb}{1.00000,0.50196,0.00000}
\definecolor{ORANGE RED}{rgb}{1.00000,0.27059,0.00000}
\definecolor{ORCHID}{rgb}{0.85490,0.43922,0.83922}
\definecolor{PINK}{rgb}{1.00000,0.75294,0.79608}
\definecolor{POWDER BLUE}{rgb}{0.69020,0.87843,0.90196}
\definecolor{PURPLE}{rgb}{0.50196,0.00000,0.50196}
\definecolor{RASPBERRY}{rgb}{0.52941,0.14902,0.34118}
\definecolor{RED}{rgb}{1.00000,0.00000,0.00000}
\definecolor{ROYAL BLUE}{rgb}{0.25490,0.41176,0.88235}
\definecolor{SALMON}{rgb}{0.98039,0.50196,0.44706}
\definecolor{SANDY BROWN}{rgb}{0.95686,0.64314,0.37647}
\definecolor{SEA GREEN}{rgb}{0.32941,1.00000,0.62353}
\definecolor{SEPIA}{rgb}{0.36863,0.14902,0.07059}
\definecolor{SIENNA}{rgb}{0.62745,0.32157,0.17647}
\definecolor{SILVER}{rgb}{0.75294,0.75294,0.75294}
\definecolor{SKY BLUE}{rgb}{0.52941,0.80784,0.92157}
\definecolor{SLATEBLUE}{rgb}{0.41569,0.35294,0.80392}
\definecolor{SLATE GRAY}{rgb}{0.43922,0.50196,0.56471}
\definecolor{SPRING GREEN}{rgb}{0.00000,1.00000,0.49804}
\definecolor{STEEL BLUE}{rgb}{0.27451,0.50980,0.70588}
\definecolor{TAN}{rgb}{0.82353,0.70588,0.54902}
\definecolor{TEAL}{rgb}{0.00000,0.50196,0.50196}
\definecolor{THISTLE}{rgb}{0.84706,0.74902,0.84706}
\definecolor{TOMATO }{rgb}{1.00000,0.38824,0.27843}
\definecolor{TURQUOISE}{rgb}{0.25098,0.87843,0.81569}
\definecolor{VIOLET}{rgb}{0.93333,0.50980,0.93333}
\definecolor{VIOLET RED}{rgb}{0.81569,0.12549,0.56471}
\definecolor{WHITE}{rgb}{1.00000,1.00000,1.00000}
\definecolor{YELLOW}{rgb}{1.00000,1.00000,0.00000}

\floatstyle{boxed}
\newfloat{notebox}{H}{lon}
\newfloat{warning}{H}{low}

% Set default longtable alignment
\setlength\LTleft{0pt}
\setlength\LTright{0pt}

% Prevent large paragraph separations
\raggedbottom

% Allow multi-line equations to span page breaks
\allowdisplaybreaks

% Conditional to activate Unstructured Geometry text:
\newif\ifcompgeom
\compgeomtrue


\begin{document}

\vspace{1.0in}
\section{Unstructured Geometry: The \texorpdfstring{{\tt GEOM}}{GEOM} Namelist Group (Table \ref{tbl:GEOM})}
\label{info:GEOM}
\subsection{Defining Surfaces}

\subsubsection{General Surfaces}
Unstructured geometric surfaces are defined using the {\tt \&GEOM}\ namelist.
These surfaces consist of a collection of triangular faces where each face consists of three vertices.
A simple form of the {\tt \&GEOM}\ namelist defining one triangular face is given by

\begin{verbatim}
&GEOM ID='triangle'
      VERTS=0.0,0.0,0.0, 1.0,0.0,0.0, 1.0,0.0,1.0,
      FACES=1,2,3
/
\end{verbatim}

\noindent where {\tt ID}\ specifies the object name, in this case {\tt triangle},
{\tt VERTS}\ specifies a list of one or more $(x,y,z)$ coordinates and {\tt FACES}\ specifies a list of vertices, 3 vertex indices for each
face. Each index ranges from 1 to the number of vertices found on this {\tt \&GEOM}\ line.

One may also define particular kinds of geometric objects,  blocks using {\tt XB}, spheres using {\tt SPHERE\_ORIGIN}\ and {\tt SPHERE\_RADIUS}
and a 2D surface using {\tt ZVALS}.
FDS generate vertices and faces to represent these objects, equivalent to what would have been defined using {\tt VERTS}\ and {\tt FACES}.

\subsubsection{Blocks}
A {\tt \&GEOM}\ namelist defining a block is given by

\begin{verbatim}
&GEOM ID='block'
      XB=0.0,1.0,0.0,1.0,0.0,1.0 /
\end{verbatim}

\noindent where {\tt XB=xmin,xmax,ymin,ymax,zmin,zmax} defines the min and max bounds of the block.
The {\tt XB}\ parameter is used in the same way as on an {\tt \&OBST}\ or {\tt \&VENT}\ line.
A block may be discretized into many parts by
specifying the {\tt IJK} parameter.  For example, {\tt IJK=8,6,4} would split the block
into 8 parts along the $x$ dimension, 6 parts along the $y$ dimension and 4 parts along
the $z$ dimension. By default, blocks are discretized
so that the block faces are consistent in size with the grid resolution.

\subsubsection{Spheres}
A {\tt \&GEOM}\ namelist defining a sphere centered at $(0,0,0)$ with radius $1$ is given by

\begin{verbatim}
&GEOM ID='sphere'
      SPHERE_ORIGIN=0.0,0.0,0.0,SPHERE_RADIUS=1.0 /
\end{verbatim}

\noindent Spheres are discretized by default so that each face is consistent in size with the grid resolution.
One may specify a {\tt N\_LEVEL}\ parameter which specifies the number of times the sphere is split.

\subsubsection{2D Surfaces}
A {\tt \&GEOM}\ namelist defining a 2D surface is given by

\begin{verbatim}
&GEOM ID='terrain'
      IJK=ivals,jvals,XB=xmin,xmax,ymin,ymax,ZVALS=..... /
\end{verbatim}

\noindent where {\tt XB}\ defines a rectangular region bounded by xmin, xmax, ymin and ymax where elevation data is defined with {\tt ZVALS}.
{\tt IJK}\ specifies how many vertices occur in this region along each dimension in this region.
In this example {\tt ivals}\ values occur along the x dimension and {\tt jval}\ values occur along the y dimension.
The {\tt ZVALS} keyword is used to specify elevations at each $(x,y)$ location.
The elevation data specified after the {\tt ZVALS}\ keyword is arranged in row major order.
The first row contains {\tt ivals} elevation values occurring at the the ymax position from xmin to xmax.
Similarly, the last row contains
{\tt ivals}\ elevation values for the ymin position again from xmin to xmax.
There are then {\tt jvals}\ rows and {\tt ivals}\ columns of elevation data.
As with the blocks and spheres, FDS uses the information
provided by these keywords to construct
vertices and triangular faces.

\subsection{Defining Solids}
Unstructured geometric solids are also defined using the {\tt \&GEOM}\ namelist.
These solids consist of a collection of tetrahedrons where each tetrahedron consists of four vertices.
A simple form of the {\tt \&GEOM}\ namelist defining one tetrahedron is given by

\begin{verbatim}
&GEOM ID='tetrahedron'
      VERTS=0.0,0.0,0.0, 1.0,0.0,0.0, 1.0,1.0,0.0,  1.0,1.0,1.0,
      VOLUS=1,2,3,4
/
\end{verbatim}

\noindent where {\tt ID}\ specifies the object name, in this case {\tt tetrahedron},
{\tt VERTS}\ specifies a list of one or more $(x,y,z)$ coordinates and {\tt VOLUS}\ specifies a list of vertices, 4 vertex indices for each
tetrahedron. Each index ranges from 1 to the number of vertices found on this {\tt \&GEOM}\ line.


\subsection{Transforming Objects}
An object may be transformed.
It  may be translated using {\tt XYZ}, scaled using {\tt SCALE} and rotated about {\tt XYZ0}\ using {\tt AZIM}\ and/or {\tt ELEV}.
For example, in the following {\tt \&GEOM}\ namelist,

\begin{verbatim}
&GEOM ID='chair'

      VERTS=X1,Y1,Z1,...,XM,YM,ZM,

      FACES=F1_1,F1_2,F1_3,...,FN_1,FN_2,FN_3,

      SCALE=sx,sy,sz,
      AZIM=az,
      ELEV=elev,

      XYZ0=x0,y0,z0,
      XYZ=x,y,z,
/
\end{verbatim}

\noindent the chair object is translated by $(x,y,z)$ by specifying {\tt XYZ=x,y,z}.
Similarly, the chair is scaled by $sx$, $sy$, $sz$ along each coordinate direction using {\tt SCALE=sx,sy,sz}.
An object may be flipped by using a negative scale factor.
An object is rotated using {\tt AZIM=az}\ and {\tt ELEV=elev}.
{\tt AZIM}\ is used to specify an azimuthal rotation angle about a vertical (z) axis centered at an origin defined by {\tt XYZ0}.
{\tt ELEV}\ specifies a rotation angle relative to a horizontal plane again containing an origin defined by {\tt XYZ0}..
An object may also be rotated about an arbitrary axis.
{\tt GAXIS} is used to specify this axis.
{\tt GROTATE}\ is used to specify the amount of rotation about this axis.
All rotation angles are specified in degrees.

\subsection{Grouping Objects}
One may group objects to form a new object.  In the following example,
{\tt chair}\ and {\tt couch}\ are objects that have been defined previously.  A new {\tt living room set}\ object
is defined by referencing {\tt chair}\ and {\tt couch}\ in the {\tt GEOM\_IDS}\ keyword.  The chair and couch
are place at a particular locations using the {\tt DXYZ}\ keyword.

{%\footnotesize
\begin{verbatim}
&GEOM ID='chair' ..... /

&GEOM ID='sofa' ...... /

&GEOM ID='living room set'

      XYZ=[0,0,0],
      AZIM=[0],
      ELEV=[0],

      GEOM_IDS(1)='chair',  DSCALE(1)=..., DAZIM(1)=..., DELEV(1)...
                            DXYZ0(1:3,1)=..., DXYZ(1:3,1)=...,
      GEOM_IDS(2)='couch',  DSCALE(2)=..., DAZIM(2)=..., DELEV(2)...
                            DXYZ0(1:3,2)=..., DXYZ(1:3,2)=...,
/
\end{verbatim}
}

\noindent Referenced objects such as {\tt chair}\ and {\tt couch} in this case must be defined in
previous {\&GEOM}\ lines.  Forward references are not permitted.
One may scale, rotate and translate objects as they are placed in the group using
{\tt DSCALE}\, for scaling, {\tt DAZIM} and {\tt DELEV} for rotation and {\tt DXYZ} for translation.

The {\tt DXYZ0}\ and {\tt DXYZ}\ keywords use two coordinates (array indices) to specify data.
The first coordinate specifies the spatial component (1 for x, 2 for y, 3 for z or 1:3 for all three).
The second coordinate specifies the object number according to its position in the {\tt GEOM\_IDS}\ array.
For example,
DXYZ(1,3) would specify the translation along the x direction for the third object in the {\tt GEOM\_IDS}\ array.

The {\tt COMPONENT\_ONLY}\ keyword if .TRUE., indicates that the object being defined is only a component found in other geometries.
It will not appear by itself in the FDS model.

\subsection{Dynamic Objects}
An object's position and orientation may change with time.
These changes are specified using various keywords that end with {\tt \_DOT}.
See Table \ref{tbl:GEOM} for a complete list of these keywords.
In the following example, {\tt AZIM\_DOT} is used to vary the azimuthal rotation angle by 1 deg/s.
Similarly, the position may be changed by using {\tt XYZ\_DOT}.
For example, {\tt XYZ\_DOT=0.0,0.0,1.0}\ would cause an object to move 1~m/s in a vertical direction.
\begin{verbatim}
&GEOM VERTS=...,FACES=....,AZIM=0.0,AZIM_DOT=1.0 /
\end{verbatim}

\vspace{\baselineskip}

\section{\texorpdfstring{{\tt GEOM}}{GEOM} (Unstructured Geometry Parameters)}

\begin{longtable}{@{\extracolsep{\fill}}|l|l|l|l|l|}
\caption[Unstructured geometry parameters ({\ct GEOM} namelist group)]{For more information see Section~\ref{info:GEOM}.}
\label{tbl:GEOM} \\
\hline
\multicolumn{5}{|c|}{{\ct GEOM} (Unstructured Geometry Parameters)} \\
\hline \hline
\endfirsthead
\caption[]{Continued} \\
\hline
\multicolumn{5}{|c|}{{\ct GEOM} (Unstructured Geometry Parameters)} \\
\hline \hline
\endhead
{\ct AZIM}         & Real                   & Section~\ref{info:GEOM}            &  deg      &    0.0                   \\ \hline
{\ct AZIM\_DOT}    & Real                   & Section~\ref{info:GEOM}            &  deg/s    &    0.0                   \\ \hline
{\ct COMPONENT\_ONLY} & LOGICAL             & Section~\ref{info:GEOM}            &           &  {\ct .FALSE.}           \\ \hline
{\ct DAZIM}        & array of Reals         & Section~\ref{info:GEOM}            &  deg      &    0.0                   \\ \hline
{\ct DELEV}        & array of Reals         & Section~\ref{info:GEOM}            &  deg      &    0.0                   \\ \hline
{\ct DSCALE}       & array of Real Triplets & Section~\ref{info:GEOM}            &           &   1.0                    \\ \hline
{\ct DXYZ0}        & array of Real Triplets & Section~\ref{info:GEOM}            &   m       &   0.0                    \\ \hline
{\ct DXYZ}         & array of Real Triplets & Section~\ref{info:GEOM}            &   m       &   0.0                    \\ \hline
{\ct ELEV}         & Real                   & Section~\ref{info:GEOM}            &  deg      &    0.0                   \\ \hline
{\ct ELEV\_DOT}    & Real                   & Section~\ref{info:GEOM}            &  deg/s    &    0.0                   \\ \hline
{\ct FACES}        & array of Integer Triplets     & Section~\ref{info:GEOM}     &           &    0                     \\ \hline
{\ct GAXIS}        & Real Triplet           & Section~\ref{info:GEOM}            &           &                          \\ \hline
{\ct GEOM\_IDS}    & Real                   & Section~\ref{info:GEOM}            &           &                          \\ \hline
{\ct GROTATE}      & Real Triplet           & Section~\ref{info:GEOM}            &  deg      &    0.0                   \\ \hline
{\ct GROTATE\_DOT}  & Real Triplet          & Section~\ref{info:GEOM}            & deg/s     &    0.0                   \\ \hline
{\ct ID}           & Character              & Section~\ref{info:GEOM}            &           &   {\ct 'geom'}           \\ \hline
{\ct IJK}          & Integer Triplet        & Section~\ref{info:GEOM}            &           &   0,0,0                  \\ \hline
{\ct MATL\_ID}     & Character              & Section~\ref{info:GEOM}            &           &  {\ct 'INERT'}           \\ \hline
{\ct N\_LAT}       & Integer                & Section~\ref{info:GEOM}            &           &   0                      \\ \hline
{\ct N\_LEVELS}    & Integer                & Section~\ref{info:GEOM}            &           &   0                      \\ \hline
{\ct N\_LONG}      & Integer                & Section~\ref{info:GEOM}            &           &   0                      \\ \hline
{\ct SCALE}        & Real Triplet           & Section~\ref{info:GEOM}            &           &   1.0,1.0,1.0            \\ \hline
{\ct SCALE\_DOT}   & Real Triplet           & Section~\ref{info:GEOM}            &  1/s      &   1.0                    \\ \hline
{\ct SPHERE\_ORIGIN}& Real Triplet          & Section~\ref{info:GEOM}            &   m       &  0.0,0.0,0.0             \\ \hline
{\ct SPHERE\_RADIUS}& Real                  & Section~\ref{info:GEOM}            &   m       &  1.0                     \\ \hline
{\ct SPHERE\_TYPE} & Integer                & Section~\ref{info:GEOM}            &           &  1                       \\ \hline
{\ct SURF\_ID}     & Character              & Section~\ref{info:GEOM}            &           &  {\ct 'INERT'}           \\ \hline
{\ct TEXTURE\_MAPPING}& Character           & Section~\ref{info:GEOM}            &           & {\ct 'RECTANGULAR'}      \\ \hline
{\ct TEXTURE\_ORIGIN} & Real Triplet        & Section~\ref{info:GEOM}            &   m       &   0.0,0.0,0.0            \\ \hline
{\ct TEXTURE\_SCALE}& Real                  & Section~\ref{info:GEOM}            &           &   1.0                    \\ \hline
{\ct VERTS}        & array of Real Triplets & Section~\ref{info:GEOM}            &   m       &   0.0                    \\ \hline
{\ct VOLUS}        & array of Integer Quadruplets     & Section~\ref{info:GEOM}  &           &    0                     \\ \hline
{\ct XB}           & Real sex-tuplet        & Section~\ref{info:GEOM}            &   m       &   0.0                    \\ \hline
{\ct XYZ0}         & Real Triplet           & Section~\ref{info:GEOM}            &   m       &   0.0                    \\ \hline
{\ct XYZ}          & Real Triplet           & Section~\ref{info:GEOM}            &   m       &   0.0                    \\ \hline
{\ct XYZ\_DOT}     & Real Triplet           & Section~\ref{info:GEOM}            &   m/s     &   0.0                    \\ \hline
{\ct ZVALS}        & Real                   & Section~\ref{info:GEOM}            &   m/s     &   0.0                    \\ \hline

\end{longtable}


\vspace{\baselineskip}

\section{File Formats}

\subsection{Unstructured Geometry}
\label{out:GEOMETRY}

(add to FDS user's guide 20.13 when ready)

Immersed geometric objects (generalized obstructions) are stored using a file format described below.
These objects are defined in terms of vertices, triangles and tetrahedrons.
A vertex is represented as an $(x,y,z)$ coordinate (three floating point values).
A triangle is represented as three vertex (integer) indices.
A tetrahedron is represented
as four vertex (integer) indices.
The file format allows one to specify objects that change with time.
Static geometry is defined once and displayed by Smokeview unchanged at each time step. An associated file format used to store data  on the geometric object is described in the next section.
Dynamic geometry is defined at each time step.
These files are written out from {\ct dump.f90} using lines equivalent to the following:

\begin{lstlisting}
! header

WRITE(LU_GEOM) ONE
WRITE(LU_GEOM) VERSION
WRITE(LU_GEOM) N_FLOATS, N_INTS, FIRST_FRAME_STATIC
IF (N_FLOATS>0) WRITE(LU_GEOM) (FLOAT_HEADER(I),I=1,N_FLOATS)
IF (N_INTS>0) WRITE(LU_GEOM) (INT_HEADER(I),I=1,N_INTS)

! geometry frame
! STIME ignored if first frame is static ( FIRST_FRAME_STATIC set to 1)

WRITE(LU_GEOM) STIME
WRITE(LU_GEOM) N_VERT, N_FACE, N_VOL
WRITE(LU_GEOM) HAS_SURF, HAS_MATL, HAS_TEXTURE
IF (N_VERT>0) WRITE(LU_GEOM)(Xvert(I),Yvert(I),Zvert(I),I=1,N_VERT)
IF (N_FACE>0) THEN
   WRITE(LU_GEOM) (FACE1(I),FACE2(I),FACE3(I),I=1,N_FACE)
   IF (HAS_SURF.EQ.1) WRITE(LU_GEOM) (SURF(I),I=1,N_FACE)
   IF (HAS_TEXTURE.EQ.1) WRITE(LU_GEOM) (Xtext(I),Ytext(I),I=1,3*N_FACE)
ENDIF
IF (N_VOL>0) THEN
   WRITE(LU_GEOM) (VOL1(I),VOL2(I),VOL3(I),VOL4(I),I=1,N_VOL)
   IF (HAS_MATL.EQ.1) WRITE(LU_GEOM) (MATL(I),I=1,N_VOL)
ENDIF
              .
\end{lstlisting}

\begin{itemize}
\item {\ct ONE}\ has the value 1. Smokeview uses this number to determine whether the computer creating the geometry file and the computer viewing the geometry file use the same or different byte swap (endian) conventions for storing floating point numbers.
\item {\ct VERSION}\ currently has value 2 and indicates the version number of this file format.
\item {\ct N\_FLOATS, N\_INTS} The number of floating point and integer data items stored at the beginning of the file.
\item {\ct FLOAT\_HEADER, INT\_HEADER} Floating point and integer data stored at the beginning of the file.
\item {\ct STIME} is the FDS simulation time.
\item {\ct N\_VERT, N\_FACE, N\_VOL}  are the number of vertices, faces and volumes.
\item {\ct HAS\_SURF, HAS\_MATL, HAS\_TEXTURE}\ these parameters are set to 1 if SURF, MATL or TEXTURE data is present, these parameters are set to 0 otherwise.
\item {\ct Xtext, Ytext}\ are the texture coordinates.
\item {\ct Xvert, Yvert, Zvert}\ are the vertex coordinates.
\item {\ct FACE1, FACE2, FACE3}\ are the vertex indices for each face (triangle).
    The indices range from 1 to the
    number of vertices.
\item {\ct VOL1, VOL2, VOL3, VOL4}\ are the verrtex indices for each volume (tetrahedron).  The indices are numbered from 1 to the number of vertices.
\item {\ct SURF}\ are the SURF indices for each face.
\item {\ct MATL}\ are the MATL indices for each volume.
\end{itemize}


\subsection{Unstructured Data}

\label{out:GEOMETRYDATA}

(add to FDS user's guide 20.13 when ready)

This section describes the file format used for
storing data associated with unstructured geometric objects.  This format will be used to unify file formats for storing 3D smoke, boundary, particle and slice data.
Unstructured data files are written out from {\ct dump.f90} using lines equivalent to the following:

\begin{lstlisting}

! header

WRITE(LU_GEOM) ONE
WRITE(LU_GEOM) VERSION
WRITE(LU_GEOM) N_FLOATS
IF (N_FLOATS>0) WRITE(LU_GEOM) (FLOAT_HEADER(I),I=1,N_FLOATS)
WRITE(LU_GEOM) N_INTS
IF (N_INTS>0) WRITE(LU_GEOM) (INT_HEADER(I),I=1,N_INTS)

! data for each time step

WRITE(LU_GEOM) STIME
WRITE(LU_GEOM) N_VERT, N_FACE, N_VOL
WRITE(LU_GEOM) HAS_VERT_ID
IF (N_VERT>0) THEN
   WRITE(LU_GEOM)(ValVert(I),I=1,N_VERT)
   IF (HAS_VERT_ID.EQ.1) WRITE(LU_GEOM)(Vert_ID(I),I=1,N_VERT)
ENDIF
IF (N_FACE>0) WRITE(LU_GEOM)(ValFace(I),I=1,N_FACE)
IF (N_VOLS>0) WRITE(LU_GEOM)(ValVol(I),I=1,N_VOLS)

\end{lstlisting}

\begin{itemize}
\item {\ct ONE}\ has the value 1. Smokeview uses this number to determine whether the computers creating and viewing this data file use the same or different conventions for storing floating point numbers.
\item {\ct VERSION}\ version number of the format used to store data in this file.
\item {\ct N\_FLOATS, N\_INTS} The number of floating point and integer data items stored at the beginning of the file.
\item {\ct FLOAT\_HEADER, INT\_HEADER} Floating point and integer data stored at the beginning of the file.
\item {\ct STIME} is the simulation time.
\item {\ct N\_VERT, N\_FACE, N\_VOLS}\ are the number of data values associated with vertices, faces and volumes.
\item {\ct ValVert}\ is data associated with the vertices in the unstructured geometric object.
\item {\ct ValFace}\ is data associated with the faces in the unstructured geometric object.
\item {\ct ValVol}\ is data associated with the volumes in the unstructured geometric object.
\end{itemize}

\subsection{Isosurface Data}
\label{out:ISOSURFACE}

FDS generated isosurfaces are stored using a
file format described in this section.
Iso-surface files are used to store one or more surfaces where the
surface has the value specified in the {\ct QUANTITY}\ keyword.
FDS outputs iso-surface data at fixed time intervals.
These surfaces are defined in terms of vertices and triangles.
A vertex is represented as an $(x,y,z)$ coordinate (three floating point values).
A triangle is represented as three vertex (integer) indices.
These files are written out from {\ct dump.f90} using lines equivalent to the following:
\begin{lstlisting}
! header

WRITE(LU_GEOM) ONE
WRITE(LU_GEOM) VERSION
WRITE(LU_GEOM) N_FLOATS
IF (N_FLOATS>0) WRITE(LU_GEOM) (FLOAT_HEADER(I),I=1,N_FLOATS)
WRITE(LU_GEOM) N_INTS
IF (N_INTS>0) WRITE(LU_GEOM) (INT_HEADER(I),I=1,N_INTS)

! static geometry - geometry specified once and appearing at all time steps

WRITE(LU_GEOM) N_VERT_S, N_FACE_S
IF (N_VERT_S>0)  WRITE(LU_GEOM) (Xvert_S(I),Yvert_S(I),Zvert_S(I),I=1,N_VERT_S)
IF (N_FACE_S>0)  THEN
   WRITE(LU_GEOM) (FACE1_S(I),FACE2_S(I),FACE3_S(I),I=1,N_FACE_S)
   WRITE(LU_GEOM) (SURF_S(I),I=1,N_FACE_S
ENDIF

! dynamic geometry - geometry specified and appearing for each time step

WRITE(LU_GEOM) STIME, unused
WRITE(LU_GEOM) N_VERT_D, N_FACE_D
IF (N_VERT_D>0) WRITE(LU_GEOM)(Xvert_D(I),Yvert_D(I),Zvert_D(I),I=1,N_VERT_D)
IF (N_FACE_D>0) THEN
   WRITE(LU_GEOM) (FACE1_D(I),FACE2_D(I),FACE3_D(I),I=1,N_FACE_D)
   WRITE(LU_GEOM) (SURF_D(I),I=1,N_FACE_D)
ENDIF
\end{lstlisting}

\begin{itemize}
\item {\ct ONE}\ has the value 1. Smokeview uses this number to determine whether the computer creating the geometry file and the computer viewing the geometry file use the same or different byte swap (endian) conventions for storing floating point numbers.
\item {\ct VERSION}\ has value 0 (for isosurface files but has value 1 for geometry files) and indicates the version number of this file format.
\item {\ct N\_FLOATS, N\_INTS} The number of floating point and integer data items stored at the beginning of the file.
\item {\ct FLOAT\_HEADER, INT\_HEADER} Floating point and integer data stored at the beginning of the file.
\item {\ct STIME} is the FDS simulation time.
\item {\ct N\_VERT\_S, N\_FACE\_S, N\_VERT\_D, N\_FACE\_D} are the number of static and dynamic vertices and faces.
\item {\ct Xvert\_S, Yvert\_S, Zvert\_S, Xvert\_D, Yvert\_D, Zvert\_D}\ are the static and dynamic vertex coordinates.
\item {\ct FACE1\_S, FACE2\_S, FACE3\_S, FACE1\_S, FACE2\_S, FACE3\_S}\ are the static and dynamic vertex indices for each face (triangle).
    The indices range from 1 to the number of vertices.
\item {\ct SURF\_S, SURF\_D}\ are the static and dynamic SURF indices for each face (color indices for isosurface files.
\end{itemize}

\section{Some Geometry Implementation Notes}
These implementation notes will not appear as is in an FDS or Smokeview technical manual. They are written down here to help verify various geometric algorithms coded in FDS and Smokeview.

\subsection{Spheres}
\subsubsection{Latitude/Longitude Discretization}
\subsubsection{Recursive Discretization}
Vertex locations for initial discretization
\begin{eqnarray*}
v_1&=&(0,0,1)\\
v_i&=&\left(\cos\left(2\pi\frac{i-1}{5}\right)\sqrt{5/6},\sin\left(2\pi\frac{i-1}{5}\right)\sqrt{5/6},\sqrt{1/6}\right)i=2,\cdots, 6\\
v_i&=&\left(\cos\left(2\pi\frac{i-7}{5}\right)\sqrt{5/6},\sin\left(2\pi\frac{i-7}{5}\right)\sqrt{5/6},-\sqrt{1/6}\right)i=7,\cdots, 11\\
v_{12}&=&(0,0,-1)
\end{eqnarray*}

Euler formula for relating the number vertices (V), edges (E) and faces (F) of a closed geometric surface used to discretize the sphere.
\begin{eqnarray*}
V+F-E&=&2
\end{eqnarray*}
Number of vertices, face and edges for the initial discretization.
\begin{eqnarray*}
V_0&=&12\\
F_0&=&20\\
E_0&=&30\\
\end{eqnarray*}
Recursion formulas for relating the number of vertices, faces and edges at the $n+1$'st refinement step given the corresponding number at the $n$'th refinement step.
\begin{eqnarray}
\label{eq:recursion1}
V_{n+1}&=&V_n+E_n\\
\label{eq:recursion2}
F_{n+1}&=&4 F_n\\
\label{eq:recursion3}
E_{n+1}&=&3 F_n+2 E_n
\end{eqnarray}

For $n\ge 0$ show that $V_{n+1}+F_{n+1}-E_{n+1}=2$.
Substituting $n=0$ into Eqs. (\ref{eq:recursion1}) through (\ref{eq:recursion3})  results in
\begin{eqnarray*}
V_1+F_1-E_1&=&V_0+E_0+4 F_0 -(3 F_0+2 E_0)\\
&=&12+30+80-(60+60)\\
&=&2
\end{eqnarray*}

Assuming the recursion formulas in Eqs. (\ref{eq:recursion1}) through (\ref{eq:recursion3}) and that $V_n+F_n-E_n=2$ (the induction step) results in
\begin{eqnarray*}
V_{n+1}+F_{n+1}-E_{n+1}&=&V_n+E_n+4 F_n-(3 F_n+2 E_n)\\
&=&V_n+F_n-E_n\\
&=&2
\end{eqnarray*}

Table \ref{table:refinement} gives the number of vertices, faces and edges for the 0'th through 5'th refinements.
This table will be used to verify that the sphere discretization routine, {\tt INIT\_SPHERE}, found in FDS generates the correct number of faces and vertices..
\vskip 0.1in
\par
\begin{table}
\caption{Number of vertices, faces and edges for the 0'th through 5'th discretization refinements}
\begin{tabular}{|c|c|c|c|}
  \hline
n& V&F&E\\
\hline
0&	12&	20&	30\\
1&	42&	80&	120\\
2&	162&	320&	480\\
3&	642&	1280&	1920\\
4&	2562&	5120&	7680\\
5&	10242&	20480&	30720\\
  \hline
\end{tabular}
\label{table:refinement}
\end{table}

Table \ref{table:facelist} gives the list of vertices for each face in
in the initial refinement where the vertices are numbered as given  in
Figure \ref{figure:facelist}.
\begin{table}
\caption{Vertex list for each face}
\begin{center}
\begin{tabular}{ccccc}
  \hline
  (1,2,3)&   (1, 3,4)&  (1, 4, 5)&   (1, 5, 6)&   (1, 6,2)\\
  (2,11,7)&  (2,7,3)&   (3, 7,8)&  (3, 8, 4)&   (4, 8, 9)\\
  (4, 9,5)&  (5,9,10)&   (5,10,6)& (6,10,11)&   (6,11, 2)   \\
  (11,12,7)& (7,12,8)&   (8,12,9)&   (9,12,10)&   (10,12,11)   \\
  \hline
\end{tabular}
\end{center}
\label{table:facelist}
\end{table}

\begin{figure}[\figoptions]
\begin{center}
\includegraphics[width=4.0in]{../SMV_Technical_Reference_Guide/FIGURES/icosahedron_setup}
\end{center}
\caption{Schematic of vertex locations for defining face lists}
\label{figure:facelist}
\end{figure}

\subsection{Boxes}
\subsection{2D Surfaces}

\section{Examples}
This section describes some of the test cases used to verify that FDS and Smokeview are computing
and visualizing unstructured geometric objects correctly.  This section will be used as source material for adding
more details to the various guides on how to setup unstructured geometries for use with FDS. (i.e. this section
is not intended to be included verbatim in the FDS UG).

\newcommand{\geominput}[1]{{\scriptsize\verbatiminput{../../Verification/Immersed_Boundary_Method/#1}}}
\newcommand{\figheightD}{2.75in}

\subsection{geom\_simple.fds}
Figure \ref{fig:geom_simple} was created using the following \&GEOM namelist.
The {\tt VERTS}\ keyword is used to specify 3 vertices and the {\tt FACES}\ keyword
is used to specify the indices of the triangular face.

{\scriptsize
\begin{verbatim}
&GEOM ID='geom1',
VERTS=0.0,0.0,0.0,1.0,0.5,0.0,1.0,1.0,1.0,
FACES=1,2,3,
SURF_ID='surf1'/
\end{verbatim}
}

\begin{figure}[\figoptions]
\begin{center}
\begin{tabular}{c}
 \includegraphics[width=4.0in]{SCRIPT_FIGURES/geom_simple}
  \end{tabular}
\end{center}
 \caption{A triangle created using 3 vertices and 1 face. Case: geom\_simple}
\label{fig:geom_simple}
\end{figure}

\subsection{geom\_azim.fds}
Figure \ref{fig:geom_azim} was created using the following \&GEOM namelists.
The {\tt AZIM}\ keyword is used to rotate the object about a vertical axis
centered at $(0,0,0)$.

{\scriptsize
\begin{verbatim}
&GEOM ID='geom1',
VERTS=0.0,0.0,0.0, 0.0,1.0,0.0, 0.0,1.0,1.0,
FACES=1,2,3,AZIM=0.0,SURF_ID='surf1'/

&GEOM ID='geom2',
VERTS=0.0,0.0,0.0, 0.0,1.0,0.0, 0.0,1.0,1.0,
FACES=1,2,3,AZIM=45.0,SURF_ID='surf2'/

&GEOM ID='geom3',
VERTS=0.0,0.0,0.0, 0.0,1.0,0.0, 0.0,1.0,1.0,
FACES=1,2,3,AZIM=90.0,SURF_ID='surf3'/
\end{verbatim}
}

\begin{figure}[\figoptions]
\begin{center}
\begin{tabular}{c}
 \includegraphics[width=3.0in]{SCRIPT_FIGURES/geom_azim}
  \end{tabular}
\end{center}
 \caption{Three triangles generated at different azmuthal angles using the {\tt AZIM}\ keyword. Case: geom\_azim.fds}
\label{fig:geom_azim}
\end{figure}

\subsection{geom\_elev.fds}
Figure \ref{fig:geom_elev} was created using the following \&GEOM namelists.
The {\tt ELEV}\ keyword is used to rotate the object with respect to a horizontal plane
passing through $(0,0,0)$.

{\scriptsize
\begin{verbatim}
&GEOM ID='geom1',
VERTS=0.0,0.0,1.0,1.0,0.0,1.0,1.0,1.0,1.0,
FACES=1,2,3,ELEV=0,XYZ0=0.0,0.0,1.0,SURF_ID='surf1'/

&GEOM ID='geom2',
VERTS=0.0,0.0,1.0,1.0,0.0,1.0,1.0,1.0,1.0,
FACES=1,2,3,ELEV=45,XYZ0=0.0,0.0,1.0,SURF_ID='surf2'/

&GEOM ID='geom3',
VERTS=0.0,0.0,1.0,1.0,0.0,1.0,1.0,1.0,1.0,
FACES=1,2,3,ELEV=90,XYZ0=0.0,0.0,1.0,SURF_ID='surf3'/
\end{verbatim}
}

\begin{figure}[\figoptions]
\begin{center}
\begin{tabular}{c}
 \includegraphics[width=3.0in]{SCRIPT_FIGURES/geom_elev}
  \end{tabular}
\end{center}
 \caption{Three triangles generated at different elevation angles using the {\tt ELEV}\ keyword. Case: geom\_elev.fds}
\label{fig:geom_elev}
\end{figure}

\subsection{geom\_scale.fds}
Figure \ref{fig:geom_scale} was created using the following \&GEOM namelists.
The {\tt SCALE}\ keyword is used to change the size of the object.

{\scriptsize
\begin{verbatim}
&GEOM ID='geom4',
VERTS=-2.0,0.0,0.0,-1.0,0.0,0.0,-1.0,0.0,1.0,
FACES=1,2,3,XYZ0=-2.0,0.0,0.0,SCALE=1.0,1.0,1.0,SURF_ID='surf1'/

&GEOM ID='geom4',
VERTS= 0.0,0.0,0.0,1.0,0.0,0.0,1.0,0.0,1.0,
FACES=1,2,3,XYZ0= 0.0,0.0,0.0,SCALE=1.0,1.0,2.0,SURF_ID='surf2'/

&GEOM ID='geom4',
VERTS= 2.0,0.0,0.0,3.0,0.0,0.0,3.0,0.0,1.0,
FACES=1,2,3,XYZ0= 2.0,0.0,0.0,SCALE=2.0,1.0,1.0,SURF_ID='surf3'/
\end{verbatim}
}

\begin{figure}[\figoptions]
\begin{center}
\begin{tabular}{c}
 \includegraphics[width=6.0in]{SCRIPT_FIGURES/geom_scale}
  \end{tabular}
\end{center}
\caption{Three triangles generated at different scale sizes using the {\tt SCALE}\ keyword. Case: geom\_scale.fds}
\label{fig:geom_scale}
\end{figure}

\subsection{geom\_obst.fds}
Figure \ref{fig:geom_obst} was created using the following \&GEOM namelist.
The {\tt XB}\ keyword is used the same way to specify a block as on
a {\tt \&OBST}\ or {\tt \&VENT}\ line.

{\scriptsize
\begin{verbatim}
&GEOM ID='obst',SURF_ID='surf1', XB=-0.6,0.6,-0.6,0.6,-0.2,0.2,AZIM=90.0,ELEV=30.0 /
\end{verbatim}
}

\begin{figure}[\figoptions]
\begin{center}
\begin{tabular}{c}
 \includegraphics[height=\figheightD]{SCRIPT_FIGURES/geom_obst}
  \end{tabular}
\end{center}
 \caption{A block generated using the {\tt XB}\ keyword.  The block is refined automatically to be consistent with the underlying grid resolution. Case: geom\_obst.fds}
\label{fig:geom_obst}
\end{figure}

\subsection{geom\_sphere1a.fds,...,geom\_sphere3f.fds}
Figure \ref{fig:geom_sphere} was created using {\tt LEVEL=n}\
in the following \&GEOM namelist
where {\tt n}\ ranges from 0 to 5.

{\scriptsize
\begin{verbatim}
&GEOM ID='sphere',SURF_ID='surf1',SPHERE_RADIUS=0.5,N_LEVELS=n,SPHERE_ORIGIN=0.0,0.0,0.0 /
\end{verbatim}
}

The {\tt N\_LEVELS}\ keyword is used
to specify the resolution of the sphere, larger {\tt N\_LEVELS}\ values result
in a more highly resolved sphere.{\tt LEVEL=0} produces a 20 sided sphere approximation, an icosahedron .
{\tt LEVEL=n} produces a sphere approximation with four times as many triangles as the
sphere produced with {\tt LEVEL=n-1}.
The {\tt SPHERE\_RADIUS}\ and {\tt SPHERE\_ORIGIN}\ keywords are used to specify
the size and location of the sphere.  This discretization technique results in equilateral triangles at each recursion level {(\em check to make sure this statement is true - when verified remove this comment)}.


\begin{figure}[\figoptions]
\begin{center}
\begin{tabular}{cc}
 \includegraphics[width=2.4in]{SCRIPT_FIGURES/geom_sphere1a}&
 \includegraphics[width=2.4in]{SCRIPT_FIGURES/geom_sphere1b}\\
 LEVEL=0&LEVEL=1\\
 \includegraphics[width=2.4in]{SCRIPT_FIGURES/geom_sphere1c}&
 \includegraphics[width=2.4in]{SCRIPT_FIGURES/geom_sphere1d}\\
 LEVEL=2&LEVEL=3\\
 \includegraphics[width=2.4in]{SCRIPT_FIGURES/geom_sphere1e}&
 \includegraphics[width=2.4in]{SCRIPT_FIGURES/geom_sphere1f}\\
 LEVEL=4&LEVEL=5\\
  \end{tabular}
\end{center}
 \caption{Recursive sphere discretizations.  A sphere at a given level is
 obtained by splitting each triangle from the previous level into four parts and renormalizing added vertices. Cases: geom\_sphere1a.fds,...,geom\_sphere1f.fds}
\label{fig:geom_sphere}
\end{figure}

Figure \ref{fig:geom_sphere2} was created by using {\tt N\_LAT}\ and {\tt N\_LONG}\ keywords in a {\tt \&GEOM}\ namelist
to split a sphere in latitudinal (north/south) and longitudinal (east/west) direcions respectively. The minimum values of {\tt N\_LAT}\ and {\tt N\_LONG} permitted are 3 and 6.  This discretization technique results in triangles with high aspect ratios near the poles at higher discretization levels.

\begin{figure}[\figoptions]
\begin{center}
\begin{tabular}{cc}
 \includegraphics[width=2.3in]{SCRIPT_FIGURES/geom_sphere3a}&
 \includegraphics[width=2.3in]{SCRIPT_FIGURES/geom_sphere3b}\\
 N\_LAT=3,N\_LONG=6&N\_LAT=6,N\_LONG=12\\
 \includegraphics[width=2.3in]{SCRIPT_FIGURES/geom_sphere3c}&
 \includegraphics[width=2.3in]{SCRIPT_FIGURES/geom_sphere3d}\\
 N\_LAT=12,N\_LONG=24&N\_LAT=24,N\_LONG=48\\
 \includegraphics[width=2.3in]{SCRIPT_FIGURES/geom_sphere3e}&
 \includegraphics[width=2.3in]{SCRIPT_FIGURES/geom_sphere3f}\\
 N\_LAT=48,N\_LONG=96&N\_LAT=96,N\_LONG=192\\
  \end{tabular}
\end{center}
 \caption{Latitude/Longitude sphere discretizations.  Spheres are
 split in longitudinal (east/west) and latitudinal (north/south) directions using the {\tt N\_LAT}\ and {\tt N\_LONG} keywords. Cases: geom\_sphere3a.fds,...,geom\_sphere3f.fds}
\label{fig:geom_sphere2}
\end{figure}

%\geominput{geom_terrain.fds}
\subsection{geom\_terrain.fds}
Figure \ref{fig:geom_terrain} was created using the {\tt ZVALS}\ keyword.

\begin{figure}[\figoptions]
\begin{center}
\begin{tabular}{c}
 \includegraphics[height=\figheightD]{SCRIPT_FIGURES/geom_terrain}
  \end{tabular}
\end{center}
 \caption{Elevations are defined on a rectangular array of grid points using the {\tt ZVALUES}\ keyword.  Case: gridgeom\_terrain.fds}
\label{fig:geom_terrain}
\end{figure}

\subsection{geom\_texture.fds}
Figure \ref{fig:geom_texture} was created using the following \&SURF and \&GEOM namelists.
The {\tt TEXTURE\_MAP}\ keyword on the {\tt \&SURF}\ is used to specify the name of the image
file used to texture map the geometric object.

{\scriptsize
\begin{verbatim}
&SURF ID='surf1',TEXTURE_MAP='nistleft.jpg',TEXTURE_WIDTH=0.6,TEXTURE_HEIGHT=0.2,COLOR='BLUE' /

&GEOM ID='texture',
VERTS=0.0,0.0,0.0, 1.0,0.0,0.0, 1.0,1.0,0.0,
FACES=1,2,3,SURF_ID='surf1'/
\end{verbatim}
}

\begin{figure}[\figoptions]
\begin{center}
\begin{tabular}{c}
 \includegraphics[width=4.0in]{SCRIPT_FIGURES/geom_texture}
  \end{tabular}
\end{center}
 \caption{A texture is applied to a triangle. Case: geom\_texture.fds}
\label{fig:geom_texture}
\end{figure}

\subsection{geom\_texture2.fds}
Figure \ref{fig:geom_texture2} was created using the following \&SURF and \&GEOM namelists.
The {\tt TEXTURE\_MAP}\ keyword on the {\tt \&SURF}\ is used to specify the name of the image
file used to texture map the geometric object. This example has two textures.

{\scriptsize
\begin{verbatim}
&SURF ID='surf1' TEXTURE_MAP='nistleft.jpg',TEXTURE_WIDTH=0.6,TEXTURE_HEIGHT=0.2,COLOR='BLUE' /
&SURF ID='surf2',TEXTURE_MAP='grass.jpg',TEXTURE_WIDTH=0.6,TEXTURE_HEIGHT=0.2,COLOR='GREEN' /

&GEOM ID='texture',
VERTS=0.0,0.0,0.0, 1.0,0.0,0.0, 1.0,1.0,0.0,
FACES=1,2,3,SURF_ID='surf1'/

&GEOM ID='texture2',
VERTS=0.0,0.0,0.0, 1.0,1.0,0.0, 0.0,1.0,0.0,
FACES=1,2,3,SURF_ID='surf2'/
\end{verbatim}
}

\begin{figure}[\figoptions]
\begin{center}
\begin{tabular}{c}
 \includegraphics[width=4.0in]{SCRIPT_FIGURES/geom_texture2}
  \end{tabular}
\end{center}
 \caption{Two texture maps are applied to two separate triangles.  Case: geom\_texture2.fds}
\label{fig:geom_texture2}
\end{figure}

\subsection{geom\_texture3a.fds, geom\_texture3b.fds}
Figure \ref{fig:geom_texture2} was created using the following \&SURF and \&GEOM namelists.
The {\tt TEXTURE\_MAP}\ keyword on the {\tt \&SURF}\ is used to specify the name of the image
file used to texture map the geometric object. The {\tt SPHERICAL}\ parameter
on the {\tt TEXTURE\_MAPPING}\ keyword indicates that that the texture map image
is applied to the object using spherical coordinates.


{\scriptsize
\begin{verbatim}
&SURF ID='surf1',TEXTURE_MAP='sphere_cover_01.png'COLOR='BLUE',TEXTURE_WIDTH=1.0,TEXTURE_HEIGHT=1.0/
&SURF ID='surf2'TEXTURE_MAP='sphere_cover_03.png'COLOR='RED',TEXTURE_WIDTH=1.0,TEXTURE_HEIGHT=1.0/

&GEOM ID='sphere1',SURF_ID='surf1',
N_LAT=50,N_LONG=50,SPHERE_RADIUS=0.25,SPHERE_ORIGIN=0.25,0.5,0.5,
TEXTURE_ORIGIN=0.25,0.5,0.5,TEXTURE_MAPPING='SPHERICAL' /

&GEOM ID='sphere2',SURF_ID='surf2',
N_LEVELS=3,SPHERE_RADIUS=0.25,SPHERE_ORIGIN=0.75,0.5,0.5,
TEXTURE_ORIGIN=0.75,0.5,0.5,TEXTURE_MAPPING='SPHERICAL' /
\end{verbatim}
}

\begin{figure}[\figoptions]
\begin{center}
\begin{tabular}{cc}
 \includegraphics[width=3.0in]{SCRIPT_FIGURES/geom_texture3a}&
 \includegraphics[width=3.0in]{SCRIPT_FIGURES/geom_texture3b}\\
 lat/long discretization&recursive discretization
  \end{tabular}
\end{center}
 \caption{Two texture maps applied to two separate spheres.
 The sphere on the left is discretized by splitting the sphere along longitudinal (east/west) and latitudinal (north/south) directions.
 The sphere on the right is discretized recursively starting with a 12 sided approximation to a sphere.  Cases: geom\_texture3a.fds, geom\_texture3b.fds}
\label{fig:geom_texture3}
\end{figure}

\subsection{geom\_arch.fds}
Figure \ref{fig:geom_arch} was created using VERTS and FACES keywords.

\begin{figure}[\figoptions]
\begin{center}
\begin{tabular}{c}
 \includegraphics[width=4.0in]{SCRIPT_FIGURES/geom_arch}
  \end{tabular}
\end{center}
 \caption{An arch structure created with FACES and VERTS keywords.
 Case: geom\_arch.fds . }
\label{fig:geom_arch}
\end{figure}

\subsection{geom\_time.fds}
Figure \ref{fig:geom_time} was created using the following \&GEOM namelist.
The {\tt AZIM}\ keyword is used to specify the initial rotational angle
and the {\tt AXIM\_DOT}\ is used to specify that rate at which
the azimuthal angle changes, in this  case 0.36 deg/s.

{\scriptsize
\begin{verbatim}
&GEOM ID='geom4',
VERTS=0.0,0.0,0.0,0.0,1.0,0.0,0.0,1.0,1.0,
FACES=1,2,3,AZIM=0.0,AZIM_DOT=0.36,SURF_ID='surf1'/
\end{verbatim}
}

\begin{figure}[\figoptions]
\begin{center}
\begin{tabular}{ccc}
 \includegraphics[width=2.0in]{SCRIPT_FIGURES/geom_time_050}&
 \includegraphics[width=2.0in]{SCRIPT_FIGURES/geom_time_100}&
 \includegraphics[width=2.0in]{SCRIPT_FIGURES/geom_time_150}\\
 \SI{50.0}{s}&\SI{100.0}{s}&\SI{150.0}{s}
  \end{tabular}
\end{center}
 \caption{A triangle rotating about the scene center. Case: geom\_time.fds}
\label{fig:geom_time}
\end{figure}

\subsection{geom\_time2.fds, geom\_time3.fds, geom\_time4.fds}
Figure \ref{fig:geom_time2} was created using the following \&GEOM namelists.
The {\tt AZIM}\ keyword is used to specify the initial rotational angle
and the {\tt AXIM\_DOT}\ is used to specify that rate at which
the azimuthal angle changes, in this  case 0.72 deg/s.
The {\tt blade}\ object is used as components to create four blade objects
{\tt blade1}, {\tt blade2}, {\tt blade3} and {\tt blade4} of the
{\tt prop}\  object. The blades are positioned within the {\tt prop} using the {\tt DAZIM}\ keyword. Figures \ref{fig:geom_time3} and \ref{fig:geom_time4} were created similarly using rotation keywords to incline the geometric objects.

{\tt \scriptsize
\begin{verbatim}
&GEOM ID='blade',AZIM=0.0,AZIM_DOT=0.72
VERTS=0.0,0.0,0.0,1.0,0.0,0.0,1.0,0.0,1.0,
FACES=1,2,3,COMPONENT_ONLY=.TRUE.,SURF_ID='surf4'/

&GEOM ID='blade1',GEOM_IDS(1)='blade',SURF_ID='surf1',COMPONENT_ONLY=.TRUE. /
&GEOM ID='blade2',GEOM_IDS(1)='blade',SURF_ID='surf2',COMPONENT_ONLY=.TRUE. /
&GEOM ID='blade3',GEOM_IDS(1)='blade',SURF_ID='surf3',COMPONENT_ONLY=.TRUE. /
&GEOM ID='prop' GEOM_IDS(1)='blade1',DAZIM(1)=0.0,
GEOM_IDS(2)='blade2',DAZIM(2)=90.0,
GEOM_IDS(3)='blade1',DAZIM(3)=180.0,
GEOM_IDS(4)='blade3',DAZIM(4)=270.0,
\end{verbatim}
}

\begin{figure}[\figoptions]
\begin{center}
\begin{tabular}{ccc}
 \includegraphics[width=2.0in]{SCRIPT_FIGURES/geom_time2_050}&
 \includegraphics[width=2.0in]{SCRIPT_FIGURES/geom_time2_100}&
 \includegraphics[width=2.0in]{SCRIPT_FIGURES/geom_time2_150}\\
 \SI{50.0}{s}&\SI{100.0}{s}&\SI{150.0}{s}
  \end{tabular}
\end{center}
 \caption{Four triangles defined as a group rotating about the scene center. A Case: geom\_time2.fds}
\label{fig:geom_time2}
\end{figure}

\begin{figure}[\figoptions]
\begin{center}
\begin{tabular}{ccc}
 \includegraphics[width=2.0in]{SCRIPT_FIGURES/geom_time3_050}&
 \includegraphics[width=2.0in]{SCRIPT_FIGURES/geom_time3_100}&
 \includegraphics[width=2.0in]{SCRIPT_FIGURES/geom_time3_150}\\
 \SI{50.0}{s}&\SI{100.0}{s}&\SI{150.0}{s}
  \end{tabular}
\end{center}
 \caption{Four inclined triangles defined as a group rotating and about the scene center. A Case: geom\_time3.fds}
\label{fig:geom_time3}
\end{figure}

\begin{figure}[\figoptions]
\begin{center}
\begin{tabular}{ccc}
 \includegraphics[width=2.0in]{SCRIPT_FIGURES/geom_time4_050}&
 \includegraphics[width=2.0in]{SCRIPT_FIGURES/geom_time4_100}&
 \includegraphics[width=2.0in]{SCRIPT_FIGURES/geom_time4_150}\\
 \SI{50.0}{s}&\SI{100.0}{s}&\SI{150.0}{s}
  \end{tabular}
\end{center}
 \caption{Two sets of four triangles rotating about the scene center. Each set of 4 triangles are defined as a group. Case: geom\_time4.fds}
\label{fig:geom_time4}
\end{figure}

\subsection{geom\_sphere2.fds}
Figure \ref{fig:geom_sphere2} tests motion of a scaled,  moving, rotating sphere. The sphere was scaled using {\tt SCALE=0.6,0.4,0.2}\, placed initially at the righ side of the domain using {\tt XYZ=10.0,0.5,1.0}, translated to the left and down using {\tt XYZ\_DOT=-0.01,0.0,-0.001} and rotated using {\tt GROTATE\_DOT=0.36}.  The axis of rotation was specified using {\tt GAXIS=0.0,1.0,0.0}.

\begin{figure}[\figoptions]
\begin{center}
\begin{tabular}{c}
\includegraphics[width=6.0in]{SCRIPT_FIGURES/geom_sphere2_000}\\
\includegraphics[width=6.0in]{SCRIPT_FIGURES/geom_sphere2_200}\\
\includegraphics[width=6.0in]{SCRIPT_FIGURES/geom_sphere2_400}\\
\includegraphics[width=6.0in]{SCRIPT_FIGURES/geom_sphere2_600}\\
\includegraphics[width=6.0in]{SCRIPT_FIGURES/geom_sphere2_800}\\
\includegraphics[width=6.0in]{SCRIPT_FIGURES/geom_sphere2_1000}
  \end{tabular}
\end{center}
 \caption{Motion of a scaled, moving, rotating sphere.  Case: geom\_sphere2.fds}
\label{fig:geom_sphere_fire}
\end{figure}

\end{document}


\subsection{geom\_sphere\_fire.fds}
Figure \ref{fig:geom_sphere_fire} tests visualization of a boundary file data on a sphere.

\begin{figure}[\figoptions]
\begin{center}
\begin{tabular}{cc}
 \includegraphics[width=3.5in]{SCRIPT_FIGURES/geom_sphere_fire_000}&
 \includegraphics[width=3.5in]{SCRIPT_FIGURES/geom_sphere_fire_020}\\
 \SI{0.0}{s}&\SI{2.0}{s}\\
 \includegraphics[width=3.5in]{SCRIPT_FIGURES/geom_sphere_fire_040}&
 \includegraphics[width=3.5in]{SCRIPT_FIGURES/geom_sphere_fire_060}\\
 \SI{4.0}{s}&\SI{6.0}{s}\\
 \includegraphics[width=3.5in]{SCRIPT_FIGURES/geom_sphere_fire_080}&
 \includegraphics[width=3.5in]{SCRIPT_FIGURES/geom_sphere_fire_100}\\
 \SI{8.0}{s}&\SI{10.0}{s}\\
  \end{tabular}
\end{center}
 \caption{Boundary file data visualized on a sphere.  Case: geom\_sphere\_fire.fds}
\label{fig:geom_sphere_fire}
\end{figure}

\end{document}



