\chapter{Survey of Past Validation Work}


In this  chapter, a survey of  FDS validation work  will be presented. Some of the work has been  performed at NIST, some by its grantees and some by
engineering firms using the model.  Because each organization has its  own reasons for  validating the model, the  referenced papers and reports do
not follow any particular guidelines. Some of the works only provide  a qualitative assessment  of the model,  concluding that the  model  agreement
with  a  particular  experiment  is ``good''  or ``reasonable.'' Sometimes, the conclusion is that the model works well in certain cases, not as well
in others. These studies are included in the survey because the references  are useful to other model users who may have a similar application  and
are interested in even qualitative assessment. It is important to note  that some of the papers point out flaws in early releases of FDS that have
been corrected or improved in more recent  releases. Some of  the issues raised, however,  are still subjects of  active research. The  research
agenda for FDS  is greatly influenced  by   the  feedback   provided  by  users,   often  through publication of validation efforts.


\section{Validation Work with Pre-Release Versions of FDS}

FDS was officially released in  2000. However, for two decades various CFD codes using the basic FDS hydrodynamic framework were developed at NIST
for  different applications and  for research. In the  mid 1990s, many of  these different codes were consolidated  into what eventually became FDS.
Before FDS, the various  models were referred  to as LES, NIST-LES, LES3D,  IFS (Industrial Fire Simulator), and  ALOFT (A Large Outdoor Fire Plume
Trajectory).

The  NIST LES model  describes the  transport of  smoke and  hot gases during  a fire  in an  enclosure using  the  Boussinesq approximation, where
it is assumed that the density and temperature variations in the flow                           are                          relatively
small~\cite{Rehm:1,Rehm:SIAM83,Rehm:ANM85,Rehm:IAFSS3}.     Such    an approximation  can be  applied  to a  fire  plume away  from the  fire itself.
Much of  the early  work  with this  form of  the model  was devoted  to  the  formulation of  the  low  Mach  number form  of  the Navier-Stokes
equations and  the development  of the  basic numerical algorithm.   Early validation  efforts  compared the  model with  salt water
experiments~\cite{Baum:1,McGrattan:1,Rehm:IAFSS5},   and  fire plumes~\cite{Baum:IAFSS5,Baum:2,Baum:3,Baum:4}.  Clement validated the hydrodynamic
model  in FDS by  measuring salt water flows  using Laser Induced   dye   Fluorescence~(LIF)~\cite{Clement:1}.  An   interesting finding  of this
work was  that the  transition from  a laminar  to a turbulent plume is very difficult  to predict with any technique other than DNS.

Eventually, the  Boussinesq approximation was  dropped and simulations began  to   include  more  fire-specific   phenomena.  Simulations  of
enclosure   fires   were   compared   to  experiments   performed   by Steckler~\cite{McGrattan:4}.  Mell~{\em  et al.}~\cite{Mell:1} studied small
helium  plumes, with particular attention to  the relative roles of  baroclinic torque and  buoyancy as  sources of  vorticity.  Cleary {\em  et
al.}~\cite{LES:6}  used   the  LES  model  to  simulate  the environment  seen by  multi-sensor fire  detectors and  performed some simple validation
work to check the model before using it.  Large fire experiments were performed by NIST  at the FRI test facility in Japan, and at US Naval aircraft
hangars in Hawaii and Iceland~\cite{Davis:1}. Room   airflow   applications   were   considered  by   Emmerich   and
McGrattan~\cite{Emmerich:1,Emmerich:2}.

These early validation efforts were encouraging, but still pointed out the  need  to  improve  the  hydrodynamic  model  by  introducing  the
Smagorinsky form of large eddy simulation.  This addition improved the stability  of the  model  because of  the  relatively simple  relation between
the  local strain rate  and the turbulent viscosity.  There is both   a   physical  and   numerical   benefit   to  the   Smagorinsky model.
Physically,  the viscous term used  in the model  has the right functional  form to describe  sub-grid mixing  processes. Numerically, local
oscillations in the computed  flow quantities are damped if they become  large   enough  to  threaten  the  stability   of  the  entire calculation.

During the 1980s and 1990s,  the Building and Fire Research Laboratory at NIST studied the burning of  crude oil under the sponsorship of the US
Minerals Management Service.  The aim of the work was to assess the feasibility of using burning as a means to remove spilled oil from the sea
surface. As part of the  effort, Rehm and Baum developed a special application of the LES model called ALOFT. The model was a spin-off of the
two-dimensional LES enclosure  model, in which a three-dimensional steady-state plume was computed  as a two-dimensional evolution of the lateral
wind field  generated by a large fire blown  in a steady wind. The ALOFT model is based on  large eddy simulation in that it attempts to resolve the
relevant scales of a large, bent-over plume. Validation work  was  performed  by  simulating  the plumes  from  several  large experimental burns of
crude oil in which aerial and ground sampling of smoke       particulate       was       performed~\cite{McGrattan:4a}. Yamada~\cite{ALOFT:2}
performed  a validation  of the ALOFT  model for 10~m oil  tank fire. The results  indicate that the  prediction of the plume  cross  section  500~m
from   the  fire  agree  well  with  the experimental observations.




\section{Validation of FDS since 2000}

There is an  on-going effort at NIST and elsewhere  to validate FDS as new capabilities are  added. To date, most of  the validation work has
evaluated the  model's ability  to predict the  transport of  heat and exhaust products from  a fire through an enclosure.  In these studies, the
heat release rate is usually prescribed, along with the production rates of  various products  of combustion.  More  recently, validation efforts
have moved  beyond  just transport  issues  to consider  fire growth, flame spread,  suppression, sprinkler/detector activation, and other
fire-specific phenomena.

The  validation work  discussed below  can be  organized  into several categories: Comparisons with full-scale tests conducted especially for the
chosen  evaluation,   comparisons   with  previously   published full-scale  test data,  comparisons with  standard  tests, comparisons with
documented  fire experience,  and  comparisons with  engineering correlations.  There is no single  method by which the predictions and measurements
are compared.   Formal, rigorous validation exercises are time-consuming  and  expensive.  Most  validation exercises  are  done simply to assess if
the model can be used for a very specific purpose. While  not  comprehensive on  their  own,  these studies  collectively constitute a valuable
assessment of the model.


\subsection{Comparison with Full-Scale Tests Conducted Specifically for the Chosen Evaluation}

As part of the NIST investigation  of the World Trade Center fires and collapse,  a series  of large  scale fire  experiments  were performed
specifically  to  validate  FDS~\cite{Hamins:WTC1}.   The  tests  were performed in  a rectangular  compartment 7.2~m long  by 3.6~m  wide by 3.8~m
tall.   The fires were  fueled by heptane  for some tests  and a heptane/toluene mixture  for the others.  The results of the experiments and simulations
are included in detail in this Guide.

A second set of experiments to validate FDS for use in the World Trade Center  investigation is  documented  in Ref.~\cite{Hamins:WTC2}.  The experiments
are not described as part of this Guide. The intent
of  these tests  was to  evaluate the ability  of the  model to simulate the growth  of a fire burning 3  office workstations within a compartment of
dimensions 11~m by 7~m by 4~m, open at one end to mimic the ventilation  of 5  windows similar to  those in  WTC 1 and  2. Six tests  were performed
with various  initial conditions  exploring the effect of jet fuel spray and ceiling tiles covering the surface of the desks and carpet. Measurements
were  made of the heat release rate and compartment  gas   temperatures  at  four   locations  using  vertical thermocouple arrays. Six different
material samples were tested in the NIST  cone calorimeter:  desk,  chair, paper,  computer case,  privacy panel, and  carpet. Data for the  carpet,
desk and  privacy panel were input directly into FDS, with the other 3 materials lumped together to form an  idealized fuel type.  Open burns of
single  workstations were used to  calibrate the simplified fuel  package. Then FDS  was used to make  blind  predictions  of   the  3  workstation
fires  within  the compartment. Peak  heat release rates and  temperatures were predicted to within 20~\% for all tests.


\subsection{Comparison with Engineering Correlations}

There are  several examples of  fire flows that have  been extensively studied, so much  so that a set of  engineering correlations combining the
results  of   many  experiments   have  been   developed.  These correlations are  useful to modelers because of  their simplicity. The most studied
phenomena include fire  plumes, ceiling jets,  and flame heights.

Although much  of the  early validation work  before FDS  was released involved fire plumes, it remains an active area of interest. One study by
Chow  and Yin~\cite{Chow:1}  surveys  the  performance of  various models in predicting plume temperatures and entrainment.  They compare various
correlations, a  RANS (Reynolds-Averaged Navier-Stokes) model, and FDS.  Simulations  were carried out that replicated  a 470~kW fire with a
diameter of  1~m and an  unbounded ceiling.  A  numerical grid size of  96 by  96 by  96 cells was  used in  the FDS  calculation and provided
results  that agreed  well with those  predicted by  the RANS model and the various correlations.

Battaglia~{\em  et al.}~\cite{Battaglia:1} used  FDS to  simulate fire whirls.   First,  the  model  was  shown to  reproduce  the  McCaffrey
correlation  of  a  fire  plume,   then  it  was  shown  to  reproduce qualitatively certain features  of fire whirls. At the  time, FDS used
Lagrangian elements to introduce heat  from the fire (no longer used), and this  combustion model could not replicate  the extreme stretching of the
core of the flame zone.

Quintiere and Ma~\cite{Ma:2,Ma:3} compared predicted flame heights and plume  centerline temperatures to  empirical correlations.   For plume
temperature,   the  Heskestad   correlation~\cite{SFPE:Heskestad}  was chosen.  Favorable  agreement was found  in the plume region,  but the results
near  the  flame  region  were found  to  be  grid-dependent, especially for  low $Q^*$  fires.  At this  same time,  researchers at NIST were
reaching similar  conclusions, and it  was noticed  by both teams  that a  critical parameter  for the  model is  $D^*/\dx$, where $D^*$ is the
characteristic fire diameter and $\dx$  is the grid cell size.  If  this parameter  is  sufficiently  large,  the fire  can  be considered
well-resolved  and  agreement  with various  flame  height correlations was found. If the parameter is not large enough, the fire is not
well-resolved and adjustments  must be made to  the combustion routine to account for it.



\subsection{Comparisons with Previously Published Full-Scale Test Data}
\label{prevpub}

Experiments  conducted  solely   for  model  validation  are  somewhat rare.  More common  are validation  studies  that use  data from  past
experiments.  This   section  contains  brief   descriptions  of  work published  comparing  FDS with  past  experiments  or correlations  of
experimental data.

\subsubsection{Pool Fires}


Xin~{\em et  al.}~\cite{Xin:JSS2005} used FDS to model  a 1~m diameter methane pool  fire.  The  computational domain was  2~m by 2~m  by 4~m with  a
uniform  grid  size of  2.5  cm.  The  predicted results  were compared  to   experimental  data  and  found   to  qualitatively  and quantitatively
reproduce   the  velocity  field.   The  same  authors performed a similar study of a 7.1~cm methane burner~\cite{Xin:CF2005} and a helium
plume~\cite{Xin:CS2002}.

Hostikka~{\em  et al.}~\cite{Hostikka:3} modeled  small pool  fires of methane and methanol to test  the FDS radiation solver for low-sooting fires.
They conclude that  the predicted  radiative fluxes  for both fuels  are  higher than  measured  values,  especially  at small  heat release rates,
due to an over-prediction of the gas temperature.

Hietaniemi,  Hostikka and  Vaari~\cite{Hietaniemi:1}  consider heptane pool fires of various diameters.  Predictions of the burning rate as a
function  of  diameter  follow  the  trend observed  in  a  number  of experimental studies.  Their results show an improvement  in the model over
the earlier work with  methanol fires, due to improvements in the radiation  routine  and the  fact  that  heptane  is more  sooty  than methanol,
simplifying  the treatment of radiation.   The authors point out  that reliable  predictions of  the burning  rate of  liquid fuels require roughly
twice as fine a grid spanning the burner than would be necessary to predict plume velocities and temperatures. The reason for this is  the prediction
of the heat  feedback to the  burning surface necessary to {\em predict} rather  than to {\em prescribe} the burning rate.


\subsubsection{Airflows in Non-Fire Compartments}

The low Mach number assumption in FDS is appropriate not only to fire, but to  most building  ventilation scenarios.  An  example of  how the model
can  be used  to  assess indoor  air  quality  is presented  by Musser~{\em  et  al.}~\cite{Musser:1}.   The  test compartment  was  a displacement
ventilation  test   room   that  contained   computers, furniture, and  lighting fixtures as well as  heated rectangular boxes intended to  represent
occupants.  A detailed description  of the test configuration is  given by Yuan~{\em et  al.}~\cite{Yuan:1}.  The room is ventilated with  cool
supply air introduced via  a diffuser that is mounted on a side  wall near the floor. The air rises  as it is warmed by heat sources  and exits
through a return duct  located in the upper portion  of  the  room.  The   flow  pattern  is  intended  to  remove contaminants by sweeping  them
upward at the source  and removing them from the room.  Sulphur  hexafluoride, SF$_6$, was introduced into the compartment during the  experiment as
a tracer gas  near the breathing zone  of  the   occupants.   Temperature,  tracer  concentration,  and velocity were  measured during the
experiments.   For temperature, the two finest grids (50 by 36 by  24 and 64 by 45 by 30) produced results in  which the  agreement between  the
measurement  and  prediction was considered  acceptable.  The agreement  for the  tracer concentrations were  not as  good.  It  was suggested  that
the  difference  could be related to  the way  the source  of the tracer  gas was  modeled.  The comparison  of   velocity  data  was  deemed
reasonable,  given  the limitations of the velocity probes at low velocities.

In another  study, Musser and  Tan~\cite{Musser:2} used FDS  to assess the  design  of ventilation  systems  for  facilities  in which  train
locomotives  operate.  Although there  is  only  a  limited amount  of validation, the  study is useful  in demonstrating a practical  use of FDS for
a non-fire scenario.

Mniszewski~\cite{Mniszewski:1}  used  FDS  to  model  the  release  of flammable gases in simple enclosures and open areas. In this work, the gases
were not ignited.

Kerber  and Walton  provided a  comparison between  FDS version  1 and experiments on positive pressure ventilation in a full-scale enclosure without
a fire.   The model predictions of velocity  were within 10~\% to 20~\% of the experimental values~\cite{Kerber:1}.


\subsubsection{Wind Engineering}

Most applications  of FDS involve fires within  buildings. However, it can be used to model thermal  plumes in the open and wind impinging on the
exterior   of   a   building.    Rehm,   McGrattan,   Baum   and Simiu~\cite{LES:4} use the LES solver to estimate surface pressures on simple
rectangular blocks in  a crosswind, and compare these estimates to experimental measurements.  In a subsequent paper~\cite{Rehm:WS02}, they consider
the qualitative  effects of multiple buildings and trees on a wind field.

A   different    approach   to   wind    is   taken   by    Wang   and Joulain~\cite{Wang:IAFSS2002}. They  consider a  small fire in  a wind tunnel
0.4~m wide  and  0.7~m tall  with  flow speeds  of 0.5~m/s  to 2.5~m/s.  Much  of  the  comparison with  experiment  is  qualitative, including
flame shape,  lean,  length.  They also  use  the model  to determine  the  predominant  modes  of  heat  transfer  for  different operating
conditions. To  assess  the combustion,  they implement  an ``Eddy Break-up''  combustion model~\cite{Magnussen:1} and  compare it to the mixture
fraction approach used by FDS.  The two models perform better or  worse, depending on  the operating conditions. Some  of the weaknesses of the
mixture fraction model as implemented in FDS version 2 are addressed in subsequent versions. The Eddy Break-up approach has not been implemented in
the official version of FDS.

Chang and Meroney~\cite{ChangJWE2003} compared the results of FDS with the  commercial CFD  package  FLUENT in  simulating  the transport  of
pollutants   from  steady   point  sources   in  an   idealized  urban environment.  FLUENT  employs a  variety  of  RANS (Reynolds  Averaged
Navier-Stokes)  closure  methods,   whereas  FDS  employs  large  eddy simulation (LES).   The results of the numerical  models were compared with
wind tunnel measurements within a 1:50 scale physical model of an urban street "canyon".


\subsubsection{Growing Fires}
\label{growingfires}

Vettori~\cite{Vettori:1} modeled two different fire growth rates in an obstructed ceiling geometry.  The rectangular compartment was 9.2~m by 5.6~m
by 2.4~m  with a hollow steel door to  the outside that remained closed during the tests. An open wooden stairway led to an upper floor with the same
dimensions as the fire compartment below.  Wooden joists measuring 0.038~m by 0.24~m were spaced at 0.41~m intervals across the ceiling and  were
supported  by a single  steel beam that  spanned the width of the  room.  A rectangular methane gas  burner measuring 0.7~m by 1.0~m by 0.31~m was
placed  in the corner of the chamber.  Slow and fast  burning  fires  that  reached  1055~kW  in  600  s  and  150  s, respectively,  were
monitored.   Four   vertical  arrays  of  Type  K thermocouples were used to measure temperatures during the tests.  The FDS model used four grid
refinements and piecewise linear grid spacing for each fire growth rate (slow  and fast). For the fast growing fire, the predicted  temperatures were
within  20~\% of the  measured values and within  10~\% for the slow  growing fire. In  general, finer grids produced better agreement.

In a follow-up report,  Vettori~\cite{Vettori:2} extended his study to include sloped ceilings, with  and without obstructions. He found that the
difference between  predicted and  measured  sprinkler activation times varied  between 4~\%  and 26~\% for  all cases studied.  He also noted that
FDS was able to predict the first activation of a sprinkler twice  as far  from  the fire  as  another; caused  presumably by  the re-direction of
smoke by the beams on the ceiling.

Floyd~\cite{Floyd:5,Floyd:6} validated  FDS by comparing  the modeling results with  measurements from fire tests at  the Heiss-Dampf Reaktor (HDR)
facility.  The structure was originally the containment building for a nuclear power reactor  in Germany. The cylindrical structure was 20~m in
diameter and  50~m in height  topped by a  hemispherical dome 10~m  in radius.   The building  was divided  into eight  levels.  The total  volume of
the building  was approximately  11,000~m$^3$.  From 1984  to 1991, four  fire test  series were  performed within  the HDR facility.  The T51  test
series consisted of eleven  propane gas tests and three  wood crib  tests.  To avoid  permanently damaging  the test facility, a special set of  test
rooms were constructed, consisting of a fire  room with a narrow  door, a long corridor  wrapping around the reactor vessel  shield wall, and  a
curtained area centered  beneath a maintenance  hatch.   The  fire   room  walls  were  lined  with  fire brick. The doorway and corridor walls had
the same construction as the test chamber. Six gas burners were mounted in the fire room.  The fuel source was propane gas mixed with  10~\% air fed
at a constant rate to one of the  six burners.  For comparison with the  FDS model, only the fire room, hallway and curtained  region was input into
the model, for a total of 450,000 grid cells.  The burners were defined within FDS as separate vents  with a constant  inlet velocity.  Two sets  of
burners were created, the first set at the physical location of the burners as the source  of fuel and second set  directly above the first  set as a
source for  ambient air.   The data was  presented at  fifteen minutes into  the fire.
The FDS model predicted the  layer height and temperature of the space to within 10~\% of the experimental values~\cite{Floyd:5}.

FDS predictions of fire growth and smoke movement in large spaces were presented by  Kashef~\cite{Kashef:1}.  The experiments  were conducted at the
National Research Council  Canada.  The tests were performed in a compartment with  dimensions of 9~m by 6~m by  5.5~m with 32 exhaust inlets and a
single supply fan.  A burner  generated fires ranging in size  from 15~kW  to 1000~kW.   FDS produced  good predictions  of the experimental layer
temperatures and interface heights,  but there was some disagreement in the shape of the temperature profiles.


\subsubsection{Flame Spread}
\label{flame spread}

FDS was  evaluated to predict  the heat transfer  to the wall  from an adjacent pool fire.   The experimental results were based  on the work by Beck
{\em et al.}  The  predicted heat flux was  in agreement with the experimental  results.  The temperatures  are within 30~\%  of the measured values
near the base of  the wall but  decrease more rapidly than  the  experimental  measurements.   The  difference  between  the experimental and
predicted values  can be attributed to the combustion model within FDS.

The flame spread  calculations from FDS were compared  to the vertical flame  spread over  a 5~m  slab of  PMMA performed  by  Factory Mutual
Research  Corporation (FMRC).   The  predicted flame  spread rate  was within  0.3~m/s  for any  point  in  time  during the  analysis.   The
comparison at  the quasi-steady  burning rate once  the full  slab was burning     shows    that     FDS    over-estimated     the    burning
rate~\cite{Ma:2,Ma:3}.

A   charring  model   was   implemented  in   FDS   by  Hostikka   and McGrattan~\cite{Hostikka:2}.  The model  is a  simplification  of work done at
NIST by Ritchie  {\em et al.}~\cite{Ritchie:1}.  The charring model was  first used to  predict the burning  rate of a  small wooden sample in the
cone calorimeter.  The results were  more favorable for higher imposed heat fluxes. For  low imposed fluxes, the heat transfer at the edge  of the
sample was more pronounced,  and more difficult to model  accurately.   Full-scale room  tests  with  wood paneling  were modeled, but  the results
were  judged to be grid-dependent.  This was likely a consequence of the  gas phase spatial resolution, rather than the solid phase. The authors
concluded that it is difficult to predict the growth rate of a fire  in a wood-lined room without ``tuning'' the pyrolysis rate  coefficients. For
real  wood products, it  is unlikely that all  of the  necessary properties can  be obtained  easily. Thus, grid sensitivity  and uncertain  material
properties make  {\em blind} predictions of fire  growth on real materials beyond  the reach of the current version of the model. However, the model
can still be used for a qualitative assessment  of fire behavior as long  as the uncertainty in the flame spread rate is recognized.


\subsubsection{Response of Active and Passive Fire Protection}

A   significant  validation  effort   for  sprinkler   activation  and suppression was  a project entitled the  International Fire Sprinkler, Smoke
and Heat Vent, Draft  Curtain Fire Test Project organized by the National   Fire  Protection   Research  Foundation~\cite{McGrattan:5}. Thirty-nine
large scale  fire  tests were  conducted at  Underwriters Laboratories in  Northbrook, IL.  The  tests were aimed  at evaluating the performance of
various  fire protection systems in large buildings with  flat ceilings, like  warehouses and  ``big box''  retail stores. All the  tests were
conducted  under a 30~m by  30~m adjustable-height platform in a 37~m by 37~m by 15~m high test bay. At the time, FDS had not been publicly released
and  was referred to as the Industrial Fire Simulator (IFS), but it was essentially the same as FDS version 1.

For  model  validation  of  sprinkler activation,  the  most  valuable experiments performed were a series  of heptane spray fires.  With the spray
burner in different  locations, with and without draft curtains, with  and  without  vertical  vents,  the model  made  predictions  of sprinkler
activation and upper layer temperatures.  For all tests, the first ring  of sprinklers surrounding the fire  activated within 15~\% of the
experimental  times; within 25~\% for the  second ring. The gas temperatures near the ceiling were  predicted to within about 15~\% of the measured
values.

Most of the full-scale experiments performed during the project used a heptane  spray  burner  to   generate  controlled  fires  of  1~MW  to 10~MW.
However, 5  experiments  were performed  with  6~m high  racks containing the  Factory Mutual Standard Plastic Commodity,  or Group A Plastic. To
model these  fires, bench scale experiments were performed to characterize the burning behavior of the commodity, and larger test fires  provided
validation  data  with  which  to   test  the  model predictions    of    the     burning    rate    and    flame    spread
behavior~\cite{Hamins:1,Hamins:IAFSS2002}.     Two   to    four   tier configurations  were  evaluated.  For  the  period  of  time prior  to
application of water, the simulated heat release rate was within 20~\% of the experimental  heat release rates.  It should  be noted that the model
was very  sensitive to the thermal parameters  and the numerical grid when used to model the fire growth in the piled commodity tests.


High rack storage fires of pool chemicals were modeled by Olenick~{\em et  al.}~\cite{Olenick:1}  to  determine  the  validity  of  sprinkler
activation predictions  of FDS.  The model was  compared to full-scale fires conducted  in January, 2000  at Southwest Research  Institute in San
Antonio,  Texas.  The results indicated that  the model accurately predicted sprinkler activation and the over-pressurization of the test
compartment.

FDS  has  been  used  to  study  the behavior  of  a  fire  undergoing suppression     by     a    water     mist     system.     Kim     and
Ryou~\cite{Kim:BE2003,Kim:IJACR2004}   compared  FDS   predictions  to results of compartment fire tests  with and without the application of a water
mist. The cooling and oxygen dilution were predicted to within about 10~\% of the measurements, but the simulations failed to predict the complete
extinguishment of a hexane pool fire. The authors suggest that this is a result of the combustion model rather than the spray or droplet model.

Another study  of water  mist suppression using  FDS was  conducted by Hume   at   the    University   of   Canterbury,   Christchurch,   New
Zealand~\cite{Hume:Masters}. Full-scale  experiments were performed in which a fine  water mist was combined with  a displacement ventilation system
to protect occupants and electrical equipment in the event of a fire.  Simulations of  these experiments  with FDS  showed qualitative agreement, but
the version of the  model used in the study (version 3) was not able  to predict accurately the decrease  in heat release rate of the fire.

Hostikka    and   McGrattan~\cite{Hostikka:FSJ2006}    evaluated   the absorption of  thermal radiation by water sprays.  They considered two sets of
experimental data and concluded  that FDS has  the ability to predict the  attenuation of thermal radiation  ``when the hydrodynamic interaction
between   the  droplets  is   weak.''  However,  modeling interacting sprays would require a more costly coalescence model. They also note that  the
results of the model were  sensitive to grid size, angular discretization, and droplet sampling.

\subsubsection{Airflows in Fire Compartments}

Cochard~\cite{Cochard:1} used  FDS to  study the ventilation  within a tunnel. He  compared the model  results with a full-scale  tunnel fire
experiment conducted  as part of the  Massachusetts Highway Department Memorial Tunnel Fire Ventilation  Test Program.  The test consisted of a
single  point supply of  fresh air through  a 28~m$^2$ opening  in a 135~m tunnel.  The ventilation was started 2 min after the ignition of a  40~MW
fire.  Fifteen  temperature  measurement  trees were  placed within  the  tunnel and  replicated  within  the  model. Depending  on location,  the
difference between  predicted and  measured temperature rise ranged from 10~\% to 20~\%.

McGrattan and Hamins~\cite{McGrattan:HST} also applied FDS to simulate two of the Memorial Tunnel Fire Tests as validation for the use of the model
in  studying  an  actual  fire in  the  Howard  Street  Tunnel, Baltimore,  Maryland,  July  2001.  The  experiments  chosen  for  the comparison
were unventilated. One  experiment was  a 20~MW  fire; the other a 50~MW fire.  FDS predictions of peak near-ceiling temperatures were within
50~$^\circ$C of the measured peak temperatures, which were 600~$^\circ$C and 800~$^\circ$C, respectively.

Friday studied the  use of FDS in large  scale mechanically ventilated spaces.   The ventilated  enclosure  was provided  with air  injection rates
of  1 to 12 air  changes per hour  and a fire with  heat release rates ranging  from 0.5~MW to  2~MW.  The test measurements  and model output were
compared to assess the accuracy of FDS~\cite{Friday:1}.

Zhang {\em  et al.}~\cite{Zhang:2} utilized  the FDS model  to predict turbulence characteristics  of the flow and temperature  fields due to fire
in a compartment.   The experimental  data was  acquired through tests that replicated a half-scale ISO Room Fire Test.  Two cases were explored:
the heat  source in  the center  of the  room and  the heat source adjacent to a wall.  The heat source was a heating element with an output of
12~kW/m$^2$ and was  assumed stable after 300 s.  For the first case,  the predicted  average velocity and  temperature profiles were found to
``agree reasonably  well.''  Near the ceiling, the model under-predicted   temperature   and   over-predicted  velocity.    The predicted  intensity
of  the temperature  fluctuation  ``agree[d] very well'' at  all points  except those directly  adjacent to  the burner. The turbulent heat flux was
found to be larger in the region above the heat source.

The  second case also  used a  burner with  a 12~kW/m$^2$  heat source located  at the  wall.  As  with the  first case,  the  predicted mean
velocities  agree  with  the  experimental  results  except  near  the ceiling.   The  temperatures  near   the  ceiling  were  found  to  be
over-predicted by FDS.  The  intensity of the velocity fluctuation was found to  ``agree well''  with the experimental  data except  near the
ceiling.   The  predicted  intensity  of the  temperature  fluctuation agrees ``very well''  with the experimental data except  in the region near
the  middle of the room.  This  might be due to  the influence of the door  sill.  Overall, in  both cases, the predicted  values agreed well  with
the  experimental values  in  all regions  except near  the ceiling.


The ability of  version 1 of FDS to  accurately predict smoke detector activation was studied by D'Souza~\cite{DSouza:1}. The smoke transport model
within FDS  was tested and compared with UL  217 test data.  The second step  in this research was  to further validate  the model with full-scale
multi-compartment fire tests.  The results  indicated that FDS is capable of predicting  smoke detector activation when used with smoke  detector
lag correlations  that  correct  for  the time  delay associated with smoke having to penetrate the detector housing.

Another study of smoke detector  activation was carried out by Brammer at  the University  of Canterbury,  New  Zealand~\cite{Brammer:1}. Two fire
tests from  a series  performed  in a  two-story residence  were simulated, and  smoke detector  activation times were  predicted using three
different methods. The methods consisted of either a temperature correlation,  a time-lagged  function  of the  optical  density, or  a thermal
device much like a heat detector.  The purpose was to identify ways to reliably predict smoke detector activation using typical model output like
temperature and  smoke concentration. It was remarked that simulating  the  early stage  of  the  fire  is critical  to  reliable prediction.

Cleary~\cite{Cleary:1} also provided a comparison between FDS computed gas  velocity,  temperature  and  concentrations at  various  detector
locations.   The research concluded  that multi-room  fire simulations with the FDS model can accurately predict the conditions that a sensor might
experience during a real fire  event.  The FDS model was able to predict the smoke and gas concentrations, heat, and flow velocities at various
detector locations to within 15~\% of measurements.

Piergoirgio~{\em  et al.}~\cite{Piergiorgio:1} provided  a qualitative analysis of FDS applied to a  truck fire within a tunnel.  The goal of their
analysis  was to describe the  spread of the  toxic gases within the tunnels, to determine the  places not involved in the spreading of combustion
products  and to quantify  the oxygen, carbon  monoxide and hydrochloric acid concentrations during the fire.

Edwards~{\em  et al.}~\cite{Edwards:SME2005,Edwards:FSJ}  used  FDS to determine the critical air velocity  for smoke reversal in a tunnel as a
function of  the fire intensity, and his  results compared favorably with   experimental  results.   In  a   further  study,   Edwards  and
Hwang~\cite{Edwards:SME2006}  applied FDS to  study fire  spread along combustibles in  a ventilated mine  entry. Analyses such as  these are
intended for planning and implementation of ventilation changes during mine fire fighting and rescue operations.


\subsubsection{Combustion Model}

Floyd~{\em et al.}~\cite{Floyd:1,Floyd:6} compared the radiation model of  FDS version 2  with full-scale  data from  the Virginia  Tech Fire
Research Laboratory (VTFRL).  The  test compartment was outfitted with equipment   capable   of  taking   temperature,   air  velocity,   gas
concentrations, unburned hydrocarbon  and heat flux measurements.  The test facility consisted of  a single compartment geometrically similar to the
ISO 9705 standard compartment with dimensions of 1.2~m by 1.8~m by  1.2~m  in height.   The  ceiling  and  walls were  constructed  of fiberboard
over  a steel  shell  with  a  floor of  concrete.   Three baseline experiments  were completed with  fires ranging in  size from 90~kW  to 440~kW.
Overall,  FDS  predicted  the  temperatures  to within  15~\%  of  the measured  temperatures.  The  FDS velocity  measurements  followed the trend
of  the test  data  but did  not  replicate  it.  The  outgoing velocities  were under-predicted by  30~\% to  40~\% and  the incoming velocities
were  over-predicted by 40~\%. FDS predicted  the heat flux gauge response to within 10~\%  of the measured values.  The radiation model  in FDS
predicted the  measured  fluxes to  within 15~\%.

Xin   and  Gore~\cite{Xin:JSS2003}   compared   FDS  predictions   and measurements  of the  spectral radiation  intensities of  small fires. The
fuel flow rates for  methane and ethylene burners were selected so that the  Froude numbers  matched that of  liquid toluene  pool fires. The heat
release rate was 4.2~kW  for the methane flame and 3.4~kW for the ethylene flame.  Line of sight spectral radiation intensities were measured  at
six  downstream  locations.    The  spectral  radiation intensity calculations were performed by post-processing the transient scalar  distributions
provided  by FDS.   The calculated  and measured spectral  radiation  intensities were  found  to  be in  ``excellent'' agreement for the gas
radiation bands.

Zhang~{\em et al.}~\cite{Zhang:1} compared the experimental results of a  circular methane  gas burner  to predictions  computed by  FDS. The
compartment was 2.8~m by 2.8~m  by 2.2~m high with natural ventilation from a  standard door.  Good agreement was  found for  the temperature
prediction at the doorway where  the radiation model was used. The FDS model predicted the temperatures at the corner of the room better than other
models  compared by the group.  It was found  that, overall, FDS predicted temperatures  well but the prescribed  turbulent Prandtl and Schmidt
numbers play an important  role in determining the accuracy of the model.


Bundy,  Dillon and  Hamins~\cite{Dillon:1,Hamins:FPE2005}  studied the use of FDS  in providing data and correlations  for fire investigators to
support their investigations.   A paraffin  wax candle  was placed within  a  0.61~m by  0.61~m  by  0.76~m  plexi-glass enclosure.   The chamber was
raised 20~mm off the surface to reveal 44 uniformly spaced 6~mm diameter holes.   The holes provided oxygen to  the flame without subjecting the
flame to a draft.   A 150~mm hole was  provided at the top of the enclosure to allow  for the heat and combustion products to exit the space.  The
heat flux  from the candle flame was modeled with FDS.  The model  provides a prediction of the heat  flux of the candle at a height of  56~mm above
the base of the flame  with an accuracy of 5~\%. The flux is under predicted  by 16~\% at 76~mm above the base of the  flame. The remainder  of the
predictions show  flux measurements were under-predicted by 15~\% to 40~\% of the measured values.



\subsection{Comparison with Standard Tests}

Standard fire tests are  performed at various testing laboratories and universities  around  the world.   While  most  were  not designed  as
validation tools, they nevertheless  can be used as relatively simple, well characterized fire experiments.

An extensive  amount of  validation work with  FDS version 4  has been performed    by   Hietaniemi,    Hostikka,   and    Vaari    at   VTT,
Finland~\cite{Hietaniemi:1}.  The  case studies are  comprised of fire experiments ranging  in scale  from the cone  calorimeter (ISO~5660-1, 2002)
to full-scale fire tests such as the room corner test (ISO~9705, 1993).  Comparisons are  also  made  between FDS  4  results and  data obtained  in
the  SBI (Single  Burning Item)  Euro-classification test apparatus (EN  13823, 2002) as  well as data  obtained in two  {\em ad hoc} experimental
configurations: one is  similar to the  room corner test but  has only partial linings and  the other is a  space to study fires in building
cavities. In the study of upholstered furniture, the experimental configurations  are the cone  and furniture calorimeters, and the  ISO room. For
liquid  pool fires, comparison is  made to data obtained  by  numerous  researchers.   The burning  materials  include spruce  timber, MDF  (Medium
Density  Fiber) board,  PVC  wall carpet, upholstered furniture, cables with plastic sheathing, and heptane.

The scope of the VTT work is considerable. Assessing the accuracy of the model must be done on a case by case basis. In some cases, predictions of
the burning rate of the material were based solely on its fundamental properties, as in the heptane pool fire simulations. In other cases, some
properties of the material are unknown, as in the spruce timber simulations. Thus, some of the simulations are true predictions, some are
calibrations. The intent of the authors was to provide guidance to engineers using the model as to appropriate grid sizes and material properties. In
many cases, the numerical grid was made fairly coarse to account for the fact that in practice, FDS is used to model large spaces of which the fuel
may only comprise a small fraction.


\subsection{Comparison with Documented Fire Experience}

Documented fire experience includes known behavior of materials in fires, eyewitness accounts of real fires, observed post fire conditions, and other
means. To date, several actual fires have been reconstructed using FDS. One case study performed by NIST is documented in Ref.~\cite{Madrzykowski:1}.
Two fire fighters were killed and one severely injured in a townhouse fire in Washington, D.C. during the evening of May 30, 1999. Questions arose
about the injuries the fire fighters had sustained, the lack of thermal damage in the living room where a fallen fire fighter was found and why the
fire fighters never opened their hose lines to protect themselves or to extinguish the fire.

To answer some of the questions, a rectangular volume of 10~m by 6~m by 5.1~m was divided into 76,500 cells in the FDS model. The FDS results that
best replicated the observed fire behavior indicated that the opening of the basement sliding glass door provided oxygen to a pre-heated,
under-ventilated fire. Flashover was estimated to occur in less than 60 s following the entry of fire fighters into the basement. The resulting fire
gases flowed up the basement stairs and moved across the living room ceiling towards the back wall of the townhouse. These hot gases came in direct
contact with the fire fighters who were killed. The hot gases traversed the townhouse in less than 2 s, giving the fire fighters little time to
respond. The model showed that the oxygen level was too low to support flaming and, therefore, the fire fighters did not have a visual cue of the
thermal conditions until it was too late. Results of the FDS study were shared with the D.C. fire department and have been made available via a
multi-media CD-ROM to other fire departments across the country.

Another case study performed at NIST involved a fire in a Houston restaurant~\cite{Texas}. On the morning of February 14, 2000, a fire started in the
office area of a fast food restaurant. Two fire fighters died when the roof collapsed. The FDS model was used to simulate the fire. The fuel was
assumed to be the contents of a typical office, and the fire was assumed to have a slowly growing heat release rate peaking at 6~MW. Multiple vents
were modeled and the time at which they opened replicated the fire fighters' actions after arrival.  The model provided a visual representation of
the fire during the initial phases until the collapse of the roof.

NIST also performed a case study on a fire that killed three children and three fire fighters on the morning of December 22, 1999~\cite{Iowa}. The
fire started on top of a stove in a two-story residence. FDS was used to simulate the fire.  The fuel packages consisted of several furniture items
in the kitchen and living room with heat release rates reaching 5.2~MW. The model results indicated the critical event in the fire was flashover of
the kitchen. The fire became a multi-room event after flashover with temperatures increasing to over 600~$^\circ$C. The hot gases spread quickly from
the living room to the stairway on the second floor trapping the fire fighters.

Outside of NIST, FDS has been used to investigate many actual fires, but very few of these studies are documented in the literature. Exceptions
include a study by Rein~{\em et al.}~\cite{Rein:Interflam2004} looking at several fire events using an analytical fire growth model, the NIST zone
model CFAST, and FDS. A similar study was performed several years earlier by Spearpoint {\em et al.}~\cite{Spearpoint:ICFRE3} as a class exercise at
the University of Maryland. During the SFPE Professional Development Week in the fall of 2001, a workshop was held in which several engineers related
their experiences using FDS as a forensic tool~\cite{Carpenter:SFPE2001}. The role of carbon monoxide in the deaths of three fire fighters was
studied by Christensen and Icove~\cite{Christensen:JFS}. There is little quantitative validation of the model afforded by these studies. However, the
degree to which the model is able to reproduce observed behavior can be used as an indicator of the model's strengths and
