\chapter{Gas Species and Smoke}

For most applications, FDS uses a single step chemical reaction whose products are tracked via
a two-parameter mixture fraction model.  The mixture fraction is a conserved
scalar quantity that represents the mass fraction of one or more components of the gas at
a given point in the flow field.  By default, two components of the mixture fraction are explicitly
computed. The first is the mass fraction of unburned fuel and
the second is the mass fraction of burned fuel (i.e. the mass of the combustion products
that originated as fuel). When the default model is used, O$_2$, CO$_2$ and smoke concentrations are obtained
from the explicitly computed mixture fraction variables. Their yields are specified by the user and do not
change. 

FDS has an optional two-step chemical reaction with a three parameter
mixture fraction decomposition with the first step being oxidation of fuel
to carbon monoxide and the second step the oxidation of carbon monoxide to carbon dioxide.
The three mixture fraction components for the two step reaction
are unburned fuel, mass of fuel that has completed the first reaction step, and the mass
of fuel that has completed the second reaction step.  The mass fractions of all of the major
reactants and products can be derived from the mixture fraction parameters by means of
``state relations.'' Examples of this more detailed model can be found later in this chapter.




\section{WTC and NIST/NRC Test Series, Oxygen and CO$_2$}

The following pages present comparisons of oxygen and carbon dioxide concentration predictions and measurements for the
WTC and NIST/NRC series. In the WTC tests, there was only one measurement of each made near the ceiling of the compartment roughly 2~m from the
seat of the fire. In the NIST/NRC tests, there were two oxygen measurements, one in the upper layer, one in the lower.  There was only one carbon
dioxide measurement in the upper layer.

\begin{figure}[p]
\begin{tabular*}{\textwidth}{l@{\extracolsep{\fill}}r}
\includegraphics[height=2.2in]{FIGURES/WTC/WTC_01_v5_Oxygen} &
\includegraphics[height=2.2in]{FIGURES/WTC/WTC_02_v5_Oxygen} \\
\includegraphics[height=2.2in]{FIGURES/WTC/WTC_03_v5_Oxygen} &
\includegraphics[height=2.2in]{FIGURES/WTC/WTC_04_v5_Oxygen} \\
\includegraphics[height=2.2in]{FIGURES/WTC/WTC_05_v5_Oxygen} &
\includegraphics[height=2.2in]{FIGURES/WTC/WTC_06_v5_Oxygen}
\end{tabular*}
\label{NIST_WTC_Oxygen}
\end{figure}


\begin{figure}[p]
\begin{tabular*}{\textwidth}{l@{\extracolsep{\fill}}r}
\includegraphics[height=2.2in]{FIGURES/NIST_NRC/NIST_NRC_01_v5_Oxygen} &
\includegraphics[height=2.2in]{FIGURES/NIST_NRC/NIST_NRC_07_v5_Oxygen} \\
\includegraphics[height=2.2in]{FIGURES/NIST_NRC/NIST_NRC_02_v5_Oxygen} &
\includegraphics[height=2.2in]{FIGURES/NIST_NRC/NIST_NRC_08_v5_Oxygen} \\
\includegraphics[height=2.2in]{FIGURES/NIST_NRC/NIST_NRC_04_v5_Oxygen} &
\includegraphics[height=2.2in]{FIGURES/NIST_NRC/NIST_NRC_10_v5_Oxygen} \\
\includegraphics[height=2.2in]{FIGURES/NIST_NRC/NIST_NRC_13_v5_Oxygen} &
\includegraphics[height=2.2in]{FIGURES/NIST_NRC/NIST_NRC_16_v5_Oxygen}
\end{tabular*}
\label{NIST_NRC_Gas_Closed}
\end{figure}

\begin{figure}[p]
\begin{tabular*}{\textwidth}{l@{\extracolsep{\fill}}r}
\includegraphics[height=2.2in]{FIGURES/NIST_NRC/NIST_NRC_17_v5_Oxygen} &
 \\
\includegraphics[height=2.2in]{FIGURES/NIST_NRC/NIST_NRC_03_v5_Oxygen} &
\includegraphics[height=2.2in]{FIGURES/NIST_NRC/NIST_NRC_09_v5_Oxygen} \\
\includegraphics[height=2.2in]{FIGURES/NIST_NRC/NIST_NRC_05_v5_Oxygen} &
\includegraphics[height=2.2in]{FIGURES/NIST_NRC/NIST_NRC_14_v5_Oxygen} \\
\includegraphics[height=2.2in]{FIGURES/NIST_NRC/NIST_NRC_15_v5_Oxygen} &
\includegraphics[height=2.2in]{FIGURES/NIST_NRC/NIST_NRC_18_v5_Oxygen}
\end{tabular*}
\label{NIST_NRC_Gas_Open}
\end{figure}


\begin{figure}[p]
\begin{tabular*}{\textwidth}{l@{\extracolsep{\fill}}r}
\includegraphics[height=2.2in]{FIGURES/WTC/WTC_01_v5_CO2} &
\includegraphics[height=2.2in]{FIGURES/WTC/WTC_02_v5_CO2} \\
\includegraphics[height=2.2in]{FIGURES/WTC/WTC_03_v5_CO2} &
\includegraphics[height=2.2in]{FIGURES/WTC/WTC_04_v5_CO2} \\
\includegraphics[height=2.2in]{FIGURES/WTC/WTC_05_v5_CO2} &
\includegraphics[height=2.2in]{FIGURES/WTC/WTC_06_v5_CO2}
\end{tabular*}
\label{NIST_WTC_CO2}
\end{figure}

\begin{figure}[p]
\begin{tabular*}{\textwidth}{l@{\extracolsep{\fill}}r}
\includegraphics[height=2.2in]{FIGURES/NIST_NRC/NIST_NRC_01_v5_CO2} &
\includegraphics[height=2.2in]{FIGURES/NIST_NRC/NIST_NRC_07_v5_CO2} \\
\includegraphics[height=2.2in]{FIGURES/NIST_NRC/NIST_NRC_02_v5_CO2} &
\includegraphics[height=2.2in]{FIGURES/NIST_NRC/NIST_NRC_08_v5_CO2} \\
\includegraphics[height=2.2in]{FIGURES/NIST_NRC/NIST_NRC_04_v5_CO2} &
\includegraphics[height=2.2in]{FIGURES/NIST_NRC/NIST_NRC_10_v5_CO2} \\
\includegraphics[height=2.2in]{FIGURES/NIST_NRC/NIST_NRC_13_v5_CO2} &
\includegraphics[height=2.2in]{FIGURES/NIST_NRC/NIST_NRC_16_v5_CO2}
\end{tabular*}
\end{figure}

\begin{figure}[p]
\begin{tabular*}{\textwidth}{l@{\extracolsep{\fill}}r}
\includegraphics[height=2.2in]{FIGURES/NIST_NRC/NIST_NRC_17_v5_CO2} &
 \\
\includegraphics[height=2.2in]{FIGURES/NIST_NRC/NIST_NRC_03_v5_CO2} &
\includegraphics[height=2.2in]{FIGURES/NIST_NRC/NIST_NRC_09_v5_CO2} \\
\includegraphics[height=2.2in]{FIGURES/NIST_NRC/NIST_NRC_05_v5_CO2} &
\includegraphics[height=2.2in]{FIGURES/NIST_NRC/NIST_NRC_14_v5_CO2} \\
\includegraphics[height=2.2in]{FIGURES/NIST_NRC/NIST_NRC_15_v5_CO2} &
\includegraphics[height=2.2in]{FIGURES/NIST_NRC/NIST_NRC_18_v5_CO2}
\end{tabular*}
\end{figure}


\begin{figure}[p]
\begin{center}
\begin{tabular}{c}
\includegraphics[width=4.0in]{FIGURES/ScatterPlots/Carbon_Dioxide_Concentration} \\
\vspace{0.25in} \\
\includegraphics[width=4.0in]{FIGURES/ScatterPlots/Oxygen_Concentration}\\
\vspace{0.25in}
\end{tabular}
\end{center}
\caption{Summary of Gas Species Results.}
\end{figure}

\clearpage


\section{NIST/NRC Test Series, Smoke}

FDS treats smoke like all other combustion products, basically a tracer gas whose mass fraction is a function of the mixture fraction.
To model smoke movement, the user need only prescribe the smoke yield, that is, the fraction of the fuel mass that is
converted to smoke particulate.  For the simulations of the NIST/NRC tests, the smoke yield is specified as one of the test parameters.
Figure and Figure contain comparisons of measured and predicted smoke concentration at one measuring station in the upper layer.
There are two obvious trends in the figures: first, the predicted concentrations are about 50 \% higher than the measured
in the open door tests.  Second,
the predicted concentrations are roughly three times the measured concentrations in the closed door tests.
As a contrast, Figure displays the time history of CO concentration for 6 of the NIST/NRC tests.
Like smoke, the CO is specified in FDS via a fixed yield, measured along with smoke and reported in the test document.
The large differences between model and measurement seen in the smoke data do not appear in the CO data.

\begin{figure}[p]
\begin{tabular*}{\textwidth}{l@{\extracolsep{\fill}}r}
\includegraphics[height=2.2in]{FIGURES/NIST_NRC/NIST_NRC_01_v5_Smoke} &
\includegraphics[height=2.2in]{FIGURES/NIST_NRC/NIST_NRC_07_v5_Smoke} \\
\includegraphics[height=2.2in]{FIGURES/NIST_NRC/NIST_NRC_02_v5_Smoke} &
\includegraphics[height=2.2in]{FIGURES/NIST_NRC/NIST_NRC_08_v5_Smoke} \\
\includegraphics[height=2.2in]{FIGURES/NIST_NRC/NIST_NRC_04_v5_Smoke} &
\includegraphics[height=2.2in]{FIGURES/NIST_NRC/NIST_NRC_10_v5_Smoke} \\
\includegraphics[height=2.2in]{FIGURES/NIST_NRC/NIST_NRC_13_v5_Smoke} &
\includegraphics[height=2.2in]{FIGURES/NIST_NRC/NIST_NRC_16_v5_Smoke}
\end{tabular*}
\end{figure}

\begin{figure}[p]
\begin{tabular*}{\textwidth}{l@{\extracolsep{\fill}}r}
\includegraphics[height=2.2in]{FIGURES/NIST_NRC/NIST_NRC_17_v5_Smoke} &
 \\
\includegraphics[height=2.2in]{FIGURES/NIST_NRC/NIST_NRC_03_v5_Smoke} &
\includegraphics[height=2.2in]{FIGURES/NIST_NRC/NIST_NRC_09_v5_Smoke} \\
\includegraphics[height=2.2in]{FIGURES/NIST_NRC/NIST_NRC_05_v5_Smoke} &
\includegraphics[height=2.2in]{FIGURES/NIST_NRC/NIST_NRC_14_v5_Smoke} \\
\includegraphics[height=2.2in]{FIGURES/NIST_NRC/NIST_NRC_15_v5_Smoke} &
\includegraphics[height=2.2in]{FIGURES/NIST_NRC/NIST_NRC_18_v5_Smoke}
\end{tabular*}
\end{figure}



\begin{figure}[p]
\begin{center}
\begin{tabular}{c}
\includegraphics[width=4.0in]{FIGURES/ScatterPlots/Smoke_Concentration} \\
\vspace{0.25in} \\
\end{tabular}
\end{center}
\caption{Summary of Smoke Results.}
\end{figure}



\clearpage

\section{Smyth Slot Burner Experiment}

The two-step, CO production model in FDS was used to simulate a methane/air slot burner diffusion flame.  Figure~\ref{WP_CH4_Temp} shows predicted and
measured temperatures at three elevations above the burner.  The model predicts a flame that is slightly narrower
(5.5~mm vs. 6.5~mm or a 15~\% error) and cooler (1700~$^\circ$C vs. 1800~$^\circ$C or a 5~\% error) than measured.  The model
also predicts higher centerline temperatures.  These results are not surprising.  The two-step combustion model considers
the first step, F~$\longrightarrow$~CO, to be infinitely fast, assuming that the local oxygen concentration satisfies a
flammability criterion.  This is true in the vicinity of the lip of the burner.  In reality, the cold fuel and air
streams do not react infinitely fast at this location and some oxygen penetrates the flame at the base, resulting in cooler
gases being entrained into the core of the flame with a resulting drop in the centerline temperature.

Figure~\ref{WP_CH4_Species} shows predicted and
measured values of CH$_4$, O$_2$, CO,  and CO$_2$ at three elevations above the burner along with a contour plot of CO near the
burner lip (black rectangle is burner).  Note, as discussed above, that the test data shows that a small quantity of oxygen exists along
the burner centerline which is not captured in the simulation.  Along the centerline, the model predicts higher values of fuel and higher values of
products than measured.  The species profiles are also slightly narrower than measured, consistent with the temperature prediction.  
The peak CH$_4$ values along the burner centerline have errors ranging from 6~\% to 13~\%,
the peak CO$_2$ values in the flame have errors ranging from 3~\% to 10`\%, and the peak CO values in the flame have
errors ranging from 9~\% to 25~\%.  The reported uncertainty in the species concentration measurements ranges from 10~\% to 20~\%. 

The contour of CO illustrates the combustion model quite well.  At the burner lip, fuel and oxygen first meet.
Since ambient levels of oxygen exist on the air side of the slot and the temperatures are relatively low at the
burner lip, combustion occurs and converts $Z_1$ to $Z_2$.  This consumes the oxygen present and the CO does not have an
opportunity to fully convert to $CO_2$.  As the CO rises in the flame, oxygen is entrained and the temperatures are high
enough that the CO reacts with the entrained oxygen to form $CO_2$.  Eventually, at a point high enough in the flame,
all of the fuel is consumed and all of the CO is oxidized.

\begin{figure}
\includegraphics[width=6in]{FIGURES/WP/WP_CH4_Temp}
\caption{Predicted and measured temperature at three elevations (7 mm, 9mm and 11 mm) above a methane-air slot burner}
\label{WP_CH4_Temp}
\end{figure}

\begin{figure}
\includegraphics[width=6in]{FIGURES/WP/WP_CH4_species}
\caption{Predicted and measured mole fractions at three elevations (7 mm, 9mm and 11 mm) above a methane-air slot burner,
along with contours of CO near the burner lip}
\label{WP_CH4_Species}
\end{figure}

\clearpage

\section{Beyler Hood Experiments}

Fig.~\ref{Beyler_Species} shows species predictions made by the two-step model compared with measured data for a
range of fire sizes and burner positions.  The dotted lines indicate the estimated measurement uncertainty.  The
model predicts the time-averaged species concentration at the hood exhaust vent.  CO$_2$ predictions are within the
measurement uncertainty for all but one of the simulations performed.  For the well-ventilated fires (burner 10~cm below the edge of the hood), CO,
CO$_2$, and unburned fuel predictions match the data.  As the fires become under-ventilated, CO is over-predicted while
fuel and O$_2$ are under-predicted.  The most likely explanation for the discrepancy is that the model assumes fuel and
oxygen react infinitely fast in the vicinity of near ambient conditions.  This occurs at the lower edge of the hood
were the vitiated layer is adjacent to the ambient air below the hood, and as a result layer burning is occurring in
the model which depletes the fuel and O$_2$ and creates CO.  This is not unexpected, and indicates that more work is
required to establish the conditions under which combustion in the first step, conversion of fuel to CO, will be
allowed.

\begin{figure}[p]
\includegraphics[width=\textwidth]{FIGURES/Beyler_Hood/Beyler_species}
\caption{Comparison of measured and predicted species concentrations in the Beyler hood experiments,
dotted lines show experimental uncertainty }
\label{Beyler_Species}
\end{figure}

\clearpage

\section{NIST Reduced Scale Enclosure (RSE) Test Series}

The RSE natural gas experiments were selected to assess the CO production capability rather than soot production. 
Nine fire sizes were simulated: 50~kW, 75~kW, 100~kW, 150~kW, 200~kW, 300~kW, 400~kW,
500~kW, and 600~kW.  The experiments were modeled using properties of the natural gas supplied to the test facility.
The model geometry included the compartment interior along with a 0.6~m deep region outside the door.
Figure~\ref{RSE_Pitts_Species} shows the measured and predicted CO$_2$ and CO concentrations.  The measured values are
from the test series performed by Bryner, Johnsson, and Pitts~\cite{Bryner:1}.  The model matches the data up to a fire size of
300~kW, including the location of the peak CO$_2$ concentration at 200~kW.  For larger fires the model predicts more
CO surviving in the upper layer than measured, along with correspondingly lower CO$_2$ levels.  
As the compartment becomes under-ventilated, the model under-predicts
CO$_2$ and over-predicts CO. The relative error increases as the compartment becomes more under-ventilated.  However,
note that the under-prediction of CO$_2$ concentration in the rear (1.5~\%) is equivalent to the
over-prediction of CO (1.7~\%).  This implies that part of the model error results from the assumption of
infinitely fast chemistry for the first step.  Furthermore, since the model as implemented performs the two steps
sequentially, higher CO is predicted where the first step consumes all of the available oxygen.

\begin{figure}
\includegraphics[width=6in]{FIGURES/RSE_Pitts/RSE_Pitts_Species}
\caption{Predicted and measured CO2 and CO concentrations for the NIST RSE experiments of Bryner, Johnsson, and Pitts}
\label{RSE_Pitts_Species}
\end{figure}
