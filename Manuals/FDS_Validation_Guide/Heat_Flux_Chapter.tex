% !TEX root = FDS_Validation_Guide.tex

\chapter{Heat Flux}

Radiative heat transfer is included in FDS via the solution of the radiation transport equation for a gray gas, and in some limited cases using a wide band model.  The equation is solved using a technique similar to finite volume methods for convective transport, thus the name given to it is the Finite Volume Method (FVM).  Using approximately 100 discrete angles, the finite volume solver requires about 20~\% of the total CPU time of a calculation, a modest cost given the complexity of radiation heat transfer. The absorption coefficients of the gas-soot mixtures are computed using the RadCal narrow-band model. Liquid droplets can absorb and scatter thermal radiation. This is important in cases involving mist sprinklers, but also plays a role in all sprinkler cases. The absorption and scattering coefficients are based on Mie theory.

This chapter contains a wide variety of heat flux measurements, ranging from less than a kW/m$^2$ from very small methane gas burners up to about 150~kW/m$^2$ in full-scale compartment fires.


\clearpage

\section{Heat Flux to Walls, Ceiling, and Floor}


\subsection{FAA Cargo Compartments}

Measurements of heat flux and surface temperature were made at two ceiling locations (denoted A and B in Fig.~\ref{FAA_Cargo_probe_locations}). The heat flux measurements are shown below.

\begin{figure}[h!]
\begin{tabular*}{\textwidth}{l@{\extracolsep{\fill}}r}
\includegraphics[height=2.15in]{SCRIPT_FIGURES/FAA_Cargo_Compartments/FAA_Cargo_Compartments_Test_1_Heat_Flux} &
\includegraphics[height=2.15in]{SCRIPT_FIGURES/FAA_Cargo_Compartments/FAA_Cargo_Compartments_Test_2_Heat_Flux} \\
\multicolumn{2}{c}{\includegraphics[height=2.15in]{SCRIPT_FIGURES/FAA_Cargo_Compartments/FAA_Cargo_Compartments_Test_3_Heat_Flux}}
\end{tabular*}
\caption{FAA Cargo Compartment experiments, heat flux to ceiling.}
\end{figure}

\clearpage


\subsection{FM Parallel Panel Experiments}

Predicted and measured vertical heat flux profiles for three propane and three propylene fires (30~kW, 60~kW, and 100~kW) sandwiched between two 2.4~m high, 0.6~m wide panels are presented below.

\begin{figure}[h!]
\begin{tabular*}{\textwidth}{l@{\extracolsep{\fill}}r}
\includegraphics[height=2.15in]{SCRIPT_FIGURES/FM_Parallel_Panel/FM_Parallel_Panel_1_Heat_Flux} &
\includegraphics[height=2.15in]{SCRIPT_FIGURES/FM_Parallel_Panel/FM_Parallel_Panel_4_Heat_Flux} \\
\includegraphics[height=2.15in]{SCRIPT_FIGURES/FM_Parallel_Panel/FM_Parallel_Panel_2_Heat_Flux} &
\includegraphics[height=2.15in]{SCRIPT_FIGURES/FM_Parallel_Panel/FM_Parallel_Panel_5_Heat_Flux} \\
\includegraphics[height=2.15in]{SCRIPT_FIGURES/FM_Parallel_Panel/FM_Parallel_Panel_3_Heat_Flux} &
\includegraphics[height=2.15in]{SCRIPT_FIGURES/FM_Parallel_Panel/FM_Parallel_Panel_6_Heat_Flux}
\end{tabular*}
\label{FM_Parallel_Panel}
\caption[FM Parallel Panel experiments, side wall heat flux]
{FM Parallel Panel experiments, side wall heat flux.}
\end{figure}


\clearpage

\subsection{NIST/NRC Experiments}

Heat flux gauges and thermocouples were positioned at various locations on the walls, floor, and ceiling of the compartment. The locations are given in Table~\ref{NIST_NRC_Wall_Coords}. The heat flux gauges were not water cooled; thus, they measured the {\em net} rather than the {\em gauge} heat flux. However, the net heat flux is a function of the temperature of the heat flux gauge itself, which is not something that is modeled. To better compare model and measurement, the measured net heat flux is converted into a gauge heat flux using the following formula:
\begin{equation}
\dot{q}''_{\subscript{gauge}} = \dot{q}''_{\subscript{net}} + \sigma \left( T^4_{\subscript{gauge}} - T^4_\infty \right) + h  \left( T_{\subscript{gauge}}-T_\infty \right) \quad \hbox{kW/m}^2
\end{equation}
where $\sigma=5.67 \times 10^{-11}$~kW/m$^2$/K$^4$ and $h=0.005$~kW/m$^2$/K.

Also, over the course of 15 experiments, numerous heat flux gauges failed, most often due to loss of contact with the wall or faulty thermocouples. All of the measurements from Test~13 and 16 were found to be flawed.

\newpage

\begin{figure}[p]
\begin{tabular*}{\textwidth}{l@{\extracolsep{\fill}}r}
\includegraphics[height=2.15in]{SCRIPT_FIGURES/NIST_NRC/NIST_NRC_01_North_Wall_Flux} &
\includegraphics[height=2.15in]{SCRIPT_FIGURES/NIST_NRC/NIST_NRC_07_North_Wall_Flux} \\
\includegraphics[height=2.15in]{SCRIPT_FIGURES/NIST_NRC/NIST_NRC_02_North_Wall_Flux} &
\includegraphics[height=2.15in]{SCRIPT_FIGURES/NIST_NRC/NIST_NRC_08_North_Wall_Flux} \\
\includegraphics[height=2.15in]{SCRIPT_FIGURES/NIST_NRC/NIST_NRC_04_North_Wall_Flux} &
\includegraphics[height=2.15in]{SCRIPT_FIGURES/NIST_NRC/NIST_NRC_10_North_Wall_Flux}
\end{tabular*}
\caption{NIST/NRC experiments, heat flux to north wall, Tests 1, 2, 4, 7, 8, 10.}
\label{NIST_NRC_North_Wall_Flux_Closed}
\end{figure}

\begin{figure}[p]
\begin{tabular*}{\textwidth}{l@{\extracolsep{\fill}}r}
\includegraphics[height=2.15in]{SCRIPT_FIGURES/NIST_NRC/NIST_NRC_03_North_Wall_Flux} &
\includegraphics[height=2.15in]{SCRIPT_FIGURES/NIST_NRC/NIST_NRC_09_North_Wall_Flux} \\
\includegraphics[height=2.15in]{SCRIPT_FIGURES/NIST_NRC/NIST_NRC_05_North_Wall_Flux} &
\includegraphics[height=2.15in]{SCRIPT_FIGURES/NIST_NRC/NIST_NRC_14_North_Wall_Flux} \\
\includegraphics[height=2.15in]{SCRIPT_FIGURES/NIST_NRC/NIST_NRC_15_North_Wall_Flux} &
\includegraphics[height=2.15in]{SCRIPT_FIGURES/NIST_NRC/NIST_NRC_18_North_Wall_Flux}
\end{tabular*}
\caption{NIST/NRC experiments, heat flux to north wall, Tests 3, 5, 9, 14, 15, 18.}
\label{NIST_NRC_North_Wall_Flux_Open}
\end{figure}

\begin{figure}[p]
\begin{tabular*}{\textwidth}{l@{\extracolsep{\fill}}r}
\includegraphics[height=2.15in]{SCRIPT_FIGURES/NIST_NRC/NIST_NRC_01_South_Wall_Flux} &
\includegraphics[height=2.15in]{SCRIPT_FIGURES/NIST_NRC/NIST_NRC_07_South_Wall_Flux} \\
\includegraphics[height=2.15in]{SCRIPT_FIGURES/NIST_NRC/NIST_NRC_02_South_Wall_Flux} &
\includegraphics[height=2.15in]{SCRIPT_FIGURES/NIST_NRC/NIST_NRC_08_South_Wall_Flux} \\
\includegraphics[height=2.15in]{SCRIPT_FIGURES/NIST_NRC/NIST_NRC_04_South_Wall_Flux} &
\includegraphics[height=2.15in]{SCRIPT_FIGURES/NIST_NRC/NIST_NRC_10_South_Wall_Flux}
\end{tabular*}
\caption{NIST/NRC experiments, heat flux to south wall, Tests 1, 2, 4, 7, 8, 10.}
\label{NIST_NRC_South_Wall_Flux_Closed}
\end{figure}

\begin{figure}[p]
\begin{tabular*}{\textwidth}{l@{\extracolsep{\fill}}r}
\includegraphics[height=2.15in]{SCRIPT_FIGURES/NIST_NRC/NIST_NRC_03_South_Wall_Flux} &
\includegraphics[height=2.15in]{SCRIPT_FIGURES/NIST_NRC/NIST_NRC_09_South_Wall_Flux} \\
\includegraphics[height=2.15in]{SCRIPT_FIGURES/NIST_NRC/NIST_NRC_05_South_Wall_Flux} &
\includegraphics[height=2.15in]{SCRIPT_FIGURES/NIST_NRC/NIST_NRC_14_South_Wall_Flux} \\
\includegraphics[height=2.15in]{SCRIPT_FIGURES/NIST_NRC/NIST_NRC_15_South_Wall_Flux} &
\includegraphics[height=2.15in]{SCRIPT_FIGURES/NIST_NRC/NIST_NRC_18_South_Wall_Flux}
\end{tabular*}
\caption{NIST/NRC experiments, heat flux to south wall, Tests 3, 5, 9, 14, 15, 18.}
\label{NIST_NRC_South_Wall_Flux_Open}
\end{figure}


\begin{figure}[p]
\begin{tabular*}{\textwidth}{l@{\extracolsep{\fill}}r}
\includegraphics[height=2.15in]{SCRIPT_FIGURES/NIST_NRC/NIST_NRC_01_East_Wall_Flux} &
\includegraphics[height=2.15in]{SCRIPT_FIGURES/NIST_NRC/NIST_NRC_07_East_Wall_Flux} \\
\includegraphics[height=2.15in]{SCRIPT_FIGURES/NIST_NRC/NIST_NRC_02_East_Wall_Flux} &
\includegraphics[height=2.15in]{SCRIPT_FIGURES/NIST_NRC/NIST_NRC_08_East_Wall_Flux} \\
\includegraphics[height=2.15in]{SCRIPT_FIGURES/NIST_NRC/NIST_NRC_04_East_Wall_Flux} &
\includegraphics[height=2.15in]{SCRIPT_FIGURES/NIST_NRC/NIST_NRC_10_East_Wall_Flux}
\end{tabular*}
\caption{NIST/NRC experiments, heat flux to east wall, Tests 1, 2, 4, 7, 8, 10.}
\label{NIST_NRC_East_Wall_Flux_Closed}
\end{figure}

\begin{figure}[p]
\begin{tabular*}{\textwidth}{l@{\extracolsep{\fill}}r}
\includegraphics[height=2.15in]{SCRIPT_FIGURES/NIST_NRC/NIST_NRC_03_East_Wall_Flux} &
\includegraphics[height=2.15in]{SCRIPT_FIGURES/NIST_NRC/NIST_NRC_09_East_Wall_Flux} \\
\includegraphics[height=2.15in]{SCRIPT_FIGURES/NIST_NRC/NIST_NRC_05_East_Wall_Flux} &
\includegraphics[height=2.15in]{SCRIPT_FIGURES/NIST_NRC/NIST_NRC_14_East_Wall_Flux} \\
\includegraphics[height=2.15in]{SCRIPT_FIGURES/NIST_NRC/NIST_NRC_15_East_Wall_Flux} &
\includegraphics[height=2.15in]{SCRIPT_FIGURES/NIST_NRC/NIST_NRC_18_East_Wall_Flux}
\end{tabular*}
\caption{NIST/NRC experiments, heat flux to east wall, Tests 3, 5, 9, 14, 15, 18.}
\label{NIST_NRC_East_Wall_Flux_Open}
\end{figure}


\begin{figure}[p]
\begin{tabular*}{\textwidth}{l@{\extracolsep{\fill}}r}
\includegraphics[height=2.15in]{SCRIPT_FIGURES/NIST_NRC/NIST_NRC_01_West_Wall_Flux} &
\includegraphics[height=2.15in]{SCRIPT_FIGURES/NIST_NRC/NIST_NRC_07_West_Wall_Flux} \\
\includegraphics[height=2.15in]{SCRIPT_FIGURES/NIST_NRC/NIST_NRC_02_West_Wall_Flux} &
\includegraphics[height=2.15in]{SCRIPT_FIGURES/NIST_NRC/NIST_NRC_08_West_Wall_Flux} \\
\includegraphics[height=2.15in]{SCRIPT_FIGURES/NIST_NRC/NIST_NRC_04_West_Wall_Flux} &
\includegraphics[height=2.15in]{SCRIPT_FIGURES/NIST_NRC/NIST_NRC_10_West_Wall_Flux}
\end{tabular*}
\caption{NIST/NRC experiments, heat flux to west wall, Tests 1, 2, 4, 7, 8, 10.}
\label{NIST_NRC_West_Wall_Flux_Closed}
\end{figure}

\begin{figure}[p]
\begin{tabular*}{\textwidth}{l@{\extracolsep{\fill}}r}
\includegraphics[height=2.15in]{SCRIPT_FIGURES/NIST_NRC/NIST_NRC_03_West_Wall_Flux} &
\includegraphics[height=2.15in]{SCRIPT_FIGURES/NIST_NRC/NIST_NRC_09_West_Wall_Flux} \\
\includegraphics[height=2.15in]{SCRIPT_FIGURES/NIST_NRC/NIST_NRC_05_West_Wall_Flux} &
\includegraphics[height=2.15in]{SCRIPT_FIGURES/NIST_NRC/NIST_NRC_14_West_Wall_Flux} \\
\includegraphics[height=2.15in]{SCRIPT_FIGURES/NIST_NRC/NIST_NRC_15_West_Wall_Flux} &
\includegraphics[height=2.15in]{SCRIPT_FIGURES/NIST_NRC/NIST_NRC_18_West_Wall_Flux}
\end{tabular*}
\caption{NIST/NRC experiments, heat flux to west wall, Tests 3, 5, 9, 14, 15, 18.}
\label{NIST_NRC_West_Wall_Flux_Open}
\end{figure}

\begin{figure}[p]
\begin{tabular*}{\textwidth}{l@{\extracolsep{\fill}}r}
\includegraphics[height=2.15in]{SCRIPT_FIGURES/NIST_NRC/NIST_NRC_01_Floor_Flux} &
\includegraphics[height=2.15in]{SCRIPT_FIGURES/NIST_NRC/NIST_NRC_07_Floor_Flux} \\
\includegraphics[height=2.15in]{SCRIPT_FIGURES/NIST_NRC/NIST_NRC_02_Floor_Flux} &
\includegraphics[height=2.15in]{SCRIPT_FIGURES/NIST_NRC/NIST_NRC_08_Floor_Flux} \\
\includegraphics[height=2.15in]{SCRIPT_FIGURES/NIST_NRC/NIST_NRC_04_Floor_Flux} &
\includegraphics[height=2.15in]{SCRIPT_FIGURES/NIST_NRC/NIST_NRC_10_Floor_Flux}
\end{tabular*}
\caption{NIST/NRC experiments, heat flux to the floor, Tests 1, 2, 4, 7, 8, 10.}
\label{NIST_NRC_Floor_Flux_Closed}
\end{figure}

\begin{figure}[p]
\begin{tabular*}{\textwidth}{l@{\extracolsep{\fill}}r}
\includegraphics[height=2.15in]{SCRIPT_FIGURES/NIST_NRC/NIST_NRC_03_Floor_Flux} &
\includegraphics[height=2.15in]{SCRIPT_FIGURES/NIST_NRC/NIST_NRC_09_Floor_Flux} \\
\includegraphics[height=2.15in]{SCRIPT_FIGURES/NIST_NRC/NIST_NRC_05_Floor_Flux} &
\includegraphics[height=2.15in]{SCRIPT_FIGURES/NIST_NRC/NIST_NRC_14_Floor_Flux} \\
\includegraphics[height=2.15in]{SCRIPT_FIGURES/NIST_NRC/NIST_NRC_15_Floor_Flux} &
\includegraphics[height=2.15in]{SCRIPT_FIGURES/NIST_NRC/NIST_NRC_18_Floor_Flux}
\end{tabular*}
\caption{NIST/NRC experiments, heat flux to the floor, Tests 3, 5, 9, 14, 15, 18.}
\label{NIST_NRC_Floor_Flux_Open}
\end{figure}


\begin{figure}[p]
\begin{tabular*}{\textwidth}{l@{\extracolsep{\fill}}r}
\includegraphics[height=2.15in]{SCRIPT_FIGURES/NIST_NRC/NIST_NRC_01_Ceiling_Flux} &
\includegraphics[height=2.15in]{SCRIPT_FIGURES/NIST_NRC/NIST_NRC_07_Ceiling_Flux} \\
\includegraphics[height=2.15in]{SCRIPT_FIGURES/NIST_NRC/NIST_NRC_02_Ceiling_Flux} &
\includegraphics[height=2.15in]{SCRIPT_FIGURES/NIST_NRC/NIST_NRC_08_Ceiling_Flux} \\
\includegraphics[height=2.15in]{SCRIPT_FIGURES/NIST_NRC/NIST_NRC_04_Ceiling_Flux} &
\includegraphics[height=2.15in]{SCRIPT_FIGURES/NIST_NRC/NIST_NRC_10_Ceiling_Flux}
\end{tabular*}
\caption{NIST/NRC experiments, heat flux to the ceiling, Tests 1, 2, 4, 7, 8, 10.}
\label{NIST_NRC_Ceiling_Flux_Closed}
\end{figure}

\begin{figure}[p]
\begin{tabular*}{\textwidth}{l@{\extracolsep{\fill}}r}
\includegraphics[height=2.15in]{SCRIPT_FIGURES/NIST_NRC/NIST_NRC_03_Ceiling_Flux} &
\includegraphics[height=2.15in]{SCRIPT_FIGURES/NIST_NRC/NIST_NRC_09_Ceiling_Flux} \\
\includegraphics[height=2.15in]{SCRIPT_FIGURES/NIST_NRC/NIST_NRC_05_Ceiling_Flux} &
\includegraphics[height=2.15in]{SCRIPT_FIGURES/NIST_NRC/NIST_NRC_14_Ceiling_Flux} \\
\includegraphics[height=2.15in]{SCRIPT_FIGURES/NIST_NRC/NIST_NRC_15_Ceiling_Flux} &
\includegraphics[height=2.15in]{SCRIPT_FIGURES/NIST_NRC/NIST_NRC_18_Ceiling_Flux}
\end{tabular*}
\caption{NIST/NRC experiments, heat flux to the ceiling, Tests 3, 5, 9, 14, 15, 18.}
\label{NIST_NRC_Ceiling_Flux_Open}
\end{figure}

\clearpage

\subsection{NRCC Facade Experiments}

Figure~\ref{NRCC_Facade_Image} displays the simulation of a 10.3~MW fire inside and outside of a small enclosure. The purpose of the experiment was to measure the heat flux to the exterior facade. The FDS heat flux predictions are made at the location of the green points.

\begin{figure}[h!]
\begin{center}
\begin{tabular}{c}
\includegraphics[width=5.0in]{FIGURES/NRCC_Facade/NRCC_Facade_Win_2_10_MW_0467}
\end{tabular}
\end{center}
\caption[Smokeview rendering of NRCC Facade experiment]
{Smokeview rendering of one of the NRCC Facade experiments. The door is
0.94~m by 2.70~m tall (referred to as ``Window 2'' in the comparison plots). The
fire is 10.3~MW.}
\label{NRCC_Facade_Image}
\end{figure}

\newpage

\begin{figure}[p]
\begin{tabular*}{\textwidth}{l@{\extracolsep{\fill}}r}
\includegraphics[height=2.15in]{SCRIPT_FIGURES/NRCC_Facade/NRCC_Facade_Win_1_05_MW} &
\includegraphics[height=2.15in]{SCRIPT_FIGURES/NRCC_Facade/NRCC_Facade_Win_1_06_MW} \\
\includegraphics[height=2.15in]{SCRIPT_FIGURES/NRCC_Facade/NRCC_Facade_Win_1_08_MW} &
  \\
\includegraphics[height=2.15in]{SCRIPT_FIGURES/NRCC_Facade/NRCC_Facade_Win_2_05_MW} &
\includegraphics[height=2.15in]{SCRIPT_FIGURES/NRCC_Facade/NRCC_Facade_Win_2_06_MW} \\
\includegraphics[height=2.15in]{SCRIPT_FIGURES/NRCC_Facade/NRCC_Facade_Win_2_08_MW} &
\includegraphics[height=2.15in]{SCRIPT_FIGURES/NRCC_Facade/NRCC_Facade_Win_2_10_MW}
\end{tabular*}
\caption{NRCC Facade experiments, heat flux, window configuration 1 and 2.}
\label{NRCC_Facade_1}
\end{figure}

\begin{figure}[p]
\begin{tabular*}{\textwidth}{l@{\extracolsep{\fill}}r}
\includegraphics[height=2.15in]{SCRIPT_FIGURES/NRCC_Facade/NRCC_Facade_Win_3_05_MW} &
\includegraphics[height=2.15in]{SCRIPT_FIGURES/NRCC_Facade/NRCC_Facade_Win_3_06_MW} \\
\includegraphics[height=2.15in]{SCRIPT_FIGURES/NRCC_Facade/NRCC_Facade_Win_3_08_MW} &
\includegraphics[height=2.15in]{SCRIPT_FIGURES/NRCC_Facade/NRCC_Facade_Win_3_10_MW} \\
\includegraphics[height=2.15in]{SCRIPT_FIGURES/NRCC_Facade/NRCC_Facade_Win_4_05_MW} &
\includegraphics[height=2.15in]{SCRIPT_FIGURES/NRCC_Facade/NRCC_Facade_Win_4_06_MW} \\
\includegraphics[height=2.15in]{SCRIPT_FIGURES/NRCC_Facade/NRCC_Facade_Win_4_08_MW} &
\includegraphics[height=2.15in]{SCRIPT_FIGURES/NRCC_Facade/NRCC_Facade_Win_4_10_MW}
\end{tabular*}
\caption{NRCC Facade experiments, heat flux, window configuration 3 and 4.}
\label{NRCC_Facade_2}
\end{figure}

\begin{figure}[p]
\begin{tabular*}{\textwidth}{l@{\extracolsep{\fill}}r}
\includegraphics[height=2.15in]{SCRIPT_FIGURES/NRCC_Facade/NRCC_Facade_Win_5_05_MW} &
\includegraphics[height=2.15in]{SCRIPT_FIGURES/NRCC_Facade/NRCC_Facade_Win_5_06_MW} \\
\includegraphics[height=2.15in]{SCRIPT_FIGURES/NRCC_Facade/NRCC_Facade_Win_5_08_MW} &
\includegraphics[height=2.15in]{SCRIPT_FIGURES/NRCC_Facade/NRCC_Facade_Win_5_10_MW}
\end{tabular*}
\caption{NRCC Facade experiments, heat flux, window configuration 5.}
\label{NRCC_Facade_3}
\end{figure}


\clearpage



\subsection{NRL/HAI Experiments}

Predicted and measured vertical heat flux profiles from 9 propane sand burner fires are shown on the following pages. The parameters for each
experiment are listed in Table~\ref{NRL/HAI_Parameters} below. Note that all the FDS simulations were performed with a grid resolution such that
$D^*/\dx=10$.

\begin{table}[ht]
\caption{Summary of the NRL/HAI Wall Heat Flux Measurements. }
\begin{center}
\begin{tabular}{|c|c|c|c|c|c|}
\hline
Test     & $D$     & $D^*$      & $\dot{Q}$   & $Q^*$   & Observed  Flame \\
Number   & (m)     & (m)        & (kW)        &         & Height (m)      \\ \hline \hline
1        & 0.28    & 0.30       &  53         & 0.85    & 0.79            \\ \hline
2        & 0.70    & 0.30       &  56         & 0.09    & 0.36            \\ \hline
3        & 0.48    & 0.33       &  68         & 0.28    & 0.60            \\ \hline
4        & 0.37    & 0.39       &  106        & 0.84    & 1.00            \\ \hline
5        & 0.48    & 0.43       &  136        & 0.57    & 0.87            \\ \hline
6        & 0.48    & 0.51       &  204        & 0.85    & 1.45            \\ \hline
7        & 0.70    & 0.52       &  220        & 0.36    & 1.20            \\ \hline
8        & 0.57    & 0.60       &  313        & 0.85    & 2.20            \\ \hline
9        & 0.70    & 0.74       &  523        & 0.85    & 2.9 (based on 500~$^\circ$C)       \\ \hline
\end{tabular}
\end{center}
\label{NRL/HAI_Parameters}
\end{table}

\newpage

\begin{figure}[p]
\begin{tabular*}{\textwidth}{l@{\extracolsep{\fill}}r}
\includegraphics[height=2.15in]{SCRIPT_FIGURES/NRL_HAI/NRL_HAI_1_Heat_Flux} &
\includegraphics[height=2.15in]{SCRIPT_FIGURES/NRL_HAI/NRL_HAI_2_Heat_Flux} \\
\includegraphics[height=2.15in]{SCRIPT_FIGURES/NRL_HAI/NRL_HAI_3_Heat_Flux} &
\includegraphics[height=2.15in]{SCRIPT_FIGURES/NRL_HAI/NRL_HAI_4_Heat_Flux} \\
\includegraphics[height=2.15in]{SCRIPT_FIGURES/NRL_HAI/NRL_HAI_5_Heat_Flux} &
\end{tabular*}
\label{NRL_HAI_1}
\caption{NRL/HAI experiments, heat flux to the wall, Tests 1-5.}
\end{figure}

\begin{figure}[p]
\begin{tabular*}{\textwidth}{l@{\extracolsep{\fill}}r}
\includegraphics[height=2.15in]{SCRIPT_FIGURES/NRL_HAI/NRL_HAI_6_Heat_Flux} &
\includegraphics[height=2.15in]{SCRIPT_FIGURES/NRL_HAI/NRL_HAI_7_Heat_Flux} \\
\includegraphics[height=2.15in]{SCRIPT_FIGURES/NRL_HAI/NRL_HAI_8_Heat_Flux} &
\includegraphics[height=2.15in]{SCRIPT_FIGURES/NRL_HAI/NRL_HAI_9_Heat_Flux}
\end{tabular*}
\label{NRL_HAI_2}
\caption{NRL/HAI experiments, heat flux to the wall, Tests 6-9.}
\end{figure}



\clearpage

\subsection{PRISME DOOR Experiments}

Total and radiative heat flux gauges were positioned at various points on the walls. Each room contained a vertical array labelled, for example, FLT\_L1\_NC265. The FLT indicates a surface total heat flux measurement, L1 indicates compartment 1, which is where the fire was located, NC indicates north wall center, and 265 indicates the number of centimeters above the floor. In addition, each room contained four measurement points centered on each wall at a height of approximately 260~cm. These points are labelled, for example, FLT\_L2\_SC265, compartment 2, center of south wall, 265~cm high.

\begin{figure}[!ht]
\begin{tabular*}{\textwidth}{l@{\extracolsep{\fill}}r}
\includegraphics[height=2.15in]{SCRIPT_FIGURES/PRISME/PRS_D1_Room_1_Total_Heat_Flux_Array} &
\includegraphics[height=2.15in]{SCRIPT_FIGURES/PRISME/PRS_D2_Room_1_Total_Heat_Flux_Array} \\
\includegraphics[height=2.15in]{SCRIPT_FIGURES/PRISME/PRS_D3_Room_1_Total_Heat_Flux_Array} &
\includegraphics[height=2.15in]{SCRIPT_FIGURES/PRISME/PRS_D4_Room_1_Total_Heat_Flux_Array} \\
\includegraphics[height=2.15in]{SCRIPT_FIGURES/PRISME/PRS_D5_Room_1_Total_Heat_Flux_Array} &
\includegraphics[height=2.15in]{SCRIPT_FIGURES/PRISME/PRS_D6_Room_1_Total_Heat_Flux_Array}
\end{tabular*}
\caption{PRISME DOOR experiments, total heat flux, vertical array, Room 1.}
\label{PRISME_Wall_Array_THF_Room_1}
\end{figure}

\newpage

\begin{figure}[p]
\begin{tabular*}{\textwidth}{l@{\extracolsep{\fill}}r}
\includegraphics[height=2.15in]{SCRIPT_FIGURES/PRISME/PRS_D1_Room_1_Radiative_Heat_Flux_Array} &
\includegraphics[height=2.15in]{SCRIPT_FIGURES/PRISME/PRS_D2_Room_1_Radiative_Heat_Flux_Array} \\
\includegraphics[height=2.15in]{SCRIPT_FIGURES/PRISME/PRS_D3_Room_1_Radiative_Heat_Flux_Array} &
\includegraphics[height=2.15in]{SCRIPT_FIGURES/PRISME/PRS_D4_Room_1_Radiative_Heat_Flux_Array} \\
\includegraphics[height=2.15in]{SCRIPT_FIGURES/PRISME/PRS_D5_Room_1_Radiative_Heat_Flux_Array} &
\includegraphics[height=2.15in]{SCRIPT_FIGURES/PRISME/PRS_D6_Room_1_Radiative_Heat_Flux_Array}
\end{tabular*}
\caption{PRISME DOOR experiments, radiative heat flux, vertical array, Room 1.}
\label{PRISME_Wall_Array_RHF_Room_1}
\end{figure}

\begin{figure}[p]
\begin{tabular*}{\textwidth}{l@{\extracolsep{\fill}}r}
\includegraphics[height=2.15in]{SCRIPT_FIGURES/PRISME/PRS_D1_Room_1_Total_Heat_Flux_Circle} &
\includegraphics[height=2.15in]{SCRIPT_FIGURES/PRISME/PRS_D2_Room_1_Total_Heat_Flux_Circle} \\
\includegraphics[height=2.15in]{SCRIPT_FIGURES/PRISME/PRS_D3_Room_1_Total_Heat_Flux_Circle} &
\includegraphics[height=2.15in]{SCRIPT_FIGURES/PRISME/PRS_D4_Room_1_Total_Heat_Flux_Circle} \\
\includegraphics[height=2.15in]{SCRIPT_FIGURES/PRISME/PRS_D5_Room_1_Total_Heat_Flux_Circle} &
\includegraphics[height=2.15in]{SCRIPT_FIGURES/PRISME/PRS_D6_Room_1_Total_Heat_Flux_Circle}
\end{tabular*}
\caption{PRISME DOOR experiments, total heat flux, four walls, Room 1.}
\label{PRISME_Wall_Circle_THF_Room_1}
\end{figure}

\begin{figure}[p]
\begin{tabular*}{\textwidth}{l@{\extracolsep{\fill}}r}
\includegraphics[height=2.15in]{SCRIPT_FIGURES/PRISME/PRS_D1_Room_2_Total_Heat_Flux_Array} &
\includegraphics[height=2.15in]{SCRIPT_FIGURES/PRISME/PRS_D2_Room_2_Total_Heat_Flux_Array} \\
\includegraphics[height=2.15in]{SCRIPT_FIGURES/PRISME/PRS_D3_Room_2_Total_Heat_Flux_Array} &
\includegraphics[height=2.15in]{SCRIPT_FIGURES/PRISME/PRS_D4_Room_2_Total_Heat_Flux_Array} \\
\includegraphics[height=2.15in]{SCRIPT_FIGURES/PRISME/PRS_D5_Room_2_Total_Heat_Flux_Array} &
\includegraphics[height=2.15in]{SCRIPT_FIGURES/PRISME/PRS_D6_Room_2_Total_Heat_Flux_Array}
\end{tabular*}
\caption{PRISME DOOR experiments, total heat flux, vertical array, Room 2.}
\label{PRISME_Wall_Array_THF_Room_2}
\end{figure}

\begin{figure}[p]
\begin{tabular*}{\textwidth}{l@{\extracolsep{\fill}}r}
\includegraphics[height=2.15in]{SCRIPT_FIGURES/PRISME/PRS_D1_Room_2_Radiative_Heat_Flux_Array} &
\includegraphics[height=2.15in]{SCRIPT_FIGURES/PRISME/PRS_D2_Room_2_Radiative_Heat_Flux_Array} \\
\includegraphics[height=2.15in]{SCRIPT_FIGURES/PRISME/PRS_D3_Room_2_Radiative_Heat_Flux_Array} &
\includegraphics[height=2.15in]{SCRIPT_FIGURES/PRISME/PRS_D4_Room_2_Radiative_Heat_Flux_Array} \\
\includegraphics[height=2.15in]{SCRIPT_FIGURES/PRISME/PRS_D5_Room_2_Radiative_Heat_Flux_Array} &
\includegraphics[height=2.15in]{SCRIPT_FIGURES/PRISME/PRS_D6_Room_2_Radiative_Heat_Flux_Array}
\end{tabular*}
\caption{PRISME DOOR experiments, radiative heat flux, vertical array, Room 2.}
\label{PRISME_Wall_Array_RHF_Room_2}
\end{figure}

\begin{figure}[p]
\begin{tabular*}{\textwidth}{l@{\extracolsep{\fill}}r}
\includegraphics[height=2.15in]{SCRIPT_FIGURES/PRISME/PRS_D1_Room_2_Total_Heat_Flux_Circle} &
\includegraphics[height=2.15in]{SCRIPT_FIGURES/PRISME/PRS_D2_Room_2_Total_Heat_Flux_Circle} \\
\includegraphics[height=2.15in]{SCRIPT_FIGURES/PRISME/PRS_D3_Room_2_Total_Heat_Flux_Circle} &
\includegraphics[height=2.15in]{SCRIPT_FIGURES/PRISME/PRS_D4_Room_2_Total_Heat_Flux_Circle} \\
\includegraphics[height=2.15in]{SCRIPT_FIGURES/PRISME/PRS_D5_Room_2_Total_Heat_Flux_Circle} &
\includegraphics[height=2.15in]{SCRIPT_FIGURES/PRISME/PRS_D6_Room_2_Total_Heat_Flux_Circle}
\end{tabular*}
\caption{PRISME DOOR experiments, total heat flux, four walls, Room 2.}
\label{PRISME_Wall_Circle_THF_Room_2}
\end{figure}



\clearpage


\subsection{PRISME SOURCE Experiments}

Total and radiative heat flux gauges were positioned at various points on the walls. Each room contained a vertical array labelled, for example, FLT\_L1\_NC265. The FLT indicates a surface total heat flux measurement, L1 indicates compartment 1, which is where the fire was located, NC indicates north wall center, and 265 indicates the number of centimeters above the floor. In addition, each room contained four measurement points centered on each wall at a height of approximately 260~cm. These points are labelled, for example, FLT\_L2\_SC265, compartment 2, center of south wall, 265~cm high.

\begin{figure}[p]
\begin{tabular*}{\textwidth}{l@{\extracolsep{\fill}}r}
\includegraphics[height=2.15in]{SCRIPT_FIGURES/PRISME/PRS_SI_D1_Room_2_Total_Heat_Flux_Array} &
\includegraphics[height=2.15in]{SCRIPT_FIGURES/PRISME/PRS_SI_D2_Room_2_Total_Heat_Flux_Array} \\
\includegraphics[height=2.15in]{SCRIPT_FIGURES/PRISME/PRS_SI_D3_Room_2_Total_Heat_Flux_Array} &
\includegraphics[height=2.15in]{SCRIPT_FIGURES/PRISME/PRS_SI_D4_Room_2_Total_Heat_Flux_Array} \\
\includegraphics[height=2.15in]{SCRIPT_FIGURES/PRISME/PRS_SI_D5_Room_2_Total_Heat_Flux_Array} &
\includegraphics[height=2.15in]{SCRIPT_FIGURES/PRISME/PRS_SI_D5a_Room_2_Total_Heat_Flux_Array} \\
\includegraphics[height=2.15in]{SCRIPT_FIGURES/PRISME/PRS_SI_D6_Room_2_Total_Heat_Flux_Array} &
\includegraphics[height=2.15in]{SCRIPT_FIGURES/PRISME/PRS_SI_D6a_Room_2_Total_Heat_Flux_Array}
\end{tabular*}
\caption{PRISME SOURCE experiments, total heat flux, vertical array, Room 2.}
\label{PRISME_SOURCE_Wall_Array_THF_Room_2}
\end{figure}

\begin{figure}[p]
\begin{tabular*}{\textwidth}{l@{\extracolsep{\fill}}r}
\includegraphics[height=2.15in]{SCRIPT_FIGURES/PRISME/PRS_SI_D1_Room_2_Radiative_Heat_Flux_Array} &
\includegraphics[height=2.15in]{SCRIPT_FIGURES/PRISME/PRS_SI_D2_Room_2_Radiative_Heat_Flux_Array} \\
\includegraphics[height=2.15in]{SCRIPT_FIGURES/PRISME/PRS_SI_D3_Room_2_Radiative_Heat_Flux_Array} &
\includegraphics[height=2.15in]{SCRIPT_FIGURES/PRISME/PRS_SI_D4_Room_2_Radiative_Heat_Flux_Array} \\
\includegraphics[height=2.15in]{SCRIPT_FIGURES/PRISME/PRS_SI_D5_Room_2_Radiative_Heat_Flux_Array} &
\includegraphics[height=2.15in]{SCRIPT_FIGURES/PRISME/PRS_SI_D5a_Room_2_Radiative_Heat_Flux_Array} \\
\includegraphics[height=2.15in]{SCRIPT_FIGURES/PRISME/PRS_SI_D6_Room_2_Radiative_Heat_Flux_Array} &
\includegraphics[height=2.15in]{SCRIPT_FIGURES/PRISME/PRS_SI_D6a_Room_2_Radiative_Heat_Flux_Array}
\end{tabular*}
\caption{PRISME SOURCE experiments, radiative heat flux, vertical array, Room 2.}
\label{PRISME_SOURCE_Wall_Array_RHF_Room_2}
\end{figure}

\begin{figure}[p]
\begin{tabular*}{\textwidth}{l@{\extracolsep{\fill}}r}
\includegraphics[height=2.15in]{SCRIPT_FIGURES/PRISME/PRS_SI_D1_Room_2_Total_Heat_Flux_Circle} &
\includegraphics[height=2.15in]{SCRIPT_FIGURES/PRISME/PRS_SI_D2_Room_2_Total_Heat_Flux_Circle} \\
\includegraphics[height=2.15in]{SCRIPT_FIGURES/PRISME/PRS_SI_D3_Room_2_Total_Heat_Flux_Circle} &
\includegraphics[height=2.15in]{SCRIPT_FIGURES/PRISME/PRS_SI_D4_Room_2_Total_Heat_Flux_Circle} \\
\includegraphics[height=2.15in]{SCRIPT_FIGURES/PRISME/PRS_SI_D5_Room_2_Total_Heat_Flux_Circle} &
\includegraphics[height=2.15in]{SCRIPT_FIGURES/PRISME/PRS_SI_D5a_Room_2_Total_Heat_Flux_Circle} \\
\includegraphics[height=2.15in]{SCRIPT_FIGURES/PRISME/PRS_SI_D6_Room_2_Total_Heat_Flux_Circle} &
\includegraphics[height=2.15in]{SCRIPT_FIGURES/PRISME/PRS_SI_D6a_Room_2_Total_Heat_Flux_Circle}
\end{tabular*}
\caption{PRISME SOURCE experiments, total heat flux, four walls, Room 2.}
\label{PRISME_SOURCE_Wall_Circle_THF_Room_2}
\end{figure}

\clearpage


\subsection{Ulster SBI Experiments}

Predicted and measured vertical heat flux profiles for three propane fire sizes in the single burning item (SBI) enclosure at the University of Ulster are shown on the following page. Measurements were made on two vertical panels that form a corner, at the base of which was a triangular-shaped burner with sides of length 25~cm. Three vertical profiles were measured on each panel at distances of 3.25~cm, 16.5~cm, and 29~cm from the corner.

\begin{figure}[h!]
\begin{tabular*}{\textwidth}{l@{\extracolsep{\fill}}r}
\includegraphics[height=2.15in]{SCRIPT_FIGURES/Ulster_SBI/Ulster_SBI_30_kW_Left_Heat_Flux} &
\includegraphics[height=2.15in]{SCRIPT_FIGURES/Ulster_SBI/Ulster_SBI_30_kW_Right_Heat_Flux} \\
\includegraphics[height=2.15in]{SCRIPT_FIGURES/Ulster_SBI/Ulster_SBI_45_kW_Left_Heat_Flux} &
\includegraphics[height=2.15in]{SCRIPT_FIGURES/Ulster_SBI/Ulster_SBI_45_kW_Right_Heat_Flux} \\
\includegraphics[height=2.15in]{SCRIPT_FIGURES/Ulster_SBI/Ulster_SBI_60_kW_Left_Heat_Flux} &
\includegraphics[height=2.15in]{SCRIPT_FIGURES/Ulster_SBI/Ulster_SBI_60_kW_Right_Heat_Flux}
\end{tabular*}
\label{Ulster_SBI}
\caption[Ulster SBI experiments, corner fire heat flux]
{Comparison of predicted (lines) and measured (circles) heat fluxes to adjacent panels forming a corner in the single
burning item (SBI) apparatus at the University of Ulster.}
\end{figure}

\clearpage


\subsection{WTC Experiments}

There were a variety of heat flux gauges installed in the test compartment. Most were within 2~m of the fire. Their locations and orientations are listed in Table~\ref{WTC_Gauges}. This section contains the measurements at the floor and ceiling.

\begin{table}[!h]
\caption{Heat flux gauge positions relative to the center of the fire pan in the WTC series.}
\begin{center}
\begin{tabular}{|l|c|c|c|c|l|}
\hline
Name    & $x$ (m)   & $y$ (m) & $z$ (m)   & Orientation  & Location              \\ \hline \hline
H2FU    & 0.64      & 0.63    & 3.30      &     $+z$     & Truss Support         \\ \hline
H2RU    & 0.64      & 0.51    & 3.30      &     $+z$     & Truss Support         \\ \hline
H2FD    & 0.64      & 0.30    & 3.15      &     $-z$     & Truss Support         \\ \hline
H2RD    & 0.64      & 0.42    & 3.15      &     $-z$     & Truss Support         \\ \hline
HCoHF   & -0.90     & 0.84    & 3.46      &     $+x$     & Column, facing fire   \\ \hline
HCoHW   & -0.97     & 0.92    & 3.27      &     $+y$     & Column, facing north  \\ \hline
HCoLF   & -0.90     & 0.84    & 0.92      &     $+x$     & Column, facing fire   \\ \hline
HCoLW   & -0.97     & 0.92    & 1.02      &     $+y$     & Column, facing north  \\ \hline
HF1     & 1.06      & 0.13    & 0.13      &     $+z$     & Floor                 \\ \hline
HF2     & 1.56      & 0.10    & 0.13      &     $+z$     & Floor                 \\ \hline
HCe1    & -0.45     & 0.35    & 3.82      &     $-z$     & Ceiling               \\ \hline
HCe2    &  0.05     & 0.35    & 3.82      &     $-z$     & Ceiling               \\ \hline
HCe3    &  0.80     & 0.35    & 3.82      &     $-z$     & Ceiling               \\ \hline
HCe4    &  2.56     & 0.35    & 3.82      &     $-z$     & Ceiling               \\ \hline
\end{tabular}
\end{center}
\label{WTC_Gauges}
\end{table}

\newpage

\begin{figure}[p]
\begin{tabular*}{\textwidth}{l@{\extracolsep{\fill}}r}
\includegraphics[height=2.15in]{SCRIPT_FIGURES/WTC/WTC_01_Floor_Flux} &
\includegraphics[height=2.15in]{SCRIPT_FIGURES/WTC/WTC_02_Floor_Flux} \\
\includegraphics[height=2.15in]{SCRIPT_FIGURES/WTC/WTC_03_Floor_Flux} &
\includegraphics[height=2.15in]{SCRIPT_FIGURES/WTC/WTC_04_Floor_Flux} \\
\includegraphics[height=2.15in]{SCRIPT_FIGURES/WTC/WTC_05_Floor_Flux} &
\includegraphics[height=2.15in]{SCRIPT_FIGURES/WTC/WTC_06_Floor_Flux}
\end{tabular*}
\caption{WTC experiments, heat flux to the floor.}
\label{NIST_WTC_Floor_Flux}
\end{figure}

\begin{figure}[p]
\begin{tabular*}{\textwidth}{l@{\extracolsep{\fill}}r}
\includegraphics[height=2.15in]{SCRIPT_FIGURES/WTC/WTC_01_Ceiling_Flux_1} &
\includegraphics[height=2.15in]{SCRIPT_FIGURES/WTC/WTC_02_Ceiling_Flux_1} \\
\includegraphics[height=2.15in]{SCRIPT_FIGURES/WTC/WTC_03_Ceiling_Flux_1} &
\includegraphics[height=2.15in]{SCRIPT_FIGURES/WTC/WTC_04_Ceiling_Flux_1} \\
\includegraphics[height=2.15in]{SCRIPT_FIGURES/WTC/WTC_05_Ceiling_Flux_1} &
\includegraphics[height=2.15in]{SCRIPT_FIGURES/WTC/WTC_06_Ceiling_Flux_1}
\end{tabular*}
\caption{WTC experiments, heat flux to the ceiling.}
\label{NIST_WTC_Ceiling_Flux_1}
\end{figure}

\begin{figure}[p]
\begin{tabular*}{\textwidth}{l@{\extracolsep{\fill}}r}
\includegraphics[height=2.15in]{SCRIPT_FIGURES/WTC/WTC_01_Ceiling_Flux_2} &
\includegraphics[height=2.15in]{SCRIPT_FIGURES/WTC/WTC_02_Ceiling_Flux_2} \\
\includegraphics[height=2.15in]{SCRIPT_FIGURES/WTC/WTC_03_Ceiling_Flux_2} &
\includegraphics[height=2.15in]{SCRIPT_FIGURES/WTC/WTC_04_Ceiling_Flux_2} \\
\includegraphics[height=2.15in]{SCRIPT_FIGURES/WTC/WTC_05_Ceiling_Flux_2} &
\includegraphics[height=2.15in]{SCRIPT_FIGURES/WTC/WTC_06_Ceiling_Flux_2}
\end{tabular*}
\caption{WTC experiments, heat flux to the ceiling.}
\label{NIST_WTC_Ceiling_Flux_2}
\end{figure}


\clearpage


\subsection{Summary of Wall, Ceiling and Floor Heat Flux Predictions}
\label{Surface Heat Flux}


\begin{figure}[h!]
\begin{center}
\begin{tabular}{c}
\includegraphics[width=3.5in]{SCRIPT_FIGURES/ScatterPlots/FDS_Surface_Heat_Flux}
\end{tabular}
\end{center}
\caption[Summary of compartment surface heat flux predictions]
{Summary of surface heat flux predictions.}
\label{Summary_Surface_Heat_Flux}
\end{figure}



\clearpage

\section{Heat Flux to Targets}

\subsection{Fleury Experiments}

The plots on the following pages contain comparisons of predicted and measured heat fluxes from a series of propane burner fires. Heat flux gauges were mounted on moveable dollies that were placed in front of, and to the side of, burners with dimensions of 0.3~m by 0.3~m (1:1 burner), 0.6~m by 0.3~m (2:1 burner), and 0.9~m by 0.3~m (3:1 burner). The heat release rates were set to 100~kW, 150~kW, 200~kW, 250~kW, and 300~kW. The gauges were mounted at heights of 0~m, 0.5~m, 1.0~m, and 1.5~m relative to the top edge of the burner. Each page contains the results for a given HRR.

\newpage

\begin{figure}[p]
\begin{tabular*}{\textwidth}{l@{\extracolsep{\fill}}r}
\includegraphics[height=2.15in]{SCRIPT_FIGURES/Fleury_Heat_Flux/Fleury_1t1_100_kW_Front_Heat_Flux} &
\includegraphics[height=2.15in]{SCRIPT_FIGURES/Fleury_Heat_Flux/Fleury_1t1_100_kW_Side_Heat_Flux} \\
\includegraphics[height=2.15in]{SCRIPT_FIGURES/Fleury_Heat_Flux/Fleury_2t1_100_kW_Front_Heat_Flux} &
\includegraphics[height=2.15in]{SCRIPT_FIGURES/Fleury_Heat_Flux/Fleury_2t1_100_kW_Side_Heat_Flux} \\
\includegraphics[height=2.15in]{SCRIPT_FIGURES/Fleury_Heat_Flux/Fleury_3t1_100_kW_Front_Heat_Flux} &
\includegraphics[height=2.15in]{SCRIPT_FIGURES/Fleury_Heat_Flux/Fleury_3t1_100_kW_Side_Heat_Flux}
\end{tabular*}
\label{Fleury_Heat_Flux_100_kW}
\caption[Fleury Heat Flux, 100 kW fires.]
{Comparison of predicted (lines) and measured (circles) heat flux for the 100~kW Fleury fires.}
\end{figure}

\begin{figure}[p]
\begin{tabular*}{\textwidth}{l@{\extracolsep{\fill}}r}
\includegraphics[height=2.15in]{SCRIPT_FIGURES/Fleury_Heat_Flux/Fleury_1t1_150_kW_Front_Heat_Flux} &
\includegraphics[height=2.15in]{SCRIPT_FIGURES/Fleury_Heat_Flux/Fleury_1t1_150_kW_Side_Heat_Flux} \\
\includegraphics[height=2.15in]{SCRIPT_FIGURES/Fleury_Heat_Flux/Fleury_2t1_150_kW_Front_Heat_Flux} &
\includegraphics[height=2.15in]{SCRIPT_FIGURES/Fleury_Heat_Flux/Fleury_2t1_150_kW_Side_Heat_Flux} \\
\includegraphics[height=2.15in]{SCRIPT_FIGURES/Fleury_Heat_Flux/Fleury_3t1_150_kW_Front_Heat_Flux} &
\includegraphics[height=2.15in]{SCRIPT_FIGURES/Fleury_Heat_Flux/Fleury_3t1_150_kW_Side_Heat_Flux}
\end{tabular*}
\label{Fleury_Heat_Flux_150_kW}
\caption[Fleury Heat Flux, 150 kW fires.]
{Comparison of predicted (lines) and measured (circles) heat flux for the 150~kW Fleury fires.}
\end{figure}

\begin{figure}[p]
\begin{tabular*}{\textwidth}{l@{\extracolsep{\fill}}r}
\includegraphics[height=2.15in]{SCRIPT_FIGURES/Fleury_Heat_Flux/Fleury_1t1_200_kW_Front_Heat_Flux} &
\includegraphics[height=2.15in]{SCRIPT_FIGURES/Fleury_Heat_Flux/Fleury_1t1_200_kW_Side_Heat_Flux} \\
\includegraphics[height=2.15in]{SCRIPT_FIGURES/Fleury_Heat_Flux/Fleury_2t1_200_kW_Front_Heat_Flux} &
\includegraphics[height=2.15in]{SCRIPT_FIGURES/Fleury_Heat_Flux/Fleury_2t1_200_kW_Side_Heat_Flux} \\
\includegraphics[height=2.15in]{SCRIPT_FIGURES/Fleury_Heat_Flux/Fleury_3t1_200_kW_Front_Heat_Flux} &
\includegraphics[height=2.15in]{SCRIPT_FIGURES/Fleury_Heat_Flux/Fleury_3t1_200_kW_Side_Heat_Flux}
\end{tabular*}
\label{Fleury_Heat_Flux_200_kW}
\caption[Fleury Heat Flux, 200 kW fires.]
{Comparison of predicted (lines) and measured (circles) heat flux for the 200~kW Fleury fires.}
\end{figure}

\begin{figure}[p]
\begin{tabular*}{\textwidth}{l@{\extracolsep{\fill}}r}
\includegraphics[height=2.15in]{SCRIPT_FIGURES/Fleury_Heat_Flux/Fleury_1t1_250_kW_Front_Heat_Flux} &
\includegraphics[height=2.15in]{SCRIPT_FIGURES/Fleury_Heat_Flux/Fleury_1t1_250_kW_Side_Heat_Flux} \\
\includegraphics[height=2.15in]{SCRIPT_FIGURES/Fleury_Heat_Flux/Fleury_2t1_250_kW_Front_Heat_Flux} &
\includegraphics[height=2.15in]{SCRIPT_FIGURES/Fleury_Heat_Flux/Fleury_2t1_250_kW_Side_Heat_Flux} \\
\includegraphics[height=2.15in]{SCRIPT_FIGURES/Fleury_Heat_Flux/Fleury_3t1_250_kW_Front_Heat_Flux} &
\includegraphics[height=2.15in]{SCRIPT_FIGURES/Fleury_Heat_Flux/Fleury_3t1_250_kW_Side_Heat_Flux}
\end{tabular*}
\label{Fleury_Heat_Flux_250_kW}
\caption[Fleury Heat Flux, 250 kW fires.]
{Comparison of predicted (lines) and measured (circles) heat flux for the 250~kW Fleury fires.}
\end{figure}

\begin{figure}[p]
\begin{tabular*}{\textwidth}{l@{\extracolsep{\fill}}r}
\includegraphics[height=2.15in]{SCRIPT_FIGURES/Fleury_Heat_Flux/Fleury_1t1_300_kW_Front_Heat_Flux} &
\includegraphics[height=2.15in]{SCRIPT_FIGURES/Fleury_Heat_Flux/Fleury_1t1_300_kW_Side_Heat_Flux} \\
\includegraphics[height=2.15in]{SCRIPT_FIGURES/Fleury_Heat_Flux/Fleury_2t1_300_kW_Front_Heat_Flux} &
\includegraphics[height=2.15in]{SCRIPT_FIGURES/Fleury_Heat_Flux/Fleury_2t1_300_kW_Side_Heat_Flux} \\
\includegraphics[height=2.15in]{SCRIPT_FIGURES/Fleury_Heat_Flux/Fleury_3t1_300_kW_Front_Heat_Flux} &
\includegraphics[height=2.15in]{SCRIPT_FIGURES/Fleury_Heat_Flux/Fleury_3t1_300_kW_Side_Heat_Flux}
\end{tabular*}
\label{Fleury_Heat_Flux_300_kW}
\caption[Fleury Heat Flux, 300 kW fires.]
{Comparison of predicted (lines) and measured (circles) heat flux for the 300~kW Fleury fires.}
\end{figure}

\clearpage

\subsection{Hamins Methane Burner Experiments}

Predicted and measured radial and vertical heat flux profiles from six experiments conducted by Anthony Hamins at NIST are shown on the following pages. The relevant information about the fires is included in Table~\ref{Hamins_Table}. These are challenging simulations because the neither the gray gas assumption nor the radiative fraction is assumed. Rather, the model is calculating the temperature and species concentrations necessary to predict the radiant energy from the fire.

\begin{table}[ht]
\caption{Summary of Hamins methane burner experiments. }
\begin{center}
\begin{tabular}{|c|c|c|c|c|c|}
\hline
Case     & Test     & $D$  & $\dot{Q}$   &  $\dot{Q}''$   & $Q^*$   \\
         & Number   & (m)  & (kW)        &  (kW/m$^2$)    &         \\ \hline \hline
A        & 1        & 0.10 & 0.42        &  53.8          & 0.12    \\ \hline
B        & 5        & 0.10 & 1.88        &  240           & 0.53    \\ \hline
C        & 23       & 0.38 & 33.5        &  295           & 0.34    \\ \hline
D        & 21       & 0.38 & 175         &  1550          & 1.8     \\ \hline
E        & 7        & 1.0  & 49.0        &  62.4          & 0.044   \\ \hline
F        & 19       & 1.0  & 162         &  206           & 0.14    \\ \hline
\end{tabular}
\end{center}
\label{Hamins_Table}
\end{table}

\newpage

\begin{figure}[p]
\begin{tabular*}{\textwidth}{l@{\extracolsep{\fill}}r}
\includegraphics[height=2.15in]{SCRIPT_FIGURES/Hamins_CH4/Hamins_CH4_01_Radial_Heat_Flux} &
\includegraphics[height=2.15in]{SCRIPT_FIGURES/Hamins_CH4/Hamins_CH4_05_Radial_Heat_Flux} \\
\includegraphics[height=2.15in]{SCRIPT_FIGURES/Hamins_CH4/Hamins_CH4_23_Radial_Heat_Flux} &
\includegraphics[height=2.15in]{SCRIPT_FIGURES/Hamins_CH4/Hamins_CH4_21_Radial_Heat_Flux} \\
\includegraphics[height=2.15in]{SCRIPT_FIGURES/Hamins_CH4/Hamins_CH4_07_Radial_Heat_Flux} &
\includegraphics[height=2.15in]{SCRIPT_FIGURES/Hamins_CH4/Hamins_CH4_19_Radial_Heat_Flux}
\end{tabular*}
\label{Hamins_CH4_Radial}
\caption[Radial heat flux predictions, Hamins methane burner experiments.]
{Comparison of predicted and measured heat fluxes to the ``floor'' as a function of radial distance from a methane burner, Hamins experiments.}
\end{figure}

\begin{figure}[p]
\begin{tabular*}{\textwidth}{l@{\extracolsep{\fill}}r}
\includegraphics[height=2.15in]{SCRIPT_FIGURES/Hamins_CH4/Hamins_CH4_01_Vertical_Heat_Flux} &
\includegraphics[height=2.15in]{SCRIPT_FIGURES/Hamins_CH4/Hamins_CH4_05_Vertical_Heat_Flux} \\
\includegraphics[height=2.15in]{SCRIPT_FIGURES/Hamins_CH4/Hamins_CH4_23_Vertical_Heat_Flux} &
\includegraphics[height=2.15in]{SCRIPT_FIGURES/Hamins_CH4/Hamins_CH4_21_Vertical_Heat_Flux} \\
\includegraphics[height=2.15in]{SCRIPT_FIGURES/Hamins_CH4/Hamins_CH4_07_Vertical_Heat_Flux} &
\includegraphics[height=2.15in]{SCRIPT_FIGURES/Hamins_CH4/Hamins_CH4_19_Vertical_Heat_Flux}
\end{tabular*}
\label{Hamins_CH4_Vertical}
\caption[Vertical heat flux predictions, Hamins methane burner experiments.]
{Comparison of predicted and measured heat fluxes from a methane burner to a ``wall'' as a function of the height from the burner surface, Hamins experiments.}
\end{figure}



\clearpage


\subsection{NIST/NRC Experiments}

Cables in various types (power and control), and configurations (horizontal, vertical, in trays or free-hanging), were installed in
the test compartment. For each of the four cable targets considered, measurements of the radiative and total heat flux were made with
gauges positioned near the cables themselves.  The following pages display comparisons of these heat flux predictions and measurements for
Control Cable B, Horizontal Cable Tray D, Power Cable F and Vertical Cable Tray G.

\newpage

\begin{figure}[p]
\begin{tabular*}{\textwidth}{l@{\extracolsep{\fill}}r}
\includegraphics[height=2.15in]{SCRIPT_FIGURES/NIST_NRC/NIST_NRC_01_Cable_B_Flux} &
\includegraphics[height=2.15in]{SCRIPT_FIGURES/NIST_NRC/NIST_NRC_07_Cable_B_Flux} \\
\includegraphics[height=2.15in]{SCRIPT_FIGURES/NIST_NRC/NIST_NRC_02_Cable_B_Flux} &
\includegraphics[height=2.15in]{SCRIPT_FIGURES/NIST_NRC/NIST_NRC_08_Cable_B_Flux} \\
\includegraphics[height=2.15in]{SCRIPT_FIGURES/NIST_NRC/NIST_NRC_04_Cable_B_Flux} &
\includegraphics[height=2.15in]{SCRIPT_FIGURES/NIST_NRC/NIST_NRC_10_Cable_B_Flux} \\
\includegraphics[height=2.15in]{SCRIPT_FIGURES/NIST_NRC/NIST_NRC_13_Cable_B_Flux} &
\includegraphics[height=2.15in]{SCRIPT_FIGURES/NIST_NRC/NIST_NRC_16_Cable_B_Flux}
\end{tabular*}
\caption{NIST/NRC experiments, heat flux to Cable B, Tests 1, 2, 4, 7, 8, 10, 13, 16.}
\label{NIST_NRC_Cable_B_Flux_Closed}
\end{figure}

\begin{figure}[p]
\begin{tabular*}{\textwidth}{l@{\extracolsep{\fill}}r}
\includegraphics[height=2.15in]{SCRIPT_FIGURES/NIST_NRC/NIST_NRC_03_Cable_B_Flux} &
\includegraphics[height=2.15in]{SCRIPT_FIGURES/NIST_NRC/NIST_NRC_09_Cable_B_Flux} \\
\includegraphics[height=2.15in]{SCRIPT_FIGURES/NIST_NRC/NIST_NRC_05_Cable_B_Flux} &
\includegraphics[height=2.15in]{SCRIPT_FIGURES/NIST_NRC/NIST_NRC_14_Cable_B_Flux} \\
\includegraphics[height=2.15in]{SCRIPT_FIGURES/NIST_NRC/NIST_NRC_15_Cable_B_Flux} &
\includegraphics[height=2.15in]{SCRIPT_FIGURES/NIST_NRC/NIST_NRC_18_Cable_B_Flux}
\end{tabular*}
\caption{NIST/NRC experiments, heat flux to Cable B, Tests 3, 5, 9, 14, 15, 18.}
\label{NIST_NRC_Cable_B_Flux_Open}
\end{figure}

\begin{figure}[p]
\begin{tabular*}{\textwidth}{l@{\extracolsep{\fill}}r}
\includegraphics[height=2.15in]{SCRIPT_FIGURES/NIST_NRC/NIST_NRC_01_Cable_D_Flux} &
\includegraphics[height=2.15in]{SCRIPT_FIGURES/NIST_NRC/NIST_NRC_07_Cable_D_Flux} \\
\includegraphics[height=2.15in]{SCRIPT_FIGURES/NIST_NRC/NIST_NRC_02_Cable_D_Flux} &
\includegraphics[height=2.15in]{SCRIPT_FIGURES/NIST_NRC/NIST_NRC_08_Cable_D_Flux} \\
\includegraphics[height=2.15in]{SCRIPT_FIGURES/NIST_NRC/NIST_NRC_04_Cable_D_Flux} &
\includegraphics[height=2.15in]{SCRIPT_FIGURES/NIST_NRC/NIST_NRC_10_Cable_D_Flux} \\
\includegraphics[height=2.15in]{SCRIPT_FIGURES/NIST_NRC/NIST_NRC_13_Cable_D_Flux} &
\includegraphics[height=2.15in]{SCRIPT_FIGURES/NIST_NRC/NIST_NRC_16_Cable_D_Flux}
\end{tabular*}
\caption{NIST/NRC experiments, heat flux to Cable D, Tests 1, 2, 4, 7, 8, 10, 13, 16.}
\label{NIST_NRC_Cable_D_Flux_Closed}
\end{figure}

\begin{figure}[p]
\begin{tabular*}{\textwidth}{l@{\extracolsep{\fill}}r}
                           &
\includegraphics[height=2.15in]{SCRIPT_FIGURES/NIST_NRC/NIST_NRC_09_Cable_D_Flux} \\
\includegraphics[height=2.15in]{SCRIPT_FIGURES/NIST_NRC/NIST_NRC_05_Cable_D_Flux} &
\includegraphics[height=2.15in]{SCRIPT_FIGURES/NIST_NRC/NIST_NRC_14_Cable_D_Flux} \\
                      &
\end{tabular*}
\caption{NIST/NRC experiments, heat flux to Cable D, Tests 5, 9, 14.}
\label{NIST_NRC_Cable_D_Flux_Open}
\end{figure}

\begin{figure}[p]
\begin{tabular*}{\textwidth}{l@{\extracolsep{\fill}}r}
\includegraphics[height=2.15in]{SCRIPT_FIGURES/NIST_NRC/NIST_NRC_01_Cable_F_Flux} &
\includegraphics[height=2.15in]{SCRIPT_FIGURES/NIST_NRC/NIST_NRC_07_Cable_F_Flux} \\
\includegraphics[height=2.15in]{SCRIPT_FIGURES/NIST_NRC/NIST_NRC_02_Cable_F_Flux} &
\includegraphics[height=2.15in]{SCRIPT_FIGURES/NIST_NRC/NIST_NRC_08_Cable_F_Flux} \\
\includegraphics[height=2.15in]{SCRIPT_FIGURES/NIST_NRC/NIST_NRC_04_Cable_F_Flux} &
\includegraphics[height=2.15in]{SCRIPT_FIGURES/NIST_NRC/NIST_NRC_10_Cable_F_Flux} \\
\includegraphics[height=2.15in]{SCRIPT_FIGURES/NIST_NRC/NIST_NRC_13_Cable_F_Flux} &
\includegraphics[height=2.15in]{SCRIPT_FIGURES/NIST_NRC/NIST_NRC_16_Cable_F_Flux}
\end{tabular*}
\caption{NIST/NRC experiments, heat flux to Cable F, Tests 1, 2, 4, 7, 8, 10, 13, 16.}
\label{NIST_NRC_Cable_F_Flux_Closed}
\end{figure}

\begin{figure}[p]
\begin{tabular*}{\textwidth}{l@{\extracolsep{\fill}}r}
\includegraphics[height=2.15in]{SCRIPT_FIGURES/NIST_NRC/NIST_NRC_03_Cable_F_Flux} &
\includegraphics[height=2.15in]{SCRIPT_FIGURES/NIST_NRC/NIST_NRC_09_Cable_F_Flux} \\
\includegraphics[height=2.15in]{SCRIPT_FIGURES/NIST_NRC/NIST_NRC_05_Cable_F_Flux} &
\includegraphics[height=2.15in]{SCRIPT_FIGURES/NIST_NRC/NIST_NRC_14_Cable_F_Flux} \\
\includegraphics[height=2.15in]{SCRIPT_FIGURES/NIST_NRC/NIST_NRC_15_Cable_F_Flux} &
\includegraphics[height=2.15in]{SCRIPT_FIGURES/NIST_NRC/NIST_NRC_18_Cable_F_Flux}
\end{tabular*}
\caption{NIST/NRC experiments, heat flux to Cable F, Tests 3, 5, 9, 14, 15, 18.}
\label{NIST_NRC_Cable_F_Flux_Open}
\end{figure}

\begin{figure}[p]
\begin{tabular*}{\textwidth}{l@{\extracolsep{\fill}}r}
\includegraphics[height=2.15in]{SCRIPT_FIGURES/NIST_NRC/NIST_NRC_01_Cable_G_Flux} &
\includegraphics[height=2.15in]{SCRIPT_FIGURES/NIST_NRC/NIST_NRC_07_Cable_G_Flux} \\
\includegraphics[height=2.15in]{SCRIPT_FIGURES/NIST_NRC/NIST_NRC_02_Cable_G_Flux} &
\includegraphics[height=2.15in]{SCRIPT_FIGURES/NIST_NRC/NIST_NRC_08_Cable_G_Flux} \\
\includegraphics[height=2.15in]{SCRIPT_FIGURES/NIST_NRC/NIST_NRC_04_Cable_G_Flux} &
\includegraphics[height=2.15in]{SCRIPT_FIGURES/NIST_NRC/NIST_NRC_10_Cable_G_Flux} \\
\includegraphics[height=2.15in]{SCRIPT_FIGURES/NIST_NRC/NIST_NRC_13_Cable_G_Flux} &
\includegraphics[height=2.15in]{SCRIPT_FIGURES/NIST_NRC/NIST_NRC_16_Cable_G_Flux}
\end{tabular*}
\caption{NIST/NRC experiments, heat flux to Cable G, Tests 1, 2, 4, 7, 8, 10, 13, 16.}
\label{NIST_NRC_Cable_G_Flux_Closed}
\end{figure}

\begin{figure}[p]
\begin{tabular*}{\textwidth}{l@{\extracolsep{\fill}}r}
\includegraphics[height=2.15in]{SCRIPT_FIGURES/NIST_NRC/NIST_NRC_03_Cable_G_Flux} &
\includegraphics[height=2.15in]{SCRIPT_FIGURES/NIST_NRC/NIST_NRC_09_Cable_G_Flux} \\
\includegraphics[height=2.15in]{SCRIPT_FIGURES/NIST_NRC/NIST_NRC_05_Cable_G_Flux} &
\includegraphics[height=2.15in]{SCRIPT_FIGURES/NIST_NRC/NIST_NRC_14_Cable_G_Flux} \\
\includegraphics[height=2.15in]{SCRIPT_FIGURES/NIST_NRC/NIST_NRC_15_Cable_G_Flux} &
\includegraphics[height=2.15in]{SCRIPT_FIGURES/NIST_NRC/NIST_NRC_18_Cable_G_Flux}
\end{tabular*}
\caption{NIST/NRC experiments, heat flux to Cable G, Tests 3, 5, 9, 14, 15, 18.}
\label{NIST_NRC_Cable_G_Flux_Open}
\end{figure}

\clearpage

\subsection{WTC Experiments}

There were a variety of heat flux gauges installed in the test compartment. Most were within 2~m of the fire. Their locations and orientations are listed in Table~\ref{WTC_Gauges}.


\begin{figure}[h!]
\begin{tabular*}{\textwidth}{l@{\extracolsep{\fill}}r}
\includegraphics[height=2.15in]{SCRIPT_FIGURES/WTC/WTC_01_Station_2_Flux_High} &
\includegraphics[height=2.15in]{SCRIPT_FIGURES/WTC/WTC_02_Station_2_Flux_High} \\
\includegraphics[height=2.15in]{SCRIPT_FIGURES/WTC/WTC_03_Station_2_Flux_High} &
\includegraphics[height=2.15in]{SCRIPT_FIGURES/WTC/WTC_04_Station_2_Flux_High} \\
\includegraphics[height=2.15in]{SCRIPT_FIGURES/WTC/WTC_05_Station_2_Flux_High} &
\includegraphics[height=2.15in]{SCRIPT_FIGURES/WTC/WTC_06_Station_2_Flux_High}
\end{tabular*}
\caption{WTC experiments, heat flux at Station 2, high position.}
\label{NIST_WTC_Station_2_Flux_High}
\end{figure}

\newpage

\begin{figure}[p]
\begin{tabular*}{\textwidth}{l@{\extracolsep{\fill}}r}
\includegraphics[height=2.15in]{SCRIPT_FIGURES/WTC/WTC_01_Station_2_Flux_Low} &
\includegraphics[height=2.15in]{SCRIPT_FIGURES/WTC/WTC_02_Station_2_Flux_Low} \\
\includegraphics[height=2.15in]{SCRIPT_FIGURES/WTC/WTC_03_Station_2_Flux_Low} &
\includegraphics[height=2.15in]{SCRIPT_FIGURES/WTC/WTC_04_Station_2_Flux_Low} \\
\includegraphics[height=2.15in]{SCRIPT_FIGURES/WTC/WTC_05_Station_2_Flux_Low} &
\includegraphics[height=2.15in]{SCRIPT_FIGURES/WTC/WTC_06_Station_2_Flux_Low}
\end{tabular*}
\caption{WTC experiments, heat flux at Station 2, low position.}
\label{NIST_WTC_Station_2_Flux_Low}
\end{figure}

\begin{figure}[p]
\begin{tabular*}{\textwidth}{l@{\extracolsep{\fill}}r}
\includegraphics[height=2.15in]{SCRIPT_FIGURES/WTC/WTC_01_Upper_Column_Flux} &
\includegraphics[height=2.15in]{SCRIPT_FIGURES/WTC/WTC_02_Upper_Column_Flux} \\
\includegraphics[height=2.15in]{SCRIPT_FIGURES/WTC/WTC_03_Upper_Column_Flux} &
\includegraphics[height=2.15in]{SCRIPT_FIGURES/WTC/WTC_04_Upper_Column_Flux} \\
\includegraphics[height=2.15in]{SCRIPT_FIGURES/WTC/WTC_05_Upper_Column_Flux} &
\includegraphics[height=2.15in]{SCRIPT_FIGURES/WTC/WTC_06_Upper_Column_Flux}
\end{tabular*}
\caption{WTC experiments, heat flux to upper column.}
\label{NIST_WTC_Upper_Column_Flux}
\end{figure}

\begin{figure}[p]
\begin{tabular*}{\textwidth}{l@{\extracolsep{\fill}}r}
\includegraphics[height=2.15in]{SCRIPT_FIGURES/WTC/WTC_01_Lower_Column_Flux} &
\includegraphics[height=2.15in]{SCRIPT_FIGURES/WTC/WTC_02_Lower_Column_Flux} \\
\includegraphics[height=2.15in]{SCRIPT_FIGURES/WTC/WTC_03_Lower_Column_Flux} &
\includegraphics[height=2.15in]{SCRIPT_FIGURES/WTC/WTC_04_Lower_Column_Flux} \\
\includegraphics[height=2.15in]{SCRIPT_FIGURES/WTC/WTC_05_Lower_Column_Flux} &
\includegraphics[height=2.15in]{SCRIPT_FIGURES/WTC/WTC_06_Lower_Column_Flux}
\end{tabular*}
\caption{WTC experiments, heat flux to lower column.}
\label{NIST_WTC_Lower_Column_Flux}
\end{figure}


\clearpage

\subsection{Summary of Target Heat Flux Predictions}
\label{Target Heat Flux}

\begin{figure}[h!]
\begin{center}
\begin{tabular}{c}
\includegraphics[width=3.5in]{SCRIPT_FIGURES/ScatterPlots/FDS_Target_Heat_Flux}
\end{tabular}
\end{center}
\caption[Summary of target heat flux predictions]
{Summary of target heat flux predictions.}
\end{figure}



\clearpage

\section{Attenuation of Thermal Radiation in Water Spray}

This section presents the results of simulations of spray experiments where the reduction of thermal radiation by a fine water spray was measured.

\subsection{BRE Spray Experiments}

Attenuation of thermal radiation by a water spray was measured using three full-cone type hydraulic nozzles at eight different pressures. The initial droplet speeds were determined using a simple hydraulic relation, $v = 0.9 \sqrt{2P/\rho}$. The median drop size distributions were determined by assuming $d_m \propto p^{-1/3}$ and finding the constant of proportionality by fitting to the experimental PDPA measurement 1~m below the nozzles.  Measured median diameters, $d_{v50}$, are compared against mean diameters, $d_{43}$. The arithmetic mean of the droplets is used for vertical velocity. The comparison of predicted and measured attenuation, Fig.~\ref{BRE_LEMTA_Spray_Attenuation}, is made at a distance of 4~m from the heat source.

\begin{figure}[h!]
\begin{tabular*}{\textwidth}{l@{\extracolsep{\fill}}r}
\includegraphics[width=2.8in]{SCRIPT_FIGURES/BRE_LEMTA_Spray/BRE_Spray_W} &
\includegraphics[width=2.8in]{SCRIPT_FIGURES/BRE_LEMTA_Spray/BRE_Spray_Diameter}
\end{tabular*}
\caption[Droplet speeds and mean diameters for the three nozzles]{Comparison of experimental and predicted droplet speeds and mean diameters for the three nozzles and different pressures.}
\label{BRE_Spray_W_and_diam}
\end{figure}

\begin{figure}[h!]
\begin{center}
\begin{tabular}{c}
\includegraphics[width=3.5in]{SCRIPT_FIGURES/BRE_LEMTA_Spray/BRE_LEMTA_Spray_Attenuation}
\end{tabular}
\end{center}
\caption[Comparison of radiation attenuation, BRE and LEMTA Spray experiments]{Comparison of predicted and measured radiation attenuation in the spray experiments at BRE and at LEMTA.}
\label{BRE_LEMTA_Spray_Attenuation}
\end{figure}


\subsection{LEMTA Spray Experiments}

The attenuation of thermal radiation was measured at five heights in water sprays produced by seven full-elliptic type hydraulic nozzles. The operating pressure was 4~bar. The initial speed was deduced from the water flow rate and the orifice diameter. The droplet size at the injection point was determined by comparing the predicted and measured results at the PDPA measurement location 0.2~m below the nozzles. The comparison of predicted and measured attenuations, Fig.~\ref{BRE_LEMTA_Spray_Attenuation}, is made at five locations.

\clearpage

\section{Convective Heat Flux}

This section focuses specifically on experiments that primarily involved convective heat transfer.

\subsection{Bouchair Solar Chimney}

The plots on the following pages compare the predicted air mass flow rates through the test apparatus shown in Fig.~\ref{Bouchair_Drawing}. The measurements were made at both the inlet and outlet of the thermal cavity. Note that in Bouchair's thesis~\cite{Bouchair:Thesis}, the measurements were presented as mass flow rates per unit length of the inlet slot, 1.4~m. In the plots on the following pages, the measurements and simulation results are presented simply as a total mass flux, kg/s.

\begin{figure}[p]
\begin{tabular*}{\textwidth}{l@{\extracolsep{\fill}}r}
\includegraphics[height=2.15in]{SCRIPT_FIGURES/Bouchair_Solar_Chimney/SC_p1_p1_30} &
\includegraphics[height=2.15in]{SCRIPT_FIGURES/Bouchair_Solar_Chimney/SC_p1_p1_40} \\
\includegraphics[height=2.15in]{SCRIPT_FIGURES/Bouchair_Solar_Chimney/SC_p1_p1_50} &
\includegraphics[height=2.15in]{SCRIPT_FIGURES/Bouchair_Solar_Chimney/SC_p1_p1_60} \\
\includegraphics[height=2.15in]{SCRIPT_FIGURES/Bouchair_Solar_Chimney/SC_p4_p1_30} &
\includegraphics[height=2.15in]{SCRIPT_FIGURES/Bouchair_Solar_Chimney/SC_p4_p1_40} \\
\includegraphics[height=2.15in]{SCRIPT_FIGURES/Bouchair_Solar_Chimney/SC_p4_p1_50} &
\includegraphics[height=2.15in]{SCRIPT_FIGURES/Bouchair_Solar_Chimney/SC_p4_p1_60}
\end{tabular*}
\caption{Bouchair Solar Chimney, 0.1 m thermal cavity.}
\label{Bouchair_p1}
\end{figure}

\begin{figure}[p]
\begin{tabular*}{\textwidth}{l@{\extracolsep{\fill}}r}
\includegraphics[height=2.15in]{SCRIPT_FIGURES/Bouchair_Solar_Chimney/SC_p1_p2_30} &
\includegraphics[height=2.15in]{SCRIPT_FIGURES/Bouchair_Solar_Chimney/SC_p1_p2_40} \\
\includegraphics[height=2.15in]{SCRIPT_FIGURES/Bouchair_Solar_Chimney/SC_p1_p2_50} &
\includegraphics[height=2.15in]{SCRIPT_FIGURES/Bouchair_Solar_Chimney/SC_p1_p2_60} \\
\includegraphics[height=2.15in]{SCRIPT_FIGURES/Bouchair_Solar_Chimney/SC_p4_p2_30} &
\includegraphics[height=2.15in]{SCRIPT_FIGURES/Bouchair_Solar_Chimney/SC_p4_p2_40} \\
\includegraphics[height=2.15in]{SCRIPT_FIGURES/Bouchair_Solar_Chimney/SC_p4_p2_50} &
\includegraphics[height=2.15in]{SCRIPT_FIGURES/Bouchair_Solar_Chimney/SC_p4_p2_60}
\end{tabular*}
\caption{Bouchair Solar Chimney, 0.2 m thermal cavity.}
\label{Bouchair_p2}
\end{figure}

\begin{figure}[p]
\begin{tabular*}{\textwidth}{l@{\extracolsep{\fill}}r}
\includegraphics[height=2.15in]{SCRIPT_FIGURES/Bouchair_Solar_Chimney/SC_p1_p3_30} &
\includegraphics[height=2.15in]{SCRIPT_FIGURES/Bouchair_Solar_Chimney/SC_p1_p3_40} \\
\includegraphics[height=2.15in]{SCRIPT_FIGURES/Bouchair_Solar_Chimney/SC_p1_p3_50} &
\includegraphics[height=2.15in]{SCRIPT_FIGURES/Bouchair_Solar_Chimney/SC_p1_p3_60} \\
\includegraphics[height=2.15in]{SCRIPT_FIGURES/Bouchair_Solar_Chimney/SC_p4_p3_30} &
\includegraphics[height=2.15in]{SCRIPT_FIGURES/Bouchair_Solar_Chimney/SC_p4_p3_40} \\
\includegraphics[height=2.15in]{SCRIPT_FIGURES/Bouchair_Solar_Chimney/SC_p4_p3_50} &
\includegraphics[height=2.15in]{SCRIPT_FIGURES/Bouchair_Solar_Chimney/SC_p4_p3_60}
\end{tabular*}
\caption{Bouchair Solar Chimney, 0.3 m thermal cavity.}
\label{Bouchair_p3}
\end{figure}

\begin{figure}[p]
\begin{tabular*}{\textwidth}{l@{\extracolsep{\fill}}r}
\includegraphics[height=2.15in]{SCRIPT_FIGURES/Bouchair_Solar_Chimney/SC_p1_p5_30} &
\includegraphics[height=2.15in]{SCRIPT_FIGURES/Bouchair_Solar_Chimney/SC_p1_p5_40} \\
\includegraphics[height=2.15in]{SCRIPT_FIGURES/Bouchair_Solar_Chimney/SC_p1_p5_50} &
\includegraphics[height=2.15in]{SCRIPT_FIGURES/Bouchair_Solar_Chimney/SC_p1_p5_60} \\
\includegraphics[height=2.15in]{SCRIPT_FIGURES/Bouchair_Solar_Chimney/SC_p4_p5_30} &
\includegraphics[height=2.15in]{SCRIPT_FIGURES/Bouchair_Solar_Chimney/SC_p4_p5_40} \\
\includegraphics[height=2.15in]{SCRIPT_FIGURES/Bouchair_Solar_Chimney/SC_p4_p5_50} &
\includegraphics[height=2.15in]{SCRIPT_FIGURES/Bouchair_Solar_Chimney/SC_p4_p5_60}
\end{tabular*}
\caption{Bouchair Solar Chimney, 0.5 m thermal cavity.}
\label{Bouchair_p5}
\end{figure}

\begin{figure}[p]
\begin{tabular*}{\textwidth}{l@{\extracolsep{\fill}}r}
\includegraphics[height=2.15in]{SCRIPT_FIGURES/Bouchair_Solar_Chimney/SC_p1_1p_30} &
\includegraphics[height=2.15in]{SCRIPT_FIGURES/Bouchair_Solar_Chimney/SC_p1_1p_40} \\
\includegraphics[height=2.15in]{SCRIPT_FIGURES/Bouchair_Solar_Chimney/SC_p1_1p_50} &
\includegraphics[height=2.15in]{SCRIPT_FIGURES/Bouchair_Solar_Chimney/SC_p1_1p_60} \\
\includegraphics[height=2.15in]{SCRIPT_FIGURES/Bouchair_Solar_Chimney/SC_p4_1p_30} &
\includegraphics[height=2.15in]{SCRIPT_FIGURES/Bouchair_Solar_Chimney/SC_p4_1p_40} \\
\includegraphics[height=2.15in]{SCRIPT_FIGURES/Bouchair_Solar_Chimney/SC_p4_1p_50} &
\includegraphics[height=2.15in]{SCRIPT_FIGURES/Bouchair_Solar_Chimney/SC_p4_1p_60}
\end{tabular*}
\caption{Bouchair Solar Chimney, 1.0 m thermal cavity.}
\label{Bouchair_1p}
\end{figure}

\begin{figure}[h!]
\begin{center}
\begin{tabular}{c}
\includegraphics[width=3.5in]{SCRIPT_FIGURES/ScatterPlots/FDS_Convective_Heat_Flux}
\end{tabular}
\end{center}
\caption{Summary of Bouchair Solar Chimney results.}
\label{Mass Flow Rate}
\end{figure}


