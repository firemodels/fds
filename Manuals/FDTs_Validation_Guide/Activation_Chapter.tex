\chapter{Ceiling Jets and Device Activation}

\section{Smoke Detector Activation Time (Milke)}

The operation of detectors for smoke from typical fuels has been correlated to a temperature change from the following correlation

\be
t_{activation} = \frac{X H^{4/3}}{\dot Q^{1/3}}
\ee

\noindent where

\be
X = 4.6 \times 10^{-4} (Y^2) + 2.7 \times 10^{-15} (Y^6)
\ee

\be
Y = \frac{\Delta T_c H^{5/3}}{\dot Q^{2/3}}
\ee

\noindent Note that $\Delta T_c$ has units of $^\circ F$ and H has units of ft.

\noindent Note that some or all of these cases use a t-squared fire source as a quasi-steady approach.

\clearpage

A summary of smoke detector activation time predictions is given in Fig.~\ref{smoke_detector_activation_milke_summary}. 

\begin{figure}[ht]
\begin{center}
\begin{tabular}{l}
\includegraphics[width=4.0in]{FIGURES/Scatterplots/Smoke_Detector_Activation_Time_Milke}
\end{tabular}
\end{center}
\caption[Summary of smoke detector activation time predictions (Milke).]
{Summary of smoke detector activation time predictions (Milke).}
\label{smoke_detector_activation_milke_summary}
\end{figure}


\clearpage


\section{Smoke Detector Activation Time (Mowrer)}

The response time of a smoke detector comprises two separate times, including the transport lag time of the plume and the transport lag time of the ceiling jet, as illustrated by the following equation

\be
t_{activation} = t_{pl} + t_{cj}
\ee

\noindent where

\be
t_{pl} = C_{pl} \frac{H^{4/3}}{\dot Q^{1/3}}
\ee

\be
t_{cj} = \frac{1}{C_{cj}} \frac{r^{11/6}}{\dot Q^{1/3} H^{1/2}}
\ee

\noindent Note that some or all of these cases use a t-squared fire source as a quasi-steady approach.

\clearpage

A summary of smoke detector activation time predictions is given in Fig.~\ref{smoke_detector_activation_mowrer_summary}. 

\begin{figure}[ht]
\begin{center}
\begin{tabular}{l}
\includegraphics[width=4.0in]{FIGURES/Scatterplots/Smoke_Detector_Activation_Time_Mowrer}
\end{tabular}
\end{center}
\caption[Summary of smoke detector activation time predictions (Mowrer).]
{Summary of smoke detector activation time predictions (Mowrer).}
\label{smoke_detector_activation_mowrer_summary}
\end{figure}


\clearpage


\section{Sprinkler Activation Time (Alpert)}

For steady-state fire, the method for estimating the response of a smoke detector is based on the correlations developed by Alpert (1972) for activation of a sprinkler and is given by the following equation (Budnick, Evans, and Nelson, 1997)

\be
t_{activation} =  \frac{RTI}{\sqrt{u_{jet}}} ln \left( \frac{T_{jet} - T_\infty}{T_{jet} - T_{activation}} \right)
\ee

\noindent where

\be
T_{jet} = \frac{16.9 \dot Q^{2/3}}{H^{5/3}} + T_\infty \mathrm{\ for\ } r/H <= 0.18
\ee

\be
T_{jet} = \frac{5.38 (\dot Q / r)^{2/3}}{H} + T_\infty \mathrm{\ for\ } r/H > 0.18
\ee

\noindent and

\be
u_{jet} = 0.96 \left( \frac{\dot Q}{H} \right)^{1/3} \mathrm{\ for\ } r/H <= 0.15
\ee

\be
u_{jet} = \frac{0.195 \dot Q^{1/3} H^{1/2}}{r^{5/6}} \mathrm{\ for\ } r/H > 0.15
\ee

\noindent Note that $\dot Q$ is the total HRR and not the convective HRR.

\noindent Note that some or all of these cases use a t-squared fire source as a quasi-steady approach.

\clearpage

A summary of sprinkler activation time predictions is given in Fig.~\ref{sprinkler_activation_summary}. 

\begin{figure}[ht]
\begin{center}
\begin{tabular}{l}
\includegraphics[width=4.0in]{FIGURES/Scatterplots/Sprinkler_Activation_Time}
\end{tabular}
\end{center}
\caption[Summary of sprinkler activation time predictions.]
{Summary of sprinkler activation time predictions.}
\label{sprinkler_activation_summary}
\end{figure}

