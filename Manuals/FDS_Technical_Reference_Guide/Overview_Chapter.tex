\chapter{Model Overview}

This chapter presents general information about the Fire Dynamics Simulator, following the basic framework set forth in
ASTM E 1355~\cite{ASTM:E1355}.

\section{Basic Description of FDS}


\subsection{Type of Model}

FDS is a Computational Fluid Dynamics (CFD) model of fire-driven fluid flow. The model numerically solves a form of the Navier-Stokes equations appropriate for low-speed, thermally-driven flow with an emphasis on smoke and heat transport from fires. The partial derivatives in the conservation equations for mass, momentum and energy are approximated by finite differences, and the solution is updated in time on a three-dimensional, rectilinear grid. Thermal radiation is computed using a finite volume technique on the same grid as the flow solver. Lagrangian particles are used to simulate smoke movement, sprinkler discharge, and fuel sprays.

Smokeview is a companion program to FDS that produces images and animations of the results. In recent years, its developer, Glenn Forney, has
added to Smokeview the ability to visualize fire and smoke in a fairly realistic way. In a sense, Smokeview now is, via its three-dimensional
renderings, an integral part of the physical model, as it allows one to assess the visibility within a fire compartment in ways that ordinary
scientific visualization software cannot.

Although not part of the FDS/Smokeview suite maintained at NIST, there are several third-party and proprietary ``add-ons'' to FDS either available
commercially or privately maintained by individual users. Most notably, there are several Graphical User Interfaces (GUIs) that can be used
to create the input file containing all the necessary information needed to perform a simulation.



\subsection{Version History}

Version 1 of FDS was publicly released in February 2000, version 2 in December 2001, version 3 in November 2002, and version 4 in July 2004. The
present version of FDS is 5, first released in October, 2007.

Starting with FDS 5, a formal revision management system has been implemented to track changes to the FDS source code. The open-source program
development tools are provided by an Internet-based organization known as Google Code (code.google.com).

The version number for FDS has three parts.  For example, FDS 5.2.12 indicates that this is FDS 5, the fifth major release. The 2 indicates a
significant upgrade, but still within the framework of FDS 5.  The 12 indicates the twelveth minor upgrade of 5.2, mostly bug fixes and minor user
requests.


\subsection{Model Developers}


Currently, FDS is maintained by the Building and Fire Research Laboratory (BFRL) of National Institute of Standards and Technology. The developers at
NIST have formed a loose collaboration of interested stakeholders, including:
\begin{itemize}
\item VTT Technical Research Centre of Finland, a research and testing
laboratory similar to NIST
\item The Society of Fire Protection Engineers (SFPE) who conduct training classes on the use of FDS
\item Fire protection engineering firms that use the software
\item Engineering departments at various universities with a particular emphasis on fire
\end{itemize}
BFRL awards grants on a competitive basis to external organizations who conduct research in fire science and engineering. Some of these grants have
been used to assist the development of FDS. The role of the grantee in supporting day to day development varies. Not all of the developers outside of
NIST are grantees.

Starting with Version 5, the FDS development team uses an Internet-based development environment called GoogleCode, a free service of the search
engine company, Google. GoogleCode is a widely used service designed to assist open source software development by providing a repository for source
code, revision control, program distribution, bug tracking, and various other very useful services.

Each member of the FDS development team has an account and password access to the FDS repository. In addition, anonymous access is available to all
interested users, who can receive the latest versions of the source code, manuals, and other items. Anonymous users simply do not have the power to
commit changes to any of these items. The power to commit changes to FDS or its manuals can be granted to anyone on a case by case basis.

The FDS manuals are typeset using \LaTeX, specifically, PDF \LaTeX. The \LaTeX files are essentially text files that are under SVN (Subversion)
control. The figures are either in the form of PDF or jpeg files, depending on whether they are vector or raster format. There are a variety of
\LaTeX packages available, including MiKTeX. The FDS developers edit the manuals as part of the day to day upkeep of the model. Different editions of
the manuals are distinguished by date.


\subsection{Intended Uses}

Throughout its development, FDS has been aimed at solving practical
fire problems in fire protection engineering, while at the same time
providing a tool to study fundamental fire dynamics and combustion.
FDS can be used to model the following phenomena:
\begin{itemize}
\setlength{\itemsep}{0.0in}
\item Low speed transport of heat and combustion products from fire
\item Radiative and convective heat transfer between the gas and solid surfaces
\item Pyrolysis
\item Flame spread and fire growth
\item Sprinkler, heat detector, and smoke detector activation
\item Sprinkler sprays and suppression by water
\end{itemize}
Although FDS was designed specifically for fire simulations,
it can be used for other low-speed fluid flow simulations that do not necessarily
include fire or thermal effects. To date, about half of the
applications of the model have been for design of smoke control
systems and sprinkler/detector activation studies.
The other half consist of residential and industrial fire reconstructions.


\subsection{Input Parameters}

All of the input parameters required by FDS to describe a particular
scenario are conveyed via a single text file created by the user.
The file contains information about the numerical grid, ambient environment, building geometry, material
properties, combustion kinetics, and desired output quantities.
The numerical grid consists of one or more rectilinear meshes with (usually) uniform cells. All geometric features of the
scenario must conform to this numerical grid. Objects smaller than a single grid cell are either approximated
as a single cell, or rejected. The building geometry is input as a series of rectangular blocks. Boundary conditions are
applied to solid surfaces as rectangular patches. Materials are defined by their thermal conductivity, specific heat,
density, thickness, and burning behavior. There are various ways that this information is conveyed, depending on the
desired level of detail.

Any simulation of a real fire scenario involves specifying material properties for the walls, floor, ceiling,
and furnishings. FDS treats all of these objects as multi-layered solids, thus the physical parameters for many real
objects can only be viewed as approximations to the actual properties. Describing these materials in the input file is
the single most challenging task for the user. Thermal properties such as conductivity, specific heat,
density, and thickness can be found in various handbooks, or in manufacturers literature, or from bench-scale measurements.
The burning behavior of materials at different heat fluxes is more difficult to describe, and the properties more difficult
to obtain. Even though entire books are devoted to the
subject~\cite{Babrauskas:2}, it is still difficult to find information on a particular item.

A significant part of the FDS input file directs the code to output various quantities in various ways. Much like in an
actual experiment, the user must decide before the calculation begins what information to save. There is no way to
recover information after the calculation is over if it was not requested at the start.

A complete description of the input parameters required by FDS can be found in the FDS User's Guide~\cite{FDS_Users_Guide}.


\subsection{Output Quantities}

FDS computes the temperature, density, pressure, velocity and chemical composition within each numerical
grid cell at each discrete time step. There are typically hundreds of thousands to millions of grid
cells and thousands to hundreds of thousands of time steps. In addition, FDS computes at solid surfaces
the temperature, heat flux, mass loss rate, and various other quantities. The user must carefully select what
data to save, much like one would do in designing an actual experiment. Even though only a small fraction of
the computed information can be saved, the output typically consists of fairly large data files. Typical
output quantities for the gas phase include:
\begin{itemize}
\setlength{\itemsep}{0.0in}
\item Gas temperature
\item Gas velocity
\item Gas species concentration (water vapor, CO$_2$, CO, N$_2$)
\item Smoke concentration and visibility estimates
\item Pressure
\item Heat release rate per unit volume
\item Mixture fraction (or air/fuel ratio)
\item Gas density
\item Water droplet mass per unit volume
\end{itemize}
On solid surfaces, FDS predicts additional quantities associated with the energy balance between
gas and solid phase, including
\begin{itemize}
\setlength{\itemsep}{0.0in}
\item Surface and interior temperature
\item Heat flux, both radiative and convective
\item Burning rate
\item Water droplet mass per unit area
\end{itemize}
Global quantities recorded by the program include:
\begin{itemize}
\setlength{\itemsep}{0.0in}
\item Total Heat Release Rate (HRR)
\item Sprinkler and detector activation times
\item Mass and energy fluxes through openings or solids
\end{itemize}
Time histories of various quantities at a single point in space or global
quantities like the fire's heat release rate (HRR) are saved in simple, comma-delimited text files that
can be plotted using a spreadsheet program.
However, most field or surface data are visualized with a program called Smokeview, a tool specifically
designed to analyze data generated by FDS. FDS and Smokeview are used in concert to model and visualize fire phenomena.
Smokeview performs this visualization by presenting animated tracer particle flow,
animated contour slices of computed gas variables and animated surface data.
Smokeview also presents contours and vector plots of static data anywhere
within a scene at a fixed time.

A complete list of FDS output quantities and formats is given in Ref.~\cite{FDS_Users_Guide}.
Details on the use of Smokeview are found in Ref.~\cite{Smokeview_Users_Guide}.




\subsection{Governing Equations, Assumptions and Numerics}

Following is a brief description of
the major components of FDS. Detailed information regarding the assumptions and governing equations associated
with the model is provided in Section~\ref{govequations}.
\begin{description}
\item[Hydrodynamic Model] FDS
solves numerically a form of the Navier-Stokes equations appropriate
for low-speed, thermally-driven flow with an emphasis on
smoke and heat transport from fires. The core algorithm is an
explicit predictor-corrector scheme that is second order accurate in space
and time. Turbulence is treated by means of the Smagorinsky form of
Large Eddy Simulation (LES). It is possible to perform a Direct
Numerical Simulation (DNS) if the underlying numerical grid is fine
enough. LES is the default mode of operation.
\item[Combustion Model]
For most applications, FDS uses a combustion model based on the mixing limited, infinitely fast reaction of lumped species.
Lumped species are conserved scalar quantities that represent a mixture of species such as air which is a mixture of nitrogren, oxygen, water vapor, and carbon dioxide.  In FDS versions prior to 6, these lumped species were referred to as mixture fraction parameters.  As with FDS 5, the reaction of fuel and oxygen is not necessarily instantaneous and complete, and there are
several optional schemes that are designed to predict the extent of combustion in under-ventilated spaces.
The mass fractions of all of the major reactants and products can
be derived from the lumped species by means of ``state relations,''
expressions arrived at by a
combination of simplified analysis and measurement.
\item[Radiation Transport] Radiative heat transfer is included in the
model via the solution of the radiation transport equation for a gray
gas. In a limited number of cases, a wide band model can be used in
place of the gray gas model to provide a better spectral accuracy. The
radiation equation is solved using a technique similar to a finite
volume method for convective transport, thus the name given to it is
the Finite Volume Method (FVM). Using approximately 100 discrete
angles, the finite volume solver requires about 20~\% of the total CPU
time of a calculation, a modest cost given the complexity of radiation
heat transfer.  Water droplets can absorb and scatter thermal
radiation. This is important in cases involving mist sprinklers, but
also plays a role in all sprinkler cases. The absorption and
scattering coefficients are based on Mie theory. The scattering from
the gaseous species and soot is not included in the model.
\item[Geometry]
FDS approximates the governing equations on one or more rectilinear grids. The
user prescribes rectangular obstructions that are forced to conform
with the underlying grid.
\item[Boundary Conditions]
All solid surfaces are assigned thermal boundary conditions, plus
information about the burning behavior of the material.
Heat and mass transfer to and from solid surfaces is
usually handled with empirical correlations, although it is possible
to compute directly the heat and mass transfer when performing a
Direct Numerical Simulation (DNS).
\item[Sprinklers and Detectors] The activation of sprinklers and heat and smoke detectors
is modeled using fairly simple correlations of thermal inertia for
sprinklers and heat detectors, and transport lag for smoke detectors.
Sprinkler sprays are modeled by Lagrangian particles that represent a sampling of the
water droplets ejected from the sprinkler.
\end{description}


\subsection{Limitations}

Although FDS can address most fire scenarios, there are limitations in all of its various
algorithms. Some of the more prominent limitations of the model are listed here. More
specific limitations are discussed as part of the description of the governing equations
in Section~\ref{govequations}.
\begin{description}
\item[Low Speed Flow Assumption] The use of FDS is limited to low-speed\footnote{Mach numbers less than about 0.3} flow
with an emphasis on smoke and heat transport from fires. This assumption rules out using the model for any scenario
involving flow speeds approaching the speed of sound, such as explosions, choke flow at nozzles, and detonations.
\item[Rectilinear Geometry] The efficiency of FDS is due to the simplicity of its rectilinear numerical grid and the
use of a fast, direct solver for the pressure field.
This can be a limitation in some situations where certain geometric features
do not conform to the rectangular grid, although most building components do. There are techniques in FDS to
lessen the effect of ``sawtooth'' obstructions used to represent non-rectangular objects, but these cannot be expected
to produce good results if, for example, the intent of the calculation is to study boundary layer effects. For most
practical large-scale simulations, the increased grid resolution afforded by the fast pressure solver offsets the
approximation of a curved boundary by small rectangular grid cells.
\item[Fire Growth and Spread]
Because the model was originally designed to analyze industrial-scale fires,
it can be used reliably when the heat release rate (HRR) of the fire is specified and the
transport of heat and exhaust products is the principal aim of the simulation.
In these cases, the model predicts flow velocities and temperatures to an accuracy within
10~\% to 20~\% of experimental measurements, depending on the resolution of the numerical grid
\footnote{It is extremely rare to
find measurements of local velocities and/or temperatures from fire experiments that
have reported error estimates that are less than 10~\%. Thus, the most accurate
calculations using FDS do not introduce significantly greater errors in these quantities
than the vast majority of fire experiments.}.
However, for fire scenarios where the heat release rate is {\em predicted} rather than {\em specified},
the uncertainty of the model is higher.
There are several reasons for this: (1) properties of real materials
and real fuels are often unknown or difficult to obtain, (2) the physical processes of combustion,
radiation and solid phase heat transfer are more complicated than their mathematical representations
in FDS, (3) the results of calculations are sensitive to both the numerical and physical parameters.
Current research is aimed at improving this situation, but it is safe to say that
modeling fire growth and spread will always require a higher level of
user skill and judgment than that required for modeling the transport of smoke and heat from specified fires.
\item[Combustion]
For most applications, FDS uses a mixing-controlled, lumped species based combustion model.
Lumped species are conserved scalar quantities that represent mixtures of gas species.  For most applications, these mixtures are air, fuel plus an optional diluent, and combustion products.  In its simplest form, the model assumes that combustion is mixing-controlled, and that the
reaction of fuel and oxygen is infinitely fast, regardless of the temperature.
For large-scale, well-ventilated
fires, this is a good assumption. However, if a fire is in an
under-ventilated compartment, or if a suppression agent like water
mist or CO$_2$ is introduced, fuel and oxygen are allowed to mix and not burn, according to a few empirically-based criteria.
The physical mechanisms underlying these phenomena are complex, and are tied closely to the flame temperature and local strain rate, neither of
which are readily-available in a large scale fire simulation.
Subgrid-scale modeling of gas phase suppression and
extinction is still an area of active research in the combustion
community. Until reliable models can be developed for building-scale
fire simulations, simple empirical rules can be used that
prevent burning from taking place when the atmosphere immediately
surrounding the fire cannot sustain the combustion. Details are found in
Section~\ref{combustionsection}.
\item[Radiation] Radiative heat transfer is included in the model via
the solution of the radiation transport equation (RTE) for a gray gas, and
in some limited cases using a wide band model.  The RTE is solved
using a technique similar to finite volume methods for convective
transport, thus the name given to it is the Finite Volume Method
(FVM). There are several limitations of the model. First, the
absorption coefficient for the smoke-laden gas is a complex function
of its composition and temperature. Because of the simplified
combustion model, the chemical composition of the smokey gases,
especially the soot content, can effect both the absorption and
emission of thermal radiation.  Second, the radiation transport is
discretized via approximately 100 solid angles, although the user may
choose to use more angles. For targets far away from a localized
source of radiation, like a growing fire, the discretization can lead
to a non-uniform distribution of the radiant energy. This error is
called ``ray effect'' and can be seen in the visualization of surface
temperatures, where ``hot spots'' show the effect of the finite number
of solid angles. The problem can be lessened by the inclusion of more
solid angles, but at a price of longer computing times. In most cases,
the radiative flux to far-field targets is not as important as those
in the near-field, where coverage by the default number of angles is
much better.
\end{description}



\clearpage
\section{Peer Review Process}

FDS is reviewed both internally and externally. All documents issued by the
National Institute of Standards and Technology are formally reviewed internally by members of
the staff. The theoretical basis of FDS is laid out in the present document, and is
subject to internal review by staff members who are not active participants in the development
of the model, but who are members of the Fire Research Division and are considered experts in
the fields of fire and combustion. Externally, papers detailing various parts of FDS are
regularly published in peer-reviewed journals and conference proceedings. In addition, FDS
is used world-wide by fire protection engineering firms who review the technical details of
the model related to their particular application. Some of these firms also publish in the
open literature reports documenting internal efforts to validate the model for a particular
use. Many of these studies are referenced in Volume 3 of the FDS Technical Reference Guide~\cite{FDS_Tech_Guide}.


\subsection{Survey of the Relevant Fire and Combustion Literature}

\label{Relevantdocs}

FDS has two separate manuals --
the FDS Technical Reference Guide~\cite{FDS_Tech_Guide}
and the FDS User's Guide~\cite{FDS_Users_Guide}. The Technical Reference Guide is broken into three volumes: (1)~Mathematical Model, (2)~Verification, and (3)~Experimental Validation.
Smokeview has its own User's Guide~\cite{Smokeview_Users_Guide}. The FDS and Smokeview User Guides only describe the mechanics of using the
computer programs. The
Technical Reference Guides provides the theory, algorithm details, and verification and validation work.

There are numerous sources that describe various parts of the
model. The basic set of equations solved in FDS was formulated by Rehm
and Baum in the {\em Journal of Research of the National Bureau of
Standards}~\cite{Rehm:1}.  The basic hydrodynamic algorithm evolved at
NIST through the 1980s and 1990s, incorporating fairly well-known
numerical schemes that are documented in books by Anderson, Tannehill
and Pletcher~\cite{Anderson:1}, Peyret and Taylor~\cite{Peyret:1}, and
Ferziger and Peri\'{c}~\cite{Ferziger:1}. This last book provides a
good description of the large eddy simulation technique and provides
references to many current publications on the subject.  Numerical
techniques appropriate for combustion systems are described by Oran
and Boris~\cite{Oran:1}.  The mixture fraction combustion model is
described in a review article by Bilger~\cite{Bilger:AnnRev}. Basic
heat transfer theory is provided by Holman~\cite{Holman:1} and
Incropera~\cite{Incropera:1}. Thermal radiation is described in Siegel
and Howell~\cite{Siegel:1}.

Much of the current knowledge of fire science and engineering
is found in the {\em SFPE Handbook of Fire Protection Engineering}~\cite{SFPE}. Popular textbooks in fire protection
engineering include those by Drysdale~\cite{Drysdale:1} and Quintiere~\cite{Quintiere:2}. On-going research in
fire and combustion is documented in several periodicals and conference proceedings.
The International Association of Fire Safety Science (IAFSS)
organizes a conference every two years, the proceedings of which are frequently referenced by fire researchers.
Interscience Communications, a London-based publisher of several fire-related journals, hosts a conference known as Interflam roughly
every three years in the United Kingdom.
The Combustion Institute hosts an international symposium on combustion every two years, and in addition to the
proceedings of this symposium, the organization publishes its own journal, {\em Combustion and Flame}.
The papers appearing in the IAFSS conference proceedings,
the Combustion Symposium proceedings, and {\em Combustion and Flame} are all peer-reviewed, while those appearing in the
Interflam proceedings are selected based on the submission of a short abstract.
Both the Society for Fire Protection Engineers (SFPE) and the National Fire Protection Association (NFPA) publish
peer-reviewed technical journals entitled the {\em Journal of Fire Protection Engineering} and {\em Fire Technology}.
Other often-cited, peer-reviewed technical journals include the {\em Fire Safety Journal}, {\em Fire and Materials}, {\em Combustion
Science and Technology}, {\em Combustion Theory and Modeling} and the {\em Journal of Heat Transfer}.

Research at NIST is documented in various ways beyond contributions made by staff to external journals and conferences.
NIST publishes several forms of internal reports, special publications, and its own journal called the {\em Journal
of Research of NIST}. An internal report, referred to as a NISTIR (NIST Inter-agency Report), is a convenient means to disseminate information,
especially when the quantity of data exceeds what could normally be accepted by a journal. Often parts of a NISTIR are
published externally, with the NISTIR itself serving as the complete record of the work performed. Previous versions of the
FDS Technical Reference Guide and User's Guide were published as NISTIRs. The current FDS and Smokeview manuals are
being published as NIST Special Publications, distinguished from NISTIRs by the fact that they are permanently archived.
Work performed by an outside person or organization working
under a NIST grant or contract is published in the form of a NIST Grant/Contract Report (GCR).
All work performed by the staff of the Building and Fire Research Laboratory at NIST beyond 1993 is permanently stored in
electronic form and made freely available via the Internet and yearly-released compact disks (CDs) or other electronic media.




\subsection{Review of the Theoretical Basis of the Model}

\label{JustAA}
The technical approach and assumptions of the model have been presented in
the peer-reviewed scientific literature and at technical conferences cited in the previous section.
The major assumptions of the model, for example the large eddy simulation technique and the combustion
model, have undergone a roughly 40 year development and are now documented in popular introductory text books.
More specific sub-models, like the sprinkler spray routine or the various pyrolysis models, have yet to be developed to
this extent. As a consequence, all documents produced by NIST staff are required to go
through an internal editorial review and approval process.
This process is designed to ensure compliance with the technical requirements,
policy, and editorial quality required by NIST.
The technical review includes a critical evaluation of the technical content and
methodology, statistical treatment of data, uncertainty analysis, use of appropriate
reference data and units, and bibliographic references.
The FDS and Smokeview manuals are first reviewed by a member of the Fire Research Division,
then by the immediate supervisor of the author of the document,
then by the chief of the Fire Research Division, and finally by a reader from
outside the division. Both the immediate supervisor and the division chief are
technical experts in the field. Once the document has been reviewed, it is
then brought before the Editorial Review Board (ERB),
a body of representatives from all the NIST laboratories.
At least one reader is designated by the Board for each document that it accepts for
review. This last reader is selected based on technical competence and impartiality.
The reader is usually from outside the division producing the document and is
responsible for checking that the document conforms with NIST policy on units, uncertainty
and scope. He/she does not need to be a technical expert in fire or combustion.

Recently, the US Nuclear Regulatory Commission (US NRC) published a seven-volume report on its own verification and validation
study of five different fire models used for nuclear power plant applications~\cite{NUREG_1824}. Two of the models are essentially a set
of empirically-based correlations in the form of engineering ``spread sheets.'' Two of the models are classic two-zone fire models, one of which
is the NIST developed CFAST. FDS is the sole CFD model in the study. More on the study and its results can be found in Volume~3 of the
FDS Technical Reference Guide~\cite{FDS_Tech_Guide}.

Besides formal internal and peer review, FDS is subjected to continuous scrutiny because
it is available free of charge to the general public and is used
internationally by those involved in fire safety design and post-fire reconstruction.
The quality of the FDS and Smokeview User Guides is checked implicitly by the fact that the
majority of model users have not taken a formal training course in the actual use of the model, but
are able to read the supporting documents, perform a few sample simulations, and then systematically build
up a level of expertise appropriate for their applications. The developers receive daily feedback from
users on the clarity of the documentation and add clarifications
when needed. Before new versions of the model are released, there is a several month ``beta test'' period
in which users test the new version using the updated documentation. This process is similar,
although less formal, to that which most computer software programs undergo.
Also, the source code for FDS is released publicly, and has been used at
various universities world-wide, both in the classroom as a teaching tool as well as for research.
As a result, flaws in the theoretical development and the computer program itself
have been identified and corrected. As FDS continues to evolve, the user base will continue to
serve as a means to evaluate the model. We consider this process as important to the development of FDS as the formal
internal and external peer-review processes.


\clearpage
\section{Development Process}

Changes are made to the FDS source code daily, and tracked via revision control software. However, these daily changes do not constitute a change to
the version number. After the developers determine that enough changes have been made to the source, they release a new minor upgrade, 5.2.12 to
5.2.13, for example. This happens every few weeks. A change from 5.2 to 5.3 might happen only a few times a year, when significant improvements have
been made to the model physics.

There is no formal process by which FDS is updated. Each developer works on various routines, and makes changes as warranted. Minor bugs are fixed
without any communication (the developers are in different locations), but more significant changes are discussed via email or telephone calls. A
suite of simple verification calculations (included in this document) are routinely run to ensure that the daily bug fixes have not altered any of
the important algorithms. A suite of validation calculations (also included here) are run with each significant upgrade. Significant changes to FDS
are made based on the following criteria, in no particular order:
\begin{description}
\item[Better Physics:] The goal of any model is to be {\em predictive}, but it also must be reliable. FDS is a blend of empirical and
deterministic sub-models, chosen based on their robustness, consistency, and reliability. Any new sub-model must demonstrate that it is of comparable
or superior accuracy to its empirical counterpart.
\item[Modest CPU Increase:] If a proposed algorithm doubles the calculation time but yields only a marginal improvement in accuracy, it is
likely to be rejected. Also, the various routines in FDS are expected to consume CPU time in proportion to their overall importance. For example, the
radiation transport algorithm consumes about 25~\% of the CPU time, consistent with the fact that about one-fourth to one-third of the fire's energy
is emitted as thermal radiation.
\item[Simpler Algorithm:] If a new algorithm does what the old one did using fewer lines of code, it is almost always accepted, so long as
it does not decrease functionality.
\item[Increased or Comparable Accuracy:] The validation experiments that are part of this guide serve as the metric for new routines. It is
not enough for a new algorithm to perform well in a few cases. It must show clear improvement across the suite of experiments. If the accuracy is
only comparable to the previous version, then some other criteria must be satisfied.
\item[Acceptance by the Fire Protection Community:] Especially in regard to fire-specific devices, like sprinklers and smoke detectors, the
algorithms in FDS often are based on their acceptance among the practicing engineers.
\end{description}


