\chapter{Fire Plumes}

\section{Plume Temperature}

Heskestad (1995) provided a simple correlation for estimating the increase in centerline temperature above ambient ($T_p$ - $T_\infty$) of a fire plume as a function of ceiling height and HRR

\be
\Delta T_p = \frac{9.1 \left( \frac{T_\infty}{g c_p^2 \rho_{a}^2} \right)^{1/3} \dot Q_c^{2/3}}{(z-z_0)^{5/3}}
\label{eq:Heskestad}
\ee

\noindent where

\be
\dot Q_c = \dot Q (1 - \chi_r)
\label{eq:Heskestad_Qc}
\ee

\be
z_0 = -1.02 D + 0.083 \dot Q^{2/5}
\label{eq:Heskestad_z0}
\ee

\be
D = \sqrt{\frac{4 A}{\pi}}
\label{eq:Heskestad_D}
\ee

In Eq.~\ref{eq:Heskestad}, $\Delta T_p$ is the plume centerline temperature rise above ambient~($^\circ$C), $T_\infty$ is the ambient air temperature~($^\circ$C), $g$ is the acceleration of gravity~(m/s$^2$), $c_p$ is the specific heat of air~(kJ/kg-K), $\rho_{a}$ is the ambient air density~(kg/m$^3$), $\dot Q_c$ is the convective HRR~(kW), $z$ is the elevation above the fire source~(m), and $z_0$ is the hypothetical virtual origin of the fire~(m).

In Eq.~\ref{eq:Heskestad_Qc}, $\dot Q$ is the total HRR~(kW), and $\chi_r$ is the radiative fraction~(-). In Eq.~\ref{eq:Heskestad_z0}, $D$ is the diameter of the fire source~(m). In Eq.~\ref{eq:Heskestad_D}, $A$ is the area of the fire source~(m$^2$).

\clearpage

\section{Summary of Plume Temperature}

Summary scatter plots of the plume temperature predictions are given on the following pages.

\begin{figure}[ht]
\begin{center}
\begin{tabular}{l}
\includegraphics[width=4.0in]{FIGURES/Scatterplots/Plume_Temperature}
\end{tabular}
\end{center}
\caption[Summary of plume temperature predictions.]
{Summary of plume temperature predictions.}
\label{Plume_Summary}
\end{figure}

