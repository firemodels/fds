% !TEX root = FDTs_Validation_Guide.tex

\chapter{Ceiling Jets and Device Activation}
\label{Ceiling_Jet_Chapter}

\section{Ceiling Jet Temperature}

For steady-state fire, the ceiling jet temperature rise, $\Delta T_{jet}$, above ambient ($T_{jet} - T_\infty$) from a fire plume is given by the following expressions developed by Alpert (1972)

\be
\Delta T_{jet} = \frac{16.9 \dot Q^{2/3}}{H^{5/3}} \mathrm{\ for\ } r/H <= 0.18
\label{eq:Alpert_Tjet_less}
\ee

\be
\Delta T_{jet} = \frac{5.38 (\dot Q / r)^{2/3}}{H} \mathrm{\ for\ } r/H > 0.18
\label{eq:Alpert_Tjet_greater}
\ee

In Eqs.~\ref{eq:Alpert_Tjet_less} and \ref{eq:Alpert_Tjet_greater}, $\Delta T_{jet}$ is the ceiling jet temperature rise above ambient~($^\circ$C), $\dot Q$ is the total HRR~(kW), $H$ is the height of the ceiling above the fuel~(m), and $r$~is the radial distance to the detector~(m).

Note that some of these cases use a t-squared fire source as a quasi-steady approach.

For cases in which the fire was located against a wall or corner, these correlations are adjusted based on the method of reflection. For a fire adjacent to a flat wall, 2$\dot Q$ is substituted for $\dot Q$; and for a fire in a 90-degree corner, 4$\dot Q$ is substituted for $\dot Q$~\cite{SFPE:Alpert}. This adjustment is denoted in the input parameters as the scaling factor. For a given case, the scaling factor has a value of 1, 2, or 4, which corresponds to a fire away from walls or corners, a fire adjacent to a wall, or a fire located in a corner, respectively.

It is important to note that this ceiling jet temperature correlation was developed using data from tests that were conducted in a large facility in which the distant walls and large compartment size did not allow for the development of a significant hot gas layer (HGL). In a more typical fire scenario (i.e., a smaller compartment), the HGL develops relatively quickly, and temperatures at the ceiling are affected by the ceiling jet as well as the accumulating HGL. Thus, when compared to experimentally measured ceiling temperatures in a compartment fire, this correlation tends to underpredict the temperatures because it is not accounting for the development of the HGL. This is an important consideration when using this correlation to predict detector or sprinkler activations.

For the reasons stated above, two scatter plot comparisons are shown in Fig.~\ref{Ceiling_Jet_Temperature_Summary}. One scatter plot shows underpredicted temperature comparisons for compartment fire tests. The other scatter plot shows results for unconfined tests that were conducted under a false ceiling in which the hot plume gases did not accumulate to form an HGL, but were allowed to spill out from under a false ceiling. In the unconfined cases, the ceiling temperature predictions are in better agreement with experimental data because this scenario is more representative of a temperature rise due to only ceiling jet flow from the fire plume.

\clearpage


\section{Sprinkler Activation Time}
\label{sec:sprinkler_alpert}

For steady-state fire, the method for estimating the response of a smoke detector is based on the correlations developed by Alpert (1972) for activation of a sprinkler and is given by the following equation (Budnick, Evans, and Nelson, 1997)

\be
t_{activation} =  \frac{RTI}{\sqrt{u_{jet}}} ln \left( \frac{T_{jet} - T_\infty}{T_{jet} - T_{activation}} \right)\label{eq:Alpert}
\ee

\noindent where

\be
T_{jet} = \frac{16.9 \dot Q^{2/3}}{H^{5/3}} + T_\infty \mathrm{\ for\ } r/H <= 0.18
\label{eq:Alpert_Tjet_lt}
\ee

\be
T_{jet} = \frac{5.38 (\dot Q / r)^{2/3}}{H} + T_\infty \mathrm{\ for\ } r/H > 0.18
\label{eq:Alpert_Tjet_gt}
\ee

\noindent and

\be
u_{jet} = 0.96 \left( \frac{\dot Q}{H} \right)^{1/3} \mathrm{\ for\ } r/H <= 0.15
\label{eq:Alpert_ujet_lt}
\ee

\be
u_{jet} = \frac{0.195 \dot Q^{1/3} H^{1/2}}{r^{5/6}} \mathrm{\ for\ } r/H > 0.15
\label{eq:Alpert_ujet_gt}
\ee

In Eq.~\ref{eq:Alpert}, $t_{activation}$ is the activation time of the sprinkler~(s), $RTI$ is the response time index of the sprinkler~(m-s)$^{1/2}$, $u_{jet}$ is the ceiling jet velocity~(m/s), $T_{jet}$ is the ceiling jet temperature~($^\circ$C), $T_\infty$ is the ambient air temperature~($^\circ$C), and $T_{activation}$ is the activation temperature of the sprinkler~($^\circ$C).

In Eqs.~\ref{eq:Alpert_Tjet_lt} and \ref{eq:Alpert_Tjet_gt}, $\dot Q$ is the total HRR~(kW), $H$ is the height of the ceiling above the fuel~(m), and $r$~is the radial distance to the detector~(m).

Note that some of these cases use a t-squared fire source as a quasi-steady approach.

For cases in which the fire was located against a wall or corner, these correlations are adjusted based on the method of reflection. For a fire adjacent to a flat wall, 2$\dot Q$ is substituted for $\dot Q$; and for a fire in a 90-degree corner, 4$\dot Q$ is substituted for $\dot Q$~\cite{SFPE:Alpert}. This adjustment is denoted in the input parameters as the scaling factor. For a given case, the scaling factor has a value of 1, 2, or 4, which corresponds to a fire away from walls or corners, a fire adjacent to a wall, or a fire located in a corner, respectively.


\clearpage


\section{Smoke Detector Activation Time}

\subsection{Method of Alpert}

In this method, smoke detector activation is identical to that for a heat detector (as described in Section~\ref{sec:sprinkler_alpert}), with the exception of the response of smoke detectors to a modest rise in the ceiling jet temperature. Heskestad and Delichatsios (1977) correlated a smoke temperature change of 10~$^\circ$C (18~$^\circ$F) from typical fuels. For the purpose of calculating smoke detector response time, it is assumed that the smoke detectors are low-RTI devices ($\textrm{RTI}=5$).


\clearpage


\subsection{Method of Milke}

The operation of detectors for smoke from typical fuels has been correlated to a temperature change from the following correlation

\be
t_{activation} = \frac{X H^{4/3}}{\dot Q^{1/3}}
\label{eq:Milke}
\ee

\noindent where

\be
X = 4.6 \times 10^{-4} (Y^2) + 2.7 \times 10^{-15} (Y^6)
\label{eq:Milke_X}
\ee

\be
Y = \frac{\Delta T_c H^{5/3}}{\dot Q^{2/3}}
\label{eq:Milke_Y}
\ee

In Eq.~\ref{eq:Milke}, $t_{activation}$ is the detector activation time~(s), $H$ is the height of the ceiling above the top of the fuel~(ft), and $\dot Q$ is the HRR~(kW). In Eq.~\ref{eq:Milke_Y}, $\Delta T_c$ is the temperature rise of gases under the ceiling required for the smoke detector to activate~($^\circ$F).

Note that some of these cases use a t-squared fire source as a quasi-steady approach.


\clearpage


\subsection{Method of Mowrer}

The response time of a smoke detector comprises two separate times, including the transport lag time of the plume and the transport lag time of the ceiling jet, as illustrated by the following equation

\be
t_{activation} = t_{pl} + t_{cj}
\label{eq:Mowrer}
\ee

\noindent where

\be
t_{pl} = C_{pl} \frac{H^{4/3}}{\dot Q^{1/3}}
\label{eq:Mowrer_tpl}
\ee

\be
t_{cj} = \frac{1}{C_{cj}} \frac{r^{11/6}}{\dot Q^{1/3} H^{1/2}}
\label{eq:Mowrer_tcj}
\ee

In Eq.~\ref{eq:Mowrer}, $t_{activation}$ is the detector activation time~(s), $t_{pl}$ is the transport lag time of the plume~(s), and $t_{cj}$ is transport lag time of the ceiling jet~(s). In Eq.~\ref{eq:Mowrer_tpl}, $C_{pl}$ is the plume lag time constant~(0.67), $H$ is the height of the ceiling above the fuel~(m), and $\dot Q$ is the HRR~(kW). In Eq.~\ref{eq:Mowrer_tcj}, $C_{cj}$ is the ceiling jet time lag time constant~(1.2), and $r$ is the radial distance to the detector~(m).

Note that some of these cases use a t-squared fire source as a quasi-steady approach.


\clearpage


\section{Summary of Ceiling Jet Temperature Predictions}

Summary scatter plots of the ceiling jet temperature predictions are shown on the following pages.

\begin{figure}[ht]
\begin{center}
\begin{tabular}{l}
\includegraphics[width=4.0in]{SCRIPT_FIGURES/Scatterplots/Ceiling_Jet_Temperature_Compartment} \\
\includegraphics[width=4.0in]{SCRIPT_FIGURES/Scatterplots/Ceiling_Jet_Temperature_Unconfined}
\end{tabular}
\end{center}
\caption[Summary of ceiling jet temperature predictions]
{Summary of compartment (top) and unconfined (bottom) ceiling jet temperature predictions.}
\label{Ceiling_Jet_Temperature_Summary}
\end{figure}

\clearpage

\section{Summary of Device Activation Time Predictions}

Summary scatter plots of the device activation time predictions are shown on the following pages.

\begin{figure}[ht]
\begin{center}
\begin{tabular}{l}
\includegraphics[width=4.0in]{SCRIPT_FIGURES/Scatterplots/Sprinkler_Activation_Time}
\end{tabular}
\end{center}
\caption[Summary of sprinkler activation time predictions]
{Summary of sprinkler activation time predictions.}
\label{Sprinkler_Activation_Summary}
\end{figure}

\begin{figure}[p]
\begin{center}
\begin{tabular}{l}
\includegraphics[width=4.0in]{SCRIPT_FIGURES/Scatterplots/Smoke_Detector_Activation_Time_Alpert}
\end{tabular}
\end{center}
\caption[Summary of smoke detector activation time predictions]
{Summary of smoke detector activation time predictions using the method of Alpert.}
\label{Smoke_Detector_Activation_Summary_Alpert}
\end{figure}

\begin{figure}[p]
\begin{center}
\begin{tabular}{l}
\includegraphics[width=4.0in]{SCRIPT_FIGURES/Scatterplots/Smoke_Detector_Activation_Time_Milke} \\
\includegraphics[width=4.0in]{SCRIPT_FIGURES/Scatterplots/Smoke_Detector_Activation_Time_Mowrer}
\end{tabular}
\end{center}
\caption[Summary of smoke detector activation time predictions]
{Summary of smoke detector activation time predictions using the method of Milke (top) and Mowrer (bottom).}
\label{Smoke_Detector_Activation_Summary_Milke_Mowrer}
\end{figure}

