%
% newcommands needed by this include file
%
\newcommand{\devicewidth}{1.5in}
\newcommand{\boxwidth}{3.0in}
\newcommand{\incgraphics}[1]{
\parbox[c]{\devicewidth}{
\vspace{0.01in}
\includegraphics[width=\devicewidth]{#1}
\vspace{0.01in}
}
}

\section{Visualizing FDS Devices Using Smokeview Objects}
Smokeview generates visual representations of FDS devices using instructions found in a data file named
{\tt objects.svo}.  These instructions correspond to OpenGL library calls, the same type of calls Smokeview
uses to visualize FDS cases.  New objects may be designed and drawn without modifying Smokeview and more
importantly may be created by any user not just the FDS/Smokeview developers.  This section gives an overview
of Smokeview objects detailing what objects are available and how to modify them.  Further documentation
giving the underlying technical details may be found in the Smokeview User's Guide~\cite{Smokeview_Users_Guide_5}.

Smokeview objects may be static or dynamic.  A static object is defined entirely in terms of data and instructions
found in the {\tt objects.svo}\ file.  For example, the {\tt sensor}\ object is static, it is drawn as a small green
sphere with a fixed diameter.  Its appearance remains
the same regardless of how an FDS input file is set up.  A dynamic object is also defined using instructions found in
{\tt objects.svo}\ but uses data found in the {\tt objects.svo}\ file, data specified on the {\tt \&PROP}\ namelist statement and/or
data contained in the particle file.  As a result, the appearance of dynamic objects may change based upon the particular FDS case that
is run.   For example, the {\tt tsphere}\ object is dynamic.  The diameter and an image used to cover the sphere (known
as a texture map) is specified in an FDS input file.


\subsection{Static Smokeview Objects}

A Smokeview object consists of one or more frames or views.  Smokeview can then display an FDS device in a
normal/inactive state or in an active state.  A sprinkler, for example, is drawn differently depending on
whether it has activated or not.  When FDS determines that a device has activated it places a message in the
{\tt .smv}\ file indicating the object number, the activation time and the state (0 for inactive or 1 for active).
Smokeview then draws the corresponding frame.  Tables \ref{tab:devices_static}\ and \ref{tab:devices_mstatic}\
give a list of various objects.  Each entry shows pictures of the device in its normal state and
in its active state if it has one.  The intersection of the red and green tubes indicate the origin,
the part of the device displayed at the $(x,y,z)$ coordinate specified on the {\tt \&DEVC}\ input line.

The {\tt SMOKEVIEW\_ID}\ keyword found on the {\tt \&PROP}\ namelist statement is used to associate an FDS device with a Smokeview object.
The following lines
were used to display the target device in Table \ref{tab:devices_static}.

\footnotesize
\begin{verbatim}
&PROP ID='target' SMOKEVIEW_ID='target' /
&DEVC XYZ=0.5,0.8,0.6, QUANTITY='TEMPERATURE' PROP_ID='target' /
\end{verbatim} \normalsize

\newpage

\begin{longtable}[t!]{|l|c|}
\caption{Single Frame Static Objects}
\label{tab:devices_static}
\\ \hline
{\tt SMOKEVIEW\_ID} & Image  \\ \hline \hline
\endfirsthead
\caption{Single Frame Static Objects (continued)} \\ \hline
{\tt SMOKEVIEW\_ID} & Image  \\ \hline \hline
\endhead

sensor & \incgraphics{"scriptfigures/sensor"} \\ \hline
target & \incgraphics{"scriptfigures/target"} \\ \hline

\end{longtable}

\begin{longtable}[ht]{|l|c|c|}
\caption{Multiple Frame Static Objects}
\label{tab:devices_mstatic}
\\ \hline
\multirow{2}{*}{{\tt SMOKEVIEW\_ID}} &\multicolumn{2}{|c|}{Image}\\ \cline{2-3}
& inactive & active  \\ \hline \hline
\endfirsthead
\caption{Multiple Frame Static Objects (continued)}
\\ \hline
\multirow{2}{*}{{\tt SMOKEVIEW\_ID}} &\multicolumn{2}{|c|}{Image}\\ \cline{2-3}
& inactive & active  \\ \hline \hline
\endhead

heat\_detector      & \incgraphics{"scriptfigures/heat_detector_0"}     & \incgraphics{"scriptfigures/heat_detector_1"} \\ \hline
nozzle              & \incgraphics{"scriptfigures/nozzle_0"}            & \incgraphics{"scriptfigures/nozzle_1"} \\ \hline
smoke\_detector     & \incgraphics{"scriptfigures/smoke_detector_0"}    & \incgraphics{"scriptfigures/smoke_detector_1"} \\ \hline
sprinkler\_upright  & \incgraphics{"scriptfigures/sprinkler_upright_0"} & \incgraphics{"scriptfigures/sprinkler_upright_1"} \\ \hline
sprinkler\_pendent  & \incgraphics{"scriptfigures/sprinkler_pendent_0"} & \incgraphics{"scriptfigures/sprinkler_pendent_1"} \\ \hline

\end{longtable}

%
%  sub-section on dynamic objects
\subsection{Dynamic Smokeview Objects}
The appearance of several Smokeview objects may be modified using data specified in an FDS input
file or data generated during an FDS computation and stored in a particle file. The parameters used to make these changes are passed
to Smokeview from the FDS input file using the {\tt SMOKEVIEW\_PARAMETERS}\ keyword on the {\tt \&PROP}\
namelist statement.  For example, the {\tt \&PROP}\ statement:
\begin{verbatim}
&PROP ID='sphere' SMOKEVIEW_PARAMETERS(1:4)='R=0','G=255','B=0',
                   'D=0.5' SMOKEVIEW_ID='sphere' /
&DEVC XYZ=0.5,0.8,1.5, QUANTITY='TEMPERATURE' PROP_ID='sphere' /
\end{verbatim}
creates an FDS device drawn as a sphere colored green with diameter 0.5. Each parameter specified using the
{\tt SMOKEVIEW\_PARAMETERS} keyword
is a text string enclosed in single quotes.  The text string is of the form {\tt 'keyword=value'} where possible
keywords are found in the {\tt objects.svo}\ file (labels beginning with `:').  For example, {\tt R}, {\tt G},
{\tt B} and {\tt D} may be used as keywords to customize the following {\tt sphere} object:
\begin{verbatim}
OBJECTDEF // object for particle file sphere
 sphere
 :R=0 :G=0 :B=0 :D=0.1
 $R $G $B setrgb
 $D drawsphere
\end{verbatim}

Another, Smokeview object, the {\tt tsphere}, uses a texture map or picture to alter the appearance of the object.
The texture map is specified using {\tt SMOKEVIEW\_PARAMETERS} keyword by placing the characters {\tt t\%}\
before the texture file name ({\em e.g.}\ {\tt t\%texturefile.jpg}).

Table \ref{tab:devices_dynamic}\ gives a list of dynamic objects and the keyword/parameter pairs used to specify them.
Each entry shows a picture of the device and the parameters used to customize its appearance.

\begin{longtable}[ht]{|l|l|c|}
\caption{Dynamic Objects}
\label{tab:devices_dynamic}
\\ \hline
{\tt SMOKEVIEW\_ID}  & {\tt SMOKEVIEW\_PARAMETERS} & Image  \\ \hline \hline
\endfirsthead
\caption{Dynamic Objects (continued)}
\\ \hline
{\tt SMOKEVIEW\_ID}  & {\tt SMOKEVIEW\_PARAMETERS} & Image  \\ \hline \hline
\endhead

ball&
\parbox[c]{\boxwidth}{
{\tt SMOKEVIEW\_PARAMETERS(1:6)=}\\
{\tt 'R=128','G=192','B=255',}\\
{\tt 'DX=0.5','DY=.75','DZ=1.0'}\\  \\
R, G, B - red, green, blue color components ranging from 0 to 255\\
DX, DY, DZ - amount ball is stretched along x, y, z axis respectively
} &
\incgraphics{scriptfigures/ball} \\ \hline

cone&
\parbox[c]{\boxwidth}{
{\tt SMOKEVIEW\_PARAMETERS(1:5)=}\\
{\tt 'R=128','G=255','B=192',}\\
{\tt 'D=0.4','H=0.6'}\\ \\
R, G, B - red, green, blue color components ranging from 0 to 255\\
D, H - diameter and length of cone respectively
} &
\incgraphics{scriptfigures/cone} \\ \hline

fan&
\parbox[c]{\boxwidth}{
{\tt SMOKEVIEW\_PARAMETERS(1:11)=}\\
{\tt 'HUB\_R=0','HUB\_G=0','HUB\_B=0',}\\
{\tt 'HUB\_D=0.1','HUB\_L=0.12',}\\
{\tt 'BLADE\_R=128','BLADE\_G=64',}\\
{\tt 'BLADE\_B=32','BLADE\_ANGLE=60.0',}\\
{\tt 'BLADE\_D=0.5','BLADE\_H=0.09'}\\  \\
HUB\_R, HUB\_G, HUB\_B - red, green, blue color components of fan hub ranging from 0 to 255\\
HUB\_D, HUB\_L - diameter and length of fan hub\\
BLADE\_R, BLADE\_G, BLADE\_B - red, green, blue color components of fan blades ranging from 0 to 255\\
BLADE\_ANGLE, BLADE\_D, BLADE\_H - angle, diameter and height of a fan blade
} &
\incgraphics{scriptfigures/fan} \\ \hline

tsphere&
\parbox[c]{\boxwidth}{
    {\tt SMOKEVIEW\_PARAMETERS(1:9)=}\\
    {\tt 'R=255','G=255','B=255',}\\
    {\tt 'AX0=0.0','ELEV0=90.0',}\\
    {\tt 'ROT0=0.0','ROTATION\_RATE=10.0',}\\
    {\tt 'D=1.0',}\\
    {\tt 'tfile="t\%sphere\_cover\_04.png"'}\\ \\
R, G, B - red, green, blue color components ranging from 0 to 255\\
AX0, ELEV0, ROT0 - initial azimuth, elevation and rotation angle respectively\\
ROTATION\_RATE - rotation rate about z axis in degrees per second\\
D - diameter of sphere \\
tfile - name of texture map file

} &
\incgraphics{scriptfigures/tsphere} \\ \hline

tube&
\parbox[c]{\boxwidth}{
{\tt SMOKEVIEW\_PARAMETERS(1:5)=}\\
{\tt 'R=255','G=0','B=0',}\\
{\tt 'D=0.2','L=0.6'}\\ \\
R, G, B - red, green, blue color components ranging from 0 to 255\\
D, L - diameter and length of tube respectively
} &
\incgraphics{scriptfigures/tube} \\ \hline

vent&
\parbox[c]{\boxwidth}{
{\tt SMOKEVIEW\_PARAMETERS(1:6)=}\\
{\tt 'R=192','G=192','B=128',}\\
{\tt 'W=0.5','H=1.0', 'ROT=90.0'}\\ \\
R, G, B - red, green, blue color components ranging from 0 to 255\\
W, H - width and height of vent respectively\\
ROT - angle that vent is rotated
} &
\parbox[c]{\devicewidth}{
\vspace{0.01in}
\includegraphics[width=\devicewidth]{scriptfigures/vent1}
inactive vent\\
\vspace{0.01in}
\includegraphics[width=\devicewidth]{scriptfigures/vent2}
active vent\\
\vspace{0.01in}
}
\\ \hline
\end{longtable} 