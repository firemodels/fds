\documentclass[11pt]{book}

%%%%%%%%%%%%%%%%%%%%%%%%%%%%%%%%%%%%%%%%%%%%%%%%%%%%%%%%%%%%%%%%%%%%%%%%%%%%%%%%%%%%%%%%%%%%%%%%%%%
%                                                                                                 %
% The mathematical style of these documents follows                                               %
%                                                                                                 %
% A. Thompson and B.N. Taylor. The NIST Guide for the Use of the International System of Units.   %
%    NIST Special Publication 881, 2008.                                                          %
%                                                                                                 %
% http://www.nist.gov/pml/pubs/sp811/index.cfm                                                    %
%                                                                                                 %
%%%%%%%%%%%%%%%%%%%%%%%%%%%%%%%%%%%%%%%%%%%%%%%%%%%%%%%%%%%%%%%%%%%%%%%%%%%%%%%%%%%%%%%%%%%%%%%%%%%

% $Date$
% $Revision$
% $Author$

%%%%%%%%%%%%%%%%%%%%%%%%%%%%%%%%%%%%%%%%%%%%%%%%%%%%%%%%%%%%%%%%%%%%%%%%%%%%%%%%%%%%%%%%%%%%%%%%%%%
%                                                                                                 %
% The mathematical style of these documents follows                                               %
%                                                                                                 %
% A. Thompson and B.N. Taylor. The NIST Guide for the Use of the International System of Units.   %
%    NIST Special Publication 881, 2008.                                                          %
%                                                                                                 %
% http://www.nist.gov/pml/pubs/sp811/index.cfm                                                    %
%                                                                                                 %
%%%%%%%%%%%%%%%%%%%%%%%%%%%%%%%%%%%%%%%%%%%%%%%%%%%%%%%%%%%%%%%%%%%%%%%%%%%%%%%%%%%%%%%%%%%%%%%%%%%

% Packages which force the use of better TeX coding
% Mostly from http://tex.stackexchange.com/q/19264
%%\RequirePackage[l2tabu, orthodox]{nag}
%%\usepackage{fixltx2e}
%\usepackage{isomath} % Disabled for the moment because it changes the syntax for bold and roman Greek math symbols
%%\usepackage[all,warning]{onlyamsmath}
%\usepackage{strict} % Commented out for now because it is uncommon. A copy of style.sty is in Manuals/LaTeX_Style_Files/.

\usepackage{times,mathptmx}
\usepackage[pdftex]{graphicx} % use \usepackage[pdftex,demo]{graphicx} to suppress images
\usepackage{tabularx}
\usepackage{multirow}
%\usepackage{pdfsync}
\usepackage{tikz}
\usepackage{bm}
\usepackage{pgfplots}
%\pgfplotsset{compat=1.7}
\usepackage{tocloft}
\usepackage{color}
\definecolor{linknavy}{rgb}{0,0,0.50196}
\definecolor{linkred}{rgb}{1,0,0}
\definecolor{linkblue}{rgb}{0,0,1}
\usepackage{amsmath}
\usepackage{cancel}
\usepackage{float}
\usepackage{caption}
\usepackage{pict2e}
\usepackage{graphpap}
\usepackage{rotating}
\usepackage{geometry}
\usepackage{relsize}
\usepackage{longtable}
\usepackage{xltabular}
\usepackage{lscape}
\usepackage{booktabs}
\usepackage{colortbl}
\definecolor{lavender}{rgb}{0.9, 0.9, 0.98}
\usepackage{amssymb}
\usepackage{threeparttable}
\usepackage{makeidx} % Create index at end of document
\usepackage[nottoc,notlof,notlot]{tocbibind} % Put the bibliography and index in the ToC
\usepackage{lastpage} % Automatic last page number reference.
\usepackage[T1]{fontenc}
\usepackage{enumerate}
\usepackage{upquote}
\usepackage{moreverb}
\usepackage{morefloats}
\usepackage[section]{placeins}
\usepackage{scrextend}
\usepackage{needspace}
\usepackage[backend=biber, style=numeric, sorting=none, backref=true]{biblatex}

\newcommand{\nopart}{\expandafter\def\csname Parent-1\endcsname{}} % To fix table of contents in pdf.
\newcommand{\ct}[1]{\lstinline{#1}}
\newcommand{\tct}[1]{\lstinline[basicstyle=\scriptsize\ttfamily]!#1!}

\usepackage{siunitx}

\usepackage{listings}
\usepackage{textcomp}
\lstset{
    tabsize=4,
    rulecolor=,
    language=Fortran,
        basicstyle=\small\ttfamily,
        upquote=true,
        aboveskip={\baselineskip},
        belowskip={\baselineskip},
        columns=fixed,
        extendedchars=true,
        breaklines=true,
        breakatwhitespace=true,
        frame=none,
        showtabs=false,
        showspaces=false,
        showstringspaces=false,
        identifierstyle=\ttfamily,
        keywordstyle=\color[rgb]{0,0,0},
        commentstyle=\color[rgb]{0,0,0},
        stringstyle=\color[rgb]{0,0,0},
        literate={\_}{}{0\discretionary{\_}{}{\_}}
                 {/}{}{0\discretionary{/}{}{/}}%
}

\usepackage{xr-hyper}
\usepackage[pdftex,
        colorlinks=true,
        urlcolor=linkblue,     % \href{...}{...} external (URL)
        citecolor=linkred,     % citation number colors
        linkcolor=linknavy,    % \ref{...} and \pageref{...}
        pdfproducer={pdflatex},
        pdfpagemode=UseNone,
        bookmarksopen=true,
        plainpages=false,
        verbose]{hyperref}

% The Following commented code makes the ``Draft'' watermark on each page.
%\usepackage{eso-pic}
%\usepackage{type1cm}
%\makeatletter
%   \AddToShipoutPicture{
%     \setlength{\@tempdimb}{.5\paperwidth}
%     \setlength{\@tempdimc}{.5\paperheight}
%     \setlength{\unitlength}{1pt}
%     \put(\strip@pt\@tempdimb,\strip@pt\@tempdimc){
%     \makebox(0,0){\rotatebox{45}{\textcolor[gray]{0.75}{\fontsize{8cm}\selectfont{RC6}}}}}
% }
%\makeatother

\captionsetup[figure]{font=small}

\setlength{\textwidth}{6.5in}
\setlength{\textheight}{9.0in}
\setlength{\topmargin}{0.in}
\setlength{\headheight}{0.in}
\setlength{\headsep}{0.in}
\setlength{\parindent}{0.25in}
\setlength{\oddsidemargin}{0.0in}
\setlength{\evensidemargin}{0.0in}
\setlength{\leftmargini}{\parindent}        % Controls the indenting of the "bullets" in a list
\cftsetindents{section}{.25in}{0.40in}      % Distance from left margin to section number; Width of section number and space before section title
\cftsetindents{subsection}{0.65in}{0.60in}  % Distance from left margin to subsection number; Width of subsection number and space before subsection title
\setlength{\cftfignumwidth}{0.45in}         % Width of figure number and space before figure caption in the list of figures
\setlength{\cfttabnumwidth}{0.45in}         % Width of table number and space before table caption in the list of tables

\makeatletter
\setlength{\@fptop}{0pt}                    % Figures on separate pages pushed to the top
\setlength{\@fpbot}{0pt plus 1fil}
\makeatother

\newcommand{\authortitlesigs}
{
\begin{flushright}
Kevin McGrattan \\
Simo Hostikka \\
Jason Floyd \\
Randall McDermott \\
Marcos Vanella \\
Eric Mueller \\
Chandan Paul
\end{flushright}
}

\newcommand{\logosigs}{
\begin{minipage}[b]{6.25in}
\parbox[b]{.5\textwidth}{\flushleft{\includegraphics[height=1.5in]{../Bibliography/FDS_Logo_lock}}}
\hfill
\parbox[b]{.5\textwidth}{\flushright{\includegraphics[height=1in]{../Bibliography/nistident_flright_vec}}}
\end{minipage}
}

\newcommand{\authorsigs}
{
\begin{flushright}
Kevin McGrattan \\
Randall McDermott \\
Marcos Vanella \\
Eric Mueller \\
{\em Fire Research Division, Engineering Laboratory, Gaithersburg, Maryland} \\[.1in]
Simo Hostikka \\
{\em Aalto University, Espoo, Finland} \\[.1in]
Jason Floyd \\
{\em Fire Safety Research Institute, UL Research Institutes, Columbia, Maryland} \\[.1in]
Chandan Paul \\
{\em The George Washington University, Washington, D.C.}
\end{flushright}
}

\newcommand{\titlesigs}
{
\small
\begin{flushright}
U.S. Department of Commerce \\
{\em Howard Lutnick, Secretary} \\
\hspace{1in} \\
National Institute of Standards and Technology \\
{\em Craig Burkhardt, Acting NIST Director and Acting Under Secretary of Commerce for Standards and Technology}
\end{flushright}
}


\newcommand{\disclaimer}[1]
{
\begin{minipage}[t]{6.25in}
\fontsize{10}{12}\selectfont
\begin{flushright}
Certain commercial entities, equipment, or materials may be identified in this \\
document in order to describe an experimental procedure or concept adequately. \\
Such identification is not intended to imply recommendation or endorsement by the \\
National Institute of Standards and Technology, nor is it intended to imply that the \\
entities, materials, or equipment are necessarily the best available for the purpose.
\end{flushright}
\vspace{3in}
\large
\flushright{\bf National Institute of Standards and Technology Special Publication #1 \\
Natl.~Inst.~Stand.~Technol.~Spec.~Publ.~#1, \pageref{LastPage} pages (October 2013) \\
CODEN: NSPUE2}
\vfill
\hspace{1in}
\end{minipage}
}



\newcommand{\gforneybio}
{
\item[Glenn Forney] is a computer scientist at the Engineering Laboratory of NIST.  He received a
bachelor of science degree in mathematics from Salisbury State College and a master of
science and a doctorate in mathematics from Clemson University.  He joined NIST
in 1986 (then the National Bureau of Standards) and has since worked on developing tools that
provide a better understanding of fire phenomena, most notably Smokeview, a software tool for visualizing
Fire Dynamics Simulator data.
}

\newcommand{\smvoverview}
{
This guide is part of a three volume set of companion documents describing how to use Smokeview
in Volume I, the Smokeview User's Guide~\cite{Smokeview_Users_Guide}, describing technical details of how the visualizations are performed in Volume II, the Smokeview Technical Reference Guide~\cite{Smokeview_Tech_Guide}, and presents example cases
verifying the various visualization capabilities of Smokeview in Volume III, the Smokeview Verification Guide~\cite{Smokeview_Verification_Guide}.  Details on the use and technical background of the Fire Dynamics Simulator is contained in the FDS User's~\cite{FDS_Users_Guide} and Technical reference guide~\cite{FDS_Math_Guide}
respectively.
}

% commands to use for "official" cover and title pages
% see smokeview verification guide to see how they are used

\newcommand{\headerA}[1]{
\begin{flushright}
\fontsize{20}{24}\selectfont
\bf{NIST Special Publication #1}
\end{flushright}
}


\newcommand{\headerB}[1]{
\begin{flushright}
\fontsize{28}{33.6}\selectfont
\bf{#1}
\end{flushright}
}

\newcommand{\headerC}[1]{
\vspace{.15in}
\begin{flushright}
\fontsize{12}{14}\selectfont
#1
\end{flushright}
}

\newcommand{\headerD}[1]{
\begin{flushright}
\fontsize{12}{14}\selectfont
http://dx.doi.org/10.6028/NIST.SP.#1
\end{flushright}
}



\newcommand{\dod}[2]{\frac{\partial #1}{\partial #2}}
\newcommand{\DoD}[2]{\frac{\mathrm{D} #1}{\mathrm{D} #2}}
\newcommand{\dsods}[2]{\frac{\partial^2 #1}{\partial #2^2}}
\renewcommand{\d}{\,\mathrm{d}}
\newcommand{\dx}{\delta x}
\newcommand{\dy}{\delta y}
\newcommand{\dz}{\delta z}
\newcommand{\degF}{$^\circ$F}
\newcommand{\degC}{$^\circ$C}
\newcommand{\x}{x}
\newcommand{\y}{y}
\newcommand{\z}{z}
\newcommand{\dt}{\delta t}
\newcommand{\dn}{\delta n}
\newcommand{\cH}{H}
\newcommand{\hu}{u}
\newcommand{\hv}{v}
\newcommand{\hw}{w}
\newcommand{\la}{\lambda}
\newcommand{\bO}{{\Omega}}
\newcommand{\bo}{{\mathbf{\omega}}}
\newcommand{\btau}{\mathbf{\tau}}
\newcommand{\bdelta}{{\mathbf{\delta}}}
\newcommand{\sumyw}{\sum (Y_\alpha/W_\alpha)}
\newcommand{\oW}{\overline{W}}
\newcommand{\om}{\ensuremath{\omega}}
\newcommand{\omx}{\omega_x}
\newcommand{\omy}{\omega_y}
\newcommand{\omz}{\omega_z}
\newcommand{\erf}{\hbox{erf}}
\newcommand{\erfc}{\hbox{erfc}}
\newcommand{\bF}{{\mathbf{F}}}
\newcommand{\bG}{{\mathbf{G}}}
\newcommand{\bof}{{\mathbf{f}}}
\newcommand{\bq}{{\mathbf{q}}}
\newcommand{\br}{{\mathbf{r}}}
\newcommand{\bu}{{\mathbf{u}}}
\newcommand{\bx}{{\mathbf{x}}}
\newcommand{\bk}{{\mathbf{k}}}
\newcommand{\bv}{{\mathbf{v}}}
\newcommand{\bg}{{\mathbf{g}}}
\newcommand{\bn}{{\mathbf{n}}}
\newcommand{\bS}{{\mathbf{S}}}
\newcommand{\bW}{\overline{W}}
\newcommand{\dS}{d{\mathbf{S}}}
\newcommand{\bs}{{\mathbf{s}}}
\newcommand{\bI}{{\mathbf{I}}}
\newcommand{\hp}{H}
\newcommand{\trho}{\tilde{\rho}}
\newcommand{\dph}{{\delta\phi}}
\newcommand{\dth}{{\delta\theta}}
\newcommand{\tp}{\tilde{p}}
\newcommand{\bp}{\overline{p}}
\newcommand{\dQ}{\dot{Q}}
\newcommand{\dq}{\dot{q}}
\newcommand{\dbq}{\dot{\mathbf{q}}}
\newcommand{\dm}{\dot{m}}
\newcommand{\ha}{\frac{1}{2}}
\newcommand{\ft}{\frac{4}{3}}
\newcommand{\ot}{\frac{1}{3}}
\newcommand{\fofi}{\frac{4}{5}}
\newcommand{\of}{\frac{1}{4}}
\newcommand{\twth}{\frac{2}{3}}
\newcommand{\R}{R}
\newcommand{\be}{\begin{equation}}
\newcommand{\ee}{\end{equation}}
\newcommand{\RE}{\hbox{Re}}
\newcommand{\LE}{\hbox{Le}}
\newcommand{\PR}{\hbox{Pr}}
\newcommand{\PE}{\hbox{Pe}}
\newcommand{\NU}{\hbox{Nu}}
\newcommand{\SC}{\hbox{Sc}}
\newcommand{\SH}{\hbox{Sh}}
\newcommand{\WE}{\hbox{We}}
\newcommand{\OI}{\text{\tiny \hbox{OI}}}
\newcommand{\COTWO}{\text{\tiny \hbox{CO}$_2$}}
\newcommand{\HTWOO}{\text{\tiny \hbox{H}$_2$\hbox{O}}}
\newcommand{\OTWO}{\text{\tiny \hbox{O}$_2$}}
\newcommand{\NTWO}{\text{\tiny \hbox{N}$_2$}}
\newcommand{\CO}{\text{\tiny \hbox{CO}}}
\newcommand{\HCN}{\text{\tiny \hbox{HCN}}}
\newcommand{\F}{\text{\tiny \hbox{F}}}
\newcommand{\C}{\text{\tiny \hbox{C}}}
\newcommand{\Hy}{\text{\tiny \hbox{H}}}
\newcommand{\So}{\text{\tiny \hbox{S}}}
\newcommand{\M}{\text{\tiny \hbox{M}}}
\newcommand{\xx}{\text{\tiny \hbox{x}}}
\newcommand{\yy}{\text{\tiny \hbox{y}}}
\newcommand{\zz}{\text{\tiny \hbox{z}}}
\newcommand{\smvlines}{120~000}

\newcommand{\calH}{\mathcal{H}}
\newcommand{\calR}{\mathcal{R}}

\newcommand{\dif}{\mathrm{d}}
\newcommand{\Div}{\nabla\cdot}
\newcommand{\D}{\mbox{D}}
\newcommand{\mhalf}{\mbox{$\frac{1}{2}$}}
\newcommand{\thalf}{\mbox{\tiny $\frac{1}{2}$}}
\newcommand{\tripleprime}{{\prime\prime\prime}}
\newcommand{\ppp}{{\prime\prime\prime}}
\newcommand{\pp}{{\prime\prime}}

\newcommand{\superscript}[1]{\ensuremath{^{\textrm{\tiny #1}}}}
\newcommand{\subscript}[1]{\ensuremath{_{\textrm{\tiny #1}}}}

\newcommand{\rb}[1]{\raisebox{1.5ex}[0pt]{#1}}

\newcommand{\Ra}{$\Rightarrow$}
\newcommand{\hhref}[1]{\href{#1}{{\tt #1}}}
\newcommand{\fdsinput}[1]{{\scriptsize\verbatiminput{../../Verification/Visualization/#1}}}

\definecolor{AQUAMARINE}{rgb}{0.49804,1.00000,0.83137}
\definecolor{ANTIQUE WHITE}{rgb}{0.98039,0.92157,0.84314}
\definecolor{BEIGE}{rgb}{0.96078,0.96078,0.86275}
\definecolor{BLACK}{rgb}{0.00000,0.00000,0.00000}
\definecolor{BLUE}{rgb}{0.00000,0.00000,1.00000}
\definecolor{BLUE VIOLET}{rgb}{0.54118,0.16863,0.88627}
\definecolor{BRICK}{rgb}{0.61176,0.40000,0.12157}
\definecolor{BROWN}{rgb}{0.64706,0.16471,0.16471}
\definecolor{BURNT SIENNA}{rgb}{0.54118,0.21176,0.05882}
\definecolor{BURNT UMBER}{rgb}{0.54118,0.20000,0.14118}
\definecolor{CADET BLUE}{rgb}{0.37255,0.61961,0.62745}
\definecolor{CHOCOLATE}{rgb}{0.82353,0.41176,0.11765}
\definecolor{COBALT}{rgb}{0.23922,0.34902,0.67059}
\definecolor{CORAL}{rgb}{1.00000,0.49804,0.31373}
\definecolor{CYAN}{rgb}{0.00000,1.00000,1.00000}
\definecolor{DIM GRAY }{rgb}{0.41176,0.41176,0.41176}
\definecolor{EMERALD GREEN}{rgb}{0.00000,0.78824,0.34118}
\definecolor{FIREBRICK}{rgb}{0.69804,0.13333,0.13333}
\definecolor{FLESH}{rgb}{1.00000,0.49020,0.25098}
\definecolor{FOREST GREEN}{rgb}{0.13333,0.54510,0.13333}
\definecolor{GOLD }{rgb}{1.00000,0.84314,0.00000}
\definecolor{GOLDENROD}{rgb}{0.85490,0.64706,0.12549}
\definecolor{GRAY}{rgb}{0.50196,0.50196,0.50196}
\definecolor{GREEN}{rgb}{0.00000,1.00000,0.00000}
\definecolor{GREEN YELLOW}{rgb}{0.67843,1.00000,0.18431}
\definecolor{HONEYDEW}{rgb}{0.94118,1.00000,0.94118}
\definecolor{HOT PINK}{rgb}{1.00000,0.41176,0.70588}
\definecolor{INDIAN RED}{rgb}{0.80392,0.36078,0.36078}
\definecolor{INDIGO}{rgb}{0.29412,0.00000,0.50980}
\definecolor{IVORY}{rgb}{1.00000,1.00000,0.94118}
\definecolor{IVORY BLACK}{rgb}{0.16078,0.14118,0.12941}
\definecolor{KELLY GREEN}{rgb}{0.00000,0.50196,0.00000}
\definecolor{KHAKI}{rgb}{0.94118,0.90196,0.54902}
\definecolor{LAVENDER}{rgb}{0.90196,0.90196,0.98039}
\definecolor{LIME GREEN}{rgb}{0.19608,0.80392,0.19608}
\definecolor{MAGENTA}{rgb}{1.00000,0.00000,1.00000}
\definecolor{MAROON}{rgb}{0.50196,0.00000,0.00000}
\definecolor{MELON}{rgb}{0.89020,0.65882,0.41176}
\definecolor{MIDNIGHT BLUE}{rgb}{0.09804,0.09804,0.43922}
\definecolor{MINT}{rgb}{0.74118,0.98824,0.78824}
\definecolor{NAVY}{rgb}{0.00000,0.00000,0.50196}
\definecolor{OLIVE}{rgb}{0.50196,0.50196,0.00000}
\definecolor{OLIVE DRAB}{rgb}{0.41961,0.55686,0.13725}
\definecolor{ORANGE}{rgb}{1.00000,0.50196,0.00000}
\definecolor{ORANGE RED}{rgb}{1.00000,0.27059,0.00000}
\definecolor{ORCHID}{rgb}{0.85490,0.43922,0.83922}
\definecolor{PINK}{rgb}{1.00000,0.75294,0.79608}
\definecolor{POWDER BLUE}{rgb}{0.69020,0.87843,0.90196}
\definecolor{PURPLE}{rgb}{0.50196,0.00000,0.50196}
\definecolor{RASPBERRY}{rgb}{0.52941,0.14902,0.34118}
\definecolor{RED}{rgb}{1.00000,0.00000,0.00000}
\definecolor{ROYAL BLUE}{rgb}{0.25490,0.41176,0.88235}
\definecolor{SALMON}{rgb}{0.98039,0.50196,0.44706}
\definecolor{SANDY BROWN}{rgb}{0.95686,0.64314,0.37647}
\definecolor{SEA GREEN}{rgb}{0.32941,1.00000,0.62353}
\definecolor{SEPIA}{rgb}{0.36863,0.14902,0.07059}
\definecolor{SIENNA}{rgb}{0.62745,0.32157,0.17647}
\definecolor{SILVER}{rgb}{0.75294,0.75294,0.75294}
\definecolor{SKY BLUE}{rgb}{0.52941,0.80784,0.92157}
\definecolor{SLATEBLUE}{rgb}{0.41569,0.35294,0.80392}
\definecolor{SLATE GRAY}{rgb}{0.43922,0.50196,0.56471}
\definecolor{SPRING GREEN}{rgb}{0.00000,1.00000,0.49804}
\definecolor{STEEL BLUE}{rgb}{0.27451,0.50980,0.70588}
\definecolor{TAN}{rgb}{0.82353,0.70588,0.54902}
\definecolor{TEAL}{rgb}{0.00000,0.50196,0.50196}
\definecolor{THISTLE}{rgb}{0.84706,0.74902,0.84706}
\definecolor{TOMATO }{rgb}{1.00000,0.38824,0.27843}
\definecolor{TURQUOISE}{rgb}{0.25098,0.87843,0.81569}
\definecolor{VIOLET}{rgb}{0.93333,0.50980,0.93333}
\definecolor{VIOLET RED}{rgb}{0.81569,0.12549,0.56471}
\definecolor{WHITE}{rgb}{1.00000,1.00000,1.00000}
\definecolor{YELLOW}{rgb}{1.00000,1.00000,0.00000}

\floatstyle{boxed}
\newfloat{notebox}{H}{lon}
\newfloat{warning}{H}{low}

% Set default longtable alignment
\setlength\LTleft{0pt}
\setlength\LTright{0pt}

% Prevent large paragraph separations
\raggedbottom

% Allow multi-line equations to span page breaks
\allowdisplaybreaks

% Conditional to activate Unstructured Geometry text:
\newif\ifcompgeom
\compgeomtrue

\IfFileExists{../Bibliography/gitrevision.tex}
{\newcommand{\gitrevision}{FDS6.5.3-739-g9e39475}
}
{\newcommand{\gitrevision}{unknown} }

% Math shortcuts
\renewcommand{\sb}[1]{_\mathrm{#1}}

% PDF metadata
\hypersetup{
  pdftitle={Verification and Validation of Commonly Used Empirical Correlations for Fire Scenarios},
  pdfauthor={Overholt, Kristopher J.},
  pdfsubject={This document was created as part of a fire model verification and validation study conducted by the U.S. Nuclear Regulatory Commission (NRC), the Electric Power Research Institute (EPRI), and the National Institute of Standards and Technology (NIST). The results of this verification and validation study are summarized in NUREG-1824 Supplement 1, the Verification and Validation of Selected Fire Models for Nuclear Power Plant Applications. The focus of this document is to compare model predictions of empirical correlations that are commonly used in fire scenarios to various experimentally measured quantities (e.g., hot gas layer temperature, heat flux, plume temperature). A Fortran program was developed along with this document that automates the calculations of the empirical correlations and the verification and validation process. This automated verification and validation process is a method for maintaining the empirical correlations in the long term in a centralized location and enables model verification and validation to be performed on the empirical correlations in a systematic manner. This document is complementary to the verification and validation guides for the Consolidated Model of Fire and Smoke Transport (CFAST) and Fire Dynamics Simulator (FDS) fire models, which are maintained by NIST. The experimental data sets referred to in this study are described in more detail in the FDS Validation Guide (Volume 3 of the FDS Technical Reference Guide) and their respective test reports. For each quantity and empirical correlation, Sections 1 through 8 provide a short description of the governing equations, a verification example, and a validation scatter plot that shows model predictions compared to measured values. For each empirical correlation, the corresponding validation scatter plot lists the experimental relative standard deviation, model relative standard deviation, and bias factor.},
  pdfkeywords={verification; validation; empirical correlations; fire dynamics}
}

\begin{document}

\bibliographystyle{unsrt}
\pagestyle{empty}

\begin{minipage}[t][9in][s]{6.25in}

\headerA{
1169\\
}

\headerB{
Verification and Validation\\
of Commonly Used\\
Empirical Correlations\\
for Fire Scenarios\\
}

\normalsize

\headerC{
{
\flushright{
Kristopher J. Overholt

\vspace*{2\baselineskip}

\begingroup
\hypersetup{urlcolor=black}
\href{http://dx.doi.org/10.6028/NIST.SP.1169}{http://dx.doi.org/10.6028/NIST.SP.1169}
\endgroup
}

\vfill

\flushright{

\includegraphics[width=2.in]{../Bibliography/nistident_flright_vec} \\[.3in]
}
}
}

\end{minipage}

\newpage
\hspace{5in}
\newpage

\frontmatter

\pagenumbering{roman}

\begin{minipage}[t][9in][s]{6.25in}

\headerA{
1169\\}

\headerB{
Verification and Validation\\
of Commonly Used\\
Empirical Correlations\\
for Fire Scenarios\\
}

\headerC{
\flushright{
Kristopher J. Overholt \\
{\em Fire Research Division \\
Engineering Laboratory} \\

\vspace*{2\baselineskip}

\begingroup
\hypersetup{urlcolor=black}
\href{http://dx.doi.org/10.6028/NIST.SP.1169}{http://dx.doi.org/10.6028/NIST.SP.1169} \\
\emph{Git Repository}~\gitrevision
\endgroup

\vspace*{2\baselineskip}

March 2014}}

\vfill

\flushright{\includegraphics[width=1in]{../Bibliography/doc} }

\titlesigs

\end{minipage}

\newpage

\begin{minipage}[t][9in][s]{6.25in}

\flushright{Certain commercial entities, equipment, or materials may be identified in this \\
document in order to describe an experimental procedure or concept adequately. \\
Such identification is not intended to imply recommendation or endorsement by the \\
National Institute of Standards and Technology, nor is it intended to imply that the \\
entities, materials, or equipment are necessarily the best available for the purpose. \\
}

\vspace{3in}

\large
\flushright{\bf National Institute of Standards and Technology Special Publication 1169 \\
Natl.~Inst.~Stand.~Technol.~Spec.~Publ.~1169, \pageref{LastPage} pages (March 2014) \\
% http://dx.doi.org/10.6028/NIST.SP.1169 \\
CODEN: NSPUE2 }

\vfill

\hspace{1in}

\end{minipage}


\clearpage


\pagestyle{plain}

\chapter{Preface}

In 2007, the U.S. Nuclear Regulatory Commission (NRC), together with the Electric Power Research Institute (EPRI) and the National Institute of Standards and Technology (NIST), conducted a research project to verify and validate five fire models that have been used for nuclear power plant (NPP) applications. The results of this effort were documented in a seven-volume report, NUREG-1824 (EPRI 1011999), Verification and Validation of Selected Fire Models for Nuclear Power Plant Applications~\cite{NUREG_1824}.

In 2014, the verification and validation study was expanded, and this document was created to serve as a verification and validation guide for the empirical correlations. The full details of this expanded verification and validation study are summarized in NUREG-1824 Supplement 1 (EPRI 3002002182)~\cite{NUREG_1824_Sup_1}.

The model evaluation process consists of two main components: verification and validation. Verification is a process to check the correctness of the solution of the governing equations. Verification does not imply that the governing equations are appropriate; only that the equations are being solved correctly.

Validation is a process to determine the appropriateness of the governing equations as a mathematical model of the physical phenomena of interest. Typically, validation involves comparing model results with experimental measurement. Differences that cannot be attributed to uncertainty in the measured quantities in the experiment are attributed to the assumptions and simplifications of the physical model.

\cleardoublepage
\phantomsection
\addcontentsline{toc}{chapter}{Contents}
\tableofcontents

\cleardoublepage
\phantomsection
\addcontentsline{toc}{chapter}{List of Figures}
\listoffigures

\cleardoublepage
\phantomsection
\addcontentsline{toc}{chapter}{List of Tables}
\listoftables

\chapter{List of Acronyms}

\begin{tabbing}
\hspace{1.5in} \= \\
AST \> Adiabatic Surface Temperature \\
ATF \> Bureau of Alcohol, Tobacco, Firearms, and Explosives \\
CAROLFIRE \> Cable Response to Live Fire Test Program \\
CFAST \> Consolidated Model of Fire Growth and Smoke Transport \\
FDS \> Fire Dynamics Simulator \\
FM \> Factory Mutual Global \\
HGL \> Hot Gas Layer \\
HRR \> Heat Release Rate \\
LLNL \> Lawrence Livermore National Laboratory \\
NBS \> National Bureau of Standards (former name of NIST) \\
NFPRF \> National Fire Protection Research Foundation \\
NIST \> National Institute of Standards and Technology \\
NRC \> Nuclear Regulatory Commission \\
RTI \> Response Time Index \\
SNL \> Sandia National Laboratory \\
SP \>  Statens Provningsanstalt (Technical Research Institute of Sweden) \\
THIEF \> Thermally-Induced Electrical Failure \\
UL  \> Underwriters Laboratories \\
USN \> United States Navy \\
VTT \> Valtion Teknillinen Tutkimuskeskus (Technical Research Centre of Finland) \\
WTC \> World Trade Center \\
\end{tabbing}

\mainmatter

% !TEX root = Correlation_Guide.tex

\chapter{Introduction}
\label{Introduction_Chapter}

\section{Scope of this Document}

Various empirical correlations exist for calculating quantities of interest related to fire dynamics in a compartment (e.g., hot gas layer temperature, heat flux, plume temperature). The focus of this document is to compare predictions made using empirical correlations to various experimentally measured quantities for a fire in a compartment and to express the accuracy and uncertainty of the predictions in a consistent manner. The empirical correlations selected for use in this document are based on the correlations that are used in nuclear power plant (NPP) applications, and more details are provided in the verification and validation report, NUREG-1824 Supplement 1 (EPRI 3002002182)~\cite{NUREG_1824_Sup_1}.

A Fortran program was developed along with this document that implements the calculations for the empirical correlations and automates the verification and validation process. This automated verification and validation process is a method for maintaining the empirical correlations in the long term in a centralized location and enables model verification and validation to be performed on the empirical correlations in a systematic manner. As new empirical correlations are developed or relevant compartment fire experiments are conducted, they can be added to this verification and validation suite and documented.

This document is complementary to the verification and validation guides for the Consolidated Model of Fire Growth and Smoke Transport (CFAST)~\cite{CFAST_Tech_Guide_6} and Fire Dynamics Simulator (FDS)~\cite{FDS_Verification_Guide, FDS_Validation_Guide}. The experiments referred to in this study are described in more detail in the FDS Validation Guide~\cite{FDS_Validation_Guide} (Volume 3 of the FDS Technical Reference Guide) and their respective test reports. The source code for the empirical correlations calculation program, the verification and validation scripts used to generate this document, and the experimental data shown in this document are freely available for download from the primary website for FDS.\footnote{\href{http://fire.nist.gov/fds}{http://fire.nist.gov/fds}}


\clearpage


\section{Organization of this Document}

For each quantity and empirical correlation, Sections~\ref{HGL_Temperature_Chapter} through \ref{Smoke_Detector_Activation_Time_Chapter} provide a short description of the governing equations, a verification example, and a validation scatter plot that shows model predictions compared to measured values. For each empirical correlation, the corresponding validation scatter plot lists the experimental relative standard deviation, model relative standard deviation, and bias factor.

Section~\ref{Summary_Chapter} includes a table of summary statistics for each quantity and empirical correlation. These statistical metrics can be used to summarize the uncertainty of model predictions and the tendency of a model to underpredict or overpredict a given quantity. More detailed discussion on the application and usage of these statistical metrics is provided in the ``Quantifying Model Uncertainty'' chapter of the FDS Validation Guide~\cite{FDS_Validation_Guide}.

For each of the experimental data sets, Appendix~\ref{Inputs_Chapter} lists the input parameters for the empirical correlations that were used in each of the the validation cases.

\section{List of Experimental Data Sets}

The experimental data sets included in this validation study are shown in Table~\ref{tab:exp_data_sets}. The experiments are described in more detail in the FDS Validation Guide~\cite{FDS_Validation_Guide} (Volume 3 of the FDS Technical Reference Guide) and their respective test reports.

\begin{table}[!ht]
\caption[Experimental data sets used in this validation study]
{Experimental data sets used in this validation study.}

\begin{center}
\begin{tabular}{|l|l|c|}
\hline
Test Series           &  Description                                                        &  Reference                                  \\ \hline \hline
ATF Corridors         &  Gas burner tests in a two-story structure with long hallways       &  \cite{Sheppard:Corridors}                  \\ \hline
CAROLFIRE             &  Electrical cables within a heated test apparatus                   &  \cite{CAROLFIRE}                           \\ \hline
Fleury Heat Flux      &  Propane burner tests with measured heat flux                       &  \cite{Fleury:Masters}                      \\ \hline
FM/SNL                &  Gas and liquid pool fire tests with forced ventilation             &  \cite{Nowlen:NUREG4681, Nowlen:NUREG4527}  \\ \hline
LLNL Enclosure        &  Methane burner tests with various ventilation conditions           &  \cite{Foote:LLNL1986}                      \\ \hline
NBS Multi-Room        &  Gas burner tests in a three-room suite and corridor                &  \cite{Peacock:NBS_Multi-Room}              \\ \hline
NIST/NRC              &  Liquid spray burner tests with various ventilation conditions      &  \cite{Hamins:SP1013-1}                     \\ \hline
NIST Smoke Alarms     &  Single-story manufactured home with furniture fire tests           &  \cite{Bukowski:1}                          \\ \hline
SP AST                &  Gas burner tests in a compartment with a horizontal beam           &  \cite{Wickstrom_AST}                       \\ \hline
SP AST Column         &  Pool fire tests with a vertical column in the center               &  \cite{Sjostrom:AST}                        \\ \hline
Steckler              &  Compartment fire tests conducted at NBS (NIST)                     &  \cite{Steckler:NBSIR_82-2520}              \\ \hline
UL/NFPRF              &  Spray burner tests in a large-scale facility with sprinklers       &  \cite{Sheppard:1, McGrattan:5}             \\ \hline
USN Hawaii            &  Jet fuel fire tests in an aircraft hangar in a warm climate        &  \cite{Gott:1}                              \\ \hline
USN Iceland           &  Jet fuel fire tests in an aircraft hangar in a cold climate        &  \cite{Gott:1}                              \\ \hline
Vettori Flat Ceiling  &  Compartment tests conducted at NIST with residential sprinklers    &  \cite{Vettori:1}                           \\ \hline
VTT Large Hall        &  Heptane pool fire tests in a large-scale facility                  &  \cite{Hostikka:VTT2104}                    \\ \hline
WTC                   &  Compartment spray burner tests conducted at NIST                   &  \cite{NIST_NCSTAR_1-5B}                    \\ \hline
\end{tabular}
\end{center}
\label{tab:exp_data_sets}
\end{table}

% !TEX root = Correlation_Guide.tex

\chapter{Hot Gas Layer Temperature}
\label{HGL_Temperature_Chapter}

The empirical correlations predict an average hot gas layer (HGL) temperature. Because there are different empirical correlations for compartments that are naturally ventilated, mechanically ventilated, or unventilated, the results for HGL temperature are divided into three categories.

\section{Natural Ventilation (MQH)}
\subsection*{Description}

For a compartment with natural ventilation, the correlation of McCaffrey, Quintiere, and Harkleroad (MQH)~\cite{SFPE:Walton} predicts that the hot gas layer (HGL) temperature rise, $\Delta T\sb{g}$, is given by
\be
\Delta T\sb{g} = 6.85 \left( \frac{\dot Q^2}{A\sb{o} \sqrt{H\sb{o}} h\sb{k} A\sb{T}} \right)^{1/3} \quad ^\circ{\rm C}
\label{eq:MQH}
\ee
where $\dot Q$ is the heat release rate (HRR) of the fire~(\si{kW}), $A\sb{o}$ is the area of the ventilation opening~(\si{m^2}), $H\sb{o}$ is the height of the ventilation opening~(\si{m}), and $A\sb{T}$ is the total area of the compartment enclosing surfaces~(\si{m^2}), excluding areas of vent openings. The heat transfer coefficient, $h\sb{k}$ (\si{kW/(m^2.K})), is given by
\be
h\sb{k} = \left\{ \begin{array}{cl}
   \sqrt{k \rho c/t}  & t \le t\sb{p} \\[0.1in]
   k/\delta           & t > t\sb{p}
   \end{array} \right.
\label{eq:MQH_hk_lt}
\ee
where $k$ is the thermal conductivity of the interior lining~(\si{kW/(m.K)}), $\rho$ is its density~(\si{kg/m^3}), $c$ is its specific heat~(\si{kJ/(kg.K)}), and $\delta$ is its thickness~(\si{m}). The thermal penetration time, $t\sb{p}$ (\si{\second}), is given by
\be
t\sb{p} = \left( \frac{\rho c}{k} \right) \left( \frac{\delta}{2} \right)^2
\label{eq:MQH_tp}
\ee


\clearpage


\subsection*{Verification}

This example case is based on Test 51 from the Lawrence-Livermore National Laboratory (LLNL) series. This test involved a compartment with an open door, a methane burner, and natural ventilation.

\begin{table}[!ht]
\caption[Verification case, HGL temperature, natural ventilation]
{Verification case, HGL temperature, natural ventilation.}
\begin{center}
\begin{tabular}{|l|c|}
\hline
\multicolumn{2}{|c|}{}                                                         \\
\multicolumn{2}{|c|}{User-Specified Input}                                     \\
\multicolumn{2}{|c|}{}                                                         \\ \hline
                        &                                                      \\
\rb{Parameter}          &  \rb{Value}                                          \\ \hline \hline
$\dot Q$ (kW)           &  200                                                 \\ \hline
$L$ (m)                 &  6.0                                                 \\ \hline
$W$ (m)                 &  4.0                                                 \\ \hline
$H$ (m)                 &  3.0                                                 \\ \hline
$H\sb{o}$ (m)           &  2.06                                                \\ \hline
$W\sb{o}$ (m)           &  0.76                                                \\ \hline
$k$ (\si{kW/(m.K)})     &  0.000463                                            \\ \hline
$\rho$ (kg/m$^3$)       &  1607                                                \\ \hline
$c$ (\si{kJ/(kg.K)})    &  1.0                                                 \\ \hline
$\delta$ (m)            &  0.10                                                \\ \hline
$T_\infty$ ($^\circ$C)  &  33                                                  \\ \hline
\multicolumn{2}{c}{}                                                           \\ \hline
\multicolumn{2}{|c|}{}                                                         \\
\multicolumn{2}{|c|}{Expected Output}                                          \\
\multicolumn{2}{|c|}{}                                                         \\ \hline
                                 &                                             \\
\multicolumn{1}{|c|}{\rb{Time}}  &  \multicolumn{1}{c|}{\rb{HGL Temperature}}  \\
\multicolumn{1}{|c|}{\rb{(s)}}   &  \multicolumn{1}{c|}{\rb{($^\circ$C)}}      \\ \hline \hline
\multicolumn{1}{|c|}{60}         &  \multicolumn{1}{c|}{111.45}                \\ \hline
\multicolumn{1}{|c|}{120}        &  \multicolumn{1}{c|}{121.05}                \\ \hline
\multicolumn{1}{|c|}{180}        &  \multicolumn{1}{c|}{127.21}                \\ \hline
\end{tabular}
\end{center}
\end{table}


\clearpage


\subsection*{Validation}

A summary of the comparisons between peak predicted and measured compartment temperatures is shown in Fig.~\ref{HGL_Summary_Natural_Ventilation}.

\begin{figure}[!ht]
\begin{center}
\begin{tabular}{l}
\includegraphics[width=4.0in]{SCRIPT_FIGURES/Scatterplots/HGL_Temperature_MQH}
\end{tabular}
\end{center}
\caption[Summary of HGL temperature predictions for natural ventilation tests (MQH)]
{Summary of HGL temperature predictions for natural ventilation tests using the MQH method.}
\label{HGL_Summary_Natural_Ventilation}
\end{figure}

In Fig.~\ref{HGL_Summary_Natural_Ventilation}, the measured values are represented by the horizontal axis and the predicted values by the vertical axis. If a particular prediction and measurement are the same, then the resulting point falls on the solid diagonal line. To better make use of these results, two statistical parameters are calculated for each model and each predicted quantity. The first parameter, $\delta$, is the bias factor, which indicates the extent to which the model, on average, under or over-predicts the measurements of a given quantity. For example, a bias factor of 1.02 indicates that the model over-estimates the measured quantity by \SI{2}{\percent}, on average. The bias factor is shown graphically by the solid red line.

The second parameter, $\widetilde{\sigma}\sb{M}$, is the relative standard deviation of the model, which indicates the variability of the model. In Fig.~\ref{HGL_Summary_Natural_Ventilation}, there are two sets of off-diagonal lines. The first set, shown as dashed black lines, indicate the uncertainty of the experimental measurements in terms of a relative standard deviation, $\widetilde{\sigma}\sb{E}$. It is assumed that the experiments are unbiased; that is, the bias factor for the experimental measurements is 1. The slopes of the dashed black lines are~$1 \pm 2\widetilde{\sigma}\sb{E}$, which represents the \SI{95}{\percent} confidence intervals. The set of red dashed lines indicate the model relative standard deviation, $\widetilde{\sigma}\sb{M}$. The slopes of these lines are~$\delta \pm 2\widetilde{\sigma}\sb{M}$. If the model were as accurate as the measurements against which it is compared, then the two sets of off-diagonal lines would merge. The extent to which the data scatters outside of the experimental bounds is an indication of the degree of model uncertainty.

These symbols and nomenclature are similar for all of the remaining scatter plots in this document.


\clearpage


\section{Forced Ventilation (FPA)}

\subsection*{Description}

For a compartment with forced ventilation, the correlation of Foote, Pagni, and Alvares (FPA)~\cite{SFPE:Walton} predicts that the HGL temperature rise, $\Delta T\sb{g}$, is given by
\be
\Delta T\sb{g} = \left[ 0.63 \left( \frac{\dot Q}{\dot m\sb{g} c\sb{p} T_\infty} \right)^{0.72} \left( \frac{h\sb{k} A\sb{T}}{\dot m\sb{g} c\sb{p}} \right)^{-0.36} \right] T_\infty \quad ^\circ{\rm C}
\label{eq:FPA}
\ee
where $\dot Q$ is the HRR of the fire~(\si{kW}), $\dot m\sb{g}$ is the compartment ventilation mass flow rate~(\si{kg/s}), $c\sb{p}$ is the specific heat of air (\si{kJ/(kg.K)}), $T_\infty$ is the ambient air temperature~(\si{\celsius}), $h\sb{k}$ is the heat transfer coefficient~(\si{kW/(m^2.K)}), and $A\sb{T}$ is the total area of the compartment enclosing surfaces~(\si{m^2}), excluding areas of vent openings. The heat transfer coefficient, $h\sb{k}$ (\si{kW/(m^2.K)}), is given by
\be
h\sb{k} = \left\{ \begin{array}{cl}
   \sqrt{k \rho c/t}  & t \le t\sb{p} \\[0.1in]
   k/\delta           & t > t\sb{p}
   \end{array} \right.
\label{eq:FPA_hk_lt}
\ee
where $k$ is the thermal conductivity of the interior lining~(\si{kW/(m.K)}), $\rho$ is its density~(\si{kg/m^3}), $c$ is its specific heat~(\si{kJ/(kg.K)}), and $\delta$ is its thickness~(\si{m}). The thermal penetration time, $t\sb{p}$ (\si{\second}), is given by
\be
t\sb{p} = \left( \frac{\rho c}{k} \right) \left( \frac{\delta}{2} \right)^2
\label{eq:FPA_tp}
\ee


\clearpage


\subsection*{Verification}

This example case is based on Test 1 from the Factory Mutual and Sandia National Laboratories (FM/SNL) series. This test involved a compartment with an open door, a propylene burner, and forced ventilation.

\begin{table}[!ht]
\caption[Verification case, HGL temperature, forced ventilation]
{Verification case, HGL temperature, forced ventilation.}
\begin{center}
\begin{tabular}{|l|c|}
\hline
\multicolumn{2}{|c|}{}                                                         \\
\multicolumn{2}{|c|}{User-Specified Input}                                     \\
\multicolumn{2}{|c|}{}                                                         \\ \hline
                            &                                                  \\
\rb{Parameter}              &  \rb{Value}                                      \\ \hline \hline
$\dot Q$ (kW)               &  516                                             \\ \hline
$\dot m$ (kg/s)             &  4.5                                             \\ \hline
$c\sb{p}$ (\si{kJ/(kg.K)})  &  1.0                                             \\ \hline
$L$ (m)                     &  18.3                                            \\ \hline
$W$ (m)                     &  12.2                                            \\ \hline
$H$ (m)                     &  6.1                                             \\ \hline
$k$ (\si{kW/(m.K)})         &  0.00023                                         \\ \hline
$\rho$ (kg/m$^3$)           &  1000                                            \\ \hline
$c$ (\si{kJ/(kg.K)})        &  1.16                                            \\ \hline
$\delta$ (m)                &  0.025                                           \\ \hline
$T_\infty$ ($^\circ$C)      &  15                                              \\ \hline
\multicolumn{2}{c}{}                                                           \\ \hline
\multicolumn{2}{|c|}{}                                                         \\
\multicolumn{2}{|c|}{Expected Output}                                          \\ 
\multicolumn{2}{|c|}{}                                                         \\ \hline
                                 &                                             \\
\multicolumn{1}{|c|}{\rb{Time}}  &  \multicolumn{1}{c|}{\rb{HGL Temperature}}  \\
\multicolumn{1}{|c|}{\rb{(s)}}   &  \multicolumn{1}{c|}{\rb{($^\circ$C)}}      \\ \hline \hline
\multicolumn{1}{|c|}{60}         &  \multicolumn{1}{c|}{53.07}                 \\ \hline
\multicolumn{1}{|c|}{120}        &  \multicolumn{1}{c|}{58.13}                 \\ \hline
\multicolumn{1}{|c|}{180}        &  \multicolumn{1}{c|}{61.39}                 \\ \hline
\end{tabular}
\end{center}
\end{table}


\clearpage


\subsection*{Validation}

A summary of the comparisons between peak predicted and measured compartment temperatures is shown in Fig.~\ref{HGL_Summary_Forced_Ventilation_FPA}.

\begin{figure}[!ht]
\begin{center}
\includegraphics[width=4.0in]{SCRIPT_FIGURES/Scatterplots/HGL_Temperature_FPA}
\end{center}
\caption[Summary of HGL temperature predictions for forced ventilation tests (FPA)]
{Summary of HGL temperature predictions for forced ventilation tests using the FPA method.}
\label{HGL_Summary_Forced_Ventilation_FPA}
\end{figure}

Note that the LLNL Enclosure experiments were used to develop the FPA correlation.


\clearpage


\section{Forced Ventilation (DB)}

\subsection*{Description}

For a compartment with forced ventilation, the correlation of Deal and Beyler (DB)~\cite{SFPE:Walton} predicts that the HGL temperature rise, $\Delta T\sb{g}$, is given by
\be
\Delta T\sb{g} = \left( \frac{\dot Q}{\dot m\sb{g} c\sb{p} + h\sb{k} A\sb{T}} \right) \quad ^\circ{\rm C}
\label{eq:DB}
\ee
where $\dot Q$ is the HRR of the fire~(\si{kW}), $\dot m\sb{g}$ is the compartment ventilation mass flow rate~(\si{kg/s}), $c\sb{p}$ is the specific heat of air (\si{kJ/(kg.K)}), $T_\infty$ is the ambient air temperature~(\si{\celsius}), $h\sb{k}$ is the heat transfer coefficient~(\si{kW/(m^2.K)}), and $A\sb{T}$ is the total area of compartment enclosing surfaces~(\si{m^2}), excluding areas of vent openings. The heat transfer coefficient, $h\sb{k}$ (\si{kW/(m^2.K)}), is given by
\be
h\sb{k} = 0.4\ \mathrm{max} \left( \sqrt{\frac{k \rho c}{t}} , \frac{k}{\delta} \right)
\label{eq:DB_hk}
\ee
where $k$ is the thermal conductivity of the interior lining~(\si{kW/(m.K)}), $\rho$ is the density of the interior lining~(\si{kg/m^3}), $c$ is the specific heat of the interior lining~(\si{kJ/(kg.K)}), $t$ is the exposure time~(\si{\second}), and $\delta$ is the thickness of the interior lining~(\si{m}). This model is only valid for times up to 2000 seconds.


\clearpage


\subsection*{Verification}

This example case is based on Test 1 from the Factory Mutual and Sandia National Laboratories (FM/SNL) series. This test involved a compartment with an open door, a propylene burner, and forced ventilation.

\begin{table}[!ht]
\caption[Verification case, HGL temperature, forced ventilation]
{Verification case, HGL temperature, forced ventilation.}
\begin{center}
\begin{tabular}{|l|c|}
\hline
\multicolumn{2}{|c|}{}                                                         \\
\multicolumn{2}{|c|}{User-Specified Input}                                     \\
\multicolumn{2}{|c|}{}                                                         \\ \hline
                            &                                                  \\
\rb{Parameter}              &  \rb{Value}                                      \\ \hline \hline
$\dot Q$ (kW)               &  516                                             \\ \hline
$\dot m$ (kg/s)             &  4.5                                             \\ \hline
$c\sb{p}$ (\si{kJ/(kg.K)})  &  1.0                                             \\ \hline
$L$ (m)                     &  18.3                                            \\ \hline
$W$ (m)                     &  12.2                                            \\ \hline
$H$ (m)                     &  6.1                                             \\ \hline
$k$ (\si{kW/(m.K)})         &  0.00023                                         \\ \hline
$\rho$ (kg/m$^3$)           &  1000                                            \\ \hline
$c$ (\si{kJ/(kg.K)})        &  1.16                                            \\ \hline
$\delta$ (m)                &  0.025                                           \\ \hline
$T_\infty$ ($^\circ$C)      &  15                                              \\ \hline
\multicolumn{2}{c}{}                                                           \\ \hline
\multicolumn{2}{|c|}{}                                                         \\
\multicolumn{2}{|c|}{Expected Output}                                          \\
\multicolumn{2}{|c|}{}                                                         \\ \hline
                                 &                                             \\
\multicolumn{1}{|c|}{\rb{Time}}  &  \multicolumn{1}{c|}{\rb{HGL Temperature}}  \\
\multicolumn{1}{|c|}{\rb{(s)}}   &  \multicolumn{1}{c|}{\rb{($^\circ$C)}}      \\ \hline \hline
\multicolumn{1}{|c|}{60}         &  \multicolumn{1}{c|}{34.60}                 \\ \hline
\multicolumn{1}{|c|}{120}        &  \multicolumn{1}{c|}{40.88}                 \\ \hline
\multicolumn{1}{|c|}{180}        &  \multicolumn{1}{c|}{45.16}                 \\ \hline
\end{tabular}
\end{center}
\end{table}


\clearpage


\subsection*{Validation}

A summary of the comparisons between peak predicted and measured compartment temperatures is shown in Fig.~\ref{HGL_Summary_Forced_Ventilation_DB}.

\begin{figure}[!ht]
\begin{center}
\includegraphics[width=4.0in]{SCRIPT_FIGURES/Scatterplots/HGL_Temperature_DB}
\end{center}
\caption[Summary of HGL temperature predictions for forced ventilation tests (DB)]
{Summary of HGL temperature predictions for forced ventilation tests using the DB method.}
\label{HGL_Summary_Forced_Ventilation_DB}
\end{figure}


\clearpage


\section{No Ventilation (Beyler)}

\subsection*{Description}

For a compartment with no ventilation (closed doors) and constant HRR, the correlation of Beyler~\cite{SFPE:Walton} predicts that the HGL temperature rise, $\Delta T\sb{g}$, is given by
\be
\Delta T\sb{g} = \frac{2 K_2}{K_1^2} (K_1 \sqrt{t} - 1 + e^{-K_1 \sqrt{t}}) \quad ^\circ{\rm C}
\label{eq:Beyler}
\ee
where $t$ is the exposure time~(\si{\second}). $K_1$ is given by
\be
K_1 = \frac{2(0.4\sqrt{k \rho c}) A\sb{T}}{m c\sb{p}}
\label{eq:Beyler_K1}
\ee
where $k$ is the thermal conductivity of the interior lining~(\si{kW/(m.K)}), $\rho$ is the density of the interior lining~(\si{kg/m^3}), $c$ is the specific heat of the interior lining~(\si{kJ/(kg.K)}), $A\sb{T}$ the total area of compartment enclosing surfaces~(\si{m^2}), $m$ is the mass of gas in the compartment~(\si{kg}), and $c\sb{p}$ is the specific heat of air~(\si{kJ/(kg.K)}). $K_2$ is given by
\be
K_2 = \frac{\dot Q}{m c\sb{p}}
\label{eq:Beyler_K2}
\ee
where $\dot Q$ is the HRR of the fire~(\si{kW}).


\clearpage


\subsection*{Verification}

This example case is based on Test 1 from the Lawrence-Livermore National Laboratory (LLNL) series. This test involved a compartment with an open door, a methane burner, and no ventilation.

\begin{table}[!ht]
\caption[Verification case, HGL temperature, no ventilation]
{Verification case, HGL temperature, no ventilation.}
\begin{center}
\begin{tabular}{|l|c|}
\hline
\multicolumn{2}{|c|}{}                                                         \\
\multicolumn{2}{|c|}{User-Specified Input}                                     \\
\multicolumn{2}{|c|}{}                                                         \\ \hline
                            &                                                  \\
\rb{Parameter}              &  \rb{Value}                                      \\ \hline \hline
$\dot Q$ (kW)               &  200                                             \\ \hline
$L$ (m)                     &  6.0                                             \\ \hline
$W$ (m)                     &  4.0                                             \\ \hline
$H$ (m)                     &  4.5                                             \\ \hline
$c\sb{p}$ (\si{kJ/(kg.K)})  &  1.0                                             \\ \hline
$k$ (\si{kW/(m.K)})         &  0.000463                                        \\ \hline
$\rho$ (kg/m$^3$)           &  1607                                            \\ \hline
$c$ (\si{kJ/(kg.K)})        &  1.0                                             \\ \hline
$T_\infty$ ($^\circ$C)      &  23                                              \\ \hline
\multicolumn{2}{c}{}                                                           \\ \hline
\multicolumn{2}{|c|}{}                                                         \\
\multicolumn{2}{|c|}{Expected Output}                                          \\
\multicolumn{2}{|c|}{}                                                         \\ \hline
                                 &                                             \\
\multicolumn{1}{|c|}{\rb{Time}}  &  \multicolumn{1}{c|}{\rb{HGL Temperature}}  \\
\multicolumn{1}{|c|}{\rb{(s)}}   &  \multicolumn{1}{c|}{\rb{($^\circ$C)}}      \\ \hline \hline
\multicolumn{1}{|c|}{60}         &  \multicolumn{1}{c|}{49.87}                 \\ \hline
\multicolumn{1}{|c|}{120}        &  \multicolumn{1}{c|}{63.33}                 \\ \hline
\multicolumn{1}{|c|}{180}        &  \multicolumn{1}{c|}{73.67}                 \\ \hline
\end{tabular}
\end{center}
\end{table}


\clearpage


\subsection*{Validation}

A summary of the comparisons between peak predicted and measured compartment temperatures is shown in Fig.~\ref{HGL_Summary_No_Ventilation}.

\begin{figure}[!ht]
\begin{center}
\begin{tabular}{l}
\includegraphics[width=4.0in]{SCRIPT_FIGURES/Scatterplots/HGL_Temperature_Beyler}
\end{tabular}
\end{center}
\caption[Summary of HGL temperature predictions for no ventilation tests (Beyler)]
{Summary of HGL temperature predictions for no ventilation tests using the Beyler method.}
\label{HGL_Summary_No_Ventilation}
\end{figure}


% !TEX root = Correlation_Guide.tex

\chapter{HGL Depth}
\label{HGL_Depth_Chapter}

The HGL depth is defined as the distance between the ceiling and the HGL height.

\section{ASET Method}

\subsection*{Description}

For a compartment with no ventilation (closed doors) and constant HRR, the available safe egress time (ASET)~\cite{Walton:1}
correlation predicts that the HGL height, $z$~(\si{m}), is given by~\cite{SFPE:Milke}
\be
A\sb{s} \frac{dz}{dt} = \frac{dV\sb{ul}}{dt} = \dot V\sb{ul}
\label{eq:ASET_1}
\ee
where $A\sb{s}$ is the area of the boundary surfaces~(\si{m^2}), and $V\sb{ul}$ is the volume of the HGL~(\si{m^3}).
The change in volume of the upper layer, $\dot V\sb{ul}$~(\si{m^3/s}), is given by
\be
\dot V\sb{ul} = \dot V\sb{exp} + \dot V\sb{ent}
\label{eq:ASET_2}
\ee
The volumetric expansion rate, $\dot V\sb{exp}$~(\si{m^3/s}), is given by~\cite{SFPE:Mowrer}
\be
\dot V\sb{exp} = \frac{\dot Q\sb{net}}{\rho\sb{g} c\sb{p} T\sb{g}} \approx \frac{(1 - \chi\sb{l}) \dot Q\sb{f}}{353}
\label{eq:ASET_3}
\ee
where $\dot Q\sb{net}$ and $\dot Q\sb{f}$ are the net and actual HRRs~(\si{kW}), respectively, $\rho\sb{g}$, $c\sb{p}$ and $T\sb{g}$ are the density~(\si{kg/m^3}), specific heat~(\si{kJ/(kg.K)}), and temperature~(\si{K}) of air in the HGL, respectively, and $\chi\sb{l}$ is the heat loss fraction to the enclosure boundaries~(-).
The volumetric entrainment rate, $\dot V\sb{ent}$~(\si{m^3/s}), is given by~\cite{Zukoski:1981}
\be
\dot V\sb{ent} = k\sb{v} \dot Q^{1/3} z^{5/3} = \frac{0.21}{K\sb{f}} \left( \frac{g}{\rho_\infty T_\infty} \right)^{1/3} (K\sb{f} \dot Q)^{1/3} (z - z\sb{f})^{5/3}
\label{eq:ASET_4}
\ee
where $k\sb{v}$ is the volumetric entrainment coefficient, $g$ is the acceleration due to gravity~(\si{m/s^2}), $\rho_\infty$ and $T_\infty$ are the density~(\si{kg/m^3}) and temperature~(\si{K}) of ambient air, respectively, $K\sb{f}$ is the location factor~(-), and $z\sb{f}$ is the fuel height~(\si{m}). The location factor has a value of 1, 2, or 4, which corresponds to a fire away from walls or corners, a fire adjacent to a wall, or a fire located in a corner, respectively.

The HGL height, $z$, in Eq.~\ref{eq:ASET_1} can be calculated iteratively using
\be
z|_{t+1} = z|_t - \frac{\dot V\sb{ul}}{L W} \Delta t
\label{eq:ASET_5}
\ee
where $L$ and $W$ are the length and width of the compartment~(\si{m}), respectively, and $\Delta t$ is the time step size~(\si{s}).


\clearpage


\subsection*{Verification}

Test: NIST/NRC Test 1

\begin{table}[!ht]
\caption[Verification input parameters, HGL depth]
{Verification input parameters, HGL depth.}
\begin{center}
\begin{tabular}{|l|c|}
\hline
                        &              \\
\rb{Input Parameter}    &  \rb{Value}  \\ \hline \hline
$\dot Q$ (kW)           &  410         \\ \hline
$L$ (m)                 &  21.66       \\ \hline
$W$ (m)                 &  7.04        \\ \hline
$H$ (m)                 &  3.82        \\ \hline
$k$ (\si{kW/(m.K)})     &  0.00012     \\ \hline
$\rho$ (kg/m$^3$)       &  737         \\ \hline
$c$ (\si{kJ/(kg.K)})    &  1.42        \\ \hline
$T_\infty$ ($^\circ$C)  &  22          \\ \hline
Location Factor (-)     &  1           \\ \hline
$\chi\sb{l}$ (-)        &  0           \\ \hline
$z\sb{f}$ (-)           &  0           \\ \hline
\end{tabular}
\end{center}
\end{table}

\noindent Expected result: At 10~s, the HGL depth $z$ is 0.35~m; at 20~s, $z$ is 0.65~m; at 30~s, $z$ is 0.93~m.


\clearpage


\subsection*{Validation}

\begin{figure}[!ht]
\begin{center}
\begin{tabular}{l}
\includegraphics[width=4.0in]{SCRIPT_FIGURES/Scatterplots/HGL_Depth_ASET}
\end{tabular}
\end{center}
\caption[Summary of HGL depth predictions (ASET)]
{Summary of HGL depth predictions using ASET.}
\label{HGL_Depth_ASET}
\end{figure}


\clearpage


\section{Yamana and Tanaka Method}

\subsection*{Description}

For a compartment with no ventilation (closed doors) and constant HRR, the correlation of Yamana and Tanaka~\cite{Tanaka:1} predicts that the HGL height, $z$, is given by
\be
z = \left( \frac{2 k \dot Q^{1/3} t}{3 A\sb{c}} + \frac{1}{h\sb{c}^{2/3}} \right)^{-3/2}
\label{eq:Yamana_Tanaka}
\ee
where $\dot Q$ is the HRR~(\si{kW}), $t$ is the time after ignition~(\si{s}), $A\sb{c}$ is the compartment floor area~(\si{m^2}), and $h\sb{c}$ is the compartment height~(\si{m}). The constant $k$ is given by
\be
k = \frac{0.076}{(353/T\sb{g})}
\ee
where $T\sb{g}$ is the HGL temperature~(\si{K}).

\subsection*{Verification}

Test: NIST/NRC Test 1
\\ \\
\noindent Note: In this verification case, the method of Beyler is used to calculate the HGL temperature, $T\sb{g}$.

% In the FDTs spreadsheets, this HGL depth correlation is used with the MQH correlation (natural ventilation).

\begin{table}[!ht]
\caption[Verification input parameters, HGL depth]
{Verification input parameters, HGL depth.}
\begin{center}
\begin{tabular}{|l|c|}
\hline
                        &              \\
\rb{Input Parameter}    &  \rb{Value}  \\ \hline \hline
$\dot Q$ (kW)           &  410         \\ \hline
$L$ (m)                 &  21.66       \\ \hline
$W$ (m)                 &  7.04        \\ \hline
$H$ (m)                 &  3.82        \\ \hline
$k$ (\si{kW/(m.K)})     &  0.00012     \\ \hline
$\rho$ (kg/m$^3$)       &  737         \\ \hline
$c$ (\si{kJ/(kg.K)})    &  1.42        \\ \hline
$T_\infty$ ($^\circ$C)  &  22          \\ \hline
\end{tabular}
\end{center}
\end{table}

\noindent Expected result: At 10~s, the HGL depth $z$ is 0.28~m; at 20~s, $z$ is 0.53~m; at 30~s, $z$ is 0.75~m.


\clearpage


\subsection*{Validation}

\begin{figure}[!ht]
\begin{center}
\begin{tabular}{l}
\includegraphics[width=4.0in]{SCRIPT_FIGURES/Scatterplots/HGL_Depth_Yamana_Tanaka}
\end{tabular}
\end{center}
\caption[Summary of HGL depth predictions (Yamana and Tanaka)]
{Summary of HGL depth predictions using Yamana and Tanaka method.}
\label{HGL_Depth_YT}
\end{figure}



% !TEX root = Correlation_Guide.tex

\chapter{Plume Temperature}
\label{Plume_Chapter}

The fire plume transports hot gases into the HGL. Its temperature is greater than the ceiling jet and HGL temperature. It is particularly important scenarios that involve targets directly above a potential fire.

\section{Heskestad Method}
\label{sec:Heskestad}

\subsection*{Description}

For a fire plume, the correlation by Heskestad~\cite{SFPE:Heskestad} predicts that the increase in centerline temperature, $\Delta T_0$~(\si{\celsius}), is given by
\be
\Delta T_0 = \frac{9.1 \left( \frac{T_\infty}{g c\sb{p}^2 \rho_{\infty}^2} \right)^{1/3} \dot Q\sb{c}^{2/3}}{(z-z_0)^{5/3}}
\label{eq:Heskestad}
\ee
where $T_\infty$ is the ambient air temperature~(\si{\celsius}), $g$ is the acceleration of gravity~(m/s$^2$), $c\sb{p}$ is the specific heat of air~(\si{kJ/(kg.K)}), $\rho_{\infty}$ is the ambient air density~(\si{kg/m^3}), and $z$ is the elevation above the fire source~(\si{m}). The convective HRR, $\dot Q\sb{c}$ (\si{kW}), is given by
\be
\dot Q\sb{c} = \dot Q (1 - \chi\sb{r})
\label{eq:Heskestad_Qc}
\ee
where $\dot Q$ is the total HRR~(\si{kW}), and $\chi\sb{r}$ is the radiative fraction. Note that the total HRR $\dot Q$ is the actual HRR, not the idealized HRR. The hypothetical virtual origin of the fire, $z_0$ (\si{m}), is given by
\be
z_0 = -1.02 D + 0.083 \dot Q^{2/5}
\label{eq:Heskestad_z0}
\ee
where $D$ is the diameter of the fire source~(\si{m}) and is given by
\be
D = \sqrt{\frac{4 A}{\pi}}
\label{eq:Heskestad_D}
\ee
where $A$ is the area of the fire source~(\si{m^2}).
Note that this plume temperature correlation is only valid above the mean flame height.


\clearpage


\subsection*{Verification}

This example case is based on Test 1 from the VTT~\cite{Hostikka:VTT2104} series. This test involved a large test hall with closed doors, a heptane pool fire, and no ventilation.

\begin{table}[!ht]
\caption[Verification case, plume temperature]
{Verification case, plume temperature.}
\begin{center}
\begin{tabular}{|l|c|}
\hline
\multicolumn{2}{|c|}{}                        \\
\multicolumn{2}{|c|}{User-Specified Input}    \\
\multicolumn{2}{|c|}{}                        \\ \hline
                            &                 \\
\rb{Parameter}              &  \rb{Value}     \\ \hline \hline
$\dot Q$ (m)                &  1245           \\ \hline
$c\sb{p}$ (\si{kJ/(kg.K)})  &  1.0            \\ \hline
$z$ (m)                     &  6              \\ \hline
$A$ (m$^2$)                 &  1.075          \\ \hline
$\chi\sb{r}$                &  0.40           \\ \hline
$T_\infty$ (\si{\celsius})  &  22             \\ \hline
\multicolumn{2}{c}{}                          \\ \hline
\multicolumn{2}{|c|}{}                        \\
\multicolumn{2}{|c|}{Calculated Output}       \\
\multicolumn{2}{|c|}{}                        \\ \hline
\multicolumn{2}{|c|}{}                        \\
\multicolumn{2}{|c|}{\rb{Plume Temperature}}  \\
\multicolumn{2}{|c|}{\rb{(\si{\celsius})}}    \\ \hline \hline
\multicolumn{2}{|c|}{133.78}                  \\ \hline
\end{tabular}
\end{center}
\end{table}


\clearpage


\subsection*{Validation}

A summary of the comparisons between peak predicted and measured plume temperatures is shown in Fig.~\ref{Plume Temperature (Heskestad)}.

\begin{figure}[!ht]
\begin{center}
\begin{tabular}{l}
\includegraphics[width=4.0in]{SCRIPT_FIGURES/Scatterplots/Plume_Temperature_Heskestad}
\end{tabular}
\end{center}
\caption[Summary of plume temperature predictions (Heskestad)]
{Summary of plume temperature predictions using the Heskestad method.}
\label{Plume Temperature (Heskestad)}
\end{figure}


\clearpage


\section{McCaffrey Method}
\label{sec:McCaffrey}

\subsection*{Description}

For a fire plume, the correlation by McCaffrey~\cite{McCaffrey:NBSIR_79-1910} predicts that the increase in centerline temperature, $\Delta T_0$~(\si{\celsius}), is given by
\be
\Delta T_0 = \left[ \left( \frac{\kappa}{0.9 \sqrt{2 g}} \right)^2 \left( \frac{z}{\dot Q^{2/5}} \right)^{2 \eta - 1} \right] T_\infty
\label{eq:McCaffrey}
\ee
where $g$ is the acceleration of gravity~(\si{m/s^2}), $z$ is the elevation above the fire source~(\si{m}), $\dot Q$ is the HRR~(\si{kW}), and $T_\infty$ is the ambient air temperature~(\si{\celsius}). The constants $\eta$ and $\kappa$ are a function of the height $z$ within the plume and are listed in Table~\ref{tbl:McCaffrey_constants}.

\vspace{\baselineskip}
\begin{table}[!ht]
\begin{center}
\caption[Constants used in McCaffrey plume temperature correlation]
{Constants used in McCaffrey plume temperature correlation.}
\label{tbl:McCaffrey_constants}
\begin{tabular}{|c|c|c|c|}
\hline
Region        &  $z/\dot Q^{2/5}$  &   $\eta$  &  $\kappa$  \\ \hline \hline
Continuous    &  < 0.08            &   1/2     &  6.8       \\ \hline
Intermittent  &  < 0.08 -- 0.2     &   0       &  1.9       \\ \hline
Plume         &  > 0.2             &   -1/3    &  1.1       \\
\hline
\end{tabular}
\end{center}
\end{table}


\clearpage


\subsection*{Verification}

This example case is based on Test 1 from the VTT~\cite{Hostikka:VTT2104} series. This test involved a large test hall with closed doors, a heptane pool fire, and no ventilation.

\begin{table}[!ht]
\caption[Verification case, plume temperature]
{Verification case, plume temperature.}
\begin{center}
\begin{tabular}{|l|c|}
\hline
\multicolumn{2}{|c|}{}                        \\
\multicolumn{2}{|c|}{User-Specified Input}    \\
\multicolumn{2}{|c|}{}                        \\ \hline
                            &                 \\
\rb{Parameter}              &  \rb{Value}     \\ \hline \hline
$\dot Q$ (m)                &  1245           \\ \hline
$z$ (m)                     &  6              \\ \hline
$T_\infty$ (\si{\celsius})  &  22             \\ \hline
\multicolumn{2}{c}{}                          \\ \hline
\multicolumn{2}{|c|}{}                        \\
\multicolumn{2}{|c|}{Calculated Output}       \\
\multicolumn{2}{|c|}{}                        \\ \hline
\multicolumn{2}{|c|}{}                        \\
\multicolumn{2}{|c|}{\rb{Plume Temperature}}  \\
\multicolumn{2}{|c|}{\rb{(\si{\celsius})}}    \\ \hline \hline
\multicolumn{2}{|c|}{153.21}                  \\ \hline
\end{tabular}
\end{center}
\end{table}


\clearpage


\subsection*{Validation}

A summary of the comparisons between peak predicted and measured plume temperatures is shown in Fig.~\ref{Plume Temperature (McCaffrey)}.

\begin{figure}[!ht]
\begin{center}
\begin{tabular}{l}
\includegraphics[width=4.0in]{SCRIPT_FIGURES/Scatterplots/Plume_Temperature_McCaffrey}
\end{tabular}
\end{center}
\caption[Summary of plume temperature predictions (McCaffrey)]
{Summary of plume temperature predictions using the McCaffrey method.}
\label{Plume Temperature (McCaffrey)}
\end{figure}


% !TEX root = Correlation_Guide.tex

\chapter{Target Temperature}
\label{Target_Temperature_Chapter}

The calculation of target temperature is a common objective of fire modeling analyses. The targets in this validation study include electrical cables as well as unprotected and protected steel members.

\section{Cable Failure Time}

\subsection*{Description}

Even though an electrical cable is considered a ``target'', the cable failure time quantity is included in this study to assess the models' ability to predict the time to cable failure. This is an indirect way of assessing the model prediction of temperature. The model only predicts the interior temperature of the cable, and the failure time is considered as the time at which the predicted temperature rises above an experimentally determined value.

The thermally-induced electrical failure (THIEF) of a cable can be predicted via a simple one-dimensional heat transfer calculation, under the assumption that the cable can be treated as a homogeneous cylinder~\cite{CAROLFIRE}. The governing equation for the cable temperature,
$T(r,t)$, is given by
\be
\rho c \left( \frac{\partial T}{\partial t} \right) = \frac{1}{r} \frac{\partial}{\partial r} k r \left( \frac{\partial T}{\partial r} \right)
\label{eq:cable_temp}
\ee
where $\rho$, $c$ and $k$ are the effective density, specific heat, and thermal conductivity of the solid, respectively, and $r$ is the radius of the cable~(\si{m}).
The boundary condition is defined as
\be
\dot q'' = k \left( \frac{\partial T}{\partial r} \right) (r,t)
\ee
A finite difference approximation to Eq.~\ref{eq:cable_temp} is given by
\be
\rho c \left[ \frac{T_i^{n+1} - T_i^n}{\delta t} \right] = \frac{2 k}{(r_{i+1} + r_i)} \frac{1}{2 \delta r} \left[ r_i \frac{T_{i+1}^n - T_i^n}{\delta r} - r_{i-1} \frac{T_{i}^n - T_{i-1}^n}{\delta r} + r_i \frac{T_{i+1}^{n+1} - T_i^{n+1}}{\delta r} - r_{i-1} \frac{T_{i}^{n+1} - T_{i-1}^{n+1}}{\delta r} \right]
\ee
where the time step $\delta t$ is given by
\be
\delta t = \frac{c \rho \delta r^2}{2 k}
\ee


\clearpage


\subsection*{Verification}

This example case is based on Penlight Test 7 from the Cable Response to Live Fire (CAROLFIRE) series. This test involved a cable inside of conduit that was located in a heated cylindrical enclosure.

\begin{table}[!ht]
\caption[Verification case, cable failure time]
{Verification case, cable failure time.}
\begin{center}
\begin{tabular}{|c|c|c|c|}
\hline
\multicolumn{4}{|c|}{}                                                                                   \\
\multicolumn{4}{|c|}{User-Specified Input}                                                               \\
\multicolumn{4}{|c|}{}                                                                                   \\ \hline
\multicolumn{2}{|c|}{}                             &  \multicolumn{2}{c|}{}                              \\
\multicolumn{2}{|l|}{\rb{Parameter}}               &  \multicolumn{2}{l|}{\rb{Value}}                    \\ \hline \hline
\multicolumn{2}{|l|}{Time Ramp}                    &  \multicolumn{2}{l|}{0, 80, 820, 1240, 1800, 1900}  \\ \hline
\multicolumn{2}{|l|}{Temperature Ramp}             &  \multicolumn{2}{l|}{24, 460, 460, 460, 460, 0}     \\ \hline
\multicolumn{2}{|l|}{Cable Diameter (mm)}          &  \multicolumn{2}{l|}{16.3}                          \\ \hline
\multicolumn{2}{|l|}{Mass per Unit Length (kg/m)}  &  \multicolumn{2}{l|}{0.529}                         \\ \hline
\multicolumn{2}{|l|}{Jacket Thickness (mm)}        &  \multicolumn{2}{l|}{1.5}                           \\ \hline
\multicolumn{2}{|l|}{Conduit Diameter (mm)}        &  \multicolumn{2}{l|}{50}                            \\ \hline
\multicolumn{2}{|l|}{Conduit Thickness (mm)}       &  \multicolumn{2}{l|}{4.9}                           \\ \hline
\multicolumn{2}{|l|}{$T_\infty$ ($^\circ$C)}       &  \multicolumn{2}{l|}{24}                            \\ \hline
\multicolumn{2}{c}{}                                                                                     \\ \hline
\multicolumn{4}{|c|}{}                                                                                   \\
\multicolumn{4}{|c|}{Expected Output}                                                                    \\
\multicolumn{4}{|c|}{}                                                                                   \\ \hline
           &                    &                    &                                                   \\
           &  \rb{Exposing}     &  \rb{Cable}        &  \rb{Conduit}                                     \\
\rb{Time}  &  \rb{Temperature}  &  \rb{Temperature}  &  \rb{Temperature}                                 \\
\rb{(s)}   &  \rb{($^\circ$C)}  &  \rb{($^\circ$C)}  &  \rb{($^\circ$C)}                                 \\ \hline \hline
50         &  296.3             &  24.1              &  32.4                                             \\ \hline
80         &  460.0             &  24.6              &  52.3                                             \\ \hline
1473       &  460.0             &  400               &  440.3                                            \\ \hline
\end{tabular}
\end{center}
\end{table}


\clearpage


\subsection*{Validation}

A summary of the comparisons between predicted and measured cable failure times (the time at which the cable reaches its threshold failure temperature) is shown in Fig.~\ref{Surface_Temperature_THIEF_Summary}.

\begin{figure}[!ht]
\begin{center}
\begin{tabular}{l}
\includegraphics[width=4.0in]{SCRIPT_FIGURES/Scatterplots/Cable_Failure_Time}
\end{tabular}
\end{center}
\caption[Summary of cable failure time predictions]
{Summary of cable failure time predictions.}
\label{Surface_Temperature_THIEF_Summary}
\end{figure}

\clearpage


\section{Unprotected Steel Temperature}

\subsection*{Description}

The temperature rise, $\Delta T\sb{s}$, of an unprotected steel member exposed to fire can be predicted using~\cite{SFPE:Milke2}
\be
\Delta T\sb{s} = \frac{F}{V} \frac{1}{\rho\sb{s} c\sb{s}} \left[ h\sb{c} (T\sb{f} - T\sb{s}) + \sigma \epsilon (T\sb{f}^4 - T\sb{s}^4) \right] \Delta t
\label{eq:unprotected_steel}
\ee
where $F/V$ is the ratio of heated surface area to volume~(\si{m^{-1}}), $\rho\sb{s}$ is the density of steel~(\si{kg/m^3}), $c\sb{s}$ is the specific heat of steel~(\si{J/(kg.K)}), $h\sb{c}$ is the convective heat transfer coefficient~(\si{W/(m^2.K)}), $T\sb{f}$ is the exposing fire temperature~(\si{K}), $T\sb{s}$ is the steel temperature~(\si{K}), $\sigma$ is the Stefan-Boltzmann constant (\si{W/(m^2.K^4)}), $\epsilon$ is the flame emissivity~(-), and $\Delta t$ is the time step~(\si{s}). Note that the HGL temperature, plume temperature, or other exposing temperature can be used as the fire temperature, $T\sb{f}$.


\clearpage


\subsection*{Verification}

This example case is based on the 1.1 m Diesel Fire Test from the SP AST Column series. This test involved a large test hall with a steel column located in the middle of a diesel pool fire.

\begin{table}[!ht]
\caption[Verification case, unprotected steel temperature]
{Verification case, unprotected steel temperature.}
\begin{center}
\begin{tabular}{|c|c|c|}
\hline
\multicolumn{3}{|c|}{}                                                                   \\
\multicolumn{3}{|c|}{User-Specified Input}                                               \\
\multicolumn{3}{|c|}{}                                                                   \\ \hline
\multicolumn{2}{|c|}{}                               &  \multicolumn{1}{c|}{}            \\
\multicolumn{2}{|l|}{\rb{Parameter}}                 &  \multicolumn{1}{c|}{\rb{Value}}  \\ \hline \hline
\multicolumn{2}{|l|}{$F/V$ (1/m)}                    &  \multicolumn{1}{c|}{205}         \\ \hline
\multicolumn{2}{|l|}{$\rho\sb{s}$ (kg/m$^3$)}        &  \multicolumn{1}{c|}{7833}        \\ \hline
\multicolumn{2}{|l|}{$c\sb{s}$ (\si{kJ/(kg.K)})}     &  \multicolumn{1}{c|}{0}           \\ \hline
\multicolumn{2}{|l|}{$\epsilon$ (-)}                 &  \multicolumn{1}{c|}{0}           \\ \hline
\multicolumn{2}{|l|}{$h\sb{c}$ (\si{W/(m^2.K)})}     &  \multicolumn{1}{c|}{25}          \\ \hline \hline
\multicolumn{2}{|l|}{Correlation for $T\sb{f}$ (-)}  &  \multicolumn{1}{c|}{McCaffrey}   \\ \hline \hline
\multicolumn{2}{|l|}{$\dot Q$ (kW)}                  &  \multicolumn{1}{c|}{1434}        \\ \hline
\multicolumn{2}{|l|}{Height (m)}                     &  \multicolumn{1}{c|}{1}           \\ \hline
\multicolumn{2}{|l|}{$T_\infty$ ($^\circ$C)}         &  \multicolumn{1}{c|}{20}          \\ \hline
\multicolumn{2}{c}{}                                                                     \\ \hline
\multicolumn{3}{|c|}{}                                                                   \\
\multicolumn{3}{|c|}{Expected Output}                                                    \\
\multicolumn{3}{|c|}{}                                                                   \\ \hline
           &                    &                                                        \\
           &  \rb{Fire}         &  \rb{Steel}                                            \\
\rb{Time}  &  \rb{Temperature}  &  \rb{Temperature}                                      \\
\rb{(s)}   &  \rb{($^\circ$C)}  &  \rb{($^\circ$C)}                                      \\ \hline \hline
15         &  872.81            &  89.7                                                  \\ \hline
30         &  872.81            &  162.4                                                 \\ \hline
45         &  872.81            &  232.8                                                 \\ \hline
\end{tabular}
\end{center}
\end{table}


\clearpage


\section{Protected Steel Temperature}
\label{info:protected_steel_temperature}

\subsection*{Description}

The temperature rise, $\Delta T\sb{s}$, of a protected steel member exposed to fire can be predicted, but we must first determine if the thermal capacity of the insulation layer should be accounted for or if it can be neglected.
\be
\Delta T\sb{s} = \left\{ \begin{array}{cl}
   k\sb{i} \left( \frac{T\sb{f} - T\sb{s}}{c\sb{s} h \frac{W}{D}} \right) \Delta t        &  c\sb{s} \frac{W}{D} > 2 c\sb{i} \rho\sb{i} h \\[0.1in]
   \frac{k\sb{i}}{h} \left( \frac{T\sb{f} - T\sb{s}}{c\sb{s} \frac{W}{D} + \frac{1}{2} c\sb{i} \rho\sb{i} h} \right) \Delta t  &  c\sb{s} \frac{W}{D} < 2 c\sb{i} \rho\sb{i} h
   \end{array} \right.
\label{eq:protected_steel}
\ee
where $k\sb{i}$ is the thermal conductivity of the insulation material~(\si{W/(m.K)}), $T\sb{f}$ is the exposing fire temperature~(\si{K}), $T\sb{s}$ is the steel temperature~(\si{K}), $c\sb{s}$ is the specific heat of steel~(\si{J/(kg.K)}), $c\sb{i}$ is the specific heat of the insulation material~(\si{J/(kg.K)}), $h$ is the thickness of the insulation~(\si{m}), $W/D$ is the ratio of the weight of steel section per unit length to the heated perimeter~(\si{kg/m^2}), $\rho\sb{i}$ is the density of the insulating material~(\si{kg/m^3}), and $\Delta t$ is the time step~(\si{s}). Note that the HGL temperature, plume temperature, or other exposing temperature can be used as the fire temperature, $T\sb{f}$.


\clearpage


\subsection*{Verification}

This example case examines the temperature of a bar structural element in Test 4 of the World Trade Center (WTC) series. This test involved a simple compartment with a heptane spray burner and various structural elements with varying amounts of sprayed fire-resistive materials.

\begin{table}[!ht]
\caption[Verification case, protected steel temperature]
{Verification case, protected steel temperature.}
\begin{center}
\begin{tabular}{|c|c|c|}
\hline
\multicolumn{3}{|c|}{}                                                                   \\
\multicolumn{3}{|c|}{User-Specified Input}                                               \\
\multicolumn{3}{|c|}{}                                                                   \\ \hline
\multicolumn{2}{|c|}{}                               &  \multicolumn{1}{c|}{}            \\
\multicolumn{2}{|l|}{\rb{Parameter}}                 &  \multicolumn{1}{c|}{\rb{Value}}  \\ \hline \hline
\multicolumn{2}{|l|}{$c\sb{s}$ (\si{kJ/(kg.K)})}     &  \multicolumn{1}{c|}{0.450}       \\ \hline
\multicolumn{2}{|l|}{$W/D$ (kg/m$^2$)}               &  \multicolumn{1}{c|}{50.1}        \\ \hline
\multicolumn{2}{|l|}{$k\sb{i}$ (\si{W/(m.K)})}       &  \multicolumn{1}{c|}{0.10}        \\ \hline
\multicolumn{2}{|l|}{$\rho\sb{i}$ (kg/m$^3$)}        &  \multicolumn{1}{c|}{208}         \\ \hline
\multicolumn{2}{|l|}{$c\sb{i}$ (\si{kJ/(kg.K)})}     &  \multicolumn{1}{c|}{2.0}         \\ \hline
\multicolumn{2}{|l|}{$h\sb{i}$ (m)}                  &  \multicolumn{1}{c|}{0.0191}      \\ \hline \hline
\multicolumn{2}{|l|}{Correlation for $T\sb{f}$ (-)}  &  \multicolumn{1}{c|}{MQH}         \\ \hline \hline
\multicolumn{2}{|l|}{$\dot Q$ (kW)}                  &  \multicolumn{1}{c|}{3200}        \\ \hline
\multicolumn{2}{|l|}{$L$ (m)}                        &  \multicolumn{1}{c|}{7.04}        \\ \hline
\multicolumn{2}{|l|}{$W$ (m)}                        &  \multicolumn{1}{c|}{3.60}        \\ \hline
\multicolumn{2}{|l|}{$H$ (m)}                        &  \multicolumn{1}{c|}{3.82}        \\ \hline
\multicolumn{2}{|l|}{$H\sb{o}$ (m)}                  &  \multicolumn{1}{c|}{2.82}        \\ \hline
\multicolumn{2}{|l|}{$W\sb{o}$ (m)}                  &  \multicolumn{1}{c|}{2.4}         \\ \hline
\multicolumn{2}{|l|}{$k$ (\si{kW/(m.K)})}            &  \multicolumn{1}{c|}{0.00012}     \\ \hline
\multicolumn{2}{|l|}{$\rho$ (kg/m$^3$)}              &  \multicolumn{1}{c|}{737}         \\ \hline
\multicolumn{2}{|l|}{$c$ (\si{kJ/(kg.K)})}           &  \multicolumn{1}{c|}{1.42}        \\ \hline
\multicolumn{2}{|l|}{$\delta$ (m)}                   &  \multicolumn{1}{c|}{0.0254}      \\ \hline
\multicolumn{2}{|l|}{$T_\infty$ ($^\circ$C)}         &  \multicolumn{1}{c|}{20}          \\ \hline
\multicolumn{2}{c}{}                                                                     \\ \hline
\multicolumn{3}{|c|}{}                                                                   \\
\multicolumn{3}{|c|}{Expected Output}                                                    \\
\multicolumn{3}{|c|}{}                                                                   \\ \hline
           &                    &                                                        \\
           &  \rb{Fire}         &  \rb{Steel}                                            \\
\rb{Time}  &  \rb{Temperature}  &  \rb{Temperature}                                      \\
\rb{(s)}   &  \rb{($^\circ$C)}  &  \rb{($^\circ$C)}                                      \\ \hline \hline
15         &  313.5             &  21.0                                                  \\ \hline
30         &  349.4             &  22.3                                                  \\ \hline
45         &  372.4             &  23.66                                                 \\ \hline
\end{tabular}
\end{center}
\end{table}


\clearpage


\subsection*{Validation}

For the unprotected and protected steel cases, a summary of the comparisons between peak predicted and measured target temperatures is shown in Fig.~\ref{Surface_Temperature_Steel_Summary}.

\begin{figure}[!ht]
\begin{center}
\begin{tabular}{l}
\includegraphics[width=4.0in]{SCRIPT_FIGURES/Scatterplots/Target_Temperature}
\end{tabular}
\end{center}
\caption[Summary of target temperature predictions]
{Summary of target temperature predictions.}
\label{Surface_Temperature_Steel_Summary}
\end{figure}



% !TEX root = FDTs_Validation_Guide.tex

\chapter{Heat Flux}
\label{Heat_Flux_Chapter}

\section{Point Source Radiation Heat Flux}

The point source model assumes that radiative energy is concentrated at a point located within a flame~\cite{Beyler2:SFPE}.
Here, the point source is located at a point one-third the height of the flame.
The radiative heat flux, $\dot q_r''$, at any distance $R$ (\si{m}) from this point can be predicted using
\be
\dot q_r'' = \cos\theta \left( \frac{\chi_r \dot Q}{4 \pi R^2} \right) \quad \si{kW/m^2}
\label{eq:point_source}
\ee
where the $\cos\theta$ term (equal to $x/R$ for targets facing sideways, or $z/R$ for gauges facing upward or downward) accounts for a target that is at an angle $\theta$ from the source. The IOR orientation parameter is used to specify which direction the target or gauge is facing: 1 or -1 for the positive or negative $x$ direction, 2 or -2 for the positive or negative $y$ direction, and 3 or -3 for the positive or negative $z$ direction. $\chi_r$ is the radiative fraction~(-), $\dot Q$ is the HRR of the fire~(\si{kW}), and $R$ is the radial distance from the point source to the edge of the target~(\si{m}) and is given by
\be
R = \sqrt{x^2 + \left(z - \frac{L_f}{3} \right)^2}
\label{eq:point_source_R}
\ee
where $x$ is the horizontal distance from the point source to the edge of the target~(\si{m}), $z$ is the height of the heat flux target~(\si{m}). The flame height, $L_f$ (\si{m}), is given by
\be
L_f = D (3.7 Q^{*^{2/5}} - 1.02)
\label{eq:point_source_Lf}
\ee
where $D$ is the diameter of the fire source~(\si{m}) and is given by
\be
D = \sqrt{\frac{4 A}{\pi}}
\label{eq:point_source_D}
\ee
where $A$ is the area of the fire source~(m$^2$). The nondimensional HRR, $Q^*$, is given by
\be
Q^* = \frac{\dot Q}{\rho_\infty c_p T_\infty \sqrt{g} D^{5/2}}
\label{eq:point_source_Qstar}
\ee
where $\rho_\infty$ is the ambient air density~(\si{kg/m^3}), $c_p$ is the specific heat of air~(\si{kJ/(kg.K)}), $T_\infty$ is the ambient air temperature~(\si{K}), and $g$ is the acceleration of gravity~(\si{m/s^2}).

\begin{figure}[!ht]
\begin{center}
\begin{tabular}{l}
\includegraphics[width=4.0in]{SCRIPT_FIGURES/Scatterplots/Target_Heat_Flux}
\end{tabular}
\end{center}
\caption[Summary of point source radiation heat flux predictions]
{Summary of point source radiation heat flux temperature predictions.}
\label{Heat_Flux_Point_Source_Summary}
\end{figure}


% !TEX root = Correlation_Guide.tex

\chapter{Ceiling Jet Temperature}
\label{Ceiling_Jet_Temperature_Chapter}

The ceiling jet is the shallow layer of hot gases below the ceiling that spreads radially from the centerline of the fire plume. The ceiling jet has a higher temperature than the overall temperature of the HGL, and therefore it is important where targets are located just below the ceiling.

\subsection*{Description}

For a steady-state fire, the correlation of Alpert~\cite{SFPE:Alpert} predicts that the ceiling jet temperature rise, $\Delta T\sb{jet}$~(\si{\celsius}), from a fire plume is given by
\be
\Delta T\sb{jet} = \left\{ \begin{array}{cl}
   \frac{16.9 \dot Q^{2/3}}{H^{5/3}}  &  r/H <= 0.18 \\[0.1in]
   \frac{5.38 (\dot Q / r)^{2/3}}{H}  &  r/H >  0.18 
   \end{array} \right.
\label{eq:Alpert_Tjet}
\ee
where $\dot Q$ is the total HRR~(\si{kW}), $H$ is the height of the ceiling above the fuel~(\si{m}), and $r$~is the radial distance to the detector~(\si{m}).

Note that some of these cases assume a quasi-steady approach for a fire source $\dot Q$ that follows a specified $t$-squared growth rate, which is given by
\be
\dot Q = \alpha t^2
\label{eq:t_squared}
\ee
where $\alpha$ is the $t$-squared growth rate parameter~(\si{kW/s^2}), and $t$ is time~(\si{s}).

For cases in which the fire was located against a wall or corner, these correlations are adjusted based on the method of reflection. For a fire adjacent to a flat wall, 2$\dot Q$ is substituted for $\dot Q$; and for a fire in a 90-degree corner, 4$\dot Q$ is substituted for $\dot Q$. This adjustment is denoted in the input parameters as the location factor. For a given case, the location factor has a value of 1, 2, or 4, which corresponds to a fire away from walls or corners, a fire adjacent to a wall, or a fire located in a corner, respectively.


\clearpage


\subsection*{Verification}

This example case is based on Test 1 from the NIST and Nuclear Regulatory Commission (NIST/NRC)~\cite{Hamins:SP1013-1} series. This test involved a compartment with a closed door, a heptane spray burner, and no ventilation.

\begin{table}[!ht]
\caption[Verification case, ceiling jet temperature]
{Verification case, ceiling jet temperature.}
\begin{center}
\begin{tabular}{|l|c|}
\hline
\multicolumn{2}{|c|}{}                              \\
\multicolumn{2}{|c|}{User-Specified Input}          \\
\multicolumn{2}{|c|}{}                              \\ \hline
                              &                     \\
\rb{Parameter}                &  \rb{Value}         \\ \hline \hline
$\dot Q$ (kW)                 &  410                \\ \hline
Location Factor               &  1                  \\ \hline
$r$ (m)                       &  5.90               \\ \hline
$H$ (m)                       &  3.72               \\ \hline
$T_{\infty}$ (\si{\celsius})  &  22                 \\ \hline
\multicolumn{2}{c}{}                                \\ \hline
\multicolumn{2}{|c|}{}                              \\
\multicolumn{2}{|c|}{Calculated Output}             \\
\multicolumn{2}{|c|}{}                              \\ \hline
\multicolumn{2}{|c|}{}                              \\
\multicolumn{2}{|c|}{\rb{Ceiling Jet Temperature}}  \\
\multicolumn{2}{|c|}{\rb{(\si{\celsius})}}          \\ \hline \hline
\multicolumn{2}{|c|}{46.45}                         \\ \hline
\end{tabular}
\end{center}
\end{table}


\clearpage


\subsection*{Validation}

A summary of the comparisons between peak predicted and measured ceiling jet temperatures is shown in Fig.~\ref{Ceiling Jet Temperature, Unconfined (Alpert)}.

It is important to note that this ceiling jet temperature correlation was developed using data from tests that were conducted in a large facility in which the distant walls and large compartment size did not allow for the development of a significant hot gas layer. In a more typical fire scenario (i.e., a smaller compartment), the HGL develops relatively quickly, and temperatures at the ceiling are affected by the ceiling jet as well as the accumulating HGL. Thus, when compared to experimentally measured ceiling temperatures in a compartment fire, this correlation tends to underpredict the temperatures because it is not accounting for the development of the HGL. This is an important consideration when using this correlation to predict detector or sprinkler activations.

For the reasons stated above, two scatter plot comparisons are shown in Fig.~\ref{Ceiling Jet Temperature, Unconfined (Alpert)}. One scatter plot shows the results for unconfined tests that were conducted under a false ceiling in which the hot plume gases did not accumulate to form an HGL, but were allowed to spill out from under a false ceiling. The other scatter plot shows the results of underpredicted temperature comparisons for compartment fire tests. The use of the ceiling jet correlation in a confined compartment with the presence of an HGL can result in an underprediction of the measured ceiling jet temperature by approximately 70~\%. Therefore, the model bias factor and model relative standard deviation were only calculated for the unconfined ceiling jet cases that the correlation was developed for. In the unconfined ceiling cases, the ceiling temperature predictions are in better agreement with experimental data because this scenario is more representative of a temperature rise due to only ceiling jet flow from the fire plume. 

\begin{figure}[!ht]
\begin{center}
\begin{tabular}{l}
\includegraphics[width=3.9in]{SCRIPT_FIGURES/Scatterplots/Ceiling_Jet_Temperature_Unconfined} \\
\includegraphics[width=3.9in]{SCRIPT_FIGURES/Scatterplots/Ceiling_Jet_Temperature_Compartment}
\end{tabular}
\end{center}
\caption[Summary of ceiling jet temperature predictions]
{Summary of compartment (top) and unconfined (bottom) ceiling jet temperature predictions.}
\label{Ceiling Jet Temperature, Unconfined (Alpert)}
\end{figure}


% !TEX root = Correlation_Guide.tex

\chapter{Sprinkler Activation Time}
\label{Sprinkler_Activation_Time_Chapter}

Much like an electrical cable, a sprinkler is merely a ``target'' with a particular set of thermal properties, such as the response time index (RTI) that indicates the sensitivity of the fusible link or glass bulb. Activation is assumed to occur when the link or bulb temperatures reaches a predetermined threshold temperature.

\subsection*{Description}

For a steady-state fire, the correlation of Alpert~\cite{SFPE:Alpert} predicts that the activation time of a sprinkler, $t\sb{act}$~(\si{s}), is given by~\cite{NFPA}
\be
t\sb{act} =  \frac{\mathrm{RTI}}{\sqrt{u\sb{jet}}} \ln \left( \frac{T\sb{jet} - T_\infty}{T\sb{jet} - T\sb{act}} \right)
\label{eq:Alpert}
\ee
where RTI is the response time index of the sprinkler~(\si{(m.s)^{1/2}}), $T_\infty$ is the ambient air temperature~(\si{\celsius}), and $T\sb{act}$ is the activation temperature of the sprinkler~(\si{\celsius}). The ceiling jet temperature, $T\sb{jet}$ (\si{\celsius}), is given by
\be
T\sb{jet} = \left\{ \begin{array}{cl}
   \frac{16.9 \dot Q^{2/3}}{H^{5/3}} + T_\infty  &  r/H <= 0.18 \\[0.1in]
   \frac{5.38 (\dot Q / r)^{2/3}}{H} + T_\infty  &  r/H >  0.18
   \end{array} \right.
\label{eq:sprinkler_Tjet}
\ee
where $\dot Q$ is the total HRR~(kW), $H$ is the height of the ceiling above the fuel~(m), and $r$~is the radial distance to the detector~(m).
The ceiling jet velocity, $u\sb{jet}$ (\si{m/s}), is given by
\be
u\sb{jet} = \left\{ \begin{array}{cl}
   0.947 \left( \frac{\dot Q}{H} \right)^{1/3}  &  r/H <= 0.15 \\[0.1in]
   \frac{0.197 \dot Q^{1/3} H^{1/2}}{r^{5/6}}   &  r/H >  0.15
   \end{array} \right.
\label{eq:sprinkler_ujet}
\ee

Note that some of these cases assume a quasi-steady approach for a fire source $\dot Q$ that follows a specified $t$-squared growth rate, which is given by Eq.~\ref{eq:t_squared}.

For cases in which the fire was located against a wall or corner, these correlations are adjusted based on the method of reflection. For a fire adjacent to a flat wall, 2$\dot Q$ is substituted for $\dot Q$; and for a fire in a 90-degree corner, 4$\dot Q$ is substituted for $\dot Q$~\cite{SFPE:Alpert}. This adjustment is denoted in the input parameters as the location factor. For a given case, the location factor has a value of 1, 2, or 4, which corresponds to a fire away from walls or corners, a fire adjacent to a wall, or a fire located in a corner, respectively.


\clearpage


\subsection*{Verification}

This example case is based on Test 1 from the Vettori Flat Ceiling~\cite{Vettori:1} series. This test involved residential quick response sprinklers located on a flat ceiling in a compartment with a closed door, a methane burner, and no ventilation.

\begin{table}[!ht]
\caption[Verification case, sprinkler activation time]
{Verification case, sprinkler activation time.}
\begin{center}
\begin{tabular}{|c|c|c|}
\hline
\multicolumn{3}{|c|}{}                                                                 \\
\multicolumn{3}{|c|}{User-Specified Input}                                             \\
\multicolumn{3}{|c|}{}                                                                 \\ \hline
\multicolumn{2}{|c|}{}                             &  \multicolumn{1}{c|}{}            \\
\multicolumn{2}{|l|}{\rb{Parameter}}               &  \multicolumn{1}{c|}{\rb{Value}}  \\ \hline \hline
\multicolumn{2}{|l|}{$\alpha$ (kW/s$^2$)}          &  \multicolumn{1}{c|}{0.105}       \\ \hline
\multicolumn{2}{|l|}{Location Factor}              &  \multicolumn{1}{c|}{1}           \\ \hline
\multicolumn{2}{|l|}{RTI (\si{(m.s)^{1/2}})}       &  \multicolumn{1}{c|}{55}          \\ \hline
\multicolumn{2}{|l|}{$T\sb{act}$ (\si{\celsius})}  &  \multicolumn{1}{c|}{68}          \\ \hline
\multicolumn{2}{|l|}{$r$ (m)}                      &  \multicolumn{1}{c|}{2.20}        \\ \hline
\multicolumn{2}{|l|}{$H$ (m)}                      &  \multicolumn{1}{c|}{2.09}        \\ \hline
\multicolumn{2}{|l|}{$T_\infty$ (\si{\celsius})}   &  \multicolumn{1}{c|}{16.6}        \\ \hline
\multicolumn{2}{c}{}                                                                   \\ \hline
\multicolumn{3}{|c|}{}                                                                 \\
\multicolumn{3}{|c|}{Calculated Output}                                                \\
\multicolumn{3}{|c|}{}                                                                 \\ \hline
           &             &                                                             \\
\rb{Time}  &  \rb{HRR}   &  \rb{Activation Time}                                       \\
\rb{(s)}   &  \rb{(kW)}  &  \rb{(s)}                                                   \\ \hline \hline
50         &  262.5      &  98.2                                                       \\ \hline
\end{tabular}
\end{center}
\end{table}

% Note that the FDTs spreadsheet uses incorrect coefficients in Alpert's ceiling jet correlation, which gives an activation time of 98.74 s.


\clearpage


\subsection*{Validation}

A summary of the comparisons between predicted and measured sprinkler activation times is shown in Fig.~\ref{Sprinkler Activation Time}.

\begin{figure}[!ht]
\begin{center}
\begin{tabular}{l}
\includegraphics[width=4.0in]{SCRIPT_FIGURES/Scatterplots/Sprinkler_Activation_Time}
\end{tabular}
\end{center}
\caption[Summary of sprinkler activation time predictions]
{Summary of sprinkler activation time predictions.}
\label{Sprinkler Activation Time}
\end{figure}



% !TEX root = Correlation_Guide.tex

\chapter{Smoke Detector Activation Time}
\label{Smoke_Detector_Activation_Time_Chapter}

\section{Temperature Rise Method}

\subsection*{Description}

In this method, the prediction of smoke detector activation time is identical to that for a sprinkler (as described in Chapter~\ref{Sprinkler_Activation_Time_Chapter}). Heskestad and Delichatsios~\cite{Heskestad:4} correlated smoke detector activation to a smoke temperature change of 10~$^\circ$C (18~$^\circ$F) from typical fuels. It is assumed that the smoke detectors are low-RTI devices ($\textrm{RTI}=5$).

Note that some of these cases assume a quasi-steady approach for a fire source $\dot Q$ that follows a specified t-squared growth, which is given by Eq.~\ref{eq:t_squared}.

\subsection*{Verification}

Test: NIST Smoke Alarms, Test SDC02

\begin{table}[!ht]
\caption[Verification input parameters, smoke detector activation time]
{Verification input parameters, smoke detector activation time.}
\begin{center}
\begin{tabular}{|l|c|}
\hline
                          &              \\
\rb{Input Parameter}      &  \rb{Value}  \\ \hline \hline
Location Factor (-)       &  1           \\ \hline
$\alpha$ (kW/s$^2$)       &  0.00463     \\ \hline
$t_{fire}$ (s)            &  300         \\ \hline
$r$ (m)                   &  1.3         \\ \hline
$H$ (m)                   &  2.1         \\ \hline
$\Delta T_c$ ($^\circ$C)  &  10          \\ \hline
RTI (m-s)$^{1/2}$         &  5           \\ \hline
$T_\infty$ ($^\circ$C)    &  21          \\ \hline
\end{tabular}
\end{center}
\end{table}

\noindent Expected result: At 47~s, the HRR $\dot Q$ is 10.23~kW, and the activation time $t_{activation}$ is 30.8~s.


\clearpage


\subsection*{Validation}

\begin{figure}[!ht]
\begin{center}
\begin{tabular}{l}
\includegraphics[width=4.0in]{SCRIPT_FIGURES/Scatterplots/Smoke_Detector_Activation_Time_Temperature_Rise}
\end{tabular}
\end{center}
\caption[Summary of smoke detector activation time predictions]
{Summary of smoke detector activation time predictions using the Temperature Rise method.}
\label{Smoke_Detector_Activation_Summary_Temperature_Rise}
\end{figure}


\clearpage


\section{Milke Method}

\subsection*{Description}

The correlation of Milke~\cite{Milke:1} predicts that the time of smoke detector activation, $t_{activation}$, is given by
\be
t_{activation} = \frac{X H^{4/3}}{\dot Q^{1/3}}
\label{eq:Milke}
\ee
where $H$ is the height of the ceiling above the top of the fuel~(\si{ft}), and $\dot Q$ is the HRR~(\si{kW}). The constant $X$ is given by
\be
X = 4.6 \times 10^{-4} (Y^2) + 2.7 \times 10^{-15} (Y^6)
\label{eq:Milke_X}
\ee
and the constant $Y$ is given by
\be
Y = \frac{\Delta T_c H^{5/3}}{\dot Q^{2/3}}
\label{eq:Milke_Y}
\ee
where $\Delta T_c$ is the temperature rise of gases under the ceiling required for the smoke detector to activate~($^\circ$F).

Note that some of these cases assume a quasi-steady approach for a fire source $\dot Q$ that follows a specified t-squared growth, which is given by Eq.~\ref{eq:t_squared}.

\subsection*{Verification}

Test: NIST Smoke Alarms, Test SDC02

\begin{table}[!ht]
\caption[Verification input parameters, smoke detector activation time]
{Verification input parameters, smoke detector activation time.}
\begin{center}
\begin{tabular}{|l|c|}
\hline
                          &              \\
\rb{Input Parameter}      &  \rb{Value}  \\ \hline \hline
$\alpha$ (kW/s$^2$)       &  0.00463     \\ \hline
$t_{fire}$ (s)            &  300         \\ \hline
$H$ (m)                   &  2.1         \\ \hline
$\Delta T_c$ ($^\circ$C)  &  10          \\ \hline
\end{tabular}
\end{center}
\end{table}

\noindent Expected result: At 40~s, the HRR $\dot Q$ is 7.41~kW, and the activation time $t_{activation}$ is 47.3~s.
\\ \\
Note: The FDTs spreadsheets use an incorrect unit conversion factor from kW to Btu/s, which gives an activation time of 39.5~s.


\clearpage


\subsection*{Validation}

\begin{figure}[!ht]
\begin{center}
\begin{tabular}{l}
\includegraphics[width=4.0in]{SCRIPT_FIGURES/Scatterplots/Smoke_Detector_Activation_Time_Milke}
\end{tabular}
\end{center}
\caption[Summary of smoke detector activation time predictions]
{Summary of smoke detector activation time predictions using the method of Milke.}
\label{Smoke_Detector_Activation_Summary_Milke}
\end{figure}


\clearpage


\section{Mowrer Method}

\subsection*{Description}

The correlation of Mowrer~\cite{Mowrer:1} predicts that the time of smoke detector activation, $t_{activation}$, is given by
\be
t_{activation} = t_{pl} + t_{cj}
\label{eq:Mowrer}
\ee
where the transport lag time of the plume, $t_{pl}$, is given by
\be
t_{pl} = C_{pl} \frac{H^{4/3}}{\dot Q^{1/3}}
\label{eq:Mowrer_tpl}
\ee
where $C_{pl}$ is the plume lag time constant~(0.67), $H$ is the height of the ceiling above the fuel~(\si{m}), and $\dot Q$ is the HRR~(\si{kW}).
The transport lag time of the ceiling jet, $t_{cj}$, is given by
\be
t_{cj} = \frac{1}{C_{cj}} \frac{r^{11/6}}{\dot Q^{1/3} H^{1/2}}
\label{eq:Mowrer_tcj}
\ee
where $C_{cj}$ is the ceiling jet time lag time constant~(1.2), and $r$ is the radial distance to the detector~(\si{m}).

Note that some of these cases assume a quasi-steady approach for a fire source $\dot Q$ that follows a specified t-squared growth, which is given by Eq.~\ref{eq:t_squared}.

\subsection*{Verification}

Test: NIST Smoke Alarms, Test SDC02

\begin{table}[!ht]
\caption[Verification input parameters, smoke detector activation time]
{Verification input parameters, smoke detector activation time.}
\begin{center}
\begin{tabular}{|l|c|}
\hline
                      &              \\
\rb{Input Parameter}  &  \rb{Value}  \\ \hline \hline
$\alpha$ (kW/s$^2$)   &  0.00463     \\ \hline
$t_{fire}$ (s)        &  300         \\ \hline
$C_{pl}$ (-)          &  0.67        \\ \hline
$C_{cj}$ (-)          &  1.2         \\ \hline
$r$ (m)               &  1.3         \\ \hline
$H$ (m)               &  2.1         \\ \hline
\end{tabular}
\end{center}
\end{table}

\noindent Expected result: At 5~s, the HRR $\dot Q$ is 0.116~kW, and the activation time $t_{activation}$ is 5.6~s.


\clearpage


\subsection*{Validation}

\begin{figure}[!ht]
\begin{center}
\begin{tabular}{l}
\includegraphics[width=4.0in]{SCRIPT_FIGURES/Scatterplots/Smoke_Detector_Activation_Time_Mowrer}
\end{tabular}
\end{center}
\caption[Summary of smoke detector activation time predictions]
{Summary of smoke detector activation time predictions using the method of Mowrer (bottom).}
\label{Smoke_Detector_Activation_Summary_Mowrer}
\end{figure}

% !TEX root = Correlation_Guide.tex

\chapter{Summary}
\label{Summary_Chapter}

\section*{Summary of Uncertainty Statistics for Empirical Correlations}

For each quantity of interest, the experimental relative standard deviation, $\widetilde{\sigma}\sb{E}$, model relative standard deviation, $\widetilde{\sigma}\sb{M}$, and model bias factor are shown in Table~\ref{summary_stats}. The latter two values indicate the average scatter and bias of the model predictions. For example, for a given quantity, a model relative standard deviation of 0.15 indicates that one standard deviation of all of the model predictions is equal to 15~\%, and a model bias factor of 1.05 indicates that, on average, the model tends to overpredict that quantity by 5~\%. The number of datasets and data points are also listed for each quantity.

\IfFileExists{SCRIPT_FIGURES/Scatterplots/validation_statistics.tex}{\input{SCRIPT_FIGURES/Scatterplots/validation_statistics.tex}}{\typeout{Error: Missing file SCRIPT_FIGURES/Scatterplots/validation_statistics.tex}}


\bibliography{../Bibliography/FDS_refs,../Bibliography/FDS_general,../Bibliography/FDS_mathcomp}

\appendix

% !TEX root = Correlation_Guide.tex

\chapter{Validation Input Parameters}
\label{Inputs_Chapter}

This appendix lists all of the input parameters that were used for each test in each data set. The input parameters are arranged alphabetically by data set, then by experimental quantity.

\section{ATF Corridors}

\subsection*{Ceiling Jet Temperature (Alpert)~\cite{SFPE:Alpert}}

\begin{table}[!ht]
\caption[Validation input parameters for ATF Corridors cases, ceiling jet temperature]
{Summary of validation input parameters used for ATF Corridors cases~\cite{Sheppard:Corridors}, ceiling jet temperature.}

\begin{center}
\begin{tabular}{|l|l|}
\hline
                              &              \\
\rb{Input Parameter}          &  \rb{Value}  \\ \hline \hline
Location Factor               &  1           \\ \hline
$r$ (m)                       &  2, 9, 14.5  \\ \hline
$H$ (m)                       &  2.03        \\ \hline
$T_{\infty}$ (\si{\celsius})  &  20          \\ \hline
\end{tabular}
\end{center}

\begin{center}
\begin{tabular}{|l|c|}
\hline
                      &                 \\
\rb{Test}             &  \rb{$\dot Q$}  \\
                      &  \rb{(kW)}      \\ \hline \hline
ATF Corridors 50 kW   &  48             \\ \hline
ATF Corridors 100 kW  &  97             \\ \hline
ATF Corridors 240 kW  &  242            \\ \hline
ATF Corridors 250 kW  &  250            \\ \hline
ATF Corridors 500 kW  &  485            \\ \hline
\end{tabular}
\end{center}
\end{table}


\clearpage


\section{CAROLFIRE}

\subsection*{Cable Failure Time~\cite{CAROLFIRE}}

The cable and conduit targets were not completely surrounded by the Penlight apparatus, which had openings at its ends. Therefore, the exposing temperature ramp $T\sb{ramp}$ was calculated using a view factor of 0.90 and the experimentally measured shroud temperature.

\begin{table}[!ht]
\caption[Validation input parameters for CAROLFIRE cases, cable failure time]
{Summary of validation input parameters used for CAROLFIRE cases~\cite{CAROLFIRE}, cable failure time.}

\begin{center}
\begin{tabular}{|l|c|c|c|c|c|c|c|}
\hline
                  &  Cable     &  Mass per     &  Jacket     &  Conduit   &  Conduit    &                   &               \\
Test              &  Diameter  &  Unit Length  &  Thickness  &  Diameter  &  Thickness  &  $T_\infty$       &  $t\sb{end}$  \\
                  &  (mm)      &  (kg/m)       &  (mm)       &  (mm)      &  (mm)       &  (\si{\celsius})  &  (s)          \\ \hline \hline
Penlight Test 1   &  16.3      &  0.529        &  1.5        &  -         &  -          &  24               &  1800         \\ \hline
Penlight Test 2   &  16.3      &  0.529        &  1.5        &  -         &  -          &  24               &  1800         \\ \hline
Penlight Test 3   &  16.3      &  0.529        &  1.5        &  -         &  -          &  24               &  1800         \\ \hline
Penlight Test 4   &  15.2      &  0.459        &  1.5        &  -         &  -          &  20               &  1800         \\ \hline
Penlight Test 5   &  15.2      &  0.459        &  1.5        &  -         &  -          &  20               &  1800         \\ \hline
Penlight Test 6   &  15.2      &  0.459        &  1.5        &  -         &  -          &  20               &  1800         \\ \hline
Penlight Test 7   &  16.3      &  0.529        &  1.5        &  50        &  4.9        &  24               &  1800         \\ \hline
Penlight Test 8   &  15.2      &  0.459        &  1.5        &  50        &  4.9        &  30               &  1800         \\ \hline
Penlight Test 9   &  16.3      &  0.529        &  1.5        &  -         &  -          &  24               &  1800         \\ \hline
Penlight Test 10  &  15.2      &  0.459        &  1.5        &  -         &  -          &  20               &  1800         \\ \hline
Penlight Test 11  &  15.0      &  0.410        &  1.5        &  -         &  -          &  24               &  1800         \\ \hline
Penlight Test 12  &  15.0      &  0.410        &  1.5        &  -         &  -          &  24               &  1800         \\ \hline
Penlight Test 13  &  15.0      &  0.410        &  1.5        &  -         &  -          &  24               &  1800         \\ \hline
Penlight Test 14  &  15.0      &  0.380        &  1.1        &  -         &  -          &  24               &  1800         \\ \hline
Penlight Test 15  &  15.0      &  0.380        &  1.1        &  -         &  -          &  24               &  1800         \\ \hline
Penlight Test 16  &  15.0      &  0.380        &  1.1        &  -         &  -          &  24               &  1800         \\ \hline
Penlight Test 17  &  15.1      &  0.400        &  1.5        &  -         &  -          &  24               &  1800         \\ \hline
Penlight Test 18  &  14.5      &  0.358        &  1.0        &  -         &  -          &  24               &  1800         \\ \hline
Penlight Test 19  &  12.2      &  0.321        &  0.9        &  -         &  -          &  24               &  1800         \\ \hline
Penlight Test 20  &  15.1      &  0.388        &  1.5        &  -         &  -          &  24               &  1800         \\ \hline
Penlight Test 21  &  12.4      &  0.324        &  1.1        &  -         &  -          &  24               &  1190         \\ \hline
Penlight Test 22  &  10.2      &  0.292        &  0.5        &  -         &  -          &  24               &  1640         \\ \hline
Penlight Test 23  &  15.0      &  0.410        &  1.5        &  50        &  4.9        &  24               &  1800         \\ \hline
Penlight Test 24  &  15.0      &  0.410        &  1.5        &  50        &  4.9        &  24               &  1800         \\ \hline
Penlight Test 25  &  15.0      &  0.380        &  1.1        &  50        &  4.9        &  24               &  1800         \\ \hline
Penlight Test 26  &  15.0      &  0.380        &  1.1        &  50        &  4.9        &  24               &  1800         \\ \hline
Penlight Test 27  &  15.0      &  0.410        &  1.5        &  -         &  -          &  24               &  1800         \\ \hline
Penlight Test 28  &  15.0      &  0.410        &  1.5        &  -         &  -          &  24               &  1800         \\ \hline
Penlight Test 29  &  15.0      &  0.380        &  1.1        &  -         &  -          &  24               &  1640         \\ \hline
Penlight Test 30  &  15.0      &  0.380        &  1.1        &  -         &  -          &  24               &  1800         \\ \hline
Penlight Test 31  &  19.0      &  0.500        &  2.0        &  -         &  -          &  24               &  1800         \\ \hline
Penlight Test 62  &  12.7      &  0.231        &  1.1        &  -         &  -          &  24               &  1190         \\ \hline
Penlight Test 63  &  11.3      &  0.195        &  1.1        &  -         &  -          &  24               &  710          \\ \hline
Penlight Test 64  &   7.9      &  0.097        &  1.1        &  -         &  -          &  24               &  890          \\ \hline
Penlight Test 65  &   7.0      &  0.076        &  1.0        &  -         &  -          &  24               &  770          \\ \hline
\end{tabular}
\end{center}
\end{table}


\clearpage


\begin{table}[!ht]
\caption[Validation input parameters for CAROLFIRE cases, cable failure time (continued)]
{Summary of validation input parameters used for CAROLFIRE cases~\cite{CAROLFIRE}, cable failure time (continued).}

\begin{center}
\begin{tabular}{|l|l|l|}
\hline
                  &                                &                             \\
\rb{Test}         &  \rb{$t\sb{ramp}$}             &  \rb{$T\sb{ramp}$}          \\
                  &  \rb{(s)}                      &  \rb{(\si{\celsius})}       \\ \hline \hline
Penlight Test 1   &  0, 70, 820, 1240, 1600, 1800  &  24, 460, 460, 275, 178, 0  \\ \hline
Penlight Test 2   &  0, 80, 800, 1220, 1600, 1800  &  24, 460, 460, 275, 178, 0  \\ \hline
Penlight Test 3   &  0, 80, 700,  950, 1600, 1800  &  24, 460, 460, 334, 178, 0  \\ \hline
Penlight Test 4   &  0, 60, 820, 1470, 1800, 1900  &  24, 290, 290, 290, 188, 0  \\ \hline
Penlight Test 5   &  0, 60, 820, 1080, 1600, 1800  &  24, 290, 290, 290, 139, 0  \\ \hline
Penlight Test 6   &  0, 60, 820, 1080, 1600, 1800  &  24, 290, 290, 290, 139, 0  \\ \hline
Penlight Test 7   &  0, 80, 820, 1240, 1800, 1900  &  24, 460, 460, 460, 460, 0  \\ \hline
Penlight Test 8   &  0, 55, 820, 1080, 1800, 1900  &  24, 290, 290, 290, 290, 0  \\ \hline
Penlight Test 9   &  0, 80, 820, 1240, 1800, 1900  &  24, 451, 451, 451, 451, 0  \\ \hline
Penlight Test 10  &  0, 60, 820, 1500, 1800, 1900  &  24, 285, 285, 285, 188, 0  \\ \hline
Penlight Test 11  &  0, 80, 820, 1320, 1800, 1900  &  24, 456, 456, 456, 266, 0  \\ \hline
Penlight Test 12  &  0, 80, 820, 1320, 1800, 1900  &  24, 460, 460, 460, 266, 0  \\ \hline
Penlight Test 13  &  0, 80, 820, 1320, 1800, 1900  &  24, 460, 460, 460, 266, 0  \\ \hline
Penlight Test 14  &  0, 60, 820, 1470, 1800, 1900  &  24, 290, 290, 290, 290, 0  \\ \hline
Penlight Test 15  &  0, 60, 820, 1470, 1800, 1900  &  24, 314, 314, 314, 188, 0  \\ \hline
Penlight Test 16  &  0, 60, 820, 1470, 1800, 1900  &  24, 314, 314, 314, 314, 0  \\ \hline
Penlight Test 17  &  0, 80, 660, 1240, 1600, 1800  &  24, 460, 460, 275, 178, 0  \\ \hline
Penlight Test 18  &  0, 70, 820, 1240, 1600, 1800  &  24, 675, 675, 675, 675, 0  \\ \hline
Penlight Test 19  &  0, 80, 820, 1050, 1500, 1800  &  24, 460, 460, 460, 236, 0  \\ \hline
Penlight Test 20  &  0, 70, 600, 1240, 1600, 1800  &  24, 460, 460, 275, 178, 0  \\ \hline
Penlight Test 21  &  0, 60, 930, 1200              &  24, 290, 290, 197          \\ \hline
Penlight Test 22  &  0, 80, 1440, 1650             &  24, 460, 460, 295          \\ \hline
Penlight Test 23  &  0, 80, 1801                   &  24, 456, 456               \\ \hline
Penlight Test 24  &  0, 80, 1801                   &  24, 456, 456               \\ \hline
Penlight Test 25  &  0, 50, 1801                   &  24, 309, 309               \\ \hline
Penlight Test 26  &  0, 60, 1801                   &  24, 309, 309               \\ \hline
Penlight Test 27  &  0, 80, 1500, 1801             &  24, 456, 456, 305          \\ \hline
Penlight Test 28  &  0, 80, 1801                   &  24, 451, 451               \\ \hline
Penlight Test 29  &  0, 60, 1260, 1650             &  24, 309, 309, 188          \\ \hline
Penlight Test 30  &  0, 50, 1350, 1800             &  24, 309, 309, 188          \\ \hline
Penlight Test 31  &  0, 70, 820, 1240, 1600, 1801  &  24, 675, 675, 675, 675, 0  \\ \hline
Penlight Test 62  &  0, 80, 300, 480, 1200         &  24, 451, 451, 334, 207     \\ \hline
Penlight Test 63  &  0, 70, 720                    &  24, 309, 309               \\ \hline
Penlight Test 64  &  0, 80, 240, 900               &  24, 451, 451, 207          \\ \hline
Penlight Test 65  &  0, 65, 540, 780               &  24, 309, 309, 217          \\ \hline
\end{tabular}
\end{center}
\end{table}


\clearpage


\section{Fleury Heat Flux}

\subsection*{Point Source and Solid Flame Radiation Heat Flux~\cite{Beyler2:SFPE}}

\begin{table}[!ht]
\caption[Validation input parameters for Fleury Heat Flux cases, radiation heat flux]
{Summary of validation input parameters used for Fleury Heat Flux cases~\cite{Fleury:Masters}, radiation heat flux.}

\begin{center}
\begin{tabular}{|l|l|}
\hline
                      &                            \\
\rb{Input Parameter}  &  \rb{Value}                \\ \hline \hline
$\chi\sb{r}$          &  0.32                      \\ \hline
$x$ (m)               &  0.5, 0.75, 1.0, 1.5, 2.0  \\ \hline
$z$ (m)               &  0.0, 0.5, 1.0, 1.5        \\ \hline
IOR                   &  2                         \\ \hline
\end{tabular}
\end{center}

\begin{center}
\begin{tabular}{|l|c|c|}
\hline
                    &                 &                \\
\rb{Test}           &  \rb{$\dot Q$}  &  \rb{$A$}      \\
                    &  \rb{(kW)}      &  \rb{(m$^2$)}  \\ \hline \hline
100 kW, 1:1 Burner  &  100            &  0.09          \\ \hline
150 kW, 1:1 Burner  &  150            &  0.09          \\ \hline
200 kW, 1:1 Burner  &  200            &  0.09          \\ \hline
250 kW, 1:1 Burner  &  250            &  0.09          \\ \hline
300 kW, 1:1 Burner  &  300            &  0.09          \\ \hline
100 kW, 2:1 Burner  &  100            &  0.18          \\ \hline
150 kW, 2:1 Burner  &  150            &  0.18          \\ \hline
200 kW, 2:1 Burner  &  200            &  0.18          \\ \hline
250 kW, 2:1 Burner  &  250            &  0.18          \\ \hline
300 kW, 2:1 Burner  &  300            &  0.18          \\ \hline
100 kW, 3:1 Burner  &  100            &  0.27          \\ \hline
150 kW, 3:1 Burner  &  150            &  0.27          \\ \hline
200 kW, 3:1 Burner  &  200            &  0.27          \\ \hline
250 kW, 3:1 Burner  &  250            &  0.27          \\ \hline
300 kW, 3:1 Burner  &  300            &  0.27          \\ \hline
\end{tabular}
\end{center}
\end{table}


\clearpage


\section{FM/SNL}

\subsection*{HGL Temperature~\cite{SFPE:Walton}}

\begin{table}[!ht]
\caption[Validation input parameters for FM/SNL cases, HGL temperature]
{Summary of validation input parameters used for FM/SNL cases~\cite{Nowlen:NUREG4681, Nowlen:NUREG4527}, HGL temperature.}

\begin{center}
\begin{tabular}{|l|c|}
\hline
                            &              \\
\rb{Input Parameter}        &  \rb{Value}  \\ \hline \hline
$L$ (m)                     &  18.3        \\ \hline
$W$ (m)                     &  12.2        \\ \hline
$H$ (m)                     &  6.1         \\ \hline
$c\sb{p}$ (\si{kJ/(kg.K)})  &  1.0         \\ \hline
$k$ (\si{kW/(m.K)})         &  0.00023     \\ \hline
$\rho$ (kg/m$^3$)           &  1000        \\ \hline
$c$ (\si{kJ/(kg.K)})        &  1.16        \\ \hline
$\delta$ (m)                &  0.025       \\ \hline
\end{tabular}
\end{center}

\begin{center}
\begin{tabular}{|l|l|c|c|c|c|}
\hline
           &                    &                 &                 &                        &                    \\
\rb{Test}  &  \rb{Correlation}  &  \rb{$\dot Q$}  &  \rb{$\dot m$}  &  \rb{$T_\infty$}       &  \rb{$t\sb{end}$}  \\
           &                    &  \rb{(kW)}      &  \rb{(kg/s)}    &  \rb{(\si{\celsius})}  &  \rb{(s)}          \\ \hline \hline
Test 1     &  FPA, DB           &  516            &  4.5            &  15                    &  600               \\ \hline
Test 2     &  FPA, DB           &  516            &  4.5            &  14                    &  600               \\ \hline
Test 3     &  FPA, DB           &  2000           &  4.5            &  15                    &  300               \\ \hline
Test 4     &  FPA, DB           &  516            &  0.45           &  15                    &  600               \\ \hline
Test 5     &  FPA, DB           &  516            &  4.5            &  19                    &  600               \\ \hline
Test 6     &  FPA, DB           &  500            &  0.45           &  15                    &  600               \\ \hline
Test 7     &  FPA, DB           &  516            &  0.45           &  15                    &  600               \\ \hline
Test 8     &  FPA, DB           &  1000           &  0.45           &  21                    &  720               \\ \hline
Test 9     &  FPA, DB           &  1000           &  3.6            &  24                    &  840               \\ \hline
Test 10    &  FPA, DB           &  1000           &  2.0            &  18                    &  600               \\ \hline
Test 11    &  FPA, DB           &  500            &  2.0            &  16                    &  600               \\ \hline
Test 12    &  FPA, DB           &  2000           &  2.0            &  17                    &  600               \\ \hline
Test 13    &  FPA, DB           &  2000           &  3.6            &  21                    &  600               \\ \hline
Test 14    &  FPA, DB           &  500            &  0.45           &  15                    &  600               \\ \hline
Test 15    &  FPA, DB           &  1000           &  0.45           &  18                    &  1380              \\ \hline
Test 16    &  FPA, DB           &  500            &  0.45           &  18                    &  720               \\ \hline
Test 17    &  FPA, DB           &  500            &  4.5            &  10                    &  1380              \\ \hline
Test 21    &  FPA, DB           &  500            &  0.45           &  14                    &  1200              \\ \hline
Test 22    &  FPA, DB           &  1000           &  0.45           &  14                    &  840               \\ \hline
\end{tabular}
\end{center}
\end{table}


\clearpage


\subsection*{Plume Temperature (Heskestad and McCaffrey)~\cite{SFPE:Heskestad, McCaffrey:NBSIR_79-1910}}

\begin{table}[!ht]
\caption[Validation input parameters for FM/SNL cases, plume temperature]
{Summary of validation input parameters used for FM/SNL cases~\cite{Nowlen:NUREG4681, Nowlen:NUREG4527}, plume temperature. Note that the radiative fraction for all tests was 0.40, except for Tests~11 and 14, in which case it was 0.2.}

\begin{center}
\begin{tabular}{|l|c|}
\hline
                            &              \\
\rb{Input Parameter}        &  \rb{Value}  \\ \hline \hline
$z$ (m)                     &  5.98        \\ \hline
$A$ (m$^2$)                 &  0.64        \\ \hline
$\chi\sb{r}$                &  0.40        \\ \hline
$c\sb{p}$ (\si{kJ/(kg.K)})  &  1.0         \\ \hline
\end{tabular}
\end{center}

\begin{center}
\begin{tabular}{|l|c|c|}
\hline
           &                 &                        \\
\rb{Test}  &  \rb{$\dot Q$}  &  \rb{$T_\infty$}       \\
           &  \rb{(kW)}      &  \rb{(\si{\celsius})}  \\ \hline \hline
Test 1     &  516            &  15                    \\ \hline
Test 2     &  516            &  14                    \\ \hline
Test 3     &  2000           &  15                    \\ \hline
Test 4     &  516            &  15                    \\ \hline
Test 5     &  516            &  19                    \\ \hline
Test 6     &  500            &  15                    \\ \hline
Test 7     &  516            &  15                    \\ \hline
Test 8     &  1000           &  21                    \\ \hline
Test 9     &  1000           &  24                    \\ \hline
Test 10    &  1000           &  18                    \\ \hline
Test 11    &  500            &  16                    \\ \hline
Test 12    &  2000           &  17                    \\ \hline
Test 13    &  2000           &  21                    \\ \hline
Test 14    &  500            &  15                    \\ \hline
Test 15    &  1000           &  18                    \\ \hline
Test 16    &  500            &  18                    \\ \hline
Test 17    &  500            &  10                    \\ \hline
Test 21    &  500            &  14                    \\ \hline
Test 22    &  1000           &  14                    \\ \hline
\end{tabular}
\end{center}
\end{table}


\clearpage


\subsection*{Ceiling Jet Temperature (Alpert)~\cite{SFPE:Alpert}}

\begin{table}[!ht]
\caption[Validation input parameters for FM/SNL cases, ceiling jet temperature]
{Summary of validation input parameters used for FM/SNL cases~\cite{Nowlen:NUREG4681, Nowlen:NUREG4527}, ceiling jet temperature.}

\begin{center}
\begin{tabular}{|l|l|}
\hline
                      &              \\
\rb{Input Parameter}  &  \rb{Value}  \\ \hline \hline
$r$ (m)               &  3.05, 9.15  \\ \hline
$H$ (m)               &  5.9         \\ \hline
\end{tabular}
\end{center}

\begin{center}
\begin{tabular}{|l|c|c|c|}
\hline
           &                 &                 &                        \\
\rb{Test}  &  \rb{$\dot Q$}  &  \rb{Location}  &  \rb{$T_\infty$}       \\
           &  \rb{(kW)}      &  \rb{Factor}    &  \rb{(\si{\celsius})}  \\ \hline \hline
Test 1     &  516            &  1              &  15                    \\ \hline
Test 2     &  516            &  1              &  14                    \\ \hline
Test 3     &  2000           &  1              &  15                    \\ \hline
Test 4     &  516            &  1              &  15                    \\ \hline
Test 5     &  516            &  1              &  19                    \\ \hline
Test 6     &  500            &  2              &  15                    \\ \hline
Test 7     &  516            &  1              &  15                    \\ \hline
Test 8     &  1000           &  1              &  21                    \\ \hline
Test 9     &  1000           &  1              &  24                    \\ \hline
Test 10    &  1000           &  2              &  18                    \\ \hline
Test 11    &  500            &  2              &  16                    \\ \hline
Test 12    &  2000           &  2              &  17                    \\ \hline
Test 13    &  2000           &  2              &  21                    \\ \hline
Test 14    &  500            &  2              &  15                    \\ \hline
Test 15    &  1000           &  2              &  18                    \\ \hline
Test 16    &  500            &  4              &  18                    \\ \hline
Test 17    &  500            &  4              &  10                    \\ \hline
Test 21    &  500            &  1              &  14                    \\ \hline
Test 22    &  1000           &  1              &  14                    \\ \hline
\end{tabular}
\end{center}
\end{table}


\clearpage


\section{LLNL Enclosure}

\subsection*{HGL Temperature~\cite{SFPE:Walton}}

For the cases with natural ventilation (MQH), the door size was 2.06 m high by 0.76 m wide.

\begin{table}[!h]
\caption[Validation input parameters for LLNL Enclosure cases, HGL temperature]
{Summary of validation input parameters used for LLNL Enclosure cases~\cite{Foote:LLNL1986}, HGL temperature.}

\begin{center}
\begin{tabular}{|l|c|}
\hline
                            &              \\
\rb{Input Parameter}        &  \rb{Value}  \\ \hline \hline
$c\sb{p}$ (\si{kJ/(kg.K)})  &  1.0         \\ \hline
$k$ (\si{kW/(m.K)})         &  0.000463    \\ \hline
$\rho$ (kg/m$^3$)           &  1607        \\ \hline
$c$ (\si{kJ/(kg.K)})        &  1.0         \\ \hline
$\delta$ (m)                &  0.1         \\ \hline
Location Factor             &  1           \\ \hline
Heat Loss Fraction          &  0           \\ \hline
Fuel Height                 &  0           \\ \hline
\end{tabular}
\end{center}

\begin{center}
\begin{tabular}{|l|l|c|c|c|c|c|c|c|}
\hline
           &                    &                 &            &            &            &                 &                        &                    \\
\rb{Test}  &  \rb{Correlation}  &  \rb{$\dot Q$}  &  \rb{$L$}  &  \rb{$W$}  &  \rb{$H$}  &  \rb{$\dot m$}  &  \rb{$T_\infty$}       &  \rb{$t\sb{end}$}  \\
           &                    &  \rb{(kW)}      &  \rb{(m)}  &  \rb{(m)}  &  \rb{(m)}  &  \rb{(kg/s)}    &  \rb{(\si{\celsius})}  &  \rb{(s)}          \\ \hline \hline
Test 1     &  Beyler            &  200            &  6.0       &  4.0       &  4.5       &  -              &  23                    &  500               \\ \hline
Test 2     &  Beyler            &  200            &  6.0       &  4.0       &  4.5       &  -              &  27                    &  500               \\ \hline
Test 3     &  Beyler            &  400            &  6.0       &  4.0       &  4.5       &  -              &  27                    &  200               \\ \hline
Test 4     &  Beyler            &  300            &  6.0       &  4.0       &  4.5       &  -              &  24                    &  300               \\ \hline
Test 5     &  Beyler            &  50             &  6.0       &  4.0       &  4.5       &  -              &  28                    &  2500              \\ \hline
Test 6     &  Beyler            &  100            &  6.0       &  4.0       &  4.5       &  -              &  29                    &  1000              \\ \hline
Test 7     &  Beyler            &  100            &  6.0       &  4.0       &  4.5       &  -              &  35                    &  1000              \\ \hline
Test 8     &  Beyler            &  200            &  6.0       &  4.0       &  4.5       &  -              &  35                    &  500               \\ \hline
Test 9     &  FPA, DB           &  200            &  6.0       &  4.0       &  4.5       &  0.565          &  33                    &  4000              \\ \hline
Test 10    &  FPA, DB           &  200            &  6.0       &  4.0       &  4.5       &  0.118          &  28                    &  6000              \\ \hline
Test 11    &  FPA, DB           &  200            &  6.0       &  4.0       &  4.5       &  0.240          &  18                    &  4000              \\ \hline
Test 12    &  FPA, DB           &  200            &  6.0       &  4.0       &  4.5       &  0.366          &  21                    &  5000              \\ \hline
Test 13    &  FPA, DB           &  200            &  6.0       &  4.0       &  4.5       &  0.474          &  28                    &  5000              \\ \hline
Test 14    &  FPA, DB           &  200            &  6.0       &  4.0       &  4.5       &  0.472          &  28                    &  5000              \\ \hline
Test 15    &  FPA, DB           &  100            &  6.0       &  4.0       &  4.5       &  0.352          &  24                    &  4000              \\ \hline
Test 16    &  FPA, DB           &  200            &  6.0       &  4.0       &  4.5       &  0.356          &  21                    &  6000              \\ \hline
Test 17    &  FPA, DB           &  200            &  6.0       &  4.0       &  3.0       &  0.587          &  26                    &  3000              \\ \hline
Test 18    &  FPA, DB           &  200            &  6.0       &  4.0       &  3.0       &  0.463          &  21                    &  4000              \\ \hline
Test 19    &  FPA, DB           &  200            &  6.0       &  4.0       &  3.0       &  0.351          &  18                    &  5000              \\ \hline
Test 20    &  FPA, DB           &  200            &  6.0       &  4.0       &  3.0       &  0.240          &  16                    &  6000              \\ \hline
Test 21    &  FPA, DB           &  200            &  6.0       &  4.0       &  3.0       &  0.116          &  23                    &  6000              \\ \hline
Test 22    &  FPA, DB           &  200            &  6.0       &  4.0       &  3.0       &  0.225          &  30                    &  500               \\ \hline
Test 23    &  FPA, DB           &  200            &  6.0       &  4.0       &  3.0       &  0.249          &  28                    &  4000              \\ \hline
Test 24    &  FPA, DB           &  200            &  6.0       &  4.0       &  3.0       &  0.235          &  26                    &  1000              \\ \hline
Test 25    &  FPA, DB           &  200            &  6.0       &  4.0       &  3.0       &  0.233          &  25                    &  2000              \\ \hline
\end{tabular}
\end{center}
\end{table}

\begin{table}[!ht]
\begin{center}
\begin{tabular}{|l|l|c|c|c|c|c|c|c|}
\hline
           &                    &                 &            &            &            &                 &                        &                    \\
\rb{Test}  &  \rb{Correlation}  &  \rb{$\dot Q$}  &  \rb{$L$}  &  \rb{$W$}  &  \rb{$H$}  &  \rb{$\dot m$}  &  \rb{$T_\infty$}       &  \rb{$t\sb{end}$}  \\
           &                    &  \rb{(kW)}      &  \rb{(m)}  &  \rb{(m)}  &  \rb{(m)}  &  \rb{(kg/s)}    &  \rb{(\si{\celsius})}  &  \rb{(s)}          \\ \hline \hline
Test 26    &  FPA, DB           &  200            &  6.0       &  4.0       &  3.0       &  0.586          &  24                    &  4000              \\ \hline
Test 27    &  FPA, DB           &  200            &  6.0       &  4.0       &  3.0       &  0.116          &  23                    &  500               \\ \hline
Test 28    &  FPA, DB           &  150            &  6.0       &  4.0       &  3.0       &  0.174          &  31                    &  1000              \\ \hline
Test 29    &  FPA, DB           &  250            &  6.0       &  4.0       &  3.0       &  0.298          &  28                    &  1000              \\ \hline
Test 30    &  FPA, DB           &  250            &  6.0       &  4.0       &  3.0       &  0.349          &  34                    &  4000              \\ \hline
Test 31    &  FPA, DB           &  250            &  6.0       &  4.0       &  3.0       &  0.576          &  36                    &  4000              \\ \hline
Test 32    &  FPA, DB           &  100            &  6.0       &  4.0       &  3.0       &  0.110          &  33                    &  4000              \\ \hline
Test 33    &  FPA, DB           &  100            &  6.0       &  4.0       &  3.0       &  0.230          &  23                    &  5000              \\ \hline
Test 34    &  FPA, DB           &  100            &  6.0       &  4.0       &  3.0       &  0.342          &  34                    &  4000              \\ \hline
Test 35    &  FPA, DB           &  100            &  6.0       &  4.0       &  3.0       &  0.455          &  22                    &  4000              \\ \hline
Test 36    &  FPA, DB           &  100            &  6.0       &  4.0       &  3.0       &  0.582          &  29                    &  4000              \\ \hline
Test 37    &  FPA, DB           &  170            &  6.0       &  4.0       &  3.0       &  0.111          &  20                    &  500               \\ \hline
Test 38    &  FPA, DB           &  200            &  6.0       &  4.0       &  3.0       &  0.346          &  29                    &  4000              \\ \hline
Test 39    &  FPA, DB           &  250            &  6.0       &  4.0       &  3.0       &  0.107          &  18                    &  500               \\ \hline
Test 40    &  FPA, DB           &  200            &  6.0       &  4.0       &  3.0       &  0.467          &  28                    &  4000              \\ \hline
Test 41    &  FPA, DB           &  150            &  6.0       &  4.0       &  3.0       &  0.126          &  20                    &  500               \\ \hline
Test 42    &  FPA, DB           &  200            &  6.0       &  4.0       &  3.0       &  0.206          &  30                    &  5000              \\ \hline
Test 43    &  Beyler            &  200            &  6.0       &  4.0       &  3.0       &  -              &  32                    &  500               \\ \hline
Test 44    &  FPA, DB           &  200            &  6.0       &  4.0       &  3.0       &  0.213          &  19                    &  2000              \\ \hline
Test 45    &  Beyler            &  200            &  6.0       &  4.0       &  3.0       &  -              &  30                    &  500               \\ \hline
Test 46    &  FPA, DB           &  200            &  6.0       &  4.0       &  3.0       &  0.201          &  19                    &  500               \\ \hline
Test 47    &  Beyler            &  200            &  6.0       &  4.0       &  3.0       &  -              &  19                    &  500               \\ \hline
Test 48    &  Beyler            &  165            &  6.0       &  4.0       &  3.0       &  -              &  21                    &  500               \\ \hline
Test 49    &  FPA, DB           &  200            &  6.0       &  4.0       &  3.0       &  0.208          &  26                    &  500               \\ \hline
Test 50    &  FPA, DB           &  200            &  6.0       &  4.0       &  3.0       &  0.222          &  21                    &  4000              \\ \hline
Test 51    &  MQH               &  200            &  6.0       &  4.0       &  3.0       &  -              &  33                    &  3000              \\ \hline
Test 52    &  MQH               &  200            &  6.0       &  4.0       &  3.0       &  -              &  23                    &  4000              \\ \hline
Test 53    &  FPA, DB           &  200            &  6.0       &  4.0       &  3.0       &  0.214          &  33                    &  1000              \\ \hline
Test 54    &  FPA, DB           &  200            &  6.0       &  4.0       &  3.0       &  0.248          &  21                    &  4000              \\ \hline
Test 55    &  MQH               &  100            &  6.0       &  4.0       &  3.0       &  -              &  31                    &  4000              \\ \hline
Test 56    &  FPA, DB           &  200            &  6.0       &  4.0       &  3.0       &  0.221          &  20                    &  1000              \\ \hline
Test 57    &  FPA, DB           &  200            &  6.0       &  4.0       &  3.0       &  0.247          &  29                    &  5000              \\ \hline
Test 58    &  FPA, DB           &  200            &  6.0       &  4.0       &  3.0       &  0.220          &  18                    &  4000              \\ \hline
Test 59    &  FPA, DB           &  200            &  6.0       &  4.0       &  3.0       &  0.214          &  24                    &  4000              \\ \hline
Test 60    &  MQH               &  400            &  6.0       &  4.0       &  3.0       &  -              &  22                    &  2000              \\ \hline
Test 61    &  MQH               &  200            &  6.0       &  4.0       &  4.5       &  -              &  31                    &  2000              \\ \hline
Test 62    &  MQH               &  400            &  6.0       &  4.0       &  4.5       &  -              &  22                    &  2000              \\ \hline
Test 63    &  MQH               &  50             &  6.0       &  4.0       &  4.5       &  -              &  28                    &  3000              \\ \hline
Test 64    &  MQH               &  100            &  6.0       &  4.0       &  4.5       &  -              &  17                    &  3000              \\ \hline
\end{tabular}
\end{center}
\end{table}


\clearpage


\section{NBS Multi-Room}

\subsection*{HGL Temperature~\cite{SFPE:Walton}}

\begin{table}[!ht]
\caption[Validation input parameters for NBS Multi-Room cases, HGL temperature]
{Summary of validation input parameters used for NBS Multi-Room cases~\cite{Peacock:NBS_Multi-Room}, HGL temperature.}

\begin{center}
\begin{tabular}{|l|c|}
\hline
                      &              \\
\rb{Input Parameter}  &  \rb{Value}  \\ \hline \hline
$\dot Q$ (kW)         &  100         \\ \hline
$L$ (m)               &  2.34        \\ \hline
$W$ (m)               &  2.34        \\ \hline
$H$ (m)               &  2.16        \\ \hline
$H\sb{o}$ (m)         &  1.6         \\ \hline
$W\sb{o}$ (m)         &  0.81        \\ \hline
$k$ (\si{kW/(m.K)})   &  0.00017     \\ \hline
$\rho$ (kg/m$^3$)     &  128         \\ \hline
$c$ (\si{kJ/(kg.K)})  &  1.04        \\ \hline
$\delta$ (m)          &  0.05        \\ \hline
$t\sb{end}$ (s)       &  1200        \\ \hline
\end{tabular}
\end{center}

\begin{center}
\begin{tabular}{|l|l|c|}
\hline
           &                    &                        \\
\rb{Test}  &  \rb{Correlation}  &  \rb{$T_\infty$}       \\
           &                    &  \rb{(\si{\celsius})}  \\ \hline \hline
Test 100A  &  MQH               &  23                    \\ \hline
Test 100O  &  MQH               &  21                    \\ \hline
Test 100Z  &  MQH               &  22                    \\ \hline
\end{tabular}
\end{center}
\end{table}


\clearpage


\section{NIST/NRC}

\subsection*{HGL Temperature and Depth~\cite{SFPE:Walton, Walton:1, Tanaka:1}}

For the cases with natural ventilation (MQH), the door size was 2.0 m high by 2.0 m wide.

\begin{table}[!ht]
\caption[Validation input parameters for NIST/NRC cases, HGL temperature and depth]
{Summary of validation input parameters used for NIST/NRC cases~\cite{Hamins:SP1013-1}, HGL temperature and depth.}

\begin{center}
\begin{tabular}{|l|c|}
\hline
                            &              \\
\rb{Input Parameter}        &  \rb{Value}  \\ \hline \hline
$L$ (m)                     &  21.66       \\ \hline
$W$ (m)                     &  7.04        \\ \hline
$H$ (m)                     &  3.82        \\ \hline
$c\sb{p}$ (\si{kJ/(kg.K)})  &  1.0         \\ \hline
$k$ (\si{kW/(m.K)})         &  0.00012     \\ \hline
$\rho$ (kg/m$^3$)           &  737         \\ \hline
$c$ (\si{kJ/(kg.K)})        &  1.42        \\ \hline
$\delta$ (m)                &  0.0254      \\ \hline
Location Factor             &  1           \\ \hline
Heat Loss Fraction          &  0           \\ \hline
Fuel Height                 &  0           \\ \hline
\end{tabular}
\end{center}

\begin{center}
\begin{tabular}{|l|l|c|c|c|c|c|c|}
\hline
           &                    &                 &                 &                  &                  &                        &                    \\
\rb{Test}  &  \rb{Correlation}  &  \rb{$\dot Q$}  &  \rb{$\dot m$}  &  \rb{$H\sb{o}$}  &  \rb{$W\sb{o}$}  &  \rb{$T_\infty$}       &  \rb{$t\sb{end}$}  \\
           &                    &  \rb{(kW)}      &  \rb{(kg/s)}    &  \rb{(m)}        &  \rb{(m)}        &  \rb{(\si{\celsius})}  &  \rb{(s)}          \\ \hline \hline
Test 1     &  Beyler            &  410            &  -              &  -               &  -               &  22                    &  1350              \\ \hline
Test 2     &  Beyler            &  1190           &  -              &  -               &  -               &  26                    &  625               \\ \hline
Test 3     &  MQH               &  1190           &  -              &  2.0             &  2.0             &  30                    &  1380              \\ \hline
Test 4     &  FPA, DB           &  1200           &  1.3            &  -               &  -               &  27                    &  816               \\ \hline
Test 5     &  MQH               &  1190           &  -              &  2.0             &  2.0             &  28                    &  1380              \\ \hline
Test 7     &  Beyler            &  400            &  -              &  -               &  -               &  24                    &  1330              \\ \hline
Test 8     &  Beyler            &  1190           &  -              &  -               &  -               &  25                    &  610               \\ \hline
Test 9     &  MQH               &  1170           &  -              &  2.0             &  2.0             &  27                    &  1380              \\ \hline
Test 10    &  FPA, DB           &  1190           &  1.3            &  -               &  -               &  27                    &  826               \\ \hline
Test 13    &  Beyler            &  2330           &  -              &  -               &  -               &  31                    &  265               \\ \hline
Test 14    &  MQH               &  1180           &  -              &  2.0             &  2.0             &  28                    &  1380              \\ \hline
Test 15    &  MQH               &  1180           &  -              &  2.0             &  2.0             &  18                    &  1380              \\ \hline
Test 16    &  FPA, DB           &  2300           &  1.3            &  -               &  -               &  26                    &  380               \\ \hline
Test 17    &  Beyler            &  1160           &  -              &  -               &  -               &  29                    &  272               \\ \hline
Test 18    &  MQH               &  1180           &  -              &  2.0             &  2.0             &  27                    &  1380              \\ \hline
\end{tabular}
\end{center}
\end{table}


\clearpage


\subsection*{Point Source and Solid Flame Radiation Heat Flux~\cite{Beyler2:SFPE}}

\begin{table}[!ht]
\caption[Validation input parameters for NIST/NRC cases, radiation heat flux]
{Summary of validation input parameters used for NIST/NRC cases~\cite{Hamins:SP1013-1}, radiation heat flux.}

\begin{center}
\begin{tabular}{|l|l|}
\hline
                      &                                \\
\rb{Input Parameter}  &  \rb{Value}                    \\ \hline \hline
$x$ (m)               &  3.14, 2.33, 2.18, 1.58, 3.27  \\ \hline
$z$ (m)               &  2.05, 2.52, 2.54, 3.04, 1.76  \\ \hline
IOR                   &  -3, -3, 2, -3, -2             \\ \hline
\end{tabular}
\end{center}

\begin{center}
\begin{tabular}{|l|c|c|c|}
\hline
           &                 &                     &                \\
\rb{Test}  &  \rb{$\dot Q$}  &  \rb{$\chi\sb{r}$}  &  \rb{$A$}      \\
           &  \rb{(kW)}      &                     &  \rb{(m$^2$)}  \\ \hline \hline
Test 1     &  410            &  0.44               &  0.671         \\ \hline
Test 2     &  1190           &  0.44               &  1.028         \\ \hline
Test 3     &  1190           &  0.44               &  1.028         \\ \hline
Test 4     &  1200           &  0.44               &  1.032         \\ \hline
Test 5     &  1190           &  0.44               &  1.028         \\ \hline
Test 7     &  400            &  0.44               &  0.665         \\ \hline
Test 8     &  1190           &  0.44               &  1.028         \\ \hline
Test 9     &  1170           &  0.44               &  1.021         \\ \hline
Test 10    &  1190           &  0.44               &  1.028         \\ \hline
Test 13    &  2330           &  0.44               &  1.345         \\ \hline
Test 14    &  1180           &  0.44               &  1.025         \\ \hline
Test 15    &  1180           &  0.44               &  1.025         \\ \hline
Test 16    &  2300           &  0.44               &  1.338         \\ \hline
Test 17    &  1160           &  0.40               &  1.018         \\ \hline
Test 18    &  1180           &  0.44               &  1.025         \\ \hline
\end{tabular}
\end{center}
\end{table}


\clearpage


\subsection*{Ceiling Jet Temperature (Alpert)~\cite{SFPE:Alpert}}

\begin{table}[!ht]
\caption[Validation input parameters for NIST/NRC cases, ceiling jet temperature]
{Summary of validation input parameters used for NIST/NRC cases~\cite{Hamins:SP1013-1}, ceiling jet temperature.}

\begin{center}
\begin{tabular}{|l|c|}
\hline
                      &              \\
\rb{Input Parameter}  &  \rb{Value}  \\ \hline \hline
Location Factor       &  1           \\ \hline
$r$ (m)               &  5.9         \\ \hline
$H$ (m)               &  3.72        \\ \hline
\end{tabular}
\end{center}

\begin{center}
\begin{tabular}{|l|c|c|}
\hline
           &                 &                        \\
\rb{Test}  &  \rb{$\dot Q$}  &  \rb{$T_\infty$}       \\
           &  \rb{(kW)}      &  \rb{(\si{\celsius})}  \\ \hline \hline
Test 1     &  410            &  22                    \\ \hline
Test 2     &  1190           &  26                    \\ \hline
Test 3     &  1190           &  30                    \\ \hline
Test 4     &  1200           &  27                    \\ \hline
Test 5     &  1190           &  28                    \\ \hline
Test 7     &  400            &  24                    \\ \hline
Test 8     &  1190           &  25                    \\ \hline
Test 9     &  1170           &  27                    \\ \hline
Test 10    &  1190           &  27                    \\ \hline
Test 13    &  2330           &  31                    \\ \hline
Test 14    &  1180           &  28                    \\ \hline
Test 15    &  1180           &  18                    \\ \hline
Test 16    &  2300           &  26                    \\ \hline
Test 17    &  1160           &  29                    \\ \hline
Test 18    &  1180           &  27                    \\ \hline
\end{tabular}
\end{center}
\end{table}


\clearpage


\section{NIST Smoke Alarms}

\subsection*{Ceiling Jet Temperature (Alpert)~\cite{SFPE:Alpert}}

\begin{table}[!ht]
\caption[Validation input parameters for NIST Smoke Alarms cases, ceiling jet temperature]
{Summary of validation input parameters used for NIST Smoke Alarms cases~\cite{Bukowski:1}, ceiling jet temperature.}

\begin{center}
\begin{tabular}{|l|c|}
\hline
                          &              \\
\rb{Input Parameter}      &  \rb{Value}  \\ \hline \hline
Location Factor           &  1           \\ \hline
$H$ (m)                   &  2.1         \\ \hline
$t\sb{end}$ (s)           &  300         \\ \hline
\end{tabular}
\end{center}

\begin{center}
\begin{tabular}{|l|c|c|c|}
\hline
            &                   &            &                        \\
\rb{Test}   &  \rb{$\alpha$}    &  \rb{$r$}  &  \rb{$T_\infty$}       \\
            &  \rb{(kW/s$^2$)}  &  \rb{(m)}  &  \rb{(\si{\celsius})}  \\ \hline \hline
Test SDC02  &  0.00463          &  1.15      &  21                    \\ \hline
Test SDC05  &  0.00617          &  1.25      &  22                    \\ \hline
Test SDC07  &  0.01080          &  1.25      &  24                    \\ \hline
Test SDC10  &  0.00463          &  1.15      &  26                    \\ \hline
Test SDC33  &  0.00309          &  1.15      &  26                    \\ \hline
Test SDC35  &  0.00309          &  1.15      &  26                    \\ \hline
Test SDC38  &  0.00370          &  1.25      &  25                    \\ \hline
Test SDC39  &  0.00617          &  1.25      &  25                    \\ \hline
\end{tabular}
\end{center}
\end{table}


\clearpage


\subsection*{Smoke Detector Activation Time (Temperature Rise)~\cite{SFPE:Alpert, Bukowski:2}}

The fire growth was specified as $\dot Q = \alpha t^2$ up to a cutoff time of $t\sb{fire}$.
After the time $t\sb{fire}$, the fire HRR was steady.

\begin{table}[!ht]
\caption[Validation input parameters for NIST Smoke Alarms cases, smoke detector activation time]
{Summary of validation input parameters used for NIST Smoke Alarms cases~\cite{Bukowski:1}, smoke detector activation time.}

\begin{center}
\begin{tabular}{|l|c|}
\hline
                                  &              \\
\rb{Input Parameter}              &  \rb{Value}  \\ \hline \hline
Location Factor                   &  1           \\ \hline
$t\sb{fire}$ (s)                  &  300         \\ \hline
$\Delta T\sb{c}$ (\si{\celsius})  &  10          \\ \hline
RTI (\si{(m.s)^{1/2}})            &  5           \\ \hline
\end{tabular}
\end{center}

\begin{center}
\begin{tabular}{|l|c|c|c|c|}
\hline
            &                   &            &            &                        \\
\rb{Test}   &  \rb{$\alpha$}    &  \rb{$r$}  &  \rb{$H$}  &  \rb{$T_\infty$}       \\
            &  \rb{(kW/s$^2$)}  &  \rb{(m)}  &  \rb{(m)}  &  \rb{(\si{\celsius})}  \\ \hline \hline
Test SDC02  &  0.00463          &  1.3       &  2.1       &  21                    \\ \hline
Test SDC05  &  0.00617          &  1.8       &  2.0       &  22                    \\ \hline
Test SDC07  &  0.01080          &  1.8       &  2.0       &  24                    \\ \hline
Test SDC10  &  0.00463          &  1.3       &  2.1       &  26                    \\ \hline
Test SDC33  &  0.00309          &  1.3       &  2.1       &  26                    \\ \hline
Test SDC35  &  0.00309          &  1.3       &  2.1       &  26                    \\ \hline
Test SDC38  &  0.00370          &  1.3       &  2.1       &  25                    \\ \hline
Test SDC39  &  0.00617          &  1.8       &  2.0       &  25                    \\ \hline
\end{tabular}
\end{center}
\end{table}


\clearpage


\section{SP AST}

\subsection*{HGL Temperature~\cite{SFPE:Walton}}

\begin{table}[!ht]
\caption[Validation input parameters for SP AST cases, HGL temperature]
{Summary of validation input parameters used for SP AST cases, HGL temperature.}

\begin{center}
\begin{tabular}{|l|c|}
\hline
                            &              \\
\rb{Input Parameter}        &  \rb{Value}  \\ \hline \hline
$\dot Q$ (kW)               &  450         \\ \hline
$L$ (m)                     &  3.6         \\ \hline
$W$ (m)                     &  3.6         \\ \hline
$H$ (m)                     &  2.4         \\ \hline
$H\sb{o}$ (m)               &  2.0         \\ \hline
$W\sb{o}$ (m)               &  0.8         \\ \hline
$k$ (\si{kW/(m.K)})         &  0.0001      \\ \hline
$\rho$ (kg/m$^3$)           &  600         \\ \hline
$c$ (\si{kJ/(kg.K)})        &  0.8         \\ \hline
$\delta$ (m)                &  0.2         \\ \hline
$T_\infty$ (\si{\celsius})  &  20          \\ \hline
\end{tabular}
\end{center}

\begin{center}
\begin{tabular}{|l|l|c|}
\hline
           &                    &                    \\
\rb{Test}  &  \rb{Correlation}  &  \rb{$t\sb{end}$}  \\
           &                    &  \rb{(s)}          \\ \hline \hline
Test 1     &  MQH               &  2400              \\ \hline
Test 2     &  MQH               &  2400              \\ \hline
Test 3     &  MQH               &  3600              \\ \hline
\end{tabular}
\end{center}
\end{table}


\clearpage


\subsection*{Steel Temperature (Unprotected)~\cite{SFPE:Milke2}}

The HGL temperatures from the MQH HGL temperature correlation were used as the input fire temperature~$T\sb{f}$.

\begin{table}[!ht]
\caption[Validation input parameters for SP AST cases, unprotected steel temperature]
{Summary of validation input parameters used for SP AST cases~\cite{Wickstrom_AST}, unprotected steel temperature.}

\begin{center}
\begin{tabular}{|l|c|}
\hline
                            &              \\
\rb{Input Parameter}        &  \rb{Value}  \\ \hline \hline
$\rho\sb{s}$ (kg/m$^3$)     &  7833        \\ \hline
$c\sb{s}$ (\si{kJ/(kg.K)})  &  0.465       \\ \hline
$\epsilon$                  &  0.7         \\ \hline
$h\sb{c}$ (\si{W/(m^2.K)})  &  25          \\ \hline
\end{tabular}
\end{center}

\begin{center}
\begin{tabular}{|l|l|c|c|}
\hline
           &                      &              &                    \\
\rb{Test}  &  \rb{Correlation}    &  \rb{F/V}    &  \rb{$t\sb{end}$}  \\
           &  \rb{for $T\sb{f}$}  &  \rb{(1/m)}  &  \rb{(s)}          \\ \hline \hline
Test 1     &  MQH                 &  125         &  2400              \\ \hline
Test 2     &  MQH                 &  157         &  2400              \\ \hline
Test 3     &  MQH                 &  157         &  3600              \\ \hline
\end{tabular}
\end{center}
\end{table}


\clearpage


\subsection*{Ceiling Jet Temperature (Alpert)~\cite{SFPE:Alpert}}

\begin{table}[!ht]
\caption[Validation input parameters for SP AST cases, ceiling jet temperature]
{Summary of validation input parameters used for SP AST cases~\cite{Wickstrom_AST}, ceiling jet temperature.}

\begin{center}
\begin{tabular}{|l|l|}
\hline
                              &              \\
\rb{Input Parameter}          &  \rb{Value}  \\ \hline \hline
$\dot Q$ (kW)                 &  450         \\ \hline
$H$ (m)                       &  1.65        \\ \hline
$T_{\infty}$ (\si{\celsius})  &  20          \\ \hline
\end{tabular}
\end{center}

\begin{center}
\begin{tabular}{|l|c|c|}
\hline
           &                        &            \\
\rb{Test}  &  \rb{Location Factor}  &  \rb{$r$}  \\
           &                        &  \rb{(m)}  \\ \hline \hline
Test 1     &  4                     &  1.22      \\ \hline
Test 2     &  4                     &  1.22      \\ \hline
Test 3     &  2                     &  0.70      \\ \hline
\end{tabular}
\end{center}
\end{table}


\clearpage


\section{SP AST Column}

\subsection*{Plume Temperature (Heskestad and McCaffrey)~\cite{SFPE:Heskestad, McCaffrey:NBSIR_79-1910}}

Note that the Heskestad correlation did not use heights of 1 m and 2 m because
it is valid only above the flame region.

\begin{table}[!ht]
\caption[Validation input parameters for SP AST Column cases, plume temperature]
{Summary of validation input parameters used for SP AST Column cases~\cite{Sjostrom:AST}, plume temperature.}

\begin{center}
\begin{tabular}{|l|l|}
\hline
                            &                 \\
\rb{Input Parameter}        &  \rb{Value}     \\ \hline \hline
$z$ (m)                     &  1, 2, 3, 4, 5  \\ \hline
$\chi\sb{r}$                &  0.40           \\ \hline
$c\sb{p}$ (\si{kJ/(kg.K)})  &  1.0            \\ \hline
$T_\infty$ (\si{\celsius})  &  20             \\ \hline
\end{tabular}
\end{center}

\begin{center}
\begin{tabular}{|l|c|c|}
\hline
                &                  &                \\
\rb{Test}       &  \rb{$\dot Q$}   &  \rb{$A$}      \\
                &  \rb{(kW)}       &  \rb{(m$^2$)}  \\ \hline \hline
Diesel, 1.1 m   &  1434            &  0.95          \\ \hline
Diesel, 1.9 m   &  1873            &  0.95          \\ \hline
Heptane, 1.1 m  &  2275            &  2.83          \\ \hline
\end{tabular}
\end{center}
\end{table}


\clearpage


\subsection*{Steel Temperature (Unprotected)~\cite{SFPE:Milke2}}

The HGL temperatures from the McCaffrey plume temperature correlation were used as the input fire temperature~$T\sb{f}$.

\begin{table}[!ht]
\caption[Validation input parameters for SP AST Column cases, unprotected steel temperature]
{Summary of validation input parameters used for SP AST Column cases~\cite{Sjostrom:AST}, unprotected steel temperature.}

\begin{center}
\begin{tabular}{|l|c|}
\hline
                            &              \\
\rb{Input Parameter}        &  \rb{Value}  \\ \hline \hline
$F/V$ (1/m)                 &  205         \\ \hline
$\rho\sb{s}$ (kg/m$^3$)     &  7833        \\ \hline
$c\sb{s}$ (\si{kJ/(kg.K)})  &  0.465       \\ \hline
$\epsilon$                  &  0.7         \\ \hline
$h\sb{c}$ (\si{W/(m^2.K)})  &  25          \\ \hline
\end{tabular}
\end{center}

\begin{center}
\begin{tabular}{|l|l|c|c|}
\hline
                &                      &                 &                    \\
\rb{Test}       &  \rb{Correlation}    &  \rb{$\dot Q$}  &  \rb{$t\sb{end}$}  \\
                &  \rb{for $T\sb{f}$}  &  \rb{(kW)}      &  \rb{(s)}          \\ \hline \hline
Diesel, 1.1 m   &  McCaffrey           &  1434           &  1620              \\ \hline
Diesel, 1.9 m   &  McCaffrey           &  1873           &  1080              \\ \hline
Heptane, 1.1 m  &  McCaffrey           &  2275           &  900               \\ \hline
\end{tabular}
\end{center}
\end{table}


\clearpage


\section{Steckler}

\subsection*{HGL Temperature~\cite{SFPE:Walton}}

\begin{table}[!ht]
\caption[Validation input parameters for Steckler cases, HGL temperature]
{Summary of validation input parameters used for Steckler cases~\cite{Steckler:NBSIR_82-2520}, HGL temperature.}

\begin{center}
\begin{tabular}{|l|c|}
\hline
                          &              \\
\rb{Input Parameter}      &  \rb{Value}  \\ \hline \hline
$L$ (m)                   &  2.8         \\ \hline
$W$ (m)                   &  2.8         \\ \hline
$H$ (m)                   &  2.13        \\ \hline
$k$ (\si{kW/(m.K)})       &  0.0001      \\ \hline
$\rho$ (kg/m$^3$)         &  200         \\ \hline
$c$ (\si{kJ/(kg.K)})      &  1.0         \\ \hline
$\delta$ (m)              &  0.013       \\ \hline
\end{tabular}
\end{center}

\begin{center}
\begin{tabular}{|l|l|c|c|c|c|}
\hline
           &                    &                 &                  &                  &                        \\
\rb{Test}  &  \rb{Correlation}  &  \rb{$\dot Q$}  &  \rb{$H\sb{o}$}  &  \rb{$W\sb{o}$}  &  \rb{$T_{\infty}$}     \\
           &                    &  \rb{(kW)}      &  \rb{(m)}        &  \rb{(m)}        &  \rb{(\si{\celsius})}  \\ \hline \hline
Test 10    &  MQH               &  62.9           &  1.83            &  0.24            &  22                    \\ \hline
Test 11    &  MQH               &  62.9           &  1.83            &  0.36            &  25                    \\ \hline
Test 12    &  MQH               &  62.9           &  1.83            &  0.49            &  19                    \\ \hline
Test 612   &  MQH               &  62.9           &  1.83            &  0.49            &  19                    \\ \hline
Test 13    &  MQH               &  62.9           &  1.83            &  0.62            &  20                    \\ \hline
Test 14    &  MQH               &  62.9           &  1.83            &  0.74            &  28                    \\ \hline
Test 18    &  MQH               &  62.9           &  1.83            &  0.74            &  29                    \\ \hline
Test 710   &  MQH               &  62.9           &  1.83            &  0.74            &  13                    \\ \hline
Test 810   &  MQH               &  62.9           &  1.83            &  0.74            &  15                    \\ \hline
Test 16    &  MQH               &  62.9           &  1.83            &  0.86            &  23                    \\ \hline
Test 17    &  MQH               &  62.9           &  1.83            &  0.99            &  19                    \\ \hline
Test 22    &  MQH               &  62.9           &  1.38            &  0.74            &  26                    \\ \hline
Test 23    &  MQH               &  62.9           &  0.92            &  0.74            &  23                    \\ \hline
Test 30    &  MQH               &  62.9           &  0.92            &  0.74            &  23                    \\ \hline
Test 41    &  MQH               &  62.9           &  0.46            &  0.74            &  14                    \\ \hline
Test 19    &  MQH               &  31.6           &  1.83            &  0.74            &  29                    \\ \hline
Test 20    &  MQH               &  105.3          &  1.83            &  0.74            &  29                    \\ \hline
Test 21    &  MQH               &  158.           &  1.83            &  0.74            &  29                    \\ \hline
Test 114   &  MQH               &  62.9           &  1.83            &  0.24            &  31                    \\ \hline
Test 144   &  MQH               &  62.9           &  1.83            &  0.36            &  30                    \\ \hline
Test 212   &  MQH               &  62.9           &  1.83            &  0.49            &  25                    \\ \hline
Test 242   &  MQH               &  62.9           &  1.83            &  0.62            &  29                    \\ \hline
Test 410   &  MQH               &  62.9           &  1.83            &  0.74            &  21                    \\ \hline
\end{tabular}
\end{center}
\end{table}

\begin{table}[!ht]
\begin{center}
\begin{tabular}{|l|l|c|c|c|c|}
\hline
           &                    &                 &                  &                  &                       \\
\rb{Test}  &  \rb{Correlation}  &  \rb{$\dot Q$}  &  \rb{$H\sb{o}$}  &  \rb{$W\sb{o}$}  & \rb{$T_{\infty}$}     \\
           &                    &  \rb{(kW)}      &  \rb{(m)}        &  \rb{(m)}        & \rb{(\si{\celsius})}  \\ \hline \hline
Test 210   &  MQH               &  62.9           &  1.83            &  0.74            &  30                   \\ \hline
Test 310   &  MQH               &  62.9           &  1.83            &  0.74            &  20                   \\ \hline
Test 240   &  MQH               &  62.9           &  1.83            &  0.86            &  28                   \\ \hline
Test 116   &  MQH               &  62.9           &  1.83            &  0.99            &  29                   \\ \hline
Test 122   &  MQH               &  62.9           &  1.38            &  0.74            &  27                   \\ \hline
Test 224   &  MQH               &  62.9           &  0.92            &  0.74            &  26                   \\ \hline
Test 324   &  MQH               &  62.9           &  0.92            &  0.74            &  22                   \\ \hline
Test 220   &  MQH               &  31.6           &  1.83            &  0.74            &  25                   \\ \hline
Test 221   &  MQH               &  105.3          &  1.83            &  0.74            &  25                   \\ \hline
Test 514   &  MQH               &  62.9           &  1.83            &  0.24            &  8                    \\ \hline
Test 544   &  MQH               &  62.9           &  1.83            &  0.36            &  8                    \\ \hline
Test 512   &  MQH               &  62.9           &  1.83            &  0.49            &  20                   \\ \hline
Test 542   &  MQH               &  62.9           &  1.83            &  0.62            &  20                   \\ \hline
Test 610   &  MQH               &  62.9           &  1.83            &  0.74            &  18                   \\ \hline
Test 510   &  MQH               &  62.9           &  1.83            &  0.74            &  22                   \\ \hline
Test 540   &  MQH               &  62.9           &  1.83            &  0.86            &  13                   \\ \hline
Test 517   &  MQH               &  62.9           &  1.83            &  0.99            &  14                   \\ \hline
Test 622   &  MQH               &  62.9           &  1.38            &  0.74            &  9                    \\ \hline
Test 522   &  MQH               &  62.9           &  1.38            &  0.74            &  13                   \\ \hline
Test 524   &  MQH               &  62.9           &  0.92            &  0.74            &  8                    \\ \hline
Test 541   &  MQH               &  62.9           &  0.46            &  0.74            &  7                    \\ \hline
Test 520   &  MQH               &  31.6           &  1.83            &  0.74            &  17                   \\ \hline
Test 521   &  MQH               &  105.3          &  1.83            &  0.74            &  13                   \\ \hline
Test 513   &  MQH               &  158.           &  1.83            &  0.74            &  14                   \\ \hline
Test 160   &  MQH               &  62.9           &  1.83            &  0.74            &  6                    \\ \hline
Test 163   &  MQH               &  62.9           &  1.83            &  0.74            &  6                    \\ \hline
Test 164   &  MQH               &  62.9           &  1.83            &  0.74            &  6                    \\ \hline
Test 165   &  MQH               &  62.9           &  1.83            &  0.74            &  6                    \\ \hline
Test 162   &  MQH               &  62.9           &  1.83            &  0.74            &  6                    \\ \hline
Test 167   &  MQH               &  62.9           &  1.83            &  0.74            &  6                    \\ \hline
Test 161   &  MQH               &  62.9           &  1.83            &  0.74            &  6                    \\ \hline
Test 166   &  MQH               &  62.9           &  1.83            &  0.74            &  6                    \\ \hline
\end{tabular}
\end{center}
\end{table}


\clearpage


\section{UL/NFPRF}

\subsection*{Ceiling Jet Temperature (Alpert)~\cite{SFPE:Alpert}}

\begin{table}[!ht]
\caption[Validation input parameters for UL/NFPRF cases, ceiling jet temperature]
{Summary of validation input parameters used for UL/NFPRF cases~\cite{Sheppard:1, McGrattan:5}, ceiling jet temperature.}

\begin{center}
\begin{tabular}{|l|l|}
\hline
                              &                                                                            \\
\rb{Input Parameter}          &  \rb{Value}                                                                \\ \hline \hline
Location Factor               &  1                                                                         \\ \hline
$r$ (m)                       &  2.12, 4.74, 6.36, 10.61, 12.90, 13.58, 14.23, 14.85, 17.10, 19.09, 21.32  \\ \hline
$H$ (m)                       &  7.1                                                                       \\ \hline
$T_{\infty}$ (\si{\celsius})  &  18                                                                        \\ \hline
\end{tabular}
\end{center}

\begin{center}
\begin{tabular}{|l|c|}
\hline
           &                 \\
\rb{Test}  &  \rb{$\dot Q$}  \\
           &  \rb{(kW)}      \\ \hline \hline
I-17       &  4600           \\ \hline
I-18       &  3700           \\ \hline
I-19       &  4600           \\ \hline
I-20       &  4200           \\ \hline
I-21       &  4600           \\ \hline
I-22       &  4600           \\ \hline
\end{tabular}
\end{center}
\end{table}


\clearpage


\subsection*{Sprinkler Activation Time~\cite{SFPE:Alpert}}

The fire growth was specified as $\dot Q = \alpha t^2$ up to a cutoff time of $t\sb{fire}$.
After the time $t\sb{fire}$, the fire HRR was steady.

\begin{table}[!ht]
\caption[Validation input parameters for UL/NFPRF cases, sprinkler activation time]
{Summary of validation input parameters used for UL/NFPRF cases~\cite{Sheppard:1, McGrattan:5}, sprinkler activation time.}

\begin{center}
\begin{tabular}{|l|c|}
\hline
                             &              \\
\rb{Input Parameter}         &  \rb{Value}  \\ \hline \hline
$\alpha$ (kW/s$^2$)          &  1.778       \\ \hline
Location Factor              &  1           \\ \hline
RTI (\si{(m.s)^{1/2}})       &  148         \\ \hline
$T\sb{act}$ (\si{\celsius})  &  74          \\ \hline
$H$ (m)                      &  7.0         \\ \hline
$T_\infty$ (\si{\celsius})   &  18          \\ \hline
\end{tabular}
\end{center}

\begin{center}
\begin{tabular}{|l|c|c||l|c|c|}
\hline
           &                     &            &            &                     &            \\
\rb{Test}  &  \rb{$t\sb{fire}$}  &  \rb{$r$}  & \rb{Test}  &  \rb{$t\sb{fire}$}  &  \rb{$r$}  \\
           &  \rb{(s)}           &  \rb{(m)}  &            &  \rb{(s)}           &  \rb{(m)}  \\ \hline \hline
I-1        &  50                 &  2.12      &  II-1      &  75                 &  1.5       \\ \hline
I-2        &  50                 &  2.12      &  II-2      &  75                 &  1.5       \\ \hline
I-3        &  50                 &  2.12      &  II-3      &  75                 &  1.5       \\ \hline
I-4        &  50                 &  2.12      &  II-4      &  75                 &  1.5       \\ \hline
I-5        &  50                 &  2.12      &  II-5      &  75                 &  1.5       \\ \hline
I-6        &  50                 &  2.12      &  II-6      &  75                 &  1.5       \\ \hline
I-7        &  50                 &  2.12      &  II-7      &  75                 &  1.5       \\ \hline
I-8        &  50                 &  2.12      &  II-8      &  75                 &  1.5       \\ \hline
I-9        &  50                 &  2.12      &  II-9      &  75                 &  2.12      \\ \hline
I-10       &  50                 &  2.12      &  II-10     &  75                 &  2.12      \\ \hline
I-11       &  50                 &  2.12      &  II-11     &  75                 &  1.5       \\ \hline
I-12       &  50                 &  2.12      &  II-12     &  75                 &  1.5       \\ \hline
I-13       &  58                 &  2.12      &            &                     &  1.5       \\ \hline
I-14       &  57                 &  2.12      &            &                     &  1.5       \\ \hline
I-15       &  57                 &  2.12      &            &                     &  1.5       \\ \hline
I-16       &  106                &  2.12      &            &                     &  1.5       \\ \hline
I-17       &  51                 &  2.12      &            &                     &  1.5       \\ \hline
I-18       &  47                 &  2.12      &            &                     &  1.5       \\ \hline
I-19       &  51                 &  2.12      &            &                     &  1.5       \\ \hline
I-20       &  49                 &  2.12      &            &                     &  1.5       \\ \hline
I-21       &  51                 &  2.12      &            &                     &  1.5       \\ \hline
I-22       &  51                 &  2.12      &            &                     &  1.5       \\ \hline
\end{tabular}
\end{center}
\end{table}


\clearpage


\section{USN Hawaii}

\subsection*{Plume Temperature (Heskestad and McCaffrey)~\cite{SFPE:Heskestad, McCaffrey:NBSIR_79-1910}}

The fire growth was specified as $\dot Q = \alpha t^2$ up to a cutoff time of $t\sb{fire}$.
After the time $t\sb{fire}$, the fire HRR was steady.

\begin{table}[!ht]
\caption[Validation input parameters for USN Hawaii cases, plume temperature]
{Summary of validation input parameters used for USN Hawaii cases~\cite{Gott:1}, plume temperature.}

\begin{center}
\begin{tabular}{|l|l|}
\hline
                            &                         \\
\rb{Input Parameter}        &  \rb{Value}             \\ \hline \hline
$z$ (m)                     &  8.7, 11.8, 13.3, 14.5  \\ \hline
$\chi\sb{r}$                &  0.40                   \\ \hline
$c\sb{p}$ (\si{kJ/(kg.K)})  &  1.0                    \\ \hline
\end{tabular}
\end{center}

\begin{center}
\begin{tabular}{|l|c|c|c|c|c|}
\hline
           &                   &                     &                  &                &                        \\
\rb{Test}  &  \rb{$\alpha$}    &  \rb{$t\sb{fire}$}  &  \rb{$\dot Q$}   &  \rb{$A$}      &  \rb{$T_\infty$}       \\
           &  \rb{(kW/s$^2$)}  &  \rb{(s)}           &  \rb{(kW)}       &  \rb{(m$^2$)}  &  \rb{(\si{\celsius})}  \\ \hline \hline
Test 1     &  0.00694          &  120                &  100             &  0.09          &  27                    \\ \hline
Test 2     &  0.02222          &  150                &  500             &  0.36          &  28                    \\ \hline
Test 3     &  0.04889          &  150                &  1100            &  0.81          &  27                    \\ \hline
Test 4     &  0.09133          &  180                &  2959            &  1.69          &  27                    \\ \hline
Test 5     &  0.48597          &  120                &  6998            &  3.24          &  27                    \\ \hline
Test 6     &  0.33773          &  150                &  7599            &  4.84          &  25                    \\ \hline
Test 7     &  0.17799          &  180                &  5767            &  3.24          &  30                    \\ \hline
Test 11    &  0.00444          &  150                &  100             &  0.09          &  30                    \\ \hline
\end{tabular}
\end{center}
\end{table}


\clearpage


\section{USN Iceland}

\subsection*{Plume Temperature (Heskestad and McCaffrey)~\cite{SFPE:Heskestad, McCaffrey:NBSIR_79-1910}}

\begin{table}[!ht]
\caption[Validation input parameters for USN Hawaii cases, plume temperature]
{Summary of validation input parameters used for USN Hawaii cases~\cite{Gott:1}, plume temperature.}

\begin{center}
\begin{tabular}{|l|l|}
\hline
                            &                                \\
\rb{Input Parameter}        &  \rb{Value}                    \\ \hline \hline
$z$ (m)                     &  15.7, 17.2, 18.8, 20.3, 21.5  \\ \hline
$\chi\sb{r}$                &  0.40                          \\ \hline
$c\sb{p}$ (\si{kJ/(kg.K)})  &  1.0                           \\ \hline
\end{tabular}
\end{center}

\begin{center}
\begin{tabular}{|l|c|c|c|}
\hline
           &                  &                &                        \\
\rb{Test}  &  \rb{$\dot Q$}   &  \rb{$A$}      &  \rb{$T_\infty$}       \\
           &  \rb{(kW)}       &  \rb{(m$^2$)}  &  \rb{(\si{\celsius})}  \\ \hline \hline
Test 1     &  88              &  0.09          &  10                    \\ \hline
Test 2     &  60              &  0.09          &  10                    \\ \hline
Test 3     &  720             &  0.36          &  11                    \\ \hline
Test 4     &  684             &  0.36          &  14                    \\ \hline
Test 5     &  1134            &  0.81          &  17                    \\ \hline
Test 6     &  1296            &  0.81          &  16                    \\ \hline
Test 7     &  2736            &  1.44          &  16                    \\ \hline
Test 9     &  153             &  0.09          &  9                     \\ \hline
Test 10    &  612             &  0.36          &  9                     \\ \hline
Test 11    &  648             &  0.36          &  9                     \\ \hline
Test 12    &  1512            &  0.36          &  9                     \\ \hline
Test 13    &  2700            &  1.44          &  11                    \\ \hline
Test 14    &  7802            &  4.84          &  12                    \\ \hline
Test 15    &  15300           &  9.00          &  12                    \\ \hline
Test 17    &  15750           &  9.00          &  11                    \\ \hline
Test 18    &  4918            &  3.24          &  10                    \\ \hline
Test 19    &  8712            &  4.84          &  13                    \\ \hline
Test 20    &  14850           &  9.00          &  14                    \\ \hline
\end{tabular}
\end{center}
\end{table}


\clearpage


\section{Vettori Flat Ceiling}

\subsection*{Ceiling Jet Temperature (Alpert)~\cite{SFPE:Alpert}}

\begin{table}[!ht]
\caption[Validation input parameters for Vettori Flat cases, ceiling jet temperature]
{Summary of validation input parameters used for Vettori Flat cases~\cite{Vettori:1}, ceiling jet temperature.}

Only the smooth, unobstructed ceiling tests were included in this study.
The fire growth was specified as $\dot Q = \alpha t^2$ up to a cutoff time of $t\sb{fire}$.
After the time $t\sb{fire}$, the fire HRR was steady.

\begin{center}
\begin{tabular}{|l|c|}
\hline
                      &              \\
\rb{Input Parameter}  &  \rb{Value}  \\ \hline \hline
$H$ (m)               &  2.075       \\ \hline
\end{tabular}
\end{center}

\begin{center}
\begin{tabular}{|l|c|c|l|c|c|}
\hline
           &                   &                     &             &                 &                        \\
\rb{Test}  &  \rb{$\alpha$}    &  \rb{$t\sb{fire}$}  &  \rb{$r$}   &  \rb{Location}  &  \rb{$T_\infty$}       \\
           &  \rb{(kW/s$^2$)}  &  \rb{(s)}           &  \rb{(m)}   &  \rb{Factor}    &  \rb{(\si{\celsius})}  \\ \hline \hline
Test 1     &  0.105            &  50                 &  2.2, 2.2   &  1              &  16.6                  \\ \hline
Test 2     &  0.105            &  50                 &  2.2, 2.2   &  1              &  19.0                  \\ \hline
Test 3     &  0.105            &  50                 &  2.2, 2.2   &  1              &  20.8                  \\ \hline
Test 6     &  0.017            &  80                 &  2.2, 2.2   &  1              &  17.1                  \\ \hline
Test 7     &  0.017            &  80                 &  2.2, 2.2   &  1              &  21.6                  \\ \hline
Test 8     &  0.017            &  80                 &  2.2, 2.2   &  1              &  21.8                  \\ \hline
Test 11    &  0.0041           &  100                &  2.2, 2.2   &  1              &  18.1                  \\ \hline
Test 12    &  0.0041           &  140                &  2.2, 2.2   &  1              &  21.0                  \\ \hline
Test 13    &  0.0041           &  130                &  2.2, 2.2   &  1              &  22.0                  \\ \hline
Test 16    &  0.105            &  45                 &  2.4, 2.4   &  2              &  20.7                  \\ \hline
Test 17    &  0.105            &  42                 &  2.4, 2.4   &  2              &  20.4                  \\ \hline
Test 18    &  0.105            &  35                 &  2.4, 2.4   &  2              &  20.3                  \\ \hline
Test 21    &  0.017            &  70                 &  2.4, 2.4   &  2              &  21.1                  \\ \hline
Test 22    &  0.017            &  70                 &  2.4, 2.4   &  2              &  21.7                  \\ \hline
Test 23    &  0.017            &  70                 &  2.4, 2.4   &  2              &  21.8                  \\ \hline
Test 26    &  0.0041           &  130                &  2.4, 2.4   &  2              &  21.5                  \\ \hline
Test 27    &  0.0041           &  130                &  2.4, 2.4   &  2              &  22.5                  \\ \hline
Test 28    &  0.0041           &  120                &  2.4, 2.4   &  2              &  22.7                  \\ \hline
Test 31    &  0.105            &  37                 &  2.1, 6.59  &  4              &  20.2                  \\ \hline
Test 32    &  0.105            &  30                 &  2.1, 6.59  &  4              &  21.9                  \\ \hline
Test 33    &  0.105            &  30                 &  2.1, 6.59  &  4              &  21.7                  \\ \hline
Test 36    &  0.017            &  50                 &  2.1, 6.59  &  4              &  22.3                  \\ \hline
Test 37    &  0.017            &  47                 &  2.1, 6.59  &  4              &  22.3                  \\ \hline
Test 38    &  0.017            &  50                 &  2.1, 6.59  &  4              &  21.7                  \\ \hline
Test 41    &  0.0041           &  100                &  2.1, 6.59  &  4              &  21.6                  \\ \hline
Test 42    &  0.0041           &  85                 &  2.1, 6.59  &  4              &  22.4                  \\ \hline
Test 43    &  0.0041           &  85                 &  2.1, 6.59  &  4              &  21.6                  \\ \hline
\end{tabular}
\end{center}
\end{table}


\clearpage


\subsection*{Sprinkler Activation Time~\cite{SFPE:Alpert}}

The fire growth was specified as $\dot Q = \alpha t^2$ up to the time of the first sprinkler activation.

\begin{table}[!ht]
\caption[Validation input parameters for Vettori Flat cases, sprinkler activation time]
{Summary of validation input parameters used for Vettori Flat cases~\cite{Vettori:1}, sprinkler activation time.}

\begin{center}
\begin{tabular}{|l|c|}
\hline
                             &              \\
\rb{Input Parameter}         &  \rb{Value}  \\ \hline \hline
RTI (\si{(m.s)^{1/2}})       &  55          \\ \hline
$T\sb{act}$ (\si{\celsius})  &  68          \\ \hline
$H$ (m)                      &  2.09        \\ \hline
\end{tabular}
\end{center}

\begin{center}
\begin{tabular}{|l|c|c|c|c|}
\hline
           &                   &            &                 &                        \\
\rb{Test}  &  \rb{$\alpha$}    &  \rb{$r$}  &  \rb{Location}  &  \rb{$T_\infty$}       \\
           &  \rb{(kW/s$^2$)}  &  \rb{(m)}  &  \rb{Factor}    &  \rb{(\si{\celsius})}  \\ \hline \hline
Test 1     &  0.105            &  2.2       &  1              &  16.6                  \\ \hline
Test 2     &  0.105            &  2.2       &  1              &  19.0                  \\ \hline
Test 3     &  0.105            &  2.2       &  1              &  20.8                  \\ \hline
Test 6     &  0.017            &  2.2       &  1              &  17.1                  \\ \hline
Test 7     &  0.017            &  2.2       &  1              &  21.6                  \\ \hline
Test 8     &  0.017            &  2.2       &  1              &  21.8                  \\ \hline
Test 11    &  0.0041           &  2.2       &  1              &  18.1                  \\ \hline
Test 12    &  0.0041           &  2.2       &  1              &  21.0                  \\ \hline
Test 13    &  0.0041           &  2.2       &  1              &  22.0                  \\ \hline
Test 16    &  0.105            &  2.4       &  2              &  20.7                  \\ \hline
Test 17    &  0.105            &  2.4       &  2              &  20.4                  \\ \hline
Test 18    &  0.105            &  2.4       &  2              &  20.3                  \\ \hline
Test 21    &  0.017            &  2.4       &  2              &  21.1                  \\ \hline
Test 22    &  0.017            &  2.4       &  2              &  21.7                  \\ \hline
Test 23    &  0.017            &  2.4       &  2              &  21.8                  \\ \hline
Test 26    &  0.0041           &  2.4       &  2              &  21.5                  \\ \hline
Test 27    &  0.0041           &  2.4       &  2              &  22.5                  \\ \hline
Test 28    &  0.0041           &  2.4       &  2              &  22.7                  \\ \hline
Test 31    &  0.105            &  2.1       &  4              &  20.2                  \\ \hline
Test 32    &  0.105            &  2.1       &  4              &  21.9                  \\ \hline
Test 33    &  0.105            &  2.1       &  4              &  21.7                  \\ \hline
Test 36    &  0.017            &  2.1       &  4              &  22.3                  \\ \hline
Test 37    &  0.017            &  2.1       &  4              &  22.3                  \\ \hline
Test 38    &  0.017            &  2.1       &  4              &  21.7                  \\ \hline
Test 41    &  0.0041           &  2.1       &  4              &  21.6                  \\ \hline
Test 42    &  0.0041           &  2.1       &  4              &  22.4                  \\ \hline
Test 43    &  0.0041           &  2.1       &  4              &  21.6                  \\ \hline
\end{tabular}
\end{center}
\end{table}


\clearpage


\section{VTT Large Hall}

\subsection*{Plume Temperature (Heskestad and McCaffrey)~\cite{SFPE:Heskestad, McCaffrey:NBSIR_79-1910}}

The fire ramp was specified as $\dot Q\sb{ramp}$ with a corresponding $t\sb{ramp}$.
For example, a fire growing linearly over time 0~s to 100~s from 0~kW to 500~kW
would be specified with a $t\sb{ramp}$ of [0, 100] and a $\dot Q\sb{ramp}$ of [0, 500].

\begin{table}[!ht]
\caption[Validation input parameters for VTT Large Hall cases, plume temperature]
{Summary of validation input parameters used for VTT Large Hall cases~\cite{Hostikka:VTT2104}, plume temperature.}

\begin{center}
\begin{tabular}{|l|l|}
\hline
                            &              \\
\rb{Input Parameter}        &  \rb{Value}  \\ \hline \hline
$z$ (m)                     &  6, 12       \\ \hline
$\chi\sb{r}$                &  0.40        \\ \hline
$c\sb{p}$ (\si{kJ/(kg.K)})  &  1.0         \\ \hline
\end{tabular}
\end{center}

\begin{center}
\begin{tabular}{|l|l|l|c|c|}
\hline
           &                                  &                                            &                &                        \\
\rb{Test}  &  \rb{$t\sb{ramp}$}               &  \rb{$\dot Q\sb{ramp}$}                    &  \rb{$A$}      &  \rb{$T_\infty$}       \\
           &  \rb{(s)}                        &  \rb{(kW)}                                 &  \rb{(m$^2$)}  &  \rb{(\si{\celsius})}  \\ \hline \hline
Test 1     &  13, 90, 288, 327, 409           &  1245, 1309, 1858, 1783, 1356              &  1.075         &  25                    \\ \hline
Test 2     &  14, 30, 91, 193, 282, 340, 372  &  2151, 2542, 3063, 3259, 3129, 2737, 2281  &  2.01          &  22                    \\ \hline
Test 3     &  13, 63, 166, 256, 292, 330      &  2437, 3201, 3601, 3638, 3456, 2656        &  2.01          &  22                    \\ \hline
\end{tabular}
\end{center}
\end{table}


\clearpage


\section{WTC}

\subsection*{HGL Temperature~\cite{SFPE:Walton}}

\begin{table}[!ht]
\caption[Validation input parameters for WTC cases, HGL temperature]
{Summary of validation input parameters used for WTC cases~\cite{NIST_NCSTAR_1-5B}, HGL temperature.}

\begin{center}
\begin{tabular}{|l|c|}
\hline
                      &              \\
\rb{Input Parameter}  &  \rb{Value}  \\ \hline \hline
$L$ (m)               &  7.04        \\ \hline
$W$ (m)               &  3.60        \\ \hline
$H$ (m)               &  3.82        \\ \hline
$H\sb{o}$ (m)         &  2.82        \\ \hline
$W\sb{o}$ (m)         &  2.4         \\ \hline
$k$ (\si{kW/(m.K)})   &  0.00012     \\ \hline
$\rho$ (kg/m$^3$)     &  737         \\ \hline
$c$ (\si{kJ/(kg.K)})  &  1.42        \\ \hline
$\delta$ (m)          &  0.0254      \\ \hline
\end{tabular}
\end{center}

\begin{center}
\begin{tabular}{|l|l|c|c|c|}
\hline
           &                    &                 &                    &                        \\
\rb{Test}  &  \rb{Correlation}  &  \rb{$\dot Q$}  &  \rb{$t\sb{end}$}  &  \rb{$T_\infty$}       \\
           &                    &  \rb{(kW)}      &  \rb{(s)}          &  \rb{(\si{\celsius})}  \\ \hline \hline
Test 1     &  MQH               &  2000           &  870               &  24                    \\ \hline
Test 2     &  MQH               &  2400           &  380               &  25                    \\ \hline
Test 3     &  MQH               &  2000           &  990               &  20                    \\ \hline
Test 4     &  MQH               &  3200           &  840               &  20                    \\ \hline
Test 5     &  MQH               &  3000           &  3080              &  20                    \\ \hline
Test 6     &  MQH               &  3000           &  3030              &  20                    \\ \hline
\end{tabular}
\end{center}
\end{table}


\clearpage


\subsection*{Steel Temperature (Unprotected)~\cite{SFPE:Milke2}}

The HGL temperatures from the MQH HGL temperature correlation were used as the input fire temperature~$T\sb{f}$.
The inputs for the MQH correlation are described in the WTC HGL temperature section.

\begin{table}[!ht]
\caption[Validation input parameters for WTC cases, unprotected steel temperature]
{Summary of validation input parameters used for WTC cases~\cite{NIST_NCSTAR_1-5B}, unprotected steel temperature.}

\begin{center}
\begin{tabular}{|l|c|}
\hline
                            &              \\
\rb{Input Parameter}        &  \rb{Value}  \\ \hline \hline
$\rho\sb{s}$ (kg/m$^3$)     &  7860        \\ \hline
$c\sb{s}$ (\si{kJ/(kg.K)})  &  0.450       \\ \hline
$\epsilon$                  &  0.7         \\ \hline
$h\sb{c}$ (\si{W/(m^2.K)})  &  25          \\ \hline
\end{tabular}
\end{center}

\begin{center}
\begin{tabular}{|l|l|l|c|}
\hline
           &                      &                   &              \\
\rb{Test}  &  \rb{Correlation}    &  \rb{Structural}  &  \rb{F/V}    \\
           &  \rb{for $T\sb{f}$}  &  \rb{Element}     &  \rb{(1/m)}  \\ \hline \hline
Test 1     &  MQH                 &  Bar              &  157         \\ \hline
Test 2     &  MQH                 &  Bar              &  157         \\ \hline
Test 3     &  MQH                 &  Bar              &  157         \\ \hline
Test 1     &  MQH                 &  Column           &  159         \\ \hline
Test 2     &  MQH                 &  Column           &  159         \\ \hline
Test 3     &  MQH                 &  Column           &  159         \\ \hline
Test 1     &  MQH                 &  Truss A          &  156         \\ \hline
Test 2     &  MQH                 &  Truss A          &  156         \\ \hline
Test 3     &  MQH                 &  Truss A          &  156         \\ \hline
Test 1     &  MQH                 &  Truss B          &  156         \\ \hline
Test 2     &  MQH                 &  Truss B          &  156         \\ \hline
Test 3     &  MQH                 &  Truss B          &  156         \\ \hline
\end{tabular}
\end{center}
\end{table}


\clearpage


\subsection*{Steel Temperature (Protected)~\cite{SFPE:Milke2}}

The HGL temperatures from the MQH HGL temperature correlation were used as the input fire temperature~$T\sb{f}$.
The inputs for the MQH correlation are described in the WTC HGL temperature section.

\begin{table}[!ht]
\caption[Validation input parameters for WTC cases, protected steel temperature]
{Summary of validation input parameters used for WTC cases~\cite{NIST_NCSTAR_1-5B}, protected steel temperature.}

\begin{center}
\begin{tabular}{|l|c|}
\hline
                            &              \\
\rb{Input Parameter}        &  \rb{Value}  \\ \hline \hline
$c\sb{s}$ (\si{kJ/(kg.K)})  &  0.450       \\ \hline
$k\sb{i}$ (\si{W/(m.K)})    &  0.10        \\ \hline
$\rho\sb{i}$ (kg/m$^3$)     &  208         \\ \hline
$c\sb{i}$ (\si{kJ/(kg.K)})  &  2.0         \\ \hline
\end{tabular}
\end{center}

\begin{center}
\begin{tabular}{|l|l|l|c|c|}
\hline
           &                      &                   &                  &                   \\
\rb{Test}  &  \rb{Correlation}    &  \rb{Structural}  &  \rb{$h\sb{i}$}  &  \rb{W/D}         \\
           &  \rb{for $T\sb{f}$}  &  \rb{Element}     &  \rb{(m)}        &  \rb{(kg/m$^2$)}  \\ \hline \hline
Test 4     &  MQH                 &  Bar              &  0.0191          &  50.1             \\ \hline
Test 5     &  MQH                 &  Bar              &  0.0191          &  50.1             \\ \hline
Test 6     &  MQH                 &  Bar              &  0.0191          &  50.1             \\ \hline
Test 4     &  MQH                 &  Column           &  0.0381          &  49.4             \\ \hline
Test 5     &  MQH                 &  Column           &  0.0381          &  49.4             \\ \hline
Test 6     &  MQH                 &  Column           &  0.0381          &  49.4             \\ \hline
Test 4     &  MQH                 &  Truss A          &  0.0191          &  50.3             \\ \hline
Test 5     &  MQH                 &  Truss A          &  0.0191          &  50.3             \\ \hline
Test 6     &  MQH                 &  Truss A          &  0.0191          &  50.3             \\ \hline
Test 4     &  MQH                 &  Truss B          &  0.0381          &  50.3             \\ \hline
Test 5     &  MQH                 &  Truss B          &  0.0381          &  50.3             \\ \hline
Test 6     &  MQH                 &  Truss B          &  0.0381          &  50.3             \\ \hline
\end{tabular}
\end{center}
\end{table}


\clearpage


\subsection*{Point Source and Solid Flame Radiation Heat Flux~\cite{Beyler2:SFPE}}

\begin{table}[!ht]
\caption[Validation input parameters for WTC cases, radiation heat flux]
{Summary of validation input parameters used for WTC cases~\cite{NIST_NCSTAR_1-5B}, radiation heat flux.}

\begin{center}
\begin{tabular}{|l|l|}
\hline
                      &                                                                                      \\
\rb{Input Parameter}  &  \rb{Value}                                                                          \\ \hline \hline
$x$ (m)               &  0.90, 0.82, 0.71, 0.77, 1.23, 1.34, 1.23, 1.34, 1.07, 1.56, 0.57, 0.35, 0.87, 2.58  \\ \hline
$z$ (m)               &  3.3, 3.3, 3.15, 3.15, 3.46, 3.27, 0.92, 1.02, 0.13, 0.13, 3.82, 3.82, 3.82, 3.82    \\ \hline
IOR                   &  3, 3, -3, -3, 1, 2, 1, 2, 3, 3, -3, -3, -3, -3                                      \\ \hline
\end{tabular}
\end{center}

\begin{center}
\begin{tabular}{|l|c|c|c|}
\hline
           &                 &                     &                \\
\rb{Test}  &  \rb{$\dot Q$}  &  \rb{$\chi\sb{r}$}  &  \rb{$A$}      \\
           &  \rb{(kW)}      &                     &  \rb{(m$^2$)}  \\ \hline \hline
Test 1     &  2000           &  0.44               &  1.258         \\ \hline
Test 2     &  2400           &  0.39               &  1.455         \\ \hline
Test 3     &  2000           &  0.39               &  1.258         \\ \hline
Test 4     &  3200           &  0.44               &  1.832         \\ \hline
Test 5     &  3000           &  0.44               &  1.739         \\ \hline
Test 6     &  3000           &  0.44               &  1.739         \\ \hline
\end{tabular}
\end{center}
\end{table}


\clearpage


\subsection*{Ceiling Jet Temperature (Alpert)~\cite{SFPE:Alpert}}

\begin{table}[!ht]
\caption[Validation input parameters for WTC cases, ceiling jet temperature]
{Summary of validation input parameters used for WTC cases, ceiling jet temperature.}

\begin{center}
\begin{tabular}{|l|l|}
\hline
                      &              \\
\rb{Input Parameter}  &  \rb{Value}  \\ \hline \hline
Location Factor       &  1           \\ \hline
$r$ (m)               &  2, 3        \\ \hline
$H$ (m)               &  3.72        \\ \hline
\end{tabular}
\end{center}

\begin{center}
\begin{tabular}{|l|c|c|}
\hline
           &                 &                        \\
\rb{Test}  &  \rb{$\dot Q$}  &  \rb{$T_\infty$}       \\
           &  \rb{(kW)}      &  \rb{(\si{\celsius})}  \\ \hline \hline
Test 1     &  2000           &  24                    \\ \hline
Test 2     &  2400           &  25                    \\ \hline
Test 3     &  2000           &  20                    \\ \hline
Test 4     &  3200           &  20                    \\ \hline
Test 5     &  3000           &  20                    \\ \hline
Test 6     &  3000           &  20                    \\ \hline
\end{tabular}
\end{center}
\end{table}



\end{document}
