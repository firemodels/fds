% !TEX root = FDTs_Validation_Guide.tex

\chapter{Heat Flux}
\label{Heat_Flux_Chapter}

\section{Point Source Radiation Heat Flux}

The point source model assumes that radiative energy is concentrated at a point located within a flame.
Here, the point source is located at a point one-third the height of the flame.
The radiative heat flux, $\dot q_r''$ (\si{kW/m^2}), at any distance $R$ (\si{m}) from this point can be predicted using
\be
\dot q_r'' = \cos\theta \left( \frac{\chi_r \dot Q}{4 \pi R^2} \right)
\label{eq:point_source}
\ee
where the $\cos\theta$ term (equal to $x/R$ for targets facing sideways, or $z/R$ for gauges facing upward or downward) accounts for a target that is at an angle $\theta$ from the source. The IOR orientation parameter is used to specify which direction the target or gauge is facing: 1 or -1 for the positive or negative $x$ direction, 2 or -2 for the positive or negative $y$ direction, and 3 or -3 for the positive or negative $z$ direction. $\chi_r$ is the radiative fraction~(-), $\dot Q$ is the HRR of the fire~(\si{kW}), and $R$ is the radial distance from the point source to the edge of the target~(\si{m}) and is given by
\be
R = \sqrt{x^2 + \left(z - \frac{L_f}{3} \right)^2}
\label{eq:point_source_R}
\ee
where $x$ is the horizontal distance from the point source to the edge of the target~(\si{m}), $z$ is the height of the heat flux target~(\si{m}). The flame height, $L_f$ (\si{m}), is given by
\be
L_f = D (3.7 Q^{*^{2/5}} - 1.02)
\label{eq:point_source_Lf}
\ee
where $D$ is the diameter of the fire source~(\si{m}) and is given by
\be
D = \sqrt{\frac{4 A}{\pi}}
\label{eq:point_source_D}
\ee
where $A$ is the area of the fire source~(m$^2$). The nondimensional HRR, $Q^*$, is given by
\be
Q^* = \frac{\dot Q}{\rho_\infty c_p T_\infty \sqrt{g} D^{5/2}}
\label{eq:point_source_Qstar}
\ee
where $\rho_\infty$ is the ambient air density~(\si{kg/m^3}), $c_p$ is the specific heat of air~(\si{kJ/(kg.K)}), $T_\infty$ is the ambient air temperature~(\si{K}), and $g$ is the acceleration of gravity~(\si{m/s^2}).

\begin{figure}[!ht]
\begin{center}
\begin{tabular}{l}
\includegraphics[width=4.0in]{SCRIPT_FIGURES/Scatterplots/Point_Source_Radiation_Heat_Flux}
\end{tabular}
\end{center}
\caption[Summary of point source radiation heat flux predictions]
{Summary of point source radiation heat flux temperature predictions.}
\label{Heat_Flux_Point_Source_Summary}
\end{figure}

% \section{Immersed HGL Radiation Heat Flux}

% The immersed HGL radiation heat flux can be calculated by

% \be
% \dot q_r'' = \sigma (T_g^4 - T_\infty^4)
% \label{eq:immersed_HGL_radiation}
% \ee

% In Eq.~\ref{eq:immersed_HGL_radiation}, $\dot q_r''$ is the radiative heat flux~(kW/m$^2$), $\sigma$ is the Stefan-Boltzmann constant (W/m$^2$/K$^4$), $T_g$ is the hot gas layer temperature~(K) from the appropriate HGL temperature correlation described in Chapter~\ref{HGL_Chapter}, and $T_\infty$ is the ambient temperature (K).

% Note: The immersed HGL radiation heat flux method was found in some cases to underpredict the measured heat flux. This underprediction was because the HGL temperature is lower than the temperature near the heat flux gauges due to energy averaging of the HGL method.


% \section{Immersed Plume Radiation Heat Flux}

% The immersed plume radiation heat flux can be calculated by

% \be
% \dot q_r'' = \sigma (T_p^4)
% \label{eq:immersed_plume_radiation}
% \ee

% In Eq.~\ref{eq:immersed_plume_radiation}, $\dot q_r''$ is the radiative heat flux~(kW/m$^2$), $\sigma$ is the Stefan-Boltzmann constant (W/m$^2$/K$^4$), and $T_p$ is the plume temperature (K) at a given height using the McCaffrey centerline plume temperature correlation described in Eq.~\ref{eq:McCaffrey}.

% \clearpage

% \begin{figure}[!ht]
% \begin{center}
% \begin{tabular}{l}
% \includegraphics[width=4.0in]{SCRIPT_FIGURES/Scatterplots/Immersed_HGL_Radiation_Heat_Flux}
% \end{tabular}
% \end{center}
% \caption[Summary of immersed HGL radiation heat flux predictions]
% {Summary of immersed HGL radiation heat flux predictions.}
% \label{Heat_Flux_Immersed_HGL_Summary}
% \end{figure}

% \begin{figure}[!ht]
% \begin{center}
% \begin{tabular}{l}
% \includegraphics[width=4.0in]{SCRIPT_FIGURES/Scatterplots/Immersed_Plume_Radiation_Heat_Flux}
% \end{tabular}
% \end{center}
% \caption[Summary of immersed plume radiation heat flux predictions]
% {Summary of immersed plume radiation heat flux predictions.}
% \label{Heat_Flux_Immersed_Plume_Summary}
% \end{figure}



