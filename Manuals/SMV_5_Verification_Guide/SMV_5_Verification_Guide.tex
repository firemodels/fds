\documentclass[11pt,twoside]{book}
\usepackage{times,mathptm,multirow,fancyvrb,color,array}
\usepackage{picins}

%\usepackage{wrapfig}
\usepackage[pdftex]{graphicx}
\usepackage[pdftex,
        colorlinks=true,
        urlcolor=linkblue,     % \href{...}{...} external (URL)
        citecolor=linkred,     % citation number colors
        linkcolor=linknavy,    % \ref{...} and \pageref{...}
        pdftitle={User's Guide for Smokeview Version 5 - A Tool for Visualizing Fire Dynamics Simulation Data},
        pdfauthor={Glenn Forney},
        pdfsubject={Verification Guide},
        pdfkeywords={FDS, Fire Model, NIST, BFRL},
        pdfproducer={pdflatex},
       % pagebackref,
        pdfpagemode=UseNone,
        bookmarksopen=true,
        plainpages=false]{hyperref}
\usepackage{hyperref}
\pdfcompresslevel=9
\DeclareGraphicsExtensions{.jpg,.pdf,.png}
\usepackage{setspace,moreverb}

%\usepackage{eso-pic}
\usepackage{color}
\definecolor{linknavy}{rgb}{0,0,0.50196}
\definecolor{linkred}{rgb}{1,0,0}
\definecolor{linkblue}{rgb}{0,0,1}

%\makeatletter
%   \AddToShipoutPicture{%
%     \setlength{\@tempdimb}{.5\paperwidth}%
%     \setlength{\@tempdimc}{.5\paperheight}%
%     \setlength{\unitlength}{1pt}%
%     \put(\strip@pt\@tempdimb,\strip@pt\@tempdimc){
%
%\makebox(0,0){\rotatebox{45}{\textcolor[gray]{0.75}{\fontsize{8cm}{8cm}\selectfont{DRAFT
%}}}}
%    }
% }
%\makeatother

\pagestyle{empty}

\setlength{\textwidth}{6.5in}
\setlength{\oddsidemargin}{0.00in}
\setlength{\evensidemargin}{0.0in}



\setlength{\textheight}{9.0in}
\setlength{\topmargin}{0.00in}
\setlength{\parindent}{0.2in}

\setlength{\headheight}{0.00in}
\setlength{\headsep}{0.0in}
\setlength{\paperheight}{11.0in}
\setlength{\paperwidth}{8.5in}
%
% new commands for this paper
%
\newcommand{\fdsinput}[1]{
{
\scriptsize
\verbatiminput{../../Test_cases/Visualization/#1}
}
}

\newcommand{\svini}{{\tt smokeview.ini}\ }
\newcommand{\bverb}{
\begin{Verbatim}[frame=single,rulecolor=\color{blue},
framerule=3pt,framesep=1pc,fillcolor=\color{yellow}]
}
\newcommand{\everb}{
\end{Verbatim}
}
\newcommand{\degC}{$^\circ$C}
\newcommand{\figoptions}{hbp}
\newcommand{\hhref}[1]{\href{#1}{{\tt #1}
}}
\newcommand{\figheight}{1.5in}
\newcommand{\infigheight}{0.75in}
\newcommand{\figheightA}{2.5in}
\newcommand{\figheightAbar}{2.2in}
\newcommand{\figheightC}{2.5in}
\newcommand{\figwidth}{3.333333in}
\newcommand{\figwidthb}{2.0in}
\newcommand{\FDS}{{FDS}}
\newcommand{\fds}{{FDS}}
\newcommand{\Smokeview}{{Smokeview}}
\newcommand{\smokeview}{{Smokeview}}
\newcommand{\parma}{.75}
\newcommand{\parmb}{.5}
\newcommand{\parmc}{0.25}
\newcommand{\bold}[1]{{\bf #1}}
\newcommand{\etc}{{\em etc}}
\newcommand{\ie}{{\em i.e.}}
\newcommand{\eg}{{\em e.g.}}
\newcommand{\via}{{via\ }}
\newcommand{\dialoguenmenu}{\fbox{\tt Dialog} }
\newcommand{\optionmenu}{\fbox{\tt Option} }
\newcommand{\loadmenu}{\fbox{\tt Load/Unload} }
\newcommand{\tourmenu}{\fbox{\tt Tour} }
\newcommand{\helpmenu}{\fbox{\tt Help} }
\newcommand{\setbounds}{\fbox{\tt Set Bounds} }
\newcommand{\showmenu}{\fbox{\tt Show/Hide} }
\newcommand{\frameit}[1]{\fbox{\tt #1}}
\newcommand{\blist}{
\begin{list}
{}{
\setlength{\leftmargin}{\parma in}
\setlength{\labelwidth}{\parmb in}
\setlength{\labelsep}{\parmc in}
\setlength{\listparindent}{0.3in}
\setlength{\topsep}{.3in}
\setlength{\parsep}{.0in}
}}
\newcommand{\elist}{\end{list}}
\newcommand{\hitem}[1]{\item[{\bf #1} \hfill]}

\bibliographystyle{unsrt}
%\doublespace
\begin{document}
%
% ----------------------  first cover/title page --------------------------
%
\begin{minipage}[t][9in][s]{6.5in}

\huge
\flushright{NIST Special Publication xxxx}


\vspace{1in}

\Huge
\flushright{Verification Guide for Smokeview Version 5 }

\vspace{.5in}
\normalsize
\flushright{Glenn P. Forney}

\vfill

%\flushright{\includegraphics[width=4.0in]{figures/nistlogo_1line}}
\flushright{\includegraphics[width=2.in]{figures/nistident_flright_vec}}
\end{minipage}

\newpage

\hspace{5in}
\newpage

%
% ----------------------  second cover/title page --------------------------
%
\begin{minipage}[t][9in][s]{6.5in}

\huge
\flushright{NIST Special Publication xxxx}

\vspace{1.in}

\Huge
\flushright{Verification Guide for Smokeview Version 5 }

\vspace{.5in}

\normalsize
\flushright{Glenn P. Forney\\
%\includegraphics[width=1in]{figures/bfrl}  \\
{\em Fire Research Division} \\
{\em Building and Fire Research Laboratory}  \\
}

\vspace{.25in}


\flushright{July 2008\\
$SVN Repository$~$Revision: 3090 $}
%
\vfill

\flushright{\includegraphics[width=1in]{figures/doc} }

\small
\flushright{U.S. Department of Commerce \\
{\em Carlos M. Gutierrez, Secretary} \\
\hspace{1in} \\
National Institute of Standards and Technology \\
{\em Patrick Gallagher, Acting Director} }

\end{minipage}


\date{}

\setlength{\parindent}{0.25in}

\newpage

\begin{minipage}[t][9in][s]{6.5in}


\flushright{Certain commercial entities, equipment, or materials may be identified in this \\
document in order to describe an experimental procedure or concept adequately. Such \\
identification is not intended to imply recommendation or endorsement by the \\
National Institute of Standards and Technology, nor is it intended to imply that the \\
entities, materials, or equipment are necessarily the best available for the purpose.
}

\vspace{3in}


\vspace{3in}

\large
\flushright{\bf National Institute of Standards and Technology Special Publication 1017-1 \\
Natl.~Inst.~Stand.~Technol.~Spec.~Publ.~1017-1, 142 pages (July 2008) \\
CODEN: NSPUE2 }

\vfill

\flushright{U.S. GOVERNMENT PRINTING OFFICE \\
WASHINGTON: 2007 \\
\rule{3.5in}{0.01in} \\
For sale by the Superintendent of Documents, U.S. Government Printing Office \\
Internet: bookstore.gpo.gov -- Phone: (202) 512-1800 -- Fax: (202) 512-2250 \\
Mail: Stop SSOP, Washington, DC 20402-0001 }

\end{minipage}


\frontmatter

\pagestyle{plain}

%---------------------------------------------------------------------------------
%------------------------ Preface ------------------------------------------------
%---------------------------------------------------------------------------------

\chapter{Preface}
\Smokeview\ is a software tool designed to visualize numerical
calculations generated by fire models such as the Fire Dynamics Simulator (\fds),
a computational fluid dynamics (CFD) model of fire-driven fluid
flow or CFAST, a zone fire model. Smokeview visualizes smoke and other attributes of the fire
using traditional scientific methods such as displaying tracer
particle flow, 2D or 3D shaded contours of gas flow data such as
temperature and flow vectors showing flow direction and magnitude.
Smokeview also visualizes fire attributes realistically so that one can
{\em experience}\ the fire. This is done by displaying a series of
partially transparent planes where the transparencies in each
plane (at each grid node) are determined from soot densities
computed by FDS.  \Smokeview\ also visualizes static data at
particular times again using 2D or 3D contours of data such as
temperature and flow vectors showing flow direction and magnitude.

\Smokeview\ and associated
documentation for Windows, Linux and Mac/OSX may be downloaded from the web site {\bf
\hhref{http://fire.nist.gov/fds}} at no cost.

%---------------------------------------------------------------------------------
%------------------------ Disclaimer ---------------------------------------------
%---------------------------------------------------------------------------------

\chapter{Disclaimer}

The US Department of Commerce makes no warranty,
expressed or implied, to users of Smokeview, and accepts no
responsibility for its use. Users of Smokeview assume sole
responsibility under Federal law for determining the
appropriateness of its use in any particular application; for any
conclusions drawn from the results of its use; and for any actions
taken or not taken as a result of analysis performed using this
tools.

Smokeview and the companion program FDS is intended for use only
by those competent in the fields of fluid dynamics,
thermodynamics, combustion, and heat transfer, and is intended
only to supplement the informed judgment of the qualified user.
These software packages may or may not have predictive capability
when applied to a specific set of factual circumstances. Lack of
accurate predictions could lead to erroneous conclusions with
regard to fire safety. All results should be evaluated by an
informed user.

Throughout this document, the mention of computer hardware or
commercial software does not constitute endorsement by NIST,
nor does
it indicate that the products are necessarily those
best suited for the
intended purpose.
 glenn.forney@nist.gov .

\tableofcontents
\listoffigures

\mainmatter

\pagenumbering{arabic}

%
% .............. new section ..............................
%
%\pagestyle{fancy}
%\newcounter{picno}
%\setcounter{picno}{24}
%\newcounter{picnoe}
%\setcounter{picnoe}{24}
%\fancyhead{} \fancyfoot[RO] {
% \setlength{\unitlength}{1mm}
% %\stepcounter{picno}
% \addtocounter{picno}{-1}
% \begin{picture}(0,0)
%   \put(-15,0){
%     \epsfig{height=1.125in,figure=figures/movies/tankfarm_\number\value{picno}.eps}
%   }
% \end{picture}
%}
%
%\fancyfoot[LE]
%{
%  \setlength{\unitlength}{1mm}
%  %\stepcounter{picnoe}{-1}
%  \addtocounter{picnoe}{-1}
%  \begin{picture}(0,0)
%    \put(-15,0){
%      \epsfig{height=1.25in,figure=figures/movies/townhouse3_\number\value{picnoe}.eps}
%    }
%  \end{picture}
%}
%\renewcommand{\headrulewidth}{0.0pt}
%\renewcommand{\footrulewidth}{0.0pt}
%\renewcommand{\footskip}{1.5in}

%---------------------------------------------------------------------------------
%------------------------ Introduction ----------------------------------------
%---------------------------------------------------------------------------------
\chapter{Introduction}
\section{Overview}
\Smokeview\ is an advanced scientific software tool designed to visualize numerical
predictions generated by fire models such as the Fire Dynamics Simulator (\fds),
a computational fluid dynamics (CFD) model of fire-driven fluid
flow\cite{FDS_Tech_Guide_5} and CFAST, a zone model of compartment fire phenomena\cite{Jones:2004A}. This report documents version 5 of
\smokeview\. For details on setting up and
running FDS cases read the FDS User's
guide\cite{FDS_Users_Guide_5}.

\FDS\ and \smokeview\ are used to model and visualize time-varying fire
phenomena. However, FDS and Smokeview are not limited to fire
simulation. For example, one may use FDS and \smokeview\ to model
other applications such as contaminant flow in a building.
\Smokeview\ performs this visualization by displaying time
dependent tracer particle flow, animated contour slices of
computed gas variables and surface data. \Smokeview\ also presents
contours and vector plots of static data anywhere within a
simulation scene at a fixed time. Several examples using these
techniques to investigate fire incidents are documented in Refs.
\cite{CHERRYROAD,IOWA,HOUSTON,WTC}.


%---------------------------------------------------------------------------------
%------------------------ Visualization Test Cases %---------------------------------------------------------------------------------

\newcommand{\figheightD}{2.75in}
\newcommand{\figheightE}{2.5in}
\newcommand{\figheightF}{2.25in}
\newcommand{\figheightG}{1.9in}
\newcolumntype{M}[1]{>{\centering\arraybackslash}m{#1}}

\chapter{Verifying Smokeview}
This chapter presents FDS results in the form of Smokeview images used
to verify that Smokeview is working as expected.  The version of Smokeview being
tested is
{
%\scriptsize
\verbatiminput{scriptfigures/smokeview.version}
}

FDS generated data is presumed
to be correct.  FDS has its own set of verification cases to test the correctness of the data.
The purpose of the cases here is to confirm that data is drawn correctly.
In particular these cases confirm
that correct files are loaded, data is scaled and drawn correctly, geometry is drawn correctly
{\em etc.}   Three types of verification cases are presented. The first set are the most important, those cases that verify that data is drawn correctly.  The second set of
cases verify that various geometric elements are drawn correctly and the third set verifies
that the various options and underlying features are implemented and perform properly.



\section{Data Presentation}
Figure \ref{figboundtest} shows a series of boundary file images drawn at 5.0, 10.0 and 30.0
seconds.
The first column of images colors data according to temperature using temperature while the second column of images
highlights regions where ignition has occurred defined by wherever the surface temperature exceeds a specified value.
Figure \ref{figisotest} shows a series of temperature isosurface file drawn images at 5.0, 10.0 and 30.0
seconds.  The three columns of images present the isosurfaces drawn as a solid, outline and points.
Figure \ref{figparttest} shows a series of particle file images drawn at 5.0, 10.0 and 30.0
seconds.
The first column shows particles while the second and third columns shows streaks of length 0.5 s and 1.0 s.
Figure \ref{figslicetest} shows a series of slice file images drawn at 5.0, 10.0 and 30.0
seconds.
The first column draws all of the data.  The second data discards or chops data below 140~\degC.
Figure \ref{figvslicetest} shows a series of vector slice file images drawn at 5.0, 10.0 and 30.0
seconds.
Similar to the slice file images, the first column draws all of the data while the second column discards or chops data below 140~\degC.
Figure \ref{figsmoketest} shows a series of 3D smoke images drawn at 5.0, 10.0 and 30.0
seconds.  The images contain both opacity slices derived from soot densities and heat release rate  per unit volume (HRRPUV).

\begin{figure}[\figoptions]
\begin{center}
\begin{tabular}{cccl}
 \includegraphics[width=\figheightG]{scriptfigures/colorconv_slice_00000}&
 \includegraphics[width=\figheightG]{scriptfigures/colorconv_slice_00025}&
 \includegraphics[width=\figheightG]{scriptfigures/colorconv_slice_00050}\\
 \includegraphics[width=\figheightG]{scriptfigures/colorconv_slice_00075}&
 \includegraphics[width=\figheightG]{scriptfigures/colorconv_slice_00100}&
 \includegraphics[width=\figheightG]{scriptfigures/colorconv_slice_00125}\\
 \includegraphics[width=\figheightG]{scriptfigures/colorconv_slice_00150}&
 \includegraphics[width=\figheightG]{scriptfigures/colorconv_slice_00175}&
 \includegraphics[width=\figheightG]{scriptfigures/colorconv_slice_00200}\\
 \includegraphics[width=\figheightG]{scriptfigures/colorconv_slice_00225}&
 \includegraphics[width=\figheightG]{scriptfigures/colorconv_slice_00250}&
 \includegraphics[width=\figheightG]{scriptfigures/colorconv_slice_00275}\\
 \includegraphics[width=\figheightG]{scriptfigures/colorconv_slice_00300}&
 \includegraphics[width=\figheightG]{scriptfigures/colorconv_slice_00325}&
 \includegraphics[width=\figheightG]{scriptfigures/colorconv_slice_10000}\\
&&&\raisebox{0.0in}[0pt]{\includegraphics[height=8.0in]{figures/colorbar_20_100}}\\
\end{tabular}
\end{center}
 \caption[Color conversion test.]{Color conversion test.}
\label{figboundtest}%
\end{figure}


\begin{figure}[\figoptions]
\begin{center}
\begin{tabular}{rccl}
 1.0 s
 & \includegraphics[height=\figheightD]{scriptfigures/plume5c_bound_01}
 & \includegraphics[height=\figheightD]{scriptfigures/plume5c_bound_cell_01}\\
 10.0 s&
 \includegraphics[height=\figheightD]{scriptfigures/plume5c_bound_10}&
 \includegraphics[height=\figheightD]{scriptfigures/plume5c_bound_cell_10}\\
 30.0 s&
 \includegraphics[height=\figheightD]{scriptfigures/plume5c_bound_30}&
 \includegraphics[height=\figheightD]{scriptfigures/plume5c_bound_cell_30}\\
&\\
&cell averaged  data&cell centered data\\
 &&&\raisebox{1.0in}[0pt]{\includegraphics[height=7.0in]{figures/colorbar_20_620}}\\
  \end{tabular}
\end{center}
 \caption[Boundary file test.]{Boundary file test.}
\label{figboundtest}%
\end{figure}


\begin{figure}[\figoptions]
\begin{center}
\begin{tabular}{rcc}
 1.0 s&
 \includegraphics[height=\figheightD]{scriptfigures/plume5c_iso_solid_01}&
 \includegraphics[height=\figheightD]{scriptfigures/plume5c_iso_solid_normal_01}\\
 10.0 s&
 \includegraphics[height=\figheightD]{scriptfigures/plume5c_iso_solid_10}&
 \includegraphics[height=\figheightD]{scriptfigures/plume5c_iso_solid_normal_10}\\
 30.0 s&
 \includegraphics[height=\figheightD]{scriptfigures/plume5c_iso_solid_30}&
 \includegraphics[height=\figheightD]{scriptfigures/plume5c_iso_solid_normal_30}\\
 &solid&solid with normals
  \end{tabular}
\end{center}
 \caption{Isosurface file test 1.}
\label{figisotest}%
\end{figure}

\begin{figure}[\figoptions]
\begin{center}
\begin{tabular}{rcc}
 1.0 s&
 \includegraphics[height=\figheightD]{scriptfigures/plume5c_iso_outline_01}&
 \includegraphics[height=\figheightD]{scriptfigures/plume5c_iso_points_01}\\
 10.0 s&
 \includegraphics[height=\figheightD]{scriptfigures/plume5c_iso_outline_10}&
 \includegraphics[height=\figheightD]{scriptfigures/plume5c_iso_points_10}\\
 30.0 s&
 \includegraphics[height=\figheightD]{scriptfigures/plume5c_iso_outline_30}&
 \includegraphics[height=\figheightD]{scriptfigures/plume5c_iso_points_30}\\
 &outline&points
  \end{tabular}
\end{center}
 \caption{Isosurface file test 2.}
\label{figisotest}%
\end{figure}

\begin{figure}[\figoptions]
\begin{center}
\begin{tabular}{rccc}
 1.0 s&
 \includegraphics[height=\figheightD]{scriptfigures/plume5c_part_01}&
 \includegraphics[height=\figheightD]{scriptfigures/plume5c_part_streak_01}&
 \includegraphics[height=\figheightD]{scriptfigures/plume5c_part_streak2_01}\\
 10.0 s&
 \includegraphics[height=\figheightD]{scriptfigures/plume5c_part_10}&
 \includegraphics[height=\figheightD]{scriptfigures/plume5c_part_streak_10}&
 \includegraphics[height=\figheightD]{scriptfigures/plume5c_part_streak2_10}\\
  30.0 s&
 \includegraphics[height=\figheightD]{scriptfigures/plume5c_part_30}&
 \includegraphics[height=\figheightD]{scriptfigures/plume5c_part_streak_30}&
 \includegraphics[height=\figheightD]{scriptfigures/plume5c_part_streak2_30}\\
 &points&0.5 s streaks&1.0 s streaks\\
  \end{tabular}
\end{center}
 \caption{Particle file test.}
\label{figparttest}%
\end{figure}

\begin{figure}[\figoptions]
\begin{center}
\begin{tabular}{cccl}
 \includegraphics[height=\figheightD]{scriptfigures/plume5c_slice_01}&
 \includegraphics[height=\figheightD]{scriptfigures/plume5c_slice_chop_01}&
 \includegraphics[height=\figheightD]{scriptfigures/plume5c_slice_cell_01}\\

 \includegraphics[height=\figheightD]{scriptfigures/plume5c_slice_10}&
 \includegraphics[height=\figheightD]{scriptfigures/plume5c_slice_chop_10}&
 \includegraphics[height=\figheightD]{scriptfigures/plume5c_slice_cell_10}\\

 \includegraphics[height=\figheightD]{scriptfigures/plume5c_slice_30}&
 \includegraphics[height=\figheightD]{scriptfigures/plume5c_slice_chop_30}&
 \includegraphics[height=\figheightD]{scriptfigures/plume5c_slice_cell_30}\\

 &data chopped below 140~\degC&cell centered data\\
 &&&\raisebox{1.0in}[0pt]{\includegraphics[height=7.0in]{figures/colorbar_20_620}}\\
 \end{tabular}
\end{center}
 \caption{Slice file test.}
\label{figslicetest}%
\end{figure}

\begin{figure}[\figoptions]
\begin{center}
\begin{tabular}{rccl}
 1.0 s&
 \includegraphics[height=\figheightD]{scriptfigures/plume5c_vslice_01}&
 \includegraphics[height=\figheightD]{scriptfigures/plume5c_vslicechop_01}\\
 10.0 s&
 \includegraphics[height=\figheightD]{scriptfigures/plume5c_vslice_10}&
 \includegraphics[height=\figheightD]{scriptfigures/plume5c_vslicechop_10}\\
 30.0 s&
 \includegraphics[height=\figheightD]{scriptfigures/plume5c_vslice_30}&
 \includegraphics[height=\figheightD]{scriptfigures/plume5c_vslicechop_30}\\
 &&data chopped below 140~\degC\\
 &&&\raisebox{1.0in}[0pt]{\includegraphics[height=7.0in]{figures/colorbar_20_620}}\\

 \end{tabular}
\end{center}
 \caption{Vector slice file test.}
\label{figvslicetest}%
\end{figure}

\begin{figure}[\figoptions]
\begin{center}
\begin{tabular}{cc}
\includegraphics[height=\figheightD]{scriptfigures/fire_line_fireline_10}&
\includegraphics[height=\figheightD]{scriptfigures/fire_line_fireline_20}\\
10 s&20 s\\

\includegraphics[height=\figheightD]{scriptfigures/fire_line_fireline_30}&
\includegraphics[height=\figheightD]{scriptfigures/fire_line_fireline_40}\\
30 s&40 s\\

 \end{tabular}
\end{center}
 \caption{Fire line test.}
\label{figfirelinetest}%
\end{figure}

\begin{figure}[\figoptions]
\begin{center}
\begin{tabular}{rc}
 1.0 s&
 \includegraphics[height=\figheightD]{scriptfigures/plume5c_smoke_01}\\
 10.0 s&
 \includegraphics[height=\figheightD]{scriptfigures/plume5c_smoke_10}\\
 30.0 s&
 \includegraphics[height=\figheightD]{scriptfigures/plume5c_smoke_30}\\

 \end{tabular}
\end{center}
 \caption{3D smoke file test.}
\label{figsmoketest}%
\end{figure}

\begin{figure}[\figoptions]
\begin{center}
 \centering
\begin{tabular}{c}
\includegraphics[height=\figheightF]{scriptfigures/smoke_sensor_l}\\
\includegraphics[height=\figheightF]{scriptfigures/smoke_sensor_c}\\
\includegraphics[height=\figheightF]{scriptfigures/smoke_sensor_r}\\

 \end{tabular}
\end{center}
\caption[Smoke sensor test.]{Smoke sensor test.
A small white (255,255,255) smokesensor appears in front of a grey (128,128,128) obstacle.
The red dot indicates where the smoke opacity is recorded.
}
\label{figsmokesensor}%
\end{figure}

\begin{figure}[\figoptions]
\begin{center}
 \centering
\begin{tabular}{c}
\includegraphics[width=6.0in]{figures/graysquares2}\\
\includegraphics[width=6.0in]{figures/graysquares3}\\
 \end{tabular}
\end{center}
 \caption[Shade of gray resolution test.]{Shade of gray resolution test.
 The number within each square represents the shade of gray used to color that square,
 0 for black and 255 for white.  Adjacent squares are drawn with nearly equal shades
 testing the ability of sensors such as the eye, computer monitor or the printed page
 to distinguish them.
 }
\label{figgraysquare}%
\end{figure}

\begin{figure}[\figoptions]
\begin{center}
 \centering
\begin{tabular}{m{1in}m{3in}m{3in}}
 &0.1 m grid&0.2 m grid\\
 all planes&\includegraphics[height=\figheightF]{scriptfigures/smoke_test_all}&
 \includegraphics[height=\figheightF]{scriptfigures/smoke_test2_all}\\
 every 2nd plane&\includegraphics[height=\figheightF]{scriptfigures/smoke_test_every2}&
 \includegraphics[height=\figheightF]{scriptfigures/smoke_test2_every2}\\
 every 3rd plane&\includegraphics[height=\figheightF]{scriptfigures/smoke_test_every3}&
  \includegraphics[height=\figheightF]{scriptfigures/smoke_test2_every3}\\
 theoretical&\multicolumn{2}{c}{\includegraphics[height=1.0in]{figures/graysquares}}\\

 \end{tabular}
\end{center}
 \caption[3D smoke file test 2.]{3D smoke file test 2.
 A quantitative test of the smoke opacity calculation in Smokeview.  This test simplifies
  the general case by assuming a uniform distribution of smoke.  Smoke grey levels are computed
  using $grey level (GL) = 255*exp(-K*S*\Delta X)$
  where $K=8700$~m2/kg is the mass extinction value, $S=79.67$~mg/m3 is the soot density
  and $\Delta x$ is the smoke path length.  Path lengths (smoke sensor locations) are chosen to obtain grey levels of 192, 128, 64, 32, 16 and 8.
 }
\label{figsmoketest2}%
\end{figure}


\begin{figure}[\figoptions]
\begin{center}
\begin{tabular}{cc}
 \includegraphics[height=\figheightD]{../SMV_5_User_Guide/scriptfigures/thouse5_smoke_005}&
 \includegraphics[height=\figheightD]{../SMV_5_User_Guide/scriptfigures/thouse5_smoke_gpu_005}\\
 \includegraphics[height=\figheightD]{../SMV_5_User_Guide/scriptfigures/thouse5_smoke_010}&
 \includegraphics[height=\figheightD]{../SMV_5_User_Guide/scriptfigures/thouse5_smoke_gpu_010}\\
 \includegraphics[height=\figheightD]{../SMV_5_User_Guide/scriptfigures/thouse5_smoke_030}&
 \includegraphics[height=\figheightD]{../SMV_5_User_Guide/scriptfigures/thouse5_smoke_gpu_030}\\
 CPU corrected&GPU corrected\\
 \end{tabular}
\end{center}
 \caption[3D smoke file test 3 - GPU Test]{3D smoke file test 3 - GPU Test}
\label{figsmoketest3}%
\end{figure}

\begin{figure}[\figoptions]
\begin{center}
\begin{tabular}{c}
 \includegraphics[height=\figheightE]{../SMV_5_User_Guide/scriptfigures/thouse5_plot3d_step}\\
 step contours\\
 \includegraphics[height=\figheightE]{../SMV_5_User_Guide/scriptfigures/thouse5_plot3d_line}\\
 line contours\\
 \includegraphics[height=\figheightE]{../SMV_5_User_Guide/scriptfigures/thouse5_plot3d_cont}\\
 continuous contours
 \end{tabular}
\end{center}
 \caption{PLOT3D file test}
\label{figPLOT3Dtest}%
\end{figure}
\section{Geometry}
\begin{figure}[\figoptions]
\begin{center}
\begin{tabular}{c}
 \includegraphics[height=\figheightE]{../SMV_5_User_Guide/scriptfigures/thouse5_solid}\\
 solid\\
 \includegraphics[height=\figheightE]{../SMV_5_User_Guide/scriptfigures/thouse5_outline}\\
 outline\\
 \includegraphics[height=\figheightE]{../SMV_5_User_Guide/scriptfigures/thouse5_hidden}\\
 hidden\\

 \end{tabular}
\end{center}
 \caption{Obstacle view test.}
\label{figobstest}%
\end{figure}

\begin{figure}[\figoptions]
\begin{center}
\begin{tabular}{c}
 \includegraphics[height=\figheightE]{../SMV_5_User_Guide/scriptfigures/thouse5_solid}\\
 all vents\\
 \includegraphics[height=\figheightE]{../SMV_5_User_Guide/scriptfigures/thouse5_noopen}\\
 no open vents\\
 \includegraphics[height=\figheightE]{../SMV_5_User_Guide/scriptfigures/thouse5_novents}\\
 no vents\\

 \end{tabular}
\end{center}
 \caption{Vent view test.}
\label{figventest}%
\end{figure}


\section{Other}


\end{document}
