
\chapter{Burning Rate}

This chapter contains a series of validation exercises where the aim is to {\em predict} the burning rate of the fuel. Most of the simulations included in the previous chapters involved
a {\em specified} burning or heat release rate. Here, the objective is to apply measured thermophysical properties of the material and predict its burning rate, either with a
specified heat flux or as a free burn.



\section{FAA Polymers}

The U.S.~Federal Aviation Administration (FAA) has studied various plastics that are commonly used aboard commercial aircraft.

This section presents measured properties of various polymers and the numerical predictions of their mass loss and/or burning rates under constant heat heating.
Two types of experiments are considered. First, the NIST Gasification Apparatus is used to measure the mass loss rate of non-burning samples in a nitrogen atmosphere.
Second, the standard Cone Calorimeter~\cite{conecal} is used to measure the heat release rate of materials in a normal atmosphere.
When just the mass loss rate of a non-burning sample has been measured, FDS is run in ``solid phase only'' mode; that is, a 1-D heat conduction calculation is performed in a
single grid cell. The result is the predicted mass loss rate as a function of time. To simulate a cone calorimeter experiment, FDS simulates the burning of a 10~cm by 10~cm sample with
a specified heat flux to represent the effect of the cone heater. The cone itself is not included in the simulation. As the sample burns, FDS predicts the additional radiative and
convective heating of the sample as a result of the fire.

In general, the burning/gasification rate of a charring polymer is more difficult to predict than a non-charring one because there are more parameters that need to be measured and
more complicated behavior, like intumescence, need to be considered.


\newpage

\subsection{Non-Charring Polymers in the Gasification Apparatus}

A non-charring polymer is considered one of the easier solids to model because it typically involves only a single, first order reaction that converts solid plastic to fuel vapor.
No residue is formed and the plastic is completely pyrolyzed. Table~\ref{FAA_Properties}
lists nine parameters for each polymer studied. These values have been input directly into FDS, and the predicted mass loss rates are compared with measured values from the NIST
Gasification Apparatus, a device that pyrolyzes the solid in a nitrogen environment to prevent combustion of fuel gases. The results are shown in Fig.~\ref{FAA_Polymers}. The exposing
heat flux was 52~kW/m$^2$. A 1~cm layer of insulation was placed under the sample. Its properties are given in Ref.~\cite{Stoliarov:CF2009}.


\begin{table}[h!]
\caption[FAA non-charring polymer properties.]{Input parameters for FAA Polymers non-charring samples. Courtesy S.~Stoliarov, M.~McKinnon and J.~Li, University of Maryland.
See Section~\ref{glossary} for an explanation of terms.}
\begin{center}
\begin{tabular}{|l|c|c|c|c|c|l|l|}
\hline
Property                    & Units         & HDPE                  & HIPS                  & PMMA                  & Unc. (\%) & Method                &  Ref.                         \\ \hline \hline
Density                     & kg/m$^3$      & 860                   & 950                   & 1100                  & 5         & Direct                &  \cite{Stoliarov:CF2009}      \\ \hline
Conductivity                & W/m/K         & 0.29                  & 0.22                  & 0.20                  & 15        & Transient Line Source &  \cite{Stoliarov:CF2009}      \\ \hline
Specific Heat               & kJ/kg/K       & 3.5                   & 2.0                   & 2.2                   & 15        & DSC                   &  \cite{Stoliarov:PDS2008}     \\ \hline
Emissivity                  &               & 0.92                  & 0.86                  & 0.85                  & 20        & Integrating Sphere    &  \cite{Hallman:PES1974}       \\ \hline
Absorption Coef.            & m$^{-1}$      & 1300                  & 2700                  & 2700                  & 50        & FTIR                  &  \cite{Tsilingiris:ECM2003}   \\ \hline
Pre-Exp.~Factor             & s$^{-1}$      & $4.8 \times 10^{22}$  & $1.2 \times 10^{16}$  & $8.5 \times 10^{12}$  & 50        & TGA                   &  \cite{Stoliarov:CF2009}      \\ \hline
Activation Energy           & kJ/kmol       & $3.49 \times 10^{5}$  & $2.47 \times 10^{5}$  & $1.88 \times 10^{5}$  & 3         & TGA                   &  \cite{Stoliarov:CF2009}      \\ \hline
Heat of Reaction            & kJ/kg         & 920                   & 1000                  & 870                   & 15        & DSC                   &  \cite{Stoliarov:PDS2008}     \\ \hline
\end{tabular}
\end{center}
\label{FAA_Properties}
\end{table}



\begin{figure}[h!]
\begin{tabular*}{\textwidth}{l@{\extracolsep{\fill}}r}
\includegraphics[height=2.2in]{FIGURES/FAA_Polymers/FAA_Polymers_HDPE} &
\includegraphics[height=2.2in]{FIGURES/FAA_Polymers/FAA_Polymers_HIPS} \\
\includegraphics[height=2.2in]{FIGURES/FAA_Polymers/FAA_Polymers_PMMA}&
\end{tabular*}
\caption[Results of FAA Polymers, non-charring, comparison]{Comparison of predicted and measured mass loss rates for three non-charring polymers exposed to a heat flux of 52~kW/m$^2$ in a
nitrogen environment.}
\label{FAA_Polymers}
\end{figure}

\clearpage

\subsection{Polycarbonate (PC) in the Cone Calorimeter}

Table~\ref{Properties_PC} lists the measured properties of polycarbonate. These values have been input directly into FDS, and the predicted heat release rates
are compared with measured values from the Cone
Calorimeter. The results for samples of various thicknesses and imposed heat fluxes are shown in Fig.~\ref{HRR_PC}.
A 1~cm layer of Kaowool insulation was placed under the sample. Its properties are given in Ref.~\cite{Stoliarov:CF2010}.
It is assumed that the polymer undergoes a single step reaction
that forms fuel gas and char.



\begin{table}[h!]
\caption[Properties of polycarbonate (PC).]{Properties of polycarbonate (PC). Courtesy S.~Stoliarov, University of Maryland. See Section~\ref{glossary} for an explanation of terms.}
\begin{center}
\begin{tabular}{|l|c|c|c|c|c|l|l|}
\hline
Property                    & Units         & Value                             & Method                &  Reference                                \\ \hline \hline
Polymer Density             & kg/m$^3$      & 1180 $\pm$ 60                     & Direct                &  \cite{Stoliarov:CF2010}                  \\ \hline
Polymer Conductivity        & W/m/K         & 0.22 $\pm$ 0.03                   & Literature            &  \cite{Stoliarov:CF2010}                  \\ \hline
Polymer Specific Heat       & kJ/kg/K       & 1.9 $\pm$ 0.3                     & DSC                   &  \cite{Stoliarov:PDS2008}                 \\ \hline
Polymer Emissivity          &               & 0.90 $\pm$ 0.05                   & Integrating  Sphere   &  \cite{Hallman:PES1974}                   \\ \hline
Polymer Absorption Coef.    & m$^{-1}$      & 1770 $\pm$ 590                    & FTIR                  &  \cite{Tsilingiris:ECM2003}               \\ \hline
Char Density                & kg/m$^3$      & 248                               & Cone Calorimeter      &  \cite{Stoliarov:CF2010}                  \\ \hline
Char Conductivity           & W/m/K         & 0.37                              & Cone Calorimeter      &  \cite{Stoliarov:CF2010}                  \\ \hline
Char Specific Heat          & kJ/kg/K       & 1.72 $\pm$ 0.17                   & Pulsed Current        &  \cite{Stoliarov:CF2010,Matsumoto:1996}   \\ \hline
Char Emissivity             &               & 0.85 $\pm$ 0.05                   & Pulsed Current        &  \cite{Stoliarov:CF2010,Matsumoto:1996}   \\ \hline
Char Absorption Coef.       & m$^{-1}$      & Opaque                            & Assumption            &  \cite{Stoliarov:CF2010}                  \\ \hline
Pre-Exp.~Factor             & s$^{-1}$      & $(1.9 \pm 1.1) \times 10^{18}$    & TGA                   &  \cite{Stoliarov:CF2010}                  \\ \hline
Activation Energy           & kJ/kmol       & $(2.95 \pm 0.06) \times 10^{5}$   & TGA                   &  \cite{Stoliarov:CF2010}                  \\ \hline
Heat of Reaction            & kJ/kg         & 830 $\pm$ 140                     & DSC                   &  \cite{Stoliarov:PDS2008}                 \\ \hline
Heat of Combustion          & kJ/kg         & 25600 $\pm$ 130                   & MCC                   &  \cite{Stoliarov:CF2010}                  \\ \hline
Combustion Efficiency       &               & 0.84 $\pm$ 0.03                   & Cone Calorimeter      &  \cite{Stoliarov:CF2010}                  \\ \hline
\end{tabular}
\end{center}
\label{Properties_PC}
\end{table}

\begin{figure}[p]
\begin{tabular*}{\textwidth}{l@{\extracolsep{\fill}}r}
\includegraphics[height=2.2in]{FIGURES/FAA_Polymers/FAA_Polymers_PC_6_75} &
\includegraphics[height=2.2in]{FIGURES/FAA_Polymers/FAA_Polymers_PC_6_92} \\
\includegraphics[height=2.2in]{FIGURES/FAA_Polymers/FAA_Polymers_PC_6_50} &
\includegraphics[height=2.2in]{FIGURES/FAA_Polymers/FAA_Polymers_PC_3_75} \\
\includegraphics[height=2.2in]{FIGURES/FAA_Polymers/FAA_Polymers_PC_9_75} &
\end{tabular*}
\caption[Heat release rate of polycarbonate (PC).]{Comparison of predicted and measured heat release rates for polycarbonate (PC).}
\label{HRR_PC}
\end{figure}

\clearpage


\subsection{Poly(vinyl chloride) (PVC) in the Cone Calorimeter}

Table~\ref{Properties_PVC} lists the measured properties of poly(vinyl chloride). These values have been input directly into FDS, and the predicted heat release rates
are compared with measured values from the Cone
Calorimeter. The results for samples of various thicknesses and imposed heat fluxes are shown in Fig.~\ref{HRR_PVC}.
A 1~cm layer of Kaowool insulation was placed under the sample. Its properties are given in Ref.~\cite{Stoliarov:CF2010}.

It is assumed that the polymer decomposes via a two-step reaction:
\begin{eqnarray}
   \hbox{Polymer} &\to& \hbox{Char 1} + \hbox{Gas 1}  \\
   \hbox{Char 1}  &\to& \hbox{Char 2} + \hbox{Gas 2}
\end{eqnarray}


\begin{table}[h!]
\caption[Properties of poly(vinyl chloride) (PVC).]{Properties of poly(vinyl chloride) (PVC). Courtesy S.~Stoliarov, University of Maryland. See Section~\ref{glossary} for an explanation of terms.}
\begin{center}
\begin{tabular}{|l|c|c|c|c|c|l|l|}
\hline
Property                    & Units         & Value                             & Method                    &  Reference                                \\ \hline \hline
Polymer Density             & kg/m$^3$      & 1430 $\pm$ 70                     & Direct                    &  \cite{Stoliarov:CF2010}                  \\ \hline
Polymer Conductivity        & W/m/K         & 0.17 $\pm$ 0.01                   & Literature                &  \cite{Stoliarov:CF2010}                  \\ \hline
Polymer Specific Heat       & kJ/kg/K       & 1.55 $\pm$ 0.25                   & DSC                       &  \cite{Stoliarov:PDS2008}                 \\ \hline
Polymer Emissivity          &               & 0.90 $\pm$ 0.05                   & Integrating Sphere        &  \cite{Hallman:PES1974}                   \\ \hline
Polymer Absorption Coef.    & m$^{-1}$      & 2145 $\pm$ 715                    & FTIR                      &  \cite{Tsilingiris:ECM2003}               \\ \hline
Char 1 Density              & kg/m$^3$      & 629                               & Assumed Constant Volume   &  \cite{Stoliarov:CF2010}                  \\ \hline
Char 1 Conductivity         & W/m/K         & 0.17                              & Assumed Same as Polymer   &  \cite{Stoliarov:CF2010}                  \\ \hline
Char 1 Specific Heat        & kJ/kg/K       & 1.55 $\pm$ 0.25                   & Assumed Same as Polymer   &  \cite{Stoliarov:CF2010}                  \\ \hline
Char 1 Emissivity           &               & 0.90 $\pm$ 0.05                   & Assumed Same as Polymer   &  \cite{Stoliarov:CF2010}                  \\ \hline
Char 1 Absorption Coef.     & m$^{-1}$      & 2453                              & Fit to Cone Data          &  \cite{Stoliarov:CF2010}                  \\ \hline
Char 2 Density              & kg/m$^3$      & 296                               & Assumed Constant Volume   &  \cite{Stoliarov:CF2010}                  \\ \hline
Char 2 Conductivity         & W/m/K         & 0.26                              & Fit to Cone Data          &  \cite{Stoliarov:CF2010}                  \\ \hline
Char 2 Specific Heat        & kJ/kg/K       & 1.72 $\pm$ 0.17                   & Pulsed Current            &  \cite{Stoliarov:CF2010,Matsumoto:1996}   \\ \hline
Char 2 Emissivity           &               & 0.85 $\pm$ 0.05                   & Pulsed Current            &  \cite{Stoliarov:CF2010,Matsumoto:1996}   \\ \hline
Char 2 Absorption Coef.     & m$^{-1}$      & Opaque                            & Assumption                &  \cite{Stoliarov:CF2010}                  \\ \hline
Reac 1 Pre-Exp.~Factor      & s$^{-1}$      & $(1.4 \pm 0.8) \times 10^{33}$    & TGA                       &  \cite{Stoliarov:CF2010}                  \\ \hline
Reac 1 Activation Energy    & kJ/kmol       & $(3.67 \pm 0.07) \times 10^{5}$   & TGA                       &  \cite{Stoliarov:CF2010}                  \\ \hline
Reac 1 Char Yield           &               & $0.44 \pm 0.01$                   & TGA                       &  \cite{Stoliarov:CF2010}                  \\ \hline
Reac 1 Heat of Reaction     & kJ/kg         & 170 $\pm$ 17                      & DSC                       &  \cite{Stoliarov:PDS2008}                 \\ \hline
Gas 1 Heat of Combustion    & kJ/kg         & 2700 $\pm$ 300                    & MCC                       &  \cite{Stoliarov:CF2010}                  \\ \hline
Gas 1 Combustion Efficiency &               & 0.75 $\pm$ 0.03                   & Cone Calorimeter          &  \cite{Stoliarov:CF2010}                  \\ \hline
Reac 2 Pre-Exp.~Factor      & s$^{-1}$      & $(3.5 \pm 2.1) \times 10^{12}$    & TGA                       &  \cite{Stoliarov:CF2010}                  \\ \hline
Reac 2 Activation Energy    & kJ/kmol       & $(2.07 \pm 0.04) \times 10^{5}$   & TGA                       &  \cite{Stoliarov:CF2010}                  \\ \hline
Reac 2 Char Yield           &               & $0.47 \pm 0.01$                   & TGA                       &  \cite{Stoliarov:CF2010}                  \\ \hline
Reac 2 Heat of Reaction     & kJ/kg         & 1200 $\pm$ 900                    & DSC                       &  \cite{Stoliarov:PDS2008}                 \\ \hline
Gas 2 Heat of Combustion    & kJ/kg         & 36500 $\pm$ 1800                  & MCC                       &  \cite{Stoliarov:CF2010}                  \\ \hline
Gas 2 Combustion Efficiency &               & 0.75 $\pm$ 0.03                   & Cone Calorimeter          &  \cite{Stoliarov:CF2010}                  \\ \hline
\end{tabular}
\end{center}
\label{Properties_PVC}
\end{table}

\begin{figure}[p]
\begin{tabular*}{\textwidth}{l@{\extracolsep{\fill}}r}
\includegraphics[height=2.2in]{FIGURES/FAA_Polymers/FAA_Polymers_PVC_6_75} &
\includegraphics[height=2.2in]{FIGURES/FAA_Polymers/FAA_Polymers_PVC_6_92} \\
\includegraphics[height=2.2in]{FIGURES/FAA_Polymers/FAA_Polymers_PVC_6_50} &
\includegraphics[height=2.2in]{FIGURES/FAA_Polymers/FAA_Polymers_PVC_3_75} \\
\includegraphics[height=2.2in]{FIGURES/FAA_Polymers/FAA_Polymers_PVC_9_75} &
\end{tabular*}
\caption[Heat release rate of poly(vinyl chloride) (PVC).]{Comparison of predicted and measured heat release rates for poly(vinyl chloride) (PVC).}
\label{HRR_PVC}
\end{figure}

\clearpage



\subsection{Poly(aryl ether ether ketone)) (PEEK) in the Cone Calorimeter}

Table~\ref{Properties_PEEK} lists the measured properties of poly(aryl ether ether ketone). Its trade name is VICTREX PEEK 450G.
It has been thoroughly dried. Its property values have been input directly into FDS, and the predicted heat release rates
are compared with measured values from the Cone
Calorimeter. The results for 3.9~mm samples at imposed heat fluxes of 50~kW/m$^2$, 70~kW/m$^2$, and 90~kW/m$^2$ are shown in Fig.~\ref{HRR_PEEK}.
A 1~cm layer of Kaowool insulation was placed under the sample. Its properties are given in Ref.~\cite{Stoliarov:CF2010}.

It is assumed that the polymer decomposes via a four-step reaction:
\begin{eqnarray}
   \hbox{Polymer} &\to& \hbox{Char 1} + \hbox{Gas 1}  \\
   \hbox{Char 1}  &\to& \hbox{Char 2} + \hbox{Gas 2}  \\
   \hbox{Char 2}  &\to& \hbox{Char 3} + \hbox{Gas 2}  \\
   \hbox{Char 3}  &\to& \hbox{Gas 2}
\end{eqnarray}
It is also assumed that the gaseous fuel molecule is C$_{19}$H$_{12}$O$_3$.

\begin{figure}[h]
\begin{tabular*}{\textwidth}{l@{\extracolsep{\fill}}r}
\includegraphics[height=2.2in]{FIGURES/FAA_Polymers/FAA_Polymers_PEEK_50} &
\includegraphics[height=2.2in]{FIGURES/FAA_Polymers/FAA_Polymers_PEEK_70} \\
\multicolumn{2}{c}{\includegraphics[height=2.2in]{FIGURES/FAA_Polymers/FAA_Polymers_PEEK_90}}
\end{tabular*}
\caption[Heat release rate of poly(aryl ether ether ketone) (PEEK).]{Comparison of predicted and measured heat release rates for poly(aryl ether ether ketone) (PEEK).}
\label{HRR_PEEK}
\end{figure}


\begin{table}[h!]
\caption[Properties of poly(aryl ether ether ketone) (PEEK).]{Properties of poly(aryl ether ether ketone) (PEEK). Courtesy E.~Oztekin, U.S.~FAA and S.~Stoliarov,
University of Maryland. See Section~\ref{glossary} for an explanation of terms.}
\begin{center}
\begin{tabular}{|l|c|c|c|c|c|l|l|}
\hline
Property                    & Units         & Value                             & Method                    &  Reference                              \\ \hline \hline
Polymer Density             & kg/m$^3$      & 1300                              & Direct                    &  \cite{Oztekin:CF2012}                  \\ \hline
Polymer Conductivity        & W/m/K         & 0.28                              & Fit to cone data          &  \cite{Oztekin:CF2012}                  \\ \hline
Polymer Specific Heat       & kJ/kg/K       & 2.05                              & Fit to cone data          &  \cite{Oztekin:CF2012}                  \\ \hline
Polymer Emissivity          &               & 0.90                              & Fit to cone data          &  \cite{Oztekin:CF2012}                  \\ \hline
Polymer Absorption Coef.    & m$^{-1}$      & 1690                              & Fit to cone data          &  \cite{Oztekin:CF2012}                  \\ \hline
Char 1 Density              & kg/m$^3$      & 810                               & Assumed constant volume   &  \cite{Oztekin:CF2012}                  \\ \hline
Char 1 Conductivity         & W/m/K         & 0.37                              & Fit to cone data          &  \cite{Oztekin:CF2012}                  \\ \hline
Char 1 Specific Heat        & kJ/kg/K       & 0.24                              & Assumed                   &  \cite{Oztekin:CF2012}                  \\ \hline
Char 1 Emissivity           &               & 1                                 & Assumed                   &  \cite{Oztekin:CF2012}                  \\ \hline
Char 1 Absorption Coef.     & m$^{-1}$      & 81000                             & Assumed opaque            &  \cite{Oztekin:CF2012}                  \\ \hline
Char 2 Density              & kg/m$^3$      & 710                               & Assumed constant volume   &  \cite{Oztekin:CF2012}                  \\ \hline
Char 2 Conductivity         & W/m/K         & 0.37                              & Fit to cone data          &  \cite{Oztekin:CF2012}                  \\ \hline
Char 2 Specific Heat        & kJ/kg/K       & 0.27                              & Assumed                   &  \cite{Oztekin:CF2012}                  \\ \hline
Char 2 Emissivity           &               & 1                                 & Assumed                   &  \cite{Oztekin:CF2012}                  \\ \hline
Char 2 Absorption Coef.     & m$^{-1}$      & 71000                             & Assumed opaque            &  \cite{Oztekin:CF2012}                  \\ \hline
Reac 1 Pre-Exp.~Factor      & s$^{-1}$      & $1.0 \times 10^{32}$              & TGA                       &  \cite{Oztekin:CF2012}                  \\ \hline
Reac 1 Activation Energy    & kJ/kmol       & $5.57 \times 10^5$                & TGA                       &  \cite{Oztekin:CF2012}                  \\ \hline
Reac 1 Char Yield           &               & 0.62                              & TGA                       &  \cite{Oztekin:CF2012}                  \\ \hline
Reac 1 Heat of Reaction     & kJ/kg         & 350                               & Fit to cone data          &  \cite{Oztekin:CF2012}                  \\ \hline
Gas 1 Heat of Combustion    & kJ/kg         & 16000                             & Cone calorimetry          &  \cite{Oztekin:CF2012}                  \\ \hline
Gas 1 Combustion Efficiency &               & 1                                 & Assumed                   &  \cite{Oztekin:CF2012}                  \\ \hline
Reac 2 Pre-Exp.~Factor      & s$^{-1}$      & $1.0 \times 10^3$                 & TGA                       &  \cite{Oztekin:CF2012}                  \\ \hline
Reac 2 Activation Energy    & kJ/kmol       & $8.9 \times 10^4$                 & TGA                       &  \cite{Oztekin:CF2012}                  \\ \hline
Reac 2 Char Yield           &               & 0.88                              & TGA                       &  \cite{Oztekin:CF2012}                  \\ \hline
Reac 2 Heat of Reaction     & kJ/kg         & 0                                 & Assumed                   &  \cite{Oztekin:CF2012}                  \\ \hline
Gas 2 Heat of Combustion    & kJ/kg         & 27000                             & Cone Calorimetry          &  \cite{Oztekin:CF2012}                  \\ \hline
Gas 2 Combustion Efficiency &               & 1                                 & Assumed                   &  \cite{Oztekin:CF2012}                  \\ \hline
Reac 3 Pre-Exp.~Factor      & s$^{-1}$      & $1.0 \times 10^5$                 & TGA                       &  \cite{Oztekin:CF2012}                  \\ \hline
Reac 3 Activation Energy    & kJ/kmol       & $1.47 \times 10^5$                & TGA                       &  \cite{Oztekin:CF2012}                  \\ \hline
Reac 3 Char Yield           &               & 0.88                              & TGA                       &  \cite{Oztekin:CF2012}                  \\ \hline
Reac 3 Heat of Reaction     & kJ/kg         & 0                                 & Assumed                   &  \cite{Oztekin:CF2012}                  \\ \hline
Reac 4 Pre-Exp.~Factor      & s$^{-1}$      & $1.0 \times 10^3$                 & TGA                       &  \cite{Oztekin:CF2012}                  \\ \hline
Reac 4 Activation Energy    & kJ/kmol       & $1.29 \times 10^5$                & TGA                       &  \cite{Oztekin:CF2012}                  \\ \hline
Reac 4 Char Yield           &               & 0                                 & TGA                       &  \cite{Oztekin:CF2012}                  \\ \hline
Reac 4 Heat of Reaction     & kJ/kg         & 0                                 & Assumed                   &  \cite{Oztekin:CF2012}                  \\ \hline
\end{tabular}
\end{center}
\label{Properties_PEEK}
\end{table}


\clearpage


\subsection{Poly(butylene terephtalate) (PBT) in the Gasification Apparatus and Cone Calorimeter}

Samples of poly(butylene terephtalate) (PBT) have been burned without oxygen in the Gasification Apparatus and
with oxygen in the Cone Calorimeter. The properties of PBT are listed in Table~\ref{Properties_PBT}. It is assumed that the polymer undergoes a single step reaction
that forms fuel gas and no char.


\begin{table}[h!]
\caption[Properties of poly(butylene terephtalate) (PBT).]{Properties of poly(butylene terephtalate) (PBT). Courtesy S.~Stoliarov, University of Maryland.
See Section~\ref{glossary} for an explanation of terms.}
\begin{center}
\begin{tabular}{|l|c|c|c|c|}
\hline
Property                & Units     & Value                             & Method                & Reference                     \\ \hline \hline
Density                 & kg/m$^3$  & 1300 $\pm$ 70                     & Direct                & \cite{Kempel:1}               \\ \hline
Specific Heat           & kJ/kg/K   & 2.23 $\pm$ 0.34                   & DSC                   & \cite{Kempel:1}               \\ \hline
Conductivity            & W/m/K     & 0.29 $\pm$ 0.05                   & Transient Line Source & \cite{Kempel:1}               \\ \hline
Emissivity              &           & 0.88 $\pm$ 0.05                   & FTIR                  & \cite{Linteris:2}             \\ \hline
Absorbtion Coefficient  & m$^{-1}$  & 2561 $\pm$ 140                    & FTIR                  & \cite{Linteris:2}             \\ \hline
Pre-Exp. Factor         & s$^{-1}$  & $(2.49 \pm 0.62) \times 10^{14}$  & TGA                   & \cite{Kempel:1}               \\ \hline
Activation Energy       & kJ/kmol   & $(2.12 \pm 0.53) \times 10^{5}$   & TGA                   & \cite{Kempel:1}               \\ \hline
Heat of Reaction        & kJ/kg     & 507                               & DSC, Literature       & \cite{Kempel:1,Lyon:Ency2005} \\ \hline
Heat of Combustion      & kJ/kg     & 19500                             & Cone Calorimeter      & \cite{Kempel:1}               \\ \hline
Combustion Efficiency   &           & 1                                 & Assumption            & \cite{Kempel:1}               \\ \hline
\end{tabular}
\end{center}
\label{Properties_PBT}
\end{table}


\begin{figure}[h!]
\begin{tabular*}{\textwidth}{l@{\extracolsep{\fill}}r}
 &
\includegraphics[height=2.2in]{FIGURES/FAA_Polymers/FAA_Polymers_PBT_35} \\
\includegraphics[height=2.2in]{FIGURES/FAA_Polymers/FAA_Polymers_PBT} &
\includegraphics[height=2.2in]{FIGURES/FAA_Polymers/FAA_Polymers_PBT_50} \\
 &
\includegraphics[height=2.2in]{FIGURES/FAA_Polymers/FAA_Polymers_PBT_70}
\end{tabular*}
\caption[Mass loss rate of poly(butylene terephtalate) (PBT).]{Comparison of predicted and measured mass loss rates for poly(butylene terephtalate) (PBT)
in both the Gasification Apparatus and Cone Calorimeter.}
\label{HRR_PBT}
\end{figure}


\clearpage


\subsection{PBT with Glass Fibers (PBT-GF) in the Gasification Apparatus and Cone Calorimeter}

Samples of poly(butylene terephtalate) (PBT), blended with 30~\% by mass glass fibers, have been burned without oxygen in the Gasification Apparatus and
with oxygen in the Cone Calorimeter. The properties of PBT-GF are listed in Table~\ref{Properties_PBT-GF}. It is assumed that the polymer undergoes a single step reaction
that forms fuel gas and char.

\begin{table}[h!]
\caption[Properties of poly(butylene terephtalate) with glass fibers (PBT-GF).]{Properties of poly(butylene terephtalate) with glass fibers (PBT-GF).
Courtesy S.~Stoliarov, University of Maryland. See Section~\ref{glossary} for an explanation of terms.}
\begin{center}
\begin{tabular}{|l|c|c|c|c|}
\hline
Property                &      Units    &      Value                        & Method                    & Reference                     \\ \hline \hline
Polymer Density         &     kg/m$^3$  &  1520 $\pm$ 80                    & Direct                    & \cite{Kempel:1}               \\ \hline
Polymer Specific Heat   &    kJ/kg/K    & 1.68 $\pm$ 0.26                   & DSC                       & \cite{Kempel:1}               \\ \hline
Polymer Conductivity    &      W/m/K    & 0.36 $\pm$ 0.06                   & Transient Line Source     & \cite{Kempel:1}               \\ \hline
Polymer Emissivity      &               & 0.87 $\pm$ 0.05                   & FTIR                      & \cite{Linteris:2}             \\ \hline
Polymer Absorbtion Coef.&  m$^{-1}$     & 2860 $\pm$ 150                    & FTIR                      & \cite{Linteris:2}             \\ \hline
Char Density            &     kg/m$^3$  &        482                        & Assumed Constant Volume   & \cite{Kempel:1}               \\ \hline
Char Specific Heat      &    kJ/kg/K    &        0.85                       & Literature                & \cite{SCHOTT}                 \\ \hline
Char Conductivity       &      W/m/K    & 0.07 $\pm$ 0.02                   & Laser Flash               & \cite{Kempel:1}               \\ \hline
Char Emissivity         &               &       0.85                        & Literature                & \cite{Braeuer:1}              \\ \hline
Char Absorbtion Coef.   &      m$^{-1}$ &       10000                       & Estimated                 & \cite{Kempel:1}               \\ \hline
Pre-Exp. Factor         &      s$^{-1}$ & $(2.49 \pm 0.63) \times 10^{14}$  & TGA                       & \cite{Kempel:1}               \\ \hline
Activation Energy       &    kJ/kmol    & $(2.12 \pm 0.53) \times 10^{5}$   & TGA                       & \cite{Kempel:1}               \\ \hline
Heat of Reaction        &      kJ/kg    &        355                        & DSC, Literature           & \cite{Kempel:1,Lyon:Ency2005} \\ \hline
Heat of Combustion      &      kJ/kg    & 19500                             & Cone Calorimeter          & \cite{Kempel:1}               \\ \hline
Char Yield              &               & 0.32 $\pm$ 0.05                   & Gasification Device       & \cite{Kempel:1}               \\ \hline
Combustion Efficiency   &               &          1                        & Assumption                & \cite{Kempel:1}               \\ \hline
\end{tabular}
\end{center}
\label{Properties_PBT-GF}
\end{table}

\begin{figure}[h!]
\begin{tabular*}{\textwidth}{l@{\extracolsep{\fill}}r}
 &
\includegraphics[height=2.2in]{FIGURES/FAA_Polymers/FAA_Polymers_PBTGF_35} \\
\includegraphics[height=2.2in]{FIGURES/FAA_Polymers/FAA_Polymers_PBTGF} &
\includegraphics[height=2.2in]{FIGURES/FAA_Polymers/FAA_Polymers_PBTGF_50} \\
 &
\includegraphics[height=2.2in]{FIGURES/FAA_Polymers/FAA_Polymers_PBTGF_70}
\end{tabular*}
\caption[Mass loss rate of poly(butylene terephtalate) with glass fibers (PBT-GF).]
{Comparison of predicted and measured mass loss rates for poly(butylene terephtalate) with glass fibers (PBT-GF) in both the Gasification Apparatus and Cone Calorimeter.}
\label{HRR_PBTGF}
\end{figure}



\clearpage

\section{Glossary of Terms}
\label{glossary}

\begin{description}
\item[Assumed Constant Volume:] The sample was assumed to have a constant volume. With volume known and mass determined density can be found.
\item[Assumed Same as Polymer:] The material properties of the product were assumed to be the same as the known properties of the polymer.
\item[Assumption:]  Characteristics were assumed from known properties in similar materials.
\item[Cone Calorimeter] (ASTM E 1354 \cite{conecal}) The Cone Calorimeter exposes a small sample to a constant external radiant heat flux simulating exposure of the sample to a large scale fire. The device records mass loss data along with heat release data through oxygen consumption calorimetry. From this a variety of heat release related properties can be found including heat of combustion.
\item[Direct:]  Direct measurement of densities is performed by measuring the dimensions and mass of the sample.
\item[DSC:] (ASTM E 2070 \cite{diffscancal}) A Differential Scanning Calorimeter precisely raises the temperature of a small sample of material at a constant rate. This coupled with knowledge of heat absorbed by the sample allows for the calculation of the specific heat function of a material as well as heats of reaction and phase change.
\item[Estimated:] Characteristics were approximated based on known properties in similar materials.
\item[Fit to Cone Data:] Property was established by fitting a model to experimental values obtained in the Cone Calorimeter.
\item[FTIR:] Fourier Transform Infrared Spectroscopy uses a spectrometer to simultaneously characterize the absorption of all frequencies of infrared light. In testing a sample is exposed to infrared light and a detector records light that has passed through the sample. A Fourier transform of detector measurement is then translated into absorption information.
\item[Gasification Apparatus:] Similar to the Cone Calorimeter however flaming is prevented. This is done typically through the introduction of inert purge gases.
\item[Integrating Sphere:] (ASTM E 1175 \cite{intgsphere}) An Integrating Sphere, or an Ulbricht Sphere, is a hollow cavity whose interior has a high diffuse reflectivity. A sample placed inside the sphere is exposed to incident radiation and reflectivity measured. Emissivity can be determined from this information. The standard above is for measurement of Solar reflectivity, and was not necessarily precisely followed.
\item[Laser Flash:] (ASTM E1461 \cite{laserflash}) In the Laser Flash Method one surface of a sample is rapidly heated using a single pulse from a laser. Heat sensors on the opposite side of the sample record the arrival of the resulting temperature disturbance. From this thermal diffusivity/thermal conductivity can be calculated.
\item[Literature:] Results were found within previously published literature.
\item[MCC:] (ASTM D 7309 \cite{microcc}) The Microscale Combustion Calorimeter (MCC)  rapidly pyrolizes a milligram size sample in an inert atmosphere. The pyrolyzate is then exposed to an abundance of oxygen.  Heat release history is obtained from oxygen consumption. Similar to TGA with  heat release recorded rather than mass loss rate.
\item[Pulsed Current:] Can refer to different types of tests. Generally, a sample is positioned between two electrodes in a sealed chamber with an inert atmosphere. The sample is heated through pulses of current. Measurements of the sample and the chamber can give information regarding specific heat, emmisivity, or other material properties.
\item[TGA:] (ASTM E 1131 \cite{thermalga}) In Thermal Gravimetric Analysis (TGA)  a small sample is heated at uniform rate, generally in an Nitrogen (N$_2$) atmosphere. The percentage weight loss of the sample is recorded relative to the sample's temperature. Rate constants can then be fitted to the data. Similar to MCC with mass loss recorded instead of heat release.
\item[Transient Line Source:] (ASTM D 5930 \cite{transline}) The Transient Line Source method records temperature of a single point at a fixed distance in a sample over time using a probe. Given knowledge of the heat exposure of the sample the thermal conductivity can be found from the slope of the recorded data.
\end{description}

\newpage
