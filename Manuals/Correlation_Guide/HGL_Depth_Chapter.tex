% !TEX root = Correlation_Guide.tex

\chapter{HGL Depth}
\label{HGL_Depth_Chapter}

The HGL depth is defined as the distance between the ceiling and the HGL height.

\section{ASET Method}

\subsection*{Description}

For a compartment with no ventilation (closed doors) and constant HRR, the available safe egress time (ASET)~\cite{Walton:1}
correlation predicts that the HGL height, $z$~(\si{m}), is given by~\cite{SFPE:Milke}
\be
A\sb{s} \frac{dz}{dt} = \frac{dV\sb{ul}}{dt} = \dot V\sb{ul}
\label{eq:ASET_1}
\ee
where $A\sb{s}$ is the area of the boundary surfaces~(\si{m^2}), and $V\sb{ul}$ is the volume of the HGL~(\si{m^3}).
The change in volume of the upper layer, $\dot V\sb{ul}$~(\si{m^3/s}), is given by
\be
\dot V\sb{ul} = \dot V\sb{exp} + \dot V\sb{ent}
\label{eq:ASET_2}
\ee
The volumetric expansion rate, $\dot V\sb{exp}$~(\si{m^3/s}), is given by~\cite{SFPE:Mowrer}
\be
\dot V\sb{exp} = \frac{\dot Q\sb{net}}{\rho\sb{g} c\sb{p} T\sb{g}} \approx \frac{(1 - \chi\sb{l}) \dot Q\sb{f}}{353}
\label{eq:ASET_3}
\ee
where $\dot Q\sb{net}$ and $\dot Q\sb{f}$ are the net and actual HRRs~(\si{kW}), respectively, $\rho\sb{g}$, $c\sb{p}$ and $T\sb{g}$ are the density~(\si{kg/m^3}), specific heat~(\si{kJ/(kg.K)}), and temperature~(\si{K}) of air in the HGL, respectively, and $\chi\sb{l}$ is the heat loss fraction to the enclosure boundaries~(-).
The volumetric entrainment rate, $\dot V\sb{ent}$~(\si{m^3/s}), is given by~\cite{Zukoski:1981}
\be
\dot V\sb{ent} = k\sb{v} \dot Q^{1/3} z^{5/3} = \frac{0.21}{K\sb{f}} \left( \frac{g}{\rho_\infty T_\infty} \right)^{1/3} (K\sb{f} \dot Q)^{1/3} (z - z\sb{f})^{5/3}
\label{eq:ASET_4}
\ee
where $k\sb{v}$ is the volumetric entrainment coefficient, $g$ is the acceleration due to gravity~(\si{m/s^2}), $\rho_\infty$ and $T_\infty$ are the density~(\si{kg/m^3}) and temperature~(\si{K}) of ambient air, respectively, $K\sb{f}$ is the location factor~(-), and $z\sb{f}$ is the fuel height~(\si{m}). The location factor has a value of 1, 2, or 4, which corresponds to a fire away from walls or corners, a fire adjacent to a wall, or a fire located in a corner, respectively.

The HGL height, $z$, in Eq.~\ref{eq:ASET_1} can be calculated iteratively using
\be
z|_{t+1} = z|_t - \frac{\dot V\sb{ul}}{L W} \Delta t
\label{eq:ASET_5}
\ee
where $L$ and $W$ are the length and width of the compartment~(\si{m}), respectively, and $\Delta t$ is the time step size~(\si{s}).


\clearpage


\subsection*{Verification}

Test: NIST/NRC Test 1

\begin{table}[!ht]
\caption[Verification input parameters, HGL depth]
{Verification input parameters, HGL depth.}
\begin{center}
\begin{tabular}{|l|c|}
\hline
                        &              \\
\rb{Input Parameter}    &  \rb{Value}  \\ \hline \hline
$\dot Q$ (kW)           &  410         \\ \hline
$L$ (m)                 &  21.66       \\ \hline
$W$ (m)                 &  7.04        \\ \hline
$H$ (m)                 &  3.82        \\ \hline
$k$ (\si{kW/(m.K)})     &  0.00012     \\ \hline
$\rho$ (kg/m$^3$)       &  737         \\ \hline
$c$ (\si{kJ/(kg.K)})    &  1.42        \\ \hline
$T_\infty$ ($^\circ$C)  &  22          \\ \hline
Location Factor (-)     &  1           \\ \hline
$\chi\sb{l}$ (-)        &  0           \\ \hline
$z\sb{f}$ (-)           &  0           \\ \hline
\end{tabular}
\end{center}
\end{table}

\noindent Expected result: At 10~s, the HGL depth $z$ is 0.35~m; at 20~s, $z$ is 0.65~m; at 30~s, $z$ is 0.93~m.


\clearpage


\subsection*{Validation}

\begin{figure}[!ht]
\begin{center}
\begin{tabular}{l}
\includegraphics[width=4.0in]{SCRIPT_FIGURES/Scatterplots/HGL_Depth_ASET}
\end{tabular}
\end{center}
\caption[Summary of HGL depth predictions (ASET)]
{Summary of HGL depth predictions using ASET.}
\label{HGL_Depth_ASET}
\end{figure}


\clearpage


\section{Yamana and Tanaka Method}

\subsection*{Description}

For a compartment with no ventilation (closed doors) and constant HRR, the correlation of Yamana and Tanaka~\cite{Tanaka:1} predicts that the HGL height, $z$, is given by
\be
z = \left( \frac{2 k \dot Q^{1/3} t}{3 A\sb{c}} + \frac{1}{h\sb{c}^{2/3}} \right)^{-3/2}
\label{eq:Yamana_Tanaka}
\ee
where $\dot Q$ is the HRR~(\si{kW}), $t$ is the time after ignition~(\si{s}), $A\sb{c}$ is the compartment floor area~(\si{m^2}), and $h\sb{c}$ is the compartment height~(\si{m}). The constant $k$ is given by
\be
k = \frac{0.076}{(353/T\sb{g})}
\ee
where $T\sb{g}$ is the HGL temperature~(\si{K}).

\subsection*{Verification}

Test: NIST/NRC Test 1
\\ \\
\noindent Note: In this verification case, the method of Beyler is used to calculate the HGL temperature, $T\sb{g}$.

% In the FDTs spreadsheets, this HGL depth correlation is used with the MQH correlation (natural ventilation).

\begin{table}[!ht]
\caption[Verification input parameters, HGL depth]
{Verification input parameters, HGL depth.}
\begin{center}
\begin{tabular}{|l|c|}
\hline
                        &              \\
\rb{Input Parameter}    &  \rb{Value}  \\ \hline \hline
$\dot Q$ (kW)           &  410         \\ \hline
$L$ (m)                 &  21.66       \\ \hline
$W$ (m)                 &  7.04        \\ \hline
$H$ (m)                 &  3.82        \\ \hline
$k$ (\si{kW/(m.K)})     &  0.00012     \\ \hline
$\rho$ (kg/m$^3$)       &  737         \\ \hline
$c$ (\si{kJ/(kg.K)})    &  1.42        \\ \hline
$T_\infty$ ($^\circ$C)  &  22          \\ \hline
\end{tabular}
\end{center}
\end{table}

\noindent Expected result: At 10~s, the HGL depth $z$ is 0.28~m; at 20~s, $z$ is 0.53~m; at 30~s, $z$ is 0.75~m.


\clearpage


\subsection*{Validation}

\begin{figure}[!ht]
\begin{center}
\begin{tabular}{l}
\includegraphics[width=4.0in]{SCRIPT_FIGURES/Scatterplots/HGL_Depth_Yamana_Tanaka}
\end{tabular}
\end{center}
\caption[Summary of HGL depth predictions (Yamana and Tanaka)]
{Summary of HGL depth predictions using Yamana and Tanaka method.}
\label{HGL_Depth_YT}
\end{figure}

