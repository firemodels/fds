\chapter{Gas Velocity}

Gas velocity is often measured at compartment inlets and outlets as part of a global assessment of mass and
energy conservation.  This chapter contains measurements of gas velocity and related quantities.

\section{Steckler Compartment Doorway Velocity Profiles}

Steckler {\em et al.}~\cite{Steckler:NBSIR_82-2520} mapped the doorway/window flows in 55 compartment fire experiments. The test matrix
is presented in Table~\ref{Steckler_Table}. Shown on
the following pages are the centerline profiles, compared with model predictions. Off-center profiles are not shown. The vertical spacing
of the measurements was approximately 11~cm, with the uppermost velocity probe centered 5.7~cm below the 10~cm thick soffit. The FDS
simulations were uniformly gridded with cells of 5~cm on each side. To quantify the difference between prediction and measurement, the
maximum outward velocities, which always occurred at the uppermost measurement location, were compared. As can be seen in Fig.~\ref{Steckler_Scatterplot}, the
uppermost velocity measurement is under-predicted by FDS. It has been found that relatively minor changes in the velocity boundary conditions at the
edges and bottom of the door soffit can have a noticeable impact on these results. It is believed that the resolution of the grid is not fine enough to
capture the steep gradients in the velocity profile in the uppermost grid cell. Work is on-going to improve the situation.

\newpage





\begin{figure}[p]
\begin{tabular*}{\textwidth}{l@{\extracolsep{\fill}}r}
\includegraphics[height=2.2in]{FIGURES/Steckler_Compartment/Steckler_010_Vel} &
\includegraphics[height=2.2in]{FIGURES/Steckler_Compartment/Steckler_011_Vel} \\
\includegraphics[height=2.2in]{FIGURES/Steckler_Compartment/Steckler_012_Vel} &
\includegraphics[height=2.2in]{FIGURES/Steckler_Compartment/Steckler_612_Vel} \\
\includegraphics[height=2.2in]{FIGURES/Steckler_Compartment/Steckler_013_Vel} &
\includegraphics[height=2.2in]{FIGURES/Steckler_Compartment/Steckler_014_Vel} \\
\includegraphics[height=2.2in]{FIGURES/Steckler_Compartment/Steckler_018_Vel} &
\includegraphics[height=2.2in]{FIGURES/Steckler_Compartment/Steckler_710_Vel}
\end{tabular*}
\label{Steckler_Vel_1}
\end{figure}

\begin{figure}[p]
\begin{tabular*}{\textwidth}{l@{\extracolsep{\fill}}r}
\includegraphics[height=2.2in]{FIGURES/Steckler_Compartment/Steckler_810_Vel} &
\includegraphics[height=2.2in]{FIGURES/Steckler_Compartment/Steckler_016_Vel} \\
\includegraphics[height=2.2in]{FIGURES/Steckler_Compartment/Steckler_017_Vel} &
\includegraphics[height=2.2in]{FIGURES/Steckler_Compartment/Steckler_022_Vel} \\
\includegraphics[height=2.2in]{FIGURES/Steckler_Compartment/Steckler_023_Vel} &
\includegraphics[height=2.2in]{FIGURES/Steckler_Compartment/Steckler_030_Vel} \\
\includegraphics[height=2.2in]{FIGURES/Steckler_Compartment/Steckler_041_Vel} &
\includegraphics[height=2.2in]{FIGURES/Steckler_Compartment/Steckler_019_Vel}
\end{tabular*}
\label{Steckler_Vel_2}
\end{figure}

\begin{figure}[p]
\begin{tabular*}{\textwidth}{l@{\extracolsep{\fill}}r}
\includegraphics[height=2.2in]{FIGURES/Steckler_Compartment/Steckler_020_Vel} &
\includegraphics[height=2.2in]{FIGURES/Steckler_Compartment/Steckler_021_Vel} \\
\includegraphics[height=2.2in]{FIGURES/Steckler_Compartment/Steckler_114_Vel} &
\includegraphics[height=2.2in]{FIGURES/Steckler_Compartment/Steckler_144_Vel} \\
\includegraphics[height=2.2in]{FIGURES/Steckler_Compartment/Steckler_212_Vel} &
\includegraphics[height=2.2in]{FIGURES/Steckler_Compartment/Steckler_242_Vel} \\
\includegraphics[height=2.2in]{FIGURES/Steckler_Compartment/Steckler_410_Vel} &
\includegraphics[height=2.2in]{FIGURES/Steckler_Compartment/Steckler_210_Vel}
\end{tabular*}
\label{Steckler_Vel_3}
\end{figure}

\begin{figure}[p]
\begin{tabular*}{\textwidth}{l@{\extracolsep{\fill}}r}
\includegraphics[height=2.2in]{FIGURES/Steckler_Compartment/Steckler_310_Vel} &
\includegraphics[height=2.2in]{FIGURES/Steckler_Compartment/Steckler_240_Vel} \\
\includegraphics[height=2.2in]{FIGURES/Steckler_Compartment/Steckler_116_Vel} &
\includegraphics[height=2.2in]{FIGURES/Steckler_Compartment/Steckler_122_Vel} \\
\includegraphics[height=2.2in]{FIGURES/Steckler_Compartment/Steckler_224_Vel} &
\includegraphics[height=2.2in]{FIGURES/Steckler_Compartment/Steckler_324_Vel} \\
\includegraphics[height=2.2in]{FIGURES/Steckler_Compartment/Steckler_220_Vel} &
\includegraphics[height=2.2in]{FIGURES/Steckler_Compartment/Steckler_221_Vel}
\end{tabular*}
\label{Steckler_Vel_4}
\end{figure}

\begin{figure}[p]
\begin{tabular*}{\textwidth}{l@{\extracolsep{\fill}}r}
\includegraphics[height=2.2in]{FIGURES/Steckler_Compartment/Steckler_514_Vel} &
\includegraphics[height=2.2in]{FIGURES/Steckler_Compartment/Steckler_544_Vel} \\
\includegraphics[height=2.2in]{FIGURES/Steckler_Compartment/Steckler_512_Vel} &
\includegraphics[height=2.2in]{FIGURES/Steckler_Compartment/Steckler_542_Vel} \\
\includegraphics[height=2.2in]{FIGURES/Steckler_Compartment/Steckler_610_Vel} &
\includegraphics[height=2.2in]{FIGURES/Steckler_Compartment/Steckler_510_Vel} \\
\includegraphics[height=2.2in]{FIGURES/Steckler_Compartment/Steckler_540_Vel} &
\includegraphics[height=2.2in]{FIGURES/Steckler_Compartment/Steckler_517_Vel}
\end{tabular*}
\label{Steckler_Vel_5}
\end{figure}

\begin{figure}[p]
\begin{tabular*}{\textwidth}{l@{\extracolsep{\fill}}r}
\includegraphics[height=2.2in]{FIGURES/Steckler_Compartment/Steckler_622_Vel} &
\includegraphics[height=2.2in]{FIGURES/Steckler_Compartment/Steckler_522_Vel} \\
\includegraphics[height=2.2in]{FIGURES/Steckler_Compartment/Steckler_524_Vel} &
\includegraphics[height=2.2in]{FIGURES/Steckler_Compartment/Steckler_541_Vel} \\
\includegraphics[height=2.2in]{FIGURES/Steckler_Compartment/Steckler_520_Vel} &
\includegraphics[height=2.2in]{FIGURES/Steckler_Compartment/Steckler_521_Vel} \\
\includegraphics[height=2.2in]{FIGURES/Steckler_Compartment/Steckler_513_Vel} &
\includegraphics[height=2.2in]{FIGURES/Steckler_Compartment/Steckler_160_Vel}
\end{tabular*}
\label{Steckler_Vel_6}
\end{figure}

\begin{figure}[p]
\begin{tabular*}{\textwidth}{l@{\extracolsep{\fill}}r}
\includegraphics[height=2.2in]{FIGURES/Steckler_Compartment/Steckler_163_Vel} &
\includegraphics[height=2.2in]{FIGURES/Steckler_Compartment/Steckler_164_Vel} \\
\includegraphics[height=2.2in]{FIGURES/Steckler_Compartment/Steckler_165_Vel} &
\includegraphics[height=2.2in]{FIGURES/Steckler_Compartment/Steckler_162_Vel} \\
\includegraphics[height=2.2in]{FIGURES/Steckler_Compartment/Steckler_167_Vel} &
\includegraphics[height=2.2in]{FIGURES/Steckler_Compartment/Steckler_161_Vel} \\
\includegraphics[height=2.2in]{FIGURES/Steckler_Compartment/Steckler_166_Vel} &

\end{tabular*}
\label{Steckler_Vel_7}
\end{figure}

\begin{figure}[p]
\begin{center}
\begin{tabular}{l}
\includegraphics[width=5.0in]{FIGURES/ScatterPlots/Velocity_Steckler}
\end{tabular}
\end{center}
\caption[Summary of velocity predictions, Steckler compartment experiments.]
{Summary of comparisons of predicted and measured maximum velocities in the doorway/window of the Steckler compartment experiments. The uncertainty
in the measurements (dotted lines) were reported by Steckler to be about 10~\%.}
\label{Steckler_Scatterplot}
\end{figure}


\clearpage

\section{Bryant Doorway Experiments}

On the following page there are seven plots comparing the predicted and measured centerline velocity\footnote{Note that the quantity 
that is being compared is the total velocity multiplied by the sign of its normal component.} profiles
in a doorway of a standard ISO~9705 compartment. The measurements shown are based on PIV (Particle Image Velocimetry).
Note that the measurements do not extend to the top of the
doorway 1.96~m above the compartment floor because the heat from the fire prevented adequate laser resolution of
the particles. Velocity measurements were also made using bi-directional probes~\cite{Bryant:FSJ2009}, but these
measurements were shown to be up to 20~\% greater in magnitude than the comparable PIV measurement.



\begin{figure}[p]
\begin{tabular*}{\textwidth}{l@{\extracolsep{\fill}}r}
\includegraphics[height=2.2in]{FIGURES/Bryant_Doorway/Bryant_Doorway_034_kW} &
\includegraphics[height=2.2in]{FIGURES/Bryant_Doorway/Bryant_Doorway_065_kW} \\
\includegraphics[height=2.2in]{FIGURES/Bryant_Doorway/Bryant_Doorway_096_kW} &
\includegraphics[height=2.2in]{FIGURES/Bryant_Doorway/Bryant_Doorway_128_kW} \\
\includegraphics[height=2.2in]{FIGURES/Bryant_Doorway/Bryant_Doorway_160_kW} &
\includegraphics[height=2.2in]{FIGURES/Bryant_Doorway/Bryant_Doorway_320_kW} \\
\includegraphics[height=2.2in]{FIGURES/Bryant_Doorway/Bryant_Doorway_511_kW} &
\end{tabular*}
\label{Bryant_Doorway}
\end{figure}


%\clearpage


%\section{NIST/WTC Test Series}

%\begin{figure}[p]
%\begin{tabular*}{\textwidth}{l@{\extracolsep{\fill}}r}
%\includegraphics[height=2.2in]{FIGURES/WTC/WTC_01_v5_Inlet_Velocity} &
%\includegraphics[height=2.2in]{FIGURES/WTC/WTC_01_v5_Outlet_Velocity} \\
%\includegraphics[height=2.2in]{FIGURES/WTC/WTC_02_v5_Inlet_Velocity} &
%\includegraphics[height=2.2in]{FIGURES/WTC/WTC_02_v5_Outlet_Velocity} \\
%\includegraphics[height=2.2in]{FIGURES/WTC/WTC_03_v5_Inlet_Velocity} &
%\includegraphics[height=2.2in]{FIGURES/WTC/WTC_03_v5_Outlet_Velocity}
%\end{tabular*}
%\label{NIST_WTC_Velocity_1}
%\end{figure}


%\begin{figure}[p]
%\begin{tabular*}{\textwidth}{l@{\extracolsep{\fill}}r}
%\includegraphics[height=2.2in]{FIGURES/WTC/WTC_04_v5_Inlet_Velocity} &
%\includegraphics[height=2.2in]{FIGURES/WTC/WTC_04_v5_Outlet_Velocity} \\
%\includegraphics[height=2.2in]{FIGURES/WTC/WTC_05_v5_Inlet_Velocity} &
%\includegraphics[height=2.2in]{FIGURES/WTC/WTC_05_v5_Outlet_Velocity} \\
%\includegraphics[height=2.2in]{FIGURES/WTC/WTC_06_v5_Inlet_Velocity} &
%\includegraphics[height=2.2in]{FIGURES/WTC/WTC_06_v5_Outlet_Velocity}
%\end{tabular*}
%\label{NIST_WTC_Velocity_2}
%\end{figure}



%\begin{figure}[ht]
%\begin{tabular*}{\textwidth}{l@{\extracolsep{\fill}}r}
%\includegraphics[width=3.0in]{FIGURES/ScatterPlots/Velocity} &
%\end{tabular*}
%\caption{Summary of Velocity Results.}
%\end{figure}


\clearpage

\section{Restivo Experiment}

The results of a simulation of Restivo's room ventilation experiment are presented below.
To capture the forced inlet flow, the volume near the supply slot needs a fairly
fine grid to capture the mixing of air at the shear layer. For the results shown here, the height of the inlet
was spanned with 6 grid cells, roughly 3~cm in the vertical dimension, 6~cm in the other two. Finer grids
were used in the Musser study~\cite{Musser:1}, but with no appreciable change in results. The component
of velocity in the lengthwise direction was measured in four arrays: two vertical arrays located 3~m and 6~m  from the inlet along the
centerline of the room, and two horizontal arrays located 8.4~cm above the floor and below the ceiling, respectively.
These measurements were taken using hot-wire anemometers. While data on the specific
instrumentation used are not readily available, hot-wire systems tend to have limitations at low velocities,
with typical thresholds of approximately 0.1~m/s.

\begin{figure}[h!]
\begin{tabular*}{\textwidth}{l@{\extracolsep{\fill}}r}
\includegraphics[height=2.2in]{FIGURES/Restivo_Experiment/Restivo_3m_Velocity} &
\includegraphics[height=2.2in]{FIGURES/Restivo_Experiment/Restivo_6m_Velocity} \\
\includegraphics[height=2.2in]{FIGURES/Restivo_Experiment/Restivo_Ceiling_Velocity} &
\includegraphics[height=2.2in]{FIGURES/Restivo_Experiment/Restivo_Floor_Velocity}
\end{tabular*}
\label{Restivo_Velocity}
\end{figure}
