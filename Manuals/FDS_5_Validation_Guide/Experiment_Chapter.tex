\chapter{Description of Experiments}

This chapter contains a brief description of the experiments that were used for model validation. Only enough detail is included here to provide a
general understanding of the model simulations. Anyone wishing to use the experimental measurements for validation ought to consult the cited test reports for a 
comprehensive description.



\section{VTT Large Hall Tests}

The experiments are described in Ref.~\cite{Hostikka:Hall}. The series consisted of 8 experiments, but because of replicates only three unique fire
scenarios. The experiments were undertaken to study the movement of smoke in a large hall with a sloped ceiling. The tests were conducted inside the
VTT Fire Test Hall, with dimensions of 19 m (62 ft) high by 27 m (89 ft) long by 14 m (46 ft) wide. Figure shows detailed plan, side and perspective
schematic diagrams of the experimental arrangement. Each test involved a single heptane pool fire, ranging from 2~MW to 4~MW. Figure is a photo of a
2~MW fire. Four types of measurements were used in the present evaluation -- the HGL temperature and depth, average flame height and the plume
temperature. Three vertical arrays of thermocouples, plus two thermocouples in the plume, were compared to model simulation results. The HGL
temperature and height were reduced from an average of the three TC trees using the standard algorithm. The ceiling jet temperature was not
considered, because the ceiling in the test hall is not flat, and the standard model algorithm is not appropriate for these conditions.

The VTT test report lacks some information needed to model the experiments, so some information was based on private communications with the
principal investigator, Simo Hostikka. The information used to conduct the model simulations is presented in Table 2-3, including information on the
fire, the compartment, and the ventilation.

Surface Materials: The walls and ceiling of the test hall consist of a 1 mm (0.039 in) thick layer of sheet metal on top of a 5 cm (2 in) layer of
mineral wool. The floor was constructed of concrete. The report does not provide thermal properties of these materials. Thermophysical properties of
the materials that were used in the simulations are given in Chapter 3.

Natural Ventilation: In Cases 1 and 2, all doors were closed, and ventilation was restricted to infiltration through the building envelope. Precise
information on air infiltration during these tests is not available. The scientists who conducted the experiments recommend a leakage area of about 2
m2 (20 ft2), distributed uniformly throughout the enclosure. By contrast, in Case 3, the doors located in each end wall (Doors 1 and 2, respectively)
were open to the external ambient environment. These doors are each 0.8 m (2.6 ft) wide by 4 m (5 ft) high, and are located such that their centers
are 9.3 m (30.5 ft) from the south wall.

Mechanical Ventilation: The test hall had a single mechanical exhaust duct, located in the roof space, running along the center of the building. This
duct had a circular section with a diameter of 1 m (40 in), and opened horizontally to the hall at a distance of 12 m (39 ft) from the floor and 10.5
m (34.4 ft) from the west wall. Mechanical exhaust ventilation was operational for Case 3, with a constant volume flow rate of 11 m3/s drawn through
the 1 m (40 in) diameter exhaust duct. 2-14 Heat Release Rate: Each test used a single fire source with its center located 16 m (52 ft) from the west
wall and 7.4 m (24.3 ft) from the south wall. For all tests, the fuel was heptane in a circular steel pan that was partially filled with water. The
pan had a diameter of 1.17 m (46.0 in) for Case 1 and 1.6 m (63 in) for Cases 2 and 3. In each case, the fuel surface was 1 m (40 in) above the
floor. The trays were placed on load cells, and the HRR was calculated from the mass loss rate (see definition in Chapter 3). For the three cases,
the fuel mass loss rate was averaged from individual replicate tests. In the HRR estimation, the heat of combustion (taken as 44.6 kJ/g) and the
combustion efficiency for n-heptane was used. Hostikka suggests a value of 0.8 for the combustion efficiency. Bundy [Ref. 23] estimates the
efficiency of a 500 kW heptane pool fire to be equal to 0.97. Tewarson reports a value of 0.93 for a 10 cm pool [Ref. 17]. The magnitude of the
combustion efficiency is a complicated function of fire size, ventilation, and other effects. Consideration of the chemical structure of a fire
suggests that the combustion efficiency should decrease as the fire size grows. Available data confirms this [Ref. 24]. The size of a compartment may
also impact this parameter, but there is little information in the fire literature that addresses this point. In summary, there is little certainty
in the actual value of the combustion efficiency in this experiment. In this report, a combustion efficiency of 0.85 � 0.12
(or � 14 %) is recommended for the BE #2 pool fire tests, based on engineering judgment. Due
to the relatively large value of the uncertainty associated with .a, the uncertainty in HRR is dominated by the uncertainty in the combustion
efficiency. Uncertainty in the mass loss rate measurement also contributed to the overall uncertainty, and the uncertainty in HRR was estimated
as 15 %. Figures 2-9 to 2-11 show the prescribed HRR as a function of time during Cases 1 to 3,
respectively. Tables 2-4 to 2-6 represent the mass loss and estimated HRR associated with Figures 2-9 to 2-11, respectively.

Radiative Fraction: The radiative fraction was assigned a value of 0.35, similar to many smoky hydrocarbons [Ref. 19]. The relative combined expanded
(2s) uncertainty in this parameter was assigned a
value of �20 %, which is typical of uncertainty values reported in the literature for this
parameter.



\clearpage

\section{UL/NFPRF Sprinkler, Vent, and Draft Curtain Study}
\label{UL_NFPRF_Description}

In January, 1997, a series of 22 heptane spray burner experiments was conducted at the Large Scale Fire Test Facility at Underwriters Laboratories
(UL) in Northbrook, Illinois~\cite{Sheppard:1}. The objective of the experiments was to characterize the temperature and flow field for fire
scenarios with a controlled heat release rate in the presence of sprinklers, draft curtains and a single smoke \& heat vent.

\subsubsection{Test Facility}

The Large Scale Fire Test Facility at UL contains a 37~m by 37~m (120~ft by 120~ft) main fire test cell, equipped with a 30.5~m by 30.5~m (100~ft by
100~ft) adjustable height ceiling. The layout of the experiments is shown in Fig.~\ref{layout}. One 1.2~m by 2.4~m (4~ft by 8~ft) vent was installed
among 49 upright sprinklers on a 3~m by 3~m (10~ft by 10~ft) spacing.

\begin{figure}[p]
\begin{center}
\setlength{\unitlength}{.05416667in}
\begin{picture}(120,120)

\linethickness{1.mm} \put(0,0){\framebox(120,120)[tc]{North Wall}} \linethickness{.5mm} \put(10,10){\framebox(100,100)[tc]{Adjustable Height
Ceiling}}

\thinlines \put(117,67){\vector(0,-1){67}} \put(117,73){\vector(0, 1){47}} \put(117,70){\makebox(0,0){$120'$}} \put(113,57){\vector(0,-1){47}}
\put(111,110){\line(1,0){4.}} \put(111, 10){\line(1,0){4.}} \put(113,63){\vector(0, 1){47}} \put(113,60){\makebox(0,0){$100'$}}
\put(30.9,12.83){\dashbox{1}(67.1,71.17)[tc]{Draft Curtains}} \put(27.9,40){\vector(0,-1){27.17}} \put(27.9,46){\vector(0, 1){38.0}}
\put(25.9,84.){\line(1,0){4.}} \put(25.9,12.83){\line(1,0){4.}} \put(27.9,43){\makebox(0,0){$71'2''$}} \put(64.0,87.){\vector(-1,0){33.1}}
\put(72.0,87.){\vector( 1,0){26.0}} \put(30.9,85.){\line(0,1){4.}} \put(98.0,85.){\line(0,1){4.}} \put(68.0,87.){\makebox(0,0){$67'1''$}}

\put(16.0,87.){\vector(-1,0){6.}} \put(24.0,87.){\vector( 1,0){6.92}} \put(20.0,87.){\makebox(0,0){$20'11''$}} \put(101.,87.){\vector(-1,0){3.}}
\put(107.,87.){\vector( 1,0){3.}} \put(104.,87.){\makebox(0,0){$12'$}}

\put(27.9,100){\vector(0,1){10.}} \put(27.9,94){\vector(0,-1){10.}} \put(27.9,97){\makebox(0,0){$26'$}}

\put(27.9,8){\vector(0,1){2.}} \put(27.9,8){\line(1,0){3.}} \put(30.9,8){\makebox(0,0)[l]{$2'10''$}}

\put(55.08,14.83){\line(-1,0){2.}} \put(54.08,16.83){\vector(0,-1){2.}} \put(54.08,7.83){\vector(0,1){5.}} \put(54.08,7.83){\line(1,0){3.}}
\put(57.08,7.83){\makebox(0,0)[l]{$2'$}}

\put(85.08,24.83){\line(0,-1){2.}} \put(95.08,24.83){\line(0,-1){2.}} \put(93.08,23.83){\vector(1,0){2.}} \put(87.08,23.83){\vector(-1,0){2.}}
\put(103.00,23.83){\vector(-1,0){5.}} \put(103.00,23.83){\line(0,-1){3.}} \put(103.00,20.83){\makebox(0,0)[ct]{$2'11''$}}
\put(90.08,23.83){\makebox(0,0)[c]{$10'$}}

\thicklines \put(78.08,55.83){\framebox(4,8){ }}

\put(78.58,58.33){\framebox(3,3)[c]{A}} \put(78.58,68.33){\framebox(3,3)[c]{B}} \put(88.58,58.33){\framebox(3,3)[c]{C}}
\put(58.58,38.33){\framebox(3,3)[c]{D}}

\thinlines

\multiput(35.08,14.83)(0,10){7}{\circle*{.8}} \multiput(45.08,14.83)(0,10){7}{\circle*{.8}} \multiput(55.08,14.83)(0,10){7}{\circle*{.8}}
\multiput(65.08,14.83)(0,10){7}{\circle*{.8}} \multiput(75.08,14.83)(0,10){7}{\circle*{.8}} \multiput(85.08,14.83)(0,10){7}{\circle*{.8}}
\multiput(95.08,14.83)(0,10){7}{\circle*{.8}} \tiny \put(35.48,15.23){98} \put(45.48,15.23){91} \put(55.48,15.23){84} \put(65.48,15.23){81}
\put(75.48,15.23){78} \put(85.48,15.23){75} \put(95.48,15.23){72} \put(35.48,25.23){99} \put(45.48,25.23){92} \put(55.48,25.23){85}
\put(65.48,25.23){82} \put(75.48,25.23){79} \put(85.48,25.23){76} \put(95.48,25.23){73} \put(35.48,35.23){100} \put(45.48,35.23){93}
\put(55.48,35.23){86} \put(65.48,35.23){83} \put(75.48,35.23){80} \put(85.48,35.23){77} \put(95.48,35.23){74} \put(35.48,45.23){101}
\put(45.48,45.23){94} \put(55.48,45.23){87} \put(65.48,45.23){62} \put(75.48,45.23){58} \put(85.48,45.23){54} \put(95.48,45.23){50}
\put(35.48,55.23){102} \put(45.48,55.23){95} \put(55.48,55.23){88} \put(65.48,55.23){63} \put(75.48,55.23){59} \put(85.48,55.23){55}
\put(95.48,55.23){51} \put(35.48,65.23){103} \put(45.48,65.23){96} \put(55.48,65.23){89} \put(65.48,65.23){64} \put(75.48,65.23){60}
\put(85.48,65.23){56} \put(95.48,65.23){52} \put(35.48,75.23){104} \put(45.48,75.23){97} \put(55.48,75.23){90} \put(65.48,75.23){65}
\put(75.48,75.23){61} \put(85.48,75.23){57} \put(95.48,75.23){53} \put(70.08,49.83){\makebox(0,0)[c]{68}} \put(70.08,59.83){\makebox(0,0)[c]{69}}
\put(70.08,69.83){\makebox(0,0)[c]{70}} \put(80.08,49.83){\makebox(0,0)[c]{67}} \put(90.08,49.83){\makebox(0,0)[c]{66}}
\put(90.08,69.83){\makebox(0,0)[c]{71}}

\multiput(80.08,56.83)(0,1){7}{\circle*{.2}} \put(80.58,62.83){\line(1,0){22.5}} \put(104.,62.83){\makebox(0,0)[l]{43}}
\put(104.,61.33){\makebox(0,0)[l]{44}} \put(104.,59.83){\makebox(0,0)[l]{45}} \put(104.,58.33){\makebox(0,0)[l]{46}}
\put(104.,56.83){\makebox(0,0)[l]{47}} \put(104.,55.33){\makebox(0,0)[l]{48}} \put(104.,53.83){\makebox(0,0)[l]{49}}

\normalsize

\end{picture}
\end{center}
\caption[Plan view of the UL/NFPRF Experiments.] {\bf Plan view of the UL/NFPRF Experiments. The sprinklers are indicated by the solid circles and
are spaced 3~m (10~ft) apart. The number beside each sprinkler location indicates the channel number of the nearest thermocouple. The vent dimensions
are 4~ft by 8~ft. The boxed letters A, B, C and D indicate burner positions. Corresponding to each burner position is a vertical array of
thermocouples. Thermocouples 1--9 hang 7, 22, 36, 50, 64, 78, 92, 106 and 120~in from the ceiling, respectively, above Position A. Thermocouples 10
and 11 are positioned above and below the ceiling tile directly above Position B, followed by 12--20 that hang at the same levels below the ceiling
as 1--9. The same pattern is followed at Positions C and D, with thermocouples 21--31 at C and 32--42 at D.} \label{layout}
\end{figure}

The ceiling was raised to a height of 7.6~m (25~ft) and instrumented with thermocouples and other measurement devices. The ceiling was constructed of
0.6~m by 1.2~m by 1.6~cm (2~ft by 4~ft by 5/8~in) UL fire rated Armstrong Ceramaguard (Item 602B) ceiling tiles. The manufacturer reported the
thermal properties of the material to be: specific heat 753 J/kg$\cdot$K, thermal diffusivity $2.6 \times 10^{-7}$~m$^2$/s, conductivity
0.0611~W/m$\cdot$K, and density 313~kg/m$^3$.

Draft curtains 1.8~m (6~ft) deep were installed for 16 of the 22 tests, enclosing an area of about 450~m$^2$ (4,800~ft$^2$). The curtains were
constructed of 1.4~m (54~in) wide sheets of 18 gauge sheet metal.

The sprinklers used were Central ELO-231 (Extra Large Orifice) uprights. The orifice diameter of this sprinkler is reported by the manufacturer to be
nominally 0.64~in, the reference actuation temperature is reported by the manufacturer to be 74$^\circ$C (165$^\circ$F). The RTI (Response Time
Index) and C-factor (Conductivity factor) were reported by UL to be 148~(m$\cdot$s)$^\ha$ (268~(ft$\cdot$s)$^\ha$) and 0.7~(m/s)$^\ha$
(1.3~(ft/s)$^\ha$), respectively~\cite{Sheppard:1}. When installed, the sprinkler deflector was located 8~cm (3~in) below the ceiling. The thermal
element of the sprinkler was located 11~cm (4.25~in) below the ceiling. The sprinklers were installed with 3~m by 3~m (10~ft by 10~ft) spacing in a
system designed to deliver a constant 0.34~L/(s$\cdot$m$^2$) (0.50 gpm/ft$^2$) discharge density when supplied by a 131~kPa (19~psi) discharge
pressure

A single UL listed double leaf fire vent with steel covers and steel curb was installed in the adjustable height ceiling in the position shown in
Fig.~\ref{layout}. The vent is designed to open manually or automatically. The vent doors were recessed into the ceiling about 0.3~m (1~ft).

\subsubsection{Fire and Heat Release Rate}

The heptane spray burner consisted of a 1~m by 1~m (40~in by 40~in) square of 1/2~in pipe supported by four cement blocks 0.6~m (2~ft) off the floor.
Four atomizing spray nozzles were used to provide a free spray of heptane that was then ignited. For all but one of the tests, the total heat release
rate from the fire was manually ramped up following the curve
$$ \dot{Q} = \dot{Q}_0 \; \left( \frac{t}{\tau} \right)^2 $$
where $\dot{Q}_0=10$~MW and $\tau=75$~s ($\tau=150$~s was used in Test I-16). The fire growth curve was followed until a specified fire size was
reached or the first sprinkler activated. After either of these events, the fire size was maintained at that level until conditions reached roughly a
steady state, {\em i.e.} the temperatures recorded near the ceilings remained steady and no more sprinkler activations occurred.

The heat release rate from the burner was confirmed by placing it under the large product calorimeter at UL, ramping up the flow of heptane in the
same manner as in the tests, and measuring the total and convective heat release rates. It was found that the convective heat release rate was
0.65$\pm$0.02 of the total.

\subsubsection{Instrumentation}

The instrumentation for the tests consisted of thermocouples, gas analysis equipment, and pressure transducers. The locations of the instrumentation
are referenced in the plan view of the facility (Fig.~\ref{layout}).

Temperature measurements were recorded at 104 locations. Type K 0.0625~in diameter Inconel sheathed thermocouples were positioned to measure (i)
temperatures near the ceiling, (ii) temperatures of the ceiling jet, and (iii) temperatures near the vent. The thermocouples numbered 50--65 were
positioned near the sprinklers, 10~cm (4~in) below the ceiling. These were intended to measure near-sprinkler gas temperatures as well as to detect
sprinkler activation when wetted. Thermocouples 66--104 were placed 5~cm (2~in) below the ceiling. Thermocouples 43--49 ran down the centerline of
the vent at the level of the ceiling, and were spaced 0.3~m (1~ft) apart. Thermocouples 1--42 were mounted on arrays hanging above each fire
location. The positions are noted in the caption to Fig.~\ref{layout}.

Oxygen, carbon dioxide and carbon monoxide sampling probes were placed at the ground (5~cm (3~in) from the floor, 2~m (6~ft) from the burner), and at
the vent (15~cm (6~in) below the ceiling, vent center).


\clearpage

\section{NIST/NRC Test Series}

These experiments, sponsored by the US NRC and conducted at NIST, consisted of 15 large-scale experiments performed in June 2003. All 15 tests were
included in the validation study. The experiments are documented in Ref.~\cite{Hamins:SP1013-1}. The fire sizes ranged from 350 kW to 2.2 MW in a compartment with dimensions
21.7~m by 7.1~m by 3.8~m high, designed to represent a compartment in a nuclear power plant containing power and control cables.
The walls and ceiling were covered with two layers of marinate boards, each layer 0.0125~m thick. The floor
was covered with one layer of gypsum board on top of a layer of plywood. Thermo-physical and optical properties of the marinate
and other materials used in the compartment are given in Ref.~\cite{Hamins:SP1013-1}. The room had one door and a mechanical air injection and extraction
system. Ventilation conditions, the fire size, and fire location were varied. Numerous measurements (approximately 350 per test) were made including
gas and surface temperatures, heat fluxes and gas velocities.

Natural Ventilation: The compartment had a 2 m by 2 m door in the middle of the west wall. Some of the tests had a closed door and no mechanical
ventilation (Tests 2, 7, 8, 13, and 17), and in those tests the measured compartment leakage was an important consideration. Ref. [6] reports leakage
area based on measurements performed prior to Tests 1, 2, 7, 8, and 13. For the closed door tests, the leakage area used in the simulations ought to
be based on the last available measurement. It should be noted that the chronological order of the tests differed from the numerical order [Ref. 6].
For Test 4, it is recommended that the leakage area measured before Test 2 be used. For Tests 10 and 16, it is recommended that the leakage area
measured before Test 7 be used.

Mechanical Ventilation: The mechanical ventilation and exhaust was used during Tests 4, 5, 10, and 16, providing about 5 air changes per hour. The
door was closed during Test 4 and open during Tests 5, 10, and 16. The supply duct was positioned on the south wall, about 2 m off the floor. An
exhaust duct of equal area to the supply duct was positioned on the opposite wall at a comparable location. The flow rates through the supply and
exhaust ducts were measured in detail during breaks in the testing, in the absence of a fire. During the tests, the flows were monitored with single
bi-directional probes during the tests themselves.

Heat Release Rate: A single nozzle was used to spray liquid hydrocarbon fuels onto a 1 m by 2 m fire pan that was about 0.02 m deep. The test plan
originally called for the use of two nozzles to provide the fuel spray. Experimental observation suggested that the fire was less unsteady with the
use of a single nozzle. In addition, it was observed that the actual extent of the liquid pool was well-approximated by a 1 m circle in the
center of the pan. For safety reasons, the fuel flow was terminated when the lower-layer oxygen concentration
dropped to approximately 15~\% by volume.
The fuel used in 14 of the tests was heptane, while toluene was used for one test. The HRR was
determined using oxygen consumption calorimetry. The recommended uncertainty values
were 17~\% for all of the tests.

Radiative Fraction: The radiative fraction was measured in an independent study for the same fuels using the same spray burner as used in the test
series [Ref. 18]. The value of the radiative fraction and its
uncertainty were reported as 0.44 � 16 \% and 0.40 � 23 \% for heptane and toluene, respectively.

\clearpage

\section{McCaffrey's Plume Experiments}

In 1979, at the National Bureau of Standards (now NIST), Bernard McCaffrey measured centerline temperature and velocity profiles above a porous,
refractory burner. There were five distinct heat release rates, ranging from 14~kW to 57~kW. The fuel was natural gas. The burner was square, 0.3~m on each side.
The results of the experiments are reported in Reference~\cite{McCaffrey:NBSIR_79-1910}.



\section{NIST Diffusion Flame Test Series}

Smyth {\em et al.} conducted diffusion flame experiments at NIST used a methane/air Wolfhard-Parker slot burner.  The experiments are described in
detail in Refs.~\cite{Norton:1,Smyth:1}.  The Wolfhard-Parker slot burner consists of an 8~mm wide
central slot flowing fuel surrounded by two 16~mm wide slots flowing dry air with 1~mm separations between the slots.
The slots are 41~mm in length.  Measurements were made of all major species and a number of minor species along with temperature
and velocity.  Experimental uncertainties have been reported as 5~\% for temperature  and 10~\% to 20~\%
for the major species.

\section{Beyler Hood Experiments}

Beyler performed a large number of experiments involving a variety of fuels, fire sizes, burner diameters, and
burner distances beneath a hood~\cite{Beyler:Hood}.  The hood consisted of concentric cylinders separated
by an air gap.  The inner cylinder was shorter than the outer and this allowed combustion products to be removed
uniformly from the hood perimeter.  The exhaust gases were then analyzed to determine species concentrations.
The burner could be raised and lowered with respect to the bottom edge of the hood.  Based on the published
measurement uncertainties, species errors are estimated at 6~\%.

\section{NIST Reduced Scale Enclosure CO Production Test Series}

The CO production test series used the NIST Reduced Scale Enclosure (RSE)~\cite{Bryner:1}.  The RSE is a
40~\% scaled version of the ISO 9705 compartment.
It measures 0.98 m wide by 1.46 m deep by 0.98 m tall.  The compartment contains a door centered on the small
face that measures 0.48 m wide by 0.81 m tall.  A 15 cm diameter natural gas burner was positioned in the
center of the compartment.  The burner was on a stand so that its top was 15~cm above the floor.
Species measurements were made inside the upper layer of the compartment at the front near the door
and near the rear of the compartment.

\section{NIST Flame Radiation Test Series}

Hamins {\em et al.} performed a series of tests on circular gas
burners measuring the radial and vertical radiative heat flux profiles
outside the flame region. The tests are described
in~\cite{Hostikka:3}. Tests at three burner diameters, 0.10 m, 0.38 m
and 1.0 m are used for validation. The parameters of the tests are
listed in the table below.

\begin{table}[h!]
\begin{center}
%\caption{Details of the Hamins CH4 tests}
\label{tab:Hamins_CH4}
\vspace{0.1in}
\begin{tabular}{|l|c|c|c|}
\hline
Test no. & Diameter  & Fuel        & HRRPUA      \\
         &   (m)     &             & (kW/m$^2$)  \\ \hline
\hline
1        & 0.1       & CH$_4$      & 53.8        \\ \hline
5        & 0.1       & CH$_4$      & 240         \\ \hline
23       & 0.38      & CH$_4$      & 295         \\ \hline
21       & 0.38      & CH$_4$      & 1550        \\ \hline
7        & 1.0       & Natural gas & 62.4        \\ \hline
19       & 1.0       & Natural gas & 206         \\ \hline
\end{tabular}
\end{center}
\end{table}
