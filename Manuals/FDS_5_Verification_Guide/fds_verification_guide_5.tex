\documentclass[11pt]{book}
\usepackage{mathptm,times}
\usepackage[pdftex]{graphicx}
\usepackage{hyperref}

%\usepackage{eso-pic}
%\usepackage{graphicx}
%\usepackage{color}
%\usepackage{type1cm}

%\makeatletter
%   \AddToShipoutPicture{%
%     \setlength{\@tempdimb}{.5\paperwidth}%
%    \setlength{\@tempdimc}{.5\paperheight}%
%   \setlength{\unitlength}{1pt}%
%  \put(\strip@pt\@tempdimb,\strip@pt\@tempdimc){%
%     \makebox(0,0){\rotatebox{45}{\textcolor[gray]{0.75}{\fontsize{8cm}{8cm}\selectfont{DRAFT}}}}}}
%\makeatother

\setlength{\textwidth}{6.5in}
\setlength{\textheight}{9.0in}
\setlength{\topmargin}{0.in}
\setlength{\headheight}{0.in}
\setlength{\headsep}{0.in}
\setlength{\parindent}{0.25in}
\setlength{\oddsidemargin}{0.0in}
\setlength{\evensidemargin}{0.0in}

\begin{document}

\bibliographystyle{unsrt}

\newcommand{\dod}[2]{\frac{\partial #1}{\partial #2}}
\newcommand{\DoD}[2]{\frac{D #1}{D #2}}
\newcommand{\dsods}[2]{\frac{\partial^2 #1}{\partial #2^2}}
\newcommand{\dx}{\delta x}
\newcommand{\dy}{\delta y}
\newcommand{\dz}{\delta z}
\newcommand{\x}{x}
\newcommand{\y}{y}
\newcommand{\z}{z}
\newcommand{\dt}{\delta t}
\newcommand{\dn}{\delta n}
\newcommand{\cH}{{\cal H}}
\newcommand{\hu}{u}
\newcommand{\hv}{v}
\newcommand{\hw}{w}
\newcommand{\la}{\lambda}
%\newcommand{\bO}{\mbox{\boldmath $\Omega$}}
\newcommand{\bO}{{\Omega}}
\newcommand{\bo}{{\bf \omega}}
%\newcommand{\btau}{\mbox{\boldmath $\tau$}}
\newcommand{\btau}{{\bf \tau}}
\newcommand{\bdelta}{{\bf \delta}}
\newcommand{\sumym}{\sum (Y_i/W_i)}
\newcommand{\oW}{\overline{W}}
\newcommand{\om}{\omega}
\newcommand{\omx}{\omega_x}
\newcommand{\omy}{\omega_y}
\newcommand{\omz}{\omega_z}
\newcommand{\erf}{\hbox{erf}}
\newcommand{\bF}{{\bf F}}
\newcommand{\bof}{{\bf f}}
\newcommand{\bq}{{\bf q}}
\newcommand{\br}{{\bf r}}
\newcommand{\bu}{{\bf u}}
\newcommand{\bx}{{\bf x}}
\newcommand{\bk}{{\bf k}}
\newcommand{\bv}{{\bf v}}
\newcommand{\bg}{{\bf g}}
\newcommand{\bn}{{\bf n}}
\newcommand{\bS}{{\bf S}}
\newcommand{\dS}{d{\bf S}}
\newcommand{\bs}{{\bf s}}
\newcommand{\bI}{{\bf I}}
\newcommand{\hp}{{\cal H}}
\newcommand{\trho}{\tilde{\rho}}
\newcommand{\dph}{{\delta\phi}}
\newcommand{\dth}{{\delta\theta}}
\newcommand{\tp}{\tilde{p}}
\newcommand{\dQ}{\dot{Q}}
\newcommand{\dq}{\dot{q}}
\newcommand{\dm}{\dot{m}}
\newcommand{\ha}{\frac{1}{2}}
\newcommand{\ft}{\frac{4}{3}}
\newcommand{\ot}{\frac{1}{3}}
\newcommand{\fofi}{\frac{4}{5}}
\newcommand{\of}{\frac{1}{4}}
\newcommand{\twth}{\frac{2}{3}}
\newcommand{\R}{{\cal R}}
\newcommand{\be}{\begin{equation}}
\newcommand{\ee}{\end{equation}}
\newcommand{\RE}{\hbox{Re}}
\newcommand{\LE}{\hbox{Le}}
\newcommand{\PR}{\hbox{Pr}}
\newcommand{\PE}{\hbox{Pe}}
\newcommand{\NU}{\hbox{Nu}}
\newcommand{\SC}{\hbox{Sc}}
\newcommand{\SH}{\hbox{Sh}}
\newcommand{\WE}{\hbox{We}}
\newcommand{\COTWO}{{\tiny \hbox{CO}_2}}
\newcommand{\OTWO}{{\tiny \hbox{O}_2}}
\newcommand{\CO}{{\tiny \hbox{CO}}}
\newcommand{\F}{{\tiny \hbox{F}}}

\newfont{\ct}{cmtt10 at 9pt}

\pagestyle{empty}

\begin{minipage}[t][9in][s]{6.5in}

\huge
\flushright{NIST Special Publication XXXX}

\vspace{1in}

\Huge \flushright{Fire Dynamics Simulator (Version 5) \\ Verification \& Validation Guide \\ \LARGE Volume 1: Verification \\ (Draft, August 7, 2007)}

\vspace{.5in}

\large
\flushright{
Kevin McGrattan \\
Anthony Hamins \\
Simo Hostikka \\
Jason Floyd \\
Bryan Klein}

\vspace{0.5in}

\flushright{In cooperation with: \\
%\includegraphics[width=1in]{FIGURES/VTT_GREY_L}  \\
VTT Technical Research Centre of Finland  }


\vfill

\includegraphics[width=\textwidth]{FIGURES/nistlogo_1line}

\end{minipage}

\newpage

\hspace{5in}

\newpage

\begin{minipage}[t][9in][s]{6.5in}

\huge
\flushright{NIST Special Publication XXXX}

\vspace{0.75in}

\Huge \flushright{Fire Dynamics Simulator (Version 5) \\ Verification \& Validation Guide \\ \LARGE Volume 1: Verification}

\vspace{.25in}

\normalsize
\flushright{
Kevin McGrattan \\
Anthony Hamins \\
Bryan Klein \\
{\em Fire Research Division} \\
{\em Building and Fire Research Laboratory}  \\
\hspace{1in} \\
Simo Hostikka \\
{\em VTT Technical Research Centre of Finland} \\
{\em Espoo, Finland} \\
\hspace{1in} \\
Jason Floyd \\
{\em Hughes Associates, Inc.}  \\
{\em Baltimore, Maryland, USA}}

\vspace{.25in}

\flushright{September 2007}

\vfill

\flushright{\includegraphics[width=1in]{FIGURES/doc} }

\small
\flushright{U.S. Department of Commerce \\
{\em Carlos M. Gutierrez, Secretary} \\
\hspace{1in} \\
Technology Administration \\
{\em Robert Cresanti, Under Secretary for Technology}  \\
\hspace{1in} \\
National Institute of Standards and Technology \\
{\em William A. Jeffrey, Director} }


\end{minipage}

\newpage

\begin{minipage}[t][9in][s]{6.5in}

\flushright{Certain commercial entities, equipment, or materials may be identified in this \\
document in order to describe an experimental procedure or concept adequately. Such \\
identification is not intended to imply recommendation or endorsement by the \\
National Institute of Standards and Technology, nor is it intended to imply that the \\
entities, materials, or equipment are necessarily the best available for the purpose.
}

\vspace{3in}

\large
\flushright{\bf National Institute of Standards and Technology Special Publication XXXX \\
Natl.~Inst.~Stand.~Technol.~Spec.~Publ.~XXXX, 94 pages (May 2007) \\
CODEN: NSPUE2 }

\vfill

\flushright{U.S. GOVERNMENT PRINTING OFFICE \\
WASHINGTON: 2004 \\
\rule{3.5in}{0.01in} \\
For sale by the Superintendent of Documents, U.S. Government Printing Office \\
Internet: bookstore.gpo.gov -- Phone: (202) 512-1800 -- Fax: (202) 512-2250 \\
Mail: Stop SSOP, Washington, DC 20402-0001 }

\end{minipage}



\newpage

\frontmatter

\pagestyle{plain}


\chapter{Preface}


This guide is based in part on the ``Standard Guide for
Evaluating the Predictive Capability of Deterministic Fire Models,'' ASTM~E~1355~\cite{ASTM:E1355}.
ASTM~E~1355 defines {\em model evaluation} as ``the process of quantifying
the accuracy of chosen results from a model when applied for a specific use.''
The model evaluation process consists of two main components: verification and validation.
{\em Verification} is a process to check the correctness of the solution of the
governing equations. Verification does not imply that the governing equations are
appropriate; only that the equations are being solved correctly.
{\em Validation} is a process to determine the appropriateness of the governing equations as a mathematical
model of the physical phenomena of interest. Typically, validation involves comparing
model results with experimental measurement. Differences that cannot be explained in terms of
numerical errors in the model or uncertainty in the measurements
are attributed to the assumptions and simplifications of the physical model.

Evaluation is critical to establishing both the acceptable uses
and limitations of a model. Throughout its development, FDS has undergone various forms of evaluation,
both at NIST and beyond. This guide provides a survey of work conducted to date to evaluate FDS.



\chapter{Disclaimer}

The US Department of Commerce makes no warranty, expressed or implied,
to users of the Fire Dynamics Simulator (FDS), and accepts no responsibility for its use.
Users of FDS assume sole responsibility under Federal law for determining
the appropriateness of its use in any particular application;
for any conclusions drawn from the results of its use; and for any
actions taken or not taken as a result of analysis performed using
these tools.

Users are warned that FDS is intended for use only by those competent
in the fields of fluid dynamics, thermodynamics, heat transfer,
combustion, and fire science, and is intended only to supplement the
informed judgment of the qualified user. The software package is a
computer model that may or may not have predictive capability when
applied to a specific set of factual circumstances. Lack of accurate
predictions by the model could lead to erroneous conclusions with
regard to fire safety. All results should be evaluated by an informed
user.

Throughout this document, the mention of computer hardware or commercial
software does not constitute endorsement by NIST, nor does it indicate that
the products are necessarily those best suited for the intended purpose.




\chapter{About the Authors}

\begin{description}
\item[Kevin McGrattan] is a mathematician in the Building and Fire
Research Laboratory of NIST. He received a bachelors of science degree
from the School of Engineering and Applied Science of Columbia
University in 1987 and a doctorate at the Courant Institute of New
York University in 1991. He joined the NIST staff in 1992 and has
since worked on the development of fire models, most notably the Fire
Dynamics Simulator.
\item[Anthony Hamins] is the leader of the Analysis and Prediction Group in the Building and Fire Research Laboratory of NIST.
He received a bachelors degree in physics from the University of California, Berkeley, and a doctorate
in engineering physics from the University of California, San Diego. At NIST, he has performed a number of large
scale validation experiments, and developed techniques to quantify the uncertainty of the experiments and the fire
models.
\item[Simo Hostikka] is a Senior Research Scientist at VTT Technical
Research Centre of Finland.  He is the principal developer of the
radiation and solid phase sub-models within FDS.
\item[Jason Floyd] is a Senior Engineer at Hughes Associates, Inc., in
Baltimore, Maryland. He received a bachelors of science degreee and a
doctorate from the Nuclear Engineering Program of the University of
Maryland. After graduating, he won a National Research Council
Post-Doctoral Fellowship at the Building and Fire Research Laboratory
of NIST, where he developed the combustion algorithm within FDS. He is
currently funded by NIST under grant 60NANB5D1205 from the Fire
Research Grants Program (15 USC 278f).  He is the principal developer
of the multi-parameter mixture fraction combustion model and control
logic within FDS.
\item[Bryan Klein] is an Information Technology Specialist in the
Building and Fire Research Laboratory of NIST.  Before coming to NIST,
Bryan worked for five years with Western Fire Center, Inc., performing a
wide range of activities including fire modeling, data acquisition programming,
and quantitative fire measurements. His current focus is on FDS development and
user support.
\end{description}



\chapter{Acknowledgments}

\label{acksection}

Support for the preparation of the FDS
manuals has been provided by the Office of Nuclear Regulatory Research
of the US Nuclear Regulatory Commission (US NRC). Special thanks to
Mark Salley and Jason Dreisbach for their efforts.

Thanks to Chris Lautenburger and Carlos Fernandez-Pello for their assistance with the ``two-reaction'' test case.

Thanks to Ian Thomas, Khalid Moinuddin, and Ian Bennetts for their description of and data for the ethanol pan fire example.




\tableofcontents

\mainmatter






\chapter{How FDS is Developed and Maintained}

This chapter outlines the basic framework under which the FDS software is developed and maintained, often referred to as a
{\em Configuration Management Plan}. Formal guidelines for such a plan are detailed in IEEE Standard~828-1998.


\subsubsection{Who are the FDS Developers?}

Currently, FDS is maintained by the Building and Fire Research Laboratory (BFRL) of National Institute of Standards and Technology. The developers
at NIST have formed a loose collaboration of interested stakeholders, including:
\begin{itemize}
\item VTT Technical Research Centre of Finland, a research and testing
laboratory similar to NIST
\item The Society of Fire Protection Engineers (SFPE) who conduct training classes on the use of FDS
\item Fire protection engineering firms that use the software
\item Engineering departments at various universities with a particular emphasis on fire
\end{itemize}
BFRL awards grants on a competitive basis to external organizations who conduct research in fire science and engineering. Some of these grants have
been used to assist the development of FDS. The role of the grantee in supporting day to day development varies. Not all of the developers outside
of NIST are grantees.


\subsubsection{What Does FDS Consist of?}

The Fire Dynamics Simulator is a numerical model of fire and thermally-driven fluid flow. The FDS software consists of the following
items:
\begin{itemize}
\item Fortran source code files
\item Compiled executables for various types of computers and operating systems
\item Documentation, including the Technical Reference Guide, User's Guide, and the Verification and Validation (V\&V) Guide
\item Sample input files, experimental test data, and various spread sheet programs used to assess model accuracy
\end{itemize}
In addition, NIST maintains a separate program called Smokeview to visualize the output of FDS. Outside of NIST, a number of
special purpose programs have been developed to generate FDS input files, convert Computer Aided Design (CAD) drawings into FDS
format, and so on.


\subsubsection{How is FDS Developed?}

Starting with Version 5, the FDS development team uses
an Internet-based development
environment called GoogleCode, a free service of the search engine company, Google. GoogleCode is a widely used service designed to assist
open source software development by providing a repository for source
code, revision control, program distribution, bug tracking, and
various other very useful services.

Each member of the FDS development team has an account and password
access to the FDS repository. In
addition, anonymous access is available to all interested users, who
can receive the latest versions of the source code, manuals, and other
items. Anonymous users simply do not have the power to commit changes
to any of these items. The power to commit changes to FDS or its
manuals can be granted to anyone on a case by case basis.

The FDS manuals are typeset using \LaTeX, specifically, PDF \LaTeX. The \LaTeX files are essentially text files that are under
SVN (Subversion) control. The figures are either in the form of PDF or jpeg files, depending on whether they are vector or
raster format. There are a variety of \LaTeX packages available, including MiKTeX. The FDS developers edit the manuals as part of the
day to day upkeep of the model. Different editions of the manuals are distinguished by date.


\subsubsection{How are Changes made to FDS?}

The version number for FDS has three parts.  For example, FDS 5.2.12
indicates that this is FDS 5, the fifth major release. The 2 indicates
a significant upgrade, but still within the framework of FDS 5.  The
12 indicates the twelveth minor upgrade of 5.2, mostly bug fixes and
minor user requests.

Changes are made to the FDS source code daily, and tracked via
revision control software. However, these
daily changes do not constitute a change to the version number. After
the developers determine that enough changes have been made to the
source, they release a new minor upgrade, 5.2.12 to 5.2.13, for
example. This happens every few weeks. A change from 5.2 to 5.3 might
happen only a few times a year, when significant improvements have
been made to the model physics.

There is no formal process by which FDS is updated. Each developer
works on various routines, and makes changes as warranted. Minor bugs
are fixed without any communication (the developers are in different
locations), but more significant changes are discussed via email or
telephone calls. A suite of simple verification calculations (included
in this document) are routinely run to ensure that the daily bug fixes
have not altered any of the important algorithms. A suite of
validation calculations (also included here) are run with each
significant upgrade, and the results are replotted in this Guide.
Significant changes to FDS are made based on the following criteria, in no particular order:
\begin{description}
\item[Better Physics:] The goal of any model is to be {\em predictive}, but it also must be reliable. FDS is a blend of empirical and
deterministic sub-models, chosen based on their robustness, consistency, and reliability. Any new sub-model must demonstrate that it is
of comparable or superior accuracy to its empirical counterpart.
\item[Modest CPU Increase:] If a proposed algorithm doubles the calculation time but yields only a marginal improvement in accuracy, it is
likely to be rejected. Also, the various routines in FDS are expected to consume CPU time in proportion to their overall importance. For example,
the radiation transport algorithm consumes about 25~\% of the CPU time, consistent with the fact that about one-fourth to one-third of the
fire's energy is emitted as thermal radiation.
\item[Simpler Algorithm:] If a new algorithm does what the old one did using less lines of code, it is almost always accepted, so long as
it does not decrease functionality.
\item[Increased or Comparable Accuracy:] The validation experiments that are part of this guide serve as the metric for new routines. It is
not enough for a new algorithm to perform well in a few cases. It must show clear improvement across the suite of experiments. If the
accuracy is only comparable to the previous version, then some other criteria must be satisfied.
\item[Acceptance by the Fire Protection Community:] Especially in regard to fire-specific devices, like sprinklers and smoke detectors, the
algorithms in FDS often are based on their acceptance among the practicing engineers.
\end{description}








\chapter{Forms of Verification}

ASTM~E~1355~\cite{ASTM:E1355} outlines methods to evaluate the
mathematical and numerical robustness of deterministic fire
models. This process, often referred to as {\em model verification},
ensures the accuracy of the numerical solution of the governing
equations. The methods include comparison with analytical solutions,
code checking, and numerical tests.

\section{Comparison with Analytical Solutions}
\label{Analytical Tests}

Most complex combustion processes, including fire, are turbulent and
time-dependent. There are no closed-form mathematical solutions for
the fully-turbulent, time-dependent Navier-Stokes equations. CFD
provides an approximate solution for the non-linear partial
differential equations by replacing them with discretized algebraic
equations that can be solved using a powerful computer. While there is
no general analytical solution for fully-turbulent flows, certain
sub-models address phenomenon that do have analytical solutions, for
example, one-dimensional heat conduction through a solid. These
analytical solutions can be used to test sub-models within a complex
code such as FDS. The developers of FDS routinely use such practices
to verify the correctness of the coding of the
model~\cite{Mell:1,McGrattan:4}. Such verification efforts are
relatively simple and routine and the results may not always be
published nor included in the documentation. Examples of routine
analytical testing include:
\begin{itemize}
\item The radiation solver has been verified with scenarios where
simple objects, like cubes or flat plates, are positioned in simple,
sealed compartments. All convective motion is turned off, the object
is given a fixed surface temperature and emissivity of one (making it
a black body radiator). The heat flux to the cold surrounding walls is
recorded and compared to analytical solutions.  These studies help
determine the appropriate number of solid angles to be set as the
default.
\item Solid objects are heated with a fixed heat flux, and the
interior and surface temperatures as a function of time are compared
to analytical solutions of the one-dimensional heat transfer
equation. These studies help determine the number of nodes to use in
the solid phase heat transfer model. Similar studies are performed to
check the pyrolysis models for thermoplastic and charring solids.
\item Early in its development, the hydrodynamic solver that evolved
to form the core of FDS was checked against analytical solutions of
simplified fluid flow phenomena. These studies were conducted at the
National Bureau of Standards (NBS)\footnote{The National Institute of
Standards and Technology (NIST) was formerly known as the National
Bureau of Standards.} by Rehm, Baum and
co-workers~\cite{Rehm:SIAM83,Rehm:SIAM84,Baum:CST84,Rehm:ANM85}. The
emphasis of this early work was to test the stability and consistency
of the basic hydrodynamic solver, especially the velocity-pressure
coupling that is vitally important in low Mach number
applications. Many numerical algorithms developed up to that point in
time were intended for use in high-speed flow applications, like
aerospace. Many of the techniques adopted by FDS were originally
developed for meteorological models, and as such needed to be tested
to assess whether they would be appropriate to describe relatively
low-speed flow within enclosures.
\item A fundamental decision made by Rehm and Baum early in the FDS
development was to use a direct (rather than iterative) solver for the
pressure. In the low Mach number formulation of the Navier-Stokes
equations, an elliptic partial differential equation for the pressure
emerges, often referred to as the Poisson equation. Most CFD methods
use iterative techniques to solve the governing conservation equations
to avoid the necessity of directly solving the Poisson equation. The
reason for this is that the equation is time-consuming to solve
numerically on anything but a rectilinear grid. Because FDS is
designed specifically for rectilinear grids, it can exploit fast,
direct solvers of the Poisson equation, obtaining the pressure field
with one pass through the solver to machine accuracy. FDS employs
double-precision (8 byte) arithmetic, meaning that the relative
difference between the computed and the exact solution of the
discretized Poisson equation is on the order of $10^{-12}$. The
fidelity of the numerical solution of the entire system of equations
is tied to the pressure/velocity coupling because often simulations
can involve hundreds of thousands of time steps, with each time step
consisting of two solutions of the Poisson equation to preserve
second-order accuracy. Without the use of the direct Poisson solver,
build-up of numerical error over the course of a simulation could
produce spurious results. Indeed, an attempt to use single-precision
(4 byte) arithmetic to conserve machine memory led to spurious results
simply because the error per time step built up to an intolerable
level.
\end{itemize}


\section{Code Checking}
\label{Code Checking}

An examination of the structure of the computer program can be used to
detect potential errors in the numerical solution of the governing
equations.  The coding can be verified by a third party either
manually or automatically with profiling programs to detect
irregularities and inconsistencies~\cite{ASTM:E1355}.

At NIST and elsewhere, FDS has been compiled and run on computers
manufactured by IBM, Hewlett-Packard, Sun Microsystems, Digital
Equipment Corporation, Apple, Silicon Graphics, Dell, Compaq, and
various other personal computer vendors. The operating systems on
these platforms include Unix, Linux, Microsoft Windows, and Mac
OSX. Compilers used include Lahey Fortran, Digital Visual Fortran,
Intel Fortran, IBM XL Fortran, HPUX Fortran, Forte Fortran for SunOS,
the Portland Group Fortran, and several others. Each combination of
hardware, operating system and compiler involves a slightly different
set of compiler and run-time options and a rigorous evaluation of the
source code to test its compliance with the Fortran 90 ISO/ANSI
standard~\cite{F90}. Through this process, out-dated and potentially
harmful code is updated or eliminated, and often the code is
streamlined to improve its optimization on the various
machines. However, simply because the FDS source code can be compiled
and run on a wide variety of platforms does not guarantee that the
numerics are correct. It is only the starting point in the process
because it at least rules out the possibility that erratic or spurious
results are due to the platform on which the code is running.

Beyond hardware issues, there are several useful techniques for
checking the FDS source code that have been developed over the
years. One of the most best ways is to exploit symmetry. FDS is filled
with thousands of lines of code in which the partial derivatives in
the conservation equations are approximated as finite differences. It
is very easy in this process to make a mistake. Consider, for example,
the finite difference approximation of the thermal diffusion term in
the $ijk$th cell of the three-dimensional grid:
\begin{eqnarray*}
(\nabla \cdot k \nabla T)_{ijk} &\approx&
              \frac{1}{\dx}
         \left[k_{i+\ha,jk}\frac{T_{i+1,jk}-T_{ijk}}{\dx}
              -k_{i-\ha,jk}\frac{T_{ijk}-T_{i-1,jk}}{\dx}\right]+  \nonumber \\
            &&\frac{1}{\dy}
         \left[k_{i,j+\ha,k}\frac{T_{i,j+1,k}-T_{ijk}}{\dy}
              -k_{i,j-\ha,k}\frac{T_{ijk}-T_{i,j-1,k}}{\dy}\right]+ \nonumber \\
            &&\frac{1}{\dz}
         \left[k_{ij,k+\ha}\frac{T_{ij,k+1}-T_{ijk}}{\dz}
              -k_{ij,k-\ha}\frac{T_{ijk}-T_{ij,k-1}}{\dz}\right]
\end{eqnarray*}
which is written as follows in the Fortran source code:
\begin{verbatim}
      DTDX = (TMP(I+1,J,K)-TMP(I,J,K))*RDXN(I)
      KDTDX(I,J,K) = .5*(KP(I+1,J,K)+KP(I,J,K))*DTDX
      DTDY = (TMP(I,J+1,K)-TMP(I,J,K))*RDYN(J)
      KDTDY(I,J,K) = .5*(KP(I,J+1,K)+KP(I,J,K))*DTDY
      DTDZ = (TMP(I,J,K+1)-TMP(I,J,K))*RDZN(K)
      KDTDZ(I,J,K) = .5*(KP(I,J,K+1)+KP(I,J,K))*DTDZ

      DELKDELT = (KDTDX(I,J,K)-KDTDX(I-1,J,K))*RDX(I) +
     .           (KDTDY(I,J,K)-KDTDY(I,J-1,K))*RDY(J) +
     .           (KDTDZ(I,J,K)-KDTDZ(I,J,K-1))*RDZ(K)
\end{verbatim}
This is one of the simpler constructs because the pattern that emerges
within the lines of code make it fairly easy to check. However, a
mis-typing of an {\ct I} or a {\ct J}, a plus or a minus sign, or any
of a hundred different mistakes can cause the code to fail, or worse
produce the wrong answer. A simple way to eliminate many of these
mistakes is to run simple scenarios that have perfectly symmetric
initial and boundary conditions.  For example, put a hot cube in the
exact center of a larger cold compartment, turn off gravity, and watch
the heat diffuse from the hot cube into the cold gas. Any simple error
in the coding of the energy equation will show up almost
immediately. Then, turn on gravity, and in the absence of any coding
error, a perfectly symmetric plume will rise from the hot cube. This
checks both the coding of the energy and the momentum
equations. Similar checks can be made for all of the three dimensional
finite difference routines. So extensive are these types of checks
that the release version of FDS has a routine that generates a tiny
amount of random noise in the initial flow field so as to eliminate
any false symmetries that might arise in the numerical solution.

The process of adding new routines to FDS is as follows: typically the
routine is written by one person (not necessarily a NIST staffer) who
takes the latest version of the source code, adds the new routine, and
writes a theoretical and numerical description for the FDS Technical
Reference Guide, plus a description of the input parameters for the
FDS User's Guide. The new version of FDS is then tested at NIST with a
number of benchmark scenarios that exercise the range of the new
parameters.  Provisional acceptance of the new routine is based on
several factors: (1) it produces more accurate results when compared
to experimental measurement, (2) the theoretical description is sound,
and (3) any empirical parameters are obtainable from the open
literature or standard bench-scale apparatus.  If the new routine is
accepted, it is added to a test version of the software and evaluated
by external users and/or NIST grantees whose research is related to
the subject. Assuming that there are no intractable issues that arise
during the testing period, the new routine eventually becomes part of
the release version of FDS.

Even with all the code checking performed at NIST, it is still
possible for errors to go unnoticed. One remedy is the fact that the
source code for FDS is publicly released. Although it consists of on
the order of 10,000 lines of Fortran statements, various researchers
outside of NIST have been able to work with it, add enhancements
needed for very specific applications or for research purposes, and
report back to the developers bugs that have been detected. The source
code is organized into 14 separate files, each containing subroutines
related to a particular feature of the model, like the mass, momentum,
and energy conservation equations, sprinkler activation and sprays,
the pressure solver, {\em etc.} The lengthiest routines are devoted to
input, output and initialization.  Most of those working with the
source code do not concern themselves with these lengthy routines but
rather focus on the finite-difference algorithm contained in a few of
the more important files. Most serious errors are found in these
files, for they contain the core of the algorithm. The external
researchers provide feedback on the organization of the code and its
internal documentation, that is, comments within the source code
itself.  Plus, they must compile the code on their own computers,
adding to its portability. Some of the work performed by researchers
who have modified the source code is discussed in Volume 2.  However, most of the routine
error reports are via electronic mail and are undocumented.  Most of
the current error reports involve routines that are not frequently
used by the FDS developers.  For example, the opening of compartment
doors or the breaking of windows, especially upon activation of a heat
detector, is a feature of FDS commonly used in the fire protection
engineering community but less so at NIST. As a result, numerous
reports have been made over the years in which a complicated sequence
of events prescribed by the user is not carried out by the program.
The errors are easy to fix, but the number of possible permutations of
events make it difficult to check them all. Another problem reported
by users are scenarios that extend the parameter range beyond which
the model was originally conceived.  Walls made of foam, fires in
refrigerators, gas leaks, fuel spills, {\em etc.}, are just some of
the phenomena that users have attempted to model but have run into
difficulty because the model parameters either have never been
exercised ({\em e.g.} very low thermal conductivities) or are not
allowed ({\em e.g.} temperatures below ambient). These reports by the
users help to improve and extend the use of the model.


\section{Numerical Tests}
\label{Numerical Tests}

Numerical techniques used to solve the governing equations within a
model can be a source of error in the predicted results.  The
hydrodynamic model within FDS is second-order accurate in space and
time.  This means that the error terms associated with the
approximation of the spatial partial derivatives by finite differences
is of the order of the square of the grid cell size, and likewise the
error in the approximation of the temporal derivatives is of the order
of the square of the time step. As the numerical grid is refined, the
``discretization error'' decreases, and a more faithful rendering of
the flow field emerges.  The issue of grid sensitivity is extremely
important to the proper use of the model and will be taken up in the
next chapter.

A common technique of testing flow solvers is to systematically refine
the numerical grid until the computed solution does not change, at
which point the calculation is referred to as a Direct Numerical
Solution (DNS) of the governing equations.  For most practical fire
scenarios, DNS is not possible on conventional computers. However, FDS
does have the option of running in DNS mode, where the Navier-Stokes
equations are solved without the use of sub-grid scale turbulence
models of any kind. Because the basic numerical method is the same for
LES and DNS, DNS calculations are a very effective way to test the
basic solver, especially in cases where the solution is steady-state.
Throughout its development, FDS has been used in DNS mode for special
applications.  For example, FDS (or its core algorithms) have been
used at a grid resolution of roughly 1~mm to look at flames spreading
over paper in a microgravity
environment~\cite{McGrattan:C&F1996,Kashiwagi:CS1996,Mell:CS98,Mell:CS00,Prasad:CS2002,Nakamura:C&F2002},
as well as "g-jitter" effects aboard spacecraft~\cite{Mell:g-jitter}.
Simulations have been compared to experiments performed aboard the US
Space Shuttle.  The flames are laminar and relatively simple in
structure, and the comparisons are a qualitative assessment of the
model solution. Similar studies have been performed comparing DNS
simulations of a simple burner flame to laboratory
experiments~\cite{Mukhopadhyay:1}. Another study compared FDS
simulations of a counterflow diffusion flames to experimental
measurements and the results of a one-dimensional multi-step kinetics
model~\cite{Hamins:NASA}.

Early work with the hydrodynamic solver compared two-dimensional
simulations of gravity currents with salt-water
experiments~\cite{McGrattan:1}. In these tests, the numerical grid was
systematically refined until almost perfect agreement with experiment
was obtained. Such convergence would not be possible if there were a
fundamental flaw in the hydrodynamic solver.




\chapter{Verification Test Suite}
\label{verification_suite}

This section contains a description of a set of relatively simple
calculations that are used to {\em verify} the major physical
algorithms within FDS.  That is, these samples confirm that the
equations have been properly coded.  They do not imply that the
equations actually represent some physical phenomena.  That is left to Volume 2. Note that the names in parentheses in each section header
correspond to the names of the input files for the cases.



\section{Hydrodynamics}

There are no analytical solutions of the fully-turbulent Navier-Stokes equations, but it is possible to simulate well
known fluid flows to determine if the basic fluid flow solver in FDS is working properly.



\subsection{Axially-Symmetric Helium Plume ({\bf helium\_2d}) }

The governing equations solved in FDS are written in terms of a
three dimensional Cartesian coordinate system. However,
a two dimensional Cartesian or two dimensional cylindrical
(axially-symmetric) calculation can be performed by setting the number of
cells in the $y$ direction to 1.
An example of an axially-symmetric helium plume is shown here, along with the input file:

\vspace{0.2in}
\scriptsize
\noindent
\begin{minipage}{1.1in}
\includegraphics[height=2in]{FIGURES/helium_2d}
\end{minipage}
\hfill
\begin{minipage}{5.5in}
\begin{verbatim}
&HEAD CHID='helium_2d',TITLE='Axisymmetric Helium Plume' /
&MESH IJK=72,1,144 XB=0.00,0.08,-0.001,0.001,0.00,0.16, CYLINDRICAL=.TRUE. /
&TIME TWFIN=5.0 /
&MISC DNS=.TRUE., ISOTHERMAL=.TRUE. /
&SPEC ID='HELIUM'  /
&SURF ID='HELIUM', VEL=-0.673, MASS_FRACTION(1)=1.0, TAU_MF(1)=0.3 /
&VENT MB='XMAX' ,SURF_ID='OPEN' /
&VENT MB='ZMAX' ,SURF_ID='OPEN' /
&OBST XB= 0.0,0.036,-0.001,0.001,0.00,0.02, SURF_IDS='HELIUM','INERT','INERT' /
&DUMP PLOT3D_QUANTITY(1)='PRESSURE',PLOT3D_QUANTITY(5)='HELIUM' /
&SLCF PBY=0.000,QUANTITY='DENSITY', VECTOR=.TRUE. /
&SLCF PBY=0.000,QUANTITY='HELIUM' /
&TAIL /
\end{verbatim}
\end{minipage}
\normalsize


\clearpage

\subsection{Pressure Rise in a Sealed Enclosure ({\bf pressure\_rise})  }

This example tests several basic features of FDS. A narrow channel, 3~m long, 0.002~m wide, and 1~m tall, has air injected at a rate of
0.1~kg/m$^2$/s over an area of 0.2~m by 0.002~m for 60 s, with a linear ramp-up and ramp-down over 1 s. The total mass of air in the channel at the start
is 0.00718~kg. The total mass of air injected is 0.00244~kg.
The domain is assumed two-dimensional, the walls are adiabatic, and {\ct STRATIFICATION} is set to {\ct .FALSE.}. The domain is divided into three
meshes each 1~m long with identicle gridding.  We expect the pressure,
 temperature and density to rise during the 60~s injection period. Afterwards, the
temperature, density, and pressure should remain constant. A hand computation is performed at 10 second intervals using the First Law of
Thermodynamics and the equation of state.  The figures below show the results of this verification case.  As is seen denisty
matches exactly showing that FDS is injecting the appropriate quantity of mass and is properly initializing the domain.
Pressure rise and temperature rise; however, are overpredicted by 3 \% and 12 \% respectively.  Also a slight drop in pressure
is seen from 60 s to 120 s, indicating that the current implementation adiabatic boundary condition has a slight error in it.

\noindent
\begin{tabular*}{\textwidth}{lr}
\includegraphics[width=3.2in]{FIGURES/pressure_rise_T} &
\includegraphics[width=3.2in]{FIGURES/pressure_rise_P} \\
\includegraphics[width=3.2in]{FIGURES/pressure_rise_R} &
\includegraphics[width=3.2in]{FIGURES/pressure_rise_1000}
\end{tabular*}




\clearpage

\subsection{Leaks and Fans in a Sealed Enclosure ({\bf leak\_test} and {\bf leak\_test\_2})  }

A new feature of FDS 5 is the idea of a ``pressure zone.''  Unlike traditional compartment or ``zone'' fire models, FDS was not designed under the assumption that there exist
rooms connected by doors or ducts. Rather, the geometry in FDS is completely specified by the user. However, there are features of simpler models that we want to retain. For example,
a leak through a small crack, or the transport of air through a ventilation duct.  In the following example, a simple compartment (3.6~m by 2.4~m by 2.4~m) has a small fan at one
end and one leak under the door at the other end. It is assumed for this example that the compartment is contained within a larger compartment that is perfectly sealed. The fan draws
air into the compartment from the plenum space, increasing the pressure inside and decreasing it outside.  A steady state is achieved when the volume flow into and out of the compartment
falls into balance.

The volume flow rate of the fan is given by the ``fan curve''
\be \dot{V}_{\hbox{\footnotesize fan}} = A_{\hbox{\footnotesize duct}} U_{\hbox{\footnotesize max}} \;
   \hbox{sign} (\Delta p_{\hbox{\footnotesize max}}-\Delta p)
   \sqrt{ \frac{ |\Delta p - \Delta p_{\hbox{\footnotesize max}}|}{\Delta p_{\hbox{\footnotesize max}} } }  \ee
where $\Delta p$ is the difference in pressure and $A_{\hbox{\scriptsize duct}}=0.16$~m$^2$, $U_{\hbox{\scriptsize max}}=0.1$~m/s, and $\Delta p_{\hbox{\scriptsize max}}=1000$~Pa.
The volume flow due to the leak is given by:
\be \dot{V}_{\hbox{\scriptsize leak}} = A_{\hbox{\scriptsize leak}} \sqrt{ \frac{ 2 \Delta p}{ \rho_\infty} } \ee
where $A_{\hbox{\scriptsize leak}}=0.0001$~m$^2$ and $\rho_\infty=1.2$~kg/m$^3$.
After 5~min the pressure difference is 938.2~Pa. The theoretical value, obtained by equating the fan and leak volume flow rates and solving for $\Delta_p$, is 938.9~Pa. The
slight difference is due to the fact that the solid boundaries within the interior of the computational domain admit a slight volume flux related to details of the
numerical solver.

Just for fun, we add another leak to the compartment, only this time the leak is to the exterior of the entire computational domain, an infinite void at ambient pressure.
Now the fan flow rate ought to balance the sum of the flow rates from the two leaks. After 5~min, the pressure difference is 935.2~Pa.

The two cases are summarized in the following plots:

\vspace{0.1in}
\noindent
\begin{tabular*}{\textwidth}{lr}
\includegraphics[width=3.1in]{FIGURES/leak_test_Pressure_Case_1} &
\includegraphics[width=3.1in]{FIGURES/leak_test_Pressure_Case_2}
\end{tabular*}

\clearpage


\subsection{Two Fans in a Wall ({\bf fan\_test})  }

Consider two simple compartments divided by a wall with two fans installed, blowing in opposite directions.



\clearpage

\subsection{Stack Effect ({\bf stack\_effect})  }

If the interior temperature of a building is at a different temperature than the surrounding atmosphere, upward or
downward air flows within shafts or stairwells connected to the ambient via leakage paths will occur.  This
phenomena is known as the stack effect.  The {\bf stack\_effect} test case is a 2D simulation of a 304~m tall building
initialized to a temperature of 20~$^\circ$ with the surround ambient temperature initialized to 10~$^\circ$.
Two small openings in the building are defined 2.5~m above the ground floor of the building and 2.5~m below the
roof of the building.

The initial density stratification is defined by assuming a lapse rate of 0~$^\circ$C/m.
\be
   \rho_0(z) = \rho_\infty \, e^{ \frac{g W}{{\cal R}_0 T_0} z}
\ee
Applying this to the external and internal locations at the lower and upper vents results in densities of 1.2392,
1.1969, 1.1954, and 1.1546~kg/m$^3$, respectively.  FDS computes the same values to within machine precision.  Since the
openings in the building are equally spaced over its height, the neutral plane of the building will be close to its midpoint.
The pressure gradient across the building's wall can be computed as
\be
   \delta P = \frac{W P_0 g} {R_{0}} \left( \frac {1}{T_{ambient}} - \frac {1}{T_{building}} \right) h
\ee
where h is the distance from the neutral plane.  Using this pressure gradient in Bernoulli's equation (and assuming it remains constant)
results in a velocity of 10.09~m/s through the vent.  FDS computes a peak velocity of 10.13~m/s or an error of 0.5~\%.

\clearpage



\subsection{Sawtooth ({\bf sawtooth})  }

Sometimes it is desired to have stair-stepped obstructions representing curved or sloped geometry. A concern is that this may change the flow pattern near the wall. To lessen the impact of stair-stepped boundaries near the edges of the obstructions, one may specify the option {\ct SAWTOOTH=.FALSE.} If {\ct SAWTOOTH} is set to {\ct FALSE}, then the velocity boundary conditions will be applied in such a way as to minimize the impact of the boundaries due to vortices at sharp corners, as shown in the following example:

\scriptsize
\begin{verbatim}
&OBST XB= 0.00, 0.05,-0.01, 0.01, 0.00, 0.05, SAWTOOTH=.FALSE., COLOR='EMERALD GREEN' /
&OBST XB= 0.05, 0.10,-0.01, 0.01, 0.00, 0.10, SAWTOOTH=.FALSE., COLOR='EMERALD GREEN' /
&OBST XB= 0.10, 0.15,-0.01, 0.01, 0.05, 0.15, SAWTOOTH=.FALSE., COLOR='EMERALD GREEN' /
&OBST XB= 0.15, 0.20,-0.01, 0.01, 0.10, 0.20, SAWTOOTH=.FALSE., COLOR='EMERALD GREEN' /
\end{verbatim}\normalsize

In the figure below, the top set of obstructions are using the default {\ct SAWTOOTH=.TRUE.} and the bottom set of obstructions are using {\ct SAWTOOTH=.FALSE.} The adjacent obstructions that have {\ct SAWTOOTH=.FALSE.} are displayed in Smokeview as one smooth obstruction, shown in green. Notice that as the air moves across the different sets of obstructions, the air velocity on the bottom set of obstructions is not affected as much by the vortices.

\begin{center}
\includegraphics[width=3in]{FIGURES/sawtooth.jpg}
\end{center}

\clearpage





\section{Combustion}


\subsection{A Simple Under-Ventilated Compartment Fire ({\bf door\_crack}) }

This example uses the same simple compartment that was used to test leakage and fan curves in the previous section. Now, we add a small (160~kW) fire, with the same
fan and leak under the door. The compartment now opens to the atmosphere, not a sealed plenum. We expect a rapid pressure rise in the compartment due to the effect of the
fire and the fan. Initially, the pressure rise is approximately:
\be \frac{d \overline{p}_1}{dt} \approx (\gamma - 1) \; \frac{\dQ}{V} + \gamma \, \overline{p} \; \frac{ \dot{V} }{V} \approx 3200 \; \hbox{Pa/s}  \approx 0.03 \; \hbox{atm/s}   \ee
where $\gamma \approx 1.4$, $\dot{Q}=160,000$~W, $V=20.7$~m$^3$, and $\dot{V}=0.016$~m$^3$/s. In roughly 150~s, the pressure rises about 0.6~atm, at which point the fire
dies due to lack of oxygen. Then the pressure decreases, and the fan starts up again (it had stalled due to high pressure in the compartment). The fan, and the leak under the
door, increase the oxygen concentration, at least near these openings, and the fuel-rich gases in the compartment continue to burn.

\vspace{0.1in}
\noindent
\begin{tabular*}{\textwidth}{lr}
\includegraphics[width=3.1in]{FIGURES/door_crack_Pressure} &
\includegraphics[width=3.1in]{FIGURES/door_crack_HRR}
\end{tabular*}

While this case has a number of interesting physical effects, and it {\em verifiies} several features of FDS, it is very important to note the following:
\begin{itemize}
\item Although there is smoke seen flowing backwards out the fan duct, in reality there would have been much more. Most conventionally built structures will not withstand over-pressures
of 0.6~atm without some sort of relief. The fan and the crack under the door obey simple formulae based on pressure differences, but these assumptions have limits.
\item It is likely that the fire in this scenario would indeed extinguish itself as the oxygen volume fraction decreased below about 15~\%. {\bf But,} its re-ignition at the door crack
and fan opening would depend on the presence of a spark or hot spot of some sort. FDS continues to flow fuel into the compartment past the point of local extinction, but the compartment
cools. The default combustion algorithm in FDS assumes that in every grid cell there is a ``virtual spark plug'' that initiates combustion if the local ratio of fuel and oxygen are
appropriate.
\end{itemize}





\clearpage

\section{Radiation}

\subsection{Radiation inside a box ({\bf radiation\_in\_a\_box}) }

This verification case tests the computation of three-dimensional
configuration factor $\Phi$ inside a cube box with one hot wall and
five cold (0~K) walls. An overview of the test geometry is shown here:
\begin{center}
\includegraphics[width=4.0in]{FIGURES/box}
\end{center}
The configuration factors are calculated at the
diagonal of the cold wall opposite to the hot wall. The exact values of the configuration factor
from plane element $dA$ to parallel rectangle $H$
are calculated using the analytical solution~\cite{Siegel:1}

\begin{center}
\begin{tabular}{|c|c|c|c|}
\hline (y,z) & $\Phi_{HdA}$ & (y,z) & $\Phi_{HdA}$ \\ \hline \hline
0.025   &0.1457 & 0.275 &0.2135 \\
0.075   &0.1603 & 0.325 &0.2233 \\
0.125   &0.1748 & 0.375 &0.2311 \\
0.175   &0.1888 & 0.425 &0.2364 \\
0.225   &0.2018 & 0.475 &0.2391 \\ \hline
\end{tabular}
\end{center}

\noindent
Different variations of the case include the mesh resolution (20$^3$ and 100$^3$ cells)
and the number of radiation angles (50, 100, 300, 1000, 2000).
The exact and FDS results are shown here:

\begin{tabular*}{\textwidth}{lr}
\includegraphics[width=3.2in]{FIGURES/box_results_20} &
\includegraphics[width=3.2in]{FIGURES/box_results_100}
\end{tabular*}





\clearpage

\subsection{Radiation from a plane layer ({\bf radiation\_plane\_layer}) }

This case tests the computation of three-dimensional
radiation from a homogenous, infinitely wide layer of radiating
material.  The temperature of the layer is 1273.15~K and absorption
coefficients are varied. The thickness of the layer is fixed at 1.0~m,
and the optical depth is $1.0 \kappa$. Wall temperatures are set to 0~K.
The results are compared against the exact solution $S(\tau)$ presented
in~\cite{Zeldovich:1}
\be
S(\tau) = S_b\left[1-2E_3(\tau)\right]
\ee
where $S_b = \sigma T^4$ is the black-body heat flux from
the radiating plane and $E_3(\tau)$ is the exponential
integral function (order 3) of optical depth $\tau$.

The FDS results are computed at two mesh resolutions in the
$x$-direction (I = 20 and I = 150). For I=20, both one-band and
six-band versions are included to test the correct integration of heat
fluxes over multiple bands. For I=20, 2D-versions are also computed
(J=1). A special case with {\ct KAPPA0} = 0 and
an opposite wall temperature of 1273.15~K is computed to test the wall
heat flux computation. The exact values and FDS predictions of the
wall heat fluxes are given here:
\begin{center}
\begin{tabular}{|c|c|c|c|c|c|c|} \hline
$\tau$ & $S(\tau)$ & \multicolumn{2}{c}{FDS (I=20,J=20)}
           & \multicolumn{2}{c}{FDS (I=20,J=1)}
                   & FDS (I=150) \\
     &          & 1 band  & 6 bands   & 1 band & 6 bands & 1 band \\
                            \hline\hline
0    & 148.9709 & 148.9709&  148.4037 & 147.9426 &147.3793 &148.9709 \\
0.01 & 2.8970   & 2.9180  &  2.9069   & 2.8364   &2.8256   &2.9258   \\
0.1  & 24.9403  & 25.5501 &  25.4529  & 25.1078  &25.0122  &25.7045  \\
0.5  & 82.9457  & 83.1309 &  82.8144  & 84.3719  &84.0506  &84.0264  \\
1.0  & 116.2891 & 115.4051&  114.9656 & 117.801  &117.353  &116.7751 \\
10.  & 148.9698 & 148.9616&  148.3947 & 148.9677 &148.4005 &148.9695 \\ \hline
\end{tabular}
\end{center}



\clearpage

\section{Solid Phase Phenomena}

This section contains examples that test the one-dimensional heat conduction solver in FDS, along with those that include
pyrolysis.


\subsection{Simple Heat Conduction Through a Solid Slab ({\bf heat\_conduction}) }

Analytical solutions of transient, one-dimensional heat conduction through a
slab can be found in Refs.~\cite{Drysdale:1} and
\cite{Carslaw:1}. Four cases are examined here. In each, a slab of
thickness $L=0.1$~m is exposed on one face to an air temperature of $T_g=120$~$^\circ$C. The other face is insulated (adiabatic). The convective
heat transfer from the gas to the slab is $\dq_c'' = h \, (T_g - T_s)$,
where $h$ is constant, and $T_s$ is the slab face temperature. No thermal radiation is included.
\begin{center}
\begin{tabular}{|c|c|c|c|c|c|}
\hline
Case  &  $k$      & $\rho$       &  $c$          &   $h$         &  Bi     \\
      & (W/m/K)   & (kg/m$^3$)   &  (kJ/kg/K)    &  (W/m$^2$/K)  &  $hL/k$ \\ \hline \hline
A     &   0.1     & 100          &  1            &  100          &  100    \\ \hline
B     &   0.1     & 100          &  1            &  10           &  10     \\ \hline
C     &   1.0     & 1000         &  1            &  10           &  1      \\ \hline
D     &  10.0     & 10000        &  1            &  10           & 0.1     \\ \hline
\end{tabular}
\end{center}

\noindent
\begin{tabular*}{\textwidth}{lr}
\includegraphics[width=3.2in]{FIGURES/heat_conduction_Slab_A_Temperatures} &
\includegraphics[width=3.2in]{FIGURES/heat_conduction_Slab_B_Temperatures} \\
\includegraphics[width=3.2in]{FIGURES/heat_conduction_Slab_C_Temperatures} &
\includegraphics[width=3.2in]{FIGURES/heat_conduction_Slab_D_Temperatures}
\end{tabular*}


\clearpage

\subsection{Heat Conduction in Different Geometries With
Temperature-dependent Thermal Properties ({\bf heat\_conduction\_kc})}

This example demostrates the 1D heat conduction in cartesian,
cylindrical and spherical geometries with temperature-dependent
thermal properties. The reference results were computed using a
HEATING-code (version 7.3). HEATING is a multi-dimensional finite
difference general purpose heat transfer model~\cite{Childs}. In
cartesian and cylindrical cases, the results have also  been verified
using a commercial finite element solver ABAQUS.

The sample of homogenous material is initially at 0$^\circ$C and
at $t>0$ exposed to a gas at 700 $^\circ$C. A fixed heat transfer
coefficient of 10~W/Km$^2$ is assumed. The density
of the material is 10000 kg/m$^3$. The conductivity and specific heat
are functions of temperature with following values: $k(0)=0.10$ W/Km$^2$,
$k(200)=0.20$ W/Km$^2$, $c(0)=1.0$ kJ/kgK, $c(100)=1.2$ kJ/kgK,
$c(200)=1.0$ kJ/kgK. The thickness (radius) of the sample is 0.01 m. In
cartesian case, the back surface of the material is exposed to a gas
at 0$^\circ$C. In the Figure below, the solid lines are FDS results and open
symbols are the HEATING results. An example input with cylindrical
geometry looks like

\scriptsize
\begin{verbatim}
&MATL ID='MAT_1'
      EMISSIVITY = 0.0
      CONDUCTIVITY_RAMP='K_RAMP'
      SPECIFIC_HEAT_RAMP = 'C_RAMP'
      DENSITY=10000. /

&RAMP ID = 'K_RAMP' T=0,   F= 0.10 /
&RAMP ID = 'K_RAMP' T=100, F= 0.15 /
&RAMP ID = 'K_RAMP' T=200, F= 0.20 /
&RAMP ID = 'C_RAMP' T=0,   F= 1.00 /
&RAMP ID = 'C_RAMP' T=100, F= 1.20 /
&RAMP ID = 'C_RAMP' T=200, F= 1.00 /

&SURF ID='SLAB'
      STRETCH_FACTOR = 1.0
      GEOMETRY = 'CYLINDRICAL'
      MATL_ID='MAT_1'
      THICKNESS=0.01 /
\end{verbatim} \normalsize

\noindent
\begin{center}
\includegraphics[width=4.5in]{FIGURES/heat_conduction_kc}
\end{center}


\subsection{A Simple Two-Step Pyrolysis Example ({\bf two\_step\_solid\_reaction}) }

Consider  the set  of ordinary  differential equations  describing the
mass  fraction of  three  components of  a  solid material  undergoing
thermal degradation:
\begin{eqnarray}
    \frac{dY_a}{dt} &=&  -K_{ab} Y_a \nonumber  \\
    \frac{dY_b}{dt} &=&   K_{ab} Y_a  -  K_{bc}  Y_b  \\
    \frac{dY_c}{dt} &=&   K_{bc} Y_a
    \nonumber
\end{eqnarray}
where  the  mass  fraction  of  component  $a$  is  1  initially.  The
analytical solution is~\cite{Lautenberger:2006}
\begin{eqnarray}
    Y_a(t)    &=&    \exp(-K_{ab}t)    \nonumber   \\
    Y_b(t)    &=&    \frac{K_{ab}}{K_{bc}-K_{ab}} \exp(-K_{ab} t)  - \exp(-K_{bc} t) \\
    Y_c(t)    &=&    \left[   K_{ab}    (1-\exp(-K_{bc} t) )    +  K_{bc}*(\exp(-K_{ab} t) -1) \right] / (K_{ab}-K_{bc})
\end{eqnarray}
The analytical and  numerical solution
for the parameters $K_{ab} = 0.389$ and $K_{bc} = 0.262$ are shown here:

\begin{center}
\includegraphics[width=4.5in]{FIGURES/two_step_solid_reaction_Mass_Fractions}
\end{center}




\clearpage
\subsection{Wall Internal Radiation ({\bf wall\_internal\_radiation}) }


The radiative flux inside the walls is computed using a two-flux
model.  In this case, the accuracy of the two-flux model is tested in
the computation of emissive flux from a 0.10 m thick, homogenous layer
of material at temperature of 1273.15 K and at ambient temperature of
10 K. The absorption coefficient is varied to cover a range [0.01, 10]
of optical depths.

The exact solutions for radiative flux are the analytical solutions
of plane layer emission~\cite{Zeldovich:1}
\be
S(\tau) = S_b\left[1-2E_3(\tau)\right]
\ee
where $S_b = \sigma T^4$ is the black-body heat flux from
the radiating plane and $E_3(\tau)$ is the exponential
integral function (order 3) of optical depth $\tau$. The exact
solutions and FDS results are shown below

\begin{center}
\begin{tabular}{|c|c|c|} \hline
$\tau$ & $S(\tau) {\rm (kW/m^2)}$ & FDS $\rm(kW/m^2)$ \\ \hline\hline
0.01  & 2.897       & 2.950 \\
0.1   & 24.94       & 26.98 \\
0.5   & 82.95       & 93.90 \\
1.0   & 116.3       & 128.4 \\
10.   & 149.0       & 149.0 \\ \hline
\end{tabular}
\end{center}


\clearpage
\subsection{A Liquid Pool Fire ({\bf ethanol\_pan}) }

In this example, a steel pan (0.7~m by 0.8~m) is filled with a thin layer (about 5~L, 9~mm) of ethanol, which burns out within about 10~min. This case tests a number of features -- burning
liquids, multiple layers of solids/liquids, and, most importantly, the absorption coefficient of the liquid. The pyrolysis models in FDS prior to version 5
assumed that radiative feedback from the fire and hot gases within a compartment were absorbed at the surface. In reality, this energy is absorbed in
depth; the extent of which is characterized by the absorption coefficient, $\kappa$. This is a property of the liquid, as well as the gaseous vapors. FDS now uses
an absorption coefficient for both the gas and solid/liquid phases. Here are the input lines that describe the properties of ethanol, and the pan in which it lies:

\scriptsize
\begin{verbatim}
&MATL ID = 'ETHANOL LIQUID'
      EMISSIVITY = 1.0
      NU_FUEL =0.97
      HEAT_OF_REACTION=880.
      CONDUCTIVITY=0.17
      SPECIFIC_HEAT= 2.45
      DENSITY= 787.
      ABSORPTION_COEFFICIENT = 40.
      BOILING_TEMPERATURE=76. /

&SURF ID='ETHANOL POOL'
      FYI = '4 kg of ethanol in a 0.7 m x 0.8 m pan'
      COLOR = 'YELLOW'
      MATL_ID = 'ETHANOL LIQUID','STEEL','CONCRETE'
      THICKNESS=0.0091,0.001,0.05
      TMP_INNER = 18. /
\end{verbatim} \normalsize

\noindent
The results of three calculations are shown below, each identical except for the value of the \\ {\ct ABSORPTION\_COEFFICIENT}. The results of a single experiment
are also shown, courtesy of Ian Thomas, Victoria University, Australia~\cite{Thomas:JFPE}.


\begin{center}
\includegraphics[width=4.5in]{FIGURES/ethanol_pan_HRR}
\end{center}



\clearpage
\subsection{A Thermoplastic ({\bf thermoplastic}) }

This example just exercises the solid phase algorithm in
FDS. Essentially, the gas phase is shut off except for the imposition
of a 50~kW/m$^2$ ``virtual'' heat flux. The solid in this example is a
slab of plastic, similar in composition to PMMA.

\scriptsize
\begin{verbatim}
&MATL ID='PMMA'
      CONDUCTIVITY=0.25
      SPECIFIC_HEAT=1.0
      DENSITY=500.
      N_REACTIONS=1
      NU_FUEL=1.
      HEAT_OF_REACTION=1578.
      HEAT_OF_COMBUSTION=25200.
      REFERENCE_TEMPERATURE=330. /

&SURF ID='PMMA SLAB'
      COLOR='BLACK'
      MATL_ID='PMMA'
      THICKNESS=0.025
      EXTERNAL_FLUX=50. /  External Flux is ONLY for this simple demo exercise
\end{verbatim} \normalsize

\noindent
The material undergoes only one reaction -- the conversion of solid to
gaseous fuel vapor. In this case, the {\ct HEAT\_OF\_REACTION} is
essentially the latent heat of vaporization. The {\ct
REFERENCE\_TEMPERATURE} indicates that the reaction is to occur at a
(default) rate of 0.1~s$^{-1}$ at 330~$^\circ$C. The {\ct
HEAT\_OF\_COMBUSTION} refers to the combustion of the gaseous fuel
vapor, which does not occur in this example.

\noindent
\begin{tabular*}{\textwidth}{lr}
\includegraphics[width=3.2in]{FIGURES/thermoplastic_Surface_Temperature} &
\includegraphics[width=3.2in]{FIGURES/thermoplastic_Burning_Rate}
\end{tabular*}

The figures above show that both the temperature and burning rate of the thermoplastic are more or less constant over
the burning phase. The slab ``burns away'' after about 10~min.






\clearpage
\subsection{A Charring Solid ({\bf charring\_solid}) }

This example just exercises the solid phase algorithm in FDS. Essentially, the gas phase is shut off except for the imposition of a 50~kW/m$^2$ ``virtual''
heat flux. The reaction mechanism is fairly complicated, as it includes various solids like cellulose, char, and ash. Each are input via {\ct MATL} lines
as follows:

\scriptsize
\begin{verbatim}
&SURF ID='SPRUCE'
      STRETCH_FACTOR = 1.
      CELL_SIZE_FACTOR = 0.5
      MATL_ID(1,1:3) = 'CELLULOSE','WATER','LIGNIN'
      MATL_MASS_FRACTION(1,1:3) = 0.70,0.1,0.20
      MATL_ID(2,1) = 'CASI'
      THICKNESS(1:2)  =  0.01,0.01
      EXTERNAL_FLUX = 50. /

&MATL ID               = 'CELLULOSE'
      CONDUCTIVITY_RAMP = 'k_cell'
      SPECIFIC_HEAT    = 2.3
      DENSITY          = 400.
      N_REACTIONS      = 1
      A                = 2.8E19
      E                = 2.424E5
      HEAT_OF_REACTION = 0.
      NU_RESIDUE       = 1.0
      RESIDUE          = 'ACTIVE'/

&MATL ID                    = 'ACTIVE'
      EMISSIVITY            = 1.0
      CONDUCTIVITY_RAMP     = 'k_cell'
      SPECIFIC_HEAT         = 2.3
      DENSITY               = 400.
      N_REACTIONS           = 2
      A(1:2)                = 1.3E10,  3.23E14
      E(1:2)                = 1.505E5, 1.965E5
      HEAT_OF_REACTION(1:2) = 418.,    418.
      NU_RESIDUE(1:2)       = 0.35,    0.0
      NU_FUEL(1:2)          = 0.65,    1.0
      RESIDUE(1)            = 'CHAR' /

&MATL ID               = 'WATER'
      EMISSIVITY       = 1.0
      DENSITY          = 1000.
      CONDUCTIVITY     = 0.6
      SPECIFIC_HEAT    = 4.19
      N_REACTIONS      = 1
      A                = 1E20
      E                = 1.62E+05
      NU_WATER         = 1.0
      HEAT_OF_REACTION = 2260. /

&MATL ID                = 'CASI'
      CONDUCTIVITY_RAMP = 'k_CASI'
      DENSITY           = 200.
      SPECIFIC_HEAT     = 1.0 /

&MATL ID               = 'LIGNIN'
      EMISSIVITY       = 1.0
      DENSITY          = 550.
      CONDUCTIVITY     = 0.1
      SPECIFIC_HEAT    = 1.1 /

&MATL ID               = 'CHAR'
      EMISSIVITY       = 1.0
      DENSITY          = 140.
      CONDUCTIVITY_RAMP = 'k_char'
      SPECIFIC_HEAT    = 1.1 /

&RAMP ID='k_cell', T= 20., F=0.15 /
&RAMP ID='k_cell', T=500., F=0.29 /

&RAMP ID='k_char', T= 20., F=0.08 /
&RAMP ID='k_char', T=900., F=0.25 /

&RAMP ID='k_CASI', T= 20., F=0.06 /
&RAMP ID='k_CASI', T=400., F=0.25 /
\end{verbatim} \normalsize

\noindent
Note the edition of the parameter {\ct EXTERNAL\_FLUX} on the {\ct SURF} line. This produces a 50~kW/m$^2$ flux on the sample, without any
additional input lines in the file. It is just to test the solid phase model and should not be copied into an actual fire simulation.

The figures below show the surface temperature and burning rate of the wood under the 50~kW/m$^2$ external heat flux. The burning rate peaks at the
start of the simulation, decreases throughout the burning phase, and then peaks again at the end due to presence of an external backing material. The
initial peak is typical of char-forming solids.

\noindent
\begin{tabular*}{\textwidth}{lr}
\includegraphics[width=3.2in]{FIGURES/charring_solid_Surface_Temperature} &
\includegraphics[width=3.2in]{FIGURES/charring_solid_Burning_Rate}
\end{tabular*}





\clearpage
\subsection{Testing the ``Burn-Away'' Feature ({\bf box\_burn\_away}) }

This is a silly example of a solid block of ``foam'' that is ignited
and burns until it is completely consumed.  The properties of the
block of foam were chosen simply to assure a quick calculation. The
objective of the test is to check that the integrated heat release
rate is consistent with the material properties of the block.  The
block is 0.4~m on a side, with a density of 20~kg/m$^3$. Its heat of
combustion is 20000~kJ/kg. The ignitor is a 10~kW burner placed beside
the block. The integrated heat release rate for a 120 s calculation
ought to be:
\be
(0.4)^3 \; \hbox{m}^3 \times 20 \; \hbox{kg/m}^3
\times 20000 \; \hbox{kJ/kg} + 10 \; \hbox{kW} \times 120 \; \hbox{s} =
26.8 \; \hbox{MJ}
\ee



\clearpage
\subsection{A Couch Fire ({\bf couch}) }

In residential fires, upholstered furniture makes up a significant fraction of the combustible load. A single couch can generate several
megawatts of energy and sometimes lead to compartment flashover. Modeling a couch fire requires a simplification of its structure and materials.
At the very least, we want the upholstery to be described as fabric covering foam:

\footnotesize
\begin{verbatim}
&MATL ID                    = 'FABRIC'
      FYI                   = 'Properties completely fabricated'
      SPECIFIC_HEAT         = 1.0
      CONDUCTIVITY          = 0.1
      DENSITY               = 100.0
      N_REACTIONS           = 1
      NU_FUEL               = 1.
      REFERENCE_TEMPERATURE = 350.
      HEAT_OF_REACTION      = 3000.
      HEAT_OF_COMBUSTION    = 15000. /

&MATL ID                    = 'FOAM'
      FYI                   = 'Properties completely fabricated'
      SPECIFIC_HEAT         = 1.0
      CONDUCTIVITY          = 0.05
      DENSITY               = 40.0
      N_REACTIONS           = 1
      NU_FUEL               = 1.
      REFERENCE_TEMPERATURE = 350.
      HEAT_OF_REACTION      = 1500.
      HEAT_OF_COMBUSTION    = 30000. /

&SURF ID             = 'UPHOLSTERY'
      FYI            = 'Properties completely fabricated'
      COLOR          = 'PURPLE'
      BURN_AWAY      = .TRUE.
      MATL_ID(1:2,1) = 'FABRIC','FOAM'
      THICKNESS(1:2) = 0.002,0.1
      PART_ID        = 'smoke' /
\end{verbatim} \normalsize

\noindent
Both the fabric and the foam decompose into fuel gases via single-step reactions. The fuel gases from each have different
composition and heats of combustion. FDS automatically adjusts the mass loss rate of each so that the ``effective'' fuel gas
is that specified by the user on the {\ct REAC} line. The attribute {\ct BURN\_AWAY} forces FDS to break up the couch into
individual cell-sized blocks that will disappear from the calculation as soon as the fuel is exhausted. The surface is specified
as consisting of two layers, with a thickness of 2~mm for the {\ct FABRIC} and 10~cm for the {\ct FOAM}. The 10~cm is chosen to be the
same as the mesh cell size.





\clearpage
\subsection{Flame Spread along a Cable Tray ({\bf cable\_tray}) }

A common combustible in industrial settings are power, control, and instrument cables. The cables may be bundled in a variety of conduits; the most
common of which is a ladder-like ``tray.'' From the point of view of FDS, the pile of cables in a tray is a composite of a variety of plastics, insulators,
and metal, usually copper. Here is one way to describe a tray of cables:

\footnotesize
\begin{verbatim}
&MATL ID                    = 'PLASTIC'
      CONDUCTIVITY          = 0.2
      SPECIFIC_HEAT         = 1.5
      DENSITY               = 1500.
      N_REACTIONS           = 1
      HEAT_OF_REACTION      = 3000.
      HEAT_OF_COMBUSTION    = 25000.
      REFERENCE_TEMPERATURE = 400.
      NU_FUEL               = 1.0 /

&MATL ID            = 'COPPER'
      SPECIFIC_HEAT = 0.38
      CONDUCTIVITY  = 387.
      DENSITY       = 8940.   /

&SURF ID                        = 'Loose Cable'
      COLOR                     = 'IVORY BLACK'
      MATL_ID(1,1:2)            = 'PLASTIC','COPPER'
      MATL_MASS_FRACTION(1,1:2) = 0.4,0.6
      BACKING                   = 'EXPOSED'
      THICKNESS                 = 0.02 /

&OBST XB=-2.00, 2.00,-0.14, 0.14, 0.51, 0.55, SURF_ID='Loose Cable' /

&OBST XB=-2.00, 2.00,-0.15,-0.15, 0.50, 0.60, SURF_ID='SHEET METAL' / Tray Side
&OBST XB=-2.00, 2.00, 0.15, 0.15, 0.50, 0.60, SURF_ID='SHEET METAL' / Tray Side
&OBST XB=-1.95,-1.90,-0.15, 0.15, 0.50, 0.50, SURF_ID='SHEET METAL' / Rung
&OBST XB=-1.60,-1.55,-0.15, 0.15, 0.50, 0.50, SURF_ID='SHEET METAL' / Rung
...
&OBST XB= 1.90, 1.95,-0.15, 0.15, 0.50, 0.50, SURF_ID='SHEET METAL' / Rung
\end{verbatim} \normalsize

\noindent
The pile of cables is assumed to be a solid slab, 28~cm wide and 2~cm deep. The tray is slightly wider and deeper, and because it is listed second in
the input file, its surface properties take precedence wherever the cable slab and tray coincide. The mesh cells in this example are 5~cm on a side, but the
heat transfer within the cable slab are governed by the 2~cm {\ct THICKNESS}. The slab is 60~\% copper, by mass. Note that we are not assuming multiple
layers in this example -- the slab is a single layer composite of plastic and copper. The plastic burns at about 400~$^\circ$, but the copper remains. Thus,
the cable does not ``burn away.''

The point of this test case is merely to propose a simple model of flame spread along a tray of assorted cable. Detailed thermo-physical property data for
industrial-grade cable is usually not available, and even if it were, it would probably not improve upon the given model. The properties given in this example
are almost completely fabricated. What is important here are the
{\ct HEAT\_OF\_REACTION} and {\ct REFERENCE\_TEMPERATURE}, obtained in most cases by a bench-scale measurement device like the cone calorimeter.






\clearpage

\section{Detectors}

\subsection{Aspiration Detector ({\bf beam\_detector}) }

A 10~m x 10~m x 4~m compartment is filled with 0.006~kg/kg of {\ct MIXTURE\_FRACTION\_2} with the default soot yield
0.01~kg/kg.  This results in an initial soot density of 71.9~mg/m$^3$ which using the default extinction coefficient of
8700~m$^2$/kg results in an optical depth of 0.626~m$^{-1}$.  The compartment has a series of obstructions placed at varying depths that are
multiples of 1~m.  Using the correlation for the output quantity {\ct visibility}, one obtains a visibility distance of 4.8~m.
When viewing the smoke levels with Smokeview, one can just barely see the fifth obstacle which is at a distance of 5~m.
Smokeview, therefore, is properly displaying the obscuration of the initial soot density.  Three beam detectors are also
placed in the compartment.  These all have a path length of 10~m, but are at different orientations within the compartment.
Using the optical depth of 0.626~m$^{-1}$ and the path length of 10~m, the expected total obscuration is 99.81~\%, which is the
result computed by FDS for each of the three detectors.
\begin{tabular*}{\textwidth}{lr}
\includegraphics[width=3.2in]{FIGURES/beam_detector_nosmoke} &
\includegraphics[width=3.2in]{FIGURES/beam_detector_smoke}
\end{tabular*}

\clearpage

\subsection{Aspiration Detector ({\bf aspiration\_detector}) }

A cubical compartment, 2~m on a side with a fire has a three sampling location aspiration system.  The three locations
have flow rates of 0.1, 0.5, and 0.8~kg/s, respectively, and transport times of 0.2, 0.1, and 0.3~s, respectively.  No bypass flow rate is
specified for the aspiration detector.  The input file fixes the initial time step to 0.01~s so that the initial output
times in the {\bf aspiration\_detector\_devc.csv} file will line up exactly with the transport times.  At 0.75~s, when
FDS begins reducing the time step below 0.01~s, the time delayed soot densities at the three sampling locations are
$7.4x10^{-6}$~kg/m$^3$, $9.5x10^{-4}$~kg/m$^3$, and $1.6x10^{-18}$~kg/m$^3$, respectively.  Using these values along with the
respective flow rates results in a detector obscuration of 0.000823~\%/m which is the same obscuration as predicted by FDS.

\clearpage

\section{Droplets and Sprays}

This section considers cases involving evaporating droplets, both water and fuel.


\subsection{Water Droplet Evaporation ({\bf water\_evaporation}) }

The case called {\bf water\_evaporation} is nothing more than stationary water droplets in an adiabatic box with dimensions of 1~m on a side. The air
within the box is stirred to maintain uniform conditions, and there are no leaks or heat losses. The initial air
 temperature is 40~$^\circ$C.
Initially, the droplets have a median volumetric diameter of 100~$\mu$m, a temperature of 90~$^\circ$C, and a total mass of 0.02~kg.
It is expected that a steady-state will be achieved after several minutes.  The initial energy content, sum of the
air and water enthalpies, of the box is 360,000~kJ.  After a short period of time, 0.0141 kg of water evaporate and the box
reaches an equilibrium temperature of 16.2~$^\circ$C, see the figure below.  At this point the energy content of the box is 372,000~kJ
 or a 3~\% error.  At 16.2~$^\circ$C, the expected evaporation is 0.0142~kg.

\begin{center}
\includegraphics[width=4.5in]{FIGURES/water_evaporation}
\end{center}


%\subsection{Fuel Droplet Evaporation ({\bf fuel\_evaporation}) }

%In the exact same geometry as the above example, stationary droplets of heptane at ambient temperature vaporize and burn.




\clearpage

\subsection{A Liquid Fuel Spray Burner ({\bf spray\_burner}) }

Controlled fire experiments are often conducted using a spray burner,
where a liquid fuel is sprayed out of a nozzle and ignited. In this
example ({\bf spray\_burner.data}), heptane from two nozzles is
sprayed downwards into a steel pan.  The flow rate is increased
linearly so that the fire grows to 2~MW in 20 s, burns steadily for
another 20 s, and then ramps down linearly in 20 s. The key input
parameters are given here:

\footnotesize
\begin{verbatim}
&DEVC ID='nozzle_1', XYZ=4.0,-.3,0.5, PROP_ID='nozzle', QUANTITY='TIME', SETPOINT=0. /
&DEVC ID='nozzle_2', XYZ=4.0,0.3,0.5, PROP_ID='nozzle', QUANTITY='TIME', SETPOINT=0. /

&PART ID='heptane droplets', FUEL=.TRUE., VAPORIZATION_TEMPERATURE=98.,
      HEAT_OF_VAPORIZATION=316., SPECIFIC_HEAT=2.25, DENSITY=688.,
      QUANTITIES(1:2)='DIAMETER','DROPLET_TEMPERATURE',
      DROPLETS_PER_SECOND=2000, DIAMETER=1000., HEAT_OF_COMBUSTION=44500.,
      DT_INSERT=0.02, SAMPLING_FACTOR=1 /

&PROP ID='nozzle', CLASS='NOZZLE', PART_ID='heptane droplets',
      FLOW_RATE=1.96, FLOW_RAMP='fuel', DROPLET_VELOCITY=10.,
      SPRAY_ANGLE=0.,30.   /
&RAMP ID='fuel', T= 0.0, F=0.0  /
&RAMP ID='fuel', T=20.0, F=1.0  /
&RAMP ID='fuel', T=40.0, F=1.0  /
&RAMP ID='fuel', T=60.0, F=0.0  /
\end{verbatim}
\normalsize
Many of these parameters are self-explanatory and the
units are given in the User's Guide~\cite{FDS_Users_Guide_5}. Note
that a 2~MW fire is achieved via 2 nozzles flowing heptane at
1.96~L/min each:
\be 2 \times 1.96 \; \frac{\hbox{L}}{\hbox{min}} \times \frac{1}{60} \; \frac{\hbox{min}}{\hbox{s}} \times 688 \;
\frac{\hbox{kg}}{\hbox{m}^3} \times \frac{1}{1000} \; \frac{\hbox{m}^3}{\hbox{L}} \times 44500 \;
\frac{\hbox{kJ}}{\hbox{kg}} = 2000 \; \hbox{kW} \ee
The parameter {\ct HEAT\_OF\_COMBUSTION} over-rides that for the overall reaction scheme. Thus, if other droplets
or solid objects have different heats of combustion, the effective burning rates are adjusted so that the
total heat release rate is that which the user expects. However, exercises like this ought to be conducted just to
ensure that this is the case. The HRR curve for this example is given here:

\begin{center}
\includegraphics[width=4.5in]{FIGURES/spray_burner_HRR}
\end{center}



\clearpage

\subsection{Measuring Water Flux ({\bf bucket\_test}) }

A common way of measuring the spray distribution for any fire sprinkler is called a ``bucket test.''  Usually, one or several sprinklers is
mounted a specified distance above an array of catch bins, the water flows for a given period of time, and the water flux distribution is calculated
from the accumulated water mass in each bin. In the test case {\bf bucket\_test}, a single sprinkler is mounted 10~cm below a 5~m ceiling. Water
flows for 5~s at a constant rate of 60~L/min.  The simulation continues for another 5~s to allow water drops time to reach the
floor. The total mass of water discharged is
\be
  \mathrm{ 60 \; \frac{L}{min} \times 1 \; \frac{kg}{L} \times \frac{1}{60} \; \frac{min}{s} \times 5 \; s = 5 \; kg }
\ee
In the simulation, the boundary quantity {\ct water\_drops\_AMPUA} (Accumulated Mass Per Unit Area) records the total water mass per unit area (kg/m$^2$), analogous to
actual buckets the size of a grid cell.
Summing the values of {\ct water\_drops\_AMPUA} over the entire floor yields 4.96~kg. Where is the missing water? Some droplets evaporate, and some droplets
fly beyond the computational domain. Also, there remain a small number of suspended drops at the end of the simulation. Note that
there are no actual ``buckets'' in the simulation.

The accumulated water mass at the floor is extracted from the boundary ({\ct BNDF}) file via the command line program {\bf fds2ascii}. Here is a transcript of the session
used to convert the binary FDS output file into ASCII format:

\footnotesize
\begin{verbatim}
>> fds2ascii
  Enter Job ID string (CHID):
bucket_test
  What type of file to parse?
  PL3D file? Enter 1
  SLCF file? Enter 2
  BNDF file? Enter 3
3
  Enter Sampling Factor for Data?
  (1 for all data, 2 for every other point, etc.)
1
  Limit the domain size? (y or n)
y
  Enter min/max x, y and z
-5 5 -5 5 0 1
  1   MESH  1, water_drops_AMPUA
  Enter starting and ending time for averaging (s)
9 10
  Enter orientation: (plus or minus 1, 2 or 3)
3
  Enter number of variables
1
 Enter boundary file index for variable 1
1
 Enter output file name:
bucket_test_fds2ascii.csv
  Writing to file...      bucket_test_fds2ascii.csv
\end{verbatim}

\normalsize \noindent


\subsection{Complex Spray Patterns ({\bf bucket\_test\_2}) }

The test case from the prior section is modified to create two jets of water.

\footnotesize
\begin{verbatim}
&PROP ID='K-11', QUANTITY='SPRINKLER LINK TEMPERATURE', OFFSET=0.10, PART_ID='water_drops',
      FLOW_RATE=60.,SPRAY_PATTERN_TABLE='TABLE1', SMOKEVIEW_ID='sprinkler_upright' /

&TABL ID='TABLE1',TABLE_DATA=30,31,0,1,5,0.2/
&TABL ID='TABLE1',TABLE_DATA=30,31,179,180,5,0.8/
\end{verbatim}
\normalsize

The jets are seperated by 180 degrees.
The jet in the {\ct -x} direction is given one quarter the flow rate of the jet in the {\ct +x} direction.

Viewing the particles in Smokeview shows two distinct jets of droplets in opposite directions.
Following the post processing instructions above,  you can observe that
the {\ct +x} jet does have four times the flow of the {\ct -x} jet.

\clearpage

\section{General Functionality}

This section contains a variety of simple tests to check the functionality of the code. These examples are good demonstrations of how to
make things happen in FDS.


\subsection{Creating and Removing {\ct HOLE}s and {\ct OBST}ructions ({\bf create\_remove}) }

It is often convenient to create or remove solid obstructions, or conversely, remove or create empty holes. They are essentially the same thing as far as the logic in
FDS is concerned, but the input can be tricky. To avoid confusion, here is an example of holes and obstructions being created and removed.

\vspace{0.2in}
\tiny
\noindent
\begin{minipage}{1.4in}
\includegraphics[height=1.4in]{FIGURES/create_remove}
\end{minipage}
\hfill
\begin{minipage}{5.5in}
\begin{verbatim}
&HOLE XB=0.25,0.45,0.20,0.30,0.20,0.30, COLOR='RED',   DEVC_ID='timer 1' /
&HOLE XB=0.25,0.45,0.70,0.80,0.70,0.80, COLOR='GREEN', DEVC_ID='timer 2' /

&OBST XB=0.70,0.80,0.20,0.30,0.20,0.30, COLOR='BLUE',  DEVC_ID='timer 3' /
&OBST XB=0.70,0.80,0.60,0.70,0.60,0.70, COLOR='PINK',  DEVC_ID='timer 4' /

&DEVC XYZ=0.1,0.1,0.1, ID='timer 1', SETPOINT= 1.0, QUANTITY='TIME', INITIAL_STATE=.FALSE./
&DEVC XYZ=0.2,0.1,0.1, ID='timer 2', SETPOINT= 2.0, QUANTITY='TIME', INITIAL_STATE=.TRUE. /
&DEVC XYZ=0.1,0.1,0.1, ID='timer 3', SETPOINT= 3.0, QUANTITY='TIME', INITIAL_STATE=.FALSE./
&DEVC XYZ=0.2,0.1,0.1, ID='timer 4', SETPOINT= 4.0, QUANTITY='TIME', INITIAL_STATE=.TRUE./
\end{verbatim}
\end{minipage}
\normalsize








\chapter{Sensitivity Analysis}

A sensitivity  analysis considers the  extent to which  uncertainty in
model  inputs influences  model output.  Model parameters  can  be the
physical properties of solids  and gases, boundary conditions, initial
conditions, {\em  etc.} The parameters  can also be  purely numerical,
like the size  of the numerical grid. FDS  typically requires the user
to  provide several  dozen different  types of  input  parameters that
describe the geometry, materials,  combustion phenomena, {\em etc.} By
design,  the user  is  not expected  to  provide numerical  parameters
besides the grid size,  although the optional numerical parameters are
described in both the Technical Reference Guide and the User's Guide.

FDS does not  limit the range of most of  the input parameters because
applications often push beyond the  range for which the model has been
validated.  FDS is still used  for research at NIST and elsewhere, and
the developers do not presume to know in all cases what the acceptable
range  of   any  parameter  is.  Plus,  FDS   solves  the  fundamental
conservation  equations  and  is   much  less  susceptible  to  errors
resulting  from  input parameters  that  stray  beyond  the limits  of
simpler empirical models.  However, the user is warned  that he/she is
responsible for  the prescription of all parameters.   The FDS manuals
can only provide guidance.

The grid size is the  most important numerical parameter in the model,
as it  dictates the spatial  and temporal accuracy of  the discretized
partial  differential equations.  The heat  release rate  is  the most
important physical parameter,  as it is the source  term in the energy
equation. Property data, like  the thermal conductivity, density, heat
of vaporization,  heat capacity, {\em  etc.}, ought to be  assessed in
terms of their influence on  the heat release rate. Validation studies
have shown that FDS predicts well the transport of heat and smoke when
the HRR is prescribed. In  such cases, minor changes in the properties
of  bounding  surfaces  do  not  have  a  significant  impact  on  the
results. However, when the HRR is not prescribed, but rather predicted
by the  model using  the thermophysical properties  of the  fuels, the
model output is sensitive to even minor changes in these properties.

The sensitivity  analyses described in this chapter  are all performed
in basically the same way. For a given scenario, best estimates of all
the  relevant  physical  and  numerical  parameters are  made,  and  a
``baseline'' simulation is performed. Then, one by one, parameters are
varied by a given percentage, and the changes in predicted results are
recorded.  This is  the simplest  form of  sensitivity  analysis. More
sophisticated  techniques that involve  the simultaneous  variation of
several  parameters  are impractical  with  a  CFD  model because  the
computation time is too long and the number of parameters too large to
perform  the  necessary  number  of calculations  to  generate  decent
statistics.



\section{Grid Sensitivity}

\label{gridsen}

The most  important decision made by a  model user is the  size of the
numerical grid. In  general, the finer the numerical  grid, the better
the numerical solution of  the equations. FDS is second-order accurate
in  space and  time,  meaning that  halving  the grid  cell size  will
decrease  the discretization  error in  the governing  equations  by a
factor  of 4.  Because  of  the non-linearity  of  the equations,  the
decrease in discretization error does not necessarily translate into a
comparable decrease  in the error of  a given FDS  output quantity. To
find out  what effect a  finer grid has  on the solution,  model users
usually  perform some  form of  grid  sensitivity study  in which  the
numerical grid  is systematically refined until  the output quantities
do not change  appreciably with each refinement. Of  course, with each
halving of  the grid cell size,  the time required  for the simulation
increases by  a factor of $2^4=16$  (a factor of two  for each spatial
coordinate, plus  time). In  the end, a  compromise is  struck between
model accuracy and computer capacity.

Some   grid    sensitivity   studies   have    been   documented   and
published. Since FDS was  first publicly released in 2000, significant
changes   in  the   combustion  and   radiation  routines   have  been
incorporated into the model. However, the basic transport algorithm is
the  same, as  is  the  critical importance  of  grid sensitivity.  In
compiling   sensitivity  studies,   only  those   that   examined  the
sensitivity of routines no longer used have been excluded.

As part of  a project to evaluate  the use of FDS version  1 for large
scale   mechanically  ventilated   enclosures,  Friday~\cite{Friday:1}
performed a  sensitivity analysis to find  the approximate calculation
time based on varying grid sizes. A propylene fire with a nominal heat
release rate was  modeled in FDS. There was  no mechanical ventilation
and  the fire  was assumed  to grow  as a  function of  the  time from
ignition  squared.  The  compartment  was   a  3~m  by  3~m  by  6.1~m
space. Temperatures  were sampled 12~cm  below the ceiling.  Four grid
sizes   were   chosen  for   the   analysis:   30~cm,  15~cm,   10~cm,
7.5~cm. Temperature  estimates were  not found to  change dramatically
with different grid dimensions.

Using FDS  version 1, Bounagui {\em  et al.}~\cite{Bounagui:1} studied
the effect of grid size on simulation results to determine the nominal
grid size for future work. A propane burner 0.1~m by 0.1~m was modeled
with  a heat  release rate  of  1500~kW. The  grid sizes  used in  the
modeling are  given in Table~\ref{gridsize}. The results  of the model
were         compared         against        Heskestad's         plume
correlation.   Table~\ref{Heskestad}   shows   the  plume   centerline
temperature comparisons.  Obviously, the smaller  grids provide better
agreements among predictive methods.  A similar analysis was performed
using  Alpert's ceiling  jet correlation~\cite{SFPE:Alpert}  that also
showed better  predictions and reasonable agreement  with smaller grid
sizes.  In a  related study,  Bounagui {\em  et al.}~\cite{Bounagui:2}
used  FDS to  evaluate  the emergency  ventilation  strategies in  the
Louis-Hippolyte-La Fontaine Tunnel in Montreal, Canada.

\begin{table}[ht]
\begin{center}
\caption{Grid    Sizes    for    Computational    Domain    used    in
Ref.~\cite{Bounagui:1}} \label{gridsize} \vspace{0.1in}
\begin{tabular}{|c|c|}
\hline Cases&Grid Size (m)\\ \hline \hline Case 1&0.20 x 0.20 x 0.20\\
\hline Case 2&0.14 x 0.14 x  0.14\\ \hline Case 3&0.10 x 0.10 x 0.10\\
\hline Case 4&0.08 x 0.08 x 0.08\\ \hline
\end{tabular}
\end{center}
\end{table}

\begin{table}[ht]
\begin{center}
\label{Heskestad} \caption{Plume Centerline Temperature Comparisons~\cite{Bounagui:1}} \vspace{0.1in}
\begin{tabular}{|c|c|c|c|c|}        \hline
Heskestad & Case 1 & Case 2 & Case  3 & Case 4 \\ \hline 839 C & 479 C
& 593 C & 962 C & 967 C \\ \hline
\end{tabular}
\end{center}
\end{table}

Xin~\cite{Xin:NFPA2004}  used FDS  to  model a  methane fueled  square
burner (1~m  by 1~m) in  the open. Engineering correlations  for plume
centerline temperature and velocity  profiles were compared with model
predictions to assess the influence of the numerical grid and the size
of the computational domain. The  results showed that FDS is sensitive
to grid size effects, especially  in the region near the fuel surface,
and domain size  effects when the domain width is  less than twice the
plume  width.  FDS  uses   a  constant  pressure  assumption  at  open
boundaries.  This assumption  will affect  the plume  behavior  if the
boundary of the computational domain is too close to the plume.

Ierardi  and Barnett~\cite{Ierardi:1} used  FDS version  3 to  model a
0.3~m square methane diffusion burner with heat release rate values in
the range of 14.4~kW to 57.5~kW. The physical domain used was 0.6~m by
0.6~m with uniform grid spacings of  15, 10, 7.5, 5, 3, 1.5~cm for all
three coordinate  directions. For both  fire sizes, a grid  spacing of
1.5~cm  was found  to  provide  the best  agreement  when compared  to
McCaffrey's     centerline    plume    temperature     and    velocity
correlations~\cite{SFPE:Heskestad}.  Two similar  scenarios  that form
the basis for Alpert's ceiling  jet correlation were also modeled with
FDS. The first scenario was a  1~m by 1~m, 670~kW ethanol fire under a
7~m   high  unconfined   ceiling.   The  planar   dimensions  of   the
computational domain were 14~m by  14~m. Four uniform grid spacings of
50, 33.3, 25, and 20~cm were  used in the modeling. The best agreement
for  maximum  ceiling  jet  temperature  was  with  the  33.3~cm  grid
spacing. The best  agreement for maximum ceiling jet  velocity was for
the  50~cm grid  spacing. The  second scenario  was a  0.6~m  by 0.6~m
1000~kW ethanol fire under a 7.2~m high unconfined ceiling. The planar
dimensions of  the computational domain  were 14.4~m by  14.4~m. Three
uniform  grid  spacings  of  60,  30,  and  20~cm  were  used  in  the
modeling. The  results show that  the 60~cm grid spacing  exhibits the
best  agreement with  the correlations  for both  maximum  ceiling jet
temperature and velocity on a qualitative basis.

Petterson~\cite{Petterson:1} also completed work assessing the optimal
grid size for FDS version 2. The FDS model predictions of varying grid
sizes were  compared to two separate fire  experiments: The University
of Canterbury  McLeans Island  Tests and the  US Navy Hangar  Tests in
Hawaii.  The first  set  of  tests utilized  a  room with  approximate
dimensions of  2.4~m by  3.6~m by  2.4~m and fire  sizes of  55~kW and
110~kW. The  Navy Hangar  tests were performed  in a  hangar measuring
98~m by 74~m by 15~m in height and had fires in the range of 5.5~MW to
6.6~MW. The results  of this study indicate that  FDS simulations with
grids of 0.15~m had temperature predictions as accurate as models with
grids as  small as 0.10~m. Each  of these grid  sizes produced results
within   15~\%   of   the   University   of   Canterbury   temperature
measurements. The 0.30~m grid  produced less accurate results. For the
comparison of the Navy Hangar tests, grid sizes ranging from 0.60~m to
1.80~m yielded results of comparable accuracy.

Musser~{\em et  al.}~\cite{Musser:1} investigated  the use of  FDS for
course grid  modeling of non-fire and fire  scenarios. Determining the
appropriate  grid  size was  found  to  be  especially important  with
respect  to heat  transfer  at heated  surfaces.  The convective  heat
transfer  from the  heated surfaces  was most  accurate when  the near
surface grid cells were smaller than the depth of the thermal boundary
layer.  However, a  finer grid  size  produced better  results at  the
expense of computational time. Accurate contaminant dispersal modeling
required a significantly finer grid. The results of her study indicate
that  non-fire simulations  can be  completed more  quickly  than fire
simulations because  the time  step is not  limited by the  large flow
speeds in a fire plume.


\section{Sensitivity of Large Eddy Simulation Parameters}

FDS  uses the  Smagorinsky form  of  the Large  Eddy Simulation  (LES)
technique.  This  means  that   instead  of  using  the  actual  fluid
viscosity, the model uses a viscosity of the form \be \mu_{\hbox{\tiny
LES}} =  \rho \,  (C_s\, \Delta)^2  \, |S|^\ha \ee  where $C_s$  is an
empirical constant, $\Delta$ is a length on the order of the size of a
grid  cell,  and  the  deformation   term  $|S|$  is  related  to  the
Dissipation Function, given by Eq.~(\ref{dissipation}). Related to the
``turbulent  viscosity'' are  comparable expressions  for  the thermal
conductivity  and  material diffusivity:  \be  k_{\hbox{\tiny LES}}  =
\frac{\mu_{\hbox{\tiny  LES}}   \;  c_p}{\PR}  \quad   ;  \quad  (\rho
D)_{\hbox{\tiny  LES}}  =\frac{\mu_{\hbox{\tiny  LES}}}{\SC}  \ee  The
Prandtl  number  $\PR$  and  the  Schmidt number  $\SC$  are  likewise
considered to  be ``turbulent'' values.  Thus, $C_s$, $\PR$  and $\SC$
are  a set  of  empirical constants.  Most  FDS users  simply use  the
default values  of (0.2,0.5,0.5), but some have  explored their effect
on the solution of the equations.

In an effort  to validate FDS with some  simple room temperature data,
Zhang~{\em et al.}~\cite{Zhang:2}  tried different combinations of the
Smagorinsky parameters,  and suggested the current  default values. Of
the  three parameters,  the  Smagorinsky constant  $C_s$  is the  most
sensitive.   Smagorinsky~\cite{Smagorinsky:1}  originally  proposed  a
value of 0.23,  but researchers over the past  three decades have used
values ranging  from 0.1  to 0.23. There  are also refinements  of the
original  Smagorinsky  model~\cite{Deardorff:1,Germano:1,Lilly:1} that
do  not  require the  user  to  prescribe  the constants,  but  rather
generate them automatically as part of the numerical scheme.

\section{Sensitivity of Radiation Parameters}

Radiative heat  transfer is  included in FDS  via the solution  of the
radiation  transport equation for  a non-scattering  gray gas,  and in
some limited  cases using  a wide band  model. The equation  is solved
using  a technique  similar to  finite volume  methods  for convective
transport,  thus the  name given  to it  is the  Finite  Volume Method
(FVM).  There  are  several  limitations  of  the  model.  First,  the
absorption coefficient  for the smoke-laden gas is  a complex function
of  its  composition  and   temperature.  Because  of  the  simplified
combustion  model,  the  chemical  composition of  the  smokey  gases,
especially  the  soot content,  can  effect  both  the absorption  and
emission  of thermal  radiation.  Second, the  radiation transport  is
discretized via  approximately 100 solid angles. For  targets far away
from  a  localized source  of  radiation,  like  a growing  fire,  the
discretization can  lead to a non-uniform distribution  of the radiant
energy. This can be seen in the visualization of surface temperatures,
where  ``hot spots'' show  the effect  of the  finite number  of solid
angles. The  problem can  be lessened by  the inclusion of  more solid
angles, but at  a price of longer computing times.  In most cases, the
radiative flux  to far-field targets is  not as important  as those in
the near-field, where coverage by the default number of angles is much
better.

Hostikka {\em et al.} examined the sensitivity of the radiation solver
to changes in  the assumed soot production, number  of spectral bands,
number  of control  angles, and  flame temperature.  Some of  the more
interesting findings were:
\begin{itemize}
\item  Changing  the  soot  yield  from 1~\%  to  2~\%  increased  the
radiative flux from a simulated methane burner about 15~\%
\item Lowering  the soot  yield to zero  decreased the  radiative flux
about 20~\%.
\item Increasing  the number of  control angles by  a factor of  3 was
necessary  to  ensure  the  accuracy  of the  model  at  the  discrete
measurement locations.
\item Changing the number of spectral  bands from 6 to 10 did not have
a strong effect on the results.
\item Errors of 100~\% in heat  flux were caused by errors of 20~\% in
absolute temperature.
\end{itemize}
The  sensitivity  to  flame   temperature  and  soot  composition  are
consistent with  combustion theory, which states that  the source term
of the  radiative transport equation  is a function of  the absorption
coefficient  multiplied  by the  absolute  temperature  raised to  the
fourth  power. The  number of  control angles  and spectral  bands are
user-controlled numerical  parameters whose sensitivities  ought to be
checked  for  each  new  scenario.  The  default  values  in  FDS  are
appropriate for  most large scale fire  scenarios, but may  need to be
refined for  more detailed simulations  such as a  low-sooting methane
burner.


\section{Sensitivity of Thermophysical Properties of Solid Fuels}

An  extensive amount  of  verification and  validation  work with  FDS
version 4  has been  performed by Hietaniemi,  Hostikka, and  Vaari at
VTT,  Finland~\cite{Hietaniemi:1}. The case  studies are  comprised of
fire  experiments   ranging  in   scale  from  the   cone  calorimeter
(ISO~5660-1) to  full-scale fire  tests such as  the room  corner test
(ISO~9705).  Comparisons are also  made between  FDS results  and data
obtained  in the  SBI (Single  Burning Item)  Euro-classification test
apparatus (EN  13823) as  well as  data obtained in  two {\em  ad hoc}
experimental configurations:  one is similar  to the room  corner test
but has only  partial linings and the other is a  space to study fires
in building cavities.

All of the  case studies involve real materials  whose properties must
be prescribed  so as to conform  to the assumption in  FDS that solids
are of uniform composition backed by a material that is either cold or
totally insulating. Sensitivity of the various physical properties and
the boundary conditions were tested. Some of the findings were:
\begin{itemize}
\item  The measured  burning  rates of  various  materials often  fell
between two FDS  predictions in which cold or  insulated backings were
assumed for the solid surfaces. FDS lacks a multi-layer solid model.
\item  The ignition  time of  upholstery is  sensitive to  the thermal
properties  of the  fabric covering,  but the  steady burning  rate is
sensitive to the properties of the underlying foam.
\item Moisture content of wooden fuels is very important and difficult
to measure.
\item Flame spread  over complicated objects, like cables  laid out in
trays, can be modeled if the  surface area of the simplified object is
comparable to that  of the real object. This  suggests sensitivity not
only to  physical properties,  but also geometry.  It is  difficult to
quantify the extent of the geometrical sensitivity.
\end{itemize}
There is  little quantification of  the observed sensitivities  in the
study. Fire  growth curves can be  linear to exponential  in form, and
small  changes in  fuel  properties  can lead  to  order of  magnitude
changes  in heat  release rate  for unconfined  fires. The  subject is
discussed  in the  FDS  Validation Guide~\cite{FDS_Validation_Guide_5}
where it is  noted in many of the studies  that predicting fire growth
is difficult.

Recently,              Lautenberger,              Rein             and
Fernandez-Pello~\cite{Lautenberger:FSJ} developed a method to automate
the process of  estimating material properties to input  into FDS. The
methodology involves simulating a  bench-scale test with the model and
iterating  via a  "genetic"  algorithm  to obtain  an  optimal set  of
material  properties for  that  particular item.  Such techniques  are
necessary because most bench-scale apparatus do not provide a complete
set of thermal properties.

\section{Summary}

The basis of  large eddy simulation is that  accuracy increases as the
numerical mesh is refined. For fire applications, the grid sensitivity
studies have shown that the accuracy of the model is a function of the
characteristic fire diameter  $D^*$ divided by the grid  cell size. It
is not enough to describe  the resolution of the calculation solely in
terms of the grid cell size, but rather the grid cell size relative to
the heat release rate.  For non-fire applications, there are no simple
means to evaluate ``good resolution.''

As  a  rule  of  thumb,  in  simulations  of  limited  resolution  FDS
predictions are more reliable in the far-field because the substantial
numerical diffusion mimics the  unresolved sub-grid scale mixing. This
is hard to quantify other than through comparisons with experiment. In
some of the sensitivity  studies discussed above, the authors conclude
that the model works best with a cell size of a given value, and often
this cell is not the smallest  one tested. In these cases, the authors
have found a  flow scenario where the unresolved  convective mixing is
almost exactly offset by  numerical diffusion. This is fortuitous, but
the  conclusion does not  necessarily extend  to other  scenarios. The
disadvantage of any turbulence  model, large eddy simulation included,
is  that  good  results  are   not  guaranteed  on  grids  of  limited
resolution. The advantage of LES  over other turbulence models is that
the  solution of  the actual  governing equations,  not a  temporal or
spatial average, is obtained as the mesh is refined.

The same  can be said  for phenomena closer  in to the  fire. However,
grid  resolution is  more  critical for  near-field phenomena  because
numerical diffusion  near the fire on  coarse grids does  not have the
same fortuitous  effect as it  does on far-field results.  In general,
coarse  resolution  will   decrease  temperatures  and  velocities  by
smearing the  values over  the large grid  cells. This can  affect the
radiative flux, convection to surrounding solids, and ultimately flame
spread and fire growth.








\backmatter


% FDSVVBiB is split into FDS_refs, FDSVVBiBnew and FDS_mathcomp
%    SV docs would typically use FDS_ref, FDS_mathcomp but not FDSVVBiBnew
%
%         FDS_refs:  FDS and SV reference documents (user, tech guides etc.)
%         FDS_mathcomp:  mathematical, computer references
%         FDSVVBiBnew: fire references, what is left over

%\bibliography{../Bibliography/FDSVVBiB}
\bibliography{../Bibliography/FDS_refs,../Bibliography/FDSVVBiBnew,../Bibliography/FDS_mathcomp}

\addcontentsline{toc}{chapter}{References}

\end{document}
