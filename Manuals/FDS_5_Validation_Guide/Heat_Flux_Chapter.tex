\chapter{Heat Flux}

Radiative heat transfer is included in FDS via the solution of the radiation transport equation for a
gray gas, and in some limited cases using a wide band
model.  The equation is solved using a technique similar to finite
volume methods for convective transport, thus the name given to it is
the Finite Volume Method (FVM).  Using approximately 100 discrete
angles, the finite volume solver requires about 20~\% of the total CPU
time of a calculation, a modest cost given the complexity of radiation
heat transfer. The absorption coefficients of the gas-soot mixtures
are computed using RADCAL narrow-band model.  Liquid droplets can
absorb and scatter thermal radiation. This is important in cases
involving mist sprinklers, but also plays a role in all sprinkler
cases.  The absorption and scattering coefficients are based on Mie
theory.

This chapter contains a wide variety of heat flux measurements,
ranging from less than a kW/m$^2$ from very small methane gas burners
up to about 150~kW/m$^2$ in full-scale compartment fires.




\section{Hamins Methane Burner Heat Flux Measurements}

Predicted and measured radial and vertical heat flux profiles from six experiments conducted by Anthony Hamins at NIST are shown on the following pages. The
relevant information about the fires is included in Table~\ref{Hamins_Table}. These are challenging simulations because the neither the gray gas assumption
nor the radiative fraction is assumed. Rather, the model is calculating the temperature and species concentrations necessary to predict the radiant energy from the fire.

\begin{table}[ht]
\caption{Summary of Hamins methane burner experiments. }
\begin{center}
\begin{tabular}{|c|c|c|c|c|c|}
\hline
Case     & Test     & $D$  & $\dot{Q}$   &  $\dot{Q}''$   & $Q^*$   \\
         & Number   & (m)  & (kW)        &  (kW/m$^2$)    &         \\ \hline \hline
A        & 1        & 0.10 & 0.42        &  53.8          & 0.12    \\ \hline
B        & 5        & 0.10 & 1.88        &  240           & 0.53    \\ \hline
C        & 23       & 0.38 & 33.5        &  295           & 0.34    \\ \hline
D        & 21       & 0.38 & 175         &  1550          & 1.8     \\ \hline
E        & 7        & 1.0  & 49.0        &  62.4          & 0.044   \\ \hline
F        & 19       & 1.0  & 162         &  206           & 0.14    \\ \hline
\end{tabular}
\end{center}
\label{Hamins_Table}
\end{table}

\newpage

\begin{figure}[p]
\begin{tabular*}{\textwidth}{l@{\extracolsep{\fill}}r}
\includegraphics[height=2.2in]{FIGURES/Hamins_CH4/Hamins_CH4_01_Radial_Heat_Flux} &
\includegraphics[height=2.2in]{FIGURES/Hamins_CH4/Hamins_CH4_05_Radial_Heat_Flux} \\
\includegraphics[height=2.2in]{FIGURES/Hamins_CH4/Hamins_CH4_23_Radial_Heat_Flux} &
\includegraphics[height=2.2in]{FIGURES/Hamins_CH4/Hamins_CH4_21_Radial_Heat_Flux} \\
\includegraphics[height=2.2in]{FIGURES/Hamins_CH4/Hamins_CH4_07_Radial_Heat_Flux} &
\includegraphics[height=2.2in]{FIGURES/Hamins_CH4/Hamins_CH4_19_Radial_Heat_Flux}
\end{tabular*}
\label{Hamins_CH4_Radial}
\caption[Radial heat flux predictions, Hamins methane burner experiments.]
{Comparison of predicted and measured heat fluxes to the ``floor'' as a function of radial distance from a methane burner, Hamins experiments.}
\end{figure}

\begin{figure}[p]
\begin{tabular*}{\textwidth}{l@{\extracolsep{\fill}}r}
\includegraphics[height=2.2in]{FIGURES/Hamins_CH4/Hamins_CH4_01_Vertical_Heat_Flux} &
\includegraphics[height=2.2in]{FIGURES/Hamins_CH4/Hamins_CH4_05_Vertical_Heat_Flux} \\
\includegraphics[height=2.2in]{FIGURES/Hamins_CH4/Hamins_CH4_23_Vertical_Heat_Flux} &
\includegraphics[height=2.2in]{FIGURES/Hamins_CH4/Hamins_CH4_21_Vertical_Heat_Flux} \\
\includegraphics[height=2.2in]{FIGURES/Hamins_CH4/Hamins_CH4_07_Vertical_Heat_Flux} &
\includegraphics[height=2.2in]{FIGURES/Hamins_CH4/Hamins_CH4_19_Vertical_Heat_Flux}
\end{tabular*}
\label{Hamins_CH4_Vertical}
\caption[Vertical heat flux predictions, Hamins methane burner experiments.]
{Comparison of predicted and measured heat fluxes from a methane burner to a ``wall'' as a function of the height from the burner surface, Hamins experiments.}
\end{figure}

\clearpage


\section{NRL/HAI Wall Heat Flux Measurements}

Predicted and measured vertical heat flux profiles from 9 propane sand burner fires are shown on the following pages. The parameters for each
experiment are listed in Table~\ref{NRL/HAI_Parameters} below. Note that all the FDS simulations were performed with a grid resolution such that
$D^*/\dx=10$.

\begin{table}[ht]
\caption{Summary of the NRL/HAI Wall Heat Flux Measurements. }
\begin{center}
\begin{tabular}{|c|c|c|c|c|c|}
\hline
Test     & $D$     & $D^*$      & $\dot{Q}$   & $Q^*$   & Observed  Flame \\
Number   & (m)     & (m)        & (kW)        &         & Height (m)      \\ \hline \hline
1        & 0.28    & 0.30       &  53         & 0.85    & 0.79            \\ \hline
2        & 0.70    & 0.30       &  56         & 0.09    & 0.36            \\ \hline
3        & 0.48    & 0.33       &  68         & 0.28    & 0.60            \\ \hline
4        & 0.37    & 0.39       &  106        & 0.84    & 1.00            \\ \hline
5        & 0.48    & 0.43       &  136        & 0.57    & 0.87            \\ \hline
6        & 0.48    & 0.51       &  204        & 0.85    & 1.45            \\ \hline
7        & 0.70    & 0.52       &  220        & 0.36    & 1.20            \\ \hline
8        & 0.57    & 0.60       &  313        & 0.85    & 2.20            \\ \hline
9        & 0.70    & 0.74       &  523        & 0.85    & 2.9 (based on 500~$^\circ$C)       \\ \hline
\end{tabular}
\end{center}
\label{NRL/HAI_Parameters}
\end{table}

\newpage

\begin{figure}[p]
\begin{tabular*}{\textwidth}{l@{\extracolsep{\fill}}r}
\includegraphics[height=2.2in]{FIGURES/NRL_HAI/NRL_HAI_1_Heat_Flux} &
\includegraphics[height=2.2in]{FIGURES/NRL_HAI/NRL_HAI_2_Heat_Flux} \\
\includegraphics[height=2.2in]{FIGURES/NRL_HAI/NRL_HAI_3_Heat_Flux} &
\includegraphics[height=2.2in]{FIGURES/NRL_HAI/NRL_HAI_4_Heat_Flux} \\
\includegraphics[height=2.2in]{FIGURES/NRL_HAI/NRL_HAI_5_Heat_Flux} &
\end{tabular*}
\label{NRL_HAI_1}
\caption[Wall heat flux predictions, NRL/HAI experiments.]
{Comparison of predicted and measured heat fluxes to the wall from an adjacent propane sand burner, NRL/HAI experiments.}
\end{figure}

\begin{figure}[p]
\begin{tabular*}{\textwidth}{l@{\extracolsep{\fill}}r}
\includegraphics[height=2.2in]{FIGURES/NRL_HAI/NRL_HAI_6_Heat_Flux} &
\includegraphics[height=2.2in]{FIGURES/NRL_HAI/NRL_HAI_7_Heat_Flux} \\
\includegraphics[height=2.2in]{FIGURES/NRL_HAI/NRL_HAI_8_Heat_Flux} &
\includegraphics[height=2.2in]{FIGURES/NRL_HAI/NRL_HAI_9_Heat_Flux}
\end{tabular*}
\label{NRL_HAI_2}
\caption[Wall heat flux predictions, NRL/HAI experiments.]
{Comparison of predicted and measured heat fluxes to the wall from an adjacent propane sand burner, NRL/HAI experiments.}
\end{figure}



\clearpage

\section{Ulster SBI Heat Flux Measurements}

Predicted and measured vertical heat flux profiles for three propane fire sizes in the
single burning item (SBI) enclosure at the University of Ulster are shown on the following page.
Measurements were made on two
vertical panels that form a corner, at the base of which was a triangular-shaped burner with sides of
length 25~cm. Three vertical profiles were measured on each panel at distances of 3.25~cm, 16.5~cm, and
29~cm from the corner.

Note that all the FDS simulations were performed with a grid resolution of 2~cm, and note also that
the parameter {\ct RADIATIVE\_FRACTION} has been set to zero, meaning that FDS is basing the source term
of the radiation transport equation on the temperature rather than the heat release rate. In addition, the
soot yield has been set to 0.05 as a crude way to represent the excess soot that is present in the flames but is
eventually oxidized. The version of FDS used was 5.2.1, and this version does not have a soot model to
account for higher levels of soot in the flame region.

\begin{warning}
\noindent
{\bf These results should be regarded as preliminary because FDS is being used in a way that one would
ordinarily not use it.}
\end{warning}

\newpage

\begin{figure}[p]
\begin{tabular*}{\textwidth}{l@{\extracolsep{\fill}}r}
\includegraphics[height=2.2in]{FIGURES/Ulster_SBI/Ulster_SBI_30_kW_Left_Heat_Flux} &
\includegraphics[height=2.2in]{FIGURES/Ulster_SBI/Ulster_SBI_30_kW_Right_Heat_Flux} \\
\includegraphics[height=2.2in]{FIGURES/Ulster_SBI/Ulster_SBI_45_kW_Left_Heat_Flux} &
\includegraphics[height=2.2in]{FIGURES/Ulster_SBI/Ulster_SBI_45_kW_Right_Heat_Flux} \\
\includegraphics[height=2.2in]{FIGURES/Ulster_SBI/Ulster_SBI_60_kW_Left_Heat_Flux} &
\includegraphics[height=2.2in]{FIGURES/Ulster_SBI/Ulster_SBI_60_kW_Right_Heat_Flux}
\end{tabular*}
\label{Ulster_SBI}
\caption[Corner heat flux predictions, Ulster SBI experiments.]
{Comparison of predicted and measured heat fluxes to adjacent panels forming a corner in the single
burning item (SBI) apparatus at the University of Ulster.}
\end{figure}

\clearpage

\section{WTC Heat Flux Measurements}

There were a variety of heat flux gauges installed in the test compartment. Most were within 2~m of the fire. Their locations are orientations are listed in Table~\ref{WTC_Gauges}.

\begin{table}[h!]
\caption{Heat flux gauge positions relative to the center of the fire pan in the WTC series.}
\begin{center}
\begin{tabular}{|l|c|c|c|c|l|}
\hline
Name    & $x$ (m)   & $y$ (m) & $z$ (m)   & Orientation  & Location  \\ \hline \hline
H2FU    & 0.64      & 0.63    & 3.30      &     $+z$     & Truss Support         \\ \hline
H2RU    & 0.64      & 0.51    & 3.30      &     $+z$     & Truss Support          \\ \hline
H2FD    & 0.64      & 0.30    & 3.15      &     $-z$     & Truss Support          \\ \hline
H2RD    & 0.64      & 0.42    & 3.15      &     $-z$     & Truss Support          \\ \hline
HCoHF   & -0.90     & 0.84    & 3.46      &     $+x$     & Column, facing fire          \\ \hline
HCoHW   & -0.97     & 0.92    & 3.27      &     $+y$     & Column, facing north          \\ \hline
HCoLF   & -0.90     & 0.84    & 0.92      &     $+x$     & Column, facing fire          \\ \hline
HCoLW   & -0.97     & 0.92    & 1.02      &     $+y$     & Column, facing north          \\ \hline
HF1     & 1.06      & 0.13    & 0.13      &     $+z$     & Floor          \\ \hline
HF2     & 1.56      & 0.10    & 0.13      &     $+z$     & Floor          \\ \hline
HCe1    & -0.45     & 0.35    & 3.82      &     $-z$     & Ceiling          \\ \hline
HCe2    &  0.05     & 0.35    & 3.82      &     $-z$     & Ceiling          \\ \hline
HCe3    &  0.80     & 0.35    & 3.82      &     $-z$     & Ceiling          \\ \hline
HCe4    &  2.56     & 0.35    & 3.82      &     $-z$     & Ceiling          \\ \hline
\end{tabular}
\end{center}
\label{WTC_Gauges}
\end{table}

\newpage

\begin{figure}[p]
\begin{tabular*}{\textwidth}{l@{\extracolsep{\fill}}r}
\includegraphics[height=2.2in]{FIGURES/WTC/WTC_01_v5_Station_2_Flux_High} &
\includegraphics[height=2.2in]{FIGURES/WTC/WTC_02_v5_Station_2_Flux_High} \\
\includegraphics[height=2.2in]{FIGURES/WTC/WTC_03_v5_Station_2_Flux_High} &
\includegraphics[height=2.2in]{FIGURES/WTC/WTC_04_v5_Station_2_Flux_High} \\
\includegraphics[height=2.2in]{FIGURES/WTC/WTC_05_v5_Station_2_Flux_High} &
\includegraphics[height=2.2in]{FIGURES/WTC/WTC_06_v5_Station_2_Flux_High}
\end{tabular*}
\label{NIST_WTC_Station_2_Flux_High}
\end{figure}

\begin{figure}[p]
\begin{tabular*}{\textwidth}{l@{\extracolsep{\fill}}r}
\includegraphics[height=2.2in]{FIGURES/WTC/WTC_01_v5_Station_2_Flux_Low} &
\includegraphics[height=2.2in]{FIGURES/WTC/WTC_02_v5_Station_2_Flux_Low} \\
\includegraphics[height=2.2in]{FIGURES/WTC/WTC_03_v5_Station_2_Flux_Low} &
\includegraphics[height=2.2in]{FIGURES/WTC/WTC_04_v5_Station_2_Flux_Low} \\
\includegraphics[height=2.2in]{FIGURES/WTC/WTC_05_v5_Station_2_Flux_Low} &
\includegraphics[height=2.2in]{FIGURES/WTC/WTC_06_v5_Station_2_Flux_Low}
\end{tabular*}
\label{NIST_WTC_Station_2_Flux_Low}
\end{figure}

\begin{figure}[p]
\begin{tabular*}{\textwidth}{l@{\extracolsep{\fill}}r}
\includegraphics[height=2.2in]{FIGURES/WTC/WTC_01_v5_Upper_Column_Flux} &
\includegraphics[height=2.2in]{FIGURES/WTC/WTC_02_v5_Upper_Column_Flux} \\
\includegraphics[height=2.2in]{FIGURES/WTC/WTC_03_v5_Upper_Column_Flux} &
\includegraphics[height=2.2in]{FIGURES/WTC/WTC_04_v5_Upper_Column_Flux} \\
\includegraphics[height=2.2in]{FIGURES/WTC/WTC_05_v5_Upper_Column_Flux} &
\includegraphics[height=2.2in]{FIGURES/WTC/WTC_06_v5_Upper_Column_Flux}
\end{tabular*}
\label{NIST_WTC_Upper_Column_Flux}
\end{figure}

\begin{figure}[p]
\begin{tabular*}{\textwidth}{l@{\extracolsep{\fill}}r}
\includegraphics[height=2.2in]{FIGURES/WTC/WTC_01_v5_Lower_Column_Flux} &
\includegraphics[height=2.2in]{FIGURES/WTC/WTC_02_v5_Lower_Column_Flux} \\
\includegraphics[height=2.2in]{FIGURES/WTC/WTC_03_v5_Lower_Column_Flux} &
\includegraphics[height=2.2in]{FIGURES/WTC/WTC_04_v5_Lower_Column_Flux} \\
\includegraphics[height=2.2in]{FIGURES/WTC/WTC_05_v5_Lower_Column_Flux} &
\includegraphics[height=2.2in]{FIGURES/WTC/WTC_06_v5_Lower_Column_Flux}
\end{tabular*}
\label{NIST_WTC_Lower_Column_Flux}
\end{figure}

\begin{figure}[p]
\begin{tabular*}{\textwidth}{l@{\extracolsep{\fill}}r}
\includegraphics[height=2.2in]{FIGURES/WTC/WTC_01_v5_Floor_Flux} &
\includegraphics[height=2.2in]{FIGURES/WTC/WTC_02_v5_Floor_Flux} \\
\includegraphics[height=2.2in]{FIGURES/WTC/WTC_03_v5_Floor_Flux} &
\includegraphics[height=2.2in]{FIGURES/WTC/WTC_04_v5_Floor_Flux} \\
\includegraphics[height=2.2in]{FIGURES/WTC/WTC_05_v5_Floor_Flux} &
\includegraphics[height=2.2in]{FIGURES/WTC/WTC_06_v5_Floor_Flux}
\end{tabular*}
\label{NIST_WTC_Floor_Flux}
\end{figure}

\begin{figure}[p]
\begin{tabular*}{\textwidth}{l@{\extracolsep{\fill}}r}
\includegraphics[height=2.2in]{FIGURES/WTC/WTC_01_v5_Ceiling_Flux_1} &
\includegraphics[height=2.2in]{FIGURES/WTC/WTC_02_v5_Ceiling_Flux_1} \\
\includegraphics[height=2.2in]{FIGURES/WTC/WTC_03_v5_Ceiling_Flux_1} &
\includegraphics[height=2.2in]{FIGURES/WTC/WTC_04_v5_Ceiling_Flux_1} \\
\includegraphics[height=2.2in]{FIGURES/WTC/WTC_05_v5_Ceiling_Flux_1} &
\includegraphics[height=2.2in]{FIGURES/WTC/WTC_06_v5_Ceiling_Flux_1}
\end{tabular*}
\label{NIST_WTC_Ceiling_Flux_1}
\end{figure}

\begin{figure}[p]
\begin{tabular*}{\textwidth}{l@{\extracolsep{\fill}}r}
\includegraphics[height=2.2in]{FIGURES/WTC/WTC_01_v5_Ceiling_Flux_2} &
\includegraphics[height=2.2in]{FIGURES/WTC/WTC_02_v5_Ceiling_Flux_2} \\
\includegraphics[height=2.2in]{FIGURES/WTC/WTC_03_v5_Ceiling_Flux_2} &
\includegraphics[height=2.2in]{FIGURES/WTC/WTC_04_v5_Ceiling_Flux_2} \\
\includegraphics[height=2.2in]{FIGURES/WTC/WTC_05_v5_Ceiling_Flux_2} &
\includegraphics[height=2.2in]{FIGURES/WTC/WTC_06_v5_Ceiling_Flux_2}
\end{tabular*}
\label{NIST_WTC_Ceiling_Flux_2}
\end{figure}

\begin{figure}[p]
\begin{center}
\begin{tabular}{c}
\includegraphics[width=5.0in]{FIGURES/ScatterPlots/WTC_Heat_Flux}
\end{tabular}
\end{center}
\caption[Summary of heat flux predictions, WTC test series.]
{Summary of heat flux predictions for the WTC test series.}
\end{figure}



\clearpage

\section{NIST/NRC Test Series, Heat Flux to Cables}

Cables in various types (power and control), and configurations (horizontal, vertical, in trays or free-hanging), were installed in
the test compartment. For each of the four cable targets considered, measurements of the radiative and total heat flux were made with
gauges positioned near the cables themselves.  The following pages display comparisons of these heat flux predictions and measurements for
Control Cable B, Horizontal Cable Tray D, Power Cable F and Vertical Cable Tray G.

\newpage

\begin{figure}[p]
\begin{tabular*}{\textwidth}{l@{\extracolsep{\fill}}r}
\includegraphics[height=2.2in]{FIGURES/NIST_NRC/NIST_NRC_01_v5_Cable_B_Flux} &
\includegraphics[height=2.2in]{FIGURES/NIST_NRC/NIST_NRC_07_v5_Cable_B_Flux} \\
\includegraphics[height=2.2in]{FIGURES/NIST_NRC/NIST_NRC_02_v5_Cable_B_Flux} &
\includegraphics[height=2.2in]{FIGURES/NIST_NRC/NIST_NRC_08_v5_Cable_B_Flux} \\
\includegraphics[height=2.2in]{FIGURES/NIST_NRC/NIST_NRC_04_v5_Cable_B_Flux} &
\includegraphics[height=2.2in]{FIGURES/NIST_NRC/NIST_NRC_10_v5_Cable_B_Flux} \\
\includegraphics[height=2.2in]{FIGURES/NIST_NRC/NIST_NRC_13_v5_Cable_B_Flux} &
\includegraphics[height=2.2in]{FIGURES/NIST_NRC/NIST_NRC_16_v5_Cable_B_Flux}
\end{tabular*}
\label{NIST_NRC_Cable_B_Flux_Closed}
\end{figure}

\begin{figure}[p]
\begin{tabular*}{\textwidth}{l@{\extracolsep{\fill}}r}
\includegraphics[height=2.2in]{FIGURES/NIST_NRC/NIST_NRC_03_v5_Cable_B_Flux} &
\includegraphics[height=2.2in]{FIGURES/NIST_NRC/NIST_NRC_09_v5_Cable_B_Flux} \\
\includegraphics[height=2.2in]{FIGURES/NIST_NRC/NIST_NRC_05_v5_Cable_B_Flux} &
\includegraphics[height=2.2in]{FIGURES/NIST_NRC/NIST_NRC_14_v5_Cable_B_Flux} \\
\includegraphics[height=2.2in]{FIGURES/NIST_NRC/NIST_NRC_15_v5_Cable_B_Flux} &
\includegraphics[height=2.2in]{FIGURES/NIST_NRC/NIST_NRC_18_v5_Cable_B_Flux}
\end{tabular*}
\label{NIST_NRC_Cable_B_Flux_Open}
\end{figure}

\begin{figure}[p]
\begin{tabular*}{\textwidth}{l@{\extracolsep{\fill}}r}
\includegraphics[height=2.2in]{FIGURES/NIST_NRC/NIST_NRC_01_v5_Cable_D_Flux} &
\includegraphics[height=2.2in]{FIGURES/NIST_NRC/NIST_NRC_07_v5_Cable_D_Flux} \\
\includegraphics[height=2.2in]{FIGURES/NIST_NRC/NIST_NRC_02_v5_Cable_D_Flux} &
\includegraphics[height=2.2in]{FIGURES/NIST_NRC/NIST_NRC_08_v5_Cable_D_Flux} \\
\includegraphics[height=2.2in]{FIGURES/NIST_NRC/NIST_NRC_04_v5_Cable_D_Flux} &
\includegraphics[height=2.2in]{FIGURES/NIST_NRC/NIST_NRC_10_v5_Cable_D_Flux} \\
\includegraphics[height=2.2in]{FIGURES/NIST_NRC/NIST_NRC_13_v5_Cable_D_Flux} &
\includegraphics[height=2.2in]{FIGURES/NIST_NRC/NIST_NRC_16_v5_Cable_D_Flux}
\end{tabular*}
\label{NIST_NRC_Cable_D_Flux_Closed}
\end{figure}

\begin{figure}[p]
\begin{tabular*}{\textwidth}{l@{\extracolsep{\fill}}r}
                           &
\includegraphics[height=2.2in]{FIGURES/NIST_NRC/NIST_NRC_09_v5_Cable_D_Flux} \\
\includegraphics[height=2.2in]{FIGURES/NIST_NRC/NIST_NRC_05_v5_Cable_D_Flux} &
\includegraphics[height=2.2in]{FIGURES/NIST_NRC/NIST_NRC_14_v5_Cable_D_Flux} \\
                      &
\end{tabular*}
\label{NIST_NRC_Cable_D_Flux_Open}
\end{figure}

\begin{figure}[p]
\begin{tabular*}{\textwidth}{l@{\extracolsep{\fill}}r}
\includegraphics[height=2.2in]{FIGURES/NIST_NRC/NIST_NRC_01_v5_Cable_F_Flux} &
\includegraphics[height=2.2in]{FIGURES/NIST_NRC/NIST_NRC_07_v5_Cable_F_Flux} \\
\includegraphics[height=2.2in]{FIGURES/NIST_NRC/NIST_NRC_02_v5_Cable_F_Flux} &
\includegraphics[height=2.2in]{FIGURES/NIST_NRC/NIST_NRC_08_v5_Cable_F_Flux} \\
\includegraphics[height=2.2in]{FIGURES/NIST_NRC/NIST_NRC_04_v5_Cable_F_Flux} &
\includegraphics[height=2.2in]{FIGURES/NIST_NRC/NIST_NRC_10_v5_Cable_F_Flux} \\
\includegraphics[height=2.2in]{FIGURES/NIST_NRC/NIST_NRC_13_v5_Cable_F_Flux} &
\includegraphics[height=2.2in]{FIGURES/NIST_NRC/NIST_NRC_16_v5_Cable_F_Flux}
\end{tabular*}
\label{NIST_NRC_Cable_F_Flux_Closed}
\end{figure}

\begin{figure}[p]
\begin{tabular*}{\textwidth}{l@{\extracolsep{\fill}}r}
\includegraphics[height=2.2in]{FIGURES/NIST_NRC/NIST_NRC_03_v5_Cable_F_Flux} &
\includegraphics[height=2.2in]{FIGURES/NIST_NRC/NIST_NRC_09_v5_Cable_F_Flux} \\
\includegraphics[height=2.2in]{FIGURES/NIST_NRC/NIST_NRC_05_v5_Cable_F_Flux} &
\includegraphics[height=2.2in]{FIGURES/NIST_NRC/NIST_NRC_14_v5_Cable_F_Flux} \\
\includegraphics[height=2.2in]{FIGURES/NIST_NRC/NIST_NRC_15_v5_Cable_F_Flux} &
\includegraphics[height=2.2in]{FIGURES/NIST_NRC/NIST_NRC_18_v5_Cable_F_Flux}
\end{tabular*}
\label{NIST_NRC_Cable_F_Flux_Open}
\end{figure}

\begin{figure}[p]
\begin{tabular*}{\textwidth}{l@{\extracolsep{\fill}}r}
\includegraphics[height=2.2in]{FIGURES/NIST_NRC/NIST_NRC_01_v5_Cable_G_Flux} &
\includegraphics[height=2.2in]{FIGURES/NIST_NRC/NIST_NRC_07_v5_Cable_G_Flux} \\
\includegraphics[height=2.2in]{FIGURES/NIST_NRC/NIST_NRC_02_v5_Cable_G_Flux} &
\includegraphics[height=2.2in]{FIGURES/NIST_NRC/NIST_NRC_08_v5_Cable_G_Flux} \\
\includegraphics[height=2.2in]{FIGURES/NIST_NRC/NIST_NRC_04_v5_Cable_G_Flux} &
\includegraphics[height=2.2in]{FIGURES/NIST_NRC/NIST_NRC_10_v5_Cable_G_Flux} \\
\includegraphics[height=2.2in]{FIGURES/NIST_NRC/NIST_NRC_13_v5_Cable_G_Flux} &
\includegraphics[height=2.2in]{FIGURES/NIST_NRC/NIST_NRC_16_v5_Cable_G_Flux}
\end{tabular*}
\label{NIST_NRC_Cable_G_Flux_Closed}
\end{figure}

\begin{figure}[p]
\begin{tabular*}{\textwidth}{l@{\extracolsep{\fill}}r}
\includegraphics[height=2.2in]{FIGURES/NIST_NRC/NIST_NRC_03_v5_Cable_G_Flux} &
\includegraphics[height=2.2in]{FIGURES/NIST_NRC/NIST_NRC_09_v5_Cable_G_Flux} \\
\includegraphics[height=2.2in]{FIGURES/NIST_NRC/NIST_NRC_05_v5_Cable_G_Flux} &
\includegraphics[height=2.2in]{FIGURES/NIST_NRC/NIST_NRC_14_v5_Cable_G_Flux} \\
\includegraphics[height=2.2in]{FIGURES/NIST_NRC/NIST_NRC_15_v5_Cable_G_Flux} &
\includegraphics[height=2.2in]{FIGURES/NIST_NRC/NIST_NRC_18_v5_Cable_G_Flux}
\end{tabular*}
\label{NIST_NRC_Cable_G_Flux_Open}
\end{figure}



\begin{figure}[p]
\begin{center}
\begin{tabular}{c}
\includegraphics[width=5.0in]{FIGURES/ScatterPlots/NIST_NRC_Heat_Flux}
\end{tabular}
\end{center}
\caption[Summary of heat flux predictions to cables, NIST/NRC test series.]
{Summary of heat flux predictions to the cables in the NIST/NRC test series.}
\end{figure}
