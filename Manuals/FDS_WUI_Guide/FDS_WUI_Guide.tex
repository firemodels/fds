\documentclass[11pt]{book}
\usepackage{times,mathptm}
\usepackage[pdftex]{graphicx}
\usepackage[pdftex,
        colorlinks=true,
        urlcolor=linkblue,     % \href{...}{...} external (URL)
        citecolor=linkred,     % citation number colors
        linkcolor=linknavy,    % \ref{...} and \pageref{...}
        pdftitle={Fire Dynamics Simulator (Version 5) User's Guide},
        pdfauthor={Kevin McGrattan, Bryan Klein, Simo Hostikka, Jason Floyd},
        pdfsubject={User Guide},
        pdfkeywords={FDS, Fire Model, NIST, BFRL},
        pdfproducer={pdflatex},
        pagebackref,
        pdfpagemode=UseNone,
        bookmarksopen=true,
        plainpages=false]{hyperref}
\usepackage{color}
\definecolor{linknavy}{rgb}{0,0,0.50196}
\definecolor{linkred}{rgb}{1,0,0}
\definecolor{linkblue}{rgb}{0,0,1}
\usepackage{caption}
\usepackage{graphpap}
\usepackage{rotating}
\usepackage{epsfig,psfrag}
%\usepackage{wrapfig}
%\usepackage{picins}
\usepackage{geometry}
\usepackage{tabularx}
\usepackage{longtable}
\usepackage{lscape}
\usepackage{amssymb}
\usepackage{makeidx} % Create index at end of document
\usepackage[nottoc,notlof,notlot]{tocbibind} % Put the bibliography and index in the ToC
\usepackage{float}
\usepackage{lastpage} % Automatic last page number reference.
\usepackage[T1]{fontenc}
\usepackage{upquote}
\usepackage{array,eqnarray}
\newcommand{\nopart}{\expandafter\def\csname Parent-1\endcsname{}} % To fix table of contents in pdf.
\newcommand{\ct}{\tt\small}
%\newfont{\ct}{cmtt10 at 9pt}


% The Following commented code makes the ``Draft'' watermark on each page.
%\usepackage{eso-pic}
%\usepackage{type1cm}
%\makeatletter
%   \AddToShipoutPicture{
%     \setlength{\@tempdimb}{.5\paperwidth}
%     \setlength{\@tempdimc}{.5\paperheight}
%     \setlength{\unitlength}{1pt}
%     \put(\strip@pt\@tempdimb,\strip@pt\@tempdimc){
%     \makebox(0,0){\rotatebox{45}{\textcolor[gray]{0.75}{\fontsize{8cm}\selectfont{RC6}}}}}
% }
%\makeatother

\setlength{\textwidth}{6.5in}
\setlength{\textheight}{9.0in}
\setlength{\topmargin}{0.in}
\setlength{\headheight}{0.in}
\setlength{\headsep}{0.in}
\setlength{\parindent}{0.25in}
\setlength{\oddsidemargin}{0.0in}
\setlength{\evensidemargin}{0.0in}



\newcommand{\dod}[2]{\frac{\partial #1}{\partial #2}}
\newcommand{\DoD}[2]{\frac{D #1}{D #2}}
\newcommand{\dsods}[2]{\frac{\partial^2 #1}{\partial #2^2}}
\newcommand{\dx}{\delta x}
\newcommand{\dy}{\delta y}
\newcommand{\dz}{\delta z}
\newcommand{\x}{x}
\newcommand{\y}{y}
\newcommand{\z}{z}
\newcommand{\dt}{\delta t}
\newcommand{\dn}{\delta n}
\newcommand{\hu}{u}
\newcommand{\hv}{v}
\newcommand{\hw}{w}
\newcommand{\bo}{{\bf \omega}}
\newcommand{\oW}{\overline{W}}
\newcommand{\om}{\omega}
\newcommand{\omx}{\omega_x}
\newcommand{\omy}{\omega_y}
\newcommand{\omz}{\omega_z}
\newcommand{\bF}{{\bf F}}
\newcommand{\bof}{{\bf f}}
\newcommand{\dS}{{d\bf S}}
\newcommand{\dA}{{dA}}
\newcommand{\bq}{{\bf q}}
\newcommand{\br}{{\bf r}}
\newcommand{\bu}{{\bf u}}
\newcommand{\bx}{{\bf x}}
\newcommand{\bk}{{\bf k}}
\newcommand{\bv}{{\bf v}}
\newcommand{\bg}{{\bf g}}
\newcommand{\bn}{{\bf n}}
\newcommand{\bS}{{\bf S}}
\newcommand{\bs}{{\bf s}}
\newcommand{\bI}{{\bf I}}
\newcommand{\hp}{{\cal H}}
\newcommand{\trho}{\tilde{\rho}}
\newcommand{\tp}{\tilde{p}}
\newcommand{\bp}{\overline{p}}
\newcommand{\dQ}{\dot{Q}}
\newcommand{\dq}{\dot{q}}
\newcommand{\dm}{\dot{m}}
\newcommand{\dW}{\dot{W}}
\newcommand{\ha}{\frac{1}{2}}
\newcommand{\ft}{\frac{4}{3}}
\newcommand{\ot}{\frac{1}{3}}
\newcommand{\of}{\frac{1}{4}}
\newcommand{\twth}{\frac{2}{3}}
\newcommand{\R}{{\cal R}}
\newcommand{\be}{\begin{equation}}
\newcommand{\ee}{\end{equation}}
\newcommand{\RE}{\hbox{Re}}
\newcommand{\LE}{\hbox{Le}}
\newcommand{\PR}{\hbox{Pr}}
\newcommand{\PE}{\hbox{Pe}}
\newcommand{\NU}{\hbox{Nu}}
\newcommand{\SC}{\hbox{Sc}}
\newcommand{\WE}{\hbox{We}}
\newcommand{\COTWO}{{\tiny \hbox{CO}_2}}
\newcommand{\HTWOO}{{\tiny \hbox{H}_2\hbox{O}}}
\newcommand{\OTWO}{{\tiny \hbox{O}_2}}
\newcommand{\NTWO}{{\tiny \hbox{N}_2}}
\newcommand{\CO}{{\tiny \hbox{CO}}}
\newcommand{\F}{{\tiny \hbox{F}}}
\newcommand{\C}{{\tiny \hbox{C}}}
\newcommand{\Hy}{{\tiny \hbox{H}}}
\newcommand{\So}{{\tiny \hbox{S}}}
\newcommand{\M}{{\tiny \hbox{M}}}
\newcommand{\xx}{{\tiny \hbox{x}}}
\newcommand{\yy}{{\tiny \hbox{y}}}
\newcommand{\zz}{{\tiny \hbox{z}}}

\newcommand{\figheight}{1.5in}
\newcommand{\figwidth}{3.333333in}
\newcommand{\figwidthb}{2.0in}
\newcommand{\Ra}{$\Rightarrow$}
\newcommand{\parma}{.75}
\newcommand{\parmb}{.5}
\newcommand{\parmc}{0.25}
%\newcommand{\bold}[1]{{\bf #1}}
\newcommand{\etc}{{\em etc}}
\newcommand{\ie}{{\em i.e.}}
\newcommand{\eg}{{\em e.g.}}
\newcommand{\via}{{\em via\ }}
\newcommand{\loadmenu}{\fbox{\ct Load/Unload}}
\newcommand{\blist}{
\begin{list}
{}{
\setlength{\leftmargin}{\parma in}
\setlength{\labelwidth}{\parmb in}
\setlength{\labelsep}{\parmc in}
\setlength{\listparindent}{0.3in}
\setlength{\topsep}{.3in}
\setlength{\parsep}{.0in}
}}
\newcommand{\elist}{\end{list}}
\newcommand{\hitem}[1]{\item[{\bf #1} \hfill]}



\floatstyle{boxed}
\newfloat{notebox}{H}{lon}
\newfloat{warning}{H}{low}

\makeindex

\begin{document}
\bibliographystyle{unsrt}

\setlength{\leftmargini}{\parindent} % Controls the indenting of the "bullets" in a list

\pagestyle{empty}
\pagenumbering{alph}

\begin{minipage}[t][9in][s]{6.25in}

\huge
\flushright{NIST Special Publication 1019-5}

\vspace{1in}

\Huge \flushright{Fire Dynamics Simulator (Version 5) \\ User's Guide }

\vspace{.5in}

\normalsize

\large
\flushright{
Kevin McGrattan \\
Bryan Klein \\
Simo Hostikka \\
Jason Floyd \\
 }

 \vspace{0.5in}

\flushright{In cooperation with: \\
VTT Technical Research Centre of Finland  }

\vfill

\flushright{\includegraphics[width=2.in]{FIGURES/nistident_flright_vec}}


\end{minipage}

\newpage
\hspace{5in}
\newpage

\begin{minipage}[t][9in][s]{6.25in}

\huge
\flushright{NIST Special Publication 1019-5}

\vspace{.75in}

\Huge
\flushright{Fire Dynamics Simulator (Version 5) \\ User's Guide}

\vspace{.25in}

\normalsize
\flushright{
Kevin McGrattan \\
Bryan Klein \\
{\em NIST Building and Fire Research Laboratory} \\
{\em Gaithersburg, Maryland, USA}  \\
\hspace{1in} \\
Simo Hostikka \\
{\em VTT Technical Research Centre of Finland} \\
{\em Espoo, Finland} \\
\hspace{1in} \\
Jason Floyd \\
{\em Hughes Associates, Inc.} \\
{\em Baltimore, Maryland, USA}}

\vspace{.25in}

\flushright{\today \\
FDS Version 5.2 \\
$SVN Repository$~$Revision$}

\vfill

\flushright{\includegraphics[width=1in]{FIGURES/doc.pdf} }

\small
\flushright{U.S. Department of Commerce \\
{\em John E. Bryson, Secretary} \\
\hspace{1in} \\
National Institute of Standards and Technology \\
{\em Patrick D. Gallagher, Under Secretary of Commerce for Standards and Technology and Director} }


\end{minipage}

\newpage

\begin{minipage}[t][9in][s]{6.25in}

\flushright{Certain commercial entities, equipment, or materials may be identified in this \\
document in order to describe an experimental procedure or concept adequately. Such \\
identification is not intended to imply recommendation or endorsement by the \\
National Institute of Standards and Technology, nor is it intended to imply that the \\
entities, materials, or equipment are necessarily the best available for the purpose.
}

\vspace{3in}

\large
\flushright{\bf National Institute of Standards and Technology Special Publication 1019-5 \\
Natl.~Inst.~Stand.~Technol.~Spec.~Publ.~1019-5, \pageref{LastPage} pages (October 2007) \\
CODEN: NSPUE2 }

\vfill

\flushright{U.S. GOVERNMENT PRINTING OFFICE \\
WASHINGTON: 2007 \\
\rule{3.5in}{0.01in} \\
For sale by the Superintendent of Documents, U.S. Government Printing Office \\
Internet: bookstore.gpo.gov -- Phone: (202) 512-1800 -- Fax: (202) 512-2250 \\
Mail: Stop SSOP, Washington, DC 20402-0001 }
\end{minipage}

\clearpage

\frontmatter

\pagestyle{plain}
\pagenumbering{roman}


\chapter{Preface}

This Guide describes how to use the Fire Dynamics Simulator (FDS). Most of the content pertains to version 5, although some features
have been added since the release of FDS 5.0. The current version is 5.2.

Note that this Guide does not provide the background theory for FDS. A three volume set of companion documents, referred to
collectively as the FDS Technical Reference Guide~\cite{FDS_Tech_Guide_5}, contains details about the governing
equations and numerical methods, model verification, and experimental validation.
The FDS User's Guide contains limited information on how to operate Smokeview, the companion
visualization program for FDS. Its full capability is described in the ``User's Guide for
Smokeview Version~5''~\cite{Smokeview_Users_Guide_5}.


\chapter{Disclaimer}

The US Department of Commerce makes no warranty, expressed or implied, to
users of the Fire Dynamics Simulator (FDS), and accepts no responsibility for its
use. Users of FDS assume sole responsibility under Federal law for
determining the appropriateness of its use in any particular application;
for any conclusions drawn from the results of its use; and for any actions
taken or not taken as a result of analyses performed using these tools.

Users are warned that FDS is intended for use only by those competent in
the fields of fluid dynamics, thermodynamics, combustion, and heat transfer,
and is intended only to supplement the
informed judgment of the qualified user. The software package is a
computer model that may or may not have predictive capability when applied
to a specific set of factual circumstances. Lack of accurate predictions by
the model could lead to erroneous conclusions with regard to fire safety.
All results should be evaluated by an informed user.

Throughout this document, the mention of computer hardware or
commercial software does not constitute endorsement by NIST, nor does
it indicate that the products are necessarily those best suited for the
intended purpose.


\chapter{About the Authors}

\begin{description}
\item[Kevin McGrattan] is a mathematician in the Building and Fire
Research Laboratory of NIST. He received a bachelors of science degree
from the School of Engineering and Applied Science of Columbia
University in 1987 and a doctorate at the Courant Institute of New
York University in 1991. He joined the NIST staff in 1992 and has
since worked on the development of fire models, most notably the Fire
Dynamics Simulator.
\item[Simo Hostikka] is a Senior Research Scientist at VTT Technical
Research Centre of Finland. He received a master of science
(technology) degree in 1997 and a doctorate in 2008 from
the Department of Engineering Physics and Mathematics of the
Helsinki University of Technology.  He is the principal developer of the
radiation and solid phase sub-models within FDS.
\item[Jason Floyd] is a Senior Engineer at Hughes Associates, Inc., in
Baltimore, Maryland. He received a bachelors of science degree and a
doctorate from the Nuclear Engineering Program of the University of
Maryland. After graduating, he won a National Research Council
Post-Doctoral Fellowship at the Building and Fire Research Laboratory
of NIST, where he developed the combustion algorithm within FDS. He is
currently funded by NIST under grant 60NANB5D1205 from the Fire
Research Grants Program (15 USC 278f).  He is the principal developer
of the multi-parameter mixture fraction combustion model and control
logic within FDS.
\item[Bryan Klein] is an Information Technology Specialist in the
Building and Fire Research Laboratory of NIST.  Before coming to NIST,
Bryan worked for five years with Western Fire Center, Inc., performing a
wide range of activities including fire modeling, data acquisition programming,
and quantitative fire measurements. His current focus is on FDS development and
user support, along with experimental model validation work.
\end{description}



\chapter{Acknowledgments}

The Fire Dynamics Simulator, in various forms, has been under development for almost 25 years. However,
the publicly released software has only existed since 2000. Since its first release, continued improvements
have been made to the software based largely on feedback from its users.
Included here are some who made important contributions.

At NIST, thanks to Dan Madrzykowski, Doug Walton, Bob Vettori, Dave Stroup, Steve Kerber and Nelson Bryner,
who have used FDS and Smokeview as part of several investigations of fire fighter line of duty deaths.
As part of these studies, they have provided valuable information on the model's usability and accuracy
when compared to large scale measurements made during fire reconstructions.

The US Nuclear Regulatory Commission has provided financial support for the maintenance and development of FDS,
along with valuable insights into how fire models are used as part of probabilistic risk assessments of nuclear
facilities. Special thanks to Mark Salley and Jason Dreisbach of NRC, and Francisco Joglar of SAIC.

The Society of Fire Protection Engineers (SFPE) sponsors a training course on the use of FDS and Smokeview.
Chris Wood of ArupFire, Dave Sheppard of the US Bureau of Alcohol, Tobacco and Firearms (ATF), and
Doug Carpenter of Combustion Science and Engineering developed the materials for the course, along with
Morgan Hurley of the SFPE.

Prof.~David McGill of Seneca College, Ontario, Canada has conducted a remote-learning course
on the use of FDS, and he has also maintained a web site that has provided valuable suggestions from users.


Prof.~Ian Thomas of Victoria University has also presented short courses on the use of FDS in Australia.
His students have also performed some validation work on compartment fires.

Prof.~Charles Fleischmann and his students at the University of Canterbury, New Zealand, have provided valuable assistance
in improving the documentation and usability of the model.

James White Jr.~of the Western Fire Center has provided valuable feedback on how to
improve the functionality of the model in the area of forensic science.

Paul Hart of Swiss Re, GAP Services, and Pravinray Gandhi of Underwriters Laboratories provided useful suggestions about
water droplet transport on solid objects.

Finally, on the following pages is a list of individuals and organizations who have volunteered their time and effort to
``beta test'' FDS and Smokeview prior to its official release. Their contribution is invaluable because there is simply no other way
to test all of the various features of the model.











\tableofcontents
\listoffigures
\listoftables

\mainmatter



\part{Running FDS}


\chapter{Introduction}

The software described in this document, Fire Dynamics Simulator (FDS), is a computational
fluid dynamics (CFD) model of fire-driven fluid flow. FDS solves numerically a form of the
Navier-Stokes equations\index{Navier-Stokes} appropriate for low-speed, thermally-driven flow
with an emphasis on smoke and heat transport from fires.
The formulation of the equations and the numerical algorithm are contained the FDS Technical Reference Guide~\cite{FDS_Tech_Guide_5}.

Smokeview is a separate visualization program that is used to display the
results of an FDS simulation.
A detailed description of Smokeview is found in the
{\em User's Guide for Smokeview Version 5}~\cite{Smokeview_Users_Guide_5}.


\section{Features of FDS}

The first version of FDS was publicly released in February 2000.
To date, about half of the applications of the model have been for design of smoke
handling systems and sprinkler/detector activation studies. The other half consist of
residential and industrial fire reconstructions. Throughout its development, FDS has
been aimed at solving practical fire problems in fire protection engineering, while
at the same time providing a tool to study fundamental fire dynamics and combustion.




\backmatter
\nopart %To Fix TOC in PDF output.

\bibliography{../Bibliography/FDS_refs,../Bibliography/FDS_general,../Bibliography/FDS_mathcomp}

\end{document}

% Just a test comment
