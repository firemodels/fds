\documentclass[11pt]{book}
\usepackage{mathptm,times}
\usepackage[pdftex]{graphicx}
%\usepackage{hyperref}

%\usepackage{array,eqnarray}

%\usepackage{eso-pic}
%\usepackage{graphicx}
%\usepackage{color}
%\usepackage{type1cm}

%\makeatletter
%   \AddToShipoutPicture{%
%     \setlength{\@tempdimb}{.5\paperwidth}%
%    \setlength{\@tempdimc}{.5\paperheight}%
%   \setlength{\unitlength}{1pt}%
%  \put(\strip@pt\@tempdimb,\strip@pt\@tempdimc){%
%     \makebox(0,0){\rotatebox{45}{\textcolor[gray]{0.75}{\fontsize{8cm}{8cm}\selectfont{DRAFT}}}}}}
%\makeatother

\usepackage[pdftex,
        colorlinks=true,
        urlcolor=linkblue,     % \href{...}{...} external (URL)
        citecolor=linkred,     % citation number colors
        linkcolor=linknavy,    % \ref{...} and \pageref{...}
        pdftitle={Fire Dynamics Simulator (Version 5) Configuration Management Plan},
        pdfauthor={Kevin McGrattan, Bryan Klein, Simo Hostikka, Jason Floyd},
        pdfsubject={User Guide},
        pdfkeywords={FDS, Fire Model, NIST, BFRL},
        pdfproducer={pdflatex},
        pagebackref,
        pdfpagemode=UseNone,
        bookmarksopen=true,
        plainpages=false]{hyperref}
\usepackage{color}
\definecolor{linknavy}{rgb}{0,0,0.50196}
\definecolor{linkred}{rgb}{1,0,0}
\definecolor{linkblue}{rgb}{0,0,1}
\usepackage{caption}
\usepackage{graphpap}
\usepackage{rotating}
\usepackage{epsfig,psfrag}
\usepackage{geometry}
\usepackage{tabularx}
\usepackage{longtable}
\usepackage{lscape}
\usepackage{amssymb}
\usepackage{makeidx} % Create index at end of document
\usepackage[nottoc,notlof,notlot]{tocbibind} % Put the bibliography and index in the ToC
\usepackage{float}
\usepackage{lastpage} % Automatic last page number reference.
\usepackage[T1]{fontenc}
\usepackage{upquote}
\usepackage{array,eqnarray}
\newcommand{\nopart}{\expandafter\def\csname Parent-1\endcsname{}} % To fix table of contents in pdf.


\setlength{\textwidth}{6.5in}
\setlength{\textheight}{9.0in}
\setlength{\topmargin}{0.in}
\setlength{\headheight}{0.in}
\setlength{\headsep}{0.in}
\setlength{\parindent}{0.25in}
\setlength{\oddsidemargin}{0.0in}
\setlength{\evensidemargin}{0.0in}

\begin{document}

\bibliographystyle{unsrt}

\newcommand{\dod}[2]{\frac{\partial #1}{\partial #2}}
\newcommand{\DoD}[2]{\frac{D #1}{D #2}}
\newcommand{\dsods}[2]{\frac{\partial^2 #1}{\partial #2^2}}
\newcommand{\dx}{\delta x}
\newcommand{\dy}{\delta y}
\newcommand{\dz}{\delta z}
\newcommand{\x}{x}
\newcommand{\y}{y}
\newcommand{\z}{z}
\newcommand{\dt}{\delta t}
\newcommand{\dn}{\delta n}
\newcommand{\cH}{{\cal H}}
\newcommand{\hu}{u}
\newcommand{\hv}{v}
\newcommand{\hw}{w}
\newcommand{\la}{\lambda}
%\newcommand{\bO}{\mbox{\boldmath $\Omega$}}
\newcommand{\bO}{{\Omega}}
\newcommand{\bo}{{\bf \omega}}
%\newcommand{\btau}{\mbox{\boldmath $\tau$}}
\newcommand{\btau}{{\bf \tau}}
\newcommand{\bdelta}{{\bf \delta}}
\newcommand{\sumyw}{\sum (Y_\alpha/W_\alpha)}
\newcommand{\oW}{\overline{W}}
\newcommand{\om}{\omega}
\newcommand{\omx}{\omega_x}
\newcommand{\omy}{\omega_y}
\newcommand{\omz}{\omega_z}
\newcommand{\erf}{\hbox{erf}}
\newcommand{\bF}{{\bf F}}
\newcommand{\bG}{{\bf G}}
\newcommand{\bof}{{\bf f}}
\newcommand{\bq}{{\bf q}}
\newcommand{\br}{{\bf r}}
\newcommand{\bu}{{\bf u}}
\newcommand{\bx}{{\bf x}}
\newcommand{\bk}{{\bf k}}
\newcommand{\bv}{{\bf v}}
\newcommand{\bg}{{\bf g}}
\newcommand{\bn}{{\bf n}}
\newcommand{\bS}{{\bf S}}
\newcommand{\bW}{\overline{W}}
\newcommand{\dS}{d{\bf S}}
\newcommand{\bs}{{\bf s}}
\newcommand{\bI}{{\bf I}}
\newcommand{\hp}{{\cal H}}
\newcommand{\trho}{\tilde{\rho}}
\newcommand{\dph}{{\delta\phi}}
\newcommand{\dth}{{\delta\theta}}
\newcommand{\tp}{\tilde{p}}
\newcommand{\bp}{\overline{p}}
\newcommand{\dQ}{\dot{Q}}
\newcommand{\dq}{\dot{q}}
\newcommand{\dbq}{\dot{\bf q}}
\newcommand{\dm}{\dot{m}}
\newcommand{\ha}{\frac{1}{2}}
\newcommand{\ft}{\frac{4}{3}}
\newcommand{\ot}{\frac{1}{3}}
\newcommand{\fofi}{\frac{4}{5}}
\newcommand{\of}{\frac{1}{4}}
\newcommand{\twth}{\frac{2}{3}}
\newcommand{\R}{{\cal R}}
\newcommand{\be}{\begin{equation}}
\newcommand{\ee}{\end{equation}}
\newcommand{\RE}{\hbox{Re}}
\newcommand{\LE}{\hbox{Le}}
\newcommand{\PR}{\hbox{Pr}}
\newcommand{\PE}{\hbox{Pe}}
\newcommand{\NU}{\hbox{Nu}}
\newcommand{\SC}{\hbox{Sc}}
\newcommand{\SH}{\hbox{Sh}}
\newcommand{\WE}{\hbox{We}}
\newcommand{\COTWO}{{\tiny \hbox{CO}_2}}
\newcommand{\HTWOO}{{\tiny \hbox{H}_2\hbox{O}}}
\newcommand{\OTWO}{{\tiny \hbox{O}_2}}
\newcommand{\NTWO}{{\tiny \hbox{N}_2}}
\newcommand{\CO}{{\tiny \hbox{CO}}}
\newcommand{\F}{{\tiny \hbox{F}}}
\newcommand{\C}{{\tiny \hbox{C}}}
\newcommand{\Hy}{{\tiny \hbox{H}}}
\newcommand{\So}{{\tiny \hbox{S}}}
\newcommand{\M}{{\tiny \hbox{M}}}
\newcommand{\xx}{{\tiny \hbox{x}}}
\newcommand{\yy}{{\tiny \hbox{y}}}
\newcommand{\zz}{{\tiny \hbox{z}}}
\newcommand{\ct}{\tt\small}
\newcommand{\dif}{\mathrm{d}}
\newcommand{\Div}{\nabla\cdot}
\newcommand{\mhalf}{\mbox{$\frac{1}{2}$}}
\newcommand{\tripleprime}{{\prime\prime\prime}}

\pagestyle{empty}

\begin{minipage}[t][9in][s]{6.5in}

\huge \flushright{NIST Special Publication 1018-5}

\vspace{1in}

\Huge \flushright{Fire Dynamics Simulator (Version 5) \\ Technical Reference Guide \\ \Large Volume 4: Configuration Management Plan}

\vspace{.5in}

\large
\flushright{
Kevin McGrattan \\
Simo Hostikka \\
Jason Floyd \\
Bryan Klein}

\vspace{0.5in}

\flushright{In cooperation with: \\
VTT Technical Research Centre of Finland  }



\vfill


\flushright{\includegraphics[width=2.in]{FIGURES/nistident_flright_vec}}


\end{minipage}

\newpage

\hspace{5in}

\newpage

\begin{minipage}[t][9in][s]{6.5in}

\huge \flushright{NIST Special Publication 1018-5}

\vspace{.75in}

\Huge \flushright{Fire Dynamics Simulator (Version 5) \\ Technical Reference Guide \\ \Large Volume 4: Configuration Management Plan}

\vspace{.25in}

\normalsize
\flushright{
Kevin McGrattan \\
Bryan Klein \\
{\em Fire Research Division} \\
{\em Building and Fire Research Laboratory}  \\
\hspace{1.in} \\
Simo Hostikka \\
{\em VTT Technical Research Centre of Finland} \\
{\em Espoo, Finland}  \\
\hspace{1.in} \\
Jason Floyd \\
{\em Hughes Associates, Inc.}  \\
{\em Baltimore, Maryland, USA}}

\vspace{.25in}

\flushright{\today \\
FDS Version 5.2 \\
$SVN Repository$~$Revision$}

\vfill

\flushright{\includegraphics[width=1in]{FIGURES/doc} }

\small
\flushright{U.S. Department of Commerce \\
{\em Carlos M. Gutierrez, Secretary} \\
\hspace{1in} \\
National Institute of Standards and Technology \\
{\em James M. Turner, Acting Director} }

\end{minipage}

\newpage

\begin{minipage}[t][9in][s]{6.5in}

\flushright{Certain commercial entities, equipment, or materials may be identified in this \\
document in order to describe an experimental procedure or concept adequately. Such \\
identification is not intended to imply recommendation or endorsement by the \\
National Institute of Standards and Technology, nor is it intended to imply that the \\
entities, materials, or equipment are necessarily the best available for the purpose.
}

\vspace{3in}

\large
\flushright{\bf National Institute of Standards and Technology Special Publication 1018-5 \\
Natl.~Inst.~Stand.~Technol.~Spec.~Publ.~1018-5, \pageref{LastPage} pages (October 2007) \\
CODEN: NSPUE2 }

\vfill

\flushright{U.S. GOVERNMENT PRINTING OFFICE \\
WASHINGTON: 2007 \\
\rule{3.5in}{0.01in} \\
For sale by the Superintendent of Documents, U.S. Government Printing Office \\
Internet: bookstore.gpo.gov -- Phone: (202) 512-1800 -- Fax: (202) 512-2250 \\
Mail: Stop SSOP, Washington, DC 20402-0001 }

\end{minipage}



\newpage

\frontmatter

\pagestyle{plain}

\chapter{Disclaimer}

The US Department of Commerce makes no warranty, expressed or implied,
to users of the Fire Dynamics Simulator (FDS), and accepts no responsibility for its use.
Users of FDS assume sole responsibility under Federal law for determining
the appropriateness of its use in any particular application;
for any conclusions drawn from the results of its use; and for any
actions taken or not taken as a result of analysis performed using these tools.

Users are warned that FDS is intended for use only by those competent
in the fields of fluid dynamics, thermodynamics, heat transfer, combustion, and fire science,
and is intended only to supplement the informed judgment of the qualified user.
The software package is a computer model that may or may not have predictive
capability when applied to a specific set of factual circumstances.
Lack of accurate predictions by the model could lead to erroneous
conclusions with regard to fire safety. All results should be evaluated by an informed user.

Throughout this document, the mention of computer hardware or commercial
software does not constitute endorsement by NIST, nor does it indicate that
the products are necessarily those best suited for the intended purpose.


\chapter{About the Authors}

\begin{description}
\item[Kevin McGrattan] is a mathematician in the Building and Fire
Research Laboratory of NIST. He received a bachelors of science degree
from the School of Engineering and Applied Science of Columbia
University in 1987 and a doctorate at the Courant Institute of New
York University in 1991. He joined the NIST staff in 1992 and has
since worked on the development of fire models, most notably the Fire
Dynamics Simulator.
\item[Simo Hostikka] is a Senior Research Scientist at VTT Technical
Research Centre of Finland. He received a master of science
(technology) degree in 1997 and a doctorate in 2008 from
the Department of Engineering Physics and Mathematics of the
Helsinki University of Technology.  He is the principal developer of the
radiation and solid phase sub-models within FDS.
\item[Jason Floyd] is a Senior Engineer at Hughes Associates, Inc., in
Baltimore, Maryland. He received a bachelors of science degree and a
doctorate from the Nuclear Engineering Program of the University of
Maryland. After graduating, he won a National Research Council
Post-Doctoral Fellowship at the Building and Fire Research Laboratory
of NIST, where he developed the combustion algorithm within FDS. He is
currently funded by NIST under grant 60NANB5D1205 from the Fire
Research Grants Program (15 USC 278f).  He is the principal developer
of the multi-parameter mixture fraction combustion model and control
logic within FDS.
\item[Bryan Klein] is an Information Technology Specialist in the
Building and Fire Research Laboratory of NIST.  Before coming to NIST,
Bryan worked for five years with Western Fire Center, Inc., performing a
wide range of activities including fire modeling, data acquisition programming,
and quantitative fire measurements. His current focus is on FDS development and
user support, along with experimental model validation work.
\end{description}


\chapter{Acknowledgments}

\label{acksection}

The development and maintenance of the Fire Dynamics Simulator has been made possible through
a partnership of public and private organizations, both in the United States and abroad. Following
is a list of contributors from the various sectors of the fire research, fire protection engineering and
fire services communities:

FDS is supported financially via internal funding at both NIST and
VTT, Finland. In addition, support is provided by other agencies of
the US Federal Government, most notably the Nuclear Regulatory Agency
Office of Research. The US NRC Office of Research has funded key
validation experiments, the preparation of the FDS manuals, and the
development of various sub-models that are of importance in the area
of nuclear power plant safety. Special thanks to Mark Salley and Jason
Dreisbach for their efforts and support.  The Office of Nuclear
Material Safety and Safeguards, another branch of the NRC, has
supported modeling studies of tunnel fires under the direction of
Chris Bajwa and Allen Hansen.

Another source of support for FDS development has been the Microgravity Combustion Program of the National Aeronautics and Space
Administration (NASA).

Originally, the basic hydrodynamic solver was designed by Ronald Rehm
and Howard Baum with programming help from Darcy Barnett, Dan Lozier
and Hai Tang of the Computing and Applied Mathematics Laboratory
(CAML) at NIST, and Dan Corley of the Building and Fire Research
Laboratory (BFRL). Jim Sims of CAML developed the original
visualization software.  The direct pressure solver was written by
Roland Sweet of the National Center for Atmospheric Research (NCAR),
Boulder, Colorado.  Kuldeep Prasad added the multiple-mesh data
structures, paving the way for parallel processing.   Charles Bouldin
devised the basic framework of the parallel version of the code.

At NIST, Glenn Forney developed the visualization tool Smokeview that
not only made the public release possible, but it also serves as the
principal diagnostic tool for the continuing development of
FDS.

William Grosshandler and Tom Cleary, both currently at NIST, developed
an enhancement to the smoke detector activation algorithm, originally
conceived by Gunnar Heskestad of Factory Mutual. Steve
Olenick of Combustion Science and Engineering (CSE) implemented the
smoke detector model into FDS.

William Grosshandler is also the developer of RadCal, a library of
subroutines that have been incorporated in FDS to provide the
radiative properties of gases and smoke.

Prof.~Nick Dembsey of Worcester Polytechnic Institute (WPI) has provided valuable feedback about the pyrolysis model used within FDS.

Professor Fred Mowrer of the University of Maryland provided a simple
of model of gas phase extinction to FDS.

Chris Lautenburger of the University of California, Berkeley, and Jose Torero and Guillermo Rein of the University of Edinburg. have
provided valuable insight in the development of the solid phase model.








\tableofcontents

\mainmatter

\chapter{Introduction}

This plan is applicable to Fire Dynamics Simulator (FDS).  This document follows guidelines set forth in IEEE 
STD 828-1998, Standard for Software Configuration Management Plans.

\section{Purpose}

The purpose of this document is to identify and describe the overall policies and methods for CM to be used for
FDS.  This plan will be updated as the necessity arises. 

This CM Plan (CMP) will establish and provide the basis for a uniform and concise CM practice for FDS
The primary intention of this CMP is to provide information on the CM policies and methods to be adopted and
implemented by the FDS development team.

The primary objective of the FDS CMP is twofold.  First is to establish the process for tracking and controlling
FDS requirements and enhancements.  Second is to establish a change management status and reporting system for
officially released FDS versions, documentation, and associated input data files.

\section{Scope}

This plan covers the Modeling and Simulation (M\&S) configuration management activities associated with FDS.

\section{Fire Dynamics Simulator (FDS) Model Description}

FDS is a Computational Fluid Dynamics (CFD) model of fire-driven fluid flow.
The model solves numerically a form of the Navier-Stokes equations appropriate
for low-speed, thermally-driven flow with an emphasis on smoke and heat transport
from fires. The partial derivatives of the conservation equations of mass, momentum and energy are approximated
as finite differences, and the solution is updated in time on a three-dimensional, rectilinear grid.
Thermal radiation is computed using a finite volume technique on the same grid as the flow solver.
Lagrangian particles are used to simulate smoke movement, sprinkler discharge, and fuel sprays.

\chapter{CM Management}

Configuration management is defined as part of the project management system required for the technical and
administrative direction of the project.  This section describes the relationship of configuration management
to the project management structure as it relates to controlling the configuration of FDS.

\section{FDS CM Manager Responsibilities}

The CM Manager for FDS is Kevin McGrattan, Fire Modeling Group Leader of the Building and Fire Research
Laboratory at the National Insitute of Standards and Technology located in Gaithersburg, MD.  The responsibilities
of the CM Manager include:

\begin{enumerate}

\item Maintaining configuration control over the officially released software and data for FDS.

\item Final approval of changes to the officially released software and data for FDS.

\item Developing and maintaining the FDS Configuration Management Plan (CMP).

\item Maintaining all officially released software and data for the FDS configuration baselines and documenting
the baseline contents.

\item Authorize individuals to become members of the software development team (SDT).

\item Authorize establish of software baselines and identification of Computer Software Configuration Items (CSCIs)

\item Recording new FDS change proposals and tracking them to completion

\end{enumerate}

The CM Manager may designate a member of the SDT to be responsible for one or more CM Manager responsibilties.

\section{FDS Software Development Team Member Responsibilities}

FDS Development Team Members are those individuals allowed to make changes to the official FDS CSCI repository.
Their specific responsibilities include:

\begin{enumerate}

\item Represent interests of project management and all groups who may be affected by changes
to the software baselines. 

\item Assign, review, and provide for disposition of action items.

\item Determine or review the availability of resources required to complete the proposed change or
modification, assess the impact of the proposed change upon the system, examine cost considerations,
and determine the impact of the change on development schedules.

\end{enumerate}

\chapter{CM Process}

The IRM Suite Configuration Management (CM) process consists of four primary functions: configuration
identification, configuration control, configuration status accounting, and configuration audits.
The responsibilities of each CM function are listed in the paragraphs below, as they relate to
controlling the configuration of the IRM Suite.

\section{Configuration Identification}

Configuration identification consists of placing significant FDS related files under configuration control.
These products are called Computer Software Configuration Items (CSCIs).  CSCIs are defined as a collection
of software, data, or documentation required for an end use function and designated by the Project Engineer
for separate configuration management.  The CSCIs include:

\begin{enumerate}

\item Compiled FDS executables
\item FDS source code files
\item FDS documentation
\item Input files for software testing, verification testing, and validation testing
\item Experimental data files used for validation testing
\item Scripts and post-processing utilities used for software testing
\item FDS web pages and wikis

\end{enumerate}

\subsection{Maintaining CSCIs}

All CSCIs are maintained using the online service GoogleCode.  GoogleCode is a free service offered by Google
to support configuration management and software development for open source applications.  GoogleCode uses 
Subversion (SVN) for revision control.  Under this system a centralized repository containing all CSCIs resides
on a GoogleCode server.  Versions are indentified by a version number which represents the version of the
entire repository rather than of a specific file (i.e. anytime a change is made to the repository all files are
incremented in version number).  A record of version number when a specific file was last changed is maintained.

As an open source program, any individual can obtain a copy of the repository or retrieve specific versions
repository.  Only CM Manager approved members of the SDT; however, can make committ changes to the repository.

The current location of the FDS repository is \href{http://fds-smv.googlecode.com/svn/trunk/}
{{\ct http://fds-smv.googlecode.com/svn/trunk/}}

\subsection{Identifying Specific CSCIs in Compiled Executables}

At the start of an FDS simulation FDS writes header information to the Smokeview output file, FDS output file,
and the FDS log file.  This header information contains the version of FDS used to perform that simulation.
While each release is tagged with a specific version number (e.g. 5.0.1), there may be many commits of source
code, documentation, or other files to the SVN repository before a new version is released with an incremented
version number.  Thus, if a developer or a user who performs their own compilation between baseline releases
discovers an error, the version number written to the output files may not be sufficient to identify the
specific set of source code files used.  Rather one would need to know the SVN revision number of the most
recently committed source file.

This is accomplished by using source file tagging.  In each source file a series of character parameter
strings are defined.  For example the strings in main.f90 are:

\footnotesize
\begin{verbatim}
CHARACTER(255), PARAMETER :: mainid='$Id$'
CHARACTER(255), PARAMETER :: mainrev='$Revision$'
CHARACTER(255), PARAMETER :: maindate='$Date$'
\end{verbatim}
\normalsize

The string contains the name of the source file, the version of the file (this reflects only the source
file version and not the overall release version number), the date and time of the file version, and
the person who checked the file in to the archive.  Upon compiling, these strings will be stored in the
executable file.  The user is then capable of searching the executable file, for example, for strings
beginning with {\ct \$Id:}.  This will result in a list of all source files compiled and their version.
Within the SVN archive any specific version of a source file can be extracted and differences between
versions can be determined.

Within the source code a series of subroutines were created, one per source file, which parses the {\ct \$Id:} in
each file and extracts from it the SVN revision number.  Each of these subroutines is called at the start
of an FDS run and the largest (and hence most recent) revision number is determined.  This number is written
along with the FDS version number to the output header information.  A user can now identify specifically
the source code used for a particular compilation of FDS when reporting an error.

\section{Configuration Control}

Configuration control is the exercising of established procedures to classify, approve or disapprove, release,
implement, and confirm changes to officially released models and databases.  The CM Manager is the
ultimate is the gatekeeper for all changes required to an established officially released FDS Suite baseline.
These change requests could include:

\begin{enumerate}

\item Modifications of defined requirements

\item Corrections for deficiecies

\item Changes to existing capabilities 

\item Model enhancements

\end{enumerate}

\section{Software Changes}

\subsection{Creating a Change Request}

Change requests are submitted using the FDS Issue Tracker.  The Issue Tracker is an online service that is part of
GoogleCode.  The current location of the Issue Tracker is \href{http://code.google.com/p/fds-smv/issues/list}
{{\ct http://code.google.com/p/fds-smv/issues/list}}.  A change request is initiated by opening a new issue.
The issue report contains the baseline identification (version number, complie date, and SVN revision number),
operating system, and a description of the defect or ehancement request.  Input files or other supporting 
documentation can be attached.

\begin{figure}[ht!]
\includegraphics[width=\textwidth]{FIGURES/defectreport.jpg}
\caption{\bf Screenshot of issue tracker reporting form}
\label{fig:issueform}
\end{figure}

If the issue is opened by a user, it will be given a status of 'New' until it is reviewed by a developer.  If the
issue is opened by a developer, the developer can immeadiate assign a status and an owner.

\subsection{Processing a Change Request}

A change request may be evaluated by any member of the development team. 

If the request duplicates an existing request, then the issue status is changed to {\ct Duplicate} and the 
requestor is sent the issue number of the existing request.

If the request is for an enhancement to FDS, then then the issue type is changed to {\ct Type-Ehancement} and the
component type is declared. The request is evaluated for suitability.  If it is a valid enhancement 
request, then the status is changed to {\ct Accepted}.  If the request is not to
be addressed under the next baseline, then the status will be changed to {\ct OnHold}.  If the request is denied, then
the status is changed to {\ct WontFix}.  An accepted request will be assigned to a developer.

If the request is identifying a defect in FDS, then the issue type is changed to {\ct Type-Defect} and the
component type is declared.  The defect is evaluated for validity.  If insufficient information has
been provided in the request, then the status is changed to {\ct MoreInfo} and a description of the additional 
information required is sent to the requestor.  If the defect is due to user error or is the intended function of FDS,
then the status is changed to {\ct Invalid}.  If the defect is valid, then the request reviewer will change the 
status to {\ct Accepted} and assign the request to a developer.

Once a change request has be addressed by a developer and changes submitted to the repositoty, the request status
is changed to {\ct Fixed}.  Once either the requestor or another developer has verified that the changes address 
the original request, then the status is changed to {\ct Verified}.  

\section{Establishing Baselines}

The decision to establish a new baseline is made by the CM Manager.  The new baseline is named, tested, and the
released based on the procedures that follow.

\subsection{Naming Baselines}

New FDS baselines are identified using a specific naming convention. Baselines are identified as app.\#.\#.\#\_os. 
Where app is the application name (fds), \#.\#.\# is a version number, and \_os is the operating system that the
baseline executable was compiled for.  The version number consists of three integers where the first number
is the {\em major} release, the second is the {\em minor} release, and the third is the {\em maintenance}
release.  Major releases occur every few years, and as the name implies dramatically changed functionality of the
model. Minor releases occur every few months, and may cause minor changes in functionality. 
Release notes can help you decide whether the changes should effect the type of applications that you typically do.
Maintenance releases are either just bug fixes or the addition of minor enhancements (such as a new output quantity),
and should not affect code functionality.

\subsection{Baseline Software Testing}

Each proposed baseline release will undergo software testing.  Three suites of test cases exists: a functional test suite,
a verification test suite, and a validation test suite.  Testing will depend upon the type of baseline release: 
maintenance, minor, or major.

Each maintenance release will be tested with the functional test suite.  It will be verified that the proposed baseline
successfully executes all of the functional test cases.  

Each minor release will be tested with the functional test suite and the verification test suite.  It will be verified
that the proposed baseline successfully executes all of the functional and verification test cases.  The
verification manual will be updated with the results.

Each major release will be tested with all three test suites.    It will be verified
that the proposed baseline successfully executes all of the functional and verification test cases.  The verification
manual will be updated with the results.  The results of the validation cases will be evaluated to ensure that the
predictive performance of the proposed baseline as either remained equivalent or improved.  The validation manual will
be updated.

\subsection{Release of a New Baseline}

Following successful completion of the required baseline testing, a baseline can be released.  Prior to release, the 
version identification information within the FDS source code will be updated to reflect the new baseline.  FDS
documentation will be updated to reflect the new baseline.  The baseline will be compiled and new executable files 
or installation packages will be placed on the FDS download site.  Prior baselines will be deprecated.  The
current FDS dowload site is \href{http://code.google.com/p/fds-smv/downloads/list}
{{\ct http://code.google.com/p/fds-smv/downloads/list}}.

\chapter{Coding Standard}

\section{Indentation}

Each new level of source code will be indented by three spaces.  Indentation will occur following a {\ct DO},
{\ct IF} {\ct ELSE}, {\ct ELSEIF}, {\ct SELECT CASE}, or {\ct CASE} statement. Comment lines are not indented.

\section{Source File Organization}

Each source file in FDS shall contain subroutines and functions specific to narrow subsegment or physical submodel of
the software.  Currently the FDS source consists of the following files:

\begin{table}[ht]
\begin{center}
\caption{FDS Source Code Files}\index{Source Code Files}
\label{tbl:sourcecodefiles}
\begin{tabular}{|c|c|c|}
\hline
File Name    & Filetype & Description    \\ \hline 
cons.f90     & Fortran  & Definition of physical constants, parameters, global variables \\ \hline 
ctrl.f90     & Fortran  & Control functions  \\ \hline 
devc.f90     & Fortran  & Definition of device, control, and property variables \\ \hline 
divg.f90     & Fortran  & Divergence computation \\ \hline 
dump.f90     & Fortran  & Output processing \\ \hline 
evac.f90     & Fortran  & Evacuation model \\ \hline 
fire.f90     & Fortran  & Combustion model \\ \hline 
func.f90     & Fortran  & Miscellaneous functions \\ \hline 
ieva.f90     & Fortran  & Defintion of evacuation variables and evacuation functions \\ \hline 
init.f90     & Fortran  & Intialization \\ \hline 
irad.f90     & Fortran  & Radiation transport variables, RADCAL, and Mie solver \\ \hline 
isob.c       & C        & Functions for generating iso surfaces \\ \hline 
main.f90     & Fortran  & Main timestepping routine \\ \hline 
main\_mpi.f90 & Fortran  & Main timestepping routine for parallel version of FDS \\ \hline 
mesh.f90     & Fortran  & Defintion of mesh variables \\ \hline 
part.f90     & Fortran  & Droplet and particle mass and energy transport \\ \hline 
pois.f90     & Fortran  & CRAYFISHPAK Poisson solver \\ \hline 
prec.f90     & Fortran  & Kind defintions \\ \hline 
pres.f90     & Fortran  & Pressure solver \\ \hline 
radi.f90     & Fortran  & Radiation transport solver and initialization  \\ \hline 
read.f90     & Fortran  & Input processsing \\ \hline 
smvv.f90     & Fortran  & Interface routines for Smokeview \\ \hline 
turb.f90     & Fortran  & Module for testing turbulence models \\ \hline 
type.f90     & Fortran  & Variable type defintions \\ \hline 
vege.f90     & Fortran  & WUI vegetation model \\ \hline 
velo.f90     & Fortran  & Velocity solver \\ \hline 
wall.f90     & Fortran  & Wall heat transfer and boundary conditions                   \\ \hline
\end{tabular}s
\end{center}
\end{table}

\section{Explicit Variable Typing}

All variables in FDS will be explicitly typed.  To aid in this, all source code files will contain the expression
{\ct IMPLICIT NONE}.  All {\ct REAL} numbers in FDS will have their kind indicated.  For example:

\begin{verbatim}
REAL(EB) :: x
x = 0._EB
\end{verbatim}

and not:

\begin{verbatim}
REAL(EB) :: x
x = 0.
\end{verbatim}

\section{Loop and Branch Naming}

Any {\ct DO} loop, {\ct IF} block, or {\ct SELECT CASE} block that continues for more than 20 lines will be named.
For example:

\begin{verbatim}
LOOPNAME: DO 
   more than 20 lines of code
END DO LOOPNAME
\end{verbatim}

and not:

\begin{verbatim}
DO 
   more than 20 lines of code
END DO
\end{verbatim}

\section{Named Constants}

Named integer constants shall be used rather than integers or text strings for controlling code execution.  This
is done to enhance readability and improve execution speed by avoiding string comparisons.  For example:

\begin{verbatim}
INTEGER :: OPTIONS
INTEGER, PARAMETER :: OPTION1=1, OPTION2=2

SELECT CASE (OPTIONS) 
   CASE(OPTION1)
      ...
   CASE(OPTION2)
      ...
END SELECT
\end{verbatim}

and neither:
\begin{verbatim}
INTEGER :: OPTIONS

SELECT CASE (OPTIONS) 
   CASE(1)
      ...
   CASE(2)
      ...
END SELECT
\end{verbatim}

nor:
\begin{verbatim}
CHARACTER(10):: OPTIONS

SELECT CASE (OPTIONS) 
   CASE('OPTION1')
      ...
   CASE('OPTION2')
      ...
END SELECT
\end{verbatim}

\section{Redundant Code}

Where a block of code must be used repeatedly in different subroutines a {\ct FUNCTION} or {\ct SUBROUTINE}
shall be created.  For example, there are a number of places where FDS must compute the average molecular weight
of a gas.  Rather than having that block of code repeated in each location, a single function was created to
perform this function.  This accomplishes two goals.  By using a function or subroutine, a meaningful name can
be used to represent the function being performed which improves source code readibilty.  If the form of the
function must be changed do to a bug fix or an enhancement, then having only one location to fix reduces that
chance of the error that would occur if one of multiple locations was missed.

\backmatter
\nopart %To Fix TOC in PDF output.

\bibliography{../Bibliography/FDS_refs,../Bibliography/FDS_general,../Bibliography/FDS_mathcomp}

\printindex

\end{document}

% Just a test comment
