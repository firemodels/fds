\chapter{Combustion}

\label{combustionsection}

There are two approaches to modeling a chemical reaction within FDS.  The first approach is a mixing-controlled combustion model.  In this approach, the reaction rate is computed based on the local mixing.  That is the reactants (e.g. fuel and air) react inifintely fast at the rate that the underlying gas transport mixes them together.  For the second model, individual gas species react according to
specified Arrhenius reaction parameters. This latter model is most often used in a
direct numerical simulation (DNS) where the diffusion of fuel and oxygen can be
modeled directly.
However, most often for large eddy simulations (LES), where the grid is not
fine enough to resolve the diffusion of fuel and oxygen,
the mixing-controlled combustion model is assumed.

\section{Mixing-Controlled Combustion Model}

For an infinitely-fast reaction, reactant species in a given grid cell are converted to product species at a rate determined by a
characteristic mixing time.  It is assumed that the
grid resolution is too coarse to resolve the flame sheet and compute directly the reaction rate.  Instead, if any grid cell contains all the reactants of the the chemical reaction and the temperature of the grid cell meets certain criteria, the consumption rate of fuel is given by~\cite{Poinsot:TNC}
\be \dm_f''' = -\rho \min \left( Y_\F , \frac{Y_\OTWO}{s}, \beta \frac{Y_{\hbox{\tiny P}}}{1+s} \right) \; \left( 1 - e^{-\dt/\tau} \right)   \quad ; \quad
   s=\frac{W_\F}{\nu_{\OTWO} W_{\OTWO} }  \label{EDC} \ee
Here, $\tau$ is a mixing time scale and $\beta$ is an empirical parameter equal to 1.

While the reaction rate given in Eq.~(\ref{EDC}) is an easily computed and robust subgrid-scale model of
turbulent combustion, there is still a need, in certain situations, to put an upper bound on the local heat release rate per unit volume. The reason for
this is that FDS is applied over length scales ranging from millimeters to tens of meters, and the resolution of the numerical grid
is sometimes too coarse to expect the simple mixing time model to work effectively.
A scaling analysis of pool fires by Orloff and De Ris~\cite{Orloff:19th_Symposium} suggests that the spatial average of the
heat release rate of a fire is approximately 1200~kW/m$^3$. FDS uses by default a value of 2500~kW/m$^3$ as an upper bound\footnote{Note that
for DNS, FDS imposes a less restrictive upper bound on the local heat release rate per unit volume. It is
\be \dq_{\max}''' = 200/\dx + 2500 \quad \hbox{kW/m}^3 \ee
The value of 200~kW/m$^2$ is an upper bound on the heat release rate per unit area of flame sheet.}
on the local value of the heat release rate per unit volume. Typically, this bound only affects fires whose value of $Q^*$
is less than one\footnote{The non-dimensional quantity, $Q^*$, is a measure of the fire's heat release rate divided by the
area of its base. It is expressed as $Q^*=\dQ/(\rho_\infty c_p T_\infty \sqrt{g D} D^2)$.}.

\subsection{A Mixing-Controlled Reaction, but with Local Extinction}

\label{extinction}

The physical limitation of the mixing-controlled reaction model described in the previous section is that it assumes that fuel and oxygen burn instantaneously when mixed. For large-scale, well-ventilated
fires, this is a good assumption. However, if a fire is in an
under-ventilated compartment, or if a suppression agent like water
mist or CO$_2$ is introduced, or if the shear layer between fuel and oxidizing streams
has a sufficiently large local strain rate,
fuel and oxygen may mix but may not burn.
The physical mechanisms underlying these phenomena are complex, and
even simplified models still rely on an accurate prediction
of the flame temperature and local strain rate.
Subgrid-scale modeling of gas phase suppression and
extinction is still an area of active research in the combustion
community.

Simple empirical rules can be used to predict local
extinction based on the species and temperature present in the flame sheet.  The FDS extinction model consists of two parts. The first part, checks to see if the local temperature is above an auto-ignition temperature for the fuel.  If the temperature is too low, then combustion will not occur.  Note that this temperature is by default set to absolute zero so that typical users do not need to specify an ignition source.  The second part uses the concept of a limiting flame temperature.  If the local combustion cannot raise the local temperature above the limiting flame temperature, then combustion will not occur.  This is done in the following steps:

\begin{enumerate}
\item Search over all reactant species to find the limiting species and express that species in terms of the reaction's fuel:

\be \Delta Z_\F = \min_{i \; = \; reactant} \left(\frac{Z_i W_\F}{W_i \nu_i} \right) \ee
\item Remove that amount of fuel from the local gas, the resulting gas is the "air" for the reaction.
\item Search over all the non-fuel reactant species in the "air" to determine the limiting reactant.  This defines how much air is required to burn the fuel:

\be \Delta Z_{Air} = \min_{i \; = \; non-fuel \; reactant} \left(\frac{\Delta Z_\F W_i \nu_i}{Z_i W_\F} \right) \ee

\item Compute the enthalpy for the fuel and the air at both the current temperature and the limiting flame temperature.
\item Combustion is allowed if 

\be \Delta Z_{Air} h_{air}(T)+\Delta Z_F \left( h_\F(T)+\Delta H_\F \right) > \Delta Z_{Air} h_{air}(T_{LFT})+ \Delta Z_\F h_\F(T_{LFT}) \ee

\end{enumerate}

\subsection{Simplified Chemistry}

\label{simplechemistry}

Most ordinary combustibles can be represented by the simple single-step reaction:

\be  \mathrm{C_xH_yO_zN_a} +  \nu_\OTWO \, \mathrm{O_2}  \rightarrow  \nu_\COTWO \, \mathrm{CO_2} + \nu_\HTWOO \, \mathrm{H_2O} + \nu_\CO \, \mathrm{CO} +
     \nu_\So \, \mathrm{S}  + \nu_\NTWO \, \mathrm{N_2}  \label{stoich}
\ee
Note that the nitrogen in the fuel molecule is assumed to form $\mathrm{N_2}$ only. Soot is assumed to be a mixture of carbon and hydrogen with the hydrogen atomic
fraction given by $X_\Hy$. The stoichiometric coefficient, $\nu_\So$, represents the amount of fuel that is converted to soot. It is related to the
{\em soot yield}, $y_\So$, via the relation:
\be
   \nu_\So = \frac{W_\F}{W_\So} \; y_\So  \quad ; \quad  W_\So = X_\Hy W_\Hy + (1 - X_\Hy) W_\C  \label{soot_yield}
\ee
Likewise, the stoichiometric coefficient of CO, $\nu_\CO$, is related to the CO {\em yield}, $y_\CO$, via:
\be
   \nu_\CO = \frac{W_\F}{W_\CO} \; y_\CO  \label{CO_yield}
\ee
The yields of soot and CO are based on ``well-ventilated'' or ``post-flame'' measurements. The increased production of CO and soot in an under-ventilated
compartment will be addressed in the following sections.

This reaction can also be represented as:

\be  \mathrm{Fuel} +  \nu_\mathrm{Air} \, \mathrm{Air}  \rightarrow  \mathrm{Products}\label{stoichair}
\ee

Air is a lumped species consisting of a mixture of nitrogen, oxygen, water vapor, and carbon dioxide (carbon dioxide is included to attenuate radiation over large distances and water vapor is included for attenuation and to support simulations with sprinklers).  Products is a lumped species consisting of all the products listed in \ref{stoich} plus the nitrogen, water vapor, and carbon dioxide from the Air that reacted with the fuel.  If we allow for the presence of a diluent gas in the fuel stream, then the Fuel becomes a lumped species consisting of fuel and diluent.  We name these species $Z_0$ for Air, $Z_1$ for Fuel, and $Z_2$ for Products.  When using simple chemistry, $Z_1$ and $Z_2$ are tracked explicitly and $Z_0$ is tracked implicitly as the background species.  The mass fractions of the component gases in these species are given as:

\noindent
$Z_0$: Air
\begin{eqnarray}
Y_{\NTWO}(Z_0) & = & Y_\NTWO^\infty\\*[.1in]
Y_{\OTWO}(Z_0) & = & Y_\OTWO^\infty\\*[.1in]
Y_{\COTWO}(Z_0) & = & Y_\COTWO^\infty\\*[.1in]
Y_{\HTWOO}(Z_0) & = & Y_\HTWOO^\infty
\end{eqnarray}
$Z_1$: Fuel
\begin{eqnarray}
Y_{\F}(Z_1) & = & Y_\F
\end{eqnarray}
$Z_2$: Products
\begin{eqnarray}
Y_{\NTWO}(Z_2) & = & \frac{\nu_{Air} W_{Air}  Y_\NTWO^\infty +\nu_{\NTWO} W_{\NTWO}}{W_F + \nu_{Air} W_{Air}}\\*[.1in]
Y_{\COTWO}(Z_2) & = & \frac{\nu_{Air} W_{Air}  Y_\COTWO^\infty +\nu_{\COTWO} W_{\COTWO}}{W_F + \nu_{Air} W_{Air}}\\*[.1in]
Y_{\HTWOO}(Z_2) & = & \frac{\nu_{Air} W_{Air}  Y_\HTWOO^\infty +\nu_{\HTWOO} W_{\HTWOO}}{W_F + \nu_{Air} W_{Air}}\\*[.1in]
Y_{\CO}(Z_2) & = & \frac{\nu_{\CO} W_{\CO}}{W_F + \nu_{Air} W_{Air}}\\*[.1in]
Y_{\So}(Z_2) & = & \frac{\nu_{\So} W_{\So}}{W_F + \nu_{Air} W_{Air}}
\end{eqnarray}
Species yields of combinations of $Z_0$, $Z_1$, and $Z_2$ are given as:
\be
Y_\alpha(Z_0,Z_1,Z_2)=Y_\alpha(Z_0) \, (1 - Z_1 - Z_2) + Y_\alpha(Z_1) \, Z_1 + Y_\alpha(Z_2) \, Z_2
\ee
The stoichiometric coefficients in the $Z_2$ species yields are:

\parbox{2.5in}{
\begin{eqnarray*}  \nu_\NTWO  &=& \frac{\hbox{a}}{2}\\*[.1in]
                  \nu_\OTWO  &=& \nu_\COTWO + \frac{\nu_\CO+\nu_\HTWOO-\hbox{z}}{2}\\*[.1in]
                  \nu_\COTWO &=& \hbox{x} - \nu_\CO - (1-X_\Hy) \nu_\So  \\*[.1in]
\end{eqnarray*} }
\hfill \parbox{3.5in}{\begin{eqnarray}
                  \nu_\HTWOO &=& \frac{\hbox{y}}{2}- X_\Hy \nu_\So\\*[.1in]
                  \nu_\CO    &=& \frac{W_\F}{W_\CO} \; y_\CO \\*[.1in]
                  \nu_\So    &=& \frac{W_\F}{W_\So} \; y_\So
\end{eqnarray} }
Remember that x is the number of carbon atoms, y is the number of hydrogen atoms, z is the number of oxygen atoms, and a is the number of nitrogen atoms in the fuel molecule.
It is important to note that the definitions of $Z_0$, $Z_1$, and $Z_2$ do not imply anything regarding the rate of combustion, only that the combustion occurs in a single step.


\subsection{CO Production (Two-Step Reaction with Extinction)}

\label{co_production}

The previous section describes the ``complete'' reaction as the conversion of fuel to
products such that the production rate of each product species is proportional to the fuel consumption rate.
This means that for each fuel molecule, fixed amounts of CO$_2$, H$_2$O, CO, and soot are formed and these products
persist in the plume indefinitely with no further reaction. This is not an unreasonable assumption if
the purpose of the fire simulation is to assess the impact of the fire on the larger space.
However, in under-ventilated fires, soot and CO are produced at higher rates,
and exist within the fuel-rich flame envelope at higher concentrations,
than would otherwise be predicted with a single set of fixed yields that are based on post-flame measurements. To account for the
production of CO and its eventual oxidation at the flame envelope or within a hot upper layer,
an additional reaction is now needed:
\begin{eqnarray}
\mathrm{C_xH_yO_zN_a} +  \nu_\OTWO' \mathrm{O_2}  &\rightarrow&  \nu_\HTWOO \mathrm{H_2O} + (\nu_\CO'+ \nu_\CO) \, \mathrm{CO} +
     \nu_\So \, \mathrm{S}  + \nu_\NTWO \, \mathrm{N_2}   \\*[.1in]
\nu_\CO' \; \Big[ \mathrm{CO} + \ha \mathrm{O_2}  &\rightarrow&  \mathrm{CO_2}  \Big]
\label{3reac} \end{eqnarray}
The brackets around the second reaction are there merely to emphasize that the sum of the two reactions equal Eq.~(\ref{stoich}).
There are two stoichiometric coefficients for CO -- the first, $\nu_\CO'=\hbox{x}-(1-X_\Hy) \nu_\So-\nu_\CO$,
represents CO that is produced in the first
step of the reaction that can potentially be converted to CO$_2$ assuming the conditions are favorable. $\nu_\CO'$ is equivalent to $\nu_\COTWO$ in
Eq.~(\ref{stoich}). The second coefficient, $\nu_\CO$,
is the so-called ``well-ventilated,'' or ``post-flame,'' value that was introduced in the previous section. The proposed model of CO production
still does not contain the necessary kinetic mechanism to predict the ``post-flame'' concentration of CO without the prescription of the
measured value of the post-flame CO yield. Rather, the proposed model includes the production of large amounts of CO in the first step of a two-step
reaction, followed by a partial conversion to CO$_2$ if there is a sufficient amount of oxygen present.

To describe the composition of the gas species, an additional lumped species is required bringing the total to four lumped species.  We can rewrite the above two step reaction as:

\begin{eqnarray}
Fuel + \nu_{Air,1} Air &\rightarrow&  Incomplete \; Products\\*[.1in]
Incomplete \; Products + \nu_{Air,2} Air &\rightarrow&  Complete \; Products
\label{4lumped}
\end{eqnarray}

It can be seen from this that the four lumped species are: $Z_0$ for Air, $Z_1$ for Fuel, $Z_2$ for the products of incomplete combustion, and $Z_3$ for the products of complete combustion.  The species yields are:

$Z_0$: Air

\begin{eqnarray}
Y_{\NTWO}(Z_0) & = & Y_\NTWO^\infty\\*[.1in]
Y_{\OTWO}(Z_0) & = & Y_\OTWO^\infty\\*[.1in]
Y_{\COTWO}(Z_0) & = & Y_\COTWO^\infty\\*[.1in]
Y_{\HTWOO}(Z_0) & = & Y_\HTWOO^\infty
\end{eqnarray}

$Z_1$: Fuel

\begin{eqnarray}
Y_{\F}(Z_1) & = & Y_\F
\end{eqnarray}

$Z_2$: Products of Incomplete Combustion

\begin{eqnarray}
Y_{\NTWO}(Z_2) & = & \frac{\nu_{Air,1} W_{Air} Y_\NTWO^\infty +\nu_{\NTWO} W_{\NTWO}}{W_F + \nu_{Air} W_{Air}}\\*[.1in]
Y_{\COTWO}(Z_2) & = & \frac{\nu_{Air} W_{Air} Y_\COTWO^\infty}{W_F + \nu_{Air} W_{Air}} \\*[.1in]
Y_{\HTWOO}(Z_2) & = & \frac{\nu_{Air} W_{Air}  Y_\HTWOO^\infty +\nu_{\HTWOO} W_{\HTWOO}}{W_F + \nu_{Air} W_{Air}}\\*[.1in]
Y_{\CO}(Z_2) & = & \frac{\nu_{\CO'} W_{\CO}}{W_F + \nu_{Air} W_{Air}}\\*[.1in]
Y_{\So}(Z_2) & = & \frac{\nu_{\So} W_{\So}}{W_F + \nu_{Air} W_{Air}}
\end{eqnarray}

$Z_3$: Products of Complete Combustion

\begin{eqnarray}
Y_{\NTWO}(Z_2) & = & \frac{\nu_{Air,2} W_{Air}  Y_\NTWO^\infty +\nu_{\NTWO} W_{\NTWO}}{W_F + \nu_{Air} W_{Air}}\\*[.1in]
Y_{\COTWO}(Z_2) & = & \frac{\nu_{Air} W_{Air}  Y_\COTWO^\infty +\nu_{\COTWO} W_{\COTWO}}{W_F + \nu_{Air} W_{Air}}\\*[.1in]
Y_{\HTWOO}(Z_2) & = & \frac{\nu_{Air} W_{Air}  Y_\HTWOO^\infty +\nu_{\HTWOO} W_{\HTWOO}}{W_F + \nu_{Air} W_{Air}}\\*[.1in]
Y_{\CO}(Z_2) & = & \frac{\nu_\CO W_{\CO}}{W_F + \nu_{Air} W_{Air}}\\*[.1in]
Y_{\So}(Z_2) & = & \frac{\nu_{\So} W_{\So}}{W_F + \nu_{Air} W_{Air}}
\end{eqnarray}


The stoichiometric coefficients are defined:

\parbox{2.5in}{
\begin{eqnarray*} \nu_\NTWO  &=& \frac{\hbox{a}}{2}\\*[.1in]
                  \nu_\OTWO' &=& \frac{\nu_\CO'+\nu_\HTWOO-z}{2}\\*[.1in]
                  \nu_\OTWO  &=& \nu_\COTWO + \frac{\nu_\CO+\nu_\HTWOO-z}{2}\\*[.1in]
                  \nu_\COTWO &=& \hbox{x} - (1-X_\Hy) \nu_\So \\*[.1in]
                  \nu_\M     &=& \hbox{b} \end{eqnarray*} }
\hfill \parbox{3.5in}{\begin{eqnarray}
                  \nu_\HTWOO &=& \frac{\hbox{y}}{2}- X_\Hy \nu_\So\\*[.1in]
                  \nu_\CO'   &=& x - \nu_\CO - (1-X_\Hy) \nu_\So \\*[.1in]
                  \nu_\CO    &=& \frac{W_\F}{W_\CO} \; y_\CO \\*[.1in]
                  \nu_\So    &=& \frac{W_\F}{W_\So} \; y_\So
\end{eqnarray} }
Although these formulae appear complicated, most are determined directly from the composition of the fuel molecule. The only information
expected of the modeler are the fuel composition, the soot and CO yields, and the atomic fraction of hydrogen in the soot.


\subsection{Heat Release Rate}

The discussion of the various multi-step reactions above is essentially book-keeping, the accounting of the gas
molecules formed in the combustion process. But what of the heat released?

When the gas constituents are characterized by three lumped species variables, there is a single step reaction that
converts fuel and oxygen into a
fixed, predefined set of combustion products. Combustion is either allowed or
disallowed depending on whether there is enough fuel or oxygen locally to maintain the 
reaction\footnote{Note that the user has control over the parameters associated with local gas phase extinction.}.
If combustion is allowed to occur in a grid
cell, the single-step reaction is assumed to be infinitely fast, and the rate of fuel consumption is controlled
only by the mixing rate of fuel and oxygen. This rate is modeled as described below.

In the case of the two-step, four lumped species model, the first step converts the
fuel to CO and other combustion products, and the second step oxidizes the CO into CO$_2$.  The first step is
determined as it is in the single step reaction case.  The second step, however, is assumed to be rate-dependent.


\subsubsection{One-Step, Fast Reaction}





\subsubsection{Two-Step, Fast-Slow Reaction}

When the mixture fraction is divided into three components, $Z_1$, $Z_2$, and $Z_3$, there are two chemical
reactions that convert $Z_1$ to $Z_2$ and $Z_2$ to $Z_3$.  Recall from Section~\ref{co_production}
that this represents two-step combustion (fuel to CO and CO to CO$_2$).
The first step occurs as it does for the two-parameter mixture fraction with a modified heat of combustion that
accounts for the conversion of fuel to CO rather than CO$_2$.
The second step is performed for all grid cells that contain CO and O$_2$.   If $\dq''' \neq 0$ in a grid
cell after the first step, then additional heat is released according to
\be \dq_{\CO}''' = \min \left[ \frac{ \max \left( \rho Z_2 , s \rho Y_\OTWO \right) }{\dt} \,
  \Delta H_\CO \; , \; \dq_{\max}'''-\dq'''  \right]     \ee
If $\dq'''=0$ after the first step, then it is presumed that the cell is out of the combustion region (say in the upper layer of
smoke-filled compartment), and a finite-rate reaction computation is performed to convert CO to CO$_2$ (see the next section for
a discussion of the algorithm for computing a finite-rate reaction).  The $\dq'''_{CO}$ computed using the finite-rate
reaction is still limited by $\dq_{\max}'''$.  Once $\dq'''_{\CO}$ is computed the mixture fraction variables are updated:
\be {Z_2}^{n+1} = {Z_2}^n - \frac {\dq'''_{\CO}{\Delta t}}{\rho \, \Delta H_{\CO}} \quad ; \quad
{Z_3}^{n+1} = {Z_3}^n + \frac {\dq'''_{\CO}{\Delta t}}{\rho \, \Delta H_{\CO}} \ee



\newpage
\section{Finite-Rate, Multiple-Step Combustion Model}

In a DNS calculation, the fine grid resolution enables the direct modeling of the diffusion of chemical species (fuel,
oxygen, and combustion products).  Since the flame is being resolved in a DNS calculation, the local gas
temperatures can be used to determine the reaction kinetics.  The general form of the rate expression is

\be \frac{d[X_\F]}{dt}    = -A \; \prod \left([X_i]^{a_i} \right) T^n \; e^{-E/RT_{ijk}} \label{Arrheniusrateeqn} \ee

note that the rate expression can be dependent on temperature and that the rate can be dependent upon species that do not directly participate in the reaction.  Since FDS solves for mass fractions rather than mole fractions, to reduce the computational expense the relation $[X_i]=\frac{Y_i \rho}{W_i}$ is used to obtain a modified rate equation:

\be \frac{dY_\F}{dt}    = -A_{mod} \rho^{\sum (a_i) -1} \; \prod \left(Y_i^{a_i} \right) T^n \; e^{-E/RT_{ijk}} \;\;\; A_{mod} = A \prod \left(W_i^{-a_i} \right) \label{Arrheniusratemode} \ee

Thus, it is possible to implement a relatively simple
set of one or more chemical reactions to model the combustion. Consider the reaction of oxygen and a hydrocarbon
fuel
\be  \mathrm{ C_xH_y + \nu_{O_2}  \, O_2 \longrightarrow
     \nu_{CO_2}  \,  CO_2 +
     \nu_{H_2 O}  \, H_2O }   \ee
If this were modeled as a single-step reaction, the reaction rate would be given by the expression
\be \frac{d[\mathrm{C_xH_y}]}{dt} = -A \, [\mathrm{C_xH_y}]^a \, [\mathrm{O_2}]^b \, e^{-E/RT}
   \label{reaction} \ee
Suggested values of $A$, $E$, $a$ and $b$ for various hydrocarbon
fuels are given
in Refs.~\cite{Puri:1,Westbrook:1}. It should be understood that the
implementation of any of these one-step reaction schemes is still very
much a research exercise because it is not universally accepted that
combustion phenomena can be represented by such a simple mechanism.
Improved predictions of the heat release rate may be possible by considering a multi-step set of reactions.
However, each additional gas species defined in the computation incurs a roughly 5~\% increase in the CPU time.  The use of lumped species may be able reduce this added computational expense.

For finite-rate chemistry, it is assumed that the chemical reaction
time scale is much shorter than any convective or diffusive
transport time scale. Thus, it makes sense to calculate the
consequences of the reaction assuming all other processes are
frozen in a state corresponding to the beginning of the time step.
For each grid cell, at the start of a time step where $t=t^n$ and
$Y_{C_x H_y,ijk}^n \rho_{ijk}/ W_F       \equiv X_\F(t^n)$ and
$Y_{\OTWO,ijk}^n   \rho_{ijk}/ W_\OTWO \equiv X_\OTWO(t^n)$,
the following set of ODEs is solved numerically with a second-order Runge-Kutta scheme
\begin{eqnarray}
\frac{dX_\F}{dt}    &=& -A \; X_\F(t)^a \, X_\OTWO(t)^b \; e^{-E/RT_{ijk}} \label{finitefuel} \\
\frac{dX_\OTWO}{dt} &=& -\frac{\nu_\OTWO}{\nu_\F} \; \frac{dX_\F}{dt} \label{finiteoxidizer}
\end{eqnarray}
The temperature $T_{ijk}$ and density $\rho_{ijk}$ are fixed at
their values at time
$t^n$ and the ODEs are iterated from $t^n$ to $t^{n+1}$ in about 20 time
steps. The pre-exponential factor $B$, the activation energy
$E$, and the exponents $a$ and $b$ are input parameters which are in typically assigned the values of
$\nu_\F$ and $\nu_\OTWO$.  At the end of each sub-time step the values of $X_\F$ and $X_\OTWO$ are updated.

The average heat release rate over the entire time step is given by
\be \dq_{ijk}^{'''n} = \Delta H \; \rho_{ijk}^n \frac{Y_\F(t^n) - Y_\F(t^{n+1})}{\dt} \ee
where $\dt = t^{n+1}-t^n$.
The species mass fractions are adjusted at this point in the calculation
(before the convection and diffusion update)
\be Y_{\alpha,ijk}^n = Y_\alpha(t^n) - \frac{\nu_\alpha \, W_\alpha}{\nu_\F \, W_\F} \big(Y_\F(t^n)-Y_\F(t^{n+1})\big) \ee
If multiple chemical reactions have been specified, equations \ref{finitefuel} and \ref{finiteoxidizer}
are evaluated for each reaction during each of the 100 time steps.  The reactions are evaluated in the order
that they are entered in the input file.

