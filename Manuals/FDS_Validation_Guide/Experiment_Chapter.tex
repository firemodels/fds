\chapter{Description of Experiments}

\label{Experiments_Chapter}

This chapter contains a brief description of the experiments that were used for model validation. Only enough detail is included here to provide a
general understanding of the model simulations. Anyone wishing to use the experimental measurements for validation ought to consult the cited test reports
or other publications for a comprehensive description.


\section{ArupFire Tunnel Fire Experiments}

Gabriele Vigne and Jimmy J\"{o}nsson of ArupFire conducted a series of fire experiments within a tunnel with a 50~m$^2$ cross section. The tunnel is located
in La Ribera del Folgoso, Spain. It is approximately 6.5~m high, 8~m wide and 300~m long.
Five replicate tests were conducted using a 1~m by 2~m steel pan filled with heptane on water. Near-ceiling temperatures
were measured 2~m, 4~m, 6~m and 8~m from the plume centerline. The peak heat release rate was approximately 5.3~MW.


\section{ATF Corridors Experiments}

A series of eighteen experiments were conducted in a two-story structure with long hallways and a connecting stairway
in the large burn room of the ATF Fire Research Laboratory in Ammendale, Maryland, in 2008~\cite{Sheppard:Corridors}.
The test enclosure consisted of two 17.0~m long hallways connected by a
stairway consisting of two staircases and an intermediary landing.
There was a door at the opposite end of the first floor hallway, which was closed during all tests.
The end of the second floor hallway was open with a soffit near the ceiling.

The walls and ceilings of the test structure were constructed of 1.2~cm gypsum wallboard.
The flooring throughout the structure, including the stairwell landing floor, consisted of one layer of 1.3~cm thick cement board on one
layer of 1.9~cm thick plywood supported by wood joists. The first set of stairs, which had eight risers, led from the first floor up to the landing area.
The second set of stairs, which had nine risers, led from the landing area up to the second floor.
The stairs were constructed of 2.5~cm thick clear pine lumber. The two set of stairs were separated by an approximately 0.42~m wide gap in the middle of the stairwell.
This gap was separated from the stairs by a 0.91~m tall barrier constructed of a single piece of gypsum board.
The flue space was open to the first floor.  The flue space was separated from the second floor by a 0.9~m tall barrier constructed of gypsum board.
There was a metal exterior type door at the end of the first floor near the burner.  The door was closed during all experiments.

The fire source was a natural gas diffusion burner.  The burner surface was horizontal, square and 0.45~m on each side, its surface was 0.37~m above the floor, and it was filled with gravel.
The burner was located near the end of the first floor away from the stairs. A diagram of the test structure is displayed in Figure~\ref{ATF Drawing}.


\begin{sidewaysfigure}[h]
\begin{center}
\includegraphics[height=6.5in]{FIGURES/ATF_Corridors/ATF}
\end{center}
\caption{Geometry of the ATF Corriodrs Experiments}
\label{ATF Drawing}
\end{sidewaysfigure}

\clearpage



\section{Beyler Hood Experiments}

Craig Beyler performed a large number of experiments involving a variety of fuels, fire sizes, burner diameters, and
burner distances beneath a hood~\cite{Beyler:Hood}.  The hood consisted of concentric cylinders separated
by an air gap.  The inner cylinder was shorter than the outer and this allowed combustion products to be removed
uniformly from the hood perimeter.  The exhaust gases were then analyzed to determine species concentrations.
The burner could be raised and lowered with respect to the bottom edge of the hood.  Based on the published
measurement uncertainties, species errors are estimated at 6~\%.



\section{BRE Spray Test for Radiation Attenuation}

Murrel {\em et al.}~\cite{Murrel:1995} measured the attenuation of thermal radiation passing through a water
spray using a heat flux gauge. The radiation was produced by a heat panel, one meter square, at 900~$^\circ$C. The horizontal distance
from the radiation panel to the spray nozzle was 2 m and to the measurement point 4 m. The nozzles were positioned at
a height 0.24 m above the panel upper edge. The heat flux gauge was positioned at the line passing through the center
of the panel. The attenuation of radiation was defined as $(q_0-q_s)/q_0$, where $q_0$ is the initial radiative heat flux,
measured without a spray, and $q_s$ is the heat flux measured during the spray operation.

Experimental results are used from three full-cone type nozzles, labeled A, B and D. The opening angles of the nozzles were between 90 and 108 degrees.
The purpose of the simulation is to compare the measured and simulated attenuation of radiation at different flow conditions. The nozzles were
specified in terms of median droplet size and mean vertical velocity using PDPA measurement in a single position, 0.7 m below the nozzle. The droplet
boundary conditions were determined by assuming $d_m \propto p^{-1/3}$ and $v \propto p^{1/2}$ type of dependences between the droplet size, speed
and pressure.

\section{Bryant Doorway Velocity Measurements}

Rodney Bryant of the Fire Research Division at NIST performed a series of velocity measurements of the gas
velocity within the doorway of a standard ISO~9705 compartment for fires ranging from
34~kW to 511~kW~\cite{Bryant:FSJ2009,Bryant:EF2009,Bryant:CS2010}. A doorway
served as the only vent for the enclosure. It included a jamb of 30~cm extending outward
to facilitate the laser measurements. The entire compartment was elevated 0.3~m off the floor of the laboratory.

The measurements were made using both bi-directional probes and PIV (Particle Image Velocimetry). The PIV measurements only cover the
lower two-thirds of the doorway because of difficulties in seeding the hot outflow gases. The bi-directional probe measurements span the
entire height of the doorway, but Bryant reports that
these measurements were up to 20~\% greater than the PIV measurements in certain regions of
the flow. Consequently, only the PIV data was used for comparison to the
model.


\section{Cable Response to Live Fire -- CAROLFIRE}

CAROLFIRE was a project sponsored by the U.S. Nuclear Regulatory Commission to study the fire-induced thermal
response and functional behavior of electrical cables~\cite{CAROLFIRE}.
The primary objective of CAROLFIRE was to characterize the various modes of electrical
failure ({\em e.g.} hot shorts, shorts to ground) within bundles of power, control and instrument cables.
A secondary objective of the project was to test a simple model to predict \underline{th}ermally-\underline{i}nduced
\underline{e}lectrical \underline{f}ailure (THIEF). The measurements used for these purposes were conducted at Sandia National Laboratories and are described in
Volume II of the CAROLFIRE test report. In brief, there were two series of experiments. The first were conducted within
a heated cylindrical enclosure known as the Penlight apparatus. Single and bundled cables were exposed to various heat fluxes and the
electrical failure modes recorded. The second series of experiments involved cables within trays in a semi-enclosed space under which a gas-fueled burner
created a hot layer to force cable failure.

Petra Andersson and Patrick Van Hees of the Swedish National Testing and Research Institute
(SP) proposed that a cable's thermally-induced electrical failure can be predicted
via a one-dimensional heat transfer calculation, under the assumption that the cable can
be treated as a homogenous cylinder~\cite{Andersson:2005}. Their results for PVC
cables were encouraging and suggested that the simplification of the analysis is reasonable and
that it should extend to other types of cables.
The assumptions underlying the THIEF model are as follows:
\begin{enumerate}
\item The heat penetration into a cable of circular cross section is largely in the radial direction.
This greatly simplifies the analysis, and it is also conservative because it is assumed that
the cable is completely surrounded by the heat source.
\item The cable is homogenous in composition. In reality, a cable is constructed of several
different types of polymeric materials, cellulosic fillers, and a conducting metal, most
often copper.
\item The thermal properties � conductivity, specific heat, and density � of the assumed
homogenous cable are independent of temperature. In reality, both the thermal
conductivity and specific heat of polymers are temperature-dependent, but this
information is very difficult to obtain from manufacturers. More discussion of this
assumption is found below.
\item It is assumed that no decomposition reactions occur within the cable during its heating,
and ignition and burning are not considered in the model. In fact, thermoplastic cables
melt, thermosets form a char layer, and both off-gas volatiles up to and beyond the point
of electrical failure.
\item Electrical failure occurs when the temperature just inside the cable jacket reaches an
experimentally determined value.
\end{enumerate}
Because the CAROLFIRE Penlight experiments tested single cables that were heated uniformly
on all sides, the one-dimensional THIEF model accurately predicted the times for the
temperature inside the cable jacket to reach ``threshold'' values that are typically observed when
the cable fails electrically.


\section{CSTB Tunnel Experiments}

Between 2005 and 2008, the French building research laboratory, Centre Scientifique et Technique du B\^{a}timent (CSTB) cooperated with
the French Tunnel Study Center (CETU), the French National Centre for Scientific Research (CNRS, Institut PPRIME) and the French Directorate for Civil Security (DSC) to conduct fire experiments in a tunnel, some of which
involved a water mist system~\cite{Meyrand}. The first aim was to improve the understanding of the interaction between water mist and a tunnel fire.
The second was to develop a database for model validation. A one-third scale was selected with the objective of studying realistic fire phenomena in an affordable way.
Twenty-eight experiments were conducted
(20 with and 8 without water mist) with varying fuels (heptane pool, wood crib and wood pallet), longitudinal velocities (with and without back layering),
and obstructions near the fire.

The tunnel was 43~m long, with a semi-circular cross section whose area was approximately 4~m$^2$. The walls were covered by a fire resistant mortar cement
with well known thermal properties. The floor was made of concrete. A fan was mounted at the downstream side of the tunnel.
Measurements were made of the following: fuel mass, gas temperature, air velocity, radiative heat flux and gas concentration (CO, CO$_2$ and O$_2$).
Sensors were located at 11 longitudinal positions.

Tests~2 and 27 have been selected because neither exhibited back layering. The longitudinal velocity in Test~2 was approximately 2.2~m/s and in Test 27 it was 3.1~m/s. Both experiments
involved a 0.5~m$^2$ area heptane pool. In Test~2, the HRR was deduced from the fuel mass loss rate only.
In Test~27, the HRR was deduced from both the mass loss rate and from oxygen consumption calorimetry.


In Test 27, a water mist system was manually activated 300~s after ignition. The water mist system was composed of six nozzles along the centerline of the tunnel,
from 4~m upstream to 3.5~m downstream of the fire, 1.5~m apart.
The operating pressure was approximately 90~bar. The water flow rate injected at each nozzle was close to 5.5~L/min, corresponding to a total mist discharge rate of approximately 33~L/min.
Test~27 is interesting because it involved a very low water injection rate.
The main consequence is that the HRR actually increased slightly after the nozzles were activated, and the fire did not extinguish.
The experiment stopped when the fuel was exhausted. This allowed for an assessment of
the model's ability to predict the gas cooling.


\section{FAA Polymers}

As part of their efforts to characterize the burning behavior of commonly used plastics, the U.S.~Federal Aviation Administration (FAA) conducted measurements of the thermal properties of
charring and non-charring polymers with the specific purpose of providing input data for numerical pyrolysis models~\cite{Stoliarov:CF2009,Stoliarov:CF2010}.
The study aimed to determine whether a one-dimensional conduction/reaction model could be used
as a practical tool for prediction and/or extrapolation of the results of fire calorimetry tests. The non-charring polymers included poly(methyl methacrylate) (PMMA),
high-impact polystyrene (HIPS), and high density
polyethylene (HDPE). The charring polymers included polycarbonate (PC) and polyvinyl chloride (PVC).


\section{Fleury Heat Flux Measurements}

Rob Fleury, a masters degree student at the University of Canterbury in Christchurch, New Zealand, measured the heat flux from a variety of propane fires~\cite{Fleury:Masters}.
The objective of the work was to evaluate a variety of empirical heat flux calculation methods. For the measurements, heat flux gauges were mounted on moveable dollies that
were placed in front of, and to the side of, burners with dimensions of 0.3~m by 0.3~m (1:1 burner), 0.6~m by 0.3~m (2:1 burner), and 0.9~m by 0.3~m (3:1 burner). The heat release
rates were set to 100~kW, 150~kW, 200~kW, 250~kW, and 300~kW. The gauges were mounted at heights of 0~m, 0.5~m, 1.0~m, and 1.5~m relative to the top edge of the burner.

\section{FM Parallel Panel Experiments}

Patricia Beaulieu made heat flux measurements within a set of vertical parallel panels as part of a cooperative research
program between Worcester Polytechnic Institute and FM Global (Factory Mutual)~\cite{Beaulieu:FM}. The experimental
apparatus consisted of two vertical parallel
panels, 2.4~m high and 0.6~m wide, with a sand burner at the base. The objective of the project was to measure the flame spread
rate over various composite wall lining materials, but there were also experiments conducted with inert walls for the purpose of measuring the
heat flux from two fuels, propane and propylene, at heat release rates of 30~kW, 60~kW, and 100~kW.



\section{FM/SNL Test Series}


The Factory Mutual and Sandia National Laboratories (FM/SNL) test series consists of 25 compartment fire
experiments conducted in 1985 for the U.S.~Nuclear Regulatory Commission (NRC) by Factory Mutual Research Corporation (FMRC), under
the direction of Sandia National Laboratories (SNL)~\cite{Nowlen:NUREG4681,Nowlen:NUREG4527}. The primary purpose of these experiments was to
provide data with which to validate computer models for various types of compartments typical of nuclear power plants. The
experiments were conducted in an enclosure measuring approximately 18~m long by 12~m wide by 6~m high, constructed at the FMRC fire test facility in Rhode Island.
A drawing is included in Figure~\ref{FM_SNL_Drawing}. All of the experiments included forced ventilation to simulate typical power plant conditions. Six of
the experiments were conducted with a full-scale control room mock-up in place. Parameters varied
during the experiments included fire intensity, enclosure ventilation rate, and fire location.
The current guide uses data from nineteen experiments (Tests 1-17, 21, and 22).
In these tests,  propylene gas burners, heptane pools, and methanol pools were used as fire sources.
Table~\ref{FM_SNL_Matrix} lists the test parameters.

The following information was provided by the test director,
Steve Nowlen of Sandia National Laboratory. In particular, Tests 4, 5, and 21 were given extra attention.
\begin{description}
\item[Heat Release Rate:] The HRR was determined using oxygen consumption calorimetry in the exhaust stack with
a correction applied for the carbon dioxide in the upper layer of the compartment. The
uncertainty of the fuel mass flow was not documented. Several tests selected for this study had
the same target peak heat release rate of 516~kW following a 4~min ``t-squared'' growth
profile. The test report contains time histories of the measured HRR, for which the average,
sustained HRR following the ramp up for Tests 4, 5, and 21 have been estimated as 510~kW, 480~kW, and 470~kW, respectively.
Once reached, the peak HRR was maintained essentially constant
during a steady-burn period of 6~min in Tests~4 and 5, and 16~min in Test~21. Note that in Test 21, Nowlen reports a
``significant'' loss of effluent from the exhaust hood that could lead to an under-estimate of the HRR towards the end of the experiment.
\item[Radiative Fraction:] The radiative fraction was not measured during the experiment, but
in this study it is assumed to equal 0.35, which is typical for a smoky hydrocarbons.
It was further assumed that the radiative fraction was about the same in
Test~21 as the other tests, as fuel burning must have occurred outside of the electrical cabinet in
which the burner was placed.
\item[Measurements:] Four types of measurements were conducted during the FM/SNL test series that are used in the
current model evaluation study, including the HGL temperature and depth, and the ceiling jet and
plume temperatures. Aspirated thermocouples (TCs) were used to make all of the temperature
measurements. Generally, aspirated TC measurements are preferable to bare-bead TC measurements,
as systematic radiative exchange measurement error is reduced.
\item[HGL Depth and Temperature:] Data from all of the vertical TC trees were used when reducing
the HGL height and temperature. For the majority of the tests, Sectors 1, 2, and 3 were used,
all weighted evenly. For Tests 21 and 22, Sectors 1 and 3 were used, evenly weighted. Sector 2 was
partially within the fire plume.
\end{description}



\begin{table}[h!]
\caption{Summary of FM/SNL Experiments.}
\begin{center}
\begin{tabular}{|c|c|c|c|c|c|c|}
\hline
Test    &  Fuel             & Nominal Peak  & Fire          & Ventilation       & Room                  & Used in  \\
No.     &  Type             & HRR (kW)      & Position      & Rate (ach)        & Configuration         & Guide?      \\ \hline \hline
1       & Propylene Burner  &     516       & Center        & 10                & Empty                 &        Yes \\ \hline
2       & Propylene Burner  &     516       & Center        & 10                & Empty                 &        Yes \\ \hline
3       & Propylene Burner  &    2000       & Center        & 10                & Empty                 &        Yes \\ \hline
4       & Propylene Burner  &     516       & Center        & 1                 & Empty                 &        Yes \\ \hline
5       & Propylene Burner  &     516       & Center        & 10                & Empty                 &        Yes \\ \hline
6       & Heptane Pool      &     500       & Wall          & 1                 & Empty                 &        Yes \\ \hline
7       & Propylene Burner  &     516       & Center        & 1                 & Empty                 &        Yes \\ \hline
8       & Propylene Burner  &    1000       & Center        & 1                 & Empty                 &        Yes \\ \hline
9       & Propylene Burner  &    1000       & Center        & 8                 & Empty                 &        Yes \\ \hline
10      & Heptane Pool      &    1000       & Wall          & 4.4               & Empty                 &        Yes \\ \hline
11      & Methanol Pool     &     500       & Wall          & 4.4               & Empty                 &        Yes \\ \hline
12      & Heptane Pool      &    2000       & Wall          & 4.4               & Empty                 &        Yes \\ \hline
13      & Heptane Pool      &    2000       & Wall          & 8                 & Empty                 &        Yes \\ \hline
14      & Methanol Pool     &     500       & Wall          & 1                 & Empty                 &        Yes \\ \hline
15      & Heptane Pool      &    1000       & Wall          & 1                 & Empty                 &        Yes \\ \hline
16      & Heptane Pool      &     500       & Corner        & 1                 & Empty                 &        Yes \\ \hline
17      & Heptane Pool      &     500       & Corner        & 10                & Empty                 &         No \\ \hline
18      &  PMMA Slab        &    1000       & Wall          & 1                 & Empty                 &         No \\ \hline
19      & Heptane Pool      &    1000       & Center        & 1                 & Furnished             &         No \\ \hline
20      & Heptane Pool      &    1000       & Corner        & 8                 & Furnished             &         No \\ \hline
21      & Propylene Burner  &     500       & Cabinet       & 1                 & Furnished             &        Yes \\ \hline
22      & Propylene Burner  &    1000       & Cabinet       & 1                 & Furnished             &        Yes \\ \hline
23      & Qualified Cable   &        N/A    & Cabinet       & 1                 & Furnished             &         No \\ \hline
24      & Unqualified Cable &        N/A    & Cabinet       & 1                 & Furnished             &         No \\ \hline
25      & Unqualified Cable &        N/A    & Cabinet       & 8                 & Furnished             &         No \\ \hline
\end{tabular}
\end{center}
\label{FM_SNL_Matrix}
\end{table}



\begin{sidewaysfigure}[p]
\begin{center}
\includegraphics[height=6.5in]{FIGURES/FM_SNL/FM_SNL_Drawing}
\end{center}
\caption{Geometry of the FM/SNL Experiments.}
\label{FM_SNL_Drawing}
\end{sidewaysfigure}

\clearpage


\section{Hamins Methane Burner Experiments}

Anthony Hamins {\em et al.} performed a series of tests on circular gas
burners measuring the radial and vertical radiative heat flux profiles
outside the flame region. The tests are described
in~\cite{Hostikka:3}. Tests at three burner diameters, 0.10 m, 0.38 m
and 1.0 m are used for validation.


\section{Harrison Spill Plumes}

Roger Harrison, a student at the University of Canterbury, New Zealand, performed a series of one-tenth scale experiments to characterize thermal spill plume
entrainment~\cite{Harrison:2009,Harrison:IAFSS2008,Harrison:FT2007,Harrison:FSJ2010}. The dimensions of the fire compartment were 1~m by 1~m by 0.5~m high.
The height of the compartment opening was equal to the height of the compartment. The width of the opening was varied from 0.2~m to 1~m.
A 0.3~m balcony was attached to the top of the compartment opening. The balcony extended 0.5~m beyond each side of the fire compartment.
The heat release rate of the fire varied from 5~kW to 15~kW.
The plume entrainment rate was measured at different heights by varying the exhaust rate of gases from a hood above the compartment.
Two different test configurations were used to model both detached and adhered spill plumes.


\section{Heskestad Flame Height Correlation}

A widely used experimental correlation for flame height is given by the expression~\cite{Heskestad:FSJ1983,SFPE:Heskestad}:
\be \frac{L_f}{D} = 3.7 \; (Q^*)^{2/5} - 1.02 \ee
where
\be Q^* = \frac{\dQ}{\rho_\infty \, c_p \, T_\infty \, \sqrt{g} \; D^{5/2} }  \ee
is a non-dimensional quantity that relates the fire's heat release rate, $\dQ$, with the diameter of its base, $D$. The greater the value of $Q^*$, the
higher the flame height relative to its base diameter.


\section{LEMTA Spray Test for Radiation Attenuation}

Lechene {\em et al.}~\cite{Lechene} measured the attenuation of thermal radiation passing through a water
spray using a heat flux gauge. The radiation was produced by a 30~cm by 35~cm heat panel whose emission is close
to a black body at 500~$^\circ$C. The horizontal distance
from the radiation panel to the spray nozzle was 1.5~m and to the measurement point 3~m. The heat flux gauge was positioned at the line passing through the center of the panel.
Water mist is produced by seven nozzles arranged in a row, 10~cm apart. They are positioned 1.5~m high. The heat panel is translated vertically during the experiment,
the distance between the panel upper edge and the nozzle row varying between 20~cm and 100~cm.

The attenuation of radiation is defined as previously described for the BRE Spray experiments.
The purpose of the simulations is to compare the measured and simulated attenuation of radiation at different heights.
The water mist nozzle has been characterized by Lechene by measuring the spray
angles and the water flow rate. The droplet size is set by using a PDPA measurement in a single position, 20~cm below the injection point.


\section{LLNL Enclosure Tests}

Sixty-four enclosure fire tests were conducted by Lawrence Livermore National Laboratory (LLNL) in 1986 to study the effects of ventilation on enclosure fires~\cite{Foote:LLNL1986}. The test
enclosure was 6~m long, 4~m wide, and 4.5~m high. It contained a methane rock burner which was placed in the center of the space. For most of the tests the burner was placed on the
floor. The fires varied in size from 50~kW to 400~kW. The burner was 0.57~m in diameter and 0.23~m hight.

The door was closed and sealed for most tests, and air was pulled through the space at rates varying from 100 to 500~g/s. In some tests the enclosure included a plenum space, where
make-up air could be injected from above or below.

\begin{sidewaysfigure}[p]
\begin{center}
\includegraphics[height=6.5in]{FIGURES/LLNL_Enclosure/LLNL_Enclosure_Drawing}
\end{center}
\caption{Geometry of the LLNL Enclosure Experiments.}
\label{LLNL_Enclosure_Drawing}
\end{sidewaysfigure}


\section{McCaffrey Plume Experiments}

In 1979, at the National Bureau of Standards (now NIST), Bernard McCaffrey measured centerline temperature and velocity profiles above a porous, refractory burner.
There were five distinct heat release rates, ranging from 14~kW to 57~kW. The fuel was natural gas. The burner was square, 0.3~m on each side.
The results of the experiments are reported in Reference~\cite{McCaffrey:NBSIR_79-1910}.


\section{NBS Multi-Room Test Series}

The National Bureau of Standards (NBS, which is now called the National Institute of Standards
and Technology, NIST) Multi-Room Test Series consisted of 45 fire tests representing
9 different sets of conditions were conducted in a three-room suite (see Fig.~\ref{NBS_Drawing}). The experiments were
conducted in 1985 and are described in detail in Ref.~\cite{Peacock:NBS_Multi-Room}. The suite consisted of two relatively
small rooms, connected via a relatively long corridor. The fire source, a gas burner, was located
against the rear wall of one of the small compartments.
Fire tests of 100~kW, 300~kW and 500~kW were conducted. For the current study, only three 100~kW fire experiments have been used,
including Test~100A from Set~1, Test~100O from Set~2, and Test~100Z from Set~4. These tests
were selected because they had been used in prior validation studies, and because these tests had the
steadiest values of measured heat release rate during the steady-burn period.

Following is additional information provided by the test director, Richard Peacock of NIST:
\begin{description}
\item[Heat Release Rate:] In the two tests for which
the door was open, the HRR during the steady-burn period measured via oxygen consumption
calorimetry was 110~kW with an uncertainty of about 15~\%, consistent with the replicate
measurements made during the experimental series and the uncertainty typical of oxygen
consumption calorimetry. It was assumed that the closed door test (Test~100O) had the same HRR as the open
door tests.
\item[Radiative Fraction:] Natural gas was used as the fuel in
Test~100A. In Tests~100O and 100Z, acetylene was added to the natural gas to increase the
smoke yield, and as a consequence, the radiative fraction increased. The radiative fraction of
natural gas has been studied previously, whereas the radiative fraction of the acetylene/natural
gas mixture has not been studied. The radiative fraction for the natural gas fire was assigned a
value of 0.20, whereas a value of 0.30 was assigned for the natural gas/acetylene fires.
\item[Measurements:] Only two types of measurements conducted during the NBS test series were used in the
evaluation considered here, because there was less confidence in the other measurements.
The measurements considered here were the HGL temperature and depth, in which bare bead
TCs were used to make these measurements. Single point measurements of temperature within
the burn room were not used in the evaluation of plume or ceiling jet algorithms. This is because
the geometry was not consistent in either case with the assumptions used in the model algorithms
of plumes or jets. Specifically, the burner was mounted against a wall, and the room width-to-height
ratio was less than that assumed by the various ceiling jet correlations.
\end{description}


\begin{sidewaysfigure}[p]
\begin{center}
\includegraphics[height=6.5in]{FIGURES/NBS/NBS_Drawing}
\end{center}
\caption{Geometry of the NBS Multi-Room Experiments.}
\label{NBS_Drawing}
\end{sidewaysfigure}





\clearpage

\section{NIST/NRC Test Series}

These experiments, sponsored by the US NRC and conducted at NIST, consisted of 15 large-scale experiments performed in June 2003. All 15 tests were
included in the validation study. The experiments are documented in Ref.~\cite{Hamins:SP1013-1}. The fire sizes ranged from 350 kW to 2.2 MW in a compartment with dimensions
21.7~m by 7.1~m by 3.8~m high, designed to represent a compartment in a nuclear power plant containing power and control cables.
The walls and ceiling were covered with two layers of marinate boards, each layer 0.0125~m thick. The floor
was covered with one layer of gypsum board on top of a layer of plywood. Thermo-physical and optical properties of the marinate
and other materials used in the compartment are given in Ref.~\cite{Hamins:SP1013-1}. The room had one door and a mechanical air injection and extraction
system. Ventilation conditions, the fire size, and fire location were varied. Numerous measurements (approximately 350 per test) were made including
gas and surface temperatures, heat fluxes and gas velocities.

Following are some notes provided by Anthony Hamins, who conducted the experiments:
\begin{description}
\item[Natural Ventilation:] The compartment had a 2~m by 2~m door in the middle of the west wall. Some of the tests had a closed door and no mechanical
ventilation (Tests 2, 7, 8, 13, and 17), and in those tests the measured compartment leakage was an important consideration. The test report lists leakage
areas based on measurements performed prior to Tests 1, 2, 7, 8, and 13. For the closed door tests, the leakage area used in the simulations was
based on the last available measurement. The chronological order of the tests differed from the numerical order.
For Test 4, the leakage area measured before Test 2 was used. For Tests 10 and 16, the leakage area
measured before Test 7 was used.
\item[Mechanical Ventilation:] The mechanical ventilation and exhaust was used during Tests 4, 5, 10, and 16, providing about 5 air changes per hour. The
door was closed during Test 4 and open during Tests 5, 10, and 16. The supply duct was positioned on the south wall, about 2~m off the floor. An
exhaust duct of equal area to the supply duct was positioned on the opposite wall at a comparable location. The flow rates through the supply and
exhaust ducts were measured in detail during breaks in the testing, in the absence of a fire. During the tests, the flows were monitored with single
bi-directional probes during the tests themselves.
\item[Heat Release Rate:] A single nozzle was used to spray liquid hydrocarbon fuels onto a 1~m by 2~m fire pan that was about 0.1~m deep. The test plan
originally called for the use of two nozzles to provide the fuel spray. Experimental observation suggested that the fire was less unsteady with the
use of a single nozzle. In addition, it was observed that the actual extent of the liquid pool was well-approximated by a 1~m circle in the
center of the pan. For safety reasons, the fuel flow was terminated when the lower-layer oxygen concentration
dropped to approximately 15~\% by volume.
The fuel used in 14 of the tests was heptane, while toluene was used for one test. The HRR was
determined using oxygen consumption calorimetry. The recommended uncertainty values
were 17~\% for all of the tests.
\item[Radiative Fraction:]  The value of the radiative fraction and its
uncertainty were reported as 0.44 and 0.40 for heptane and toluene, respectively.
\end{description}

A diagram of the test structure is displayed in Figure~\ref{NIST_NRC_Drawing}.

\begin{sidewaysfigure}[p]
\begin{center}
\includegraphics[height=6.5in]{FIGURES/NIST_NRC/NIST_NRC_Drawing}
\end{center}
\caption{Geometry of the NIST/NRC Experiments.}
\label{NIST_NRC_Drawing}
\end{sidewaysfigure}


\clearpage


\section{NIST Reduced Scale Enclosure Experiments}

The CO production test series used the NIST Reduced Scale Enclosure (RSE)~\cite{Bryner:1}.  The RSE is a
40~\% scaled version of the ISO 9705 compartment.
It measures 0.98 m wide by 1.46 m deep by 0.98 m tall.  The compartment contains a door centered on the small
face that measures 0.48 m wide by 0.81 m tall.  A 15 cm diameter natural gas burner was positioned in the
center of the compartment.  The burner was on a stand so that its top was 15~cm above the floor.
Species measurements were made inside the upper layer of the compartment at the front near the door
and near the rear of the compartment.


\section{NRCC Facade Heat Flux Measurements}

A series of experiments was conducted by the Fire Research Section of the Institute for Research in Construction, National Research Council of Canada (NRCC),
to measure the heat flux to a mock exterior building facade due to a fire within a compartment~\cite{Oleszkiewicz:ASME,Oleszkiewicz:FireTech}. The experiments selected for model
validation were conducted using a series of propane line burners within a compartment whose interior dimensions were 5.95~m wide, 4.4~m deep, and 2.75~m high (see
Fig.~\ref{NRCC_Facade_Drawing}). There
were five different door/window sizes:
\begin{enumerate}
\item 0.94 m by 2.00 m high
\item 0.94 m by 2.70 m high (door)
\item 2.60 m by 1.37 m high (shown in Fig.~\ref{NRCC_Facade_Drawing})
\item 2.60 m by 2.00 m high
\item 2.60 m by 2.70 m high (door)
\end{enumerate}
There were four fire sizes: 5.5~MW, 6.9~MW, 8.6~MW, and 10.3~MW. In all, 19 experiments were conducted, with the exception of the 10.3~MW fire with Window~1. In each
experiment, heat flux measurements were made 0.5~m, 1.5~m, 2.5~m, and 3.5~m above the top of the door/window.

\begin{sidewaysfigure}[p]
\begin{center}
\includegraphics[height=6.5in]{FIGURES/NRCC_Facade/NRCC_Facade-1}
\end{center}
\caption{Geometry of the NRCC Facade Experiments}
\label{NRCC_Facade_Drawing}
\end{sidewaysfigure}


\clearpage

\section{NRL Confined Space Experiments}

The U.S. Naval Research Laboratory (NRL) performed a multi-year series of experiments inside of a four level,
23 compartment test facility with 20 doors and ceiling vents whose exterior boundaries were airtight~\cite{Confined_JFPE} ~\cite{Confined_DTIC}.
Three HVAC systems were installed in the facility: a supply air system that takes suction from a fan room and discharges the air to the each of the compartments,
an exhaust system that takes suction from each of the compartments and discharges it to the fan room, and a smoke control control system that takes suction from an upper
level compartment and discharges it to the ambient (see Figure~\ref{confined_HVAC}).
A second set of HVAC ducts directly connected the second level with the fourth level (see Figure~\ref{confined_bypass}).

\begin{figure}[ht]
\begin{center}
%\includegraphics[width=5.in]{FIGURES/NRL_Confined_Space/confined_space_hvac_layout.pdf}
\end{center}
\caption{Confined space HVAC system layouts}
\label{confined_HVAC}
\end{figure}

\begin{figure}[ht]
\begin{center}
%\includegraphics[width=5.in]{FIGURES/NRL_Confined_Space/confined_space_bypass.pdf}
\end{center}
\caption{Confined space bypass ducts}
\label{confined_bypass}
\end{figure}

The test facility was instrumented with gas thermocouples, surface thermocouples, optical density meters, gas sampling lines (CO$_2$, CO, and O$_2$),
and velocity probes in a small number of doors, vents, and HVAC ducts.  Test variables included the number of opened doors and hatches, the number of vent openings to the ambient,
the fire location, the fire size, and the operation of the HVAC systems.  All the fires were marine diesel pool fires.

\clearpage

\section{NRL/HAI Wall Heat Flux Measurements}

Back, Beyler, DiNenno and Tatem~\cite{Back:IAFSS4} measured the heat flux from 9 different sized propane fires set up against a wall composed
of gypsum board. The experiments were sponsored by the Naval Research Laboratory and conducted by Hughes Associates, Inc., of Baltimore, Maryland. The
square sand burner ranged in size from 0.28~m to 0.70~m, and the fires ranged in size from 50~kW to 520~kW.


\section{Restivo Compartment Air Flow Experiment}

Velocity measurements for forced airflow within a 9~m by 3~m by 3~m high compartment were made by Restivo~\cite{Restivo:1979}. These measurements
have been widely used to validate CFD models designed for indoor air quality applications. It was also used to assess early versions of
FDS~\cite{Emmerich:1,Emmerich:2,Musser:1}. In the experiment, air was forced into the compartment through a 16.8~cm vertical slot along the ceiling
running the width of the compartment with a velocity of 0.455~m/s. A passive exhaust was located near the floor on the opposite wall, with
conditions specified such that there was no buildup of pressure in the enclosure. The component
of velocity in the lengthwise direction was measured in four arrays: two vertical arrays located 3~m and 6~m  from the inlet along the
centerline of the room, and two horizontal arrays located 8.4~cm above the floor and below the ceiling, respectively.
These measurements were taken using hot-wire anemometers. While data on the specific
instrumentation used are not readily available, hot-wire systems tend to have limitations at low velocities,
with typical thresholds of approximately 0.1~m/s.


\section{Sandia Plume Experiments}

The Fire Laboratory for Accreditation of Models by Experimentation (FLAME) facility \cite{OHern:2005,Blanchat:2001} at Sandia National Laboratories in Albuquerque, New Mexico, is designed specifically for validating models of buoyant fire plumes.  The plume source is 1 m in diameter surrounded by a 0.5 m steel `ground plane'. PIV/PLIF techniques are used to obtain instantaneous joint scalar and velocity fields.  O'Hern {\em et al.} \cite{OHern:2005} studied a turbulent buoyant helium plume in the FLAME facility. Earlier work to model this experiment has been performed by DesJardin {\em et al.} \cite{DesJardin:2004}. Tieszen {\em et al.} \cite{Tieszen:2004,Tieszen:2002} studied methane and hydrogen pool fires.


\section{Smyth Slot Burner Experiment}

Kermit Smyth {\em et al.} conducted diffusion flame experiments at NIST using a methane/air Wolfhard-Parker slot burner.  The experiments are described in
detail in Refs.~\cite{Norton:1,Smyth:1}.  The Wolfhard-Parker slot burner consists of an 8~mm wide
central slot flowing fuel surrounded by two 16~mm wide slots flowing dry air with 1~mm separations between the slots.
The slots are 41~mm in length.  Measurements were made of all major species and a number of minor species along with temperature
and velocity.  Experimental uncertainties have been reported as 5~\% for temperature  and 10~\% to 20~\%
for the major species.


\section{SP Adiabatic Surface Temperature Experiments}

In 2008, three compartment experiments were performed at SP Technical Research Institute of Sweden under the sponsorship of Brandforsk, the Swedish Fire Research Board~\cite{Wickstrom_AST}. The
objective of the experiments was to demonstrate how plate thermometer measurements in the vicinity of a simple steel beam can be used to supply the boundary conditions
for a multi-dimensional heat conduction calculation for the beam. The adiabatic surface temperature was derived from the plate temperatures and used by TASEF, a finite-element
thermal-structural program.

The experiments were performed inside a standard compartment designed for corner fire testing (ISO 9705).
The compartment is 3.6~m deep, 2.4~m wide and 2.4~m high and includes a door opening 0.8~m by 2.0~m. The room was constructed of 20~cm thick light weight concrete
blocks with a density of 600~kg/m$^3$ $\pm 100$~kg/m$^3$.
The heat source was a gas burner run at a constant power of 450~kW. The top of the burner, with a square opening 30~cm by 30~cm, was placed 65~cm above the floor, 2.5~cm from the walls.
A single steel beam was suspended 20~cm below the ceiling
along the centerline of the compartment. There were three measurement stations along the beam at lengths of 0.9~m (Position A), 1.8~m (Position B), and
2.7~m (Position C) from the far wall where the fire was either positioned in the corner (Tests 1 and 2), or the center (Test 3). The beam in Test 1 was
a rectangular steel tube filled with an insulation material. The beam in Tests 2 and 3 was an I-beam.

A second series of experiments involving plate thermometers was carried out in 2011~\cite{Sjostrom:AST}.
A 6~m long, 20~cm diameter vertical steel column was positioned in the center of 1.1~m and 1.9~m diesel fuel and 1.1~m heptane
pool fires. Gas, plate thermometer, and surface temperatures were measured at heights of 1~m, 2~m, 3~m, 4~m, and 5~m above the pool surface. These experiments are notable because
the column is partially engulfed in flames.




\section{Steckler Compartment Experiments}

Steckler, Quintiere and Rinkinen performed a set of 55 compartment fire tests at NBS in 1979. The compartment was 2.8~m by 2.8~m by 2.13~m high\footnote{The test report
gives the height of the compartment as 2.18~m. This is a misprint. The compartment was 2.13~m high.}, with a single door of
various widths, or alternatively a single window with various heights. A 30~cm diameter methane burner was used to generate fires with heat release rates of
31.6~kW, 62.9~kW, 105.3~kW and 158~kW. Vertical profiles of velocity and temperature were measured in the doorway, along with a single vertical profile of temperature
within the compartment.
A full description and results are reported in Reference~\cite{Steckler:NBSIR_82-2520}. The basic test matrix is listed in Table~\ref{Steckler_Table}. Note that the
test report does not include a detailed description of the compartment. However, an internal report\footnote{ {\em Technical Research Report, Fire Induced Flows
Through Room Openings - Flow Coefficients}, Project 203005-003, Armstrong Cork Company, Lancaster, Pennsylvania, May, 1981.} by the test sponsor, Armstrong Cork Company,
reports that the compartment floor was composed of 19~mm calcium silicate board on top of 12.7~mm plywood on wood joists. The walls and ceiling consisted of
12.7~mm ceramic fiber insulation board over 0.66~mm aluminum sheet attached to wood studs.

\begin{table}[h!]
\caption{Summary of Steckler compartment experiments.}
\begin{center}
\begin{tabular}{|c|c|c|c|c||c|c|c|c|c|}
\hline
        & Opening   & Opening       &  HRR       & Burner       &       & Opening   & Opening     &  HRR         & Burner        \\
Test    & Width     & Height        & $\dot{Q}$  & Location     & Test  & Width     & Height      & $\dot{Q}$    & Location      \\
        & (m)       & (m)           & (kW)       &              &       & (m)       &  (m)        & (kW)         &                \\ \hline \hline
10      & 0.24      & 1.83          &  62.9      & Center       & 224   & 0.74      & 0.92        &  62.9         & Back Corner         \\ \hline
11      & 0.36      & 1.83          &  62.9      & Center       & 324   & 0.74      & 0.92        &  62.9         & Back Corner         \\ \hline
12      & 0.49      & 1.83          &  62.9      & Center       & 220   & 0.74      & 1.83        &  31.6         & Back Corner         \\ \hline
612     & 0.49      & 1.83          &  62.9      & Center       & 221   & 0.74      & 1.83        &  105.3        & Back Corner         \\ \hline
13      & 0.62      & 1.83          &  62.9      & Center       & 514   & 0.24      & 1.83        &  62.9         & Back Wall           \\ \hline
14      & 0.74      & 1.83          &  62.9      & Center       & 544   & 0.36      & 1.83        &  62.9         & Back Wall           \\ \hline
18      & 0.74      & 1.83          &  62.9      & Center       & 512   & 0.49      & 1.83        &  62.9         & Back Wall           \\ \hline
710     & 0.74      & 1.83          &  62.9      & Center       & 542   & 0.62      & 1.83        &  62.9         & Back Wall           \\ \hline
810     & 0.74      & 1.83          &  62.9      & Center       & 610   & 0.74      & 1.83        &  62.9         & Back Wall           \\ \hline
16      & 0.86      & 1.83          &  62.9      & Center       & 510   & 0.74      & 1.83        &  62.9         & Back Wall           \\ \hline
17      & 0.99      & 1.83          &  62.9      & Center       & 540   & 0.86      & 1.83        &  62.9         & Back Wall           \\ \hline
22      & 0.74      & 1.38          &  62.9      & Center       & 517   & 0.99      & 1.83        &  62.9         & Back Wall           \\ \hline
23      & 0.74      & 0.92          &  62.9      & Center       & 622   & 0.74      & 1.38        &  62.9         & Back Wall           \\ \hline
30      & 0.74      & 0.92          &  62.9      & Center       & 522   & 0.74      & 1.38        &  62.9         & Back Wall           \\ \hline
41      & 0.74      & 0.46          &  62.9      & Center       & 524   & 0.74      & 0.92        &  62.9         & Back Wall           \\ \hline
19      & 0.74      & 1.83          &  31.6      & Center       & 541   & 0.74      & 0.46        &  62.9         & Back Wall           \\ \hline
20      & 0.74      & 1.83          &  105.3     & Center       & 520   & 0.74      & 1.83        &  31.6         & Back Wall           \\ \hline
21      & 0.74      & 1.83          &  158.0     & Center       & 521   & 0.74      & 1.83        &  105.3        & Back Wall           \\ \hline
114     & 0.24      & 1.83          &  62.9      & Back Corner  & 513   & 0.74      & 1.83        &  158.0        & Back Wall           \\ \hline
144     & 0.36      & 1.83          &  62.9      & Back Corner  & 160   & 0.74      & 1.83        &  62.9         & Center$^*$          \\ \hline
212     & 0.49      & 1.83          &  62.9      & Back Corner  & 163   & 0.74      & 1.83        &  62.9         & Back Corner$^*$     \\ \hline
242     & 0.62      & 1.83          &  62.9      & Back Corner  & 164   & 0.74      & 1.83        &  62.9         & Back Wall$^*$       \\ \hline
410     & 0.74      & 1.83          &  62.9      & Back Corner  & 165   & 0.74      & 1.83        &  62.9         & Left Wall$^*$       \\ \hline
210     & 0.74      & 1.83          &  62.9      & Back Corner  & 162   & 0.74      & 1.83        &  62.9         & Right Wall$^*$      \\ \hline
310     & 0.74      & 1.83          &  62.9      & Back Corner  & 167   & 0.74      & 1.83        &  62.9         & Front Center$^*$    \\ \hline
240     & 0.86      & 1.83          &  62.9      & Back Corner  & 161   & 0.74      & 1.83        &  62.9         & Doorway$^*$         \\ \hline
116     & 0.99      & 1.83          &  62.9      & Back Corner  & 166   & 0.74      & 1.83        &  62.9         & Front Corner$^*$    \\ \hline
122     & 0.74      & 1.38          &  62.9      & Back Corner  &  \multicolumn{5}{r|}{$^*$ Raised burner}                   \\ \hline
\end{tabular}
\end{center}
\label{Steckler_Table}
\end{table}


\clearpage





\section{UL/NFPRF Sprinkler, Vent, and Draft Curtain Study}
\label{UL_NFPRF_Description}

In 1997, a series of 34 heptane spray burner experiments was conducted at the Large Scale Fire Test Facility at Underwriters Laboratories
(UL) in Northbrook, Illinois~\cite{Sheppard:1}. The experiments were divided into two test series. Series I consisted of 22 4.4~MW experiments. Series~II consisted
of 12 10~MW experiments. The objective of the experiments was to characterize the temperature and flow field for fire
scenarios with a controlled heat release rate in the presence of sprinklers, draft curtains, and smoke \& heat vents.
The Large Scale Fire Test Facility at UL contains a 37~m by 37~m (120~ft by 120~ft) main fire test cell, equipped with a 30.5~m by 30.5~m (100~ft by
100~ft) adjustable height ceiling. The layout of the experiments is shown in Figs.~\ref{layout} and \ref{burnerlayoutA}.

\begin{figure}[p]
\begin{center}
\setlength{\unitlength}{.05416667in}
\begin{picture}(120,120)

\linethickness{1.mm} \put(0,0){\framebox(120,120)[tc]{North Wall}} \linethickness{.5mm} \put(10,10){\framebox(100,100)[tc]{Adjustable Height
Ceiling}}

\thinlines \put(117,67){\vector(0,-1){67}} \put(117,73){\vector(0, 1){47}} \put(117,70){\makebox(0,0){$120'$}} \put(113,57){\vector(0,-1){47}}
\put(111,110){\line(1,0){4.}} \put(111, 10){\line(1,0){4.}} \put(113,63){\vector(0, 1){47}} \put(113,60){\makebox(0,0){$100'$}}
\put(30.9,12.83){\dashbox{1}(67.1,71.17)[tc]{Draft Curtains}} \put(27.9,40){\vector(0,-1){27.17}} \put(27.9,46){\vector(0, 1){38.0}}
\put(25.9,84.){\line(1,0){4.}} \put(25.9,12.83){\line(1,0){4.}} \put(27.9,43){\makebox(0,0){$71'2''$}} \put(64.0,87.){\vector(-1,0){33.1}}
\put(72.0,87.){\vector( 1,0){26.0}} \put(30.9,85.){\line(0,1){4.}} \put(98.0,85.){\line(0,1){4.}} \put(68.0,87.){\makebox(0,0){$67'1''$}}

\put(16.0,87.){\vector(-1,0){6.}} \put(24.0,87.){\vector( 1,0){6.92}} \put(20.0,87.){\makebox(0,0){$20'11''$}} \put(101.,87.){\vector(-1,0){3.}}
\put(107.,87.){\vector( 1,0){3.}} \put(104.,87.){\makebox(0,0){$12'$}}

\put(27.9,100){\vector(0,1){10.}} \put(27.9,94){\vector(0,-1){10.}} \put(27.9,97){\makebox(0,0){$26'$}}

\put(27.9,8){\vector(0,1){2.}} \put(27.9,8){\line(1,0){3.}} \put(30.9,8){\makebox(0,0)[l]{$2'10''$}}

\put(55.08,14.83){\line(-1,0){2.}} \put(54.08,16.83){\vector(0,-1){2.}} \put(54.08,7.83){\vector(0,1){5.}} \put(54.08,7.83){\line(1,0){3.}}
\put(57.08,7.83){\makebox(0,0)[l]{$2'$}}

\put(85.08,24.83){\line(0,-1){2.}} \put(95.08,24.83){\line(0,-1){2.}} \put(93.08,23.83){\vector(1,0){2.}} \put(87.08,23.83){\vector(-1,0){2.}}
\put(103.00,23.83){\vector(-1,0){5.}} \put(103.00,23.83){\line(0,-1){3.}} \put(103.00,20.83){\makebox(0,0)[ct]{$2'11''$}}
\put(90.08,23.83){\makebox(0,0)[c]{$10'$}}

\thicklines \put(78.08,55.83){\framebox(4,8){ }}

\put(78.58,58.33){\framebox(3,3)[c]{A}} \put(78.58,68.33){\framebox(3,3)[c]{B}} \put(88.58,58.33){\framebox(3,3)[c]{C}}
\put(58.58,38.33){\framebox(3,3)[c]{D}}

\thinlines

\multiput(35.08,14.83)(0,10){7}{\circle*{.8}} \multiput(45.08,14.83)(0,10){7}{\circle*{.8}} \multiput(55.08,14.83)(0,10){7}{\circle*{.8}}
\multiput(65.08,14.83)(0,10){7}{\circle*{.8}} \multiput(75.08,14.83)(0,10){7}{\circle*{.8}} \multiput(85.08,14.83)(0,10){7}{\circle*{.8}}
\multiput(95.08,14.83)(0,10){7}{\circle*{.8}} \tiny \put(35.48,15.23){98} \put(45.48,15.23){91} \put(55.48,15.23){84} \put(65.48,15.23){81}
\put(75.48,15.23){78} \put(85.48,15.23){75} \put(95.48,15.23){72} \put(35.48,25.23){99} \put(45.48,25.23){92} \put(55.48,25.23){85}
\put(65.48,25.23){82} \put(75.48,25.23){79} \put(85.48,25.23){76} \put(95.48,25.23){73} \put(35.48,35.23){100} \put(45.48,35.23){93}
\put(55.48,35.23){86} \put(65.48,35.23){83} \put(75.48,35.23){80} \put(85.48,35.23){77} \put(95.48,35.23){74} \put(35.48,45.23){101}
\put(45.48,45.23){94} \put(55.48,45.23){87} \put(65.48,45.23){62} \put(75.48,45.23){58} \put(85.48,45.23){54} \put(95.48,45.23){50}
\put(35.48,55.23){102} \put(45.48,55.23){95} \put(55.48,55.23){88} \put(65.48,55.23){63} \put(75.48,55.23){59} \put(85.48,55.23){55}
\put(95.48,55.23){51} \put(35.48,65.23){103} \put(45.48,65.23){96} \put(55.48,65.23){89} \put(65.48,65.23){64} \put(75.48,65.23){60}
\put(85.48,65.23){56} \put(95.48,65.23){52} \put(35.48,75.23){104} \put(45.48,75.23){97} \put(55.48,75.23){90} \put(65.48,75.23){65}
\put(75.48,75.23){61} \put(85.48,75.23){57} \put(95.48,75.23){53} \put(70.08,49.83){\makebox(0,0)[c]{68}} \put(70.08,59.83){\makebox(0,0)[c]{69}}
\put(70.08,69.83){\makebox(0,0)[c]{70}} \put(80.08,49.83){\makebox(0,0)[c]{67}} \put(90.08,49.83){\makebox(0,0)[c]{66}}
\put(90.08,69.83){\makebox(0,0)[c]{71}}

\multiput(80.08,56.83)(0,1){7}{\circle*{.2}} \put(80.58,62.83){\line(1,0){22.5}} \put(104.,62.83){\makebox(0,0)[l]{43}}
\put(104.,61.33){\makebox(0,0)[l]{44}} \put(104.,59.83){\makebox(0,0)[l]{45}} \put(104.,58.33){\makebox(0,0)[l]{46}}
\put(104.,56.83){\makebox(0,0)[l]{47}} \put(104.,55.33){\makebox(0,0)[l]{48}} \put(104.,53.83){\makebox(0,0)[l]{49}}

\normalsize

\end{picture}
\end{center}
\caption[Plan view of the UL/NFPRF Experiments, Series~I.] {Plan view of the UL/NFPRF Experiments, Series~I. The sprinklers are indicated by the solid circles and
are spaced 3~m apart. The number beside each sprinkler location indicates the channel number of the nearest thermocouple. The vent dimensions
are 4~ft by 8~ft. The boxed letters A, B, C and D indicate burner positions. Corresponding to each burner position is a vertical array of
thermocouples. Thermocouples 1--9 hang 7, 22, 36, 50, 64, 78, 92, 106 and 120~in from the ceiling, respectively, above Position A. Thermocouples 10
and 11 are positioned above and below the ceiling tile directly above Position B, followed by 12--20 that hang at the same levels below the ceiling
as 1--9. The same pattern is followed at Positions C and D, with thermocouples 21--31 at C and 32--42 at D.}
\label{layout}
\end{figure}


\begin{figure}[p]
\begin{center}
\setlength{\unitlength}{.054166in}
\begin{picture}(120,120)

\linethickness{1mm}
\put(0,0){\framebox(120,120)[tc]{ }}
\put(60,118){\makebox(0,0){North Wall}}
\put(60,  2){\makebox(0,0){South Wall}}

\linethickness{.5mm}
\put(10,10){\framebox(100,100)[tl]{ }}

\thinlines
\put(117,67){\vector(0,-1){67}}
\put(117,73){\vector(0, 1){47}}
\put(117,70){\makebox(0,0){$120'$}}
\put(113,57){\vector(0,-1){47}}
\put(113,63){\vector(0, 1){47}}
\put(113,60){\makebox(0,0){$100'$}}
\put(10.0,86.0){\dashbox{1}(30.0,24.0)[tl]{ }}
\put(40.,10.5){\dashbox{1}(69.5,75.5)[tl]{              }}

\thicklines
\put(48.,16.){\framebox(4,8){ }}
\put(48.,67.){\framebox(4,8){ }}
\put(28.,67.){\framebox(4,8){ }}
\put(98.,67.){\framebox(4,8){ }}
\put(98.,16.){\framebox(4,8){ }}

\large
\put(68.5,49.5){\dashbox{.5}(3,3)[c]{D}}
\put(48.5,69.5){\dashbox{.5}(3,3)[c]{A}}
\put(48.5,79.5){\dashbox{.5}(3,3)[c]{B}}
\put(58.5,59.5){\dashbox{.5}(3,3)[c]{C}}
\put(38.5,54.5){\dashbox{.5}(3,3)[c]{E}}
\put(38.5,84.5){\dashbox{.5}(3,3)[c]{F}}
\normalsize

\multiput(15,11)(0,10){10}{\circle*{.8}}
\multiput(25,11)(0,10){10}{\circle*{.8}}
\multiput(35,11)(0,10){10}{\circle*{.8}}
\multiput(45,11)(0,10){10}{\circle*{.8}}
\multiput(55,11)(0,10){10}{\circle*{.8}}
\multiput(65,11)(0,10){10}{\circle*{.8}}
\multiput(75,11)(0,10){10}{\circle*{.8}}
\multiput(85,11)(0,10){10}{\circle*{.8}}
\multiput(95,11)(0,10){10}{\circle*{.8}}
\multiput(105,11)(0,10){10}{\circle*{.8}}

\end{picture}
\end{center}
\caption[Plan view of the UL/NFPRF Experiments, Series~II. ]
{Plan view of the UL/NFPRF Experiments, Series~II. The boxed letters A, B, C, D, E and F indicate burner positions. The sprinklers are indicated
by the solid circles and are spaced 10~ft apart. The branch lines run
north to south. The vents are 4~ft by 8~ft. }
\label{burnerlayoutA}
\end{figure}



\begin{description}
\item[Ceiling:] The ceiling was raised to a height of 7.6~m and instrumented with thermocouples and other measurement devices. The ceiling was constructed of
0.6~m by 1.2~m by 1.6~cm UL fire-rated Armstrong Ceramaguard (Item 602B) ceiling tiles. The manufacturer reported the
thermal properties of the material to be: specific heat 753 J/(kg$\cdot$K), thermal conductivity
0.0611~W/(m$\cdot$K), and density 313~kg/m$^3$.
\item[Draft Curtains:] Sheet metal, 1.2~mm thick and 1.8~m deep, was suspended from the ceiling for 16 of the 22 Series~I tests, enclosing an area of about 450~m$^2$ and 49 sprinklers.
The curtains were in place for all of the Series~II tests.
\item[Sprinklers:] Central ELO-231 (Extra Large Orifice) uprights were used for all the tests. The orifice diameter of this sprinkler is reported by the manufacturer to be
nominally 1.6~cm (0.64~in), the reference actuation temperature is reported by the manufacturer to be 74$^\circ$C (165$^\circ$F). The RTI (Response Time
Index) and C-factor (Conductivity factor) were reported by UL to be 148~(m$\cdot$s)$^\ha$ and 0.7~(m/s)$^\ha$, respectively~\cite{Sheppard:1}.
When installed, the sprinkler deflector was located 8~cm below the ceiling. The thermal
element of the sprinkler was located 11~cm below the ceiling. The sprinklers were installed with nominal 3~m by 3~m (exact 10~ft by 10~ft) spacing in a
system designed to deliver a constant 0.34~L/(s$\cdot$m$^2$) (0.50 gpm/ft$^2$) discharge density when supplied by a 131~kPa (19~psi) discharge
pressure
\item[Vent:] UL-listed double leaf fire vents with steel covers and steel curb were installed in the adjustable height ceiling in the position shown in
Figs.~\ref{layout} and \ref{burnerlayoutA}. The vent is designed to open manually or automatically. The vent doors were recessed into the ceiling about 0.3~m (1~ft).
\item[Heat Release Rate:] The heptane spray burner consisted of a 1~m by 1~m square of 1.3~cm pipe supported by four cement blocks 0.6~m off the floor.
Four atomizing spray nozzles were used to provide a free spray of heptane that was then ignited. For all but one of the Series~I tests, the total heat release
rate from the fire was manually ramped up following a ``t-squared'' curve to a steady-state in 75~s
(150~s was used in Test I-16). The fire was ramped to 10~MW in 75~s for the Series~II tests. The fire growth curve was followed until a specified fire size was
reached or the first sprinkler activated. After either of these events, the fire size was maintained at that level until conditions reached roughly a
steady state, {\em i.e.} the temperatures recorded near the ceilings remained steady and no more sprinkler activations occurred.
The heat release rate from the burner was confirmed by placing it under the large product calorimeter at UL, ramping up the flow of heptane in the
same manner as in the tests, and measuring the total and convective heat release rates. It was found that the convective heat release rate was
0.65$\pm$0.02 of the total.
\item[Instrumentation:] The instrumentation for the tests consisted of thermocouples, gas analysis equipment, and pressure transducers. The locations of the instrumentation
are referenced in the plan view of the facility (Fig.~\ref{layout}).
Temperature measurements were recorded at 104 locations. Type K 0.0625~in diameter Inconel sheathed thermocouples were positioned to measure (i)
temperatures near the ceiling, (ii) temperatures of the ceiling jet, and (iii) temperatures near the vent.
\end{description}


\clearpage


\section{Ulster SBI Corner Heat Flux Measurements}

Zhang {\em et al.}~\cite{Zhang:IAFSS9} measured the heat flux and flame heights from
fires in the single burning item (SBI) enclosure at the University of Ulster, Northern Ireland.
Thin steel plate probes were used to measure the surface heat flux, and flame
heights were determined by analyzing the instantaneous images extracted from the videos of the
experiments by a CCD camera. Three heat release rates were used -- 30~kW, 45~kW, and 60~kW.



\section{USCG/HAI Water Mist Suppression Tests}

The U.S. Coast Guard sponsored a series of experiments to assess the fire suppression capabilities of a variety of water mist systems in a variety of ship board configurations. The
experiments were conducted in 1999 by Hughes Associates, Inc., in a simulated machinery space aboard the test vessel {\em State of Maine} at the USCG Fire and Safety Test Detachment,
Mobile, Alabama~\cite{Back:USCG1999}.
The space had nominal dimensions of 7~m by 5~m by 3~m, containing two steel engine mock-ups each measuring 3~m by 1~m by 1.5~m. The space was equipped with a door
for natural ventilation and a forced ventilation system providing approximately 15 air changes per hour. Five commercially available water mist systems were evaluated. The
obstructed heptane spray fires ranged in size from approximately 250~kW to 1~MW.



\section{USN High Bay Hangar Experiments}

The U.S. Navy sponsored a series of 33 tests within two hangars examining fire detection and sprinkler activation in response to spill fires in large enclosures. Experiments were conducted using JP-5 and JP-8 fuels in two Navy high bay aircraft hangars located in Naval Air Stations in Barber's Point, Hawaii and Keflavik, Iceland.

The Hawaii tests were conducted in a 15~m high hangar measuring 97.8~m in length and 73.8~m in width. Of the 13 tests conducted in the facility 11 were conducted in pans ranging from .09~m$^2$ to 4.9~m$^2$ in area with heat release rates varying from 100~kW to 7.7~MW. The burner was placed in the center of the room on a scale that continuously recorded the pans weight. The facility was equipped with a number of detection devices including thermocouples, electronic smoke and spot heat detectors, projected beam smoke detectors, combination UV/IR optical flame detectors, line-type heat detectors, as well as sprinklers. Measurements were recorded at a large number of locations allowing for a through profile of compartment behavior.

It was suspected that fire plume behavior and response of detection devices in a cold building may not have been well replicated by the experiments held in the warm hangar in Hawaii. The Iceland tests were conducted under a 22~m barrel vaulted ceiling in a hangar measuring 45.7~m by 73.8~m. 22 tests in total were conducted. The majority of these tests fires burned JP-5 fuel with the remainder burning JP-8. The jet fuel fires ranged in size from .06~m$^2$ to 20.9~m$^2$ and in heat release rate from 100~kW to approximately 33~MW. The facility was equipped similarly to the Hawaii hangar.



\section{Vettori Flat Ceiling Experiments}

Vettori~\cite{Vettori:1} analyzed a series of 45 experiments conducted at NIST that were intended to compare the effects of different ceiling configurations on the activation times of quick response residential pendent sprinklers. The two ceiling configurations used consisted of an obstructed ceiling, with parallel beams 0.038~m wide by 0.24~m deep placed 0.41~m on center, and a smooth ceiling configuration, in which the beams were covered by a sheet of gypsum board.  In addition to the two ceiling configurations, there were also three fire growth rates and three burner locations used -- a total of 18 test configurations. The fire growth rate was provided by a computer controlled methane gas burner to mimic a standard t-squared fire growth rate with either a slow, medium, or fast ramp up. The burner was placed in a corner of the room, then against an adjacent wall, and then in a location removed from any wall. Measurements were taken to record sprinkler activation time, temperatures at varying heights and locations within the room, and the ceiling jet velocities at several other locations.  A diagram of the test structure is displayed in Figure~\ref{Vettori_Drawing}.

\begin{sidewaysfigure}[p]
\begin{center}
\includegraphics[height=6.5in]{FIGURES/Vettori_Flat_Ceiling/Vettori_Flat Ceiling.pdf}
\end{center}
\caption{Geometry of the Vettori Flat Ceiling.}
\label{Vettori_Drawing}
\end{sidewaysfigure}


\section{Vettori Sloped Ceiling Experiments}

Vettori~\cite{Vettori:2} performed a series of 72 compartment experiments to assess the effects of ceiling beams and slopes on the activation times of quick-response residential pendent sprinklers. There were 36 experiments (2 replicates of each) combining the following parameters:
\begin{itemize}
\item Flat, 13$^\circ$, or 24$^\circ$ Sloped Ceiling
\item Smooth or Obstructed Ceiling
\item Fast or Slow Growth Fire
\item Corner, Side Wall, or Detached Burner Location
\end{itemize}


\clearpage


\section{VTT Large Hall Tests}

The experiments are described in Ref.~\cite{Hostikka:VTT2104}. The series consisted of 8 experiments, but because of replicates only three unique fire
scenarios. The experiments were undertaken to study the movement of smoke in a large hall with a sloped ceiling. The tests were conducted inside the
VTT Fire Test Hall, with dimensions of 19~m high by 27~m long by 14~m wide. Each test involved a single heptane pool fire, ranging from 2~MW to 4~MW.
Four types of predicted output were used in the present evaluation -- the HGL temperature and depth, average flame height and the plume
temperature. Three vertical arrays of thermocouples (TC), plus two thermocouples in the plume, were compared to FDS predictions. The HGL
temperature and height were reduced from an average of the three TC
arrays using the standard algorithm described in
Chapter~\ref{HGL:Chapter}. The ceiling jet temperature was not
considered, because the ceiling in the test hall is not flat, and the
standard model algorithm is not appropriate for this geometry.

The VTT test report lacks some information needed to model the experiments, which is why some information was based on private communications with the
principal investigator, Simo Hostikka.
\begin{description}
\item[Surface Materials:] The walls and ceiling of the test hall consist of a 1~mm thick layer of sheet metal on top of a 5~cm layer of
mineral wool. The floor was constructed of concrete. The report does not provide thermal properties of these materials.
\item[Natural Ventilation:] In Cases~1 and 2, all doors were closed, and ventilation was restricted to infiltration through the building envelope. Precise
information on air infiltration during these tests is not available. The scientists who conducted the experiments recommend a leakage area of about 2~m$^2$,
distributed uniformly throughout the enclosure. By contrast, in Case~3, the doors located in each end wall (Doors 1 and 2, respectively)
were open to the external ambient environment. These doors are each 0.8~m wide by 4~m high, and are located such that their centers
are 9.3~m from the south wall.
\item[Mechanical Ventilation:] The test hall has a single mechanical exhaust duct, located in the roof space, running along the center of the building. This
duct had a circular section with a diameter of 1~m, and opened horizontally to the hall at a distance of 12~m from the floor and 10.5~m from the west wall.
Mechanical exhaust ventilation was operational for Case~3, with a constant volume flow rate of 11~m$^3$/s drawn through
the exhaust duct.
\item[Heat Release Rate:] Each test used a single liquid fuel pan with its center located 16~m from the west
wall and 7.4~m from the south wall. For all tests, the fuel was heptane in a circular steel pan that was partially filled with water. The
pan had a diameter of 1.17~m for Case~1 and 1.6~m for Cases~2 and 3. In each case, the fuel surface was 1~m above the
floor. The trays were placed on load cells, and the HRR was calculated from the mass loss rate. For the three cases,
the fuel mass loss rate was averaged from individual replicate tests. In the HRR estimation, the heat of combustion (taken as 44,600~kJ/kg) and the
combustion efficiency for n-heptane was used. Hostikka suggests a value of 0.8 for the combustion efficiency.
Tewarson reports a value of 0.93 for a 10~cm pool fire~\cite{SFPE:Tewarson}. For the calculations reported in the current
study, a combustion efficiency of 0.85 is assumed. In general, an uncertainty of 15~\% has been assumed for the reported HRR of most of the large
scale fire experiments used.
\item[Radiative Fraction:] The radiative fraction was assumed to be 0.35, similar to many smoky hydrocarbons.
\end{description}
A diagram of the test structure is displayed in Figure~\ref{VTT_Drawing}.
\begin{sidewaysfigure}[p]
\begin{center}
\includegraphics[height=6.5in]{FIGURES/VTT/VTT_Drawing}
\end{center}
\caption{Geometry of the VTT Large Fire Test Hall.}
\label{VTT_Drawing}
\end{sidewaysfigure}


\clearpage





\section{WTC Spray Burner Test Series}

As part of its investigation of the World Trade Center disaster, the Building and Fire Research Laboratory at NIST conducted several series of fire experiments to both gain insight into the
observed fire behavior and also to validate FDS for use in reconstructing the fires. The first series of experiments involved a relatively simple compartment with a liquid spray burner and
various structural elements with varying amounts of sprayed fire-resistive materials (SFRM). A diagram of the compartment is shown in Fig.~\ref{WTC_Drawing}.
A complete description of the experiments can be found in the NIST WTC report NCSTAR~1-5B~\cite{NIST_NCSTAR_1-5B}.
The overall enclosure was rectangular, as were the vents and most of the obstructions. The compartment walls and ceiling were made of 2.54~cm thick marinite. The manufacturer provided the thermal properties of the material used in the calculation. The density was 737~kg/m$^3$, conductivity 0.12~W/m/K. The specific heat ranged from 1.17~kJ/kg/K at 93~$^\circ$C to
1.42~kJ/kg/K at 425~$^\circ$C. This value was assumed for higher temperatures.
The steel used to construct the column and truss flanges was 0.64~cm thick.  The density of the steel was assumed to be 7,860~kg/m$^3$; its specific heat 0.45~kJ/kg/K.

Two fuels were used in the tests. The properties of the fuels were obtained from measurements made on a series of unconfined burns that are referenced in the test report.
The first fuel was a blend of heptane isomers, C$_7$H$_{16}$. Its soot yield was set at a constant 1.5~\%. The second fuel was a mixture (40~\% - 60~\% by volume) of toluene, C$_7$H$_8$,
and heptane. Because FDS only considers the burning of a single hydrocarbon fuel, the mixture was taken to be C$_7$H$_{12}$ with a soot yield of 11.2~\%.
The radiative fraction for the heptane blend was 0.44; for the heptane/toluene mixture it was 0.39.
The heat release rate of the simulated burner was set to that which was measured in the experiments. The spray burner was modeled using reported properties of the nozzle and
liquid fuel droplets.

\begin{sidewaysfigure}[p]
\begin{center}
\includegraphics[height=6.5in]{FIGURES/WTC/WTC_AutoDesk_Schematic}
\end{center}
\caption{Geometry of the WTC Experiments.}
\label{WTC_Drawing}
\end{sidewaysfigure}

\clearpage







\section{Summary of Experiments}

\label{experiment_summary}

Table~\ref{Test_Parameters} presents a summary of all the experiments described in this chapter in terms of quantities commonly used in fire protection engineering. This ``parameter space''
outlines the range of applicability of the validation studies performed to date. The parameters are explained below:

\begin{description}
\item[Heat Release Rate, $\dQ$,] is the range of peak heat release rates of the fires in the test series.
\item[Fire Diameter, $D$,] is the equivalent diameter of the base of the fire, calculated $D=\sqrt{4A/\pi}$, where $A$ is the area of the base.
\item[Ceiling Height, $H$,] is the distance from floor to ceiling.
\item[Fire Froude Number, $\dQ^*$,] is a useful non-dimensional quantity for plume correlations and flame height estimates.
\be Q^* = \frac{\dot{Q}}{\rho_\infty c_p T_\infty \sqrt{gD} D^2} \ee
It is essentially the ratio of the fuel gas exit velocity and the buoyancy-induced plume velocity. Jet fires are characterized by large Froude numbers. Typical accidental fires
have a Froude number near unity.
\item[Flame Height relative to Ceiling Height, $L_f/H$,] is a convenient way to express the physical size of the fire relative to the size of the room.
The height of the visible flame, based on Heskestad's correlation, is estimated by:
\be L_f = D \, \left( 3.7 \, (\dQ^*)^{2/5} - 1.02 \right) \ee
\item[Global Equivalence Ratio, $\phi$,] is the ratio of the mass flux of fuel to the mass flux of oxygen into the compartment, divided by the stoichiometric ratio.
\be \phi = \frac{\dm_f}{r\, \dm{\hbox{\tiny O$_2$}}} \equiv  \frac{\dQ \, \hbox{(kW)}}{13,100 \, \hbox{(kJ/kg)} \; \dm_{\hbox{\tiny O$_2$}} } \quad ; \quad  \dm_{\hbox{\tiny O$_2$}} = \left\{
   \begin{array}{r@{\quad:\quad}l}
      \ha \, 0.23 \, A_0 \sqrt{H_0} & \hbox{Natural Ventilation} \\
      0.23 \, \rho \, \dot{V}       & \hbox{Mechanical Ventilation} \end{array} \right.
\ee
Here, $r$ is the stoichiometric ratio, $A_0$ is the area of the compartment opening, $H_0$ is the height of the opening, $\rho$ is the density of air, and $\dot{V}$ is the
volume flow of air into the compartment. If $\phi<1$, the compartment is considered ``well-ventilated'' and if $\phi>1$, the compartment is considered ``under-ventilated.''
\item[Compartment Aspect Ratios, $W/H$ and $L/H$,] indicate if the compartment is shaped like a hallway, typical room, or vertical shaft.
\item[Relative Distance along the Ceiling, $r_{cj}/H$,] indicates the distance from the fire plume of a sprinkler, smoke detector, {\em etc.}, relative to the
compartment height, $H$.
\item[Relative Distance from the Fire, $r_{rad}/D$,] indicates whether a ``target'' is near or far from the fire.
\end{description}

\newpage \thispagestyle{empty}
\begin{sidewaystable}[p]
\caption{Summary of important experimental parameters. }
\begin{center}
\begin{tabular}{|l|c|c|c|c|c|c|c|c|c|c|c|c|}
\hline
                    & $\dot{Q}$     & $D$           & $H$   &                   &               &               &           &           &                   &                   \\
\rb{Test Series}    & (kW)          & (m)           & (m)   & \rb{$Q^*$}        & \rb{$L_f/H$}  & \rb{$\phi$}   & \rb{$W/H$}& \rb{$L/H$}& \rb{$r_{cj}/H$}   & \rb{$r_{rad}/D$}  \\ \hline \hline
Arup Tunnel         & 5344          & 1.6           & 7     & 1.4               & 0.8           & 0.03          & 1.1       & 43        & 0 -- 1.1          & N/A               \\ \hline
ATF Corridors       & 50 -- 500     & 0.5           & 2.4   & 0.3 -- 3.1        & 0.3 -- 1      & 0.01 -- 0.07  & 0.8       & 7.1       & 0.8 -- 6          & N/A               \\ \hline
Bryant Doorway      & 34 -- 511     & 0.3           & 2.4   & 0.4 -- 6.4        & 0.3 -- 1      & 0.01 -- 0.16  & 1.0       & 2.1       & 0.6 -- 0.8        & N/A               \\ \hline
CSTB Tunnel         & 1965 -- 2484  & 0.8           & 1.9   & 2.9 -- 3.7        & 2.1 -- 2.4    & 0.04 -- 0.05  & 1.3       & 28        & 1.6 -- 12.6       & N/A               \\ \hline
Fleury Heat Flux    & 100 -- 300    & 0.3 -- 0.6    & Open  &  0.3 -- 4.3       & Open          & Open          & Open      & Open      & Open              & 0.8 -- 6.7        \\ \hline
FM/SNL              & 470 -- 516    & 0.9           & 6.1   & 0.5 -- 0.6        & 0.3           & 0.2           & 2.0       & 3.0       & 0.2 -- 0.3        & N/A               \\ \hline
Hamins Burner       & 0.4 -- 162    & 0.1 -- 1.0    & Open  & 0.1               & Open          & Open          & Open      & Open      & N/A               & 0.1 -- 12         \\ \hline
Heskestad           & $10^2-10^7$   & 1.1           & Open  & $10^{-1}-10^4$    & Open          & Open          & Open      & Open      & N/A               & N/A               \\ \hline
LLNL Enclosure      & 50 -- 400     & 0.6           & 4.5   & 0.2 -- 1.4        & 0.1 -- 0.4    & 0.03 -- 0.22  & 0.9       & 1.3       & 0.3 -- 1.0        & N/A               \\ \hline
McCaffrey Plume     & 14 -- 57      & 0.3           & Open  & 0.2 -- 0.7        & Open          & Open          & Open      & Open      & N/A               & N/A               \\ \hline
NBS Multi-Room      & 110           & 0.3           & 2.4   & 1.4               & 0.4 -- 5.2    & 0.12          & 1.0       & 5.1       & 0.5 -- 0.7        & 0.9 -- 2.4        \\ \hline
NIST RSE            & 50 -- 600     & 0.15          & 1.0   & 4.9 -- 58.4       & 1 -- 2.9      & 0.1 -- 1.15   & 1.0       & 1.5       & N/A               & N/A               \\ \hline
NIST/NRC            & 350 -- 2200   & 1.0           & 3.8   & 0.3 -- 1.9        & 0.4 -- 1.1    & 0.04 -- 0.7   & 1.9       & 5.7       & 0.3 -- 2.1        & 2 -- 4            \\ \hline
NRCC Facade         & 5000 -- 10300 & 4.3           & 2.8   & 0.1 -- 0.2        & 1 -- 1.8      & 2.5 -- 5.2    & 1.6       & 2.2       & N/A               & 0                 \\ \hline
NRL/HAI             & 50 -- 520     & 0.3 -- 0.7    & Open  & 1 -- 1.1          & Open          & Open          & Open      & Open      & N/A               & 0.3 -- 8          \\ \hline
Sandia Plume        & 2025 -- 5450  & 1.0           & Open  & 1.7 -- 4.6        & Open          & Open          & Open      & Open      & N/A               & N/A               \\ \hline
SP AST              & 450           & 0.3           & 2.4   & 5.7               & 1             & 0.1           & 1.0       & 1.5       & N/A               & N/A               \\ \hline
Steckler            & 31.6 -- 158   & 0.3           & 2.1   & 0.7 -- 3.5        & 0.3 -- 0.7    & 0.01 -- 0.6   & 1.3       & 1.3       & N/A               & N/A               \\ \hline
UL/NFPRF            & 4400 -- 10000 & 1.0           & 7.6   & 3.7 -- 8.5        & 0.8 -- 1.1    & 0             & 4.9       & 4.9       & 0.1 -- 2.0        & N/A               \\ \hline
Ulster SBI          & 30 -- 60      & 0.2           & Open  & 1.4 -- 2.8        & Open          & Open          & Open      & Open      & N/A               & 1 -- 7.5          \\ \hline
USCG/HAI            & 250 -- 1000   & 0.3           & 3.0   & 5.6 -- 22.4       & 0.6 -- 1.1    & 0.3 -- 1.0    & 1.7       & 2.3       & 0 -- 0.8          & 6 -- 15           \\ \hline
USN Hawaii          & 100 -- 7700   & 0.3 -- 2.5    & 15    & 1.3 -- 0.7        & 0.1 -- 0.4    & 0             & 4.9       & 6.5       & 0 -- 1.2          & N/A               \\ \hline
USN Iceland         & 100 -- 15700  & 0.3 -- 3.4    & 22    & 1.3 -- 0.6        & 0.1 -- 0.4    & 0             & 2.1       & 3.4       & 0 -- 1.0          & N/A               \\ \hline
Vettori Flat        & 1055          & 0.7           & 2.6   & 2.3               & 1.2           & Closed        & 2.1       & 3.5       & 0.8 -- 2.9        & N/A               \\ \hline
Vettori Sloped      & 1055          & 0.7           & 2.5   & 2.3               & 1.2           & 0.23          & 2.2       & 2.9       & N/A               & N/A               \\ \hline
VTT Large Hall      & 1860 -- 3640  & 1.4 -- 1.8    & 19    & 0.7               & 0.2           & 0             & 1.0       & 1.4       & 0 -- 0.6          & N/A               \\ \hline
WTC                 & 965 -- 1460   & 1.0           & 3.8   & 0.8 -- 1.2        & 0.7 -- 0.9    & 0.5 -- 0.7    & 0.9       & 1.8       & 0.1               & 0.5 -- 2          \\ \hline
\end{tabular}
\end{center}
\label{Test_Parameters}
\nopagebreak
\end{sidewaystable}


\noindent
Table~\ref{Numerical_Parameters} lists a few important parameters related to the numerical resolution of the calculation.
\begin{description}
\item[Characteristic Fire Diameter, $D^*$,] is a useful length scale that incorporates the heat release rate of the fire.
\be D^* = \left( \frac{\dot{Q}}{\rho_\infty c_p T_\infty \sqrt{g}} \right)^{2/5}  \ee
\item[Plume Resolution Index, $D^*/\dx$,] is the number of grid cells of length $\dx$ that span the characteristic diameter of the fire. The greater its value, the more
``resolved'' are the fire dynamics.
\item[Ceiling Height relative to Fire Diameter, $H/D^*$,] is the non-dimensional height of the smoke plume.
\end{description}
Note that the calculations performed for the various validation studies described in this Guide use a wide range of values of the Plume Resolution Index, $D^*/\dx$.
There are several reasons for this. First,
typical applications of FDS often involve relatively small fires in relatively large spaces, and it is impractical to use a very fine grid that captures the detailed fire dynamics.
Second, for some applications the accuracy of calculation is highly dependent on resolving the plume well, but for others, it is less important. For those citing the validation
studies in this Guide, it is important that both the physical and numerical parameters are comparable to the given application.


\begin{table}[p]
\caption{Summary of important numerical parameters. }
\begin{center}
\begin{tabular}{|l|c|c|c|}
\hline
                    &               &               &               \\
\rb{Test Series}    & $D^*$ (m)     & \rb{$D^*/\dx$}& \rb{$H/D^*$}  \\ \hline \hline
Arup Tunnel         & 1.8           & 9             & 3.8           \\ \hline
ATF Corridors       & 0.3 -- 0.7    & 3 -- 7        & 8.5 -- 3.4    \\ \hline
Bryant Doorway      & 0.2 -- 0.7    & 5 -- 14       & 9.9 -- 3.4    \\ \hline
CSTB Tunnel         & 1.2 -- 1.3    & 12 -- 13      & 1.5 -- 1.4    \\ \hline
Fleury Heat Flux    & 0.4 -- 0.6    & 8 -- 12       & Open          \\ \hline
FM/SNL              & 0.7           & 7             & 8.8 -- 8.5    \\ \hline
Hamins Burner       & 0.04 -- 0.5   & 6             & Open          \\ \hline
Heskestad           & 0.4 -- 44     & 5 -- 20       & Open          \\ \hline
LLNL Enclosure      & 0.3 -- 0.6    & 1 -- 3        & 15.9 -- 6.9   \\ \hline
McCaffrey Plume     & 0.2 -- 0.3    & 5 -- 20       & Open          \\ \hline
NBS Multi-Room      & 0.4           & 4             & 6.2           \\ \hline
NIST RSE            & 0.3 -- 0.8    & 12 -- 32      & 3.5 -- 1.3    \\ \hline
NIST/NRC            & 0.6 -- 1.3    & 5 -- 11       & 6.5 -- 3.1    \\ \hline
NRCC Facade         & 1.8 -- 2.4    & 18 -- 24      & 1.5 -- 1.2    \\ \hline
NRL/HAI             & 0.3 -- 0.7    & 9 -- 10       & Open          \\ \hline
Sandia Plume        & 1.2 -- 1.8    & 20 -- 118     & Open          \\ \hline
SP AST              & 0.7           & 14            & 3.5           \\ \hline
Steckler            & 0.2 -- 0.4    & 5 -- 9        & 9.1 -- 4.8    \\ \hline
UL/NFPRF            & 1.7 -- 2.4    & 8 -- 12       & 4.5 -- 3.2    \\ \hline
Ulster SBI          & 0.2 -- 0.3    & 12 -- 15      & Open          \\ \hline
USCG/HAI            & 0.5 -- 0.9    & 5 -- 9        & 5.6 -- 3.2    \\ \hline
USN Hawaii          & 0.4 -- 2.1    & 2 -- 11       & 40.3 -- 7.1   \\ \hline
USN Iceland         & 0.4 -- 2.8    & 2 -- 14       & 59 -- 7.8     \\ \hline
Vettori Flat        & 1.0           & 12            & 2.8           \\ \hline
Vettori Sloped      & 1.0           & 10            & 2.6           \\ \hline
VTT Large Hall      & 1.2 -- 1.6    & 5 -- 6        & 15.8 -- 12.1  \\ \hline
WTC                 & 0.9 -- 1.1    & 9 -- 11       & 4.1 -- 3.5    \\ \hline
\end{tabular}
\end{center}
\label{Numerical_Parameters}
\nopagebreak
\end{table}






