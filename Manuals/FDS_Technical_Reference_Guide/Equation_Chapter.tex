\chapter{Governing Equations and Solution Procedure}

\label{basisformodel}

This chapter presents the governing equations of FDS and an outline of the general solution procedure. Details of the individual equations
are described in later chapters.
The governing equations are presented as a set of
partial differential equations, with appropriate simplifications and approximations
noted. The numerical method essentially consists of a finite difference approximation of the governing equations and a procedure for
updating these equations in time.

\section{Governing Equations}
\label{govequations}

This section introduces the basic conservation equations for mass, momentum and energy for a
Newtonian fluid. These are the same equations that can
be found in almost any textbook on fluid dynamics or CFD. A particularly useful
reference for a description of the equations, the notation used,
and the various approximations employed is Anderson {\em et al.}~\cite{Anderson:1}.
Note that this is a set of partial differential equations consisting of
six equations for six unknowns, all functions of three spatial dimensions and time:
the density $\rho$, the three components
of velocity $\bu=[u,v,w]^T$, the temperature $T$, and the pressure $p$.

\subsection{Mass and Species Transport}

Mass conservation can be expressed either in terms of the density, $\rho$,
\be \dod{\rho}{t} + \nabla\!\cdot (\rho \bu)  =  \dm_b'''  \label{mass} \ee
or in terms of the individual gaseous species, $Y_\alpha$:
\be \dod{ }{t}(\rho Y_\alpha) + \nabla\!\cdot (\rho Y_\alpha \bu) = \nabla\!\cdot (\rho D_\alpha \nabla Y_\alpha) + \dm_\alpha''' + \dm_{b,\alpha}''' \label{species} \ee
Here $\dm_b'''=\sum_\alpha \dm_{b,\alpha}'''$ is the production rate of species by evaporating droplets or particles.
Summing these equations over all species yields the original mass conservation equation because
$\sum Y_\alpha=1$ and $\sum \dm_\alpha''' = 0$ and $\sum \dm_{b,\alpha}'''=\dm_b'''$, by definition, and because it is assumed
that $\sum \rho D_\alpha \nabla Y_\alpha = 0$. This last assertion is not true, in general. However, transport equations are solved for total
mass and all but one of the species, implying that the diffusion coefficient of the implicit species is chosen so that the sum of all the
diffusive fluxes is zero.

\subsection{Momentum Transport}

The momentum equation in conservative form is written:
\be \dod{}{t} (\rho \bu) + \nabla\!\cdot (\rho \bu \bu) =
- \nabla p + \nabla\!\cdot \btau_{ij} + \rho \bg + \bof_b \label{momentum} \ee
where $\bg$ is the gravity vector and $\bof_b$ represents external forces such as the drag exerted by liquid droplets. The deviatoric (trace free) stress tensor $\btau_{ij}$ is:
\be \btau_{ij} = 2 \mu \left( \bS_{ij} - \frac{1}{3} \bdelta_{ij} (\nabla \cdot \bu) \right) \quad ; \quad
   \bdelta_{ij}=\left\{ \begin{array}{ll} 1 & i=j \\ 0 & i\ne j \end{array} \right.   \quad ; \quad
   \bS_{ij} = \frac{1}{2} \left( \dod{u_i}{x_j}+\dod{u_j}{x_i} \right) \quad i,j=1,2,3   \ee
The term $\bS_{ij}$ is the symmetric rate-of-strain tensor, written using conventional tensor notation. The symbol $\mu$ is the dynamic viscosity of the fluid.

The overall computation can either be treated as a Direct Numerical Simulation (DNS), in which the dissipative terms are computed directly, or as a Large Eddy Simulation (LES), in which the large-scale eddies are computed directly and the subgrid-scale dissipative processes are modeled. The numerical algorithm is designed so that LES becomes DNS as the grid is refined. Most applications of FDS require LES. For example, in simulating the flow of smoke through a large, multi-room enclosure, it is not possible to resolve the combustion and transport processes directly. However, for small-scale combustion experiments, it is possible to compute the transport and combustion processes directly.

Chapter~\ref{momentum_chapter} contains a detailed description of the numerical solution of the momentum and pressure equations. For the purpose of outlining the solution procedure below, it is sufficient to consider the momentum equation written as:
\be \dod{\bu}{t} + \bF + \nabla {\cal H} = 0  \label{momentum2} \ee
and the pressure equation as
\be \nabla^2 {\cal H} = -\dod{ }{t} (\nabla\!\cdot \bu) - \nabla\!\cdot \bF    \label{simplephi2} \ee
which is obtained by taking the divergence of the momentum equation.


\subsection{Energy Transport}

The energy conservation equation is written in terms of the {\em sensible enthalpy}, $h_s$:
\be \dod{ }{t}(\rho h_s) + \nabla\!\cdot (\rho h_s \bu) = \frac{\mbox{D}p}{\mbox{D}t}  + \dq''' - \dq_b'''
        - \nabla\!\cdot \dbq'' + \epsilon \label{energy} \ee
The sensible enthalpy is a function of the temperature:
\be
  h_s = \sum_\alpha Y_\alpha h_{s,\alpha} \quad ; \quad h_{s,\alpha}(T)=\int_{T_0}^T c_{p,\alpha}(T') \,\mbox{d}T'
\ee
Note the use of the material derivative, $\mbox{D}(\,)/\mbox{D}t\equiv\partial(\,) /\partial t + \bu \cdot \nabla (\,)$. The term
$\dq'''$ is the heat release rate per unit volume from a chemical reaction. The term $\dq_b'''$ is the energy transferred to the evaporating droplets.
The term $\dbq''$ represents the conductive and radiative heat fluxes:
\be \dbq'' = -k \nabla T - \sum_\alpha h_{s,\alpha} \rho D_\alpha \nabla Y_\alpha + \dbq_r'' \ee
where $k$ is the thermal conductivity. The viscous dissipation, $\epsilon$, which shows up as a source term in the thermal energy equation, is often neglected for low-speed flows.  Further discussion on the energy equation can be found in Appendix \ref{app_divergence}.




\subsection{Equation of State}

The ideal gas equation of state is
\be p = \frac{\rho \R T}{\bW}  \label{basicstate} \ee
An approximate form of the Navier-Stokes equations appropriate for low Mach number applications is used in the model. The approximation involves the filtering out of acoustic waves while allowing for large variations in temperature and density~\cite{Rehm:1}. This gives the equations an elliptic character, consistent with low speed, thermal convective processes. In practice, this means that the spatially resolved pressure, $p(x,y,z)$, is replaced by an ``average'' or ``background'' pressure, $\bp_m(z,t)$, that is only a function of time and height above the ground.
\be \bp_m(z,t) = \rho T {\cal R} \sum_\alpha  Y_\alpha/W_\alpha \ee
Taking the material derivative of the background pressure and substituting the result into the energy conservation equation yields an expression for the velocity divergence, $\nabla\!\cdot \bu$, that is an important term in the numerical algorithm. The source terms from the energy conservation equation are incorporated into the divergence, which appears in the mass transport equations. The temperature is found from the density and background pressure via the equation of state.


\clearpage

\section{Solution Procedure}

This section describes the basic time-marching algorithm of FDS.
FDS uses a second-order accurate finite-difference approximation to the governing equations on a series of connected recti-linear meshes.
The flow variables are updated in time using an explicit second-order Runge-Kutta scheme.
This section describes how this algorithm is used to advance in time the density, species mass fractions, velocity components, and
background and perturbation pressure. Let $\rho^n$, $Y_\alpha^n$, $\bu^n$, $\bp_m^n$ and $\cH^n$ denote these variables at the $n$th time step.

\begin{enumerate}
\item Compute the ``patch-average'' velocity field $\bar{\mathbf{u}}^n$ to force normal components of velocity to match at mesh interface boundaries (see Section \ref{app_pressure_correction}). Note that this change in the velocity field creates an error in the divergence which is to be corrected when the velocities
    are advanced in time.

\item Estimate $\rho$, $Y_\alpha$, and $\bp_m$ at the next time step with an explicit Euler step. For
example, the density is estimated by
\be \frac{\rho^*-\rho^n}{\dt} + \nabla\!\cdot \rho^n \bar{\bu}^n = 0 \ee

\item Exchange values of density and mass fraction, $\rho^*$ and $Y_\alpha^*$, at mesh boundaries. The word ``exchange'' implies that information is to be passed from one mesh to another, if necessary via MPI (Message Passing Interface) calls.

\item Apply boundary conditions for $\rho^*$ and $Y_\alpha^*$.

\item Compute the divergence, $\widehat{\nabla\!\cdot \bu^*}$, using the estimated thermodynamic quantities. Note that the hat symbol implies that the estimated velocity field, $\bu^*$, has not been computed yet. The calculation of the pressure field in the next step shall ensure that the divergence of the updated velocity field is the same as that which is
    computed here.

\item Solve the Poisson equation for the pressure fluctuation with a direct solver on each individual mesh:
\be \nabla^2 \cH^n = - \left[ \frac{ \widehat{\nabla\!\cdot \bu^*} -
  \widehat{\nabla\!\cdot \bu^n} - \nabla\!\cdot (\bar{\bu}^n - \bu^n) }{\dt} \right] - \nabla\!\cdot \bar{\bF}^n  \ee
Note that the vector $\bar{\bF}^n = \bF(\rho^n,\bar{\bu}^n)$ is computed using patch-averaged velocities. Note also that the term
$\nabla\!\cdot (\bar{\bu}^n - \bu^n)$ corrects the error in the divergence introduced by the averaging of velocity components at
mesh interfaces.

\item Estimate the velocity at the next time step
\be
\frac{\bu^* - \bar{\bu}^n}{\dt} +  \bar{\bF}^n + \nabla {\cal H}^n = 0
\ee
Note that the divergence of the estimated velocity field, $\nabla\!\cdot \bu^*$, is identically
equal to the divergence, $\widehat{\nabla\!\cdot \bu^*}$, that
was derived from the estimated thermodynamic quantities.

\item Check the time step at this point to ensure that
\be \dt \; \hbox{max} \left( \frac{|u|}{\dx},\frac{|v|}{\dy},\frac{|w|}{\dz} \right) < 1 \quad ; \quad
    2 \; \dt \; \nu \; \left(\frac{1}{\dx^2} + \frac{1}{\dy^2} + \frac{1}{\dz^2} \right) < 1 \ee
If the time step is too large, it is reduced so that it satisfies
both constraints and the procedure returns to the beginning of the time step.
If the time step satisfies the stability criteria, the procedure continues to the corrector step.
See Section~\ref{stability} for more details on stability.
\end{enumerate}

\noindent
This concludes the ``Predictor'' stage of the time step.  At this point, values of $\cH^n$ and the components of $\bu^*$ are exchanged at mesh boundaries via MPI calls.

\begin{enumerate}
\item Compute the ``patch-average'' velocity field $\bar{\bu}^*$ (see Section \ref{app_pressure_correction}).

\item Apply the second part of the Runge-Kutta update to the mass variables. For example, the density is corrected
\be
\frac{\rho^{n+1} - \ha \left(\rho^n + \rho^* \right)}{\dt/2} +  \nabla\!\cdot \rho^* \bar{\bu}^* = 0
\ee

\item Exchange values of $\rho^n$ and $Y_\alpha^n$ at mesh boundaries.

\item Apply boundary conditions for $\rho^n$ and $Y_\alpha^n$.

\item Compute the divergence, $\widehat{\nabla\!\cdot \bu^{n+1}}$ from the corrected thermodynamic quantities.

\item Compute the pressure fluctuation using estimated quantities
\be
\label{eqn_corrector_poisson2}
\nabla^2{\cal H}^* = - \left[ \frac{ \widehat{\nabla\cdot\bu^{n+1}} - \ha \left( \widehat{\nabla\cdot \bu^*} + \widehat{\nabla\cdot \bu^n} \right) }{\dt/2} \right]
   - \nabla\!\cdot \bar{\mathbf{F}}^*
\ee
Note that the same type of correction is made for the divergence error at mesh boundaries.

\item Update the velocity via the second part of the Runge-Kutta scheme
\be
\frac{ \bu^{n+1} - \ha \left( \bar{\bu}^* + \bar{\bu}^n \right)}{\dt/2} + \bar{\mathbf{F}}^* + \nabla {\cal H}^*  = 0
\ee
Note again that the divergence of the corrected velocity field is identically equal to the divergence that was computed earlier.

\item At the conclusion of the time step, values of $\cal H^*$ and the components of $\bu^{n+1}$ are exchanged at mesh boundaries via MPI calls.

\end{enumerate}





\section{Spatial Discretization}

Spatial derivatives in the governing equations are written as second-order accurate
finite differences on a rectilinear grid. The overall
domain is a rectangular box that is divided into rectangular grid cells.
Each cell is assigned indices $i$, $j$ and $k$ representing the
position of the cell in the $x$, $y$ and $z$ directions, respectively.
Scalar quantities are assigned in the center of each grid cell; thus,
$\rho_{ijk}^n$ is the density at the $n$th time step in the center
of the cell whose indices are $i$, $j$ and $k$. Vector quantities like
velocity are assigned at their appropriate cell faces. For example,
$u_{ijk}^n$ is the $x$-component of velocity at the
positive-oriented face of the $ijk$th cell; $u_{i-1,jk}^n$ is defined at the
negative-oriented face of the same cell.


