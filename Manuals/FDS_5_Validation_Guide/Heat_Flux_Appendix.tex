

\section{NIST/NRC Test Series, Compartment Walls, Floor and Ceiling}

Thirty-six heat flux gauges were positioned at various locations on all four walls of the compartment,
plus the ceiling and floor.  Comparisons between measured and predicted heat fluxes and surface temperatures are shown
on the following pages for a selected number of locations.
Over half of the measurement points are in roughly the same relative location to the fire and hence
the measurements and predictions are similar.  For this reason, data for the east and north walls are shown
because the data from the south and west walls are comparable.  Data from the south wall is used in cases where
the corresponding instrument on the north wall failed, or in cases where the fire is positioned close to the south wall.
For each test, eight locations are used for comparison, two on the long (mainly north) wall,
two on the short (east) wall, two on the floor, and two on the ceiling.  Of the two locations for each panel,
one is considered in the far-field, relatively remote from the fire; one is in the near-field,
relatively close to the fire.  How close or far varies from test to test, depending on the availability of working flux gauges.
The two short wall locations are equally remote from the fire; thus, one location is in the lower layer, one in the upper.
Table lists the locations for each test.
The heat flux gauges used on the compartment walls measured the net, not total, heat flux.
FDS predicts the net heat flux, but this prediction cannot be compared directly with the measured net heat
flux because the predicted and measured wall temperatures can differ, and this affects the net heat flux.
In a sense, the net heat flux and surface temperature are coupled, and it is difficult to assess the accuracy of the models
if the two quantities cannot be decoupled.  For the purpose of comparing prediction and measurement,
the following correction has been applied to both the measured and predicted net heat fluxes:
\be  \dq_{\hbox{\tiny total}}'' = \dq_{\hbox{\tiny net}}'' + \sigma (T_s^4-T_\infty^4) + h (T_s - T_\infty) \ee
where $T_s$ is the temperature of the surface.  A constant convective heat transfer coefficient is assumed
(5 W/m$^2$/K) and an emissivity of 1.
After applying the correction, it is easier to compare total heat fluxes that are independent of the surface temperature.

Heat flux gauges and thermocouples were attached to the compartment walls, floor, and ceiling of the NIST/NRC test compartment. Following is a list of the
coordinates of each, relative to the floor in the southwest corner of the compartment. The compartment was 21.66~m by 7.04~m by 3.82~m high. All coordinates are
accurate to the nearest centimeter.

\begin{figure}[h!]
\begin{center}
\begin{tabular}{|l|c|c|c|}
\hline
Name              & $x$   & $y$  & $z$ \\ \hline \hline
TC North U-1-2    & 3.85  & 7.04 & 1.49 \\ \hline
TC North U-2-2    & 3.86  & 7.04 & 3.71 \\ \hline
TC North U-3-2    & 9.48  & 7.04 & 1.86 \\ \hline
TC North U-4-2    & 12.07 & 7.04 & 1.88 \\ \hline
TC North U-5-2    & 17.69 & 7.04 & 1.49 \\ \hline
TC North U-6-2    & 17.69 & 7.04 & 3.69 \\ \hline
TC South U-1-2    & 3.86  & 0    & 1.49 \\ \hline
TC South U-2-2    & 3.86  & 0    & 3.82 \\ \hline
TC South U-3-2    & 9.54  & 0    & 1.86 \\ \hline
TC South U-4-2    & 12.08 & 0    & 1.86 \\ \hline
TC South U-5-2    & 17.69 & 0    & 1.50 \\ \hline
TC South U-6-2    & 17.74 & 0    & 3.70 \\ \hline
TC East U-1-2     & 21.66 & 1.52 & 1.12 \\ \hline
TC East U-2-2     & 21.66 & 1.52 & 2.40 \\ \hline
TC East U-3-2     & 21.66 & 5.68 & 1.13 \\ \hline
TC East U-4-2     & 21.66 & 5.70 & 2.42 \\ \hline
TC Floor U-1-2    & 3.08  & 3.51 & 0 \\ \hline
TC Floor U-2-2    & 9.08  & 1.94 & 0 \\ \hline
TC Floor U-3-2    & 9.06  & 5.97 & 0 \\ \hline
TC Floor U-4-2    & 10.86 & 2.38 & 0 \\ \hline
TC Floor C-5-2    & 10.93 & 5.20 & 0.01 \\ \hline
TC Floor U-6-2    & 13.13 & 1.99 & 0 \\ \hline
TC Floor U-7-2    & 13.00 & 5.92 & 0 \\ \hline
TC Floor U-8-2    & 18.63 & 3.54 & 0 \\ \hline
TC Ceiling U-1-2  & 3.04  & 3.60 & 3.82 \\ \hline
TC Ceiling C-2-2  & 8.99  & 2.00 & 3.82 \\ \hline
TC Ceiling C-3-2  & 9.03  & 5.97 & 3.82 \\ \hline
TC Ceiling C-4-2  & 10.79 & 2.38 & 3.82 \\ \hline
TC Ceiling C-5-2  & 10.79 & 5.20 & 3.82 \\ \hline
TC Ceiling C-6-2  & 13.00 & 2.07 & 3.82 \\ \hline
TC Ceiling C-7-2  & 12.84 & 5.98 & 3.82 \\ \hline
TC Ceiling U-8-2  & 18.71 & 3.54 & 3.82 \\ \hline
\end{tabular}
\end{center}
\end{figure}

% NIST/NRC Cable B Heat Flux

\begin{figure}[p]
\begin{tabular*}{\textwidth}{l@{\extracolsep{\fill}}r}
\includegraphics[height=2.2in]{FIGURES/NIST_NRC/NIST_NRC_01_v5_Long_Wall_Flux} &
\includegraphics[height=2.2in]{FIGURES/NIST_NRC/NIST_NRC_07_v5_Long_Wall_Flux} \\
\includegraphics[height=2.2in]{FIGURES/NIST_NRC/NIST_NRC_02_v5_Long_Wall_Flux} &
\includegraphics[height=2.2in]{FIGURES/NIST_NRC/NIST_NRC_08_v5_Long_Wall_Flux} \\
\includegraphics[height=2.2in]{FIGURES/NIST_NRC/NIST_NRC_04_v5_Long_Wall_Flux} &
\includegraphics[height=2.2in]{FIGURES/NIST_NRC/NIST_NRC_10_v5_Long_Wall_Flux} \\
%\includegraphics[height=2.2in]{FIGURES/NIST_NRC/NIST_NRC_13_v5_Long_Wall_Flux} &
%\includegraphics[height=2.2in]{FIGURES/NIST_NRC/NIST_NRC_16_v5_Long_Wall_Flux}
\end{tabular*}
\label{NIST_NRC_Long_Wall_Flux_Closed}
\end{figure}

\begin{figure}[p]
\begin{tabular*}{\textwidth}{l@{\extracolsep{\fill}}r}
\includegraphics[height=2.2in]{FIGURES/NIST_NRC/NIST_NRC_03_v5_Long_Wall_Flux} &
\includegraphics[height=2.2in]{FIGURES/NIST_NRC/NIST_NRC_09_v5_Long_Wall_Flux} \\
\includegraphics[height=2.2in]{FIGURES/NIST_NRC/NIST_NRC_05_v5_Long_Wall_Flux} &
\includegraphics[height=2.2in]{FIGURES/NIST_NRC/NIST_NRC_14_v5_Long_Wall_Flux} \\
\includegraphics[height=2.2in]{FIGURES/NIST_NRC/NIST_NRC_15_v5_Long_Wall_Flux} &
\includegraphics[height=2.2in]{FIGURES/NIST_NRC/NIST_NRC_18_v5_Long_Wall_Flux}
\end{tabular*}
\label{NIST_NRC_Long_Wall_Flux_Open}
\end{figure}


