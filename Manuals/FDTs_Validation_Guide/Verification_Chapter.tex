% !TEX root = FDTs_Validation_Guide.tex

\chapter{Verification}
\label{Verification_Chapter}

\section{HGL Temperature and Depth}

\subsection{HGL Temperature, Natural Ventilation}

Test: LLNL Test 51

\begin{table}[!ht]
\caption[Input parameters, HGL temperature, natural ventilation]
{Input parameters, HGL temperature, natural ventilation.}
\begin{center}
\begin{tabular}{|l|c|}
\hline
                        &              \\
\rb{Input Parameter}    &  \rb{Value}  \\ \hline \hline
$\dot Q$ (kW)           &  200         \\ \hline
$L$ (m)                 &  6.0         \\ \hline
$W$ (m)                 &  4.0         \\ \hline
$H$ (m)                 &  3.0         \\ \hline
$H_v$ (m)               &  2.06        \\ \hline
$W_v$ (m)               &  0.76        \\ \hline
$k$ (kW/m/K)            &  0.000463    \\ \hline
$\rho$ (kg/m$^3$)       &  1607        \\ \hline
$c$ (kJ/kg/K)           &  1.0         \\ \hline
$\delta$ (m)            &  0.10        \\ \hline
$T_\infty$ ($^\circ$C)  &  33          \\ \hline
\end{tabular}
\end{center}
\end{table}

\noindent Expected result: At 60~s, the HGL temperature $T_g$ is 111.45~$^\circ$C; at 120~s, $T_g$ is 121.05~$^\circ$C; at 180~s, $T_g$ is 127.21~$^\circ$C.


\clearpage


\subsection{HGL Temperature, Forced Ventilation (FPA)}

Test: FM/SNL Test 1

\begin{table}[!ht]
\caption[Input parameters, HGL temperature, forced ventilation]
{Input parameters, HGL temperature, forced ventilation.}
\begin{center}
\begin{tabular}{|l|c|}
\hline
                        &              \\
\rb{Input Parameter}    &  \rb{Value}  \\ \hline \hline
$\dot Q$ (kW)           &  516         \\ \hline
$\dot m$ (kg/s)         &  4.5         \\ \hline
$L$ (m)                 &  18.3        \\ \hline
$W$ (m)                 &  12.2        \\ \hline
$H$ (m)                 &  6.1         \\ \hline
$k$ (kW/m/K)            &  0.00023     \\ \hline
$\rho$ (kg/m$^3$)       &  1000        \\ \hline
$c$ (kJ/kg/K)           &  1.16        \\ \hline
$\delta$ (m)            &  0.025       \\ \hline
$T_\infty$ ($^\circ$C)  &  15          \\ \hline
\end{tabular}
\end{center}
\end{table}

\noindent Expected result: At 60~s, the HGL temperature $T_g$ is 53.07~$^\circ$C; at 120~s, $T_g$ is 58.13~$^\circ$C; at 180~s, $T_g$ is 61.39~$^\circ$C.


\subsection{HGL Temperature, Forced Ventilation (DB)}

Test: FM/SNL Test 1

\begin{table}[!ht]
\caption[Input parameters, HGL temperature, forced ventilation]
{Input parameters, HGL temperature, forced ventilation.}
\begin{center}
\begin{tabular}{|l|c|}
\hline
                        &              \\
\rb{Input Parameter}    &  \rb{Value}  \\ \hline \hline
$\dot Q$ (kW)           &  516         \\ \hline
$\dot m$ (kg/s)         &  4.5         \\ \hline
$L$ (m)                 &  18.3        \\ \hline
$W$ (m)                 &  12.2        \\ \hline
$H$ (m)                 &  6.1         \\ \hline
$k$ (kW/m/K)            &  0.00023     \\ \hline
$\rho$ (kg/m$^3$)       &  1000        \\ \hline
$c$ (kJ/kg/K)           &  1.16        \\ \hline
$\delta$ (m)            &  0.025       \\ \hline
$T_\infty$ ($^\circ$C)  &  15          \\ \hline
\end{tabular}
\end{center}
\end{table}

\noindent Expected result: At 60~s, the HGL temperature $T_g$ is 34.60~$^\circ$C; at 120~s, $T_g$ is 40.88~$^\circ$C; at 180~s, $T_g$ is 45.16~$^\circ$C.


\clearpage


\subsection{HGL Temperature, No Ventilation (Beyler)}

Test: LLNL Test 1

\begin{table}[!ht]
\caption[Input parameters, HGL temperature, no ventilation]
{Input parameters, HGL temperature, no ventilation.}
\begin{center}
\begin{tabular}{|l|c|}
\hline
                        &              \\
\rb{Input Parameter}    &  \rb{Value}  \\ \hline \hline
$\dot Q$ (kW)           &  200         \\ \hline
$L$ (m)                 &  6.0         \\ \hline
$W$ (m)                 &  14.0        \\ \hline
$H$ (m)                 &  4.5         \\ \hline
$k$ (kW/m/K)            &  0.000463    \\ \hline
$\rho$ (kg/m$^3$)       &  1607        \\ \hline
$c$ (kJ/kg/K)           &  1.0         \\ \hline
$\delta$ (m)            &  0.1         \\ \hline
$T_\infty$ ($^\circ$C)  &  23          \\ \hline
\end{tabular}
\end{center}
\end{table}

\noindent Expected result: At 60~s, the HGL temperature $T_g$ is 49.87~$^\circ$C; at 120~s, $T_g$ is 63.33~$^\circ$C; at 180~s, $T_g$ is 73.67~$^\circ$C.


\clearpage


\subsection{HGL Depth (ASET)}

Test: NIST/NRC Test 1

\begin{table}[!ht]
\caption[Input parameters, HGL depth]
{Input parameters, HGL depth.}
\begin{center}
\begin{tabular}{|l|c|}
\hline
                        &              \\
\rb{Input Parameter}    &  \rb{Value}  \\ \hline \hline
$\dot Q$ (kW)           &  410         \\ \hline
$L$ (m)                 &  21.66       \\ \hline
$W$ (m)                 &  7.04        \\ \hline
$H$ (m)                 &  3.82        \\ \hline
$k$ (kW/m/K)            &  0.00012     \\ \hline
$\rho$ (kg/m$^3$)       &  737         \\ \hline
$c$ (kJ/kg/K)           &  0.9         \\ \hline
$\delta$ (m)            &  0.0254      \\ \hline
$T_\infty$ ($^\circ$C)  &  22          \\ \hline
Location Factor (-)     &  1           \\ \hline
$\chi_l$ (-)            &  0           \\ \hline
$z_f$ (-)               &  0           \\ \hline
\end{tabular}
\end{center}
\end{table}

\noindent Expected result: At 10~s, the HGL depth $z$ is 3.47~m; at 20~s, $z$ is 3.17~m; at 30~s, $z$ is 2.89~m.


\subsection{HGL Depth (Yamana and Tanaka)}

Test: NIST/NRC Test 1

\noindent Note: In the FDTs spreadsheets, this HGL depth correlation is used with the MQH correlation (natual ventilation). In this verification case, the method of Beyler is used to calculate the HGL temperature,~$T_g$.

\begin{table}[!ht]
\caption[Input parameters, HGL depth]
{Input parameters, HGL depth.}
\begin{center}
\begin{tabular}{|l|c|}
\hline
                        &              \\
\rb{Input Parameter}    &  \rb{Value}  \\ \hline \hline
$\dot Q$ (kW)           &  410         \\ \hline
$L$ (m)                 &  21.66       \\ \hline
$W$ (m)                 &  7.04        \\ \hline
$H$ (m)                 &  3.82        \\ \hline
$k$ (kW/m/K)            &  0.00012     \\ \hline
$\rho$ (kg/m$^3$)       &  737         \\ \hline
$c$ (kJ/kg/K)           &  0.9         \\ \hline
$\delta$ (m)            &  0.0254      \\ \hline
$T_\infty$ ($^\circ$C)  &  22          \\ \hline
\end{tabular}
\end{center}
\end{table}

\noindent Expected result: At 10~s, the HGL depth $z$ is 3.54~m; at 20~s, $z$ is 3.29~m; at 30~s, $z$ is 3.07~m.


\clearpage


\section{Fire Plumes}

\subsection{Plume Temperature (Heskestad)}

\begin{table}[!ht]
\caption[Input parameters, plume temperature (Heskestad)]
{Input parameters, plume temperature (Heskestad).}
\begin{center}
\begin{tabular}{|l|c|}
\hline
                        &              \\
\rb{Input Parameter}    &  \rb{Value}  \\ \hline \hline
$\dot Q$ (m)            &  1245        \\ \hline
$z$ (m)                 &  6           \\ \hline
$A$ (m$^2$)             &  1.075       \\ \hline
$\chi_r$ (-)            &  0.40        \\ \hline
$T_\infty$ ($^\circ$C)  &  22          \\ \hline
\end{tabular}
\end{center}
\end{table}

\noindent Expected result: The plume temperature $T_{0}$ is 133.76~$^\circ$C.


\subsection{Plume Temperature (McCaffrey)}

\begin{table}[!ht]
\caption[Input parameters, plume temperature (McCaffrey)]
{Input parameters, plume temperature (McCaffrey).}
\begin{center}
\begin{tabular}{|l|c|}
\hline
                        &              \\
\rb{Input Parameter}    &  \rb{Value}  \\ \hline \hline
$\dot Q$ (m)            &  1245        \\ \hline
$z$ (m)                 &  6           \\ \hline
$T_\infty$ ($^\circ$C)  &  22          \\ \hline
\end{tabular}
\end{center}
\end{table}

\noindent Expected result: The plume temperature $T_{0}$ is 153.21~$^\circ$C.


\clearpage


\section{Target Temperature}

\subsection{Cable Failure Time (THIEF)}

Test: CAROLFIRE Penlight Test 1

\begin{table}[!ht]
\caption[Input parameters, cable failure time]
{Input parameters, cable failure time.}
\begin{center}
\begin{tabular}{|l|l|}
\hline
                             &                                \\
\rb{Input Parameter}         &  \rb{Value}                    \\ \hline \hline
Time Ramp                    &  0, 70, 820, 1240, 1600, 1800  \\ \hline
Temperature Ramp             &  24, 480, 480, 290, 190, 0     \\ \hline
Cable Diameter (mm)          &  16.3                          \\ \hline
Mass per Unit Length (kg/m)  &  0.529                         \\ \hline
Jacket Thickness (mm)        &  1.5                           \\ \hline
Conduit Diameter (mm)        &  -                             \\ \hline
Conduit Thickness (mm)       &  -                             \\ \hline
$T_\infty$ ($^\circ$C)       &  24                            \\ \hline
\end{tabular}
\end{center}
\end{table}

\noindent Expected result: At 50~s, the exposing temperature is 349.5~$^\circ$C, and the cable temperature is 35.2~$^\circ$C; at 80~s, the exposing temperature is 480.0~$^\circ$C, and the cable temperature is 69.7~$^\circ$C. At 592.8~s, the cable reaches a failure temperature of 400~$^\circ$C.


\subsection{Unprotected Steel Temperature}

Test: SP AST Column, 1.1 m Diesel Fire

\begin{table}[!ht]
\caption[Input parameters, unprotected steel temperature]
{Input parameters, unprotected steel temperature.}
\begin{center}
\begin{tabular}{|l|l|}
\hline
                           &              \\
\rb{Input Parameter}       &  \rb{Value}  \\ \hline \hline
$F/V$ (1/m)                &  205         \\ \hline
$\rho_{s}$ (kg/m$^3$)      &  7850        \\ \hline
$c_{s}$ (kJ/kg/K)          &  0.6         \\ \hline
$\epsilon$ (-)             &  0.7         \\ \hline
$h_c$ (W/m$^2$/K)          &  25          \\ \hline
Correlation for $T_f$ (-)  &  McCaffrey   \\ \hline
$\dot Q$ (kW)              &  1434        \\ \hline
Height (m)                 &  1           \\ \hline
\end{tabular}
\end{center}
\end{table}

\noindent Expected result: The fire temperature (plume temperature at a height of 1~m from McCaffrey) is 872.51~$^\circ$C. At 15~s, the steel temperature is 74.0~$^\circ$C; at 30~s, the steel temperature is 130.7~$^\circ$C; at 45~s, the steel temperature is 186.1~$^\circ$C.


\subsection{Protected Steel Temperature}

Test: WTC Test 4, Bar Structural Element

\begin{table}[!ht]
\caption[Input parameters, protected steel temperature]
{Input parameters, protected steel temperature.}
\begin{center}
\begin{tabular}{|l|l|}
\hline
                           &              \\
\rb{Input Parameter}       &  \rb{Value}  \\ \hline \hline
$c_{s}$ (kJ/kg/K)          &  0.450       \\ \hline
$W/D$ (kg/m$2$)            &  51.1        \\ \hline
$k_{i}$ (W/m/K)            &  0.10        \\ \hline
$\rho_{i}$ (kg/m$^3$)      &  208         \\ \hline
$c_{i}$ (kJ/kg/K)          &  2.0         \\ \hline
$h_{i}$ (m)                &  0.0191      \\ \hline
Correlation for $T_f$ (-)  &  MQH         \\ \hline
$\dot Q$ (kW)              &  3200        \\ \hline
$L$ (m)                    &  7.04        \\ \hline
$W$ (m)                    &  3.60        \\ \hline
$H$ (m)                    &  3.82        \\ \hline
$H_v$ (m)                  &  2.82        \\ \hline
$W_v$ (m)                  &  2.4         \\ \hline
$k$ (kW/m/K)               &  0.00012     \\ \hline
$\rho$ (kg/m$^3$)          &  737         \\ \hline
$c$ (kJ/kg/K)              &  0.9         \\ \hline
$\delta$ (m)               &  0.0254      \\ \hline
$T_\infty$ ($^\circ$C)     &  20          \\ \hline
\end{tabular}
\end{center}
\end{table}

\noindent Expected result: At 15~s, the HGL temperature (from MQH) is 336.7~$^\circ$C and the steel temperature is 20.94~$^\circ$C; At 30~s, the HGL temperature (from MQH) is 375.4~$^\circ$C and the steel temperature is 22.1~$^\circ$C; at 45~s, the HGL temperature (from MQH) is 400.3~$^\circ$C and the steel temperature is 23.36~$^\circ$C.


\clearpage


\section{Target Heat Flux}

\subsection{Point Source Radiation Heat Flux}

Test: Fleury 100 kW, 1:1 Burner
\\ \\
\noindent In the FDTs spreadsheets, the point source radiation heat flux is calculated without accounting for the angle/elevation
of the target with respect to the fire, and the flame height is not considered. The FDTs.f90 program considers both of these and
assumes that the point source is located one-third of the height of the flame ($L_f/3$).

\begin{table}[!ht]
\caption[Input parameters, point source radiation heat flux]
{Input parameters, point source radiation heat flux.}
\begin{center}
\begin{tabular}{|l|l|}
\hline
                      &                   \\
\rb{Input Parameter}  &  \rb{Value}       \\ \hline \hline
$\dot Q$ (kW)         &  100              \\ \hline
$\chi_r$ (-)          &  0.35             \\ \hline
$A$ (m$^2$)           &  0.09             \\ \hline
$x$ (m)               &  0.50, 0.75, 1.0  \\ \hline
$z$ (m)               &  0.0              \\ \hline
IOR (-)               &  2                \\ \hline
\end{tabular}
\end{center}
\end{table}

\noindent Expected result (FDTs spreadsheets): At a radius $R$ of 0.67~m, the radiative heat flux $q''_{r}$ is 6.22~kW/m$^2$; at an $R$ of 0.92~m, $q''_{r}$ is 3.30~kW/m$^2$; at an $R$ of 1.17~m, $q''_{r}$ is 2.04~kW/m$^2$.
\\ \\
\noindent Expected result (FDTs.f90): At a radius $R$ of 0.50~m, the radiative heat flux $q''_{r}$ is 6.01~kW/m$^2$; at an $R$ of 0.75~m, $q''_{r}$ is 3.65~kW/m$^2$; at an $R$ of 1.0~m, $q''_{r}$ is 2.33~kW/m$^2$.


\clearpage


\section{Ceiling Jets and Device Activation}

\subsection{Ceiling Jet Temperature (Alpert)}

Test: NIST/NRC Test 1

\begin{table}[!ht]
\caption[Input parameters, ceiling jet temperature (Alpert)]
{Input parameters, ceiling jet temperature (Alpert).}
\begin{center}
\begin{tabular}{|l|c|}
\hline
                          &              \\
\rb{Input Parameter}      &  \rb{Value}  \\ \hline \hline
$\dot Q$ (kW)             &  410         \\ \hline
Location Factor (-)       &  1           \\ \hline
$r$ (m)                   &  5.90        \\ \hline
$H$ (m)                   &  3.72        \\ \hline
$T_{\infty}$ ($^\circ$C)  &  22          \\ \hline
\end{tabular}
\end{center}
\end{table}

\noindent Expected result: The ceiling jet temperature $T_{jet}$ is 46.45~$^\circ$C.


\subsection{Sprinkler Activation Time}

Test: Vettori Flat Ceiling Test 1

\begin{table}[!ht]
\caption[Input parameters, sprinkler activation time]
{Input parameters, sprinkler activation time.}
\begin{center}
\begin{tabular}{|l|c|}
\hline
                              &              \\
\rb{Input Parameter}          &  \rb{Value}  \\ \hline \hline
$\alpha$ (kW/s$^2$)           &  0.105       \\ \hline
Location Factor (-)           &  1           \\ \hline
RTI (m-s)$^{1/2}$             &  55          \\ \hline
$T_{activation}$ ($^\circ$C)  &  68          \\ \hline
$r$ (m)                       &  2.20        \\ \hline
$H$ (m)                       &  2.09        \\ \hline
$T_\infty$ ($^\circ$C)        &  16.6        \\ \hline
\end{tabular}
\end{center}
\end{table}

\noindent Expected result: At 50~s, the HRR $\dot Q$ is 262.5~kW, and the activation time $t_{activation}$ is 98.74~s.


\clearpage


\subsection{Smoke Detector Activation Time (Alpert)}

\begin{table}[!ht]
\caption[Input parameters, smoke detector activation time (Alpert)]
{Input parameters, smoke detector activation time (Alpert).}
\begin{center}
\begin{tabular}{|l|c|}
\hline
                          &              \\
\rb{Input Parameter}      &  \rb{Value}  \\ \hline \hline
Location Factor (-)       &  1           \\ \hline
$\alpha$ (kW/s$^2$)       &  0.00463     \\ \hline
$t_{fire}$ (s)            &  300         \\ \hline
$r$ (m)                   &  1.3         \\ \hline
$H$ (m)                   &  2.1         \\ \hline
$\Delta T_c$ ($^\circ$C)  &  10          \\ \hline
RTI (m-s)$^{1/2}$         &  5           \\ \hline
$T_\infty$ ($^\circ$C)    &  21          \\ \hline
\end{tabular}
\end{center}
\end{table}

\noindent Expected result: At 47~s, the HRR $\dot Q$ is 10.23~kW, and the activation time $t_{activation}$ is 30.8~s.


\subsection{Smoke Detector Activation Time (Milke)}


\begin{table}[!ht]
\caption[Input parameters, smoke detector activation time (Milke)]
{Input parameters, smoke detector activation time (Milke).}
\begin{center}
\begin{tabular}{|l|c|}
\hline
                          &              \\
\rb{Input Parameter}      &  \rb{Value}  \\ \hline \hline
$\alpha$ (kW/s$^2$)       &  0.00463     \\ \hline
$t_{fire}$ (s)            &  300         \\ \hline
$H$ (m)                   &  2.1         \\ \hline
$\Delta T_c$ ($^\circ$C)  &  10          \\ \hline
\end{tabular}
\end{center}
\end{table}

\noindent Expected result: At 40~s, the HRR $\dot Q$ is 7.41~kW, and the activation time $t_{activation}$ is 47.3~s.
\\ \\
Note: The FDTs spreadsheets use an incorrect unit conversion factor from kW to Btu/s, which gives an activation time of 39.5~s.


\clearpage


\subsection{Smoke Detector Activation Time (Mowrer)}

\begin{table}[!ht]
\caption[Input parameters, smoke detector activation time (Mowrer)]
{Input parameters, smoke detector activation time (Mowrer).}
\begin{center}
\begin{tabular}{|l|c|}
\hline
                      &              \\
\rb{Input Parameter}  &  \rb{Value}  \\ \hline \hline
$\alpha$ (kW/s$^2$)   &  0.00463     \\ \hline
$t_{fire}$ (s)        &  300         \\ \hline
$C_{pl}$ (-)          &  0.67        \\ \hline
$C_{cj}$ (-)          &  1.2         \\ \hline
$r$ (m)               &  1.3         \\ \hline
$H$ (m)               &  2.1         \\ \hline
\end{tabular}
\end{center}
\end{table}

\noindent Expected result: At 5~s, the HRR $\dot Q$ is 0.116~kW, and the activation time $t_{activation}$ is 5.6~s.

