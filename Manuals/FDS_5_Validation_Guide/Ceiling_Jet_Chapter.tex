\chapter{Ceiling Jets and Device Activation}

FDS is a computational fluid dynamics (CFD) model and has no explicit ceiling jet model.
Rather, temperatures throughout the fire compartment are computed directly from the governing conservation equations.
Nevertheless, temperature measurements near the ceiling can be used to evaluate the model's ability to predict the flow of
hot gases across a relatively flat ceiling. Measurements for this category are available from the NIST/NRC and the FM/SNL series.

\section{WTC Test Series}

Aspirated thermocouples were positioned 3~m to the west (TTRW1) and 2~m to the east (TTRE1) of the fire pan, 18~cm below the ceiling.

\begin{figure}[p]
\begin{tabular*}{\textwidth}{l@{\extracolsep{\fill}}r}
\includegraphics[height=2.2in]{FIGURES/WTC/WTC_01_v5_Ceiling_Jet} &
\includegraphics[height=2.2in]{FIGURES/WTC/WTC_02_v5_Ceiling_Jet} \\
\includegraphics[height=2.2in]{FIGURES/WTC/WTC_03_v5_Ceiling_Jet} &
\includegraphics[height=2.2in]{FIGURES/WTC/WTC_04_v5_Ceiling_Jet} \\
\includegraphics[height=2.2in]{FIGURES/WTC/WTC_05_v5_Ceiling_Jet} &
\includegraphics[height=2.2in]{FIGURES/WTC/WTC_06_v5_Ceiling_Jet}
\end{tabular*}
\label{WTC_Jet}
\end{figure}

\section{NIST/NRC Test Series}

The thermocouple nearest the ceiling in Tree 7, located towards the back of the compartment,
has been chosen as a surrogate for the ceiling jet temperature.
Curiously, the difference between measured and predicted temperatures is noticeably greater for the open door tests.
Certainly, the open door changes the flow pattern of the exhaust gases.
However, the predicted HGL heights for the open door tests, shown in the previous section,
do not show a noticeable difference from their closed door counterparts.
The predicted HGL temperatures are only slightly less than those measured in the open door tests,
due in large part to the contribution of Tree 7 in the layer reduction calculation.


\begin{figure}[p]
\begin{tabular*}{\textwidth}{l@{\extracolsep{\fill}}r}
\includegraphics[height=2.2in]{FIGURES/NIST_NRC/NIST_NRC_01_v5_Ceiling_Jet} &
\includegraphics[height=2.2in]{FIGURES/NIST_NRC/NIST_NRC_07_v5_Ceiling_Jet} \\
\includegraphics[height=2.2in]{FIGURES/NIST_NRC/NIST_NRC_02_v5_Ceiling_Jet} &
\includegraphics[height=2.2in]{FIGURES/NIST_NRC/NIST_NRC_08_v5_Ceiling_Jet} \\
\includegraphics[height=2.2in]{FIGURES/NIST_NRC/NIST_NRC_04_v5_Ceiling_Jet} &
\includegraphics[height=2.2in]{FIGURES/NIST_NRC/NIST_NRC_10_v5_Ceiling_Jet} \\
\includegraphics[height=2.2in]{FIGURES/NIST_NRC/NIST_NRC_13_v5_Ceiling_Jet} &
\includegraphics[height=2.2in]{FIGURES/NIST_NRC/NIST_NRC_16_v5_Ceiling_Jet}
\end{tabular*}
\label{NIST_NRC_Jet_Closed}
\end{figure}

\begin{figure}[p]
\begin{tabular*}{\textwidth}{l@{\extracolsep{\fill}}r}
\includegraphics[height=2.2in]{FIGURES/NIST_NRC/NIST_NRC_17_v5_Ceiling_Jet} &
 \\
\includegraphics[height=2.2in]{FIGURES/NIST_NRC/NIST_NRC_03_v5_Ceiling_Jet} &
\includegraphics[height=2.2in]{FIGURES/NIST_NRC/NIST_NRC_09_v5_Ceiling_Jet} \\
\includegraphics[height=2.2in]{FIGURES/NIST_NRC/NIST_NRC_05_v5_Ceiling_Jet} &
\includegraphics[height=2.2in]{FIGURES/NIST_NRC/NIST_NRC_14_v5_Ceiling_Jet} \\
\includegraphics[height=2.2in]{FIGURES/NIST_NRC/NIST_NRC_15_v5_Ceiling_Jet} &
\includegraphics[height=2.2in]{FIGURES/NIST_NRC/NIST_NRC_18_v5_Ceiling_Jet}
\end{tabular*}
\label{NIST_NRC_Jet_Open}
\end{figure}


\section{FM/SNL Test Series}

The near-ceiling thermocouples in Sectors 1 and 3 have been chosen as surrogates for the ceiling jet temperature.
The results are shown below.  The only noticeable discrepancy is in Test 21, and it is the same pattern that
was observed in the HGL temperature comparison for this test.

\begin{figure}[p]
\begin{tabular*}{\textwidth}{l@{\extracolsep{\fill}}r}
\includegraphics[height=2.2in]{FIGURES/FM_SNL/FM_SNL_04_v5_Ceiling_Jet} &
\includegraphics[height=2.2in]{FIGURES/FM_SNL/FM_SNL_05_v5_Ceiling_Jet} \\
\includegraphics[height=2.2in]{FIGURES/FM_SNL/FM_SNL_21_v5_Ceiling_Jet} &
\end{tabular*}
\label{FM_SNL_Ceiling_Jet}
\end{figure}


\clearpage

\section{ATF Corridors Series}

This series of experiments involved two fairly long corridors connected by a staircase. The fire, a natural gas sand
burner, was located on the first level at the end of the corridor away from the stairwell, which was located at the
other end. The corridor was closed at the end where the fire was located, and open at the same position on the 
second level. Two-way flow occurred on both levels because make-up air flowed from the opening on the second level down
the stairs to the first. The only opening to the enclosure was the open end of the second-level corridor.

Temperatures were measured with 7 thermocouple trees. Tree A was located fairly close to the fire on the first level. Tree~B
was located halfway down the first-level corridor. Tree~C was close to the stairwell entrance on the first level. Tree~D was located
in the doorway of the stairwell on the first level. Tree~E was located roughly along the vertical centerline of the 
stairwell. Tree~F was located near the stairwell opening on the second level. Tree~G was located near the exit at the
other end of the second-level corridor. The graphs on the following pages show the top and bottom TC from each tree for
the given fire sizes of 50~kW, 100~kW, 250~kW, 500~kW, and a mixed HRR ``pulsed'' fire.

\begin{figure}[p]
\begin{tabular*}{\textwidth}{l@{\extracolsep{\fill}}r}
\includegraphics[height=2.2in]{FIGURES/ATF_Corridors/ATF_Corridors_Jet_Temp_A_050_kW} &
\includegraphics[height=2.2in]{FIGURES/ATF_Corridors/ATF_Corridors_Jet_Temp_B_050_kW} \\
\includegraphics[height=2.2in]{FIGURES/ATF_Corridors/ATF_Corridors_Jet_Temp_C_050_kW} &
\includegraphics[height=2.2in]{FIGURES/ATF_Corridors/ATF_Corridors_Jet_Temp_D_050_kW} \\
\includegraphics[height=2.2in]{FIGURES/ATF_Corridors/ATF_Corridors_Jet_Temp_E_050_kW} &
\includegraphics[height=2.2in]{FIGURES/ATF_Corridors/ATF_Corridors_Jet_Temp_F_050_kW} \\
\includegraphics[height=2.2in]{FIGURES/ATF_Corridors/ATF_Corridors_Jet_Temp_G_050_kW} &
\end{tabular*}
\label{ATF_Corridors_Jet_Temp_50_kW}
\end{figure}

\begin{figure}[p]
\begin{tabular*}{\textwidth}{l@{\extracolsep{\fill}}r}
\includegraphics[height=2.2in]{FIGURES/ATF_Corridors/ATF_Corridors_Jet_Temp_A_100_kW} &
\includegraphics[height=2.2in]{FIGURES/ATF_Corridors/ATF_Corridors_Jet_Temp_B_100_kW} \\
\includegraphics[height=2.2in]{FIGURES/ATF_Corridors/ATF_Corridors_Jet_Temp_C_100_kW} &
\includegraphics[height=2.2in]{FIGURES/ATF_Corridors/ATF_Corridors_Jet_Temp_D_100_kW} \\
\includegraphics[height=2.2in]{FIGURES/ATF_Corridors/ATF_Corridors_Jet_Temp_E_100_kW} &
\includegraphics[height=2.2in]{FIGURES/ATF_Corridors/ATF_Corridors_Jet_Temp_F_100_kW} \\
\includegraphics[height=2.2in]{FIGURES/ATF_Corridors/ATF_Corridors_Jet_Temp_G_100_kW} &
\end{tabular*}
\label{ATF_Corridors_Jet_Temp_100_kW}
\end{figure}

\begin{figure}[p]
\begin{tabular*}{\textwidth}{l@{\extracolsep{\fill}}r}
\includegraphics[height=2.2in]{FIGURES/ATF_Corridors/ATF_Corridors_Jet_Temp_A_250_kW} &
\includegraphics[height=2.2in]{FIGURES/ATF_Corridors/ATF_Corridors_Jet_Temp_B_250_kW} \\
\includegraphics[height=2.2in]{FIGURES/ATF_Corridors/ATF_Corridors_Jet_Temp_C_250_kW} &
\includegraphics[height=2.2in]{FIGURES/ATF_Corridors/ATF_Corridors_Jet_Temp_D_250_kW} \\
\includegraphics[height=2.2in]{FIGURES/ATF_Corridors/ATF_Corridors_Jet_Temp_E_250_kW} &
\includegraphics[height=2.2in]{FIGURES/ATF_Corridors/ATF_Corridors_Jet_Temp_F_250_kW} \\
\includegraphics[height=2.2in]{FIGURES/ATF_Corridors/ATF_Corridors_Jet_Temp_G_250_kW} &
\end{tabular*}
\label{ATF_Corridors_Jet_Temp_250_kW}
\end{figure}

\begin{figure}[p]
\begin{tabular*}{\textwidth}{l@{\extracolsep{\fill}}r}
\includegraphics[height=2.2in]{FIGURES/ATF_Corridors/ATF_Corridors_Jet_Temp_A_500_kW} &
\includegraphics[height=2.2in]{FIGURES/ATF_Corridors/ATF_Corridors_Jet_Temp_B_500_kW} \\
\includegraphics[height=2.2in]{FIGURES/ATF_Corridors/ATF_Corridors_Jet_Temp_C_500_kW} &
\includegraphics[height=2.2in]{FIGURES/ATF_Corridors/ATF_Corridors_Jet_Temp_D_500_kW} \\
\includegraphics[height=2.2in]{FIGURES/ATF_Corridors/ATF_Corridors_Jet_Temp_E_500_kW} &
\includegraphics[height=2.2in]{FIGURES/ATF_Corridors/ATF_Corridors_Jet_Temp_F_500_kW} \\
\includegraphics[height=2.2in]{FIGURES/ATF_Corridors/ATF_Corridors_Jet_Temp_G_500_kW} &
\end{tabular*}
\label{ATF_Corridors_Jet_Temp_500_kW}
\end{figure}

\begin{figure}[p]
\begin{tabular*}{\textwidth}{l@{\extracolsep{\fill}}r}
\includegraphics[height=2.2in]{FIGURES/ATF_Corridors/ATF_Corridors_Jet_Temp_A_Mix_kW} &
\includegraphics[height=2.2in]{FIGURES/ATF_Corridors/ATF_Corridors_Jet_Temp_B_Mix_kW} \\
\includegraphics[height=2.2in]{FIGURES/ATF_Corridors/ATF_Corridors_Jet_Temp_C_Mix_kW} &
\includegraphics[height=2.2in]{FIGURES/ATF_Corridors/ATF_Corridors_Jet_Temp_D_Mix_kW} \\
\includegraphics[height=2.2in]{FIGURES/ATF_Corridors/ATF_Corridors_Jet_Temp_E_Mix_kW} &
\includegraphics[height=2.2in]{FIGURES/ATF_Corridors/ATF_Corridors_Jet_Temp_F_Mix_kW} \\
\includegraphics[height=2.2in]{FIGURES/ATF_Corridors/ATF_Corridors_Jet_Temp_G_Mix_kW} &
\end{tabular*}
\label{ATF_Corridors_Jet_Temp_Mix_kW}
\end{figure}





\begin{figure}[p]
\begin{center}
\begin{tabular}{c}
\includegraphics[width=5.0in]{FIGURES/ScatterPlots/Ceiling_Jet_Temperature} \\
\vspace{0.25in}
\end{tabular}
\end{center}
\caption[Summary of ceiling jet temperature predictions, WTC, NIST/NRC and FM/SNL test series.]
{Summary of ceiling jet temperature predictions, WTC, NIST/NRC and FM/SNL test series.}
\end{figure}



\clearpage

\section{UL/NFPRF Sprinkler, Vent, and Draft Curtain Experiments}
\label{UL_NFPRF:Results}

The ceiling jet is an important fire phenomenon because of the presence of automatic fire protection devices at the ceiling, like
sprinklers and smoke/heat vents. The results of the UL/NFPRF experiments provide useful data to assess the accuracy of FDS in predicting
the velocity and temperature near the ceiling, and consequently the resulting activation of sprinklers.
The UL/NFPRF test results (Series I) are summarized in Table~\ref{ULmatrix}.

The figures on the following pages display the number of sprinklers actuated as a function of time. 
The results are then summarized in Fig.~\ref{UL_NFPRF}. Note that there are no experimental uncertainty bounds on the plot because it is difficult to estimate the
combined uncertainty related to the various parameters that are input into the model. At the bottom of Fig.~\ref{UL_NFPRF}, the results of three replicate experiments
demonstrate that the total number of actuated sprinklers in each experiment is repeatable, even though individual actuation times may vary. Based on these
three replicates, there is very little, if any, uncertainty in the total number of actuated sprinklers for each test. However, the test report~\cite{Sheppard:1} does not
include uncertainty estimates for the heat release rate, thermal properties of the ceiling, sprinkler RTI, conductivity factor, actuation temperature,
median droplet diameter, and various other parameters that have been input into the model. Consequently, it is not possible to estimate the uncertainty in the
total number of actuated sprinklers due to the uncertainty in the reported parameters. The only sensitivity analysis conducted for this set of experiments was
to change the median volumetric droplet size from 1000~$\mu$m to 750~$\mu$m, which led to a reduction of approximately 50~\% in the number of predicted sprinkler actuations.

\begin{figure}[p]
\begin{tabular*}{\textwidth}{l@{\extracolsep{\fill}}r}
\includegraphics[height=2.2in]{FIGURES/UL_NFPRF/UL_NFPRF_1_01_Actuations} &
\includegraphics[height=2.2in]{FIGURES/UL_NFPRF/UL_NFPRF_1_02_Actuations} \\
\includegraphics[height=2.2in]{FIGURES/UL_NFPRF/UL_NFPRF_1_03_Actuations} &
\includegraphics[height=2.2in]{FIGURES/UL_NFPRF/UL_NFPRF_1_04_Actuations} \\
\includegraphics[height=2.2in]{FIGURES/UL_NFPRF/UL_NFPRF_1_05_Actuations} &
\includegraphics[height=2.2in]{FIGURES/UL_NFPRF/UL_NFPRF_1_06_Actuations} \\
\includegraphics[height=2.2in]{FIGURES/UL_NFPRF/UL_NFPRF_1_07_Actuations} &
\includegraphics[height=2.2in]{FIGURES/UL_NFPRF/UL_NFPRF_1_08_Actuations} \\
\end{tabular*}
\label{UL_NFPRF_1}
\end{figure}

\begin{figure}[p]
\begin{tabular*}{\textwidth}{l@{\extracolsep{\fill}}r}
\includegraphics[height=2.2in]{FIGURES/UL_NFPRF/UL_NFPRF_1_09_Actuations} &
\includegraphics[height=2.2in]{FIGURES/UL_NFPRF/UL_NFPRF_1_10_Actuations} \\
\includegraphics[height=2.2in]{FIGURES/UL_NFPRF/UL_NFPRF_1_11_Actuations} &
\includegraphics[height=2.2in]{FIGURES/UL_NFPRF/UL_NFPRF_1_12_Actuations} \\
\includegraphics[height=2.2in]{FIGURES/UL_NFPRF/UL_NFPRF_1_13_Actuations} &
\includegraphics[height=2.2in]{FIGURES/UL_NFPRF/UL_NFPRF_1_14_Actuations} \\
\includegraphics[height=2.2in]{FIGURES/UL_NFPRF/UL_NFPRF_1_15_Actuations} &
\includegraphics[height=2.2in]{FIGURES/UL_NFPRF/UL_NFPRF_1_16_Actuations} \\
\end{tabular*}
\label{UL_NFPRF_2}
\end{figure}

\begin{figure}[p]
\begin{tabular*}{\textwidth}{l@{\extracolsep{\fill}}r}
\includegraphics[height=2.2in]{FIGURES/UL_NFPRF/UL_NFPRF_1_17_Actuations} &
\includegraphics[height=2.2in]{FIGURES/UL_NFPRF/UL_NFPRF_1_18_Actuations} \\
\includegraphics[height=2.2in]{FIGURES/UL_NFPRF/UL_NFPRF_1_19_Actuations} &
\includegraphics[height=2.2in]{FIGURES/UL_NFPRF/UL_NFPRF_1_20_Actuations} \\
\includegraphics[height=2.2in]{FIGURES/UL_NFPRF/UL_NFPRF_1_21_Actuations} &
\includegraphics[height=2.2in]{FIGURES/UL_NFPRF/UL_NFPRF_1_22_Actuations} 
\end{tabular*}
\label{UL_NFPRF_3}
\end{figure}

\begin{figure}[p]
\begin{center}
\includegraphics[width=3.5in]{FIGURES/ScatterPlots/UL_NFPRF_Actuations}
\end{center}
\begin{tabular*}{\textwidth}{l@{\extracolsep{\fill}}r}
\includegraphics[height=2.2in]{FIGURES/UL_NFPRF/UL_NFPRF_1_01_08_Actuations} &
\includegraphics[height=2.2in]{FIGURES/UL_NFPRF/UL_NFPRF_1_04_07_Actuations}
\end{tabular*}
\begin{center}
\includegraphics[height=2.2in]{FIGURES/UL_NFPRF/UL_NFPRF_1_09_10_Actuations} 
\end{center}`
\caption[Summary of sprinkler actuation predictions, UL/NFPRF test series.]
{Above: Comparison of predicted and measured sprinkler activation times for the UL/NFPRF Test Series I. Below:
The results of three replicate experiments.}
\label{UL_NFPRF}
\end{figure}
