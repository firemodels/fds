% !TEX root = FDTs_Validation_Guide.tex

\chapter{Surface Temperature}
\label{Surface_Temperature_Chapter}

\section{Time to Threshold Cable Temperature}

The thermally-induced electrical failure (THIEF) of a cable can be predicted via a simple
one-dimensional heat transfer calculation, under the assumption that the cable can be
treated as a homogeneous cylinder. The governing equation for the cable temperature,
$T(r,t)$, is given by

\be
\rho c \left( \frac{\partial T}{\partial t} \right) = \frac{1}{r} \frac{\partial}{\partial r} k r \left( \frac{\partial T}{\partial r} \right)
\label{eq:cable_temp}
\ee

\noindent with the boundary condition

\be
\dot q'' = k \left( \frac{\partial T}{\partial r} \right) (R,t)
\ee

\noindent A finite difference approximation to Eq.~\ref{eq:cable_temp} is given by

\be
\rho c \left[ \frac{T_i^{n+1} - T_i^n}{\delta t} \right] = \frac{2 k}{(r_{i+1} + r_i)} \frac{1}{2 \delta r} \left[ r_i \frac{T_{i+1}^n - T_i^n}{\delta r} - r_{i-1} \frac{T_{i}^n - T_{i-1}^n}{\delta r} + r_i \frac{T_{i+1}^{n+1} - T_i^{n+1}}{\delta r} - r_{i-1} \frac{T_{i}^{n+1} - T_{i-1}^{n+1}}{\delta r} \right]
\ee

\noindent where the time step $\delta t$ is given by

\be
\delta t = \frac{c \rho \delta r^2}{2 k}
\ee


\clearpage


\section{Unprotected Steel Temperature}

The temperature rise, $\Delta T_s$, above ambient ($T_s - T_\infty$) of an unprotected steel member exposed to fire can be approximated using a quasi-steady approach as

\be
\Delta T_s = \frac{F}{V} \frac{1}{\rho_s c_s} \left[ h_c (T_f - T_s) + \sigma \epsilon (T_f^4 - T_s^4) \right] \Delta t
\label{eq:unprotected_steel}
\ee

In Eq.~\ref{eq:unprotected_steel}, $\Delta T_s$ is the temperature rise of the steel member above ambient~($^\circ$C), $F/V$ is the ratio of heated surface area to volume~(m$^{-1}$), $\rho_s$ is the density of steel~(kg/m$^3$), $c_s$ is the specific heat of steel~(J/kg/K), $h_c$ is the convective heat transfer coefficient~(W/m$^2$/K), $T_f$ is the exposing fire temperature~(K), $T_s$ is the steel temperature~(K), $\sigma$ is the Stefan-Boltzmann constant (W/m$^2$/K$^4$), $\epsilon$ is the flame emissivity~(-), and $\Delta t$ is the time step~(s).

Note that the exposing fire temperature, $T_f$, can be replaced by the hot gas layer temperature, plume temperature, or other exposing temperature.


\clearpage


\section{Protected Steel Temperature}
\label{info:protected_steel_temperature}

If the thermal capacity of the insulation layer is neglected, then the temperature rise, $\Delta T_s$, above ambient ($T_s - T_\infty$) of a structural steel element exposed to fire can be calculated using the following equation

\be
\Delta T_s = k_i \left( \frac{T_f - T_s}{c_s h_i \frac{W}{D} + \frac{1}{2} c_i \rho_i h_i^2} \right) \Delta t
\label{eq:protected_steel}
\ee

In Eq.~\ref{eq:protected_steel}, $\Delta T_s$ is the temperature rise of the steel member above ambient~($^\circ$C), $k_i$ is the thermal conductivity of the insulation material~(W/m/K), $T_f$ is the exposing fire temperature~(K), $T_s$ is the steel temperature~(K), $c_s$ is the specific heat of steel~(J/kg/K), $h_i$ is the thickness of the insulation~(m), $W/D$ is the ratio of the weight of steel section per unit length to the heated perimeter~(kg/m$^2$), $c_i$ is the specific heat of the insulation material~(J/kg/K), $\rho_i$ is the density of the insulating material~(kg/m$^3$), and $\Delta t$ is the time step~(s).

Note that the exposing fire temperature, $T_f$, can be replaced by the hot gas layer temperature, plume temperature, or other exposing temperature.

\clearpage

\section{Summary of Surface Temperature Predictions}

Summary scatter plots of the surface temperature predictions are shown on the following pages.

\begin{figure}[ht]
\begin{center}
\begin{tabular}{l}
\includegraphics[width=4.0in]{SCRIPT_FIGURES/Scatterplots/Cable_Temperature}
\end{tabular}
\end{center}
\caption[Summary of time to threshold cable temperature predictions]
{Summary of time to threshold cable temperature predictions.}
\label{Surface_Temperature_THIEF_Summary}
\end{figure}

\begin{figure}[p]
\begin{center}
\begin{tabular}{l}
\includegraphics[width=4.0in]{SCRIPT_FIGURES/Scatterplots/Steel_Temperature}
\end{tabular}
\end{center}
\caption[Summary of steel temperature predictions]
{Summary of unprotected and protected steel temperature predictions.}
\label{Surface_Temperature_Steel_Summary}
\end{figure}

