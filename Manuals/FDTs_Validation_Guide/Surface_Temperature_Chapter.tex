\chapter{Surface Temperature}

\section{Time to Threshold Cable Temperature}

The thermally-induced electrical failure (THIEF) of a cable can be predicted via a simple
one-dimensional heat transfer calculation, under the assumption that the cable can be
treated as a homogeneous cylinder. The governing equation for the cable temperature,
$T(r,t)$, is given by

\be
\rho_s c_s \left( \frac{\partial T_s}{\partial t} \right) = \frac{1}{r} \frac{\partial}{\partial r} k_s r \left( \frac{\partial T_s}{\partial r} \right)
\label{eq:cable_temp}
\ee

\noindent with the boundary condition

\be
\dot q'' = k_s \left( \frac{T_s}{r} \right) (R,t)
\ee

\noindent A finite difference approximation to Eq.~\ref{eq:cable_temp} is given by

\be
T_i^{n+1} = T_i^n + \frac{\delta t \alpha}{r_{avg} \delta r} + \frac{r_i (T_{i+1}^n - T_i^n)}{\delta r} + \frac{r_{i-1} (T_{i}^n - T_{i-1}^n)}{\delta r}
\ee

\noindent where the time step $\delta t$ is given by

\be
\delta t = \frac{c_s \rho_s \delta r^2}{2 k_s}
\ee


\clearpage


\section{Unprotected Steel Temperature}

The following equation calculates the temperature development of an unprotected steel member, using a quasi-stationary approach, iterated for successive time steps of $\Delta t$ (sec)

\be
\Delta T_s = \frac{F}{V} \frac{1}{\rho_s c_s} \left[ h_c (T_g - T_s) + \sigma \epsilon (T_g^4 - T_s^4) \right] \Delta t
\label{eq:unprotected_steel}
\ee

In Eq.~\ref{eq:unprotected_steel}, $\Delta T_s$ is the temperature of the steel member~($^\circ$C), $F/V$ is the ratio of heated surface area to volume~(m$^{-1}$), $\rho_s$ is the density of steel~(kg/m$^3$), $c_s$ is the specific heat of steel~(J/kg-K), $h_c$ is the convective heat transfer coefficient~(W/m$^2$-K), $T_g$ is the hot gas layer temperature from the appropriate method described in Chapter~\ref{HGL:Chapter}~($^\circ$C), $T_s$ is the steel temperature~($^\circ$C), $\sigma$ is the Stefan-Boltzmann constant ($W/m^2$-$K^4$), $\epsilon$ is the emissivity~(-), and $\Delta t$ is the time step~(s).


\section{Protected Steel Temperature}
\label{info:protected_steel_temperature}

If the thermal capacity of the insulation layer is neglected, the temperature rise in the structural steel element can be calculated using the following equation

\be
\Delta T_s = k_i \left( \frac{T_p - T_s}{c_s h \frac{W}{D} + \frac{1}{2} c_i \rho_i h^2} \right) \Delta t
\label{eq:protected_steel}
\ee

In Eq.~\ref{eq:protected_steel}, $\Delta T_s$ is the temperature increase of the steel member~($^\circ$C), $k_i$ is the thermal conductivity of the insulation material~(W/m-K), $T_p$ is the plume temperature at a given height using the McCaffrey centerline plume temperature correlation given in Eq.~\ref{eq:McCaffrey}~($^\circ$C), $T_s$ is the steel temperature~($^\circ$C), $c_s$ is the specific heat of steel~(J/kg-K), $h$ is the thickness of the insulation~(m), $W/D$ is the ratio of the weight of steel section per linear foot to the heated perimeter~(kg/m$^2$), $c_i$ is the specific heat of the insulation material~(J/kg-K), $\rho_i$ is the density of the insulating material~(kg/m$^3$), and $\Delta t$ is the time step~(s).

The McCaffrey centerline plume temperature correlation is given by

\be
T_p = \left[ \left( \frac{\kappa}{0.9 \sqrt{2 g}} \right)^2 \left( \frac{Z}{\dot Q^{2/5}} \right)^{2 \eta - 1} T_\infty \right] + T_\infty
\label{eq:McCaffrey}
\ee

In Eq.~\ref{eq:McCaffrey}, $T_p$ is the plume centerline temperature~(K), $g$ is the acceleration of gravity~(m/s$^2$), $Z$ is the elevation above the fire source~(m), $\dot Q$ is the HRR~(kW), and $T_\infty$ is the ambient air temperature~(K). The constants $\eta$ and $\kappa$ are a function of the height $Z$ within the plume and are listed in Table~\ref{tbl:McCaffrey_constants}.

\vspace{\baselineskip}
\begin{table}[ht]
\begin{center}
\caption{Constants used in McCaffrey plume correlation}
\label{tbl:McCaffrey_constants}
\begin{tabular}{|c|c|c|c|}
\hline
Region & $z/\dot Q^{2/5}$      & $\eta$ & $\kappa$ \\
\hline
Continuous & < 0.08       & 1/2  & 6.8 \\
Intermittent & < 0.08-0.2 & 0     & 1.9 \\
Plume         & > 0.2         & -1/3 & 1.1 \\
\hline
\end{tabular}
\end{center}
\end{table}

\clearpage

\section{Summary of Surface Temperature}

Summary scatter plots of the surface temperature predictions are given on the following pages.

\begin{figure}[ht]
\begin{center}
\begin{tabular}{l}
\includegraphics[width=4.0in]{FIGURES/Scatterplots/Cable_Temperature}
\end{tabular}
\end{center}
\caption[Summary of time to threshold cable temperature predictions.]
{Summary of time to threshold cable temperature predictions.}
\label{Surface_Temperature_THIEF_Summary}
\end{figure}

\begin{figure}[p]
\begin{center}
\begin{tabular}{l}
\includegraphics[width=4.0in]{FIGURES/Scatterplots/Steel_Temperature_Unprotected} \\
\includegraphics[width=4.0in]{FIGURES/Scatterplots/Steel_Temperature_Protected}
\end{tabular}
\end{center}
\caption[Summary of steel temperature predictions.]
{Summary of unprotected (top) and protected (bottom) steel temperature predictions.}
\label{Surface_Temperature_Steel_Summary}
\end{figure}

