\documentclass[11pt]{book}
\usepackage{times,mathptm}
\usepackage[pdftex]{graphicx}
\usepackage[pdftex,
        colorlinks=true,
        urlcolor=linkblue,     % \href{...}{...} external (URL)
        citecolor=linkred,     % citation number colors
        linkcolor=linknavy,    % \ref{...} and \pageref{...}
        pdftitle={Fire Dynamics Simulator (Version 5) User's Guide},
        pdfauthor={Kevin McGrattan, Bryan Klein, Simo Hostikka, Jason Floyd},
        pdfsubject={User Guide},
        pdfkeywords={FDS, Fire Model, NIST, BFRL},
        pdfproducer={pdflatex},
        pagebackref,
        pdfpagemode=UseNone,
        bookmarksopen=true,
        plainpages=false]{hyperref}
\usepackage{color}
\definecolor{linknavy}{rgb}{0,0,0.50196}
\definecolor{linkred}{rgb}{1,0,0}
\definecolor{linkblue}{rgb}{0,0,1}
\usepackage{caption}
\usepackage{graphpap}
\usepackage{rotating}
\usepackage{epsfig,psfrag}
%\usepackage{wrapfig}
%\usepackage{picins}
\usepackage{geometry}
\usepackage{tabularx}
\usepackage{longtable}
\usepackage{lscape}
\usepackage{amssymb}
\usepackage{makeidx} % Create index at end of document
\usepackage[nottoc,notlof,notlot]{tocbibind} % Put the bibliography and index in the ToC
\usepackage{float}
\usepackage{lastpage} % Automatic last page number reference.
\usepackage[T1]{fontenc}
\usepackage{upquote}
\usepackage{array,eqnarray}
\newcommand{\nopart}{\expandafter\def\csname Parent-1\endcsname{}} % To fix table of contents in pdf.
\newcommand{\ct}{\tt\small}

% The Following commented code makes the ``Draft'' watermark on each page.
%\usepackage{eso-pic}
%\usepackage{type1cm}
%\makeatletter
%   \AddToShipoutPicture{
%     \setlength{\@tempdimb}{.5\paperwidth}
%     \setlength{\@tempdimc}{.5\paperheight}
%     \setlength{\unitlength}{1pt}
%     \put(\strip@pt\@tempdimb,\strip@pt\@tempdimc){
%     \makebox(0,0){\rotatebox{45}{\textcolor[gray]{0.75}{\fontsize{8cm}\selectfont{RC6}}}}}
% }
%\makeatother

\setlength{\textwidth}{6.5in}
\setlength{\textheight}{9.0in}
\setlength{\topmargin}{0.in}
\setlength{\headheight}{0.in}
\setlength{\headsep}{0.in}
\setlength{\parindent}{0.25in}
\setlength{\oddsidemargin}{0.0in}
\setlength{\evensidemargin}{0.0in}




\newcommand{\be}{\begin{equation}}
\newcommand{\ee}{\end{equation}}



\begin{document}
\bibliographystyle{unsrt}

\pagestyle{empty}
\pagenumbering{alph}

\begin{minipage}[t][9in][s]{6.25in}

\huge
\flushright{NIST Special Publication XXXX}

\vspace{1in}

\Huge \flushright{Fire Dynamics Simulator and Smokeview \\ Training Guide }

\vspace{.5in}

\normalsize

\large
\flushright{
Glenn Forney \\
Kevin McGrattan \\
Daniel Madrzykowski \\
Stephen Kerber \\
 }


\vfill

\flushright{\includegraphics[width=2.in]{FIGURES/nistident_flright_vec}}


\end{minipage}

\newpage
\hspace{5in}
\newpage

\begin{minipage}[t][9in][s]{6.25in}

\huge
\flushright{NIST Special Publication XXXX}

\vspace{.75in}

\Huge
\flushright{Fire Dynamics Simulator and Smokeview \\ Training Guide}

\vspace{.25in}

\normalsize
\flushright{
Glenn Forney \\
Kevin McGrattan \\
Daniel Madrzykowski \\
Stephen Kerber \\
{\em NIST Building and Fire Research Laboratory} \\
{\em Gaithersburg, Maryland, USA}}

\vspace{.25in}

\flushright{\today \\
FDS Version 5.2 \\
Smokeview Version 5.2 \\
$SVN Repository$~$Revision: 1815 $}

\vfill

\flushright{\includegraphics[width=1in]{FIGURES/doc.pdf} }

\small
\flushright{U.S. Department of Commerce \\
{\em Carlos M. Gutierrez, Secretary} \\
\hspace{1in} \\
National Institute of Standards and Technology \\
{\em James M. Turner, Acting Director} }


\end{minipage}

\newpage

\begin{minipage}[t][9in][s]{6.25in}

\flushright{Certain commercial entities, equipment, or materials may be identified in this \\
document in order to describe an experimental procedure or concept adequately. Such \\
identification is not intended to imply recommendation or endorsement by the \\
National Institute of Standards and Technology, nor is it intended to imply that the \\
entities, materials, or equipment are necessarily the best available for the purpose.
}

\vspace{3in}

\large
\flushright{\bf National Institute of Standards and Technology Special Publication XXXX \\
Natl.~Inst.~Stand.~Technol.~Spec.~Publ.~XXXX, \pageref{LastPage} pages (October 2008) \\
CODEN: NSPUE2 }

\vfill

\flushright{U.S. GOVERNMENT PRINTING OFFICE \\
WASHINGTON: 2007 \\
\rule{3.5in}{0.01in} \\
For sale by the Superintendent of Documents, U.S. Government Printing Office \\
Internet: bookstore.gpo.gov -- Phone: (202) 512-1800 -- Fax: (202) 512-2250 \\
Mail: Stop SSOP, Washington, DC 20402-0001 }
\end{minipage}

\clearpage

\frontmatter

\pagestyle{plain}
\pagenumbering{roman}


\chapter{Preface}

This Guide contains a description of the numerical simulation of a variety of full-scale fire experiments
using the Fire Dynamics Simulator and its visualization program, Smokeview. 

Note that this Guide does not provide the background theory for FDS nor a detailed description of Smokeview. The FDS
Technical Reference Guide~\cite{FDS_Tech_Guide_5} contains details about the governing
equations and numerical methods. The FDS User's Guide~\cite{FDS_Users_Guide_5} describes how to run FDS. Smokeview is described in the ``User's Guide for
Smokeview Version~5''~\cite{Smokeview_Users_Guide_5}.


\chapter{Disclaimer}

The US Department of Commerce makes no warranty, expressed or implied, to
users of the Fire Dynamics Simulator (FDS) or Smokeview, and accepts no responsibility for their
use. Users of FDS and Smokeview assume sole responsibility under Federal law for
determining the appropriateness of its use in any particular application;
for any conclusions drawn from the results of its use; and for any actions
taken or not taken as a result of analyses performed using these tools.

Users are warned that FDS and Smokeview are intended for use only by those competent in
the fields of fluid dynamics, thermodynamics, combustion, and heat transfer,
and is intended only to supplement the
informed judgment of the qualified user. The software package is a
computer model that may or may not have predictive capability when applied
to a specific set of factual circumstances. Lack of accurate predictions by
the model could lead to erroneous conclusions with regard to fire safety.
All results should be evaluated by an informed user.

Throughout this document, the mention of computer hardware or
commercial software does not constitute endorsement by NIST, nor does
it indicate that the products are necessarily those best suited for the
intended purpose.


\chapter{About the Authors}

\begin{description}
\item[Glenn Forney] is a computer scientist in the Building and Fire
Research Laboratory (BFRL) of NIST.
\item[Kevin McGrattan] is a mathematician in BFRL. He received a bachelors of science degree
from the School of Engineering and Applied Science of Columbia
University in 1987 and a doctorate at the Courant Institute of New
York University in 1991. He joined the NIST staff in 1992 and has
since worked on the development of fire models, most notably the Fire
Dynamics Simulator.
\item[Daniel Madrzykowski] is ...
\item[Stephen Kerber] is ...
\end{description}



\chapter{Acknowledgments}







\tableofcontents
\listoffigures
\listoftables

\mainmatter


\chapter{Introduction}

The software described in this document, Fire Dynamics Simulator (FDS), is a computational
fluid dynamics (CFD) model of fire-driven fluid flow. FDS solves numerically a form of the
Navier-Stokes equations\index{Navier-Stokes} appropriate for low-speed, thermally-driven flow
with an emphasis on smoke and heat transport from fires.




\bibliography{../Bibliography/FDS_refs,../Bibliography/FDS_general,../Bibliography/FDS_mathcomp}


\end{document}


