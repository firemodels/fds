\chapter{Training Tower}

\section{Training Objectives}

\section{Experimental Description}
\subsection{Geometry}

%\begin{figure}[\figoptions]
%\begin{center}
%\includegraphics[width=6.5in]{FIGURES/MFRI_tower_setup}
%\end{center}
%\caption {Schematic of the training tower layout}
%\label{figtowerplan}%
%\end{figure}

\begin{figure}[\figoptions]
\begin{center}
\includegraphics[width=5.0in]{scriptfigures/MFRItower_00_top}\\
a) top view\\
\includegraphics[width=5.0in]{scriptfigures/MFRItower_00_3D}\\
a) 3D perspective view\\
\end{center}
\caption {Schematic of training tower layout}
\label{figflashoverplan}%
\end{figure}
\subsection{Fire}

\subsection{Ventilation}

\subsection{Materials}

\section{Model Description}

\subsection{Geometry}

\subsection{Fire}

\subsection{Ventilation}

\subsection{Materials}

\section{Summary of Results}

\begin{figure}[\figoptions]
\begin{center}
\begin{tabular}{c}
 \includegraphics[width=3.0in]{datafigures/MFRI_Training_Tower_min_hrr}\\
 minimum HRR\\
 \\
 \includegraphics[width=3.0in]{datafigures/MFRI_Training_Tower_avg_hrr}\\
 average HRR\\
 \\
 \includegraphics[width=3.0in]{datafigures/MFRI_Training_Tower_max_hrr}\\
 maximum HRR\\
\end{tabular}
\end{center}
\caption {Experimental and FDS heat release results for the
Training Tower case}
\label{figtrainingtowerhrr}%
\end{figure}

\begin{figure}[\figoptions]
\begin{center}
\begin{tabular}{cc}
\includegraphics[width=3.0in]{datafigures/MFRI_Training_Tower_01_temp_all_layer_060}&
\includegraphics[width=3.0in]{datafigures/MFRI_Training_Tower_02_temp_all_layer_060}\\
Test 1 at 60 s&Test 2 at 60 s\\

\includegraphics[width=3.0in]{datafigures/MFRI_Training_Tower_01_temp_all_layer_120}&
\includegraphics[width=3.0in]{datafigures/MFRI_Training_Tower_02_temp_all_layer_120}\\
Test 1 at 120 s&Test 2 at 120 s\\

\includegraphics[width=3.0in]{datafigures/MFRI_Training_Tower_01_temp_all_layer_180}&
\includegraphics[width=3.0in]{datafigures/MFRI_Training_Tower_02_temp_all_layer_180}\\
Test 1 at 180 s&Test 2 at 180 s\\
\end{tabular}
\end{center}
\caption[Comparison of temperature as a function of elevation at 60 s, 120 s and 180 s for tests 1 and 2.]
{
Comparison of experimental and FDS generated temperatures for tests 1 and 2 at 60 s, 120 s and 180 s.
Temperatures were compared at 8 sensor locations spaced vertically  (0.3 m apart) located from floor to ceiling in the fire room.
The heat release rates used to generate the numerical results was obtained from Figure \ref{figtrainingtowerhrr}.
}
\label{figtrainingtemp12}%
\end{figure}

\begin{figure}[\figoptions]
\begin{center}
\begin{tabular}{cc}
\includegraphics[width=3.0in]{datafigures/MFRI_Training_Tower_03_temp_all_layer_060}&
\includegraphics[width=3.0in]{datafigures/MFRI_Training_Tower_04_temp_all_layer_060}\\
Test 3 at 60 s&Test 4 at 60 s\\

\includegraphics[width=3.0in]{datafigures/MFRI_Training_Tower_03_temp_all_layer_120}&
\includegraphics[width=3.0in]{datafigures/MFRI_Training_Tower_04_temp_all_layer_120}\\
Test 3 at 120 s&Test 4 at 120 s\\

\includegraphics[width=3.0in]{datafigures/MFRI_Training_Tower_03_temp_all_layer_180}&
\includegraphics[width=3.0in]{datafigures/MFRI_Training_Tower_04_temp_all_layer_180}\\
Test 3 at 180 s&Test 4 at 180 s\\
\end{tabular}
\end{center}
\caption[Comparison of temperature as a function of elevation at 60 s, 120 s and 180 s for tests 3 and 4.]
{
Comparison of experimental and FDS generated temperatures for tests 3 and 4 at 60 s, 120 s and 180 s.
Temperatures were compared at 8 sensor locations spaced vertically  (0.3 m apart) located from floor to ceiling in the fire room.
The heat release rates used to generate the numerical results was obtained from Figure \ref{figtrainingtowerhrr}.
}
\label{figtrainingtemp34}%
\end{figure}

\begin{figure}[\figoptions]
\begin{center}
\begin{tabular}{cc}
\includegraphics[width=3.0in]{datafigures/MFRI_Training_Tower_05_temp_all_layer_060}&
\includegraphics[width=3.0in]{datafigures/MFRI_Training_Tower_06_temp_all_layer_060}\\
Test 5 at 60 s&Test 6 at 60 s\\

\includegraphics[width=3.0in]{datafigures/MFRI_Training_Tower_05_temp_all_layer_120}&
\includegraphics[width=3.0in]{datafigures/MFRI_Training_Tower_06_temp_all_layer_120}\\
Test 5 at 120 s&Test 6 at 120 s\\

\includegraphics[width=3.0in]{datafigures/MFRI_Training_Tower_05_temp_all_layer_180}&
\includegraphics[width=3.0in]{datafigures/MFRI_Training_Tower_06_temp_all_layer_180}\\
Test 5 at 180 s&Test 6 at 180 s\\
\end{tabular}
\end{center}
\caption[Comparison of temperature as a function of elevation at 60 s, 120 s and 180 s for tests 5 and 6.]
{
Comparison of experimental and FDS generated temperatures for tests 5 and 6 at 60 s, 120 s and 180 s.
Temperatures were compared at 8 sensor locations spaced vertically  (0.3 m apart) located from floor to ceiling in the fire room.
The heat release rates used to generate the numerical results was obtained from Figure \ref{figtrainingtowerhrr}.
}
\label{figtrainingtemp56}%
\end{figure}

\begin{figure}[\figoptions]
\begin{center}
\begin{tabular}{c}
\includegraphics[width=3.0in]{datafigures/MFRI_Training_Tower_07_temp_all_layer_060}\\
Test 7 at 60 s\\

\includegraphics[width=3.0in]{datafigures/MFRI_Training_Tower_07_temp_all_layer_120}\\
Test 7 at 120 s\\

\includegraphics[width=3.0in]{datafigures/MFRI_Training_Tower_07_temp_all_layer_180}\\
Test 7 at 180 s\\
\end{tabular}
\end{center}
\caption[Comparison of temperature as a function of elevation at 60 s, 120 s and 180 s for test 7.]
{
Comparison of experimental and FDS generated temperatures for test 7 at 60 s, 120 s and 180 s.
Temperatures were compared at 8 sensor locations spaced vertically  (0.3 m apart) located from floor to ceiling in the fire room.
The heat release rates used to generate the numerical results was obtained from Figure \ref{figtrainingtowerhrr}.
}
\label{figtrainingtemp7}%
\end{figure}




\begin{figure}[\figoptions]
\begin{center}
\begin{tabular}{c}
\includegraphics[width=3.0in]{datafigures/MFRI_Training_Tower_01_temp_min}\\
FDS run with minimum HRR\\

\includegraphics[width=3.0in]{datafigures/MFRI_Training_Tower_01_temp_avg}\\
FDS run with average HRR\\

\includegraphics[width=3.0in]{datafigures/MFRI_Training_Tower_01_temp_max}\\
FDS run with maximum HRR\\
\end{tabular}
\end{center}
\caption[Comparison of temperature as a function of time for test 1 using 3 simulated HRR rates.] {
Comparison of experimental (test 1) and FDS generated temperatures for 3  simulated fire sizes as a function of time.
Temperatures were compared at 8 sensor locations spaced vertically  (0.3 m apart) located from floor to ceiling in the fire room.
The minimum, average and maximum heat release rate used to generate the numerical results was obtained from Figure \ref{figtrainingtowerhrr}.
}
\label{figtrainingtemp12}%
\end{figure}




\begin{figure}[\figoptions]
\begin{center}
\begin{tabular}{cc}
\includegraphics[width=3.0in]{datafigures/MFRI_Training_Tower_02_temp_avg}&
\includegraphics[width=3.0in]{datafigures/MFRI_Training_Tower_03_temp_avg}\\
Test 2&Test 3\\

\includegraphics[width=3.0in]{datafigures/MFRI_Training_Tower_04_temp_avg}&
\includegraphics[width=3.0in]{datafigures/MFRI_Training_Tower_05_temp_avg}\\
Test 4&Test 5\\

\includegraphics[width=3.0in]{datafigures/MFRI_Training_Tower_06_temp_avg}&
\includegraphics[width=3.0in]{datafigures/MFRI_Training_Tower_07_temp_avg}\\
Test 6&Test 7\\
\end{tabular}
\end{center}
\caption[Comparison of temperature as a function of time for tests 2 through 7.] {
Comparison of experimental and FDS generated temperatures for tests 2 through 7.
Temperatures were compared at 8 sensor locations spaced vertically  (0.3 m apart) located from floor to ceiling in the fire room.
The average heat release rate used to generate the numerical results was obtained from Figure \ref{figtrainingtowerhrr}.
}
\label{figtrainingtowertemp27}%
\end{figure}

\begin{figure}[\figoptions]
\begin{center}
\begin{tabular}{cc}
 \includegraphics[height=2.25in]{scriptfigures/MFRItower_000}&
 \includegraphics[height=2.25in]{scriptfigures/MFRItower_015}\\
a) 0 s&b) 15 s\\
 \includegraphics[height=2.25in]{scriptfigures/MFRItower_030}&
 \includegraphics[height=2.25in]{scriptfigures/MFRItower_060}\\
a) 30 s&b) 60 s\\
 \includegraphics[height=2.25in]{scriptfigures/MFRItower_120}&
 \includegraphics[height=2.25in]{scriptfigures/MFRItower_240}\\
a) 120 s&b) 240 s\\
\end{tabular}
\end{center}
\caption {Realistic visualizations of an FDS simulation of an experimental burn performed in the third floor of the MFRI training tower.}
\label{figtowersmoke}%
\end{figure}

\begin{figure}[\figoptions]
\begin{center}
\begin{tabular}{ccc}
 \includegraphics[height=2.25in]{scriptfigures/MFRItower_slice_temp_000}&
 \includegraphics[height=2.25in]{scriptfigures/MFRItower_slice_temp_015}\\
a) 0 s&b) 15 s\\
 \includegraphics[height=2.25in]{scriptfigures/MFRItower_slice_temp_030}&
 \includegraphics[height=2.25in]{scriptfigures/MFRItower_slice_temp_060}\\
a) 30 s&b) 60 s\\
 \includegraphics[height=2.25in]{scriptfigures/MFRItower_slice_temp_120}&
 \includegraphics[height=2.25in]{scriptfigures/MFRItower_slice_temp_240}\\
&&\raisebox{0.5in}[0pt]{\includegraphics[height=7.0in]{figures/colorbar_20_620}}\\
a) 120 s&b) 240 s\\
\end{tabular}
\end{center}
\caption {Shaded temperature slice visualizations of an FDS simulation of an experimental burn performed in the third floor of the MFRI training tower.}
\label{figtowersmoke}%
\end{figure}

