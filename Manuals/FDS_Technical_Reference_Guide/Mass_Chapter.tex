\chapter{Mass and Species Transport Equations}

A distinguishing feature of a CFD model is the regime of
flow speeds (relative to the speed of sound) for which it is designed. High
speed flow codes involve compressibility effects and shock waves. Low speed
solvers, however, explicitly eliminate compressibility effects that give rise
to acoustic (sound) waves. The Navier-Stokes equations describe the
propagation of information at speeds comparable to that of the fluid flow (for fire, 10-20~m/s),
but also at speeds comparable to that of sound waves (for still air,
300~m/s). Solving a discretized form of these equations would require extremely small
time steps in order to account for information traveling at the speed of sound, making
practical simulations difficult.

\section{The Low Mach Number Assumption}

Following the work of Rehm and Baum~\cite{Rehm:1}, an approximation to the equation of state~(\ref{basicstate}) is made by decomposing the pressure
into a ``background'' component and a perturbation. The original version of FDS assumed that the background component of the pressure
applied to the entire computational domain, most often a single compartment. Starting in FDS version 5, it is now assumed that
the background component of pressure can differ from compartment to compartment. If
a volume within the computational domain is isolated from other volumes, except via leak paths or ventilation ducts, it is referred to as a ``pressure
zone'' and assigned its own background pressure. The pressure within the $m$th zone, for example, is a linear combination
of its background component and the flow-induced perturbation:
\be p(\bx,t) = \bp_m(z,t) + \tp(\bx,t) \ee
Note that the background pressure is a function of $z$, the vertical spatial coordinate, and time. For most
compartment fire applications, $\bp_m$ changes very little with height or time. However, for situations where the pressure
increases due to a fire in a tightly sealed enclosure, or when the height of the domain is significant, $\bp_m$ takes these effects into
account~\cite{Baum:5}. The ambient pressure field is denoted $\bp_0(z)$. Note that the subscript 0 denotes the exterior of the computational domain, not
time 0. This is the assumed atmospheric pressure stratification that serves as both
the initial and boundary condition for the governing equations.

The purpose of decomposing the pressure is that for low-Mach number flows, it can be assumed that the temperature and density are inversely
proportional, and thus the equation of state (in the $m$th pressure zone) can be approximated
\be \bp_m  =  \rho T \R \sum_\alpha \frac{Z_\alpha}{W_\alpha} = \frac{\rho T \R}{ \bW }  \label{state} \ee
The pressure, $p$, in the state and energy equations is replaced by the background pressure $\bp_m$ to filter out sound waves
that travel at speeds that are much faster
than typical flow speeds expected in fire applications. The low Mach number assumption serves two purposes. First, the filtering of acoustic waves
means that the time step in the numerical algorithm is bound only by the flow speed as opposed to the speed of sound, and second, the modified state
equation leads to a reduction in the number of dependent variables in the system of equations by one. The energy equation (\ref{energy}) is never
explicitly solved, but its source terms are included in the expression for the flow divergence, to be derived presently.

The stratification of the atmosphere is derived from the relation
\be \frac{d \bp_0}{dz} = - \rho_0(z) \, g  \ee
where $\rho_0$ is the background density and $g=9.8$ m/s$^2$. Using Eq.~(\ref{state}), the background pressure can be written as a function of the background temperature, $T_0(z)$,
\be \bp_0(z) = p_\infty \; \exp \, \left( -\int^z_{z_\infty} \frac{\bW \, g}{\R \, T_0(z')} dz' \right)  \label{pstrat} \ee
where the subscript infinity generally refers to the ground. A linear temperature stratification of the atmosphere may be
specified by the user such that $T_0(z) = T_\infty + \Gamma z$ where $T_\infty$ is the temperature at the ground and
$\Gamma$ is the lapse rate (e.g., $\Gamma = -0.0098$~K/m is the {\em adiabatic lapse rate}).
In this case $\bp_0$ and $\rho_0$ are derived from Eqs.~(\ref{pstrat}) and (\ref{state}), respectively.
It can then be shown that for $\Gamma \ne 0$ the pressure stratification becomes
\be
   \bp_0(z) = p_\infty  \left( \frac{T_0(z)}{T_\infty} \right)^{\overline{W}g/\R \Gamma}
   \label{pstrat2}
\ee


\section{Combination of the Mass and Energy Equations via the Divergence}

Because of the low Mach number assumption, the divergence of the flow, $\nabla\!\cdot \bu$, plays a very important role in the overall solution scheme. The divergence is obtained by taking the material (substantial) derivative
of the modified Equation of State~(\ref{state}), and then substituting terms from the mass
and energy conservation equations. As shown in Appendix \ref{app_divergence}, for the $m$th zone with background pressure $\bp_m$, the divergence may be written as
\begin{equation}
\label{eqn_divfromeos}
\nabla\!\cdot \bu = {\cal D} - {\cal P}\; \dod{\bp_m}{t}
\end{equation}
where
\begin{equation}
\label{eqn_fdsP1}
\mathcal{P} = \frac{1}{\overline{p}_m}\left( 1 - \frac{\mathcal{R}}{\overline{W} c_p}  \right)
\end{equation}
and
\begin{align}
\label{eqn_fdsD1}
\mathcal{D} &= \frac{\dot{m}_b^\tripleprime}{\rho}\frac{\overline{W}}{\overline{W}_b} +
\frac{\overline{W}}{\rho} \sum_\alpha \Div \left( \rho D_\alpha \nabla[Z_\alpha/W_\alpha] \right) +
\frac{1}{\rho}\sum_\alpha \left( \frac{\overline{W}}{W_\alpha} -
\frac{h_{s,\alpha}}{c_p T} \right) \dot{m}_{\alpha}^\tripleprime  + \mathcal{P} w \rho g \nonumber  \\
&+ \frac{\mathcal{R}}{\overline{W} c_p \overline{p}_m} \left[ \dot{q}^\tripleprime - \dq_b''' -
\Div \dot{\mathbf{q}}^{\prime\prime} - \sum_\alpha h_{s,\alpha} \Div \rho D_\alpha \nabla Z_\alpha +
\dot{m}_{b}^{\tripleprime} \sum_\alpha Z_{b,\alpha} c_{p,\alpha}(T_b-T)  \right]
\end{align}
Contributions to the divergence of the flow include the heat release rate of the fire, $\dq'''$, heat losses to evaporating droplets,
$\dq_b'''$, the net heat flux from thermal conduction and radiation, $\nabla\!\cdot \dbq''$, updrafts of air over considerable heights of
the atmosphere, the net mass flux from gas species diffusion and production, and global pressure changes. The change in the background
pressure with time, $\partial \bp_m/\partial t$, is non-zero only if
it assumed that the compartment is tightly sealed, in which case the background pressure, $\bp_m$, can no longer be assumed constant due to
the increase (or decrease) in mass and thermal energy within the enclosure. The time derivative of the background pressure of the $m$th
pressure zone, $\Omega_m$, is found by integrating Eq.~(\ref{eqn_divfromeos}) over the zone volume:
\begin{equation}
\dod{\bp_m}{t} = \left( \int_{\Omega_m} {\cal D} \, dV - \int_{\partial \Omega_m} \bu \cdot d\bS \right) \Big/ \int_{\Omega_m} {\cal P} \, dV  \label{concon}
\end{equation}
Equation~(\ref{concon}) is essentially a consistency condition, ensuring that blowing air or starting a fire within a sealed
compartment leads to an appropriate decrease in the divergence within the volume.


\clearpage
\section{Numerical Method}

Due to the use of the low Mach number approximation, the mass and energy equations are combined through the divergence. The divergence of the flow field contains many of the fire-specific source terms described above.

\subsection{Discretizing the Convective and Diffusive Transport Terms}

The density at the center of the $ijk$th cell is updated in time with the following predictor-corrector scheme. Advection terms are written in ``flux divergence'' form. In the predictor step, the density at the $(n+1)$st time level is estimated based on information at the $n$th level
\be  \frac{\rho_{ijk}^{*}-\rho_{ijk}^n}{\dt}
    + \nabla\!\cdot(\overline{\rho}^{FL} \mathbf{u})_{ijk}^n = 0
\ee
The quantity $\overline{\rho}^{FL}$ indicates a \emph{flux limiter} applied to the cell face value, as discussed below in Section \ref{sec_flux_limiters}.

Following the prediction of the velocity and background pressure at the $(n+1)$st time level, the density is corrected via
\be \frac{\rho_{ijk}^{n+1}-\ha\left(\rho_{ijk}^n
     +\rho_{ijk}^{*}\right)} {\ha \dt}
    + \nabla\!\cdot(\overline{\rho}^{FL} \mathbf{u})_{ijk}^{*}
    = 0 \ee
The species conservation equations are differenced the same way, with the addition of the diffusion term (including turbulent diffusion):
\be  \frac{(\rho Z_\alpha)_{ijk}^{*}-(\rho Z_\alpha)_{ijk}^n}{\dt}
  + \nabla\!\cdot(\overline{\rho Z_\alpha}^{FL} \mathbf{u})_{ijk}^n
  = \nabla\!\cdot (\rho D_\alpha \nabla Z_\alpha)_{ijk}^n \ee
at the predictor step, and
\be \frac{(\rho Z_\alpha)_{ijk}^{n+1}-\ha\left((\rho Z_\alpha)_{ijk}^n
     +(\rho Z_\alpha)_{ijk}^{*}\right)} {\ha \dt}
    + \nabla\!\cdot(\overline{\rho Z_\alpha}^{FL} \mathbf{u})_{ijk}^*
    = \nabla\!\cdot (\rho D_\alpha \nabla Z_\alpha)_{ijk}^{*} \ee
at the corrector step.

Mass source terms due to chemistry, evaporation, or pyrolysis are time split and applied after the corrector step (see Section \ref{sec_time_splitting}).


\subsection{Flux Limiters}
\label{sec_flux_limiters}

A \emph{flux limiter} is a form of interpolation scheme which depends on the local state of the flow field and scalar data. Simple linear interpolation of the cell-centered scalar data to the cell face would result in a central differencing scheme.  Such purely centered schemes are known to generate intolerable levels of dispersion error (spurious wiggles) leading to unphysical results such as negative densities or mass fractions outside the range of [0,1].  To address this issue, FDS has relied on a \emph{flux correction} scheme (see Appendix \ref{app_boundedness}) which adds a sufficient amount of numerical diffusion to maintain boundedness.  There is, however, more to the problem.

For uniform flow velocity, a fundamental property of the exact solution to the equations governing scalar transport is that the total variation of the scalar field (the sum of the absolute values of the scalar differences between neighboring cells) is preserved or diminished (never increased).  In other words, no new extrema are created.  Numerical schemes which preserve this property are referred to as total variation diminishing (TVD) schemes.  The practical importance of using a TVD scheme for fire modelling is that such a scheme is able to accurately track coherent vortex structure in turbulent flames and does not develop spurious reaction zones.

FDS employs two popular second-order TVD schemes as options for scalar transport: Superbee and CHARM.  Superbee \cite{Roe:1986} is recommended for LES because it more accurately preserves the scalar variance for coarse grid solutions which are not expected to be smooth.  Due to the gradient steepening applied in Superbee, however, the convergence degrades at small grid spacing for smooth solutions (the method will revert to a stair-step pattern instead of the exact solution).  CHARM \cite{Zhou:1995}, though slightly more dissipative than Superbee, is convergent, and is therefore the better choice for DNS calculations where the flame front is well resolved.

When a flux limiter is chosen for scalar transport (set {\ct FLUX\_LIMITER=1-4} on {\ct MISC}; 2 by default for LES, 4 for DNS), FDS formulates the density and species advection terms in ``flux divergence'' form.  For example, the predictor step of the continuity equation is discretized as
\be  \frac{\rho_{ijk}^{(n+1)_e}-\rho_{ijk}^n}{\dt}
    + \nabla\!\cdot(\overline{\rho}^{FL} \mathbf{u})_{ijk}^n = 0
\ee
In 1D, we would have
\be  \frac{\rho_{i}^{(n+1)_e}-\rho_{i}^n}{\dt}
    + \frac{\overline{\rho}^{FL}_{i+\frac{1}{2}} u_{i+\frac{1}{2}} - \overline{\rho}^{FL}_{i-\frac{1}{2}} u_{i-\frac{1}{2}}}{\dx} = 0
\ee
Note that the `1/2' indicates a face value for a particular cell ($i,j,k$). A flux-limited scalar value (density in this case) premultiplies the staggered, face-centered velocity to form the scalar advective flux.

Consider face $i+\frac{1}{2}$ between cells $i$ and $i+1$ and let $\phi$ denote a general scalar variable.  The local ($loc$) and upstream ($up$) data variations are
\begin{eqnarray}
\delta \phi_{loc} &=& \phi_{i+1}-\phi_i \nonumber\\
\delta \phi_{up}  &=& \left\{ \begin{array}{ll} \phi_i-\phi_{i-1} & \mbox{if} \quad u_i>0 \\ \phi_{i+2}-\phi_{i+1} & \mbox{if} \quad u_i<0 \end{array} \right. \nonumber
\end{eqnarray}
The limiter function $B(r)$ depends on the upstream-to-local data ratio, $r=\delta \phi_{up}/\delta \phi_{loc}$ \cite{Toro}.  In FDS, options for this function are:
\begin{enumerate}
\item[]{\tt FLUX\_LIMITER=0\,\,\,} Central Differencing  \begin{equation} B(r) = 1 \end{equation}
\item[]{\tt FLUX\_LIMITER=1\,\,\,} First-order Upwinding (Godunov's Scheme) \begin{equation} B(r) = 0 \end{equation}
\item[]{\tt FLUX\_LIMITER=2\,\,\,} Superbee (recommended for LES) \begin{equation} B(r) = \max(0,\min(2r,1),\min(r,2)) \end{equation}
\item[]{\tt FLUX\_LIMITER=3\,\,\,} MINMOD \begin{equation} B(r) = \max(0,\min(1,r)) \end{equation}
\item[]
Once $B(r)$ has been determined, the scalar face value is found from
\begin{equation}
\label{eqn_flux_limiter}
\overline{\phi}^{FL}_{i+1/2} = \left\{ \begin{array}{lcll} \phi_i &+& B(r) \,\frac{1}{2}(\phi_{i+1}-\phi_i) & \mbox{if} \quad u_i>0 \vspace{0.2 cm}\\
\phi_{i+1} &+& B(r) \,\frac{1}{2}(\phi_i-\phi_{i+1}) & \mbox{if} \quad u_i<0 \end{array} \right.
\end{equation}
\item[]{\tt FLUX\_LIMITER=4\,\,\,} CHARM
\item[] For this limiter the FDS implementation uses the reciprocal definition of the data ratio, $r = \delta C_{loc}/\delta C_{up}$.  The limiter function is given by \cite{Zhou:1995,Kempf:2003}
\begin{equation}
B(r) = \frac{r(3r+1)}{(r+1)^2}
\end{equation}
and the scalar face value is then determined from
\begin{equation}
\label{eqn_charm_limiter}
\overline{\phi}^{FL}_{i+1/2} = \left\{ \begin{array}{lcll} \phi_i &+& B(r) \,\frac{1}{2}(\phi_i-\phi_{i-1}) & \mbox{if} \quad u_i>0 \vspace{0.2 cm}\\
\phi_{i+1} &+& B(r) \,\frac{1}{2}(\phi_{i+1}-\phi_{i+2}) & \mbox{if} \quad u_i<0 \end{array} \right.
\end{equation}
\item[]{\tt FLUX\_LIMITER=5\,\,\,} Monotonicity-Preserving 5th-Order (MP5)
\item[] The MP5 scheme of Suresh and Huynh \cite{Suresh:1997} is based on the keen observation that three points cannot distinguish between extrema and discontinuities.  The functional form of the limiter is not a simple as the three-point schemes described above, so we refer the reader to the original paper or the FDS source code for details.  But the basic idea behind the method is to use a five-point stencil, three upwind and two downwind, to reconstruct the cell face value, considering both accuracy and monotonicity-preserving constraints.  An additional benefit of the MP5 scheme is that it was designed specifically with strong stability-preserving (SSP) Runge-Kutta time discretizations in mind.  The predictor-corrector scheme used by FDS is similar to the second-order SSP scheme described in \cite{Gottlieb:2001}.
\end{enumerate}

\subsection*{Notes on Implementation}

In practice, we set $r=0$ initially and only compute $r$ if the denominator is not zero.  Note that for $\delta \phi_{loc}=0$ it does not matter which limiter (0-3) is used: all the limiters yield the same scalar face value.  For CHARM, we set both $r=0$ and $B=0$ initially and only compute $B$ if $r>0$ (this requires data variations to have the same sign), else CHARM reduces to Godunov's scheme.

Central differencing ({\tt FLUX\_LIMITER=0}), Godunov's scheme ({\tt FLUX\_LIMITER=1}), and MINMOD ({\tt FLUX\_LIMITER=3}) are essentially included for completeness, debugging, and educational purposes.  These schemes have little utility in practice.

\subsection{Time Splitting for Mass Source Terms}
\label{sec_time_splitting}
After the corrector step for the transport scheme, source terms are applied to the scalars.  The source terms are evaluated using the results from the corrected scalar transport scheme (denoted with an asterisk *):
\be
\frac{(\rho Z_\alpha)_{ijk}^{n+1}-(\rho Z_\alpha)_{ijk}^*}{\dt} =  \dm_{\alpha,ijk}'''(\mathbf{Z}^*,T^*)
\ee


\subsection{Discretizing the Divergence}
\label{div_discret}

The divergence (see Eq.~(\ref{eqn_divfromeos})) in the $m$th pressure zone in both the predictor and corrector step
is discretized
\be (\nabla\!\cdot \bu)_{ijk} = \frac{\R}{\bW c_p \bp_m} \left( \dq_{ijk}''' + (\nabla\!\cdot k \nabla T)_{ijk}
    + \ldots \right) + \frac{1}{\bp_n} \left( \frac{\R}{\bW c_p} -1 \right)
      \left( \dod{\bp_m}{t} - w_{ijk} \rho_{0,k} g \right) \label{divdis} \ee
The thermal and material diffusion terms are pure central differences,
with no upwind or downwind bias, thus they are differenced the same
way in both the predictor and corrector steps. For example, the thermal
conduction term is differenced as follows:
\begin{eqnarray}
(\nabla\!\cdot k \nabla T)_{ijk} &=&
              \frac{1}{\dx}
         \left[k_{i+\ha,jk}\frac{T_{i+1,jk}-T_{ijk}}{\dx}
              -k_{i-\ha,jk}\frac{T_{ijk}-T_{i-1,jk}}{\dx}\right]+  \nonumber \\
            &&\frac{1}{\dy}
         \left[k_{i,j+\ha,k}\frac{T_{i,j+1,k}-T_{ijk}}{\dy}
              -k_{i,j-\ha,k}\frac{T_{ijk}-T_{i,j-1,k}}{\dy}\right]+ \nonumber \\
            &&\frac{1}{\dz}
         \left[k_{ij,k+\ha}\frac{T_{ij,k+1}-T_{ijk}}{\dz}
              -k_{ij,k-\ha}\frac{T_{ijk}-T_{ij,k-1}}{\dz}\right]
\end{eqnarray}
The temperature is extracted from the density via the equation of state
\be T_{ijk} = \frac{\bp_m}{\rho_{ijk} {\cal R}\, \sum_{\alpha=0}^{N_s} (Z_{\alpha,ijk}/W_\alpha)}\ee

Because only species 1 through $N_s$ are explicitly computed

\be Z_0 =  1 -  \sum_{\alpha=0}^{N_s} (Z_{\alpha,ijk} \ee

The molecular weight, $\bW_\alpha$, of the lumped species $Z_\alpha$ is computed as

\be \bW_\alpha =  \sum_{n=0}^{N_s} \frac{Y_n}{W_n} \ee

where $Y_n$ is the mass fraction of species $n$ in the lumped species $Z_\alpha$.

In isothermal calculations involving multiple species, the density
can be extracted from the average molecular weight
\be \rho_{ijk} =  \frac{ p_m}{T_\infty {\cal R } \bW } =
   \frac{W_0\,p_m}{T_\infty {\cal R }} + \sum_{\alpha=1}^{N_s} \left( 1-\frac{W_0}{W_\alpha} \right)
   (\rho Z_\alpha)_{ijk} \ee


To describe how the background pressure of the $m$th pressure zone, $\bp_m$, is updated in time, consider the expression for the
divergence written in compact notation:
\begin{equation}
\label{eqn_divfromeos2}
\nabla\!\cdot \bu = {\cal D} - {\cal P}\; \dod{\bp_m}{t}
\end{equation}
The terms $\mathcal{D}$ and $\mathcal{P}$ are defined by Eqs.~(\ref{eqn_fdsD1}) and (\ref{eqn_fdsP1}), respectively. The subscript $m$ refers to the
number of the {\em pressure zone}; that is, a volume within the computational domain that is allowed to have its own background pressure rise. A closed room
within a building, for example, is a pressure zone.
The time derivative of the background pressure of the $m$th
pressure zone is found by integrating Eq.~(\ref{eqn_divfromeos2}) over the zone volume (denoted by $\Omega_m$):
\begin{equation}
\dod{\bp_m}{t} = \left( \int_{\Omega_m} {\cal D} \, dV - \int_{\partial \Omega_m} \bu \cdot d\bS \right) \Big/ \int_{\Omega_m} {\cal P} \, dV  \label{concon2}
\end{equation}
Equation~(\ref{concon2}) is essentially a consistency condition, ensuring that blowing air or starting a fire within a sealed
compartment leads to an appropriate decrease in the divergence within the volume.


\subsubsection{Combining Pressure Zones}

In the event that a barrier separating two pressure zones should rupture, Eq.~(\ref{concon2}) is modified so that the pressure in the
newly connected zones is driven towards an equilibrium pressure:
\be
  \bp_{eq} = \sum_m \left( \bp_m \int_{\Omega_m} {\cal P} \, dV  \right)  \Big/  \sum_m \int_{\Omega_m} {\cal P} \, dV \approx \frac{ \sum_m V_m }{ \sum_m (V_m/\bp_m) }
\ee
Note that
\be
  \int_{\Omega_m} {\cal P} \, dV \approx  \frac{ V_m}{\gamma \bp_m }
\ee
To drive the pressure within the connected zones towards each other, a volume flow, $\dot{V}_m^*$, is applied to each zone. This flow is intended to move gas
from zones with the higher pressures towards zones with lower pressures. Eq.~(\ref{concon2}) now becomes:
\be
   \dod{\bp_{eq}}{t} - r \, \frac{ \bp_m - \bp_{eq} }{ \dt} = 
   \left( \int_{\Omega_m} {\cal D} \, dV - \int_{\partial \Omega_m} \bu \cdot d\bS - \dot{V}_m^* \right) \Big/ \int_{\Omega_m} {\cal P} \, dV
\ee
This equation is solved for $\dot{V}_m^*$. 
The first term on the left is the change in the equilibrium pressure with time:
\be \dod{\bp_{eq}}{t} = \left( \sum_m \int_{\Omega_m} {\cal D} \, dV - \sum_m \int_{\partial \Omega_m} \bu \cdot d\bS \right) \Big/ \sum_m \int_{\Omega_m} {\cal P} \, dV \ee
The summation is over all connected zones, and it is essentially the net change in pressure with time for the entire connected region. If there is any opening to the
exterior of the computational domain, this term is set to zero and all connected zone pressures are driven towards ambient.
The second term on the left forces the pressure in the $m$th pressure zone towards the equilibrium. 
The constant, $r$, is a relaxation factor of 0.2, a somewhat arbitrary value designed to slow down the fairly rapid pressure equilibration.


\subsection{Enthalpy and Specific Heat}
\label{HandCp}

The sensible enthalpy is a function of the temperature:
\be
  h_s = \sum_\alpha Z_\alpha h_{s,\alpha} \quad;\quad  h_{s,\alpha}=\sum_n Y_n h_{s,n}  \quad; \quad h_{s,n}(T)=\int_{T_0}^T c_{p,n}(T') \,\mbox{d}T'
\ee
The values for $h_{s,n}$ and $c_{p,n}$ are obtained by table lookup from the NIST-JANAF tables~\cite{NIST_JANAF}.
The values are taken to the nearest degree Kelvin.


\subsection{Coupling the Gas and Solid Phase}

Gas phase temperatures are defined at cell centers; solid surfaces lie at the
interface of the bordering gas phase cell and a ``ghost'' cell inside the
solid. As far as the gas phase calculation is concerned,
the normal temperature gradient at the surface is expressed in terms of
the temperature difference between the ``gas'' cell and the ``ghost'' cell.
The solid surface temperature is not used directly in the gas phase calculation.
Rather, the ghost cell temperature is used to couple the gas and solid phases.
The ghost cell temperature
has no physical meaning on its own. It is purely a numerical construct. It does
not represent the temperature within the wall, but rather establishes
a temperature gradient at the solid surface consistent with the empirical
correlation. Only the difference between ghost and
gas cell temperatures matters, for this defines the heat transfer to the
wall.

In a DNS calculation, the solid surface temperature is assumed to be an average of the
ghost cell temperature and the temperature of the first cell in the gas,
thus the ghost cell temperature is defined
\be T_{ghost} =  2 T_s - T_{gas}  \ee
For an LES calculation, the numerical expression for the heat lost to the boundary is
equated with the empirical convective heat transfer
\be k_{\hbox{\tiny LES}} \frac{T_{gas} - T_{ghost}}{\dn} - \bar{\rho} \, u_n \bar{c}_p (\bar{T}) \, \bar{T}
   = h \; (T_{gas}-T_s) - \bar{\rho} \, u_n  \, \bar{c}_p(T_s) \, T_s  \label{ebal} \ee
where $\dn$ is the distance between the center of the ghost cell
and the center of the gas cell, and the bar over the $T$ and $\rho$ indicate the average of the gas and ghost values:
\be \overline{T} = \frac{T_{gas}+T_{ghost}}{2}  \quad ; \quad \overline{\rho} = \frac{\rho_{gas} + \rho_{ghost}^*}{2} \ee
The specific heat is defined:
\be \bar{c}_p (T) = \frac{1}{T} \int_0^T c_p(T') \, dT' \ee
Equation~(\ref{ebal}) is solved for $T_{ghost}$,
so that when the conservation equations are updated, the amount of heat
lost to the wall is equivalent to the empirical expression on the right
hand side. Note that the asterisk in the equations above denotes that the value is taken from the previous
time step.

At solid walls there is no transfer of mass, thus the boundary condition
for the $l$th species at a wall is simply
\be Z_{l,ghost} = Z_{l,gas} \ee
where the subscripts ``ghost'' and ``gas'' are the same as above since
the mass fraction, like temperature, is defined at cell centers.
At forced flow boundaries either the mass fraction $Y_{l,w}$ or
the mass flux $\dot{m}_l''$ of species $l$ may be prescribed.
Then the ghost cell mass fraction can be derived because, as with
temperature, the normal gradient of mass fraction is needed in the gas phase
calculation.
For cases where the mass fraction is prescribed
\be Z_{l,ghost} = 2 Z_{l,w} - Z_{l,gas}  \ee
For cases where the mass flux is prescribed,
the following equation must be solved iteratively
\be \dot{m}_l'' = u_n \frac{\rho_{ghost} Z_{l,ghost} + \rho_{gas} Z_{l,gas}}{2}
  - \rho D \frac{Z_{l,gas}-Z_{l,ghost}}{\dn}
  \mp \frac{\dt \, u_n^2}{2} \frac{\rho_{gas} Z_{l,gas}
   -\rho_{ghost} Z_{l,ghost}}{\dn} \ee
where $\dot{m}_l''$ is the mass flux of species $l$ per unit area,
$u_n$ is the normal component of velocity at the wall pointing into
the flow domain, and $\dn$ is the distance between the center of the ghost
cell and the center of the gas cell. Notice that the last term on the
right hand side is subtracted at the predictor step and added at the
corrector step, consistent with the biased upwinding introduced earlier.

Once the temperature and species mass fractions have been defined in the
ghost cell, the density in the ghost cell is computed from the equation of
state
\be  \rho_{ghost} = \frac{p_0}{ {\cal R} \, T_{ghost} \, \sum_l (Z_{l,ghost}/W_l) }  \ee



\subsection{Mass and Energy Transfer at Interpolated Mesh Boundaries}

In simulations involving more than one numerical mesh, information has to be passed between meshes, even when
the meshes are being processed by separate computers. If two meshes abut each other, and the mesh cells are aligned and the same size, then
one mesh simply uses the density and species mass fractions of the adjacent mesh as the ``ghost'' cell values. However, in cases where the
mesh cells are not the same size, the exchange of information must be done more carefully. Consider a case where two meshes meet:

\begin{picture}(200,110)(0,-10)
\setlength{\unitlength}{0.02in}
\put(120,10){\framebox(20,20){ }}
\put(120,30){\framebox(20,20){ }}
\put(140,10){\framebox(40,40){ }}
\put(100,30){\makebox(0,0){Mesh 1}}
\put(200,30){\makebox(0,0){Mesh 2}}
\thicklines
\put(140,0){\line(0,1){60}}
\end{picture}

\noindent
We want the total and species mass fluxes between meshes to be the same, or as close as possible.
Let the density in cell $(1,j',k')$ of Mesh 2 be denoted $\rho_{1,j'k'}^{(2)}$. Assume that this cell abuts four cells in Mesh 1. The densities in the four abutting cells
of Mesh 1 are denoted $\rho_{I,jk}^{(1)}$. Note that $j$ and $k$ are not the same as $j'$ and $k'$. $I$ is the number of cells in the $x$ direction of Mesh 1. The
ghost cell quantities in Mesh 1 have an $i$ index of $I+1$. The ghost cell quantities in Mesh 2 have an $i$ index of 0.
We want to assert mass conservation at the mesh interface:
\be
   \sum_{j,k} u_{I,jk}^{(1)} \; \frac{\rho_{I+1,jk}^{(1)}+\rho_{I,jk}^{(1)}}{2} \; \dy^{(1)} \, \dz^{(1)}  =
              u_{0,j'k'}^{(2)} \; \frac{\rho_{1,j'k'}^{(2)}+\rho_{0,j'k'}^{(2)}}{2} \; \dy^{(2)} \, \dz^{(2)}  \label{rhou}
\ee
When solving for $\rho_{0,j'k'}^{(2)}$, the ghost cell value for Mesh 2, we have to assume that the ghost cells values for Mesh 1 are simply linearly interpolated
from the gas phase values of Mesh 1 and Mesh 2:
\be \rho_{I+1,jk}^{(1)} = \rho_{I,jk}^{(1)} + \frac{2 \, \dx^{(1)}}{\dx^{(1)}+\dx^{(2)}} \left( \rho_{1,j'k'}^{(2)} - \rho_{I,jk}^{(1)} \right)  \label{ghostrho} \ee
Rearranging terms in Eq.~(\ref{rhou}) and using the expression for the ghost cells from Eq.~(\ref{ghostrho}), we get:
\be \rho_{0,j'k'}^{(2)} = -\rho_{1,j'k'}^{(2)} + \frac{1}{u_{0,j'k'}^{(2)} \dy^{(2)} \, \dz^{(2)} } \sum_{j,k}  u_{1,jk}^{(1)} \dy^{(1)} \, \dz^{(1)}
    \left[ 2 \, \rho_{I,jk}^{(1)} +  \frac{2 \, \dx^{(1)}}{\dx^{(1)}+\dx^{(2)}}  \left( \rho_{1,j'k'}^{(2)} - \rho_{I,jk}^{(1)} \right)  \right]  \ee
