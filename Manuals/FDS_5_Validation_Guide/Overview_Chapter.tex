
\chapter{What is Model Validation?}

This supplement to the Fire Dynamimcs Simulator (FDS) Technical Reference Guide~\cite{FDS_Tech_Guide_5}
is a compilation of past and present validation studies of FDS. Although there are various definitions of model validation, for example in
ASTM E 1355~\cite{ASTM:E1355}, most define it as the process of determining how well the mathematical model predicts the actual physical phenomena of
interest.
Validation typically involves (1) comparing model predictions with experimental measurements, (2) quantifying the differences in light of uncertainties in
both the measurements and the model inputs, and (3) deciding if the model is appropriate for the given application. This Guide only does (1) and (2). Number (3) is
the responsibility of the model user.

A common question asked of any mathematical model is whether it is validated. To say that FDS is
``validated'' means that the model has been shown to be of a given level of accuracy for a given range of parameters for a given
type of fire scenario. Although the FDS developers continuously perform validation studies, it is ultimately the end user of the model who
decides if the model is adequate for the job at hand. Thus, this Guide provides the raw material for a validation study, but it does not
and cannot be considered comprehensive.

The following sections discuss key issues that you must consider when deciding whether or not FDS has been validated. It depends on (a) the scenarios
of interest, (b) the predicted quantities, and (c) the desired level of accuracy. Keep in mind that FDS can be used to model most any fire scenario and predict almost
any quantity of interest, but the prediction may not be accurate because of limitations in the description of the fire physics, and also because of limited
information about the fuels, geometry, and so on.



\section{Model Scenarios}

When doing a validation study, the first question to ask is, ``What is the application?'' There are countless fire scenarios to consider, but from the
point of view of validation it is useful to divide them into two classes -- those for which the fire is {\em specified} as an input to the model and those for which the fire
is {\em predicted} by the model. The former is often the case for a design application, the latter for a forensic reconstruction. Consider each in turn.

Design  applications  typically  involve  an existing  building  or  a
building  under  design. A  so-called  ``design  fire'' is  specified
either by  a regulatory authority  or by the engineers  performing the
analysis. Because the  fire's heat release rate is  specified, the role of
the model is to predict  the transport of heat and combustion products
throughout  the room or  rooms of  interest. Ventilation  equipment is
often included  in the simulation, like fans,  blowers, exhaust hoods,
HVAC ducts,  smoke management systems,  {\em etc.} Sprinkler  and heat
and smoke detector activation are also of interest.  The effect of the
sprinkler spray on the fire is usually less of interest since the fire
is  prescribed rather  than  predicted. Detailed  descriptions of  the
contents of the building are usually not necessary because these items
are not assumed to be burning,  and even if they are, the burning rate
will be  fixed, not predicted.  Sometimes, it is necessary  to predict
the heat  flux from the fire  to a nearby ``target,''  and even though
the target  may heat up  to some prescribed ignition  temperature, the
subsequent spread  of the  fire usually goes  beyond the scope  of the
analysis because of the uncertainty  inherent in object to object fire
spread.

Forensic reconstructions require the  model to simulate an actual fire
based on  information that is collected  after the event,  such as eye
witness accounts, unburned materials,  burn signatures, {\em etc.} The
purpose  of  the simulation  is  to  connect  a sequence  of  discrete
observations  with  a continuous  description  of  the fire  dynamics.
Usually,  reconstructions  involve  more gas/solid  phase  interaction
because  virtually  all  objects  in  a  given  room  are  potentially
ignitable, especially when flashover  occurs. Thus, there is much more
emphasis on  such phenomena as  heat transfer to  surfaces, pyrolysis,
flame  spread, and suppression.  In general,  forensic reconstructions
are more challenging simulations  to perform because they require more
detailed  information  about the  room  contents,  and  there is  much
greater uncertainty in the total heat release rate as the fire spreads
from object to object.

Validation  studies  of FDS  to  date  have  focussed more  on  design
applications   than  reconstructions.  The   reason  is   that  design
applications usually involve prescribed  fires and demand a minimum of
thermophysical properties  of real materials.  Transport  of smoke and
heat  is  the  primary  focus,  and measurements  can  be  limited  to
well-placed thermocouples, a few  heat flux gauges, gas samplers, {\em
etc.} Phenomena of importance in forensic reconstructions, like second
item  ignition, flame  spread, vitiation  effects and  extinction, are
more   difficult  to   model  and   more  difficult   to   study  with
well-controlled experiments. Uncertainties  in material properties and
measurements, as  well as simplifying assumptions in  the model, often
force the  comparison between model and measurement  to be qualitative
at best.  Nevertheless, current validation  efforts are moving  in the
direction of these more difficult issues.




\section{Model Outputs}

For a given fire scenario, there are a number of different quantities that the model predicts, like gas temperature, heat flux, and so on. A typical
fire experiment can produce hundreds of time histories of point measurements, each of which can be reproduced by the model to some level of accuracy.
It is a challenge to sort out all the plots and graphs of all the different quantities and come to some general conclusion. For this reason, this Guide
is organized by output quantity, not by individual experiment or fire scenario. In this way, it is possible to assess, over a range of different experiments
and scenarios, the performance of the model in predicting a given quantity. Overall trends and biases become much more clear when the data is organized this way.

Keep in mind that for any fire experiment, FDS might predict something very well, say within the experimental uncertainty bounds, but something else not well by
any measure. For example, in the a series of 15 full-scale fire experiments conducted at NIST in 2003, sponsored by the U.S.~Nuclear Regulatory Commission, the 
average hot gas layer (HGL) temperature predictions were within the accuracy of the experiments themselves, yet the smoke concentration predictions differed from the
measurements by as much as a factor of 3. Why? Consider the following issues associated with various types of measurements:
\begin{itemize}
\item Is the measurement taken at a single point, or averaged over many points? In the example above, the HGL temperature is an average of many
thermocouple measurements, whereas the smoke concentration is based on the extinction of laser light over a short length span. Model error tends to be
reduced by the averaging process, plus most fire models, including FDS, are based on global mass and energy conservation laws that are expressed as 
spatial averages.
\item Is the measured quantity time-averaged or instantaneous? For example, a surface temperature prediction is less prone to error in comparison to a 
heat flux prediction because the former is, in some sense, a time-integral of the latter.
\item In the case of a point measurement, how close to the fire is it? The terms ``near-field'' and ``far-field'' are used throughout this Guide to describe 
the relative distance from the fire. In general, predictions of near-field phenomena are more prone to error than far-field. There are exceptions, however. For example,
a prediction of the temperature directly within the flaming region may be more accurate than that made just a fire diameter away because of the fact that temperatures
tend to stabilize at about 1000~$^\circ$C within the fire itself, but then rapidly decrease away from the flames. Less accurate predictions typically occur in regions
of steep gradients (rapid changes, both in space and time).
\end{itemize}




\section{Model Accuracy}

The desired accuracy for each predicted quantity depends on the  technical  issues  associated with  the
analysis.  You must ask the question: How accurate does  the analysis have to
be  to  answer  the  technical  question posed?  
Returning   to    the   earlier   definitions    of   ``design''   and
``reconstruction,''  design applications  typically are  more accurate
because the heat release rate is typically specified rather than predicted, and
the    initial    and    boundary    conditions   are    better
characterized -- at least in the analysis. Mathematically, a design calculation is an example of a
``well-posed''  problem  in  which   the  solution  of  the  governing
equations is  advanced in  time starting from  a known set  of initial
conditions and constrained by a known set of boundary conditions.  The
accuracy of the results is a function of the fidelity of the numerical
solution, which is  mainly dependent on the size  of the computational
grid. 

A reconstruction is an example of an ``ill-posed'' problem because the
outcome  is known  whereas  the initial  and  boundary conditions  are usually
not. There is  no single, unique solution to the  problem, that is, it
is possible to simulate numerous fires that produce the same general outcome.
There is no right or wrong answer, but rather a small set of plausible
fire scenarios that are  consistent with the collected evidence. These
simulations are then used to demonstrate to fire service personnel why
the fire behaved as it did  based on the current understanding of fire
physics  incorporated in  the model.  Most  often, the  result of  the
analysis is only  qualitative. If there is any  quantification at all,
it could be in the time to reach critical events, like a roof collapse
or room flashover.
