%
% -------------------  Introduction ------------------------
%

\section{Introduction}
This \paper\ documents the physics and associated numerical algorithms used by Smokeview\cite{Smokeview_Users_Guide} to visualize smoke and fire realistically.  Smokeview is a companion to the fire model, the Fire Dynamics Simulator (FDS), which generates   simulation data Smokeview uses to perform these visualizations. FDS is a computational fluid dynamics model of fire-driven fluid flow. FDS solves numerically a form of the Navier-Stokes equations appropriate for low-speed, thermally-driven flow with an emphasis on smoke and heat transport from fires~\cite{FDS_Tech_Guide}.

\subsection{Background}

Realistic visualization methods are important for applications where one wishes to observe effects of fire and smoke rather than determine quantitative characteristics of data such as temperature or velocity.  This would be the case for a fire fighter using a computer based fire fighting simulator. Realistic visualization methods, however, complement but do not replace other more traditional visualization methods such as 2D contouring or 3D iso-surfacing which are better suited for analyzing data quantitatively.

Complete methods for visualizing smoke and fire data taking into account interactions between light and smoke such as absorption, emission, and scattering require the solution of the radiation transport equation (RTE)~\cite{Siegel:2001} also called the volume rendering equation~\cite{levoy:1988}. This equation models how light appears to an observer after it interacts with smoke, a participating media. The RTE used by Smokeview~\citesmv\ to model smoke and fire appearance is identical to that used by the Fire Dynamics Simulator (FDS)~\cite{FDS_Tech_Guide} to model radiative heat transfer.  Smokeview solves the RTE for visible wavelengths of light while FDS solves it for infrared. Smokeview solves the RTE assuming a gray gas environment.  This is the default solution method for FDS.  One other important difference is that Smokeview requires a solution only at the observer's viewpoint while FDS requires a radiation field, {\em i.e.} solutions at all points within the solution domain.

Even so, the solution of this integro-differential equation for visualization requires significant computation and memory  resources.  Approximations are required in order to display smoke and fire at interactive frame rates.  The primary approximation is to take advantage of the low albedo character of smoke allowing one to either simplify or eliminate scattering terms in the RTE.

Two techniques discussed for visualizing smoke realistically are slice rendered and volume rendered methods~\cite{levoy:1988,Engel:2006}.    These methods both solve a form of the RTE equation.  The integrated quantity in both cases results in an opacity, the fraction of light obscured.  They differ in how the integration path is partitioned.

The integration path for a slice rendered method is split at grid planes within a 3D mesh. There is one partially transparent slice for each plane of simulated data. Planes are drawn through the data along the XY, XZ, YZ coordinate planes or diagonally to these  planes.   The separation distance between these planes becomes smaller as more planes are used to simulate a case.  As a result, due to finite precision arithmetic, the computed opacity values are subject to increased round off error.  In fact, if these planes are sufficiently close, the computed opacities truncate to zero.  The particular plane drawn is chosen to be the one most perpendicular to the viewer's line of sight.
The resulting partially transparent slice planes are drawn individually and combined by the video hardware to form one image.

The second approach, a volume rendering method, again integrates a simplified form of the radiation transport equation but over the entire data mesh, from the front (relative to the observer) of the solution domain to the back, rather than across just one slice plane. By performing the entire integral at once instead of in pieces, the volume rendered method does not suffer from the round off errors of the slice rendered method.  The intermediate computational terms in the volume rendered method are stored using full precision arithmetic rather than 8 bits.  A volume rendered method then computes opacity across multiple grid planes.  As a result, there is only one plane of data displayed.  The video hardware is again exploited, but this time to compute a line integral for each pixel in the observer's view.  Opacities and color for both methods are computed using transfer functions using soot density and temperature data obtained from a fire simulation.

\subsection{Overview}
This \paper\ is organized as follows.  A model for visualizing smoke and fire is discussed.  This model, the radiative transport equation (RTE), is simplified in several ways one of which gives the Beer-Lambert law.  Two methods are then discussed for visualizing smoke  both using a form of the RTE.  The first method, slice rendering,  uses a series of partially transparent slices to represent smoke and fire. The second method, volume rendering, solves the RTE over the entire domain. Finally, several areas where the visualization methods may be improved are given.

%
% -------------------  Visualization Model ------------------------
%

\section{Visualization Model}
The visualization model used here to simulate smoke appearance is the radiation transport equation (RTE)~\cite{Siegel:2001}.  This equation uses the quantity radiance to represent smoke appearance.  Radiance has units of watts per square meter per unit solid angle~(W/(sr$\cdot$m$^2$)).  The solid angle accounts for the fact that a light source appears brighter if it emits a given amount of light through a smaller cross-sectional area.  This has the interesting implication, ignoring atmospheric effects, as pointed out in~\cite{dutre:2002}, that the sun appears equally bright on the Earth as on Mars.  The diminished heat flux $(W/\mbox{m}^2)$ on Mars due to the increased distance from the sun is exactly offset by the reduced solid angle ($sr$) of the sun's disk.  From a visualization perspective, this implies that image brightness does not depend on image distance from the observer unless  participating media is present to absorb or scatter  intervening light.  The radiation transport equation discussed in this section models the change in radiance or appearance due to these factors.

%
% ----------  Radiation Transport Equation --------------
%

\subsection{Full Radiation Transport Equation}
\newcommand{\siga}{ \sigma_a(x) }
\newcommand{\sigt}{ \sigma_t(x) }
\newcommand{\sigs}{ \sigma_s(x) }
\newcommand{\sigts}{ \sigma_t(s) }
\newcommand{\Le}{ C_e(x) }
\newcommand{\Lexo}{ C_e(x,\omega) }
\newcommand{\Lxo}{ C(x,\omega) }
\newcommand{\dLdx}{ \frac{dC}{dx}(x)}
\newcommand{\intf}[2]{ \exp\left({\int_{#1}^{#2} \sigts ds}\right) }
\newcommand{\intff}[2]{ {\int_{#1}^{#2} \sigts ds} }
\newcommand{\intmf}[2]{ \exp\left({-\int_{#1}^{#2} \sigts ds}\right) }
\newcommand{\intmff}[2]{ {-\int_#1^#2 \sigts ds} }

The radiation transport equation is used to calculate radiance due to one or more light sources within a region possibly containing a participating media such as smoke~\cite{Siegel:2001}. The change in radiance along a ray with direction $\omega$ may be expressed using

\begin{eqnarray}
\label{eq:fullrte}
 \left(\omega\cdot\nabla\right)\Lxo =
-\underbrace{\siga\Lxo}_\mathrm{absorption}-\underbrace{\sigs\Lxo}_\mathrm{out-scattering}
+ \underbrace{\siga\Lexo}_\mathrm{emission} +
\underbrace{\sigs\int_{4\pi}p(x,\omega,\omega')C_i(x,\omega')d\omega'}_\mathrm{in-scattering}
\end{eqnarray}

\noindent where  $\Lxo$ represents the  radiance at $x$ along a direction $\omega$.
As illustrated in Figure \ref{figRadiance}, the right hand side of (\ref{eq:fullrte}) is split into four components representing absorption, in and out scattering and emission where $\siga$ is the absorption coefficient, $\sigs$ is the scattering coefficient, $\Lexo$ is the radiance emitted at $x$ along a direction $\omega$ and $p(x,\omega,\omega')$ is the fraction of light moving along direction $\omega'$ scattered along direction $\omega$. Absorption and out-scattering cause radiance to decrease while emission and in-scattering cause radiance to increase. The terms $C$, $C_e$ and $C_i$ each have units of W/m$^2$sr$^{-1}$.  The terms $\sigma_a$ and $\sigma_s$ each have units of m$^{-1}$ and specify the change per unit length to the radiance term they are applied to.

\begin{figure}[\figoptions]
\begin{center}
\includegraphics[width=6.0in]{figures/rte_setup}
\end{center}
\caption{Components of the radiation transport equation decreasing radiance along a ray are
absorption and out-scattering while components increasing radiance are emission and in-scattering.}
\label{figRadiance}
\end{figure}

%
% ----  Approximating the Radiation Transport Equation ------------------------
%

\subsection{Approximate Radiation Transport Equation}

The RTE may be simplified in several ways depending on which terms are included or ignored.  This section derives a solution to one approximation where the in-scattering integral term in (\ref{eq:fullrte}) is neglected and the absorption and out-scattering coefficients are combined (both are loss terms) using $\sigt=\siga+\sigs$.  This simplification then only includes interactions between light and smoke due to absorption,  out-scattering and emission.  Note that the Beer-Lambert law results if the emission term is also dropped.

Equation (\ref{eq:fullrte}) is then approximated by neglecting the integral term and using $\sigt=\siga+\sigs$ to obtain

\begin{eqnarray}
\label{eq:simple_rte}
\dLdx&=&-\sigt C(x) + \siga C_e(x)\\
 C(x_0)&=&C_0\nonumber
\end{eqnarray}

This equation may be solved by rearranging terms and applying the integrating factor $\exp(\int_{x_0}^x \sigts ds)$ obtaining

\begin{eqnarray*}
\intf{x_0}{x}\left(\dLdx+\sigt C(x)\right)&=&  \intf{x_0}{x}\siga \Le\\
\frac{d}{dx}\left(\intf{x_0}{x} C(x)\right)&=& \intf{x_0}{x}\siga \Le
\end{eqnarray*}

Integrating both sides and substituting the integration limits results in

\begin{eqnarray*}
\left.\intf{x_0}{x} C(x)\right|_{x_0}^{x_N}&=& \int_{x_0}^{x_N}\intf{x_0}{x}\siga \Le dx \\
\intf{x_0}{x_N} C(x_N)-C_0&=& \int_{x_0}^{x_N}\intf{x_0}{x}\siga \Le dx
\end{eqnarray*}

Solving for $C(x_N)$ after noting that $\intff{x_0}{x}-\intff{x_0}{x_N}=-\intff{x}{x_N}$ results in

\begin{eqnarray*}
C(x_N)&=&\intmf{x_0}{x_N} C_0+ \int_{x_0}^{x_N}\intmf{x}{x_N}\siga \Le dx\\
\end{eqnarray*}

which may be simplified to

\begin{equation}
\label{eq:rte_simp}
\label{eq:rtesoln}
 C(x_N)=\tau(x_0,x_N)C_0 + \int_{x_0}^{x_N}\tau(x,x_N)\siga\Le dx
\end{equation}

after defining $\tau(a,b)$ to represent the optical depth between $a$ and $b$ given by
\begin{equation}
\label{eq:optdepth}
\tau(a,b)=\intmf{a}{b}
\end{equation}
As noted earlier, if the emission term is neglected and $\sigma_t(x)=\sigma_t$ is constant over a path with length
$L=x_N-x_0$ then (\ref{eq:rte_simp}) simplifies to
\begin{equation}
\label{eq:rte_simp2}
 \frac{C(x_N)}{C_0}=\exp(-\sigma_tL)
\end{equation}
which is the Beer-Lambert law.

%
% -------------------  Slice Rendered Smoke ------------------------
%

\section{Solving the RTE using Slices}
A slice rendering algorithm for visualizing smoke consists of partitioning a 3D computational domain into a series of 2D slices.  The RTE is then solved on each slice.  Each slice solution only accounts for conditions between adjacent slices.  The individual partially transparent slice solutions are then combined using video hardware to form the final image.
 
There are many ways to {\em slice}\ a 3D data set.  The slice orientation is chosen to be the one most perpendicular to the viewer's line of sight where possible choices are slice planes parallel to the three cartesian coordinate planes (XY, XZ, YZ) or planes diagonal to the data.  The opacity at each grid node is computed using the distance $\Delta x$ between adjacent YZ planes and soot density data computed by the fire model.  If slice orientations other than YZ are displayed, then opacities are adjusted if the distance between planes is different than $\Delta x$.  Opacity data is computed and compressed using run length encoding as a preprocessing step and decompressed one frame at a time as data is displayed.


%
% -------------------  Computing Opacity ------------------------
%

\subsection{Computing Opacity}
Computing opacity at slice plane nodes is illustrated in Figure \ref{figsmokesetup}. A ray travels from the background to the observer through intervening smoke. Light is absorbed or scattered by the smoke as the ray passes each slice plane .  Emission effects are ignored. Scattering effects presently are only accounted for in the value of the total mass extinction coefficient used.  Light losses are assumed to be from both absorption and scattering. The obscuration is computed along each ray one grid plane at a time, using the Beer-Lambert law as follows.  The $\alpha=1-\tau$ values are pre-computed by FDS using the Beer-Lambert law~\cite{Siegel:2001}

\begin{equation}
\alpha=1-\exp(-\sigma_t\Delta x) \label{eq:alpha}
\end{equation}

\noindent for a particular view direction (down the x axis) where $\Delta x$ is this distance between two grid plane and as before $\sigma_t=\sigma_a+\sigma_s$ is the total mass extinction coefficient.  The Beer-Lambert law is an empirical relationship relating light absorption to the material properties of the media the light is traveling through, in this case soot or smoke.

\begin{figure}[\figoptions]
\begin{center}
\includegraphics[width=4.0in]{figures/smoke_setup}
\end{center}
\caption {Opacity, $\alpha_i$, computed at node $i$ using soot density, $S_i$ and grid spacing $\Delta x$.}
\label{figsmokesetup}
\end{figure}

The $\alpha$ parameter in equation (\ref{eq:alpha}) is used by OpenGL to blend smoke planes with the current background.  The $\alpha$ parameter used here also represents an opacity, 0.0 for completely transparent, 1.0 for completely opaque.

%
% -------------------  Adjusting Opacity ------------------------
%

\subsection{Adjusting Opacity}

The absorption parameter, $\alpha$, needs to be adjusted when the view direction is not aligned along the axis orthogonal to the viewing planes (as in Figure \ref{figray}), the distance between adjacent smoke planes changes, or viewing planes are skipped.

\begin{figure}[\figoptions]
\centerline{\includegraphics[width=3.5in]{figures/forney_figure4}}
\caption [Diagram illustrating the adjustment required to the opaqueness parameter, $\alpha$,
for non-axis aligned views.] { Diagram
illustrating the adjustment required to the opaqueness parameter, $\alpha$,
for non axis aligned views. The $\hat{\alpha}$ value along the $\Delta\hat{x}$ segment needs to account for
the longer path length. } \label{figray}
\end{figure}

Ten million exponential operations per second are required to display smoke with corrected $\alpha$'s at 10 frames per second if the simulation has grid dimensions of $100\times 100\times 100$. Recent advances in CPU and video hardware makes these types of visualizations possible. These corrections may also be performed in the video card (GPU), resulting in increased display rates because the GPU performs the corrections simultaneously at all or many of the grid nodes rather than one at a time as the CPU would.

The $\alpha$ obscurations are pre-computed using the distance $\Delta x$ between adjacent planes along the x-axis. The adjusted $\hat{\alpha}$ expressed in terms of $\Delta\hat{x}$ is given by

\begin{equation}
\label{eq:adjusted}
\hat{\alpha}=1-\exp(-\sigma_t\Delta \hat{x})\\
\end{equation}

where $\Delta\hat{x}$ is the distance between planes along the line of site.  Equations (\ref{eq:alpha}) and (\ref{eq:adjusted}) may be used to solve for $\hat{\alpha}$ in terms of $\alpha$ to obtain

\begin{equation}
\label{eq:alphahat}
\hat{\alpha}=1-(1-\alpha)^{\Delta\hat{x}/\Delta x}
\end{equation}

after noting that

\begin{eqnarray*}
1-\hat{\alpha}=\exp(-\sigma_t\Delta\hat{x})=\exp(-\sigma_t\Delta
x)^{\Delta\hat{x}/\Delta x}=(1-\alpha)^{\Delta\hat{x}/\Delta x}
\end{eqnarray*}

The computation of equation (\ref{eq:alphahat}) is expensive because the exponential is computed at each grid node for every time step.  In addition, numerical cancellation may occur for small $\alpha$ leading to loss of significant digits. Both problems may be solved by expanding equation (\ref{eq:alphahat}) in a Taylor series and keeping only the first few terms:

\begin{eqnarray*}
\hat{\alpha}\approx \alpha r -
\frac{\alpha^2}{2}r(r-1)+\frac{\alpha^3}{6}r(r-1)(r-2)
\end{eqnarray*}

where $r=\sec(\theta)=\Delta \hat{x}/\Delta x=||x_p-x_e||/n\cdot(x_p-x_e)$, $n$ is the unit vector normal to the current plane being drawn, $\theta$ is the angle between the view direction and $n$, $x_e$ is the observers position and $x_p$ is the vertex being drawn (along the view direction).  These terms are illustrated in Figure \ref{figray}.

When planes are skipped, equation (\ref{eq:alphahat}) may be simplified.  In particular, when every 2nd plane is skipped, $\Delta\hat{x}/\Delta x=2$, so that equation (\ref{eq:alphahat})
simplifies to

\begin{eqnarray*}
\hat{\alpha}=1-(1-\alpha)^2=2\alpha-\alpha^2
\end{eqnarray*}

The video hardware uses $\alpha$ values contained in the smoke planes to obscure the background much like a camera uses a neutral density filter to darken a scene.  Extending the analogy, Smokeview uses one spatial/time varying {\em numerical}\ neutral density filter for each plane of smoke data.  On a node by node basis then, each smoke plane obscures the current image stored in  the OpenGL back buffer by the amount $(1-\alpha)$ to form a new back buffer image.  Figure \ref{figplume} illustrates this process showing several snapshots of a fire plume. The final image in the lower right is the most realistic. A simplistic description of one step of this process is given by

\begin{eqnarray*}
\mbox{new buffer image} = (1-\alpha)\times \mbox{old buffer image}
\end{eqnarray*}


\begin{figure}[\figoptions]
\begin{center}
\begin{tabular}{cc}
\includegraphics[height=4.0in]{figures/splume_20_27}&
\includegraphics[height=4.0in]{figures/splume_17_27}\\
slices 20 to 27&slices 17 to 27\\
\includegraphics[height=4.0in]{figures/splume_14_27}&
\includegraphics[height=4.0in]{figures/splume_11_27}\\
slices 14 to 27&slices 11 to 27
\end{tabular}
\end{center}
\caption [Smoke plume visualized using several vertical parallel
partially transparent planes.] {Smoke plume visualized using
several vertical parallel partially transparent planes. The smoke
plume looks more realistic as more slice planes
are included to form the image. } \label{figplume}
\end{figure}

\noindent This process is repeated for each smoke plane.

Figure \ref{figsmoke3d} illustrates this process showing smoke and fire in a townhouse kitchen fire. The visualization is performed by displaying a series of partially transparent planes. For illustration, these planes are made more conspicuous (in Figure \ref{figsmoke3d}a) by skipping smoke planes (displaying every third plane) and orienting them along the `x' axis. Figure \ref{figsmoke3d}b shows the visualization as it normally appears with all slice planes shown and oriented along a
plane most perpendicular to the view direction.

\begin{figure}[\figoptions]
\begin{center}
\begin{tabular}{l}
\includegraphics[height=3.75in]{figures/thouse5c_skip}\\
a) slices skipped and oriented along `x' directions\\
\includegraphics[height=3.75in]{figures/thouse5c_full}\\
b) all slices shown and oriented towards viewer \\
\end{tabular}
\end{center}
\caption[Realistic visualization of a townhouse kitchen fire simulated using FDS.]{Realistic visualization of a townhouse kitchen fire simulated using FDS. For illustrative purposes, planes in the top image are oriented along the X axis .  Planes in the bottom image are aligned along the y axis, the axis most perpendicular to the line of sight.}
\label{figsmoke3d}%
\end{figure}

%
% -------------------  Orienting smoke planes ------------------------
%

\subsection{Orienting smoke planes}

Smoke opacity data computed as described in previous sections is stored in a 3D array. This array corresponds to the solution domain as set up in an FDS input file (or some other model). Smoke planes are drawn in Smokeview through this data.  The orientation is chosen to be most perpendicular to the viewers line of sight. A plane orientation exactly perpendicular to the view direction could be drawn if one is willing to pay the added CPU cost of interpolating opacity values between grid nodes.

Figure \ref{figDIRA} illustrates this process showing three view directions and the corresponding smoke plane orientations that would be used. Off-axis viewing is minimized by selecting the view planes orientation that minimizes the angle between the planes normal direction and the view direction. This angle, $\theta$, is illustrated in Figure \ref{figDIRB}, and is given by

\begin{eqnarray*}
\cos(\theta)=\frac{n\cdot v_e}{||n||~||v_e||}
\end{eqnarray*}

\noindent where $n$ is normal vector for the candidate smoke plane, and $v_e$ is the view direction vector.  In OpenGL, the view direction vector, $v_e$, is computed by simply obtaining the modelview matrix, $M$ and multiplying it by the vector, $(0,0,1)^T$ or equivalently the third row of $M$.

\begin{figure}
\begin{tabular}{ccc}
\includegraphics[width=2.25in]{figures/figDIR1a}&
\includegraphics[width=2.25in]{figures/figDIR1b}&
\includegraphics[width=2.25in]{figures/figDIR1c}\\
a) smoke planes parallel to the $y$ axis& b) smoke planes parallel to
the $y=x$ axis&
c) smoke planes parallel to the $x$ axis\\
\end{tabular}
\caption{View of smoke planes from above.  Smoke Planes are
oriented so that they are {\em most perpendicular}\ to the line of sight }
\label{figDIRA}
\end{figure}

\begin{figure}
\centerline{\includegraphics[width=3.5in]{figures/figDIR2}}
\caption[Diagram illustrating the angle between the line of sight
and the vector normal to the smoke plane.]{Diagram illustrating the angle between the line of sight
and the vector normal to the smoke plane.  View plane orientation is chosen to minimize
this angle.} \label{figDIRB}
\end{figure}

%
% -------------------  Compressing Smoke Data ------------------------
%

\subsection{Compressing Smoke Data}

The opacity parameters are computed at each node for all time steps. The space required to store these values can easily become quite large. Compression techniques are required to reduce storage requirements.

Storage reduction occurs in two steps.  First, four byte floating point soot densities are converted to one bye smoke opacities using the Beer-Lambert law.  Video cards presently use only one byte to represent opacity. Next, the sequence of opacity values are compressed using run-length encoding, a compression scheme where repeated ``runs'' of data are replaced with a number (number of repeats), and the value repeated.  In more detail,


\begin{enumerate}
\item Represent four or more consecutive identical characters as $\# n c$ where $\#$ is a special character denoting the beginning
of a repeated sequence, $n$ is the number of repeats and $c$ is the character repeated.  $n$ can be up to 254 (255 is used to
represent the {\em special}\ character). \item Represent characters not repeated four or more times as is.
\end{enumerate}

The character string {\tt aaaaaabbbbcc}\ would then be encoded as {\tt \#6b\#4bcc}.

Run length encoding provides a reasonably good compression ratio, is simple to implement and more importantly can be decompressed quickly. This last property is important for any compression scheme chosen because it is a rate limiting step in the process that Smokeview uses to display smoke data. The CPU time required to compute the smoke flow can easily exceed one minute of CPU time
per outputted time step, so extra time used to produce a more compact file is affordable. However, each data frame is decompressed {\em on the fly}\ so a compression format that can be rapidly decompressed is critical.

A second compression scheme is used by Smokezip, companion software to FDS and Smokeview, to compress FDS files even more compactly.  Smokezip uses the ZLIB compression library~\cite{ZLIB}.

\subsection{Limitation}
As noted earlier, the slice rendering method for visualizing smoke records opacities on grid planes using
\begin{equation}
\label{eq:alpha}
\alpha=1-\exp(-\sigma_t\Delta x)
\end{equation}
where $\sigma_t$ total mass extinction coefficient and $\Delta x$ is the distance between adjacent grid planes.  Problems with round off error can occur because $\alpha$'s are stored using only 8 bits, the size typical video hardware uses to represent color and alpha channels.
As a result, opacity values are quantized.  Only values of 0, 1/255, $\cdots$, 254/255, 1 can be represented.  In particular any $\alpha$ value smaller than $1/255$ truncates to 0.
This can occur when more grid cells are used to resolve a solution domain, {\em i.e.}\ when $\Delta x$ is sufficiently small so that $\alpha<1/256$.

For example, suppose that a 1~m column of smoke has opacity of 0.5.  Substituting $\Delta x=1$ and $\alpha=1/2$ into (\ref{eq:alpha}) gives $\sigma_t=-\ln(1/2)$.  Solving for $\Delta x$ in (\ref{eq:alpha}) using this value of $\sigma_t$ and $\alpha=1/255$ gives
\begin{equation}
\Delta x = \frac{\ln(254/255)}{\ln(1/2)}
\end{equation}
Therefore, when $\Delta x<\ln(254/255)/\ln(1/2)\approx 0.00567$ the opacity truncates to zero.  Equivalently, the opacity truncates to zero when $N>\ln(1/2)/\ln(254/255)\approx 177$ where $N$ is the number of grid planes in the 1~m column of smoke.

To get around this problem, a volume rendering algorithm for visualizing smoke has been implemented where intermediate smoke opacities are stored using full precision arithmetic instead of 8 bits.  This algorithm is discussed in the next section.


%
% -------------------  Volume Rendered Smoke ------------------------
%

\section{Solving the RTE using Volumes}
Volume rendering is the process of visualizing 3D data by projecting partially transparent colors derived from 3D data onto a 2D image.  An entire volume of data is used to generate an image rather than just a slice as in the previous section.  Transfer functions are used to map data to color and optical density (opacity).  These colors and opacities are then combined to form an image.  The transfer functions may be arbitrary designed to highlight certain portions of the data or based on physics designed to produce realistic appearing images.  Figure \ref{figsmokesetup2} illustrates this process.  Data is first mapped to color and opacity then combined to form an image.


\begin{figure}[\figoptions]
\begin{center}
\includegraphics[width=6.5in]{figures/smoke_setup2}
\end{center}
\caption[Opacity and color is computed for each pixel on an image plane by solving a line integral
representation of a simplified form of the radiation transport equation.]{Opacity and color is computed for each pixel on an image plane by solving a line integral
representation of a simplified form of the radiation transport equation.  Rays are cast from the observer through the  image plane into the 3D data set.  The line integral is computed where the ray intersects the data set.
}
\label{figsmokesetup2}
\end{figure}

A general transfer function is used to map data values to color and opacity. To produce a realistic image, the strategy is to perform this mixing in the same way that light would behave by solving an approximate form of the radiation transport equation.  The approximate RTE solution given in (\ref{eq:rtesoln})  is used to collapse 3D data into 2D images projected onto the front facing sides of the data mesh or meshes.

Figure \ref{fig:volplume_example} illustrates this process showing two images.  The image on the left is presented from the point of view of the observer.  The image on the right is the same image rotated to show more clearly the images projected on the front facing sides of the data mesh.  As the scene is moved through rotations and translations the projected surface images are constantly recomputed and redrawn presenting the illusion that the image is 3D and drawn within (rather than on the surface) of the data mesh.

\begin{figure}[\figoptions]
\begin{center}
\begin{tabular}{cc}
\includegraphics[width=3.5in]{figures/vis_test2_nonfreeze}&
\includegraphics[width=3.5in]{figures/vis_test2_freeze}\\
a) front view&b) side view\\
\end{tabular}
\end{center}
\caption[Volume rendered smoke plume shown from two directions.]{Volume rendered smoke plume shown from two directions.
The first direction is as seen by the observer.  The second direction shows the plume as it is computed on the two forward facing faces of the data mesh.
}
\label{fig:volplume_example}
\end{figure}

This section derives a discretization of the optical depth given in equation (\ref{eq:optdepth}) and the approximate RTE solution given in equation (\ref{eq:rte_simp}).  It then presents an algorithm for performing the computations.


\subsection{Discretizing the Radiation Transport Equation}
\newcommand{\htau}[1]{\tau_{#1}^{N-1}}
\newcommand{\halpha}[1]{\alpha_{#1}^{N-1}}
\newcommand{\sigai}[1]{\sigma_{a,#1}}
\newcommand{\Lei}[1]{C_{e,#1}}
\newcommand{\Lhatj}[1]{C_{#1}^N}
\newcommand{\Lhatjj}[1]{\hat{C}_{#1}^N}
\newcommand{\Chatjj}[1]{\hat{C}_{#1}^N}
\newcommand{\Leii}[1]{\hat{C}_{e,#1}}

The approximate RTE solution, equation (\ref{eq:rte_simp}), is discretized by converting integral terms present to Riemann sums. Figure \ref{fig:smokediscretesetup}\ illustrates the terms used to perform these discretizations.  The path is split into $N$ regions each with length $\Delta x=(x_N-x_0)/N$.  The coordinate system is setup so that the initial radiance, $C_0$, is located at $x_0$ furthest from the observer and the final radiance, $C_N$ located at $x_N$ closest to the observer.

\begin{figure}[\figoptions]
\begin{center}
\includegraphics[width=3.5in]{figures/smoke_discrete_setup}
\end{center}
\caption {Setup for discretizing the equations used to model
radiance within a column of 3D smoke data.}
\label{fig:smokediscretesetup}
\end{figure}

The optical depth, $\tau(a,b)$, in equation (\ref{eq:optdepth}) is discretized using a Riemann sum  after defining sample points $s_j=x_0+j\Delta s$ with spacing $\Delta s=x_N/N$ to obtain

\begin{eqnarray*}
\htau{i}=\tau(x_i,x_N)&=&\exp\left(-\int_{x_i}^{x_N}\sigma_t(s)ds\right)\\
&\approx&\exp\left(-\sum_{j=i}^{N-1}\sigma_t(s_j)\Delta s\right)\\
&=&\prod_{j=i}^{N-1}\exp\left(-\sigma_t(s_j)\Delta s\right)
\end{eqnarray*}

or

\begin{eqnarray}
\label{eq:tauhat_discrete}
\htau{i}&=&\prod_{j=i}^{N-1}\tau_j
\end{eqnarray}

where $\tau_j=\exp\left(-\sigma_t(s_j)\Delta s\right)$ represents the transparency over one discretization interval.
For $i=N-1\cdots 0$, the optical depth $\htau{i}$ may be computed recursively using 
\begin{eqnarray}
\label{eq:tauhat_recurse}
\htau{i}&=&\htau{i+1}\tau_i
\end{eqnarray}
where the recursion starts with $\htau{N}=1$.
Substituting $1-\halpha{i}=\htau{i}$ and $1-\alpha_i=\tau_i$ into (\ref{eq:tauhat_recurse}) gives
\begin{eqnarray*}
1-\halpha{i}&=&(1-\halpha{i+1})(1-\alpha_i)\\
&=&1-\halpha{i+1} - \alpha_i + \halpha{i+1}\alpha_i
\end{eqnarray*}
which simplifies to
\begin{eqnarray}
\label{eq:alpha}
\halpha{i}&=&\halpha{i+1} + (1-\halpha{i+1})\alpha_i
\end{eqnarray}

Similarly, the radiance given by the RTE solution $C(x_N)$ in equation (\ref{eq:rtesoln}) may be discretized to obtain

\begin{equation}
\label{eq:discrete_rte}
C_{N} = \htau{0}C_0 +
\sum_{i=0}^{N-1}\htau{i}\sigai{i}\Lei{i}\Delta x
\end{equation}

where $\sigma_{a,i}=\sigma_a(x_i)$, $\Lei{i}=C_e(x_i)$, $x_i=x_0+i\Delta x$ and $\Delta x=x_N/N$.
Letting $\Leii{i}=\sigma_{a,i}C_{e,i}\Delta x$, this simplifies to

\begin{equation}
\label{eq:discrete_rte2}
C_N = \htau{0}C_0 + \sum_{i=0}^{N-1}\htau{i}\Leii{i}
\end{equation}

The terms in equation (\ref{eq:discrete_rte2}) are summed from back to front meaning that the location of the $i=0$ term is furthest from the observer while the location of the $i=N-1$ term is closest.  We wish to perform this sum in reverse order, from front to back so that the sum may be terminated early if further contributions would not significantly change the result.

Therefore, to compute $C_N$, let $\Chatjj{j}$ denote the partial sum using terms $i=j$ through $i=N-1$ in the summation term in (\ref{eq:discrete_rte2}).  Using this notation $\hat{C}_N=\htau{0}C_0+\Chatjj{0}$ . Then

\begin{eqnarray}
\label{eq:recurse1}
\Chatjj{j} &= &\sum_{i=j}^{N-1}\htau{i}\Leii{i}\\
\label{eq:recurse2}
\Chatjj{j+1}     &= &\sum_{i=j+1}^{N-1}  \htau{i}\Leii{i}
\end{eqnarray}

Subtracting (\ref{eq:recurse2}) from (\ref{eq:recurse1}) and solving for $\Chatjj{j}$ results in
\begin{eqnarray}
\label{eq:color}
\Chatjj{j}&=&\Chatjj{j+1}+\htau{j}\Leii{j}
\end{eqnarray}
The basic computational strategy then is to compute a color using equation (\ref{eq:color}) and an opacity using equation (\ref{eq:alpha}) for each pixel in the 2D projected image.


\subsection{Solution Algorithm}
An algorithm for determining image opacities and colors using volume rendering is detailed below.

\begin{enumerate}

\item For each pixel in the image plane, cast a ray from the observer's viewpoint through that pixel into the 3D data set

\item Step along the ray from the front (relative to the observer) to the back of the 3D data set converting data values along the way to color and opacity.  Choose a step size (possibly varying) to capture changes occurring in the data set.

\item
Combine the colors and opacities found in step 2 using recursion~\cite[Chapter 39]{gpugems}.
Start the recursion with $\hat{\alpha}_{N}=\hat{C}_{N}=0$. Continue the recursion for $i=N-1$ to $i=0$ by computing $\hat{\alpha}_i$ and $\hat{C}_i$ using:

\begin{eqnarray}
\label{eq:alphaupdate}
\hat{\alpha}_i&=&\hat{\alpha}_{i+1}+\left(1-\hat{\alpha}_{i+1}\right)\alpha_i\\
\label{eq:colorupdate}
\hat{C}_i&=&\hat{C}_{i+1}+\left(1-\hat{\alpha}_{i+1}\right)C_i
\end{eqnarray}

where $\hat{C}_i$ and $\hat{\alpha}_i$ are color and opacity accumulated from steps $N$ to $i$ while stepping through the 3D data set from front to back.  When data is mapped appropriately to color and opacity, this recursion is simply a numerical integration of the radiation transport equation.
The values $C_i$ and $\alpha_i$ are the color and opacity at the $i$'th step.

\item Mix the volume rendered color, $\hat{C}_N$, with the colors already rendered using

\begin{eqnarray*}
\noindent\mbox{updated background color} = (1-\hat{\alpha}_N)\times \mbox{original background color} + 1\times\hat{C}_N
\end{eqnarray*}

The OpenGL call, {\tt   glBlendFunc(GL\_ONE,GL\_ONE\_MINUS\_SRC\_ALPHA); }, is used to implement this mixing mode in Smokeview.
\end{enumerate}


%
% -------------------  Combining Multiple Solutions ------------------------
%

\subsection{Volume Rendering on Multiple Meshes}
When drawing smoke for cases with multiple data meshes, it is more practical to draw smoke one mesh at a time rather than to draw smoke once using data from all meshes.  This is because data from other meshes may not be easily accessible.  The RTE solution or equivalently the computed smoke opacity requires smoke properties for the entire line of sight which may encompass more than one data mesh.  This section discusses how smoke opacities may be computed for each individual data mesh and combined appropriately to form smoke opacities for the entire data domain.

As illustrated in Figure \ref{figsmokesetup3}, consider the interval $[x_0,x_N]$ that is split at $D$ into two intervals $[x_0,D]$ and $[D,x_N]$.  Let $C$ be the RTE solution for $[x_0,x_N]$ and $C_1$ and $C_2$ be the RTE solutions for the two sub-intervals.  Likewise let $\alpha$ be the optical depth for $[x_0,x_N]$ and $\alpha_1$ and $\alpha_2$ be the optical depths for the two sub-intervals.  This section then determines how the solutions $C$ and $\alpha$ for the full interval can be related to solutions $C_1$, $C_2$, $\alpha_1$ and $\alpha_2$ for the two sub-intervals.

\begin{figure}[\figoptions]
\begin{center}
\includegraphics[width=4.0in]{figures/smoke_setup3}
\end{center}
\caption {Solutions to the RTE on two sub-intervals are combined to form a solution to the RTE
for the merger of these intervals.}
\label{figsmokesetup3}
\end{figure}

The radiance $C$ and optical depth $\alpha$ for the interval $[x_0,x_N]$ are given by
\begin{eqnarray*}
C&=&\int_{x_0}^{x_N}\tau(x,x_N)g(x)dx\\
\alpha&=&\tau(x_0,x_N)
\end{eqnarray*}
The radiances $C_1$ and $C_2$ and optical depths $\alpha_1$ and $\alpha_2$ for the intervals $[x_0,D]$ and $[D,x_N]$ are given by
\begin{eqnarray*}
C_1&=&\int_{x_0}^{D}\tau(x,D)g(x)dx\\
C_2&=&\int_{D}^{x_N}\tau(x,x_N)g(x)dx\\
\alpha_1&=&\tau(x_0,D)\\
\alpha_2&=&\tau(D,x_N)
\end{eqnarray*}

\label{eq:optdepth}

The optical depth $\alpha$ can be split into two parts giving
\begin{eqnarray*}
\alpha&=&\tau(x_0,x_N)=\tau(x_0,D)\tau(D,x_N)=\alpha_1\alpha_2
\end{eqnarray*}
since from (\label{eq:optdepth}) it can be shown that $\tau(a,b)=\tau(a,x)\tau(x,b)$.

The radiance $C$ can be split into two parts giving
\begin{eqnarray*}
C&=&\int_{x_0}^{x_N}\tau(x,x_N)g(x)dx\\
&=&\int_{x_0}^{D}\tau(x,x_N)g(x)dx+\int_{D}^{x_N}\tau(x,x_N)g(x)dx\\
&=&\tau(D,x_N)\int_{x_0}^{D}\tau(x,D)g(x)dx+\int_{D}^{x_N}\tau(x,x_N)g(x)dx\\
&=&\alpha_2C_1+C_2
\end{eqnarray*}


\section{Future Work}
This \paper\ describes the algorithms Smokeview uses to display smoke and fire realistically using physics based algorithms.  These algorithms may be improved in several ways.  Presently, only radiation from soot is used to visualize smoke.  Radiation emissions from heated gas components such as CO2 and water vapor are neglected.  The gray gas assumption may be relaxed by solving the RTE for several wavelength bands (as may be done by FDS) and combining the results.  The interaction between volume rendered smoke and interior solid objects may be improved by accounting for interior solid objects such as FDS OBST's.  The RTE line integration needs to terminate at the first blockage encountered rather than the far side of the data mesh.  Work on unstructured geometries presently researched for future incorporation into FDS may also lead to better physics there.  Finally, the transfer function relating temperature to fire color may be improved by using a physics based approach rather than a color map.


