% !TEX root = Correlation_Guide.tex

\chapter{Fire Plumes}
\label{Plume_Chapter}

\clearpage

\section{Plume Temperature}

\subsection{Heskestad Method}

\subsubsection*{Description}

For a fire plume, the correlation by Heskestad~\cite{SFPE:Heskestad} predicts that the increase in centerline temperature, $\Delta T_0$, is given by
\be
\Delta T_0 = \frac{9.1 \left( \frac{T_\infty}{g c_p^2 \rho_{\infty}^2} \right)^{1/3} \dot Q_c^{2/3}}{(z-z_0)^{5/3}} \quad ^\circ{\rm C}
\label{eq:Heskestad}
\ee
where $T_\infty$ is the ambient air temperature~($^\circ$C), $g$ is the acceleration of gravity~(m/s$^2$), $c_p$ is the specific heat of air~(kJ/kg/K), $\rho_{\infty}$ is the ambient air density~(\si{kg/m^3}), and $z$ is the elevation above the fire source~(\si{m}). The convective HRR, $\dot Q_c$ (\si{kW}), is given by
\be
\dot Q_c = \dot Q (1 - \chi_r)
\label{eq:Heskestad_Qc}
\ee
where $\dot Q$ is the total HRR~(\si{kW}), and $\chi_r$ is the radiative fraction~(-). The hypothetical virtual origin of the fire, $z_0$ (\si{m}), is given by
\be
z_0 = -1.02 D + 0.083 \dot Q^{2/5}
\label{eq:Heskestad_z0}
\ee
where $D$ is the diameter of the fire source~(\si{m}) and is given by
\be
D = \sqrt{\frac{4 A}{\pi}}
\label{eq:Heskestad_D}
\ee
where $A$ is the area of the fire source~(\si{m^2}).
Note that this plume temperature correlation is only valid above the mean flame height.

\subsubsection*{Verification}

Test: VTT Test 1

\begin{table}[!ht]
\caption[Input parameters, plume temperature (Heskestad)]
{Input parameters, plume temperature (Heskestad).}
\begin{center}
\begin{tabular}{|l|c|}
\hline
                        &              \\
\rb{Input Parameter}    &  \rb{Value}  \\ \hline \hline
$\dot Q$ (m)            &  1245        \\ \hline
$z$ (m)                 &  6           \\ \hline
$A$ (m$^2$)             &  1.075       \\ \hline
$\chi_r$ (-)            &  0.40        \\ \hline
$T_\infty$ ($^\circ$C)  &  22          \\ \hline
\end{tabular}
\end{center}
\end{table}

\noindent Expected result: The plume temperature $T_{0}$ is 133.76~$^\circ$C.


\clearpage


\subsubsection*{Validation}

\begin{figure}[!ht]
\begin{center}
\begin{tabular}{l}
\includegraphics[width=4.0in]{SCRIPT_FIGURES/Scatterplots/Plume_Temperature_Heskestad}
\end{tabular}
\end{center}
\caption[Summary of plume temperature predictions]
{Summary of plume temperature predictions using the Heskestad method.}
\label{Plume_Temperature_Heskestad}
\end{figure}


\clearpage


\subsection{McCaffrey Method}

\subsubsection*{Description}

For a fire plume, the correlation by McCaffrey~\cite{McCaffrey:NBSIR_79-1910} predicts that the increase in centerline temperature, $\Delta T_0$, is given by
\be
\Delta T_0 = \left[ \left( \frac{\kappa}{0.9 \sqrt{2 g}} \right)^2 \left( \frac{z}{\dot Q^{2/5}} \right)^{2 \eta - 1} \right] T_\infty \quad ^\circ{\rm C}
\label{eq:McCaffrey}
\ee
where $g$ is the acceleration of gravity~(\si{m/s^2}), $z$ is the elevation above the fire source~(\si{m}), $\dot Q$ is the HRR~(\si{kW}), and $T_\infty$ is the ambient air temperature~(\si{\celsius}). The constants $\eta$ and $\kappa$ are a function of the height $z$ within the plume and are listed in Table~\ref{tbl:McCaffrey_constants}.

\vspace{\baselineskip}
\begin{table}[!ht]
\begin{center}
\caption[Constants used in McCaffrey plume correlation]
{Constants used in McCaffrey plume correlation.}
\label{tbl:McCaffrey_constants}
\begin{tabular}{|c|c|c|c|}
\hline
Region        &  $z/\dot Q^{2/5}$ &  $\eta$ & $\kappa$ \\
\hline
Continuous    &  < 0.08           & 1/2     & 6.8      \\
Intermittent  &  < 0.08 -- 0.2    & 0       & 1.9      \\
Plume         &  > 0.2            & -1/3    & 1.1      \\
\hline
\end{tabular}
\end{center}
\end{table}

\subsubsection*{Verification}

Test: VTT Test 1

\begin{table}[!ht]
\caption[Input parameters, plume temperature (McCaffrey)]
{Input parameters, plume temperature (McCaffrey).}
\begin{center}
\begin{tabular}{|l|c|}
\hline
                        &              \\
\rb{Input Parameter}    &  \rb{Value}  \\ \hline \hline
$\dot Q$ (m)            &  1245        \\ \hline
$z$ (m)                 &  6           \\ \hline
$T_\infty$ ($^\circ$C)  &  22          \\ \hline
\end{tabular}
\end{center}
\end{table}

\noindent Expected result: The plume temperature $T_{0}$ is 153.21~$^\circ$C.


\clearpage


\subsubsection*{Validation}

\begin{figure}[!ht]
\begin{center}
\begin{tabular}{l}
\includegraphics[width=4.0in]{SCRIPT_FIGURES/Scatterplots/Plume_Temperature_McCaffrey}
\end{tabular}
\end{center}
\caption[Summary of plume temperature predictions]
{Summary of plume temperature predictions using the McCaffrey method.}
\label{Plume_Temperature_McCaffrey}
\end{figure}
